\documentclass[a4paper,10pt
,draft
]{article}%

\usepackage[hidelinks]{hyperref}
\hypersetup{
  % colorlinks,
  final,
  pdftitle={Equivariant Dendroidal Segal Spaces},
  pdfauthor={Bonventre, P. and Pereira, L. A.},
  % pdfsubject={Your subject here},
  % pdfkeywords={keyword1, keyword2},
  linktoc=page
}
%\usepackage[open=false]{bookmark}

\input{commands.tex}%


%-------- Tikz ---------------------------

\usepackage{tikz}%
\usetikzlibrary{matrix,arrows,decorations.pathmorphing,
cd,patterns,calc}
\tikzset{%
  treenode/.style = {shape=rectangle, rounded corners,%
                     draw, align=center,%
                     top color=white, bottom color=blue!20},%
  root/.style     = {treenode, font=\Large, bottom color=red!30},%
  env/.style      = {treenode, font=\ttfamily\normalsize},%
  dummy/.style    = {circle,draw,inner sep=0pt,minimum size=2mm}%
}%

\usetikzlibrary[decorations.pathreplacing]



% ---- Commands on draft --------

\usepackage{ifdraft}
\ifdraft{
  \color[RGB]{63,63,63}
  % \pagecolor[rgb]{0.5,0.5,0.5}
  \pagecolor[RGB]{220,220,204}
  % \color[rgb]{1,1,1}
}


\usepackage[draft]{showkeys}
\usepackage{todonotes}%[obeyDraft]


% ----- Labels Changed? --------

\makeatletter

\def\@testdef #1#2#3{%
  \def\reserved@a{#3}\expandafter \ifx \csname #1@#2\endcsname
  \reserved@a  \else
  \typeout{^^Jlabel #2 changed:^^J%
    \meaning\reserved@a^^J%
    \expandafter\meaning\csname #1@#2\endcsname^^J}%
  \@tempswatrue \fi}

\makeatother


% ---- Commands --------

\newcommand{\mycircled}[2][none]{%
  \mathbin{
    \tikz[baseline=(a.base)]\node[draw,circle,inner sep=-1.5pt, outer sep=0pt,fill=#1](a){\ensuremath #2\strut};
cf.  }
}
\newcommand{\owr}{\mycircled{\wr}}
\newcommand{\UV}{\underline{\mathcal V}}

\newcommand{\UC}{\underline{\mathfrak C}}

\renewcommand{\F}{\mathcal F}
\newcommand{\I}{\mathbb I}
\newcommand{\J}{\mathbb J}
\newcommand{\Q}{\mathcal Q}
\renewcommand{\1}{\ensuremath{\mathbb{id}}}
\renewcommand{\hat}{\widehat}
\newcommand{\lltimes}{\underline{\ltimes}}

\newcommand{\SC}{\Sigma_{\mathfrak C}}
\newcommand{\OC}{\Omega_{\mathfrak C}}

% ---- Title --------

\title{To be determined}%

\author{Peter Bonventre, Lu\'is A. Pereira}%

\date{\today}


% ---- Document body --------

\begin{document}

\maketitle

\begin{abstract}
      Things and stuff
\end{abstract}

\tableofcontents


\section{Overview}


\[
	\begin{tikzcd}
		\mathsf{PreOp}^G
\\
		\mathsf{sdSet}^G \ar{r}[swap]{(-)_0} \ar{u}{\gamma_{\**}} &
		\mathsf{dSet}^G
	\end{tikzcd}
\]


\section{Preliminaries}

\subsection{Wreath products and Grothendieck fibrations}

% -------------------- Types of Grothendieck fibrations --------------------

% \begin{remark}
%       \label{GRTH_RMK}
%       Given a functor
%       \begin{equation}
%             \mathcal B^{op} \longrightarrow \Cat,
%             \qquad
%             b \mapsto \mathcal C_b
%       \end{equation}
%       there are four possible ``Grothendieck constructions''.
%       Two are fibrations over $\mathcal B$, two over $\mathcal B^{op}$.
%       They all have as objects pairs $(b \in \mathcal B, X \in \mathcal C_b)$ and arrows pairs of maps,
%       but the directionality of these maps is different:
%       \begin{enumerate}[label = (\roman*)]
%       \item The \textit{standard} Grothendieck construction $\mathcal B \ltimes \mathcal C$
%             is the fibration over $\mathcal B$ with as arrows pairs of maps $(f, \phi)$ with
%             $f: b \to \bar b$ and
%             $\phi: X \to f^{\**} \bar X$.
%       \item $(\mathcal B \ltimes \mathcal C)^{\underline{op}}$
%             is the fibration over $\mathcal B$ with as arrows pairs of maps $(f, \phi)$ with
%             $f: b \to \bar b$ and
%             $\phi: f^{\**} \bar X \to X$.
%       \item $(\mathcal B \ltimes \mathcal C)^{op}$
%             is the fibration over $\mathcal B^{op}$ with as arrows pairs of maps $(f, \phi)$ with
%             $f: \bar b \to b$ and
%             $\phi: \bar X \to f^{\**} X$.
%       \item $\mathcal B \underline{\ltimes} \mathcal C = \left((\mathcal B \ltimes \mathcal C)^{\underline{op}}\right)^{op}$
%             is the fibration over $\mathcal B^{op}$ with as arrows pairs of maps $(f, \phi)$ with
%             $f: \bar b \to b$ and 
%             $\phi: f^{\**}X \to \bar X$.
%       \end{enumerate}
      
%       Unless otherwise specified, \textit{the} Grothendieck construction will refer to the first one.
% \end{remark}
\begin{notation}
      Given a functor $\mathcal C_{(-)}: \mathcal B \to \Cat$, $b \mapsto \mathcal C_b$, we let
      $\mathcal B \ltimes \mathcal C_{(-)}$ denote the (covariant) \textit{Grothendieck construction},
      with objects pairs $(b,X)$ with $b \in \mathcal B$ and $X \in \mathcal C_b$, and
      maps pairs $(f,g)$ with $f: b \to \bar b$ and $g: f_{\**}X \to \bar X$.
\end{notation}



% -------------------- Translation Categories --------------------

\begin{example}
      \label{G_GR_REM}
      Given a group $G$, let $BG$ denote the category which encodes \textit{left} $G$-actions (so $h \circ g = h g$).
      For any left $G$-set $A$, the \textit{translation category} or \textit{action groupoid}
      has object set $A$
      and morphism set all pairs $(g,a): a \to g.a$ for all $(g,a) \in G \times A$
      (equivalently, $\Hom(a,b) = \sets{g \in G}{b = g.a}$).
      %
      More generally, if $\mathcal C$ is a left $G$-category, the \textit{translation category}
      has object set $\mathrm{Ob}(\mathcal C)$
      and morphism set all tuples $(g,a,b,f)$ with $g \in G$, $a,b\in \mathcal C$, and $f: g.a \to b$.
      %
      These are both isomorphic to the Grothendieck construction $BG \ltimes \mathcal C$
      on the functor $BG \to \Cat$ defining the $G$-action on $\mathcal C$.
      We will mildly abuse notation and denote these categories by simply $G \ltimes \mathcal C$
      \footnote{
        The action groupoid for a $G$-set $A$ is often denoted $B_A G$, generalizing the notation for the category $B G = B_{\**} G$.}.
      
      Further, we note that if $\mathcal C$ has an action by other groups $\Sigma$ which commute with the action by $G$,
      then the iterated Grothendieck constructions above are associative and commutative.
\end{example}

\begin{remark}
      \label{GOP_REM}
      If $\mathcal C$ is a left $G$-category, then we write
      $G^{op} \ltimes \mathcal C$ to denote the standard Grothendieck construction on the functor
      $(BG^{op})^{op} = BG \xrightarrow{\mathcal C} \Cat$.

      Moreover, as $\mathcal C^{op}$ is also a left $G$-category, it is easy to check that
      $(G \ltimes \mathcal C)^{op} \simeq G^{op} \ltimes \mathcal C^{op}$,
      using the above convention.
      \todo[inline]{come back: make sure I fix what I messed up here in this Remark}
\end{remark}



% -------------------- OTHER STUFF --------------------


% First, we record a basic result.

% \begin{lemma}
%       \label{PB_GR_LEM}
%       If $\mathcal C \to \mathcal D$ is a Grothendieck fibration, then the pullback
%       \begin{equation}
%             \begin{tikzcd}
%                   \mathcal A \arrow[d] \arrow[r]
%                   &
%                   \mathcal C \arrow[d]
%                   \\
%                   \mathcal B \arrow[r, "F"]
%                   &
%                   \mathcal D
%             \end{tikzcd}
%       \end{equation}
%       is isomorphic to the Grothendieck construction on
%       \begin{equation}
%             \label{PB_GR_EQ}
%             % \begin{tikzcd}[row sep = tiny]
%             %       \mathcal B^{op} \arrow[r]
%             %       &
%             %       \mathsf{Cat}
%             %       \\
%             %       b \arrow[r, mapsto]
%             %       &
%             %       \mathcal C_{F(b)},
%             % \end{tikzcd}
%             \mathcal B^{op} \longto \Cat,
%             \qquad \qquad
%             b \longmapsto \mathcal C_{F(b)},
%       \end{equation}
%       where $\mathcal C_{d}$ is the fiber over $d \in \mathcal D$.
% \end{lemma}
% \begin{proof}
%       The fact that the map \eqref{PB_GR_EQ} is a functor follows from $\mathcal C \to \mathcal D$ being a fibration;
%       the rest follows by unpacking definitions.
% \end{proof}



% -------------------- Wreath Products --------------------

Now, recall the notation $\mathsf F \wr \mathcal C$ for a category $\mathcal C$ from \cite{BP17}.

\begin{notation}
      \label{F_WR_NOT}
      We let $\mathsf F$ denote a (fixed) full subcategory of \textit{ordered finite sets} and set maps,
      such that the only ordered isomorphisms are the identity.
      
      Given a category $\mathcal C$, let $\mathsf F \wr \mathcal C$ denote the contravariant Grothendieck construction
      $(\mathsf F^{op} \ltimes \mathcal C^{\times (-)})^{op}$ on the functor
      \begin{equation}
            \mathsf F^{op} \longto \Cat,
            \qquad \qquad
            A \mapsto \mathcal C^{\times A}.
      \end{equation}
      Explicitly, objects are tuples of elements of $\mathcal C$, and maps are composites of ``shuffles'' and tuples of maps in $\mathcal C$.
\end{notation}

\begin{remark}
      \label{WR_DIAG_REM}
      We observe that we have a natural diagonal map
      % \begin{equation}
      $
      \mathsf F \times \mathcal C \into \mathsf F \wr \mathcal C,
      $
      % \end{equation}
      for any category $\mathcal C$,
      and thus for any functor $F: \mathcal D \to \mathsf F$, we have an induced functor
      $F: \mathcal D \times \mathcal C \to \mathsf F \wr \mathcal C$.

      More generally, for any $G$-category $\mathcal C$ we have a natural ``diagonal'' map
      \begin{equation} 
            G \ltimes (\mathsf F \wr \mathcal C) \to \mathsf F \wr (G \ltimes \mathcal C).
      \end{equation}
\end{remark}

\begin{definition}[{cf. \cite[Defn 4.3]{BP17}}]
      Let $\mathsf{WSpan}^l(\mathcal C, \mathcal D)$ (resp. $\mathsf{WSpan}^r(\mathcal C, \mathcal D)$)
      denote the category of \textit{left (resp. right) weak spans}, with objects
      \begin{equation}
            \mathcal C \xleftarrow{k} \mathcal A \xrightarrow{X} \mathcal D
      \end{equation}
      and arrows those diagrams as on the left (resp. right) below
      \begin{equation}
            \begin{tikzcd}[row sep = tiny]
                  & \mathcal A_1 \arrow[dr, "X_1", ""'{name=U}] \arrow[dl, "k_1"'] \arrow[dd, "i"']
                  &
                  &&
                  &
                  \mathcal A_1 \arrow[dr, "X_1", ""'{name=A}] \arrow[dl, "k_1"'] \arrow[dd, "i"']
                  \\
                  \mathcal C
                  &&
                  \mathcal D
                  &&
                  \mathcal C
                  &&
                  \mathcal D
                  \\
                  & |[alias=V]| \mathcal A_2 \arrow[ur, "X_2"'] \arrow[ul, "k_2"]
                  &
                  &&
                  &
                  |[alias=B]| \mathcal A_2 \arrow[ur, "X_2"'] \arrow[ul, "k_2"]
                  \arrow[Rightarrow, from = U, to = V]
                  \arrow[Rightarrow, from = B, to = A]
            \end{tikzcd}
      \end{equation}
      denoted by $(i,\phi): (k_1,X_1) \to (k_2,X_2)$, with composition defined in the natural way.      
\end{definition}



The following lemmas have straightforward proofs.

\begin{lemma}
      \label{GD_PULL_LEM}
      For any functor $\mathcal C \to \mathcal D$ of $G$-categories, the squares below are Cartesian.
      \begin{equation}
            \begin{tikzcd}
                  \mathcal C \arrow[d] \arrow[r]
                  &
                  G \ltimes \mathcal C \arrow[d]
                  &&
                  G \ltimes (\Sigma \wr \mathcal C) \arrow[d] \arrow[r]
                  &
                  \Sigma \wr (G \ltimes \mathcal C) \arrow[d]
                  \\
                  \mathcal D \arrow[r]
                  &
                  G \ltimes \mathcal D
                  &&
                  G \ltimes (\Sigma \wr \mathcal D) \arrow[r]
                  &
                  \Sigma \wr (G \ltimes \mathcal D)
            \end{tikzcd}
      \end{equation}
      Moreover, the square on the left lifts Kan extensions.
\end{lemma}

\begin{lemma}
      \label{GL_PULL_LEM}
      The functor $G \ltimes (-): \Cat^{G} \to \mathsf{Fib}(G)$ preserves pullbacks and coproducts.
\end{lemma}

\begin{lemma}
      \label{GL_GR_LEM}
      $G \ltimes (-)$ preserves Grothendieck fibrations, and in fact the fibers remain constant.
\end{lemma}
\begin{proof}
      A straightforward diagram chase shows that if $f$ is a Cartesian arrow in $\mathcal E$ over $\mathcal B$,
      then for any $g\in G$, $f \circ g$ is a Cartesian arrow in $G \ltimes \mathcal E$ over $G \ltimes \mathcal B$.
      The result follows immediately.
\end{proof}

\begin{lemma}
      \label{GL_RANINIT_LEM}
      $G \ltimes (-)$ preserves Ran-initiality:
      if $\mathcal C \to \mathcal D$ is a functor of $G$-categories over another $G$-category $\mathcal E$
      which is Ran-initial, then so is $G \ltimes \mathcal C \to G \ltimes \mathcal D$ over $G \ltimes \mathcal E$.
\end{lemma}
\begin{proof} 
      This follows from the unique description of any arrow in $G \ltimes \mathcal E$ as an element of $G$ plus an arrow in $\mathcal E$. 
\end{proof}

\begin{remark}
      For $G$-categories $\mathcal C$ and $\mathcal D$, we have equivalences of categories
      \begin{equation}
            \Fun^G(\mathcal C, \mathcal D)
            \simeq \Fun_{\Fib(G)}(G \ltimes \mathcal C, G \ltimes \mathcal D)
            \simeq \Fun_{\Fib(G^{op})}(G^{op} \ltimes \mathcal C, G^{op} \ltimes \mathcal D).
      \end{equation}
\end{remark}

\begin{remark}
      $(G \ltimes \mathcal C)^{op} \cong G^{op} \ltimes \mathcal C^{op}$.
\end{remark}



\subsection{2-overcategories and pullback functors}

Include \S 8.1: 2-overcategories and \S 8.3: Pullback functors.

Random lemmas:

\begin{lemma}
      Suppose $F: \mathcal C \to \mathcal D$ is a simple 1-arrow in $\Cat \downarrow^r \mathcal B$.
      Consider the following cube in $\Cat$, where the bottom, top, and right faces are all (strict) pullbacks.
      \begin{equation}
            \begin{tikzcd}[row sep = small, column sep = small]
                  \pi^{\**} \mathcal C \arrow[rr] \arrow[dr] \arrow[dd]
                  &&
                  \mathcal E \arrow[dr, "\pi"] \arrow[dd, equal]
                  \\
                  &
                  \mathcal C \arrow[rr, crossing over]
                  &&
                  \mathcal B \arrow[dd, equal]
                  \\
                  \pi^{\**} \mathcal D \arrow[rr] \arrow[dr]
                  &&
                  \mathcal E \arrow[dr, "\pi"]
                  \\
                  &
                  \mathcal D \arrow[uu, crossing over, leftarrow] \arrow[rr]
                  &&
                  \mathcal B
            \end{tikzcd}
      \end{equation}
      Then the left square is also a pullback.
\end{lemma}
\begin{proof}
      We check that $\pi^{\**}\mathcal C$ has the correct universal property: for any category $Z$, we have
      \begin{align*}
        \Hom(Z,\mathcal C) \times_{\Hom(Z,\mathcal D)} \Hom(Z, \pi^{\**} \mathcal C)
        & =
          \Hom(Z, \mathcal C) \times_{\Hom(Z, \mathcal D)}\left(
          \Hom(Z, \mathcal D) \times_{\Hom(Z, \mathcal B)} \Hom(Z, \mathcal E)
          \right)
          \\
        & =
          \Hom(Z, \mathcal C) \times_{\Hom(Z, \mathcal B)}\Hom(Z, \mathcal E)
          = \Hom(Z, \pi^{\**}\mathcal C).
      \end{align*}
\end{proof}

\begin{corollary}
      \label{PI_GFIB_COR}
      If a simple 1-arrow $F:\mathcal C \to \mathcal D$ in $\Cat \downarrow^r \mathcal B$
      is an underlying Grothendieck fibration,
      then so is $\pi^{\**}F$.
\end{corollary}
\begin{proof}
      As Grothendieck fibrations are preserved by pullbacks, this follows from the previous lemma.
\end{proof}

\begin{lemma}
      \label{RANINIT_PULL_LEM}
      Suppose we have a triangle
      \begin{equation}
            \begin{tikzcd}
                  \mathcal C \arrow[rr, hookrightarrow] \arrow[dr, "p"] \arrow[ddr, bend right, "\delta"', ""{near end, name = V}]
                  &&
                  \mathcal D \arrow[dl, "p"'] \arrow[ddl, bend left, "\delta", ""'{near end, name = B}]
                  \\
                  &
                  |[alias = A]| \mathcal E \arrow[d, "\epsilon"]
                  \\
                  &
                  \mathsf F
                  \arrow[Rightarrow, from = A, to = B, "\Phi"]
                  \arrow[Rightarrow, from = A, to = V, "\Phi"']
            \end{tikzcd}
      \end{equation}
      in $\Cat \downarrow^r \mathsf F$,
      such that $\mathcal C \into \mathcal D$ is an inclusion of a subcategory by a simple 1-arrow.
      If $\mathcal C \to \mathcal D$ is $\Ran$-initial over $\mathcal E$,
      then $\mathcal C_{\mathfrak C} \to \mathcal D_{\mathfrak C}$ is $\Ran$-initial over $\mathcal E_{\mathfrak C}$.
\end{lemma}
\begin{proof}
      This is just a matter of unpacking definitions.
      Fixing some $(e, \epsilon(e) \xrightarrow{t} \mathfrak C)$ in $\mathcal E_{\mathfrak C}$,
      and $\ki = \begin{cases}
            (d, \delta(d) \xrightarrow{r} \mathfrak C)
            \\
            (e \longto p(d))
      \end{cases}$
      in $(e,t) \downarrow \mathcal D_{\mathfrak C}$,
      so in particular the triangle below commutes.
      \begin{equation}
            \begin{tikzcd}[row sep = small,column sep = tiny]
                  \epsilon(e) \arrow[rr] \arrow[ddr]
                  &&
                  \epsilon p(d) \arrow[d, "\Phi"]
                  \\
                  &&
                  \delta(d) \arrow[dl]
                  \\
                  &
                  \mathfrak C
            \end{tikzcd}
      \end{equation}
      We must show that $\left((e,t) \downarrow \mathcal C_{\mathfrak C} \right) \downarrow \ki$ is non-empty and connected.

      First, by hypothesis we know there exists $c \in \mathcal C$, $c \xrightarrow{f} d$, and $e \to p(c)$ such that the obvious triangle commutes.
      Now, the object 
      $\begin{cases}
            (c, \delta(c) \xrightarrow{\delta(f)} \delta(d) \to \mathfrak C)
            \\
            (e \longto p(c)
      \end{cases}$
      in $(e,t) \downarrow \mathcal C_{\mathfrak C}$ clearly maps to $\ki$.

      Second, any such object over $\ki$ must factor this way, and hence the connectedness of $(e \downarrow C) \downarrow (d,r)$
      implies the desired connectedness.
      \begin{equation}
            \begin{tikzcd}
                  &
                  \epsilon(e) \arrow[dr] \arrow[dl]
                  &
                  && %                  
                  &
                  \epsilon(e) \arrow[dr] \arrow[dl] \arrow[dd]
                  \\
                  \epsilon p(c) \arrow[rr, "{\epsilon p (f)}"] \arrow[d, "\Phi"]
                  &&
                  \epsilon p(d) \arrow[d, "\Phi"]
                  && %
                  \epsilon p (c) \arrow[dr, "{\epsilon p (f)}"'] \arrow[dd, "\Phi"'] \arrow[rr, dashed]
                  &&
                  \epsilon p (c') \arrow[dl, "{\epsilon p (f')}"] \arrow[dd, "\Phi"]
                  \\
                  \delta(c) \arrow[rr, "{P\delta(f)}"] \arrow[dr, "{r \circ \delta(f)}"']
                  &&
                  \delta(d) \arrow[dl, "r"]
                  && %
                  &
                  \epsilon p (d) \arrow[dd]
                  \\
                  &
                  \mathfrak C
                  &
                  && %
                  \delta (c) \arrow[dr, "{\delta(f)}"] \arrow[ddr] \arrow[rr, dashed]
                  &&
                  \delta(c') \arrow[dl, "{\delta(f')}"'] \arrow[ddl]
                  \\
                  &&&& %
                  &
                  \delta(d) \arrow[d]
                  \\
                  &&&& %
                  &
                  \mathfrak C
            \end{tikzcd}
      \end{equation}
\end{proof}




\subsection{Homotopy in a general model category}

\todo[inline]{come back: move whole section to appendix}

%{\color{OliveGreen}
We recall the following about homotopies in general model categories.
\begin{definition}
      For any $A \in \V$, a \textit{cylinder object for $A$} is an object $\mathbb C(A)$ equipped with a factorization of the fold map
      \begin{equation}
            \begin{tikzcd}
                  A \amalg A \arrow[r, tail]
                  &
                  \mathbb C(A) \arrow[r, "\simeq"]
                  &
                  A
            \end{tikzcd}
      \end{equation}
      into a cofibration followed by a weak equivalence.
      
      For the tensor unit $1_\V$, we write $\mathbb C = \mathbb C(1_\V)$, and call this simply a \textit{cylinder} in $\V$.
      % A cylinder for $A$ is called \textit{good} if the second map is a trivial fibration.
      
      A (left) \textit{homotopy} between maps $f,g: A \to B$ in $\V$ is a map $H_{fg}: \mathbb C(A) \to \V(A,B)$ such that
      the diagram below commutes.
      \begin{equation}
            \begin{tikzcd}[row sep = tiny]
                  A \amalg A \arrow[rr, "{(f,g)}"] \arrow[dr]
                  &&
                  B
                  \\
                  &
                  \mathbb C(A) \arrow[ur, "H_{fg}"']
            \end{tikzcd}
      \end{equation}
      We say $f$ and $g$ are \textit{homotopic} if there exists a homotopy $H_{f g}$ between them.
\end{definition}

% \begin{remark}
%       If $B$ is fibrant, we may lift any homotopy to a homotopy out of a good cylinder,
%       using the functorial factorization
%       $\mathbb C(A) \overset{\sim}{\rightarrowtail} \mathbb C'(A) \xrightarrow{\sim}{\twoheadrightarrow} A$.
% \end{remark}

\begin{remark}
      \label{CYL_REM}
      For $\mathbb C$ a cylinder in $\V$,
      if $A \in \V$ is cofibrant, then $A \otimes \mathbb C$ is a cylinder object for $A$,
      as the fold map can be written
      \begin{equation}
            A \amalg A \simeq A \otimes (1_\V \amalg 1_\V) \rightarrowtail A \otimes \mathbb C \xrightarrow{\sim} A
      \end{equation}
      as $A \otimes (-)$ preserves cofibrations, and, by Ken Brown's Lemma, weak equivalences between cofibrant objects.
\end{remark}

These cylinder objects provide another description of the mapping sets in the homotopy category $\Ho \V$ of $\V$.

\begin{proposition}       [{\cite[1.2.10]{Hov99}}]
      If $A$ is fibrant and $B$ cofibrant, then
      homotopy is an equivalence relation $\sim$ on $\V(A,B)$.
      Moreover, 
      $\Ho \V (A,B) = \V(A_f, B_c)/\sim$.
\end{proposition}


\todo[inline]{move this discussion to the operads section}
We can use these cylinder objects to extend the notion of homotopy to $\V$-categories or $\V$-operads.

\begin{definition}
      \label{HTPY_DEFN}
      Given $\mathcal C \in \Cat(\V)$, define $\pi_0 \mathcal C$ to be the (unenriched) category with
      the same objects as $\mathcal C$, and $\pi_0 \mathcal C (c,d) = \Ho(\V)(1_\V, \mathcal C(c,d))$.

      We say maps $f,g \in \mathcal C(c,d)$ are \textit{homotopic}
      if the representing maps $f,g: 1_\V \to \mathcal C(c,d)$ are homotopic in $\V$.
      If $\mathcal C$ is fibrant, this is equivalent to $[f] = [g]$ in $\pi_0\mathcal C$.

      We say maps $f,g \in \O(c;d)$ are \textit{homotopic} if they are homotopic in $j^{\**}\O$. 
\end{definition}

\begin{lemma}
      If $\O \in \Op^{G, \mathfrak C}(\V)$ is cofibrant,
      and $f,g: 1_\V \to \O(c;d)$ are homotopic, then for any $\mathfrak C(\O)$ profile $\ksi$ such that either
      \begin{itemize} %{enumerate}[label = (\arabic*)]
      \item the target of $\ksi$ is $c$, or
      \item $d$ is in the source of $\ksi$,
      \end{itemize}
      the maps
      \begin{equation}
            f_{\**}, g_{\**}: \O(\ksi) \simeq \O(\ksi) \otimes 1_\V \to \O(\ksi) \otimes \O(c;d) \xrightarrow{\circ} \O(\ksi_c^d)
      \end{equation}
      are homotopic,
      where $\ksi_c^d$ is the profile created from $\ksi$ by replacing $c$ with $d$.
\end{lemma}
\begin{proof}
      This follows from Remark \ref{CYL_REM} and the following commuting diagram.
      \begin{equation}
            \begin{tikzcd}
                  \O(\ksi) \otimes (1_\V \amalg 1_\V) \arrow[r, "{(f_{\**}, g_{\**})}"] \arrow[d, tail]
                  &
                  \O(\ksi_c^d)
                  \\
                  \O(\ksi) \otimes \mathbb C \arrow[r, "H_{fg}"]
                  &
                  \O(\ksi) \otimes \O(c;d) \arrow[u, "{(-)_{\**}}"]
            \end{tikzcd}
      \end{equation}
\end{proof}

The following is immediate.
\begin{corollary}
      \label{HTPIC_ISO_COR}
      Let $\O$ be level bifibrant in $\Op(\V)$. 
      If $f \in \O(c;c)$ is homotopic to the identity on $c$, then for any profile $\ksi$ containing $c$, the map
      $f_{\**}: \O(\ksi) \to \O(\ksi)$
     is an isomorphism in the homotopy category $\Ho(\V)$.
\end{corollary}


% ------------------------------ ASSEMBLING HOMOTOPIES ------------------------------


\begin{definition}[{cf. \cite[Defn 6.16]{BP17}}]
        We say a symmetric monoidal model category $\V$ has \textit{cofibrant symmetric pushout powers} if
        for all (trivial) cofibrations $f$, the pushout product power $f^{\square n}$
        is a $\Sigma_n$-genuine (trivial) cofibration in $\V^{\Sigma_n}$. 
\end{definition}

We may assemble homotopies in the following manner.

\begin{lemma}
      \label{ASSEM_HOM_LEM}
      Suppose $\V$ has cofibrant symmetric pushout powers.
      If $\mathbb C$ is a cylinder, then so is each $\left(\mathbb C^{\otimes n}\right)^{K}$ for all $K \leq \Sigma_n$.
\end{lemma}
\begin{proof}
      % It suffices to show there exist
      % $1_\V \amalg 1_\V \rightarrowtail (\mathbb C^{\otimes n})^K$
      % and
      % $(\mathbb C^{\otimes n})^K \xrightarrow{\sim} (1_\V^{\otimes n})^K \simeq 1_\V$
      % (where the coherence axioms imply that $1_\V^{\otimes n}$ always has a trivial $\Sigma_n$-action).
      % 
      We consider each structure map separately.
      
      We note that, as $\otimes$ commutes with colimits,
      $(1_\V \amalg 1_\V)^{\otimes n} \simeq \coprod_{\chi} 1_{\V,\chi}$
      with $\chi$ running over all set maps $\underline{n} \to \set{0,1}$,
      and $\Sigma_n$ acting by pre-composition on $\chi$.
      Now, we have the composite
      \begin{equation}
%            \begin{tikzcd}
            1_\V \amalg 1_\V = 1_{\V, 0} \amalg 1_{\V, 1}
            \simeq
            \left((1_\V \amalg 1_\V)^{\otimes n}\right)^{\Sigma_n}
            \longrightarrow
            (1_\V \amalg 1_\V)^{\otimes n}
            \longrightarrow
            \mathbb C^{\otimes n}
%            \end{tikzcd}
      \end{equation}
      where $i: \underline{n} \to \set{i} \into \set{0,1}$ is the constant map,
      the first arrow is a genuine $\Sigma_n$-cofibration
      as we may attach each $\Sigma_n$-orbit $\Sigma_n \chi$ individually via maps $\varnothing \to \amalg_{\V,\sigma\chi}$,
      % (and we've already attached the stable orbits).
      and the second arrow is a genuine $\Sigma_n$-cofibration since $\V$ has cofibrant symmetric pushout powers.

      Now,
      we note that $(\mathbb C \to 1_\V)^{\otimes n}$ is a weak equivalence by induction using Ken Brown's lemma,
      as $\mathbb C^{\otimes n}$ is cofibrant,
      $\mathbb C^{\otimes n} \to 1_\V^{\otimes n}$ is a map between cofibrant objects,
      and $\mathbb C \otimes (-)$ preserves all trivial cofibrations.
      % 
      Let $Q(n) \to \mathbb C^{\otimes n}$ denote $(1_\V \to \mathbb C)^{\square n}$,
      which is a genuine trivial cofibration in $\V^{\Sigma_n}$ by the assumption on $\V$.
      Consider the pushout $P$ and induced maps in $\V^{\Sigma_n}$ below
      \begin{equation}
            \begin{tikzcd}
                  Q(n) \arrow[r, tail, "\sim"] \arrow[d, tail, "\sim"']
                  &
                  \mathbb C^{\otimes n} \arrow[d, tail, "\sim"] \arrow[drr, bend left, "\sim"]
                  \\
                  \mathbb C^{\otimes n} \arrow[r, tail, "\sim"] \arrow[rrr, bend right, "\sim"]
                  &
                  P \arrow[r, tail, dashed, "\sim"]
                  &
                  P' \arrow[r, two heads, "\sim", dashed]
                  &
                  1_\V^{\otimes n}
            \end{tikzcd}
      \end{equation}
      where we have factored the unique map $P \to 1_\V^{\otimes n}$ into a cofibration and fibration.

      Thus, as (the proof of) \cite[Prop 6.3]{BP17} shows that $(-)^H$ preserves pushouts over genuine cofibrations
      as well as genuine \textit{trivial} cofibrations,
      and since $(-)^H$ preserves trivial fibrations by construction,
      we have the string of maps in $\V$ below for any $K \leq \Sigma_n$
      \begin{equation}
            \begin{tikzcd}
                  1_\V \amalg 1_\V \simeq \left((1_\V \amalg 1_\V)^{\otimes n}\right)^{\Sigma_n} \arrow[r, hookrightarrow]
                  &
                  \left((1_\V \amalg 1_\V)^{\otimes n}\right)^K \arrow[r, hookrightarrow]
                  &
                  (\mathbb C^{\otimes n})^K \arrow[r, tail, "\sim"]
                  &
                  P^K \arrow[r, tail, "\sim"]
                  &
                  P'^K \arrow[r, two heads, "\sim"]
                  &
                  (1_\V^{\otimes n})^K \simeq 1_\V.
            \end{tikzcd}
      \end{equation}
\end{proof}

\begin{remark}
      \label{ASSEM_HOM_REM}
      The inclusions $i_k: 1_\V \to \mathbb C$, $k \in \set{0,1}$ factors through $\mathbb C^{\otimes n}$ as the composite below
      \begin{equation}
            1_\V
            \xrightarrow{i_k}
            \mathbb C^{\otimes n}
            \xrightarrow{\sim}
            1_\V^{\otimes j-1} \otimes \mathbb C \otimes 1_\V^{\otimes n-j}
            \xrightarrow{\simeq}
            \mathbb C
      \end{equation}
      for any $j \in \set{n}$.
      Indeed, the following diagram commutes since cylinders factor the fold map $\nabla$.
      \begin{equation}
            \begin{tikzcd}
                  1_\V \amalg 1_\V \arrow[r]
                  &
                  \coprod_\chi 1_{\V, \chi} \arrow[r, "\simeq"] \arrow[d, "\nabla \otimes id \otimes \nabla"]
                  &
                  (1_\V \amalg 1_\V)^{\otimes n} \arrow[r] \arrow[d, "\nabla \otimes id \otimes \nabla"]
                  &
                  \mathbb C^{\otimes n} \arrow[d]
                  \\
                  &
                  1_{\V, j\mapsto 0} \amalg 1_{\V, j \mapsto 1} \arrow[r, "\simeq"]
                  &
                  1_\V^{\otimes j-1} \otimes (1_\V \amalg 1_\V) \otimes 1_\V^{n-j} \arrow[r]
                  &
                  1_\V^{\otimes j-1} \otimes \mathbb C \otimes 1_\V^{n-j}
            \end{tikzcd}
      \end{equation}
      
      Thus, for any homotopy $\mathbb C^{\otimes n} \to \mathcal C$,
      there are $n$ associated homotopies $x_{i,0} \sim x_{i,1}$,
      and the inclusion $1_{\V, \chi} \to \mathbb C^{\otimes n}$ ``sees'' the tuple of objects $(x_{i,\chi(i)})$.
\end{remark}

% ------------------------------ diagonals for cylinder objects ----------

% \begin{definition}
%       The category $\V$ is said to \textit{have diagonals for cylinder objects} if
%       for any cylinder object $\mathbb C$ and $n \geq 0$ there exists a map
%       \begin{equation}
%             \Delta: \mathbb C \to \left(\mathbb C^{\otimes n}\right)^{\Sigma_n}
%       \end{equation}
%       such that the following diagram commutes.
%       \begin{equation}
%             \begin{tikzcd}
%                   1_\V \amalg 1_\V \arrow[r] \arrow[d]
%                   &
%                   \left((1_\V \amalg 1_\V)^{\otimes n}\right)^{\Sigma_n} \arrow[d]
%                   \\
%                   \mathbb C \arrow[r, "\Delta"]
%                   &
%                   \left(\mathbb C^{\otimes n}\right)^{\Sigma_n}
%             \end{tikzcd}
%       \end{equation}
% \end{definition}

% This implies the diagram below commutes for all $k \in \underline{n}$ (since $\mathbb C$ factors the fold map).
% \begin{equation}
%       \begin{tikzcd}
%             1_\V \arrow[rrr, "i_k"] \arrow[d, "i_k"]
%             &&&
%             \mathbb C
%             % &&&
%             % 1_\V \arrow[lll, "i_1"'] \arrow[d, "i_1"]
%             \\
%             \mathbb C \arrow[r, "\Delta"]
%             &
%             \mathbb C^{\otimes n} \arrow[r]
%             &
%             1_\V^{\otimes k-1} \otimes \mathbb C \otimes 1_\V^{\otimes n-k} \arrow[r, "\simeq"]
%             &
%             \mathbb C \arrow[u, equal]
%             % &
%             % \mathbb C \otimes 1_\V^{\otimes n-1} \arrow[l, "\simeq"']
%             % &
%             % \mathbb C^{\otimes n} \arrow[l]
%             % &
%             % \mathbb C \arrow[l, "\Delta"']
%       \end{tikzcd}
% \end{equation}

% It suffices to show the map $\mathbb C^{\otimes n} \to 1_\V^{\otimes n} \simeq 1_\V$ is a trivial $\Sigma_n$-fibration, as a lift of
% \begin{equation}
%       \begin{tikzcd}
%             1_\V \amalg 1_\V \arrow[r] \arrow[d, tail]
%             &
%             \left((1_\V \amalg 1_\V)^{\otimes n}\right)^{\Sigma_n} \arrow[r]
%             &
%             \left(\mathbb C^{\otimes n}\right)^{\Sigma_n} \arrow[d]
%             \\
%             \mathbb C \arrow[rr, two heads, "\simeq"]
%             &&
%             1_\V \simeq 1_\V^{\otimes n} \simeq \left(1_\V^{\otimes n}\right)^{\Sigma_n}
%       \end{tikzcd}
% \end{equation}
% would satisfy these properties.
% \todo[inline]{come back: this need not happen. It may only be a weak equivalence.}
% 
% } % END OF OLIVE GREEN


\section{Colored Operads}


\subsection{Non-Equivariant Colored Operads}

Fix a closed symmetric monoidal category $(\V, \otimes)$.

\begin{definition}
      Fix a set $\mathfrak C$ of \textit{colors}.
      A tuple
      $\ksi = (c_1, \ldots, c_n; c_0) \in \mathfrak C^{\times n} \times \mathfrak C$
      is called a \textit{$\mathfrak C$-signature}.
      A collection $\ksi, \ksi_1,\dots, \ksi_n$ of $\mathfrak C$-signatures is called \textit{compatible} if
      $\ksi \in \mathfrak C^{\times n+1}$, and the target of $\ksi_i$ is the $i$-th source of $\ksi$;
      e.g.  $\xi = (c_1, \ldots, c_n; c_0)$, $\xi_i = (c_{1}^i, \ldots, c_{m_i}^i; c_i)$.
      In this case, define $\ksi \circ (\ksi_i) = (c_1^1,c_2^1,\dots,c_{m_n}^{n}; c_0)$.
\end{definition}

\begin{definition}
      For a set $\mathfrak C$, define the \textit{$\mathfrak C$-symmetric category} $\Sigma_{\mathfrak C}$
      \begin{equation}
            \Sigma_{\mathfrak C} = \coprod_{n \geq 0} \Sigma_n \ltimes \mathfrak C^{\times n+1} = \Sigma^{+1} \wr \mathfrak C,
      \end{equation}
      where $\mathfrak C^{\times n+1}$ has a natural \textit{right} $\Sigma_n$-action on the first $n$-coordinates.
      A \textit{$\mathfrak C$-symmetric sequence} is a functor $X: \Sigma_{\mathfrak C}^{op} \to \V$.
      Explicitly, this consists of
      \begin{itemize} %{enumerate}[label = (\arabic*)]
      \item An object $X(\ksi) \in \V$ for each signature $\ksi$.
      \item For all signatures $\ksi \in \mathfrak C^{\times n+1}$ and $\sigma \in \Sigma_n$,
            unital and associative times maps $X(\xi) \to X(\sigma \cdot \xi)$,
            where $\Sigma_n$ acts on the first $n$ coordinates of $\mathfrak C^{\times n+1}$;
            e.g. maps
            \begin{equation}
                  X(c_1, \ldots, c_n; c_0) \xrightarrow{\sigma} X(c_{\sigma^{-1}1}, \ldots, c_{\sigma^{-1}n}; c_0).
            \end{equation}
      \end{itemize}
\end{definition}

\begin{definition}
      A \textit{$\mathfrak C$-colored operad} in $\V$ 
      is a $\mathfrak C$-symmetric sequence $\O$ along with appropriate composition laws \footnote{
        (Colored) operads are also known as \textit{symmetric multicategories}.}.
      Explicitly, this consists of the following data:
      \begin{itemize} %{enumerate}[label = (\arabic*)]
      \item An object $\O(\ksi) \in \V$ for each signature $\ksi$.
      \item For all signatures $\ksi \in \mathfrak C^{\times n+1}$ and $\sigma \in \Sigma_n$,
            unital and associative times maps $\O(\xi) \to \O(\sigma \cdot \xi)$,
            where $\Sigma_n$ acts on the first $n$ coordinates of $\mathfrak C^{\times n+1}$;
            i.e., maps
            \begin{equation}
                  \O(c_1, \ldots, c_n; c_0) \xrightarrow{\sigma} \O(c_{\sigma^{-1}1}, \ldots, c_{\sigma^{-1}n}; c_0).
            \end{equation}
      \item For each $c \in \mathfrak C$, a \textit{unit} $1_c \in \O(c;c)$.                        
      \item For all compatible signatures $\ksi$, $\ksi_1$, $\dots$, $\ksi_n$,
            \textit{composition} maps
            \begin{equation}
                  \O(\xi) \otimes \O(\xi_1) \otimes \ldots \otimes \O(\xi_n) \to \O(\xi \circ (\xi_i)),
            \end{equation}
            which are unital, associative, and appropriately $\Sigma$-equivariant.
      \end{itemize}

      A map of $\mathfrak C$-colored operads is a collection of maps
      $\set{\O(\xi) \to \O'(\xi)}_{\xi}$
      which commutes with the above structure maps.
      
      Let $\Op^{\mathfrak C}(\V)$ denote the category of $\mathfrak C$-colored operads in $\V$.
\end{definition}

\begin{remark}
      There is a natural monad $\mathbb F^{\mathfrak C}$ on $\Sym^{\mathfrak C}(\V)$ such that
      the category of $\mathbb F^{\mathfrak C}$-algebras is precisely $\Op^{\mathfrak C}(\V)$.
      This monad will be explored more (especially in the equivariant context) in Section \ref{COMEGA_SEC}.
\end{remark}

\begin{definition}
      \label{OP_MAP_DEFN}
      Given a map of $G$-sets $F: \mathfrak C' \to \mathfrak C$ and a $\mathfrak C$-symmetric sequence $X$
      there is a natural $\mathfrak C'$-symmetric sequence $F^{\**}X$
      \begin{equation}
            F^{\**}X(\xi') = X(f(\xi')),
      \end{equation}
      which is an operad if $X$ was an operad.
      In fact, we have a pair of adjunctions as below for any such $F$.
      \begin{equation}
            \label{C_CHANGE_EQ}
            \begin{tikzcd}
                  \Op^{\mathfrak C'}(\V) \arrow[r, shift right, "F^{\**}"'] \arrow[d, "\mathsf{fgt}"']
                  &
                  \Op^{\mathfrak C}(\V) \arrow[l, shift right, "F_!"'] \arrow[d, "\mathsf{fgt}"]
                  \\
                  \Sym^{\mathfrak C'}(\V) \arrow[r, shift right, "F^{\**}"']
                  &
                  \Sym^{\mathfrak C}(\V) \arrow[l, shift right, "F_{!!}"'].
            \end{tikzcd}
      \end{equation}
      where we highlight that $F^{\**}$ commutes with $\mathsf{fgt}$, but the left adjoint does not.

      We let $\Sym(\V) = \left(\mathsf F^{op} \ltimes \Sym^{(-)}(\V)\right)^{op}$ and $\Op(\V) = \left(\mathsf F^{op} \ltimes \Op^{(-)}(\V)\right)^{op}$
      denote the categories of
      \textit{(colored) symmetric sequences} and \textit{(colored) operads}
      defined by the Grothendieck constructions on the functors
      \begin{align*}
        \mathsf F^{op} \longrightarrow \Cat,
        \qquad \qquad
        &
          \mathfrak C \longmapsto \Sym^{\mathfrak C}(\V)
          \mbox{ or }
          \mathfrak C \longmapsto \Op^{\mathfrak C}(\V).
      \end{align*}
      
      Explicitly, a map of sequences (resp. operads) $X \to Y$ is given by a map of colors
      $F: \mathfrak C(X) \to \mathfrak C(Y)$
      and a map of $\mathfrak C(X)$-colored sequences (operads)
      $X \to F^{\**} Y$.
\end{definition}

\subsection{Equivariant Colored Operads}

Now and forever, fix a finite group $G$.

\begin{convention}
      \label{G_CONV}
      We make the following conventions throughout the rest of the paper.
      \begin{itemize} %{enumerate}[label = (\roman*)]
      \item We equip $G$ with a fixed total order.
      \item All $G$-objects will be \textit{left} $G$-objects,
            and we let $BG$ denote the category which encodes left actions.            
      \end{itemize}
\end{convention}

% \begin{remark}
%       \label{G_GR_REM}
%       For any (left) $G$-set $A$, the opposite of the Grothendieck construction $(G \ltimes A)^{op}$
%       on the functor characterizing the $G$-action 
%       \begin{equation}
%             G^{op} \longto \Set, \qquad \qquad \** \longmapsto A
%       \end{equation}
%       is equal to the \textit{translation category}, often denoted $B_A G$,
%       where objects are elements of $A$, and arrows $g: a \to b$ with  $b = g a$.
%       Dually, the standard Grothendieck construction $G \ltimes A$ has arrows
%       $g: a \to b$ with $a = g b$.
      
%       Moreover, if $A$ has a (left) action by $G \times \Sigma$ for another group $\Sigma$, then
%       iterating either type of Grothendieck construction above is associative and commutative.
% \end{remark}

\begin{notation}
      [{cf. \cite{BP17}}]
      Let $\mathsf F^G$ denote the category of finite ordered $G$-sets (see Notation \ref{F_WR_NOT} for our convention on $\mathsf F$).
      Moreover, let $\mathsf O_G \into \mathsf F^G$ denote the full subcategory of \textit{transitive} $G$-sets.
      In particular, we note that the oribts $G/H$ are well-defined
      (using the chosen total order on $G$
      and the ``minimal representative'' total order on $G/H$).
\end{notation}

\begin{definition}
      The category $\Op^G(\V)$ of  \textit{$G$-colored operads} in $\V$ is the category of
      (left) $G$-objects in $\Op(\V)$.
\end{definition}

% Unpacking this definition, we see $\O \in \Op^G(\V)$ consists of the following data:
% \begin{enumerate}[label = (\arabic*)]
%       \setcounter{enumi}{-1}
% \item A $G$-set $\mathfrak C$ of colors.
% \item For each signature $\xi$ of $\mathfrak C$, an object $\O(\xi) \in \V$.
% \item where $G$ acts on $\mathfrak C^{\times n+1}$ diagonally (across all $n+1$ coordinates), and $\Sigma_n$ acts on the first $n$.
% \item For each $c \in \mathfrak C$, a \textit{unit} $1_c \in \O(c;c)^{G_c}$.
% \item For compatible signatures $\xi$, $\xi_1$, $\ldots$, $\xi_n$, \textit{composition maps}
%       \begin{equation}
%             \O(\xi) \otimes \O(\xi_1) \otimes \ldots \otimes \O(\xi_n) \to \O(\xi \circ (\xi_i)),
%       \end{equation}
% \end{enumerate}
% such that composition is
% compatible with the $G$-action on each component as well as the appropriate actions of $\Sigma$,
% and is unital and associative. 

We would like to unpack this definition, and first do so in general.

% -------------------- Equivariant Grothendieck Constructions --------------------

Suppose $E = (\mathcal D^{op} \ltimes \mathcal C_{(-)})^{op}$ is the contravariant Grothendieck construction on the functor
\begin{equation}
      % \begin{tikzcd}[row sep = tiny]
      %       \mathcal D^{op} \arrow[r]
      %       &
      %       \mathsf{Cat}
      %       \\
      %       d \arrow[r, mapsto]
      %       &
      %       \mathcal C_{d}.
      % \end{tikzcd}
      \mathcal D^{op} \longto \Cat,
      \qquad \qquad
      d \longmapsto \mathcal C_d.
\end{equation}

A (left) $G$-action on an element $(d,c_d)$ is given by compatible maps
$d \xrightarrow{g} d$ and
$c_d \xrightarrow{g_c} g^{\**}c_d$
(with the $g_c$ compatible in the sense that
\begin{equation}
      \begin{tikzcd}
            c_d \arrow[r, "g_c"] \arrow[dr, "(h g)_c"']
            &
            g^{\**} c_d \arrow[d, "g^{\**} h_c"]
            \\
            &
            (h g)^{\**} c_d
      \end{tikzcd}
\end{equation}
commutes),
and $G$-maps between such objects
are pairs of maps $(d \xrightarrow{f} d, c_d \xrightarrow{f_c} f^{\**}c_{d'})$ which commute with the action.

Equivalently, and more concretely,
an object in $E^G$ is given by a pair $(d, \underline{c}_d)$ where
$d \in \mathcal D^G$ and
$\underline{c}_d$ is a $G/e$-indexed diagram
% \footnote{
%   We recall that $(G \ltimes (G/e))^{op} = B_{G/e} G$.}
$E G := G \ltimes (G/e) \to \mathcal C_d$,
such that $\underline{c}_d(g) = g_{d}^{\**}\underline{c}_d(e)$.
A $G$-map in this context is given by a pair $(f, \Phi_c)$ where
$f: d \to d'$ in $\mathcal D^G$ and
$\Phi: \underline{c}_d \Rightarrow f^{\**}\underline{c}_{d'}: E G \to \mathcal C_d$
(where we note that $f^{\**}\underline{c}_{d'}$ is a functor since $f$ is a $G$-map).

Put another way, we have the following.
\begin{lemma}
      \label{G_GR_LEM}
      The category of $G$-objects
      $\left(\mathcal D^{op} \ltimes \mathcal C_{(-)}\right)^{op,G}$
      is given by the Grothendieck construction on the functor
      \begin{equation}
            % \begin{tikzcd}[row sep = tiny]
            %       \mathcal D^{G,op} \arrow[r]
            %       &
            %       \Cat
            %       \\
            %       d \arrow[r, mapsto]
            %       &
            %       \Fun^{G^{op}}(E G^{op}, \mathcal C_d)
            % \end{tikzcd}
            \mathcal D^{G,op} \longrightarrow \Cat,
            \qquad \qquad
            d \longmapsto \Fun^{G^{op}}(E G, \mathcal C_d)
      \end{equation}
      where $\Fun^{G^{op}}$ denotes the category of $G^{op}$-functors and $G^{op}$-natural transformations.
\end{lemma}
\begin{proof}
      We observe that $EG$ has a natural $G^{op}$-action via the \textit{right} action of $G$ on $G/e$,
      and $\mathcal C_d$ has a $G^{op}$ action induced by the (left) $G$-action on $d$.
      The rest follows by unpacking definitions.
      {\color{OliveGreen}
        A map $\underline{c}: EG \to \mathcal C_d$ gives elements $c_g$ and maps $(g,h): c_g \to c_{hg}$ such that
        $(g, kh) = (hg, k) \circ (g, h)$.
        $x \in G^{op}$ acts on $EG$ by sending $g$ to $g x$ and the map $(g,h)$ to $(g x,h)$,
        and on $\mathcal C_d$ by sending $c$ to $x^{\**}c$ and $f$ to $x^{\**}f$.
        Equivariance implies that $c_x = x^{\**}c_e$ and $(x,g) = x^{\**}(e,g)$, finishing the proof.
      }
\end{proof}

% ----------------------------------------------------------------------

We apply this result to both $\Sym(\V) = (\mathsf F^{op} \ltimes \Sym^{(-)}(\V))^{op}$ and $\Op(\V) = (\mathsf F^{op} \ltimes \Op^{(-)}(\V))^{op}$.

\begin{example}
      For the category $E = \Sym^G(\V) = (\Sym(\V))^G$ of \textit{$G$-symmetric sequences},
      we note that, for any $G$-set $\mathfrak C$, we have
      \begin{equation}
            (E G \times \Sigma_{\mathfrak C}^{op})/G \simeq G \ltimes \Sigma_{\mathfrak C}^{op},
      \end{equation}
      where $\Sigma_{\mathfrak C}$ inherits a $G$-action from the action on $\mathfrak C$;
      we refer to this later category
      \begin{equation}
            G \ltimes \Sigma_{\mathfrak C}^{op} = \coprod_{n \geq 0}(G \times \Sigma_n^{op}) \ltimes \mathfrak C^{\times n+1}
      \end{equation}
      as the \textit{$(G, \mathfrak C)$-symmetric category},
      and the category $\Sym^{G, \mathfrak C}(\V)$ of functors $X: G \ltimes \Sigma_{\mathfrak C} ^{op}\to \V$
      the category of \textit{$(G, \mathfrak C)$-symmetric sequences}.
      % 
      Then, by adjunction and Lemma \ref{G_GR_LEM}, we have that
      $\Sym^G(\V)$ is isomorphic to the Grothendieck construction on the functor
      \begin{equation}
            \mathsf F^{G,op} \longrightarrow \Cat,
            \qquad
            \qquad
            \mathfrak C \longmapsto \Sym^{G, \mathfrak C}(\V).
      \end{equation}
\end{example}

\begin{remark}
      When $G = \set{e}$ and $\mathfrak C = \set{*}$, $G \ltimes \Sigma_{\mathfrak C}^{op} = \Sigma^{op}$.
\end{remark}


\begin{example}
      Now, for $E = \Op(\V)$,
      we see that an object $\O \in \Op^G(\V)$ consists of
      a $G$-set $\mathfrak C$ of colors, and a compatible $G/e$-indexed diagram of $\mathfrak C$-colored operads;
      in particular, $\O$ has an underlying object in $\Sym^G(\V)$
      (i.e., satisfying $(GO\ 0)$ through $(GO\ 2)$ below)
      by composition with the forgetful functor $\Op^{\mathfrak C}(\V) \to \Sym^{\mathfrak C}(\V)$.
      % 
      Explicitly, an object $\O \in \Fun^{G^{op}}(E G, \Op^{\mathfrak C}(\V)) =: \Op^{G, \mathfrak C}(\V)$
      is given by the following data:
      \begin{enumerate}[label = (\arabic*), start = 0]
      \item A $G$-set $\mathfrak C = \mathfrak C(\O)$ of colors.
      \item For each $\mathfrak C$-signature $\ksi$, an object $\O(\ksi) \in \V$.
      \item \label{SACTION_LBL}
            For all signatures $\ksi \in \mathfrak C^{\times n + 1}$ and $\sigma \in \Sigma_n$,
            maps $\O(\ksi) \to \O(\sigma \cdot \ksi)$
            which are unital and associative.
      \item \label{GUNIT_LBL} 
            For each $c \in \ksi$, a \textit{unit} $1_c \in \O(c;c)$.
      \item \label{COMP_LBL}
            For all compatible signatures $\ksi, \ksi_1,\dots, \ksi_n$,
            \textit{composition maps} $\O(\ksi) \otimes \O(\ksi_1) \otimes \dots \otimes \O(\ksi_n) \to \O(\ksi \circ (\ksi_i))$
            which are unital, associative, and appropriately $\Sigma$-equivariant
      \item \label{GACTION_LBL}
            For all $g \in G$, maps $\O(\ksi) \to g^{\**}\O(\ksi) = \O(g \cdot \ksi)$
            (with $G$ acting on $\mathfrak C^{\times n+1}$ diagonally)
            which are unital and associative, and which commute with the composition maps
            (note that if $\ksi, (\ksi_i)$ are compatible, then so are $g \ksi, (g \ksi_i)$).
      \end{enumerate}
      Synthesizing, we may combine \ref{SACTION_LBL} and \ref{GACTION_LBL} into
      \begin{enumerate}
      \item[($2'$)] For all signatures $\xi \in \mathfrak C^{\times n+1}$ and $(g,\sigma) \in G\times \Sigma_n$, maps
            $\O(\xi) \to \O((g,\sigma)\cdot \xi)$
            which are unital and associative.
      \end{enumerate}
      
      and replace \ref{GUNIT_LBL} and \ref{COMP_LBL} with
      \begin{enumerate}
      \item[($3'$)] For each $c \in \mathfrak C$, a $G_c$-fixed unit $1_c \in \O(c;c)^{G_c}$.
      \item[($4'$)] For all compatible signatures $\ksi, (\ksi_i)$,
            composition maps $\O(\ksi) \otimes \bigotimes_i \O(\ksi_i) \to \O(\ksi \circ (\ksi_i))$
            which are unital, assocative, $G$-equivariant, and approriately $\Sigma$-equivariant.
      \end{enumerate}
      
      From Lemma \ref{G_GR_LEM}, we can see that
      maps of $G$-operads $\O \to \P$ are given by a pair $(f, F)$ where
      $f: \mathfrak C(\O) \to \mathfrak C(\P)$ is a map of $G$-sets, and
      $F: \O \to f^{\**}\P$ is a map in $\Op^{\mathfrak C(\O)}(\V)$ that commutes with the $G$-action
      (or, equivalently, so that the $g^{\**}F$ assemble into a map of $G$-indexed diagrams).
\end{example}

As in \eqref{C_CHANGE_EQ}, we in fact have a pair of adjoints
\begin{equation}
      \label{GC_CHANGE_EQ}
      \begin{tikzcd}
            \Op^{G, \mathfrak C'}(\V) \arrow[r, shift right, "F^{\**}"'] \arrow[d, "\mathsf{fgt}"']
            &
            \Op^{G, \mathfrak C}(\V) \arrow[l, shift right, "F_!"'] \arrow[d, "\mathsf{fgt}"]
            \\
            \Sym^{G, \mathfrak C'}(\V) \arrow[r, shift right, "F^{\**}"']
            &
            \Sym^{G, \mathfrak C}(\V) \arrow[l, shift right, "F_{!!}"'].
      \end{tikzcd}
\end{equation}
where again only the right adjoint commutes with $\mathsf{fgt}$. 
%
Each $\Op^{G, \mathfrak C}(\V) \subseteq \Op^G(\V)$
is hence the subcategory of $\mathfrak C$-colored operads and maps which are the identity on colors.

\begin{lemma}
      $\Op^G(\V)$ is isomorphic to the Grothendieck construction on the functor
      \begin{equation}
            % \begin{tikzcd}[row sep = tiny]
            %       \mathsf (F^G)^{op} \arrow[r] & \mathsf{Cat}
            %       \\
            %       \mathfrak C \arrow[r, mapsto] & \Op^{G,\mathfrak C}(\V).
            % \end{tikzcd}
            \mathsf F^{G,op} \longto \Cat,
            \qquad \qquad
            \mathfrak C \longmapsto \Op^{G, \mathfrak C}(\V).
      \end{equation}
\end{lemma}


\begin{remark}
      \label{COLOR_SQUARE_REM}
      We also record the straightforward observation that if a map $F: \O_1 \to \O_2$ is color-fixed, then
      a commuting square (resp. lifting diagram, pullback, pushout) as in the middle below is
      equivalent to such squares on the left and right.
      \begin{equation}
            \label{COLOR_SQ_EQ}
            \begin{tikzcd}
                  a_! \O_1 \arrow[d, "a_! F"'] \arrow[r, "a"]
                  &
                  \O \arrow[d, "p"]
                  &&
                  \O_1 \arrow[d, "F"] \arrow[r, "a"]
                  &
                  \O \arrow[d, "p"]
                  &&
                  \O_1 \arrow[d, "F"] \arrow[r, "a"]
                  &
                  a^{\**} \O \arrow[d]
                  \\
                  a_! \O_2 \arrow[r]
                  &
                  p^{\**} \P
                  &&
                  \O_2 \arrow[r]
                  &
                  \P
                  &&
                  \O_2 \arrow[r]
                  &
                  a^{\**} p^{\**} \P
            \end{tikzcd}
      \end{equation}
\end{remark}

\begin{remark}
      We also have a inclusion-forgetful adjunction
      \begin{equation}
            \label{JSTAR_CAT_EQ}
            \begin{tikzcd}
                  \Op^{G, \mathfrak C}(\V) \arrow[r, shift right, "j^{\**}"'] \arrow[d, "{(-)^H}"']
                  &
                  \Cat^{G, \mathfrak C}(\V) \arrow[l, shift right, "j_!"'] \arrow[d, "{(-)^H}"']
                  \\
                  \Op^{\mathfrak C^H}(\V)  \arrow[r, shift right, "j^{\**}"']
                  &
                  \Cat^{\mathfrak C^H}(\V) \arrow[l, shift right, "j_!"']
            \end{tikzcd}
      \end{equation}
      such that $j^{\**}$ commutes with taking $H$-fixed points.
\end{remark}


\begin{remark}
	Unlike in the single-colored case, $\Op^G(\V)$ does \textit{not} coincide with the category of colored operads in $\V^G$.
	Indeed, objects in $\Op(\V^G)$ have a fixed $G$-set of colors,
        and each level $\O(\xi)$ has an action by the full group $G$
	(though only a partial action by $\Sigma_{|\xi|}$).
\end{remark}




\subsection{Colored trees and a monadic description of equivariant operads}
\label{COMEGA_SEC}

We will now investigate $\SC$ and $G \ltimes \Sigma_{\mathfrak C}^{op}$ and related categories,
and build the monad on $\Sym^{\G, \mathfrak C}(\V)$ defining $\Op^{G, \mathfrak C}(\V)$.

% \begin{remark}
%       As a warning, recall Convention \ref{G_CONV} about the chirality of the action of groups and their associated categories.
% \end{remark}

\begin{remark}
      For much of this section, $\mathsf F$ will often denote not just $\mathsf F$ itself, but also any of the subcategories
      $\Sigma$, $\mathsf F_i$, or $\mathsf F_s$.
      We will specify precisely when one of the categories is needed.
\end{remark}

\begin{definition}
      Given a category $\mathcal C$ over $\mathsf F$, let $\mathcal C_{\mathfrak C}$ denote the pullback
      \begin{equation}
            \begin{tikzcd}
                  \mathcal C_{\mathfrak C} \arrow[r] \arrow[d]
                  &
                  \mathsf F \wr \mathfrak C \arrow[d]
                  \\
                  \mathcal C \arrow[r]
                  &
                  \mathsf F
            \end{tikzcd}
      \end{equation}
\end{definition}

\begin{remark}
      We observe that the above definition does not conflict with our notation $\Sigma_{\mathfrak C}$ from the previous section,
      as $\SC$ is isomorphic to the pullback on the left below,
      where $E: \Sigma \to \mathsf F$ sends $\underline{n}$ to $\underline{n+1}$.
      % and the second arrow is the diagonal functor from Remark \ref{WR_DIAG_REM}. 
      \begin{equation}
            \label{COMEGA_B_EQ}
            \begin{tikzcd}
                  \SC \arrow[r, "E"] \arrow[d]
                  &
                  \mathsf F \wr \mathfrak C \arrow[d]
                  & %
                  \OC \arrow[d] \arrow[r, "E"]
                  &
                  \mathsf F \wr \mathfrak C \arrow[d]
                  & %
                  G^{op} \ltimes \OC \arrow[d] \arrow[r, "E"]
                  &
                  G^{op} \ltimes (\mathsf F \wr \mathfrak C) \arrow[d] \arrow[r]
                  &
                  \mathsf F \wr (G^{op} \ltimes \mathfrak C) \arrow[d]
                  \\
                  \Sigma \arrow[r, "E"]
                  &
                  \mathsf F
                  & %
                  \Omega \arrow[r, "E"]
                  &
                  \mathsf F
                  & %
                  G^{op} \times \Omega \arrow[r, "E"]
                  &
                  G^{op} \times \mathsf F \arrow[r]
                  &
                  \mathsf F \wr G^{op}
            \end{tikzcd}
      \end{equation}
      More generally, let $\Omega_{\mathfrak C}$ denote the middle pullback,
      where $E: \Omega \to \mathsf F$ sends a tree $U$ to its set $E(U)$ of edges.
      We note that $G^{op} \ltimes \OC$ is given by the right pullback,
      as $G^{op} \ltimes (-)$ preserves pullbacks and the rightmost square is a pullback
      by Lemmas \ref{GL_PULL_LEM} and \ref{GD_PULL_LEM},
      and a similar pair of squares yields $G^{op} \ltimes \SC$. 

      We have a natural inclusion of categories $\Sigma_{\mathfrak C} \into \Omega_{\mathfrak C}$,
      and as such we will called elements of these categories
      \textit{colored corollas} and \textit{colored trees},
      and denote them by $(U,\mathfrak c)$, where $\mathfrak c: E(U) \to \mathfrak C$ is a map of sets,
      or just by $U$ if there is no confusion.
\end{remark}

Unpacking definitions (cf. Remark \ref{G_GR_REM}), we see that a map $(U, \mathfrak c) \to (V, \mathfrak d)$ in $G^{op} \ltimes \Omega_{\mathfrak C}$
is given by
a map $f: U \to V$ in $\Omega$ and an element $g\in G$,
such that $\mathfrak c(f(e)) = g.\mathfrak d(e)$ for all $e \in E(V)$.
\begin{equation}
      \begin{tikzcd}
            E(U) \arrow[d, "\mathfrak c"'] \arrow[r, "f"]
            &
            E(V) \arrow[d, "\mathfrak d"]
            \\
            \mathfrak C \arrow[r, "g"] 
            &
            \mathfrak C
      \end{tikzcd}
\end{equation}

In particular, we have maps of the form
\begin{equation}
      g = (id, g): g^{\**} U \to U
      % (U, E(U) \xrightarrow{\mathfrak c} \mathfrak C \xrightarrow{g \cdot} \mathfrak C)
      % \to
      % (U, E(U) \xrightarrow{\mathfrak c} \mathfrak C).
\end{equation}



Now, many of the natural functors around $\Omega$ and $\Sigma$ have generalizations to the colored setting,
which can be built through a straightforward use of the universal property of pullbacks.

\begin{definition}
      We have natural \textit{vertex} functors
      \begin{equation}
            \OC \to \Sigma \wr \SC
      \end{equation}
      and 
      \begin{equation}
            % G \ltimes \OC^{op} =
            % (G^{op} \ltimes \OC)^{op} \xrightarrow{V}
            % (G^{op} \ltimes (\Sigma \wr \SC))^{op} \to
            % (\Sigma \wr (G^{op} \ltimes \SC))^{op} =
            % (\Sigma \wr (G \ltimes \SC^{op})^{op})^{op},
            G^{op} \ltimes \OC \to G^{op} \ltimes (\Sigma \wr \SC) \to \Sigma \wr (G^{op} \wr \SC),
      \end{equation}
      as colorings of a tree restrict to colorings of each vertex corolla.

      Equivalently, the first map is $\pi_{\mathfrak C}^{\**}$ applied to the original vertex functor
      $V: \Omega \to \Sigma \wr \Sigma$;
      see {\color{red} LATER SECTIONS}).

      Similarly, there is a \textit{leaf-root} functor
      $\mathsf{lr}: G \ltimes \Omega_{\mathfrak C}^{op} \to G \ltimes \Sigma_{\mathfrak C}^{op}$,
      where the coloring of $\mathsf{lr}(T)$ is a restrict of the coloring of $T$.
\end{definition}

Moreover, the algebraic structure on these operads is determined by a monad on symmetric sequences.

\begin{definition}
      Given $X \in \Sym^{G, \mathfrak C}$, let $\mathbb F^{\mathfrak C} X$ denote the left Kan extension below.
      \begin{equation} 
            \begin{tikzcd}
                  G \ltimes \Omega_{\mathfrak C}^{op}
                  \arrow[d, "\mathsf{lr}"']
                  \arrow[r, "V"]
                  &
                  (\Sigma \wr (G \ltimes \Sigma_{\mathfrak C}^{op})^{op})^{op} \arrow[r, "\Sigma \wr X"]
                  \arrow[dl, Rightarrow]
                  &
                  (\Sigma \wr \V^{op})^{op} \arrow[r, "\otimes"]
                  &
                  \V
                  \\
                  G \ltimes \Sigma_{\mathfrak C}^{op} \arrow[urrr, "\Lan = \mathbb F^{\mathfrak C} X"']
            \end{tikzcd}
      \end{equation}
\end{definition}

We will analyze the above monad, and the colored vertex and leaf-root functors, systematically {\color{red} LATER}.
However, we first show that this generalizes the free single-colored operad monad from \cite{BP17}.

% Note that when $\mathfrak C = \set{\**}$, we have
% $G \ltimes \Omega_{\mathfrak C} = \Omega \times G$ and 
% $G \ltimes \Sigma_{\mathfrak C} = \Sigma \times G$.

\begin{notation}
      Given a functor $X : \C \to \mathsf{Fun}(\mathcal D, \V)$,
      let $\tilde X$ denote the adjoint $\tilde X: \C \times \mathcal D \to \V$.
\end{notation}

\begin{lemma}
      \label{SPAN_LAN_LEM}
      Conisder the two spans below.
      \begin{equation}
            \begin{tikzcd}
                  \C \arrow[d, "p"] \arrow[r, "X"]
                  &
                  \mathsf{Fun}(\mathcal D, \V)
                  &&
                  \C \times \mathcal D \arrow[d, "p \times \mathsf{id}"] \arrow[r, "\tilde X"]
                  &
                  \V
                  \\
                  \mathcal E
                  &
                  &&
                  \mathcal E \times \mathcal D
            \end{tikzcd}
      \end{equation}
      
      Then $\Lan_p X$ is adjoint to $\Lan_{p \times \mathsf{id}} \tilde X$. 
\end{lemma}
\begin{proof}
      Using the pointwise description of the Kan extension, we have
      \begin{align}
        \widetilde{\Lan_p X}(e,d)
        &= (\Lan_p X(e))(d)
          = \left(
          \colim\limits_{\substack{ \C \downarrow e \\ p(c) \to e}} X(c)
        \right)(d)
        = \colim\limits_{\substack{ \C \downarrow e \\ p(c) \to e}}(X(c)(d))
        = \colim\limits_{\substack{ \C \downarrow e \\ p(c) \to e}}(\tilde X(c,d))\\
        &= \colim\limits_{\substack{ \C \times \set{d} \downarrow (e,d) \\ p(c) \to e}}(\tilde X(c,d))
        \cong \colim\limits_{\substack{ \C \times \mathcal D \downarrow (e,d) \\ (p(c),d') \to (e,d)}}(\tilde X(c,d'))
        = \Lan_{p \times \mathsf{id}}\tilde X(c,d),
      \end{align}
      where the isomorphism holds by a straightforward finality argument.
      On maps, a similar argument holds.
\end{proof}

\begin{notation}[\cite{BP17}]
      Let $\mathbb F'$ denote the \textit{free single-colored operad monad} on $\V$, given by the left Kan extension of the following diagram.
      \begin{equation}
            \begin{tikzcd}
                  \Omega^{op}
                  \arrow[d, "\mathsf{lr}"']
                  \arrow[r, "V"]
                  &
                  (\Sigma \wr \Sigma)^{op} \arrow[r, "X"]
                  \arrow[dl, Rightarrow]
                  &
                  (\Sigma \wr \V^{op})^{op} \arrow[r, "\otimes"]
                  &
                  \V
                  \\
                  \Sigma^{op} \arrow[urrr, "\Lan = \mathbb F' X"']
            \end{tikzcd}
      \end{equation}
\end{notation}

\begin{proposition}
      \label{TEST_PROP}
      $\mathbb F^{\set{\**}}$ is a monad, and moreover
      the category of $\mathbb F^{\set{\**}}$-algebras in $\mathsf{Fun}(G \times \Sigma^{op}, \V)$ is equivalent to
      the category of $\mathbb F'$-algebras in $\mathsf{Fun}(\Sigma^{op}, \V^G)$.
\end{proposition}
\begin{proof}
      Let $\tau: \tilde X \mapsto X$ denote the isomorphism of categories
      $\mathsf{Fun}(G \times \Sigma^{op}, \V) \xrightarrow{\tau} \mathsf{Fun}(\Sigma^{op}, \V^G)$.
      Then $\mathbb F^{\set{\**}} = \tau^{-1} \mathbb F' \tau$ by \ref{SPAN_LAN_LEM}, and so
      $\mathbb F^{\set{\**}}$ is in fact a monad, and
      the isomorphism lifts to an isomorphism on the category of algebras.
\end{proof}

The general case will be given in Proposition \ref{FC_MONAD_PROP}.





\newpage

\section{Results Needed in the Next Section}

\subsection{Monad}

We define the monad on $\mathsf{WSpan}^r_{\Cat \downarrow^r \mathsf F}(\Sigma^{op}, \V)$.
\begin{definition}
      Let $\mathcal A$ be in this category. This consists of the data of a category $\mathcal A$ and the following diagram.
      \begin{equation}
            \begin{tikzcd}
                  |[alias = U]| \Sigma \arrow[dr, "E"']
                  &
                  \mathcal A \arrow[d, "\epsilon"'{name = V}, ""{name = D}] \arrow[l, "\lambda"'] \arrow[r]
                  &
                  |[alias = C]| \V \arrow[dl, "\varnothing"]
                  \\
                  & \mathsf F
                  \arrow[Rightarrow, from = U, to = V, "\alpha"]
                  \arrow[Rightarrow, from = C, to = D, "\beta"']
            \end{tikzcd}
      \end{equation}
      Then we define $N(\mathcal A)$ to be the span given by the pullback
      \begin{equation}
            \begin{tikzcd}
                  \Omega \wr \mathcal A \arrow[r] \arrow[d]
                  &
                  \mathsf F \wr \mathcal A \arrow[d] \arrow[r]
                  &
                  \mathsf F \wr \V \arrow[r, "\otimes"]
                  &
                  \V
                  \\
                  \Omega \arrow[d] \arrow[r, "V"]
                  &
                  \mathsf F \wr \Sigma
                  \\
                  \Sigma
            \end{tikzcd}
      \end{equation}
      with the map over $\mathsf F$ given by the pushout in $\Fun(\Omega \wr \mathcal A, \mathsf F)$ of the natural transformations
      \begin{equation}
            E(T) \Leftarrow \coprod_{v \in V(T)} (\lambda(a_v)+1) \Rightarrow \coprod_{v \in V(T)}\epsilon(a_v)
      \end{equation}
      on the objects $(T, (a_v)) \in \Omega \wr \mathcal A$,
      where the maps between these functors are given by the following diagram
      (and in particular, we have a map $\lambda(a_v) + 1 \to E(T_v)$).
      \begin{equation}
            \begin{tikzcd}
                  \Omega \wr \mathcal A \arrow[r, equal] \arrow[d, equal]
                  &
                  \Omega \wr \mathcal A \arrow[d] \arrow[r, "V"]
                  &
                  \mathsf F \wr \mathcal A \arrow[d, "\lambda"] \arrow[r, "\epsilon"]
                  &
                  \mathsf F \wr \mathsf F \arrow[r, "\amalg"] \arrow[d, equal]
                  &
                  \mathsf F \arrow[d, equal]
                  \\
                  \Omega \wr \mathcal A \arrow[d, equal] \arrow[r, "\lambda"]
                  &
                  \Omega \wr \Sigma \arrow[d, equal] \arrow[r, "V"]
                  &
                  \mathsf F \wr \Sigma \arrow[r, "{(-)+1}"] \arrow[ur, Rightarrow, "\alpha"] \arrow[drr, Rightarrow] 
                  &
                  \mathsf F \wr \mathsf F \arrow[r, "\amalg"]
                  &
                  \mathsf F \arrow[d, equal]
                  \\
                  \Omega \wr \mathcal A \arrow[r, "\lambda"]
                  &
                  \Omega \wr \Sigma \arrow[r, "\simeq"]
                  &
                  \Omega \arrow[rr, "E"] \arrow[u, "V"]
                  &&
                  \mathsf F
            \end{tikzcd}
      \end{equation}
      
      \todo[inline]{come back: check that this gives maps over $\mathsf F$}
\end{definition}

\begin{lemma}
      $\Omega \wr \mathcal A$ is the strict 2-pullback in $\Cat \downarrow^r \mathsf F$.
\end{lemma}
\begin{proof}
      This follows immediately from Remark \ref{SPANLIM REM}.
\end{proof}

\begin{proposition}
      $N$ is a monad on $\mathsf{WSpan}^r_{\Cat \downarrow^r \mathsf F}(\Sigma^{op}, \V)$.
\end{proposition}
\begin{proof}
      \todo[inline]{come back}
\end{proof}



\begin{proposition}
      \label{FC_MONAD_PROP}
      For all $G$-sets $\mathfrak C$,
      $\mathbb F^{\mathfrak C}$ is a monad on $\Sym^{G, \mathfrak C}(\V)$ with algebra category $\Op^{G, \mathfrak C}(\V)$
\end{proposition}
% \begin{proof}
%       This follows from the work in {\color{red} LATER SECTIONS}:
%       in particular, $\mathbb F^{\mathfrak C} = \Lan \circ N^{\mathfrak C} \circ \iota$,
%       where $N^{\mathfrak C}$ is the operation on spans...
%       The fact that all the necessarily diagrams, as shown in \cite{BP17}, commute/agree follows by breaking those diagrams in half:
%       the left sides never mention $\V$, and we can show that the generalized diagrams commute/agree by applying $\pi^{\**}$ to the diagrams in \cite{BP17},
%       where $\pi: \mathsf F \wr (G \ltimes \mathfrak C) \to \mathsf F$,
%       and using the fact that this is a functor which preserves standard limits;
%       the right sides only deal with the permutative structure of our underlying category $\V$.
%       These are connected by trivially commuting information about our $(G, \mathfrak C)$-symmetric sequence $X$.
%       \todo[inline]{come back: this doesn't actually say anything yet.}
% \end{proof}

\subsection{Filtration}

We mimic the discussion and construction in \cite[\S 5.1 - 5.3]{BP17}.
For this section, let $\mathbb F$ denote $\mathbb F^{G, \mathfrak C}$.

\begin{definition}
      Given $\O \in \Op^{G, \mathfrak C}(\V)$ and $u: X \to Y$ in $\Sym^{G,\mathfrak C}\V$,
      let $\mathcal \O[u]$ denote the pushout
      \begin{equation}
            \begin{tikzcd}
                  \mathbb F X \arrow[d, "u"'] \arrow[r]
                  &
                  \O \arrow[d]
                  \\
                  \mathbb F Y \arrow[r]
                  &
                  \O[u].
            \end{tikzcd}
      \end{equation}
\end{definition}

Our goal in this section is to produce the following filtration.

\begin{proposition}
      \label{FILT_PUSH_PROP}
      There exists a filtration in $\Sym^{G, \mathfrak C}(\V)$
      \begin{equation}
            \O = \O_0 \into \O_1 \into \O_2 \into \dots \into \O_\infty = \O[u]
      \end{equation}
      of the pushout map
      such that, for each $\mathfrak C$-profile $C \in G \ltimes \Sigma_{\mathfrak C}$,
      the diagram below is a pushout in $\V^{\Aut(C, \mathfrak c)}$.
      \begin{equation}
            \label{FILT_PUSHG_EQ}
            \begin{tikzcd}
                  \mathop{\coprod}\limits_{[T] \in \Iso(C \downarrow_r G \ltimes \Omega_{\mathfrak C}^a[k])}
                  \left(
                        \mathop{\bigotimes}\limits_{v \in V^{ac}(T)} \O(T_v) \otimes
                        Q^{in}_T[u]
                  \right) \cdot_{\Aut_{G \ltimes OC}(T)} \Aut_{G \ltimes \SC}(C)
                  \arrow[r]
                  \arrow[d]
                  &
                  \O_{n-1}(C) \arrow[d]
                  \\                  
                  \mathop{\coprod}\limits_{[T] \in \Iso(C \downarrow_r G \ltimes \Omega_{\mathfrak C}^a[k])}
                  \left(
                        \mathop{\bigotimes}\limits_{v \in V^{ac}(T)} \O(T_v) \otimes
                        \mathop{\bigotimes}\limits_{v \in V^{in}(T)} Y(T_v)
                  \right) \cdot_{\Aut_{G \ltimes \OC}(T)} \Aut_{G \ltimes \SC}(C)
                  \arrow[r]
                  &
                  \O_n(C)
            \end{tikzcd}
      \end{equation}
      where we define
      \begin{equation}
            \mathop{\mathlarger{\mathlarger{\mathlarger{\square}}}}_{v \in V^{in}(T)} u(T_v): Q^{in}_T[u] \to \bigotimes_{v \in V^{in}(T)} Y(T_v).
      \end{equation}
\end{proposition}

To start, we observe ---  as in $(5.3)$ through $(5.7)$ of \cite{BP17} --- that
\begin{align*}
  \O[u]
  &
    \simeq \mathrm{coeq}\left(
          \O \mathbin{\hat\amalg} \mathbb F X \mathbin{\hat\amalg} \mathbb F Y \rightrightarrows \O \mathbin{\hat\amalg} \mathbb F Y
          \right)
  \\
  &
    \simeq \colim_{n,l} B_l \left( \mathbb F^{n+1} \O, \mathbb F X, \mathbb F X, \mathbb F X, \mathbb FY \right)
  \\
  &
    \simeq \colim_{n,l} \Lan N \circ \left( N^{\circ n} \iota \O \amalg \iota X^{\amalg 2l +1} \amalg \iota Y \right),
    \stepcounter{equation}\tag{\theequation}\label{OU_EQ1}
    % =: \colim_{n,l} \Lan \hat N_{n,l}^{(\O,X,Y)},
\end{align*}
where $B$ denotes the \textit{double bar construction},
and the left Kan extension is over
\mbox{$G \ltimes \Omega_{\mathfrak C}^{0,op} \xrightarrow{\mathsf{lr}} G \ltimes \Sigma_{\mathfrak C}^{op}$}.
Through a series of reductions (Propositions \ref{OU_RED1_PROP}, \ref{OU_RED2_PROP}, and \ref{OU_RED3_PROP}),
we will identify this as a left Kan extension over an approachable category $\hat\Omega_{\mathfrak C}^e$, 
and build our filtration by filtering this category.

\subsubsection{Labeled planar strings}

\begin{definition}
      Given $n\geq -1$, $-1 \leq s \leq n$, $l \geq 0$, and a partition $\lambda = \lambda_a \amalg \lambda_i$ of $\set{1,2,\dots,l}$, 
      define $\Omega_{\mathfrak C}^{n,s,\lambda}$ to be the pullback
      \begin{equation}
            \begin{tikzcd}
                  \OC^{n,s,\lambda} \arrow[d] \arrow[r, "E"]
                  &
                  \Sigma \wr \mathfrak C \arrow[d]
                  \\
                  \Omega^{n,s,\lambda} \arrow[r, "E"]
                  &
                  \Sigma
            \end{tikzcd}
      \end{equation}
      for $\Omega^{n,s,\lambda}$ as in \cite[Defn. 5.10]{BP17}.

      Explicitly, objects are strings
      \begin{equation}
            T_{-1} \xrightarrow{\phi_0} T_0 \xrightarrow{\phi_1} T_1 \xrightarrow{\phi_2} \dots \xrightarrow{\phi_n} T_n
      \end{equation}
      of planar tall maps of trees, such that
      the $T_r$ are $l$-labeled and the $\phi_r$ are $\lambda_i$-inert label maps for $r > s$,
      and such that $T_n$ is $\mathfrak C$-colored.
      This last fact induces a \textit{unique} choice of coloring on all trees $T_i$ so that the maps $f_i$ are maps of colored trees.      
\end{definition}

\begin{remark}
      We note that if $\mathfrak C$ is a $G$-set, then
      $\OC^{n,s,\lambda}$ is a left $G$-category, with action induced by the action on $\mathfrak C$,
      and thus we may form the categories $G \ltimes \OC^n$ and $G^{op} \ltimes \OC^n$ as in Example \ref{G_GR_REM} and Remark \ref{GOP_REM}.
\end{remark}

\begin{definition}
      \label{NA_DEF}
      Given functors of $G$-categories $\A, \A_j \to \Sigma_{\mathfrak C}$, define the categories
      \begin{equation}
            (G^{op} \ltimes \Omega_{\mathfrak C}^{n}) \wr (G^{op} \ltimes \A)
            \qquad \mbox{ and } \qquad
            (G^{op} \ltimes \Omega_{\mathfrak C}^{n,s,\lambda}) \wr ((G^{op} \ltimes A_j))
      \end{equation}
      to be the pullbacks of the solid rectangles below.
      \begin{equation}
            \label{LSTRINGS_EQ}
            \begin{tikzcd}
                  \bullet \arrow[d] \arrow[r]
                  &
                  G^{op} \ltimes (\Sigma \wr \A) \arrow[d] \arrow[r]
                  &
                  \Sigma \wr (G^{op} \ltimes \A) \arrow[d] \arrow[r, dashed]
                  &
                  \Sigma \wr \V^{op} \arrow[r, dashed, "\otimes"]
                  &
                  \V^{op}
                  \\
                  G^{op} \ltimes \Omega_{\mathfrak C}^{n} \arrow[r] \arrow[d, dashed]
                  &
                  G^{op} \ltimes (\Sigma \wr \Sigma_{\mathfrak C}) \arrow[r]
                  &
                  \Sigma \wr (G^{op} \ltimes \Sigma_{\mathfrak C})
                  \\
                  \SC
                  \\ % NEW DIAGRAM ------------------------------
                  \bullet \arrow[dd] \arrow[r]
                  &
                  G^{op} \ltimes (\Sigma \wr \amalg \A_j) \arrow[d] \arrow[r]
                  &
                  \Sigma \wr (G^{op} \ltimes \amalg A_j) \arrow[d] \arrow[r, dashed]
                  &
                  \Sigma \wr \V^{op} \arrow[r, dashed, "\otimes"]
                  &
                  \V^{op}
                  \\
                  &
                  G^{op} \ltimes (\Sigma \wr \amalg \SC) \arrow[d] \arrow[r]
                  &
                  \Sigma \wr (G^{op} \ltimes \amalg \SC) \arrow[d]
                  \\
                  G^{op} \ltimes \OC^{n,s,\lambda} \arrow[r] \arrow[d, dashed]
                  &
                  G^{op} \ltimes (\Sigma \wr \OC^{-1,s-n-1,\lambda}) \arrow[r]
                  &
                  \Sigma \wr (G^{op} \ltimes \OC^{-1,s-n-1,\lambda})
                  \\
                  G^{op} \ltimes \SC
            \end{tikzcd}
      \end{equation}

      More generally, if we in fact started with spans
      \begin{equation}
            G^{op} \ltimes \SC \leftarrow \A \to \V^{op},
            \qquad
            G^{op} \ltimes \SC \leftarrow \A_j \to \V^{op},
      \end{equation}
      define $N(A) = N^{G^{op},\mathfrak C}(A)$ and $N_{n,s,\lambda}^{(A_j)}$ to be the dashed spans.
\end{definition}


\begin{lemma}
      \label{GWRA_LEM}
      For any such $\A$ and $\A_j$, we have
      \begin{gather*}
            (G^{op} \ltimes \OC^{n}) \wr (G^{op} \ltimes \A) \simeq G^{op} \ltimes (\OC^{n} \wr \A),
            \\
            (G^{op} \ltimes \OC^{n,s,\lambda}) \wr ((G^{op} \ltimes A_j)) \simeq G^{op} \ltimes (\OC^{n,s,\lambda} \wr (\A_j)).
      \end{gather*}
\end{lemma}
\begin{proof}
      All of the squares in \eqref{LSTRINGS_EQ} are pullbacks by Lemmas \ref{GL_PULL_LEM} and \ref{GD_PULL_LEM}.
\end{proof}


\begin{lemma}
      $\OC^{n,s,\lambda} \simeq \OC^s \wr \OC^{-1,-1,\lambda} \wr (\OC^{n-s-1})^{\times \lambda}$.
\end{lemma}
\begin{proof}
      The $\mathfrak C$-colored version of \cite[Prop 5.30]{BP17} (with $G = e$)
      follows from Proposition 9.2 and Corollary \ref{COLORCOR COR}
      by the same argument as in \textit{loc cite}.
      The $\mathfrak C$-colored version of \cite[Cor. 5.32]{BP17} then follows by drawing the analogous diagrams used in its proof,
      and finally the above equation combines two special cases of this result.
\end{proof}


Applying Lemma \ref{GWRA_LEM} to the above equation yields the following.
\begin{proposition}
      \label{NNSL_PROP}
      We have
      \begin{equation}
            G^{op} \ltimes \OC^{n,s,\lambda} \simeq
            (G^{op} \ltimes \OC^{s}) \wr ((G^{op} \ltimes \OC^{-1,-1,\lambda}) \wr (G^{op} \ltimes \OC^{n-s-1})^{\times \lambda}),
      \end{equation}
      and hence
      \begin{equation}
            N_{n,s,\lambda}^{(A_j)}
            \simeq
            \left(
                  N^{\circ s + 1} \circ \amalg \circ (N^{\times \lambda})^{\circ n-s}
            \right)
            ((A_j))
            % \simeq
            % \Lan_{G \ltimes \OC^{n,s,\lambda,op} \to G \ltimes \SC^{op}}(A_j)
      \end{equation}
\end{proposition}



\begin{definition}
      Define the bisimplicial object
      \begin{equation}
            N_{n,l}^{(\O,X,Y)}
            := N_{n,0,\lambda_l}^{(\O, X^{\amalg 2l+1}, Y)}
            = N \circ (N^{\circ n} \iota \O \amalg X^{\amalg 2l + 1} \amalg \iota Y).
            % \Lan_{G \ltimes \OC^{n,\lambda_l,op} \to G \ltimes \SC^{op}}(\O, X^{\amalg 2l+1}, Y).
      \end{equation}
      
      In the $n$-direction, the face maps $d_i$ for $0 < i < n$ and $i=0$ are induced by the maps
      \begin{equation}
            N \circ N \xrightarrow{\mu} N,
            \qquad
            N \circ \amalg \circ N \to N \circ N \circ \amalg \xrightarrow{\mu} N \circ \amalg            
      \end{equation}
      respectively, 
      where $\mu$ is the monad multiplication,
      and the first map for $d_0$ is the natural map $\amalg \circ N \to N \circ \amalg$
      produced from Lemma \ref{GWRA_LEM} by applying (the pullback-preserving functor) $G \ltimes (-)$
      to the $\mathfrak C$-colored adaption of the diagrams in \cite[(5.34)]{BP17}.
      % All horizontal lines in the first diagram are the maps $V_{\mathfrak C}^0$,
      % and again the diagram on the right unpacks notation.
      % \begin{equation}
      %       \begin{tikzcd}[row sep = small]
      %             G \ltimes \OC^{0,-1,\lambda} \arrow[r, "{\amalg (V^0_G)^{\times \lambda}}"] \arrow[d]
      %             &
      %             G \ltimes \left( \amalg \Sigma \wr (\OC^{-1})^{\times \lambda} \right) \arrow[r]
      %             &
      %             G \ltimes \left( \Sigma \wr \OC^{-1,-2,\lambda} \right) \arrow[d, equal]
      %             && %
      %             G \ltimes \left( \amalg (\OC^0)^{\times \lambda} \right) \arrow[d] \arrow[r]
      %             &[-15pt]
      %             G \ltimes \left( \amalg \Sigma \wr \SC \right) \arrow[r]
      %             &[-15pt]
      %             G \ltimes \left( \Sigma \wr \amalg \SC \right) \arrow[d, equal]
      %             \\
      %             G \ltimes \OC^{0,0,\lambda} \arrow[d] \arrow[rr, "V_{\mathfrak C}^0"]
      %             &&
      %             G \ltimes \left( \Sigma \wr \OC^{-1,-1,\lambda} \right) \arrow[d]
      %             && %
      %             G \ltimes \OC^{0,0,\lambda} \arrow[d] \arrow[rr]
      %             &&
      %             G \ltimes \left( \Sigma \wr \amalg \SC \right) \arrow[d]
      %             \\
      %             G \ltimes \OC^{0,-1,\lambda} \arrow[rr, "V_{\mathfrak C}^0"]
      %             &&
      %             G \ltimes \left( \Sigma \wr \OC^{-1,0,\lambda} \right)
      %             && %
      %             G \ltimes \OC^0 \arrow[rr]
      %             &&
      %             G \ltimes (\Sigma \wr \SC)
      %       \end{tikzcd}
      % \end{equation}
      The degeneracies are similar.
      Lastly, the operations in the $l$-direction are the standard operations for the double bar construction.
\end{definition}

\begin{remark}
      Generalizing Remark \ref{GOP_REM},
      we have
      \begin{gather*}
            \left((G^{op} \ltimes \OC^n) \wr (G^{op} \ltimes \A)\right)^{op} \simeq
            (G \ltimes \OC^{n,op}) \wr (G \ltimes \A^{op}),
      \end{gather*}
      and similarly for the other three constructions in Definition \ref{NA_DEF}.
\end{remark}

Applying Proposition \ref{NNSL_PROP} to \eqref{OU_EQ1} yields the following.

\begin{proposition}
      \label{OU_RED1_PROP}
      $\O[u]
      \simeq
      \mathop{\colim}\limits_{(\Delta \times \Delta)^{op}} \left(
            \Lan_{G \ltimes (\Omega_{\mathfrak C}^{n,\lambda_l} \to \Sigma_{\mathfrak C})^{op}} N_{n,l}^{(\O,X,Y)}
      \right)$.
\end{proposition}


\subsubsection{Alternating trees and the filtration}

Recall that, for any tree $T \in \Omega$ and edge $e \in T$, the \textit{input path} of $e$ is the set $I(e) = \sets{f \in T}{e \leq_d f}$.

\begin{definition}[{cf. \cite[Defns 5.38,5.44,5.48]{BP17}}]
      Let $\Omega^a$ denote the category of \textit{alternating trees}, where for any leaf $l \in T$, $|I(l)|$ is even.
      Those vertices $e^{\uparrow} \leq e$ where $|I(e)|$ is odd are called \textit{active}, and \textit{inert} otherwise.

      Let $\Omega^e$ denote the \textit{extension tree category}, with objects $\set{\O, X,Y}$-labeled trees, and arrows tall maps $\phi: T \to S$ such that for all $v \in V(T)$,
      \begin{itemize}
      \item If $T_v$ has an $X$-label, then $S_v \in \Sigma$ and has an $X$-label.
      \item If $T_v$ has a $Y$-label, then $S_v \in \Sigma$ and has either an $X$- or $Y$-label.
      \item If $T_v$ has an $\O$-label, then $S_v$ has only $X$ and $\O$ labels.
      \end{itemize}

      Finally, let $\hat\Omega^e \into \Omega^e$ denote the subcategory of $(\O,X,Y)$-labeled trees whose
      underlying tree is alternating, has active nodes labeled by $\O$ and inert nodes labeled by $X$ or $Y$.
\end{definition}

We will be realizing (twice) the bisimplicial category $\OC^{\bullet, \lambda_\bullet}$,
and the following proposition shows that realization behaves well with pullback functors such as $\pi_{\mathfrak C}^{\**}$.
\todo[inline]{might be good to move this to the Appendix, so can discuss the relations more fully}
\begin{remark}
      \label{REAL_CAT_REM}
      We recall that $|\mathcal C_\bullet|$ has object set $\Ob(\mathcal C_0)$,
      arrows generated by
      maps $f_0: a_0 \to \bar a_0$ in $\mathcal C_0$ and objects $a_1: d_1(a_1) \xrightarrow{a_1} d_0(a_1)$ in $\mathcal C_1$,
      and relations given by information in just $\mathcal C_0$, $\mathcal C_1$, and $\mathcal C_2$:
      \begin{enumerate}[label = (\roman*)]\setcounter{enumi}{-1}
      \item $[f_0] \circ [g_0] = [f_0 \circ g_0]$,
      \item $[id_{a_0}] = [\ ]_{a_0}$,
      \item $[s_0a_0] = [id_{a_0}]$,
      \item for all $f_1: a_1 \to b_1$ in $\mathcal C_1$, the following diagram commutes.
            \begin{equation}
                  \begin{tikzcd}
                        d_1 a_1 \arrow[r, "a_1"] \arrow[d, "d_1 f_1"']
                        &
                        d_0 a_1 \arrow[d, "d_0 f_1"]
                        \\
                        d_1 b_1 \arrow[r, "b_1"]
                        &
                        d_0 b_1
                  \end{tikzcd}
            \end{equation}
      \item for all $a_2 \in \mathcal C_2$, the following diagram commutes.
            \begin{equation}
                  \begin{tikzcd}
                        d_1 d_2 a_2 = d_1 d_1 a_2 \arrow[dr, "d_2 a_2"'] \arrow[rr, "d_1 a_2"]
                        &&
                        d_0 d_1 a_2 = d_0 d_0 a_2
                        \\
                        &
                        d_0 d_2 a_2 = d_1 d_0 a_2 \arrow[ur, "d_0 a_2"]
                  \end{tikzcd}
            \end{equation}
      \end{enumerate}

      See \cite[Remark A.5]{BP17} for more details.
\end{remark}
\begin{proposition}
      \label{PIREAL_PROP}
      Let $E_\bullet: \mathcal C_\bullet \to \mathcal B$ be a simplicial object in $\Cat \downarrow^r \mathcal B$
      such that the natural transformation component $\phi_0$ of the structure map $d_0: \mathcal C_1 \to \mathcal C_0$
      is an isomorphism.

      Then we have a map out of the realization $E: |\mathcal C_\bullet| \to \mathcal B$,
      acting by $E_0$ on objects and defined on the generating arrows as follows:
      $f_0 \in \mathcal C_0$ goes to $E_0(f_0)$, and $a_1 \in \mathcal C_1$ goes to the composite
      \begin{equation}
            E_0 d_0(a_1) \xrightarrow{\phi_1} E_1(a_1) \xrightarrow{\phi_0^{-1}} E_0 d_0(a_1).
      \end{equation}

      Moreover, for any (split) Grothendieck fibration $\pi: \mathcal E \to \mathcal B$,
      the map $|\pi^{\**} \mathcal C_\bullet| \to \pi^{\**} |\mathcal C_\bullet|$ is an isomorphism.
\end{proposition}
\begin{proof}
      This result is not challenging, but requires carefully putting all the pieces together.
      
      For the first claim, it suffices to check that the map described above is well-defined with respect to the relations on the set of arrows in $|\mathcal C_\bullet|$, and this is not hard to verify directly.
      {\color{OliveGreen}
        Conditions (0) - (ii) are straightforward, $(iii)$ follows as $\phi_0$ and $\phi_1$ are natural transformations,
        and $(iv)$ follows as the following diagram commutes by the simplicial identities.
        \begin{equation}
              \begin{tikzcd}
                    E_0 d_1 d_2 a_2 \arrow[d, equal] \arrow[r, "\phi_1"]
                    &
                    E_1 d_2 a_2 \arrow[dd, "\phi_2"]
                    &
                    E_0 d_0 d_2 a_2 \arrow[d, equal] \arrow[l, "\phi_0"]
                    \\
                    E_0 d_1 d_1 a_2 \arrow[d, "\phi_1"]
                    &&
                    E_0 d_1 d_0 a_2 \arrow[d, "\phi_1"]
                    \\
                    E_1 d_1 a_2 \arrow[r, "\phi_1"]
                    &
                    E_2 a_2
                    &
                    E_1 d_0 a_2 \arrow[l, "\phi_0"]
                    \\
                    &&
                    E_0 d_0 d_0 a_2 \arrow[u, "\phi_0"] \arrow[d, equal]
                    \\
                    E_0 d_0 d_1 a_2 \arrow[uu, "\phi_0"] \arrow[r, "\phi_0"]
                    &
                    E_1 d_1 a_2 \arrow[uu, "\phi_1"]
                    &
                    E_0 d_0 d_1 a_2 \arrow[l, "\phi_0"]
              \end{tikzcd}
        \end{equation}
      }
      
      Now, $|\pi^{\**} \mathcal C_\bullet|$ and $\pi^{\**} |\mathcal C_\bullet|$
      clearly both have $\Ob(\mathcal C_0) \times_{\Ob(\mathcal B)} \Ob(\mathcal E)$ as their set of objects.
      For maps, we see the category $|\pi^{\**} \mathcal C_\bullet|$ has arrows generated by
      \begin{enumerate}[label = (\Roman*)]
      \item arrows in $\pi^{\**} \mathcal C_0$, namely pairs of arrows $(f_0,\alpha)$ compatible over $\mathcal B$, and
      \item formal arrows $a_1 \wedge e$ for each object $(a_1,e)$ in $\pi^{\**}\mathcal C_1$, which are of the form
            \begin{equation}
                  d_1(a_1,e) = (d_1(a_1), (\phi_1\pi)^{\**}e) \xrightarrow{(a_1,e)} (d_0(a_1), (\phi_0 \pi)^{\**} e).
            \end{equation}
      \end{enumerate}
      These are subject to conditions analogous to those from Remark \ref{REAL_CAT_REM}.
      {\color{OliveGreen}
        \begin{enumerate}[label = (\roman*)]\setcounter{enumi}{-1}
        \item $[(f_0, \alpha)] \circ [(f_0', \alpha')] = [(f_0 f_0', \alpha \alpha')]$,
        \item $[(id_{a_0}, id_e)] = [\ ]_{(a_0,e)}$,
        \item $[s_0 a_0 \wedge e] = [(id_{a_0}, id_e)]$,
        \item For all $(f_1, \alpha): (a_1, e) \to (b_1, \bar e)$ in $\pi^{\**} \mathcal C_1$,
              the following diagram commutes.
              \begin{equation}
                    \begin{tikzcd}
                          d_1(a_1, e) = (d_1 a_1, \phi_1^{\**} e) \arrow[d, "{d_1(f_1, \alpha) = (d_1 f_1, \phi_1^{\**} \alpha)}"'] \arrow[r, "a_1 \wedge e"]
                          &
                          d_0(a_1,e)  (d_0 a_1, \phi_0^{\**} e) \arrow[d, "{d_0(f_1, \alpha) = (d_0 f_1, \phi_0^{\**} \alpha)}"]
                          \\
                          d_1(b_1, \bar e) = (d_1 b_1, \phi_1^{\**} \bar e) \arrow[r, "b_1 \wedge \bar e"]
                          &
                          d_0(b_1, \bar e) = (d_0 b_1, \phi_0^{\**} \bar e).
                    \end{tikzcd}
              \end{equation}
        \item For all $(a_2, e) \in \pi^{\**} \mathcal C_2$, the following diagram commutes.
              \begin{equation}
                    \begin{tikzcd}
                          d_1 d_2 (a_2, e) = d_1 d_1 (a_2, e) \arrow[rr, "d_1 a_2 \wedge \phi_1^{\**} e"] \arrow[dr, "d_2 a_2 \wedge \phi_2^{\**} e"]
                          &&
                          d_0 d_1(a_2, e) = d_0 d_0(a_2, e)
                          \\
                          &
                          d_0 d_2 (a_2, e) = d_1 d_0 (a_2, e) \arrow[ur, "d_1 a_2 \wedge \phi_1^{\**} e"]
                    \end{tikzcd}
              \end{equation}
        \end{enumerate}
      }

      Conversely, $\pi^{\**}|\mathcal C_\bullet|$ has arrows generated by pairs of arrows $(f,\alpha)$ in $|\mathcal C_\bullet| \times \mathcal E$ over $\mathcal B$.

      We have a natural map $F: |\pi^{\**} \mathcal C_\bullet| \to \pi^{\**} |\mathcal C_\bullet|$, which is the identity on objects,
      and acts as follows on the generating arrows:
      $(f_0, \alpha)$ gets mapped to $([f_0], \alpha)$, while
      $a_1 \wedge $ gets mapped to $([a_1], \phi_1^{\**} e \to e \to \phi_0^{\**} e)$
      (where here we are again using the $\phi_0$ is invertible).
      It is straightforward to verify that this map is well-defined.

      We will use the fact that $\pi$ is a split Grothendieck fibration in order to build an inverse $G$ of the functor $F$.
      Namely, given any word $W: a_0 \to b_0$ of composable generating arrows from $\pi^{\**}|\mathcal C_{\**}|$,
      any map $\alpha: \bar e \to e$ lifting $E([W])$ has an induced factorization into
      a map $W^{\**}\alpha$ over $E_0(a_0)$, followed by a composite of cartesian arrows $\alpha_i$, one for each letter $w$ in $W$.

      Now, we send $([W],\alpha)$ to the composite of $|W|+1$ arrows.
      If $w_i = (a_0 \xrightarrow{c_0})$ and the target of $\alpha_i$ is $e$, then the $(i+1)$-st arrow is $(f_0, f_0^{\**}<e>)$.
      If $w_i = a_1$, then the $(i+1)$-st arrow is $a_1 \wedge (\phi_0^{\-1})^{\**} e$.
      Finally, we precompose with $(id_{a_0}, W^{\**}\alpha)$.

      As an example, suppose we have a word $W$ equal to the top row below, so $E(W)$ is as in the second row,
      and an arrow $\alpha: \bar e \to e$ in $\mathcal E$ such that $\pi(\alpha) = E(W)$.
      \begin{equation}
            \begin{tikzcd}[column sep = small]
                  W:
                  &[-20pt]
                  &[20pt]
                  a_0 \arrow[r, "f_0"]
                  &[20pt]
                  b_0 = d_1 a_1 \arrow[rr, "a_1"]
                  &&
                  d_0 a_1
                  \\
                  E(W):
                  &
                  &
                  E_0 a_0 \arrow[r, "E_0 f_0"] % \arrow[d, equal]
                  &
                  E_0 b_0 = E_0 d_1 a_1 \arrow[r, "\phi_1"]
                  &
                  E_1 a_1 \arrow[r, "{\phi_0^{-1}}"]
                  &
                  E_0 d_0 a_1 % \arrow[d, equal]
                  % \\
                  % \pi(\bar e) \arrow[r, equal]
                  % &
                  % \pi(\bar e) \arrow[rrr, "{\pi(\alpha)}"]
                  % &&&
                  % \pi(e)
                  \\
                  &
                  \bar e \arrow[r, dashed, "{\exists ! W^{\**} \alpha}"]
                  &
                  f_0^{\**} (\phi_0^{-1} \phi_1)^{\**} e \arrow[r]
                  &
                  (\phi_0^{-1} \phi_1)^{\**} e \arrow[rr]
                  &&
                  e
                  \\
                  G([W], \alpha):
                  &
                  (a_0, \bar e) \arrow[r, "{(id, W^{\**} \alpha)}"]
                  &
                  (a_0, f_0^{\**} (\phi_0^{-1} \phi_1)^{\**} e) \arrow[r, "{(f_0, f_0^{\**})}"]
                  &
                  (b_0 = d_1 a_1, (\phi_0^{-1} \phi_1)^{\**} e) \arrow[rr, "a_1 \wedge \phi_0^{-1,\**} e"]
                  &&
                  (d_0 a_1, e)
            \end{tikzcd}
      \end{equation}
      The word $E(W)$ induces a factorization of $\alpha$ into two chosen cartesian maps, along with a unique map $W^{\**}\alpha$ determined by the cartesian maps.
      With this factorization, the last line displays the image of the arrow $([W], \alpha)$ in $\pi^{\**}|\mathcal C_\bullet|$. 
      
      Now, it is clear that these functors are invertible, assuming $G$ is in fact functorial.
      We begin by showing that $G$ behaves well under composition.
      It suffices to show that these end maps $W^{\**}\alpha$ ``commute'' with the more structured maps.
      We have two cases, one for each type of generating arrow.
      First, suppose we have a word/map $W = f_0: a_0 \to b_0$ in $\mathcal C_0$, and two maps
      \begin{equation}
            \bar{\bar e} \xrightarrow{\beta} \bar e \xrightarrow{\alpha} e            
      \end{equation}
      in $\mathcal E$, with $\alpha \in \mathcal E_{E_0 a_0}$ and $\pi(\alpha\beta) = E_0 f_0$.
      We must show that the composites
      \begin{align*}
        (a_0, \bar{\bar e}) & \xrightarrow{(id, W^{\**}\beta)}
                              (a_0, f_0^{\**} \bar e) \xrightarrow{(f_0, f_0^{\**})}
                              (b_0, \bar e) \xrightarrow{(id, \alpha)}
                              (b_0, e)
        \\
        (a_0, \bar{\bar e}) & \xrightarrow{(id, W^{\**}(\alpha\beta))}
                              (a_0, f_0^{\**} e) \xrightarrow{(f_0, f_0^{\**})}
                              (b_0, e)
      \end{align*}
      agree. However, these are all arrows of type (I), and so the composites are both $(f_0, \alpha\beta)$.

      Second, suppose we have a word/map $W = a_1$, with $\alpha$, $\beta$ as before such that $\alpha \in \mathcal E_{E_0 d_0 a_1}$ and $\pi(\alpha\beta) = E_0 a_1$.
      Then we have the following diagram
      \begin{equation}
            \begin{tikzcd}[column sep = large]
                  (d_1 a_1, \bar{\bar e}) \arrow[r, "{(id, W^{\**} \beta)}"] \arrow[dr, "{(id, W^{\**}(\alpha\beta))}"']
                  &
                  (d_1 a_1, (\phi_0^{-1} \phi_1)^{\**} \bar e) \arrow[d, "{(id, (\phi_0^{-1} \phi_1)^{\**} \alpha)}"] \arrow[r, "a_1 \wedge \phi_0^{-1,\**} \bar e"]
                  &
                  (d_0 a_1, \bar e) \arrow[d, "{(id, \alpha)}"]
                  \\
                  &
                  (d_1 a_1, (\phi_0^{-1} \phi_1)^{\**} e) \arrow[r, "a_1 \wedge \phi_0^{-1,\**} e"']
                  &
                  (d_0 a_1, e).
            \end{tikzcd}
      \end{equation}
      This commutes by relation (iii) in $|\pi^{\**} \mathcal C_\bullet|$ on the map $(id_{a_1}, \phi_0^{-1,\**}\alpha)$,
      as desired.

      It remains to check that $G$ is independent of the choice of word representative. We will show that the relations all hold in the image of $G$.
      (i) through (iii) are straightforward from the fact the composition works as above.
      For (iii), given $f_1: a_1 \to b_1$, we have two word decompositions
      \begin{equation}
            W:
            \quad 
            d_1 a_1 \xrightarrow{a_1} d_0 a_0 \xrightarrow{d_0 f_1} d_0 b_1,
            \qquad \qquad
            V:
            \quad
            d_1 a_1 \xrightarrow{d_1 f_1} d_1 b_1 \xrightarrow{b_1} d_0 b_1
      \end{equation}
      whose images under $E$ are equal, and we must check that for any $\alpha: \bar e \to e$ over this image,
      the images under $G$ are equal:
      \begin{align*}
        (d_1 a_1, \bar e) & \xrightarrow{(id, W^{\**}\alpha)}
                            (d_1 a_1, (\phi_0^{-1}\phi_1)^{\**} (d_0f_1)^{\**} e) \xrightarrow{a_1 \wedge \phi_0^{-1,\**} (d_0 f_1)^{\**} e}
                            (d_0 a_1, (d_0 f_1)^{|**} e) \xrightarrow{(d_0f_1, d_0 f_1^{\**})}
                            (d_0 b_1, e)
        \\
        (d_1 a_1, \bar e) & \xrightarrow{(id, V^{\**}\alpha)}
                            (d_1 a_1, (d_1f_1)^{\**} (\phi_0^{-1} \phi_1)^{\**} e) \xrightarrow{(d_1 f_1, d_1 f_1^{\**})}
                            (d_1 b_1, (\phi_0^{-1} \phi_1)^{\**} e) \xrightarrow{b_1 \wedge \phi_0^{-1, \**} e}
                            (d_0 b_1, e).
                            \stepcounter{equation}\tag{\theequation}\label{IMUG_EQ}
      \end{align*}
      But we note that $d_1 f_1^{\**} = \phi_1^{\**} \beta$ with $\beta = \phi_0^{-1,\**} (d_0 f_1)^{\**}$, as indicated by the following diagram, using relation (iii) from $|\mathcal C_\bullet|$.
      \begin{equation}
            \begin{tikzcd}[column sep = tiny, row sep = small]
                  (d_1 f_1)^{\**} (\phi_0^{-1} \phi_1)^{\**} e \arrow[rr, equal] \arrow[dd, mapsto] \arrow[dr, dashed, "\exists !"]
                  &&
                  (\phi_0^{\-1} \phi_1)^{\**} (d_0 f_1)^{\**} e \arrow[rr] \arrow[dd, mapsto] \arrow[dr, dashed, "\exists !"]
                  &&
                  \phi_0^{-1,\**} (d_0 f_1)^{\**} e \arrow[rr] \arrow[dd, mapsto] \arrow[dr, dashed, "\exists !", "\beta"']
                  &&
                  (d_0 f_1)^{\**} e \arrow[dr] \arrow[dd, mapsto]
                  \\
                  &
                  (\phi_0^{-1} \phi_1)^{\**} \arrow[rr, equal, crossing over]
                  &&
                  (\phi_0^{-1} \phi_1)^{\**} e \arrow[rr, crossing over]
                  &&
                  \phi_0^{-1,\**} e \arrow[rr, crossing over]
                  &&
                  e \arrow[dd, mapsto]
                  \\
                  E_0 d_1 a_1 \arrow[rr, equal] \arrow[dr]
                  &&
                  E_0 d_1 a_1 \arrow[dr] \arrow[rr]
                  &&
                  E_1 a_1 \arrow[rr] \arrow[dr]
                  &&
                  E_0 d_0 a_1 \arrow[dr]
                  \\
                  &
                  E_0 d_1 b_1 \arrow[uu, mapsfrom, crossing over] \arrow[rr, equal]
                  &&
                  E_0 d_1 b_1 \arrow[uu, mapsfrom, crossing over] \arrow[rr]
                  &&
                  E_1 b_1 \arrow[uu, mapsfrom, crossing over] \arrow[rr]
                  &&
                  E_0 d_0 b_1 
            \end{tikzcd}
      \end{equation}
      Hence, by relation (iii) for $|\pi^{\**} \mathcal C_\bullet|$, the maps in \eqref{IMUG_EQ} agree.

      Relation (iv) holds by a similar argument. Thus $G$ is a well-defined inverse to the map $F$, showing $|\pi^{\**} \mathcal C_\bullet|$ and $\pi^{\**}|\mathcal C_\bullet|$ are isomorphic.      
      % Moreover, since $\pi$ is a fibration, any choice of factorization of the arrow $f$ into its generating pieces
      % must in turn produce a factorization of the map $\alpha$ (up to isomorphism on the source).
      % Thus, we have that arrows here are generated by
      % \begin{enumerate}[label = (\roman*)]
      % \item pairs $(f_0: a_0 \to \bar a_0, (E_0(f_0))^{\**}e \to e)$ with $p(e) = E_0(a_0)$.
      % \item pairs $(a_1:d_1(a_1) \to d_0(a_1), \phi_1^{\**}(\phi_0^{-1})^{\**} e \to (\phi_0^{-1})^{\**}e \to e)$ with $p(e) = E_0d_0(a_1)$.
      % \end{enumerate}

      % Thus, it suffices to show that
      % (i) all maps are (isomorphic to one) of the form $E_0(f_0)^{\**}e \to e$, and
      % (ii) all objects $e$ are (isomorphic to one) of the form $(\phi_0 \pi)^{\**} \bar e$.

      % (ii) follows since $\phi_0$ is invertible, so $e = (\phi_0\pi)^{\**} (\phi_0^{-1}\pi)^{\**} e$.
      % (i).... i don't know yet.
      % \todo[inline]{come back}.
\end{proof}


\begin{proposition}
      \label{OU_RED2_PROP}
      $\O[u] \simeq \Lan_{G \ltimes (\OC^{e} \to \SC)^{op}} \tilde N^{(\O,X,Y)}$.
\end{proposition}
\begin{proof}
      We have that $|\OC^{\bullet, \lambda_l}| \simeq \pi^{\**} |\Omega^{\bullet, \lambda_l}| = \pi^{\**} \Omega^{t, \lambda_l} = \OC^{t, \lambda_l}$ % $\simeq \OC^{t,\lambda_l}$ and $|\OC^{t,\lambda_\bullet}| \simeq \OC^e$
      by combining Remark 5.36 and Proposition 5.41 of \cite{BP17} with Proposition \ref{PIREAL_PROP} above.
      Thus, Proposition \ref{PRESSTLIM PROP} tells us that
      $|G^{op} \ltimes \OC^{\bullet, \lambda_\bullet}| \simeq G^{op} \ltimes \OC^e$.
      Finally, the realization tool \cite[Prop. 5.37]{BP17} has no dependence whatsoever on $\Sigma_G$,
      and the natural transformation component of both sets of the simplicial maps in $N_{\bullet, \bullet}^{(\O,X,Y)}$ are natural isomorphisms by the same analysis as in \cite{BP17},
      and thus a double application on the output of Corollary \ref{OU_RED1_PROP} yields the result.
\end{proof}

\begin{notation}
      Let $\tilde N$ denote $\tilde N^{(\O,X,Y)}$ and the restriction of this functor to any faithful subcategory of $G \ltimes \OC^e$.
\end{notation}

We have a final reduction in this presentation.

 % \begin{lemma}
 %       If $\mathcal C \to \mathcal D$ is $\Ran$-initial over $\mathcal E$ in $\Cat^G$, then
 %       $G \ltimes \mathcal C \to G \ltimes \mathcal D$ is $\Ran$-initial over $G \ltimes \mathcal E$ in $\mathsf{Fib}(G)$. 
 % \end{lemma}


\begin{lemma}
      $G \ltimes \hat\Omega_{\mathfrak C}^e \into G \ltimes \OC^e$ is $\Ran$-initial.
\end{lemma}
\begin{proof}
      As $\hat\Omega^e \into \OC^e$ is $\Ran$-initial over $\Sigma$, this follows from Lemmas \ref{RANINIT_PULL_LEM} and \ref{GL_RANINIT_LEM}
\end{proof}
\begin{proof}
      We have a right retraction $\hat \Omega^e \into \Omega^e \to \hat \Omega^e$
      by \cite[Prop. 5.49]{BP17},
      and functoriality of $G \ltimes (-)$ and $\pi_{\mathfrak C}^{\**}$ implies
      we have a right retraction
      $G \ltimes \hat \Omega_{\mathfrak C}^e \into G \ltimes \OC^e \to G \ltimes \hat \Omega_{\mathfrak C}^e$.
      As before, this implies we have right retractions on all undercategories, showing $\Ran$-initiality.
\end{proof}

\begin{proposition}
      \label{OU_RED3_PROP}
      $\O[u] \simeq \Lan_{(G \ltimes \hat\Omega_{\mathfrak C}^{e} \to G \ltimes \SC)^{op}} \tilde N^{(\O,X,Y)}$. 
\end{proposition}

We will now filter the category $\hat\Omega_{\mathfrak C}^e$ by full subcategories (and hence also $G \ltimes \hat \OC^e$)
to produce a levelwise filtration of $\O \to \O[u]$.

\begin{definition}
      Given $T \in \OC^e$, define
      $V^X(T)$ (resp. $V^Y(T)$) to be the set of $X$-labeled (resp. $Y$-labeled) vertices of $T$.
      Further, define the \textit{degrees} of $T$ by
      \begin{equation}
            |T|_X = |V^X(T)|,
            \qquad
            |T|_Y = |V^Y(T)|,
            \qquad
            |T| = |T|_X + |T|_Y.
      \end{equation}
      Lastly, for $T \in \OC^a$, let $V^{in}(T)$ and $V^{ac}(T)$ denote the inert and active vertices of $T$,
      where we observe that $|T| = |V^{in}(T)|$.
\end{definition}

\begin{definition}[{cf. \cite[Defn. 5.56]{BP17}}]
      Define the following full subcategories of $\hat\Omega_{\mathfrak C}^e$.
      \begin{itemize}
      \item $\hat\Omega_{\mathfrak C}^e[\leq k]$ (resp. $\hat\Omega_{\mathfrak C}^e[k]$) is the subcategory spanned by
            those $T$ with $|T| \leq k$ (resp. $|T| = k$).
      \item $\hat\Omega_{\mathfrak C}^e[\leq k \setminus Y]$ (resp. $\hat\Omega_{\mathfrak C}^e[k \setminus Y]$) is the further subcategory of $\hat\Omega_{\mathfrak C}^e[\leq k]$ (resp. $\hat\Omega_{\mathfrak C}^e[k]$) with $|T|_Y \neq k$.
      \end{itemize}

      Finally, define the full subcategories $G^{op} \ltimes \hat \Omega_{\mathfrak C}^e[\bullet] \subseteq G \ltimes \hat \Omega_{\mathfrak C}^e$
      to be spanned by the same objects.
\end{definition}

\begin{remark}
      \label{OEFIB_REM}
      We have $\hat\Omega^e[\leq k]_{\mathfrak C} = \hat\Omega_{\mathfrak C}^e[\leq k]$, and similarly for the other constructions.
      In particular, applying Corollary \ref{PI_GFIB_COR} and Lemma \ref{GL_GR_LEM} to \cite[Remark 5.57]{BP17}, we have a map of Grothendieck fibrations
      \begin{equation}
            \begin{tikzcd}
                  G^{op} \ltimes \hat\Omega_{\mathfrak C}^e[k \setminus Y] \arrow[rr, hookrightarrow] \arrow[dr]
                  &&
                  G^{op} \ltimes \hat\Omega_{\mathfrak C}^e[k] \arrow[dl]
                  \\
                  &
                  G^{op} \ltimes \Omega_{\mathfrak C}^a[k]
            \end{tikzcd}
      \end{equation}
      such that fibers over $T \in \Omega_{\mathfrak C}^a[k]$ are the punctured cube and cube categories
      \begin{equation}
            (Y \to X)^{\times V^{in}(T)} - Y^{\times V^{in}(T)},
            \qquad
            (Y \to X)^{\times V^{in}(T)}.
      \end{equation}
\end{remark}

\begin{lemma}
      \label{LANINT_LEM}
      $G^{op} \ltimes \hat\Omega_{\mathfrak C}^e[\leq k-1] \into G^{op} \ltimes \hat\Omega_{\mathfrak C}^e[\leq k \setminus Y]$
      is $\Ran$-initial over $G^{op} \ltimes \SC$.
\end{lemma}
\begin{proof}
      This follows from combining Lemmas \ref{GL_RANINIT_LEM} and \ref{RANINIT_PULL_LEM} with the non-colored case \cite[Lemma 5.58]{BP17}.
\end{proof}
\begin{remark}
      This can also be done directly, completely analogously to the proof of \cite[Lemma 5.58]{BP17}.
\end{remark}

We can now construct our filtration.
\begin{definition}
      Let $\O_k$ denote the left Kan extnesion
      \begin{equation}
            \begin{tikzcd}
                  |[alias = A]| G \ltimes \hat\Omega_{\mathfrak C}^e[\leq k]^{op} \arrow[r, "\tilde N"] \arrow[d, "\mathsf{lr}"']
                  &
                  \V
                  \\
                  G \ltimes \SC^{op} \arrow[ur, "\O_k"', ""{name = B}]
                  \arrow[Rightarrow, from = A, to = B]
            \end{tikzcd}
      \end{equation}
\end{definition}

Since $G^{op} \ltimes \hat\Omega_{\mathfrak C}^e[\leq 0] \simeq G^{op} \ltimes \SC$, and
$G^{op} \ltimes \hat\Omega_{\mathfrak C}^e$ is the union of (the nerves of) the $G^{op} \ltimes \hat\Omega_{\mathfrak C}^e[\leq k]$,
we obtain our desired filtration
\begin{equation}
      \O = \O_0 \to \O_1 \to \O_2 \to \dots \to \colim_k\O_k = \O[u].
\end{equation}

To further analyze the connecting maps, consider the diagram of inclusions
\begin{equation}
      \label{FILT_PRELAN_EQ}
      \begin{tikzcd}
            G^{op} \ltimes \hat\Omega_{\mathfrak C}^e[k \setminus Y] \arrow[r] \arrow[d]
            &
            G^{op} \ltimes \hat\Omega_{\mathfrak C}^e[\leq k \setminus Y] \arrow[d]
            \\
            G^{op} \ltimes \hat\Omega_{\mathfrak C}^e[k] \arrow[r]
            &
            G^{op} \ltimes \hat\Omega_{\mathfrak C}^e[\leq k].
      \end{tikzcd}
\end{equation}
In fact, this is a pushout at the level of nerves, since
\begin{equation}
      G^{op} \ltimes \hat\Omega_{\mathfrak C}^e[k]
      \cap
      G^{op} \ltimes \hat\Omega_{\mathfrak C}^e[\leq k \setminus Y]
      =
      G^{op} \ltimes \hat\Omega_{\mathfrak C}^e[k \setminus Y],
      \qquad
      G^{op} \ltimes \hat\Omega_{\mathfrak C}^e[k]
      \cup
      G^{op} \ltimes \hat\Omega_{\mathfrak C}^e[\leq k \setminus Y]
      =
      G^{op} \ltimes \hat\Omega_{\mathfrak C}^e[\leq k],
\end{equation}
and a map $T \to S$ in $G^{op} \ltimes \hat\Omega_{\mathfrak C}^e[\leq k]$ is in one of the subcategories named in \eqref{FILT_PRELAN_EQ} iff $T$ is.

Applying left Kan extension over $G \ltimes \SC^{op}$ to \eqref{FILT_PRELAN_EQ} and reducing via Lemma \ref{LANINT_LEM}
yields the pushout diagram
\begin{equation}
      \begin{tikzcd}
            \label{FILT_LAN_EQ}
            \Lan_{G \ltimes \hat\Omega_{\mathfrak C}^e[k \setminus Y]^{op}} \tilde N \arrow[r] \arrow[d]
            &
            \O_{k-1} \arrow[d]
            \\
            \Lan_{G \ltimes \hat\Omega_{\mathfrak C}^e[k]^{op}} \tilde N \arrow[r]
            &
            \O_k
      \end{tikzcd}
\end{equation}


We can finally prove our formula for the levelwise filtration.
\begin{proof}
      [Proof of Proposition \ref{FILT_PUSH_PROP}]
      By first left Kan extending to $G \ltimes \OC^a[k]^{op}$, we can rewrite the leftmost map in \eqref{FILT_LAN_EQ} as
      \begin{equation}
            \Lan_{G \ltimes (\OC^a[k] \to \SC)^{op}}\left(
                  \bigotimes_{v \in V^{ac}(T)}\O(T_v) \otimes
                  \mathop{\mathlarger{\mathlarger{\mathlarger{\square}}}}\limits_{v \in V^{in}(T)} u(T_v)
            \right).
      \end{equation}
      using Remark \ref{OEFIB_REM}.
      Finally, the description of the leftmost map in \eqref{FILT_PUSHG_EQ} follows by noting that
      the undercategories $C \downarrow G \ltimes \OC^a[k]^{op}$ are groupoids.
\end{proof}

\begin{remark}
      We also have a description of $\O_{k-1}(C) \to \O_k(C)$ as the pushout below,
      as the analysis after Proposition \ref{OU_RED2_PROP} basically follows just as well without applying $G \ltimes (-)$.
      \begin{equation}
            \label{FILT_PUSH_EQ}
            \begin{tikzcd}
                  \mathop{\coprod}\limits_{[T] \in \Iso(C \downarrow_r \Omega_{\mathfrak C}^a[k])}
                  \left(
                        \mathop{\bigotimes}\limits_{v \in V^{ac}(T)} \O(T_v) \otimes
                        Q^{in}_T[u]
                  \right) \cdot_{\Aut_{\OC} (T)} \Aut_{\SC}(C)
                  \arrow[r]
                  \arrow[d]
                  &
                  \O_{n-1}(C) \arrow[d]
                  \\                  
                  \mathop{\coprod}\limits_{[T] \in \Iso(C \downarrow_r \Omega_{\mathfrak C}^a[k])}
                  \left(
                        \mathop{\bigotimes}\limits_{v \in V^{ac}(T)} \O(T_v) \otimes
                        \mathop{\bigotimes}\limits_{v \in V^{in}(T)} Y(T_v)
                  \right) \cdot_{\Aut_{\OC}(T)} \Aut_{\SC}(C)
                  \arrow[r]
                  &
                  \O_n(C)
            \end{tikzcd}
      \end{equation}
\end{remark}






% -------------------- ORIGINAL "PLAN OF ATTACK" ------------------------------

% \begin{proof}
%       {\color{red} Apply $\pi^{\**}$ to \cite[{\S 5.1-5.3}]{BP17}.      }
% \end{proof}

% Expanding on the above proof:

% First, suppose $G = \**$.
% Using 2-naturality of $\pi_{\mathfrak C}^{\**}$ and Proposition \ref{PRESSTLIM PROP} % preservation of standard limits
% it is immediate that $\Omega^e_{\mathfrak C} \simeq | \Omega_{\mathfrak C}^{t, \lambda_\bullet} |$,
% and as the proof of \cite[Prop. 5.37]{BP17} uses literally nothing about $\Sigma_G$, we may conclude
% \begin{equation}
%       \O[u] \simeq \Lan_{(\Omega_{\mathfrak C}^e \to \Sigma_{\mathfrak C})^{op}} \tilde N ^{\O, X, Y}.
% \end{equation}

% As a choice of coloring descends to subtrees, \cite[Prop. 5.49]{BP17} also follows, and thus
% \begin{equation}
%       \O[u] \simeq \Lan_{(\widehat\Omega_{\mathfrak C}^e \to \Sigma_{\mathfrak C})^{op}} \tilde N ^{\O, X, Y},
% \end{equation}
% where $\widehat\Omega_{\mathfrak C}^e \into \Omega_{\mathfrak C}^e$ is the full subcategory of $(\O, X,Y)$-labeled trees
% whose underlying tree is alternating, with active nodes labeled by $\O$ and inert nodes labeled by $X$ or $Y$.
% All of \cite[\S 5.3]{BP17} also follows immediately, and so we have the above filtration when $G = \**$.

% Now, for general $G$,  $|G \ltimes \C_\bullet | \simeq G \ltimes |\C_\bullet|$ by Proposition \ref{GT_REAL_PROP},
% and hence we're done:

% \begin{itemize}
% \item $G \ltimes \Omega_{\mathfrak C}^{t, \lambda} \simeq | G \ltimes \Omega_{\mathfrak C}^{\bullet, \lambda} |$.
% \item $G \ltimes \Omega_{\mathfrak C}^e \simeq | G \ltimes \Omega_{\mathfrak C}^{t, \lambda_\bullet} |$.
% \end{itemize}
% Hence $G \ltimes \Omega_{\mathfrak C}^{\bullet, \lambda_\bullet}$ is a bisimplicial category with all the properties that we need
% (since $G \ltimes (-)$ is functorial and preserves pullbacks)
% so that $\O[u] \simeq \Lan_{\left( \widehat\Omega_{\mathfrak C}^e \to \Sigma_{\mathfrak C} \right)^{op}} \tilde N^{\O,X,Y}$.
%  The rest of \cite[\S 5.3]{BP17} follows without a hitch.

 
\subsection{Model structures}

Recall the following from \cite{Ste16, BP17}.
\begin{definition}
      We say $\V$ has \textit{cellular fixed points} if
      for all finite groups $G$ and subgroups $H, K \leq G$ one has that:
      \begin{enumerate}[label = (\roman*)]
      \item fixed points $(-)^H: \V^G \to \V$ preserve direct colimits;
      \item fixed points $(-)^H$ preserve pushouts where one leg is $(G/K) \cdot f$, for $f$ a cofibration in $\V$.
      \item for each object $X \in \V$, the natural map $(G/K)^H \cdot X \to ((G/K) \cdot X)^H$ is an isomorphism.
      \end{enumerate}
\end{definition}

\begin{remark}
      \label{LEVEL_COF_REM}
      An immediate consequence of cellularity is that for any genuine cofibration $f \in \V^G_{gen}$,
      $f^H$ is a cofibration in $\V$ for all $H \leq G$.
\end{remark}


\begin{proposition}
      \label{TRANS_MODEL_PROP}
      Let $\V$ be a cocomplete model category will cellular fixed points.
      Now suppose $\mathcal D$ is a groupoid, and let $\F_d$ be a family of subgroups of $\mathsf{Aut}(d)$ for each $d \in \mathcal D$.
      Then the category of diagrams $\V^{\mathcal D}$ has an \textit{$\F$-model structure} $\V^\D_\F$, where
      a map $f: X \to Y$ is a
      weak equivalence (resp. fibration) iff $f(d): X(d) \to Y(d)$ is so in $\V^{\mathsf{Aut}(d)}_{\F_d}$ for each $d \in \mathcal D$.

      Moreover, if $X \in \V^{\mathcal D}$ is cofibrant, it is also levelwise cofibrant.
\end{proposition}
\begin{proof}
      This is the model structure transferred along the adjunction
      \begin{equation}
            \begin{tikzcd}
                  \V^\D \leftrightarrows
                  \mathop{\prod}\limits_{d \in \D}\V^{\mathsf{Aut}(d)}_{\F_d}
            \end{tikzcd}
      \end{equation}
      which exists by a straightforward exercise adapting and combining the proofs of
      \cite[Thm 11.6.1]{Hir03} and \cite[Prop 2.6]{Ste16};
      the generating (trivial) cofibrations are given by
      \begin{equation}
            I^{\mathcal D}_\F = \sets{\mathcal D(d,-) \cdot_{\mathcal D(d,d)} \mathcal D(d,d)/H \cdot i}{d \in \mathcal D, i \in I, H \in \F_d}
      \end{equation}
      where $I$ is the set of generating (trivial) cofibrations in $\V$.
\end{proof}

\begin{example}
      Let $\V$ be a cocomplete model category with cellular fixed points,
      $\mathfrak C$ a $G$-set, and $\F = \set{\F_n}$ a collection of families $\F_n$ of graph subgroups of $G \times \Sigma_n$.
      Then $\Sym^{G,\mathfrak C}(\V) = \V^{G \ltimes \Sigma_{\mathfrak C}^{op}}$ has an $\F$-model structure
      \begin{equation}
            \Sym^{G,\mathfrak C}_\F(\V) = \prod_n \V^{(G \times \Sigma_n^{op}) \ltimes \mathfrak C^{\times n+1}}_{\F_n},
      \end{equation}
      where
      $(\F_n)_{\xi}$ is all $\Gamma \in \F_n$ such that $\Gamma \leq \mathsf{Aut}(\xi)$.
      % $\V^{B_{\mathfrak C^{\times n+1}}(G \times \Sigma_n)}_{\F_n}$ has the model structure lifted from the adjunction to
      % $\prod_{(c_i)}\V^{\mathsf{Aut}(c_1,\dots,c_n; c_0)}_{\F_{(c_i)}}$ where
      % $\F_{(c_i)}$ is the set of all $\Gamma \in \F_n$ such that $\Gamma \leq \mathsf{Aut}(c_1,\dots, c_n;c_0)$.
\end{example}

% On the other hand, let $\Sym_{\F, \mathfrak C}(\V)$ denote the category $\V^{\UC\Sigma_\F}$ with the projective model structure.

% \begin{lemma}
%       \label{EXMAIN_LEM}
%       Let $\V$ be {\color{red} GOOD}.
%       Let $\P \in \Sym_{G, \mathfrak C}(\V)$ be level genuine cofibrant, and
%       $u: X\ to Y$ in $\Sym_{G, \mathfrak C}(\V)$ be a level genuine cofibration.
%       Then, for each $(T, \mathfrak c) \in \UC \Omega_G^a [k]$, and writing $C = \mathsf{lr}(T)$, the map
%       \begin{equation}
%             \left(
%                   \bigotimes_{v \in V_G^{ac}(T)}\P(T_v, \mathfrak c) \otimes
%                   {\mathlarger{\mathlarger{\mathlarger{\square}}}}u(T_v, \mathfrak d_v)
%             \right)
%             \mathop{\otimes}\limits_{\Aut(T, \mathfrak c)}\Aut(C, \mathfrak c)
%       \end{equation}
%       is a genuine cofibration in $\V^{\Aut(C, \mathfrak c)}_{\mbox{gen}}$,
%       which is trivial if $k \geq 1$ and $u$ is trivial.
% \end{lemma}
% \begin{proof}
%       This follows from \cite[Prop 6.24]{BP17} analogously as \cite[Lemma 5.72]{BP17} did.
% \end{proof}

We need a colored (but non-$G$-tree) generlization of a result from \cite{BP17}.
Let $X \to Y \in \Sym^{G, \mathfrak C}(\V)$ be called a \textit{level genuine cofibration} if for each $C \in \G \ltimes \SC^{op}$,
$X(C) \to Y(C)$ is a genuine cofibration in $\V^{\Aut(C)}$.
Similarly call $X$ \textit{level genuine cofibrant} if $\varnothing \to X$ is a level genuine cofibration.

\begin{lemma}[{cf. \cite[5.73]{BP17}}] % EXMAINLEM LEM
      \label{EXMAINLEM LEM}
      Suppose $\mathcal{V}$ is a cofibrantly generated closed monoidal model category
      with cellular fixed points and
      with cofibrant symmetric pushout powers.
      
      Let $\O \in \mathsf{Sym}^{G,\mathfrak C}(\mathcal{V})$
      be level genuine cofibrant
      and  
      $u: X \to Y$ in $\Sym^{G, \mathfrak C}(\V)$ a level genuine cofibration. 
      Then for each $T \in G \ltimes \Omega^a[k]^{op}$ and writing
      $C = \mathsf{lr}(T)$, the map	
      \begin{equation}\label{EXMAINLEM EQ}
            \left(
                  \bigotimes\limits_{v \in V^{ac}(T)}\P(T_v) \otimes
                  \underset{v \in V^{in}(T)}
                  {\mathlarger{\mathlarger{\mathlarger{\square}}}}
                  u(T_v)
            \right) 
            \mathop{\otimes}\limits_{\mathsf{Aut}(T)} \mathsf{Aut}(C).
      \end{equation}
      is a genuine cofibration in 
      $\mathcal{V}^{\mathsf{Aut}(C)}_{\text{gen}}$,
      which is trivial if $k \geq 1$ and $u$ is trivial.	
\end{lemma}
\begin{proof}
      This follows as in \cite{BP17}, as we may take the automorphism groups to live in the category $F \wr (G^{op} \ltimes \Sigma_{\mathfrak C}).$ 
\end{proof}




\begin{theorem}
      \label{THM1_C}
      Let $(\V,\otimes)$ denote either $(\sSet, \times)$ or $(\sSet_{\**} \wedge)$,
      and fix a $G$-set $\mathfrak C$.
      Then there exist model structures on $\Op^{G, \mathfrak C}(\V)$ such that
      $\O \to \O'$ is a weak equivalence (resp. fibration) if the maps
      \begin{equation}
            \label{THM1_C_EQ}
            \O(\xi)^\Gamma \to \O'(\xi)^\Gamma
      \end{equation}
      are weak equivalences (resp. fibrations) in $\V$ for all
      $\mathfrak C$-signatures $\xi$ and
      graph subgroups $\Gamma \leq \Stab(\xi)$.

      More generally, for $\F = \set{\F_n}_{n \geq 0}$ an indexing family with $\F_n$ an \textit{arbitrary} collection of subgroups of $G \times \Sigma_n$,
      there exists a model category structure on $\Op^{G, \mathfrak C}(\V)$, which we denote $\Op^{G, \mathfrak C}_\F(\V)$,
      with weak equivalences (resp. fibrations) determined by \eqref{THM1_C_EQ} for $\Gamma \leq (\F_n)_\ksi = \F_n \cap \Stab(\ksi)$.

      Lastly, analogous semi-model category structures $\Op^{G, \mathfrak C}(\V)$, $\Op^{G, \mathfrak C}_\F(\V)$ exist provided that
      $(\V, \otimes)$:
      \begin{enumerate*}[label = (\roman*)]
      \item is a cofibrantly generated model category;
      \item is a closed monoidal model category with cofibrant unit;
      \item has cellular fixed points;
      \item has cofibranty symmetric pushout powers.
      \end{enumerate*}
\end{theorem}
\begin{proof}
      This is the model structure lifted from $\Sym^{G, \mathfrak C}_\F(\V)$,
      and its existance and properties follow from
      Proposition \ref{FILT_PUSH_PROP} and Lemma \ref{EXMAINLEM LEM} above,
      completely analogously to the proof of Theorem I from \cite{BP17}.
\end{proof}

\begin{remark}
      The category $\Op^{G,\mathfrak C}(\Top)$ actually has full model structures lifted from $\Sym^{G, \mathfrak C}_\F(\Top)$
      for any indexing family $\F$,
      using an argument analogous to \cite[Thm. 3.1]{GW17}.
\end{remark}

We record a particular step in the proof of Theorem \ref{THM1_C}.
\begin{corollary}
      \label{LGC_COR}
      Suppose $\V$ has a cofibrant unit $1_\V$ and has cofibrant symmetric pushout powers.
      Let $\mathfrak C$ be a $G$-set, and $f: \O \to \P$ a map in $\Op^{G,\mathfrak C}(\V)$.
      If $f$ is a (trivial) cofibration in $\Op^{G, \mathfrak C}_\F(\V)$ for some indexing family $\F$,
      and $\O$ is level genuine cofibrant, then
      $f(\ksi)$ is a genuine (trivial) cofibration in $\V^{\Stab(\ksi)}_{gen}$ for all $\mathfrak C$-signatures $\ksi$.
\end{corollary}
% \begin{proof}
%       It suffices to show that the map $\O \to \O[u]$ given by the pushout
%       \begin{equation}
%             \begin{tikzcd}
%                   \mathbb F^{\mathfrak C}(X) \arrow[r] \arrow[d, "u"']
%                   &
%                   \O \arrow[d]
%                   \\
%                   \mathbb F^{\mathfrak C}(Y) \arrow[r]
%                   &
%                   \O[u],
%             \end{tikzcd}
%       \end{equation}
%       with $u: X \to Y$ a generating (trivial) $\F$-cofibration in $\Sym^{G, \mathfrak C}_\F(\V)$
%       (and so in particular a local genuine (trivial) cofibration),
%       is a local genuine (trivial) cofibration.
      
%       Recalling the filtration in \eqref{FILT_PUSHG_EQ}, % We have/will show {\color{red} COME BACK} that we have a filtration
%       % \begin{equation}
%       %       \O = \O_0 \to \O_1 \to \dots \to \colim_n \O_n = \O[u]
%       % \end{equation}
%       % where locally we have $\O_{n-1}(\ksi) \to \O_n(\ksi)$ is given by the pushout over
%       % \begin{gather*}
%       %       \Lan_{\left(G \ltimes \Omega_{\mathfrak C} \to G \ltimes \Sigma_{\mathfrak C}\right)^{op}}
%       %       \left(
%       %             \bigotimes_{v \in V^{ac}(T)}\O(T_v, \mathfrak d) \otimes \mathop{\square}\limits_{v \in V^{in}(T)}u(T_v, mathfrak d)
%       %       \right)
%       %       \\
%       %       \simeq
%       %       \coprod_{[T] \in \Iso\left( C \downarrow_r G \ltimes \Omega^a[k] \right)}
%       %       \left(
%       %             \bigotimes_{v \in V^{ac}(T)}\O(T_v, \mathfrak d) \otimes
%       %             \left(
%       %                   Q^{in}_T[u] \to \bigotimes_{v \in V^{in}(T)}Y(T_v, \mathfrak d)
%       %             \right)
%       %       \right)
%       %       \otimes_{\Aut(T, \mathfrak d)} \Aut(C, \mathfrak c),
%       % \end{gather*}
%       % where $(C, \mathfrak c) = (C_{|\ksi|}, i \mapsto \ksi_i)$.
%       by \cite[Remarks 5.72(i) and 5.72(ii)]{BP17}, it suffices to check that the two maps
%       \begin{gather*}
%             \left(\varnothing \to \bigotimes_{v \in V^{ac}(T)}\O(T_v)\right)
%             =
%             \mathop{\square}\limits_{v \in V^{ac}(T)}(\varnothing \to \O)(T_v),
%             \qquad
%             \mathop{\square}\limits_{v \in V^{in}(T)}u(T_v)
%       \end{gather*}
%       are $\Aut\left((T_v)_{v \in V^{ac}(T)}\right)$- and $\Aut\left(((T_v)_{v \in V^{in}(T)}\right)$-genuine cofibration,
%       with the latter trivial if $u$ is.
%       These automophism groups are taken in $\Sigma \wr \SC$, the codomain of the vertex functor,
%       and are of the form $\prod_i \Sigma_{\lambda_i} \wr \Aut_{\SC}(T_{v_i})$ for the partition
%       $\lambda$ of $V(T)$ where with two vertices in the same class iff they are isomorphic in $\SC$.
%       % CYAN = With $G \ltimes (-)$
%       % {\color{cyan}
%       %   These automorphism groups are taken in $G \ltimes (\Sigma \wr \Sigma_{\mathfrak C})$, the codomain of the vertex functor.
%       %   These groups inject into $\Sigma \wr (G \ltimes \Sigma_{\mathfrak C})$ along the diagonal,
%       % and here the automorphism group is of the form
%       % $\Pi_i \Sigma_{|\lambda_i}| \wr \Aut(T_{v_i}, \mathfrak d)$.}    
%       Thus the result follows by applying, in order, Remark 5.72(ii), Proposition 6.24, and Remark 5.72(i) of \cite{BP17}.     
%       % \todo[inline]{come back: reference the used equations when they appear in hypothetical earlier discussion}      
% \end{proof}




\begin{remark}
      In both Theorem \ref{THM1_C} and Corollary \ref{LGC_COR}, as well as in \cite{BP17},
      the cofibrancy of the unit is necessary so that the initial operad $\mathbb F(\varnothing)$ is level genuine cofibrant.
\end{remark}



The following is immediate.
\begin{corollary}
      \label{COLOR_CHANGE_Q_COR}
      The change of color adjunctions from \eqref{GC_CHANGE_EQ},
      and the forgetful functor $j^{\**}$ from \eqref{JSTAR_CAT_EQ}
      form Quillen adjunctions.
\end{corollary}
% \begin{proof}
%       For any $F: \mathfrak C \to \mathfrak C'$, $F_{\**}$ clearly preserves (trivial) fibrations.
% \end{proof}

\begin{remark}
      In particular, this implies that $j^{\**}$ commutes with fibrant replacement.
\end{remark}









\section{Model structure for all colors} 
\renewcommand{\C}{\mathfrak C}

\textbf{Fix a cofibrantly generated model category $\V$.}
We generalize and synthesize \cite{BM13}, \cite{Cav14}, and \cite{CM13b} to assemble the collection of
``color-fixed'' model structures on each $\Op^{G, \mathfrak C}(\V)$ into a single model structure on $\Op^G(\V)$.


% \begin{definition}
%       We say $\V$ \textit{admits (semi)-transfer for categories/operads} if for any set $\mathfrak C$,
%       the categories $\Cat^{\mathfrak C}(\V)$, $\Op^{\mathfrak C}(\V)$
%       may be equipped with the canonical (semi)-model structure.
% \end{definition}

% \begin{remark}
%       If $\V$ satisfies the \textit{monoid axiom} of \cite{SS00}
%       or has a monoidal fibrant replacement functor and a comonoidal Hopf interval object \cite{BM03},
%       then $\V$ admits transfers for categories and operads by \cite{Mur11,Mur14} and \cite{BM03}, respectively.
% \end{remark}


In this section, further assume that $\V$ has cellular fixed points.

% \begin{definition}
%       Given $\V$ with cellular fixed points, we say
%       $\V$ \textit{admits (semi)-transfers for $G$-categories/operads} if
%       for any $G$-set $\mathfrak C$ and indexing collection $\F$,
%       the categories $\Cat^{G, \mathfrak C}(\V)$, $\Op^{G, \mathfrak C}(\V)$
%       may be equipped with the canonical $\F$-(semi)-model structure.
% \end{definition}

% \begin{example}
%       By Theorem \ref{THM1_C}, $(\sSet, \times)$ and $(\sSet_{\**}, \wedge)$ admit transfers for $G$-operads,
%       and any $(\V, \otimes)$ satisfying the hypotheses of said theorem admits semi-transfers for $G$-operads.
% \end{example}

% \begin{example}
%       \cite[Theorem 3.1]{GW17} shows that $(\Top, \times)$ admits transfers for $G$-operads when $\mathfrak C = \set{*}$.
% \end{example}

% \textbf{\color{OliveGreen} In this section, further assume that $\V$
%   has cellular fixed points and admits (semi)-transfer for $G$-categories and $G$-operads}.

\begin{remark}
      \label{JSTAR_REM}        
      Extending \eqref{JSTAR_CAT_EQ}, we have another inclusion-forgetful adjunction,      
      \begin{equation}
            \begin{tikzcd}
                  \mathsf{Op}^G(\V) \arrow[d, "(-)^H"']
                  \arrow[r, shift right, "j^*"']
                  &
                  \mathsf{Cat}^G(\V) \arrow[l, shift right, swap, "j_!"] \arrow[d, "(-)^H"]
                  \\
                  \Op(\V) \arrow[r, shift right, "j^*"']
                  &
                  \Cat(\V) \arrow[l, shift right, swap, "j_!"]
            \end{tikzcd}
      \end{equation}
      and again $j^{\**}$ commutes with $H$-fixed points and fibrant replacement.
\end{remark}

\begin{definition}
      We highlight three particular $\V$-categories.
      We let $1_\V$ and $\varnothing$ denote the unit object and initial object of $\V$, respectively.
      \begin{itemize} %{enumerate}[label = (\roman*)]
      \item Let $\I$ be the $\V$-category that detects isomorphisms: it has objects $\set{0,1}$,
            with $\I(0,0) = \I(0,1) = \I(1,0) = \I(1,1) = 1_\V$.
      \item Let $\mathbb A$ be the $\V$-category that detects arrows: it has objects $\set{0,1}$,
            with $\mathbb A(0,0) = \mathbb A(0,1) = \mathbb A(1,1) = 1_\V$, and $\mathbb A(1,0) = \varnothing$.
      \item Let $\1$ be the $\V$-category that detects objects: it has a single object $\set{\**}$, with $\1(\**, \**) = 1_\V$.
      \end{itemize}
\end{definition}

We recall and extend the following definitions from \cite{BM13}. 

\begin{definition}
      A {\em $\V$-interval} is a cofibrant object in $\Cat^{\set{0,1}}(\V)$ (with the transfered model structure)
      equipped with a weak equivalence $\J \to \I_f$.
      A set $\mathcal{G}$ of $\V$-intervals is {\em generating} if all $\V$-intervals $\J$ can be obtained
      as a retract of a trivial extension of an element in $\mathcal{G}$ in $\Cat^{\set{0,1}}(\V)$:
      \begin{equation}
            \begin{tikzcd}
                  \mathbb{G} \arrow[r,rightarrowtail, "\sim"]
                  &
                  \mathbb{K} \arrow[r,yshift=-.3em, "r"']
                  &
                  \mathbb{J} \arrow[l,yshift=.3em, "i"']
            \end{tikzcd}
      \end{equation}
\end{definition}

\begin{definition}
      \label{PL_ES_DEFN}
      We say a functor $F: \mathcal C \to \mathcal D$ in $\Cat(\V)$ is
      \begin{itemize} %{enumerate}[label = (\roman*)]
      \item \textit{path-lifting}
            if it has the right lifting property against all maps of the form
            $\1 \to \J$
            where $\J$ is a $\V$-interval.
      \item \textit{essentially surjective}
            if for any object $d: \1 \to \mathcal D$,
            there is an object $c: \1 \to \mathcal C$
            and a map $\J \to \mathcal D$ out of a $\V$-interval fitting in to the commuting diagram below.
            \begin{equation}
                  \begin{tikzcd}
                        \1 \arrow[rr, dashed, "c"] \arrow[dr, "i_0"]
                        &&
                        \mathcal C \arrow[dd, "F"]
                        \\
                        &
                        \J \arrow[dr, dashed]
                        \\
                        \1 \arrow[ur, " i_1"] \arrow[rr,"b"]
                        &&
                        \mathcal D
                  \end{tikzcd}
            \end{equation}
      \end{itemize}
\end{definition}

Now, recall that an \textit{indexing family} is a collection $\F = \set{\F_n}_{n \geq 0}$ of families $\F_n$ of graph subgroups of $G \times \Sigma_n$.
Especially well-behaved indexing families are the \textit{(weak) indexing systems} of \cite{BP17},
of which the maximal indexing family $\mathrm{Gr} = \set{\mathrm{Gr}_n}$ is one.

We say an indexing family $\F$ \textit{has units} if
$\F_1$ contains all graph subgroups of the form $H \leq G \times \Sigma_1$.
In particular, \cite[Remark 4.50]{BP17} implies that any \textit{weak indexing system} has units.

We may now extend the categorical model theoretic notions to the equivariant context.
\begin{definition}
      \label{MODEL_DEFN}
      Fix an indexing family $\F$.
      We call a map $F: \O \to \P$ in $\mathsf{Op}^G(\V)$
      \begin{itemize}
      \item a {\em local $\F$-fibration} (resp. {\em local weak $\F$-equivalence}) if
            $F(\xi): \O(\xi)\to \P(F(\xi))$
            is a fibration (resp. weak equivalence) in $\V^{\Stab(\xi)}_{\F_\xi}$ for all $\xi\in \C(\O)^{\times n+1}$ and all $n$.
      \item a {\em local trivial $\F$-fibration} if both a local $\F$-fibration and a local weak $\F$-equivalence.
      \item {\em essentially surjective} (resp. {\em path lifting}) if $j^*F^H$ is essentially surjective (resp. path lifting) in $\Cat(\V)$ for all $H\leq G$.
      \item a {\em $\F$-fibration} if both path-lifting and a local $\F$-fibration.
      \item a {\em weak $\F$-equivalence} if both essentially surjective and a local weak $\F$-equivalence.
      % \item a \textit{DK-$\F$-equivalence} if a local weak $\F$-equivalence such that
      %       $\pi_0 j^{\**}F^H$ (cf. Definition \ref{HTPY_DEFN}) is essentially surjective.
      \item a \textit{(trivial) $\F$-cofibration} if it has the left lifting property against all trivial $\F$-fibrations (resp. $\F$-fibrations).
      \end{itemize}
\end{definition}

% \begin{remark}
%       If $\V$ has diagonals, then $F \in \Op^G(\V)$ is a $DK$-$\F$-equivalence iff
%       $F$ is a local weak $\F$-equivalence such that 
%       the associated map of \textit{$\F$-genuine equivariant operads} under the composite
%       \begin{equation}
%             \Op^G(\V) \to \Op_\F(\V) \xrightarrow{\pi_0} \Op_\F(\Set) 
%       \end{equation}
%       is an equivalence.
% \end{remark}

% \begin{remark}
%       Trivial $\F$-fibrations are precisely local $\F$-fibrations which are surjective on objects.
%       Thus $\F$-cofibrations are $f: \O \to \P$ such that
%       each $\mathfrak C(\O)^H \to \mathfrak C(\P)^H$ is injective and
%       $f_!\O \to \P$ is an $\F$-cofibration in $\Op^{G, \mathfrak C(\P)}(\V)$.
% \end{remark}

We now state the main result of this section.

\begin{theorem}
      \label{MODEL_THM}
      Fix an indexing family $\F = \set{\F_n}$ with units.
      Let $(\V, \otimes)$, denote either $(\sSet, \times)$ or $(\sSet_{\**}, \wedge)$.
      Then there exists a cofibrantly generated model structure on the category $\Op^G(\V)$,
      denoted $\Op^G_\F(\V)$, with
      weak $\F$-equivalences, $\F$-fibrations, and $\F$-cofibrations defines as in Definition \ref{MODEL_DEFN}.
           
      Moreover, analogous semi-model category structures $\Op^G_\F(\V)$ exist
      provided that $(\V, \otimes)$:
      \begin{enumerate}[label = (\roman*)]\itemsep-4pt
      \item is a cofibrantly generated model category,
      \item is a closed monoidal model category with cofibrant unit
            \footnote{Cofibrant unit also needed for \ref{J-CELL_PROP}.},
      \item has cellular fixed-point functors,
      \item \label{CSPP_LBL} has cofibrant symmetric pushout powers,
            \footnote{Also needed for Lemmas \ref{CAV_4.14_PROP2}, Prop \ref{J-CELL_PROP}}, % \ref{LOCAL_COF_LEM}            
            % --------------------
      \item \label{RP_LBL} is right proper
            \footnote{Needed for Lemma \ref{RIGHTPROPER_LEM} and Lemma \ref{2OUTOF3_PROP}.},
      \item \label{GENSET_LBL} has a set $\mathbb{G}$ of generating $\V$-intervals
            \footnote{Needed so we have a \textit{set} of generating trivial cofibrations},
      \end{enumerate}
\end{theorem}
\begin{proof}
      In the first case, we have that the model category $\Op^{G,\mathfrak C}_\F(\sSet)$ exists
      for any $G$-set $\mathfrak C$ and indexing family $\F$ by Theorem \ref{THM1_C},
      while in the second case, conditions $(i)$ -- \ref{CSPP_LBL} are sufficient to construct the
      semi-model category $\Op^{G, \mathfrak C}_\F(\V)$ from said theorem.
      
      Moreover,
      % After this difference, the proofs of the two cases are identical, as
      every object in $\sSet^G$ is genuine cofibrant by e.g. \cite[Remark 5.71]{BP17},
      \ref{RP_LBL} $\sSet$ is right proper
      by Lemma \ref{INTER_LEM} and e.g. \cite[Prop 2.1.5]{Cis06} or \cite[Lemma 1.12]{BM13}, and
      \ref{GENSET_LBL} $\sSet$ has a generating set of intervals
      by e.g. \cite[Lemma 1.12]{BM13}.
      % \ref{TCWE_LBL} the class of genuine weak equivalences in $\mathsf{Op}^G(\sSet)$ is closed under transfinite compositions
      % by an argument analogous to \cite[Lemma 1.24]{CM13b}.
      % Now, we note that condition \ref{TCWE_LBL} proves the analogous statement for any $\F$,
      % since the transfinite composite of local $\F$-equivalences is a local $\F$-equivalence.

      Since $\mathsf{Op}^G(\V)$ is complete and cocomplete, it thus suffices to prove
      following \cite[Theorem 2.1.19]{Hov98} (or \cite[Theorem 2.2.2]{WY15} in the semi-model structure case) that:
      \begin{enumerate}[label = (\arabic*)]
      \item the class of weak $\F$-equivalences has the 2-out-of-3 property and is closed under retracts;
      \item the domains of $I_{\F}$ (resp. $J_{\F}$) are small relative to $I_{\F}$-cell (resp. $J_{\F}$-cell);
      \item $I_{\F}$-inj $= W\cap J_{\F}$-inj;
      \item $J_{\F}$-cell (with cofibrant source) $\subseteq W\cap I_{\F}$-cof.
      \end{enumerate}
      (1) follows from Proposition \ref{2OUTOF3_PROP} and the fact that if $L$ is a retract of $F$, $L^H$ is a retract of $F^H$.
      (2) follows since colimits in $\mathsf{Op}^G(\V)$ are created in $\Op(\V)$, and it holds non-equivariantly.
      (3) follows from Lemma \ref{CAV_4.8}.
      (4) follows from Lemma \ref{POINT_4_LEMMA} and Proposition \ref{J-CELL_PROP}.
\end{proof}

% \begin{remark}
%       If we could show via some other method that $(\V, \otimes)$ satisfied actual transfer for $G$-operads, then
%       conditions \ref{CSPP_LBL} -- \ref{TCWE_LBL} would imply that the $\F$-model structure existed on $\Op^G(\V)$. 
% \end{remark}

\begin{remark}
      This recovers the main results of \cite{BM13, Cav14} for $G = \set{e}$. 
\end{remark}



We spend the rest of this section proving the results needed in the proof of Theorem \ref{MODEL_THM},
beginning with a description of the sets of generating (trivial) cofibrations.

Fix a graph subgroup $\Gamma \in \F_n$ of $G \times \Sigma_n$, and $X \in \V^\Gamma$.
We now construct the ``free operad with stabilizer $\Gamma$ generated by $X$''.
Consider the $G$-set of colors $\mathfrak C_\Gamma := G \cdot_\Gamma \underline{n+1}$,
and let $\xi_0$ denote the signature $([e,1],[e,2],\dots,[e,n];[e,0])$.
We define the $(G,\mathfrak C_\Gamma)$-symmetric sequence
\begin{equation}
      C_\Gamma[X](\xi) =
      \begin{cases}
            (g,\sigma)^{\**} X \qquad \qquad & \xi = (g,\sigma).\xi_0
            \\
            \varnothing & \mbox{otherwise,}
      \end{cases}
\end{equation}
where $g \in G$ and $\sigma \in \Sigma_n$ are chosen to be the \textit{minimal} elements in those groups with this property,
and $\varnothing$ is the initial object in $\V$.

Let $\mathbb F_\Gamma[X]$ denote free operad $\mathbb F^{\mathfrak C_\Gamma} C_\Gamma[X]$.
It is straightforward that the operad $\mathbb F_\Gamma[X]$ has the universal property
\begin{equation}
      \Hom_{\Op^G(\V)}(\mathbb F_\Gamma[X], \O) = \mathop\prod\limits_{\xi \in (\mathfrak C(\O)^{\times n+1})^\Gamma}\Hom_{\V^\Gamma}(X, \O(\xi)).
\end{equation}

Define $I_{\F,loc}$ and $J_{\F, loc}$ to be the sets
\begin{align*}
  \set{\mathbb F_\Gamma[\Gamma/\Gamma \cdot i]}, \qquad \qquad \set{\mathbb F_\Gamma[\Gamma/\Gamma \cdot j]},
\end{align*}
where $\Gamma$ runs over all graph subgroups of $G \times \Sigma_n$ in $\F_n$,
and $i$ (resp. $j$) runs over all generating (trivial) cofibrations in $\V$.

The universal property makes the following immediate.
\begin{corollary}[{cf. \cite[\S 4.2]{Cav14}, \cite[1.16]{CM13b}}]
      $\O \to \O'$ is a local (trivial) $\F$-fibration iff
      $\O \to \O'$ has the right lifting property against $J_{\F, loc}$ (resp. $I_{\F, loc}$).
\end{corollary}

Now, define
\begin{equation}
      I_{\F}:= I_{\F, loc} \mathbin{\cup} \set{\varnothing \to G/H \cdot \1}_{H\leq G},
      \qquad \qquad
      J_{\F} := J_{\F, loc} \mathbin{\cup} \set{G/H \cdot (\1 \to \J)}_{H\leq G,\ \J\in\mathbb{G}}
\end{equation}
where $\1$ defined as in Definition \ref{PL_ES_DEFN}, and $\mathbb{G}$ is a generating set of $\V$-intervals. 

\begin{lemma}
      [{cf. \cite[4.8]{Cav14}, \cite[2.3]{BM13}, \cite[1.18]{CM13b}}]
      \label{CAV_4.8}
      Suppose $\V$ has a generating set of intervals.
      Then the following are equivalent for a map $F$ in $\Op^G(\V)$.
      \begin{enumerate}[label = (\arabic*)]
      \item $F$ is a trivial $\F$-fibration.
      \item $F$ is a local trivial $\F$-fibration such that $F^H$ is surjective on $H$-fixed colors for all $H\leq G$.
      \item $F$ has the right lifting property against $I_{\F}$.
      \end{enumerate}
\end{lemma}
\begin{proof}
      $(2) \Leftrightarrow (3)$ is immediate by the construction of $I_{\F}$.
      For $(1) \Leftrightarrow (2)$, we have by definition that
      $F$ is a trivial $\F$-fibration
      iff
      it is a local trivial $\F$-fibration such that $j^*F^H$ is path-lifting and essentially surjective for all $H\leq G$.
      \cite[2.4]{BM13} completes the proof. 
      % Moreover, right lifting against $I_{\F, loc}$ is identical to being a local trivial $\F$-fibration, while
      % lifting against $\varnothing \to G/H\otimes \1$ precisely say that $F^H$ is surjective on colors;
      % combining these observations yields the result.
\end{proof}

\begin{lemma}
      [{cf. \cite[1.20]{CM13b}, \cite[\S 4.3]{Cav14}}]
      $F$ has right lifting against $J_{\F}$ iff $F$ is an $\F$-fibration.
\end{lemma}
\begin{proof}
      Again, lifting against $J_{\F, loc}$ is identical to being a local $\F$-fibration, while lifting against $G/H \cdot (\1 \to \J)$
      is equivalent to $F^H$ lifting against $\1 \to \J$.
\end{proof}

\begin{lemma}
      [{cf. \cite[1.19]{CM13b}}]
      \label{POINT_4_LEMMA}
      $J_{\F}\mbox{-cof} \subseteq I_{\F}\mbox{-cof}$; that is, trivial cofibrations are cofibrations.
\end{lemma}
\begin{proof}
      % It suffices to show that if $F$ has (right) lifting against $I_\F$, it has lifting aginst $J_{\F}$.
      Clearly a local trivial $\F$-fibration is a local $\F$-fibration.
      On the other hand, by locality and \eqref{COLOR_SQ_EQ},
      any cofibration in $\mathsf{Op}^{G, \mathfrak C}(\V)$ for any $G$-set $\C$
      is a cofibration when considered in $\mathsf{Op}^G(\V)$.
      Thus, since $G/H \cdot (\1 \to \1 \amalg \1)$ is a pushout of $G/H \cdot(\varnothing \to \1)$
      and hence is in $I_{\F}\mbox{-cof}$, the composite
      \begin{equation}
            \begin{tikzcd}
                  G/H \cdot \1 \arrow[r, rightarrowtail]
                  &
                  G/H \cdot (\1 \amalg \1) \arrow[r, rightarrowtail]
                  &
                  G/H \cdot \J 
            \end{tikzcd}
      \end{equation}
      is in $I_{\F}\mbox{-cof}$.
      Thus $J_\F \subseteq I_\F\mbox{-cof}$, implying the result.
\end{proof}

\subsection{Trivial cofibrations}

Similar to many cases in the literature, the two most difficult steps in the proof of Theorem \ref{MODEL_THM} are showing that
$J_\F$-cells are weak equivalences, and that weak equivalences satisfy 2-out-of-3.
We show these results in that order.

\begin{lemma}
      \label{TRANSCOMP_ES_LEM}
      Suppose $\V$ is a monoidal model category.
      The transfinite composition of essentially surjective maps in $\Op^G(\V)$ is essentially surjective.
\end{lemma}
\begin{proof}
      Since taking fixed points commutes with filtered colimits, they commute with transfinite composition,
      and hence by \cite[4.17]{Cav14}, we are done.
\end{proof}

\begin{lemma}
      \label{TRANSCOMP_LGC_LEM}
      Local genuine cofibrations in $\Op^G(\V)$ are closed under transfinite composition.
\end{lemma}
\begin{proof}
      Since for any $\mathfrak C$-signature $\ksi$ and any map of $\mathfrak C$-colored operads $f$ we have that
      the restriction map
      $\V^{\Stab(f(\ksi))}_{gen} \to \V^{\Stab(\ksi)}_{gen}$
      is left Quillen,
      and any transfinite composition of operads locally is of the form
      \begin{equation}
            \O_0(\ksi) \to \O_1(F_1(\ksi)) \to \O_2(F_2(\ksi)) \to \dots   
      \end{equation}
      in $\V^{\Stab(\ksi)}_{gen}$
      (with $F_n: \O_0 \to \O_1 \to \dots \to \O_n$),
      the result follows.
\end{proof}

\begin{proposition}
      [{c.f. \cite[4.20]{Cav14}}]
      \label{J-CELL_PROP}
      Suppose $\V$ has a cofibrant unit $1_\V$ and has cofibrant symmetric pushout powers.
      Then relative $J_{\F}$-cells with locally genuine cofibrant source are weak equivalences.
\end{proposition}
\begin{proof}
      By Lemmas \ref{TRANSCOMP_ES_LEM} and \ref{TRANSCOMP_LGC_LEM}, it suffices to show that
      the pushout of a map $j \in J_\F$ is both
      essentially surjective and a local genuine trivial cofibration.

      Firstly, if $j = \mathbb F_\Gamma[\Gamma/\Gamma \cdot i] \in J_{\F, loc}$, 
      we are considering the pushout of a trivial cofibration in $\mathsf{Op}^{G,\mathfrak C(\P)}(\V)$ (cf. \eqref{COLOR_SQ_EQ}).
      By the existance of the $\F$-model structure on $\Op^{G, \mathfrak C(\P)}(\V)$ by Theorem \ref{THM1_C},
      this is again a trivial cofibration, and hence by Corollary \ref{LGC_COR} a local genuine trivial cofibration.
      As it is the identity on colors, it is also essentially surjective.
      
      Secondly, supppose $j$ is of the form $G/H \cdot (\1 \to \J)$ for $\J$ a $\V$-interval.
      We split this pushout into a composition of two pushouts
      \begin{equation}
            \begin{tikzcd}
                  G/H \cdot \1 \arrow[r, "a"] \arrow[d, "G/H \cdot \phi"']
                  % \arrow[dr,phantom, yshift=.1em, xshift=.5em, "\lrcorner" near end]
                  &
                  \O \arrow[d,"\phi'"]
                  \\
                  G/H \cdot \J_{\set{0}} \arrow[r] \arrow[d, "G/H \cdot \psi"']
                  % \arrow[dr,phantom, yshift=.1em, xshift=.5em, "\lrcorner" near end]
                  &
                  \O' \arrow[d,"\psi'"]
                  \\
                  G/H \cdot \J \arrow[r]
                  &
                  \P
            \end{tikzcd}
      \end{equation}
      where $\J_{\set{0}}$ is the full subcategory of $\J$ spanned by the object $0$.
      It suffices to show both $\psi'$ and $\phi'$ are local genuine trivial cofibrations which are essentially surjective on fixed points. 

      We first consider the bottom pushout.
      We know that $\psi$ is injective on colors and a local isomorphism in $\Op(\V)$,
      and hence so is $G/H \cdot \psi$ in $\Op^G(\V)$.
      Since colimits are created non-equivariantly, and equivariant isomorphisms are detected by invertible equivariant maps,
      \cite[Prop B.22]{Cav14} implies that $\psi'$ is also a local isomorphism in $\Op^G(\V)$.
      % so in particular a local trivial $\F$-cofibration.
      
      Moreover, we observe that $\C(\P) = \C(\O') \amalg (G/H \times \set{1})$.
      Thus, if $x \in \C(\P)^K$ for $K \leq G$ is in $\C(\O')$, we have essential surjectivity trivially,
      as shown on the left below in \eqref{J-CELL_EQ},
      where $\I_c \to \I$ is a cofibrant replacement in $\Op^{\set{0,1}}(\V)$.
      %
      \begin{equation}
            \label{J-CELL_EQ}
            \begin{tikzcd}
                  \1 \arrow[rrr, "x"] \arrow[dr, " i_0"]
                  &&&
                  (\O')^K \arrow[dd, "\psi'"]
                  &[15pt] % ----------
                  \1 \arrow[r, "0"] \arrow[dr, "i_0"']
                  &
                  \J_{\set{0}} \arrow[d] \arrow[r, "g"]
                  &
                  (\O')^K \arrow[d, "{\psi'}"]
                  \\
                  &
                  \I_c \arrow[r]
                  &
                  \I \arrow[dr, "x"]
                  &
                  & % ----------
                  &
                  \J \arrow[r, "g"]
                  &
                  \P^K \arrow[d, equal]
                  \\
                  \1 \arrow[ur, " i_1"] \arrow[rrr,"x"]
                  &&&
                  (\P)^K
                  & % ----------
                  \1 \arrow[rr, "g \cdot 1"] \arrow[ur, "i_1"]
                  &&
                  \P^K
            \end{tikzcd}
      \end{equation}
      %
      If instead $x  = g \cdot 1 \in (G/H \cdot 1)^K \subseteq \mathfrak C(\P)^K$,
      the pushout square yields the diagram on the right above in \eqref{J-CELL_EQ},
      where the maps $\J_{\set{0}} \xrightarrow{g} (\O')^K$, $\J \xrightarrow{g} \P^K$ are adjoint to the composites
      \begin{equation}
            G/K \cdot \J_{\set{0}} \xrightarrow{g} G/H \cdot \J_{\set{0}} \longto \O',
            \qquad \qquad
            G/K \cdot \J \xrightarrow{g} G/H \cdot \J \longto \P
      \end{equation}
      (using that $(G/H)^K \simeq \Hom(G/K, G/H)$).
      % Lastly, if we consider (any element in the orbit of) the new object $1\in \C(\P)^H$,
      % there is an associated object $0 \in \C(\O')^H$ such that the essentially surjectivity diagram
      % factors through the pushout diagram for $\psi$:
      % \begin{equation}
      %       \begin{tikzcd}
      %             G/H \cdot \1 \arrow[r,"0"] \arrow[dr, "G/H \cdot i_0"']
      %             &
      %             G/H \cdot \J_{\set{0}} \arrow[r] \arrow[d]
      %             &
      %             \O' \arrow[d, "\psi'"]
      %             \\
      %             &
      %             G/H \cdot \J \arrow[r]
      %             &
      %             \P \arrow[d, equal]
      %             \\
      %             G/H \cdot \1 \arrow[ur, "G/H \cdot i_1"] \arrow[rr, "1"]
      %             &&
      %             \P.
      %       \end{tikzcd}
      % \end{equation}
      Hence $\psi'$ is also essentially surjective.

      Now, consider the top pushout. \eqref{COLOR_SQ_EQ} again implies that this pushout is created in $\Op^{G, \mathfrak C(X)}(\V)$.
      In particular, this implies $\phi'$ is bijective on objects, and hence essentially surjective.
      Further, since $1_\V$ is cofibrant in $\V$, \cite[Thm. 1.15]{BM13} implies that $\J_{\set 0}$ is cofibrant in $\Op^{\**}(\V)$,
      and since $\1$ is the initial object here, $\phi$ is a trivial cofibration here.
      Thus $a_! (G/H \cdot \phi)$ is a trivial $\F$-cofibration in $\Op^{G, \mathfrak C(X)}(\V)$ by Corollary \ref{COLOR_CHANGE_Q_COR},
      {(as $G/H \cdot \phi$ is one in $\Op^{G, G/H}(\V)$,
        since $\O \to \P$ a trivial $\F$-fibration in $\Op^{G, \mathfrak C}(\V)$
        implies $j^{\**}\O^H \to j^{\**}\P^H$ is one in $\Cat^{\mathfrak C^H}e(\V)$)}.
      Hence $\phi'$ is a trivial $\F$-cofibration in $\mathsf{Op}^{G,\C(X)}(\V)$,
      and thus a local genuine trivial cofibration by Corollary \ref{LGC_COR}.
      
      Since both $\phi'$ and $\psi'$ are essentially surjective and local genuine trivial cofibrations,
      the result is proved.
\end{proof}





\subsection{2-out-of-3}

For the proof of 2-out-of-3 in Proposition \ref{2OUTOF3_PROP}, we will need to convert between
``essential surjectivity'' and ``local equivalence'' information. 
To begin, we recall some equivalence relations on objects in a $\V$-category \cite{Cav14, BM13}:
\begin{definition}
      Given $\mathcal{C}$ in  $\Cat(\V)$ and $a,b\in\mathrm{Ob}(\mathcal C)$, we say $a$ and $b$ are
      \begin{itemize}
      \item {\em equivalent} if there exists a map $\gamma: \J \to \mathcal C$ such that
            $\gamma i_0 = a$, $\gamma i_1 = b$
            for some $\V$-interval $\J$;
      \item {\em virtually equivalent} if $a$ and $b$ are equivalent in some fibrant replacement
            $\mathcal C_f$ of $\mathcal C$ in $\Cat^{\mathrm{Ob}(\mathcal C)}(\V)$;
      \item {\em homotopy equivalent} if $a$ and $b$ are isomorphic in the unenriched category $\pi_0 \mathcal C_f$
            for some fibrant replacement $\mathcal C_f$ of $\mathcal C$;
            equivalently, if there exist maps
            $\alpha: 1_\V \to \mathcal C_f(a,b)$ and $\beta: 1_\V\to \mathcal C_f(b,a)$ such that
            $\beta\alpha$ and $\alpha\beta$ are homotopic (cf. Definition \ref{HTPY_DEFN})
            to the identity arrows
            $1_V\to \mathcal C_f(a,a)$ and $1_V \to \mathcal C_f(b,b)$, respectively.
      \end{itemize}
\end{definition}

Equivariantly, we have the following:
\begin{definition}
      Fix $\mathcal{C}\in \Cat^G(\V)$, $a,b\in \mathrm{Ob}(\mathcal{C})$, and $H \leq G$.
      We say $a$ and $b$ are 
      \textit{(virtually, homotopy) $H$-equivalent}
      if they are (resp. virtually, homotopy) equivalent in $\mathcal{C}^H$;
      % Two options for virtually equivalent:
      % (i) they are virtually equivalent in $\mathcal C^H$
      % (ii) they are $H$-equivalent in some fibrant replacement $\mathcal C_f$ of $\mathcal C$
      % in $\Cat^{G, \mathrm{Ob}(\mathcal C)}(\V)$.
\end{definition}



\begin{remark}
      \label{HK_EQUIV_REM}
      We note that if $K \leq H \leq G$, then (virtually, homotopy) $H$-equivalent implies (virtually, homotopy) $K$-equivalent
      as we have inclusions of categories
      $j^{\**}(\O^H) \to j^{\**}(\O^K)$
      (resp. by functoriality of fibrant replacement,
      $j^{\**}(\O^H)_f \to j^{\**}(\O^K)_f$,
      $\pi_0(j^{\**}(\O^H)_f) \to \pi_0(j^{\**}(\O^K)_f)$).
      % Further, if $\V$ has a fibrant replacement functor that commutes with taking fixed points for any subgroup of $G$,     
      % % (in which case the two definitions of virtually $H$-equivalent coincide)
      % then virtually (resp. homotopy) $H$-equivalent implies virtually (homotopy) $K$-equivalent,
      % as we would have an inclusion of categories
      % Moreover, this would imply that $a$ and $b$ are virtually $H$-equivalent iff
      % they are $H$-equivalent in some fibrant replacement $\mathcal C_f$ of $\mathcal C$ in $\Op^{G, \mathrm{Ob}(\mathcal C)}(\V)$.
\end{remark}

\begin{remark}
      \label{HVIRT_REM}
      If $a,b\in \C$ are virtually $H$-equivalent (which could more accurately be called ``$H$-virtually equivalent''),
      they are in fact \textit{virtually} $H$-equivalent:
      there exists a lift
      \begin{equation}
            \label{FIBFIX_LIFT_EQ}
            \begin{tikzcd}
                  \mathcal C^H \arrow[d, tail, "\sim"'] \arrow[r, "\sim"]
                  &
                  (\mathcal C_f)^H
                  \\
                  (\mathcal C^H)_f \arrow[ur, dashed, "\sim"]
            \end{tikzcd}
      \end{equation}
      and thus any equivalence in $(\mathcal C^H)_f$ induces an equivalence in $(\mathcal C_f)^H$. 
\end{remark}

\begin{definition}
      For a $G$-operad $\O\in \mathsf{Op}^G(\V)$ and $a,b\in \C(\O)$, we say $a$ and $b$ are
      {\em (virtually, homotopy) $H$-equivalent}
      if they are so in $j^*\O$. 
\end{definition}

\begin{remark}
      \label{ESS_SUR_REM}
      Unpacking definitions, we see
      $F: \O \to \P$ in $\Op^G(\V)$ is essentially surjective iff
      for any $b \in \P^H$ there exists $a \in \O^H$ such that $F(a)$ and $b$ are $H$-equivalent.
\end{remark}

\begin{notation}
      As fixed points $(-)^\Gamma$ and fibrant replacement $(-)_f$ need not commute, we will write
      \begin{equation}
            \O_f(\ksi)^\Gamma = (\O_f(\ksi))^\Gamma,
            \qquad
            \O^\Gamma(\ksi)_f = (\O(\ksi)^{\Gamma})_f.
      \end{equation}
\end{notation}



The following two lemmas follow directly from the proofs of their non-equivariant counterparts:
\begin{lemma}
      [{cf. \cite[4.10]{Cav14}}]
      \label{CAV_4.10_LEM}
      $H$-equivalence and virtual $H$-equivalence define equivalence relations on $\C(\O)^H$.
\end{lemma}
% \begin{proof}
%       {\color{OliveGreen}
%         Follows exactly as in \textit{loc cite}; either version of virtual $H$-equivalence works.

%         $ $
        
%         Indeed,
%         symmetry follows from the transposition isomorphism $\tau^{\**}\J \to \J$.
        
%         Reflexivity follows from the composition $\I_c \to \I \to \mathcal C^H$,
%         \todo{either version (i) or (ii) works here}
%         $\mathcal C_f^H$
%         of cofibrant replacement followed by the map realizing the identity map on $a$.
        
%         Transitivity follows from the amalgamation of interval objects \cite[Cor. 1.16]{BM13}
%         by the following two claims.
%         First, for any maps of $G$-sets $f: A \to \mathfrak C(\O)$,
%         we have a canonical ``identity'' map $f^{\**}\O \to \O$.
%         Second, a chase through the adjunctions yields that
%         any pair of maps $h: \J \to \mathcal C$ and $h': \J' \to \mathcal C$
%         induces a map $h \** h' : \J \** \J' \to \mathcal C$ such that      
%         $(h \** h') i_0 = h i_0$ and $(h \** h') i_1 = h' i_1$.
%       }
%   \end{proof}


\begin{lemma}
      [{cf. \cite[4.12]{Cav14}, \cite[2.10]{BM13}}]
      \label{RIGHTPROPER_LEM}
      If $\V$ is right proper, then two colors are virtually $H$-equivalent iff they are $H$-equivalent. 
\end{lemma}
% \begin{proof}
%       {\color{OliveGreen}
%         Need: virtual $H$-equivalent (i). Then it is an immediate consequence of \textit{loc cite}.
%       }
% \end{proof}

We elaborate on the following two lemmas from \cite{BM13}.

\begin{lemma}
      [{cf. \cite[4.13]{Cav14}, \cite[2.11]{BM13}}]
      \label{VIR_HTPY_LEM}
      Virtually $H$-equivalent colors are homotopy $H$-equivalent. 
\end{lemma}
\begin{proof}
      {\color{OliveGreen} We reframe and complete \cite[2.10]{BM13} (they were missing \eqref{J11_CYL_EQ}).}
      Let $\mathcal C_f$ denote $j^{\**}(\O^H)_f$.
      Suppose $H: \J \to \mathcal C_f(x,y)$ realizes a virtual equivalence between $x$ and $y$.
      Let's factor the equipped map $\J \xrightarrow{\sim} \I_f$ (a weak equivalence in $\Cat^{\set{0,1}}(\V)$)
      as a trivial cofibration and trivial fibration
      $\J \xrightarrow{\sim} \J' \xrightarrow{\sim} \I_f$,
      and then $H$ extends to $\J'$ since $\mathcal C_f$ is fibrant.

      Thus we have lifts $f_{01}$ and $f_{10}$ in the diagrams below,
      where all designations on arrows are as maps in $\Cat^{\set{0,1}}(\V)$.
      \begin{equation}
            \begin{tikzcd}
                  &
                  \J \arrow[r, "H"] \arrow[d, tail, "\sim"]
                  &
                  \mathcal C_f
                  &&
                  \J \arrow[r, "H"] \arrow[d, tail, "\sim"]
                  &
                  \mathcal C_f
                  \\
                  \1 \amalg \1 \arrow[r] \arrow[d, tail]
                  &
                  \J' \arrow[ur, dashed, "H'"'] \arrow[d, two heads, "\sim"]
                  &&
                  \1 \amalg \1 \arrow[r] \arrow[d, tail]
                  &
                  \J' \arrow[ur, dashed, "H'"'] \arrow[d, two heads, "\sim"]
                  \\
                  \mathbb A \arrow[r] \arrow[ur, dashed, "f_{01}"]
                  &
                  \I_f
                  &&
                  \mathbb A^{op} \arrow[r] \arrow[ur, dashed, "f_{10}"]
                  &
                  \I_f
            \end{tikzcd}
      \end{equation}
      This provides maps
      $\alpha = H'_{01} f_{01}: 1_\V \to \mathcal C_f(x,y)$
      and
      $\beta = H'_{10} f_{10}: 1_\V \to \mathcal C_f(y,x)$.
      By construction, the composite $\alpha\beta$ (resp. $\beta\alpha$) factors through $\J'(1,1)$ (resp. $\J'(0,0)$)
      by a map we denote $f_1$ (resp. $f_0$).
      \begin{equation}
            \begin{tikzcd}
                  1_\V \arrow[r, "\simeq"] \arrow[drr, "f_1"']
                  &
                  1_\V \otimes 1_\V \arrow[r, "f_{01} \otimes f_{10}"]
                  &
                  \J'(0,1) \otimes \J'(1,0) \arrow[d, "\circ"] \arrow[r, "H'_{01} \otimes H'_{10}"]
                  &
                  \mathcal C_f(x,y) \otimes \mathcal C_f(y,x) \arrow[d, "\circ"]
                  \\
                  &&
                  \J'(1,1) \arrow[r, "H'_{11}"]
                  &
                  \mathcal C_f(y,y)
            \end{tikzcd}
      \end{equation}
      
      Now, if $\mathbb C$ is any cylinder in $\V$, we have a lift in the following diagram, for $i \in \set{0,1}$.
      \begin{equation}
            \label{J11_CYL_EQ}
            \begin{tikzcd}
                  1_\V \amalg 1_\V \arrow[rr, "{(id, f_i)}"] \arrow[d, tail]
                  &&
                  \J'(i,i) \arrow[d, "\sim", two heads]
                  \\
                  \mathbb C \arrow[r] \arrow[urr, dashed]
                  &
                  1_\V \arrow[r]
                  &
                  (1_\V)_f
              \end{tikzcd}
        \end{equation}
        Thus $f_1$ (resp. $f_0$) factors through a cylinder,
        implying $\alpha\beta$ (resp. $\beta\alpha)$ is homotopic to $id_y$ (resp. $id_x$),
        and hence that the objects $x,y\in \mathcal C_f$ are homotopy equivalent.
  \end{proof}
  
\begin{lemma}
      [{cf. \cite[4.11]{Cav14}, \cite[2.9]{BM13}}]
      \label{REF_VIRT_LEM}
      Suppose the indexing family $\F$ has units.
      Then local weak $\F$-equivalences reflect $H$-equivalences.
      Explicitly,
      fixing $H \leq G$, 
      $a_0,a_1 \in \C(\O)^H$, and $F: \O \to \P$ in $\mathsf{Op}^G(\V)$ a local weak $\F$-equivalence,
      if $F(a_0)$ and $F(a_1)$ are virtually $H$-equivalent then so are $a_0$ and $a_1$.
\end{lemma}
\begin{proof}
      % {\color{OliveGreen}
      %   Need: virtual $H$-equivalent (i).
      % }
      Since $\F$ has units, $\F_1$ contains all $H \leq G \times \Sigma_1$ (cf. \cite[Remark 4.50]{BP17}), and thus
      $j^{\**}F^H: j^{\**}\O^H \to j^{\**}\P^H$ is a local weak equivalence in $\Cat(\V)$.
      %
      As in \cite{BM13}, we may build a fibrant replacement of $j^{\**}F^H$ which is a local trivial fibration in $\Cat(\V)$.
      \begin{equation}
            \begin{tikzcd}
                 j^{\**} \O^H \arrow[r, "\sim_l"] \arrow[d, dashed, "\sim"']
                  &
                  F^{\**} (j^{\**}\P^H) \arrow[r, "\simeq_l", two heads] \arrow[d, "\sim"']
                  &
                  j^{\**} \P^H \arrow[d, "\sim"]
                  \\
                  j^{\**} (\O^H)_f \arrow[r, dashed, two heads]
                  &
                  F^{\**} (j^{\**}(\P^H)_f) \arrow[r, "\simeq_l", two heads]
                  &
                  j^{\**}(\P^H)_f
            \end{tikzcd}
      \end{equation}
      (where all designations $(-)_l$ on arrows denote ``local'').
      Thus we have a lift on the left in \eqref{REF_VIRT_EQ},
      where the bottom arrow realizes the virtual $H$-equivalence between $F(a_0)$ and $F(a_1)$,
      as such a lift in $\Cat(\V)$ is equivalent to a lift in $\Cat^{\set{0,1}}(\V)$
      after pulling the right vertical arrow back along the inclusion
      $a: \set{0,1} \to \set{a_0,a_1} \into \mathfrak C(\O)^H$ as on the right (cf. \eqref{COLOR_SQ_EQ}),
      and this lift exists by locality.
      \begin{equation}
            \label{REF_VIRT_EQ}
            \begin{tikzcd}
                  \1 \amalg \1 \arrow[r, "{(a_0,a_1)}"] \arrow[d, tail, "{(i_0, i_1)}"']
                  &
                  j^{\**}(\O^H)_f \arrow[d, two heads, "\sim_l"]
                  &&
                  \1 \amalg \1 \arrow[r, "{(a_0,a_1)}"] \arrow[d, tail, "{(i_0, i_1)}"']
                  &
                  a^{\**}j^{\**}(\O^H)_f \arrow[d, two heads, "\sim"]
                  \\
                  \J \arrow[r] \arrow[ur, dashed]
                  &
                  j^{\**}(\P^H)_f
                  &&
                  \J \arrow[r] \arrow[ur, dashed]
                  &
                  a^{\**} F^{\**} j^{\**}(\P^H)_f.
            \end{tikzcd}
      \end{equation}
\end{proof}

We can now prove the following two similar lemmas, generalizing \cite[Prop. 4.14]{Cav14} and \cite[Prop. 2.12]{BM13},
which show that in many cases, a homotopy equivalence of colors induces a homotopy equivalence of hom-objects.

\begin{lemma}
      \label{CAV_4.14_PROP1}
      % Suppose $\V$ has \textit{Cartesian fixed points} \cite[Def 6.27]{BP17}, and
      % a fibrant replacement functor that commutes with taking fixed points for all subgroups of $G$.
      Let $\O \in \Op^G(\V)$,
      $c,d \in \mathfrak C(\O)$, 
      $H = \Stab(c)$, and suppose $c$ and $d$ are homotopy $H$-equivalent.
      %
      Then, for any $\mathfrak C(\O)$-profile $\ksi = (c_1,\dots, c_n;c)$,
      there exists a zig-zag of weak equivalences in $\V^{\Stab(\ksi)}_{\mathrm{Gr}_\ksi}$
      between $\O(\ksi)$ and $\O(\ksi_c^d)$,
      where $\ksi_c^d = (c_1,\dots, c_n;d)$.

      Finally, any functor $F: \O \to \P$ induces a functorial zig-zag of weak equivalences in $\V^{\Stab(\ksi)}_{\mathrm{Gr}_\ksi}$
      between $\P(F(\ksi))$ and $\P(\F(\ksi_c^d))$.
\end{lemma}
\begin{proof}
      For $\P = \O$, $\O_f$, or $\O_{f c} = (\O_f)_c$,
      and $(a;b) = (c;c)$, $(c;d)$, $(d;c)$, or $(d;d)$,
      we see that $\P(a;b) \in \V$ has a $\Stab(\ksi)$-action
      via the projection $\Stab(\ksi) \to H$.
      Further, naturality of composition implies that the maps
      \begin{equation}
            \alpha_{\**}: \P(\ksi) \to \P(\ksi_c^d),
            \qquad
            \beta_{\**}: \P(\ksi_c^d) \to \P(\ksi)
      \end{equation}
      and their composites are $\Stab(\ksi)$-equivariant.
      Thus we have natural composition maps
      \begin{equation}
            \P(\ksi)^\Gamma \otimes \P(c;d)^{H_\Gamma} \longrightarrow
            (\P(\ksi) \otimes \P(c;d))^\Gamma \longrightarrow
            \P(\ksi_c^d)^\Gamma
      \end{equation}
      for any graph subgroup $\Gamma \leq \Stab(\ksi) \leq G \times \Sigma_n$,
      where $H_\Gamma$ is the projection of $\Gamma$ onto $G$,
      and similarly when replacing $(c;d)$ with $(d;c)$, $(c;c)$, $(d;d)$,
      and $\ksi$ by $\theta$ when required.
      
      By assumption we have maps
      $\alpha: 1_\V \to \O^H(c;d)_f$ and $\beta: 1_\V \to \O^H(d;c)_f$
      representing the homotopy $H$-equivalence,
      with associated homotopies $H_{\beta\alpha,1}$ and $H_{\alpha\beta,1}$.
      Additionally, for any graph subgroup $\Gamma$, we have lifts of these homotopies, e.g.
      \begin{equation}
            \label{OFC_HOM_LIFT}
            \begin{tikzcd}
                  &&& \O_{f c}(c;c)^{H_\Gamma} \arrow[d, two heads, "\sim"]
                  \\
                  \mathbb C \arrow[r, "H_{\beta\alpha,1}"'] \arrow[urrr, dashed, shift left, "H_{\beta\alpha,1}"]
                  &
                  \O^H(c;c)_f \arrow[r]
                  &
                  \O^{H_\Gamma}(c;c)_f \arrow[r]
                  &
                  \O_f(c;c)^{H_\Gamma}
            \end{tikzcd}
      \end{equation}
      (using Remark \ref{HK_EQUIV_REM} and \eqref{FIBFIX_LIFT_EQ}), compatible with similar lifts of $\alpha$ and $\beta$.
      %
      Thus we have an induced homotopy $\beta\alpha \sim id_{\O_{f c}(\ksi)^{\Gamma}}$ in $\V$
      \begin{equation}
            \begin{tikzcd}
                  \O_{f c}(\ksi)^{\Gamma} \otimes (1_\V \amalg 1_\V) \arrow[d, tail] \arrow[r, "{((\beta\alpha)_{\**}, id)}"]
                  &
                  \O_{f c}(\ksi)^\Gamma
                  % &
                  % 1_\V \amalg 1_\V \arrow[d, tail] \arrow[rr, "{((\alpha\beta)_{\**}, id)}"]
                  % &&
                  % \V((\O(\ksi_c^d)_f)^\Gamma, (\O(\ksi_c^d)_f)^\Gamma)
                  \\                  
                  \O_{f c}(\ksi)^\Gamma \otimes \mathbb C \arrow[r, "H_{\beta\alpha,1}"]
                  &
                  \O_{f c}(\ksi)^\Gamma \otimes \O_{f c}(c;c)^{H_\Gamma} \arrow[u, "\circ"]
                  % &
                  % \mathbb C \arrow[r, "H_{\alpha\beta,1}"']
                  % &
                  % (\O(d;d)^{H_\Gamma})_f \arrow[r]
                  % &
                  % (\O(d;d)_f)^{H_\Gamma} \arrow[u, "{(-)_{\**}}"],
            \end{tikzcd}
      \end{equation}
      and similarly $\alpha\beta \sim id_{\O_{f c}(\ksi_c^d)^\Gamma}$.
      Hence, as both $\O_{f c}(\ksi)^\Gamma$ and $\O_{f c}(\ksi_c^d)^\Gamma$ are bifibrant by Remark \ref{LEVEL_COF_REM},
      $\alpha_{\**}: \O_{f c}(\ksi)^\Gamma \leftrightarrow \O_{f c}(\ksi_c^d)^\Gamma : \beta_{\**}$ are inverse isomorphisms in $\Ho(\V)$ for all $\Gamma$
      by Corollary \ref{HTPIC_ISO_COR}.
      %
      Thus, in particular, $\alpha_{\**}: \O_{f c}(\ksi) \to \O_{f c}(\ksi_c^d)$ is a weak equivalence in $\V^{\Stab(\ksi)}_{\mathrm{Gr}_\ksi}$.
      Composing with the functorial (co)fibrant replacement weak equivalences yields our zig-zag of $\mathrm{Gr}_\ksi$-equivalences
      \begin{equation}
            \O(\ksi) \xrightarrow{\simeq} \O_f(\ksi) \xleftarrow{\sim} \O_{f c}(\ksi)
            \xrightarrow{\simeq}
            \O_{f c}(\ksi_c^d) \xrightarrow{\sim} \O_f(\ksi_c^d) \xleftarrow{\simeq} \O(\ksi_c^d).
      \end{equation}
      
      Finally, any $F: \O \to \P$ induces a map $F_{f c}: \O_{f c} \to \P_{f c}$, and
      the images of $\alpha_{\**}$ and $\beta_{\**}$ yield
      homotopy equivalences $\P_{f c}(F(\ksi))^\Gamma \to \P_{f c}(F(\ksi_c^d))^\Gamma$ for any graph subgroup $\Gamma \leq \Stab(\ksi)$,
      as desired.
\end{proof}



\begin{lemma}
      % [{c.f. \cite[4.14]{Cav14}}]
      \label{CAV_4.14_PROP2}
      Suppose $\V$ has cofibrant symmetric pushout powers.
      Let $\O \in \Op^G(\V)$, and suppose
      $c_i$ and $d_j$ are homotopy $H$-equivalent with $H \leq \Stab(c_i) \leq \Stab(d_j)$;
      without loss of generality we may assume $i = j = 1$.

      Then, for any $\mathfrak C(\O)$-profile $\ksi = (c_1,\dots,c_n;c)$,
      there exists a zig-zag of weak equivalences between
      $\O(\ksi)$ and $\O(\theta)$
      in $\V^{\Stab(\xi)}_{\mathrm{Gr}_\xi}$,
      where
      $\theta = (d_1,\ldots, d_n; c)$, with the colors $d_i$ defined as follows:
            
      Let $\lambda \subseteq \underline{n} = \set{1,2,\ldots, n}$ denote
      the set of all $i$ such that $c_i = k_i \cdot c_1$ for some $k_i\in G$,
      and fix choices of such $k_i$.
      If $i \notin \lambda$, let $k_i$ denote the identity element of $G$.
      % \todo[inline]{if $H = \Stab{c_1}$, then $k_i$ are well-defined in $k_i H$.}
      Then, for all $i \in \underline{n}$, we define
      \begin{equation}
            \label{DCOLORS_EQ}
            d_i =
            \begin{cases}
                  k_i \cdot d_1 \qquad \qquad & i \in \lambda
                  \\
                  c_i & \mbox{otherwise.}
            \end{cases}
      \end{equation}
      
      Finally, any functor $F: \O \to \P$ induces a functorial zig-zag of weak equivalences in $\V^{\Stab(\ksi)}_{\mathrm{Gr}_\ksi}$
      between $\P(F(\ksi))$ and $\P(F(\theta))$.
\end{lemma}
\begin{proof}
      Firstly, we claim that $\Stab_{G \times \Sigma_n}(\ksi) \subseteq \Stab_{G \times \Sigma_n}(\theta)$.
      To that end, suppose $(k,\pi)\in \Stab_{G\times \Sigma_n}(\ksi)$, so that $k \cdot c_{\pi^{-1}(i)} = c_i$ for all $i$.
      Thus $\pi$ must act on $\lambda$ and $\underline{n} \setminus \lambda$ independently,
      so for $i \in \lambda$ we have % $k \cdot c_{\pi^{-1}(i)} = c_i$, or
      $k k_{\pi^{-1}(i)} \cdot c_1 = k_i \cdot c_1$, and so
      $k_i^{-1} k k_{\pi^{-1}(i)} =:h_i \in \Stab(c_1) \leq \Stab(d_1)$. Hence 
      \begin{equation}
            \label{STAB_KT_EQ}
            k \cdot d_{\pi^{-1}(i)} = k k_{\pi^{-1}(i)} \cdot d_1 = k_i h_r \cdot d_1 = k_i \cdot d_1 = d_i,
      \end{equation}
      as desired.
      On the other hand, if $i \not \in \lambda$, then
      $k \cdot d_{\pi^{-1}(i)} = k \cdot c_{\pi^{-1}(i)} = c_i = d_i$.
      
      Now, by assumption, we have maps
      $\alpha: 1_\V \to \O^H(c_1;d_1)_f$ and $\beta: 1_\V \to \O^H(d_1;c_1)_f$
      representing the homotopy $H$-equivalence,
      with associated homotopies $H_{\beta\alpha,1}$ and $H_{\alpha\beta,1}$.
      %
      Chose a set of maps (cf. \eqref{FIBFIX_LIFT_EQ})
      \begin{equation}
            \O^H(a_1;b_1)_f \to \O_f(a_1;b_1)^H,
            \qquad \qquad
            (a_1;b_1) \in  \set{(d_1;c_1), (c_1;d_1), (c_1;c_1), (d_1;d_1)},
      \end{equation}
      and use these as in \eqref{OFC_HOM_LIFT} to compatibly lift
      $\alpha_1$, $\beta_1$, their composites, and their associated homotopies to $\O_{f c}(-)^H$.
      % These provide compatable maps for all $i \in \lambda$
      % \begin{equation}
      %       \begin{tikzcd}
      %             \O^{H_i}(a_i;b_i)_f
      %             &
      %             \O^H(a_1;b_1)_f \arrow[l, "\simeq", "k_i"'] \arrow[r]
      %             &
      %             \O_f(a_1;b_1)^H \arrow[r, "\simeq"', "k_i"]
      %             &
      %             \O_f(a_i;b_i)^{H_i}.
      %       \end{tikzcd}
      % \end{equation}   
      % 
      For each $i \in \underline{n}$, define
      \begin{equation}
            \label{WEAKEQCOLORS_EQ}
            H_i =
            \begin{cases}
                  k_i H k_i^{-1} \qquad & i \in \lambda
                  \\
                  H & \mbox{else,}
            \end{cases}
            \qquad
            \qquad 
            \alpha_i =
            \begin{cases}
                  1_\V \xrightarrow{\alpha} O_{f c}(c_1;d_1)^H \xrightarrow{k_i} \O_{f c}(c_i;d_i)^{H_i} \qquad & i \in \lambda
                  \\
                  1_\V \xrightarrow{id} \O_{f c}(c_i;d_i)^H & \mbox{else},
            \end{cases}
      \end{equation}
      and $\beta_i$ similarly.
      Note that all the $H_i$, $\alpha_i$, and $\beta_i$ are independent of the choice of $k_i\in k_i H$.
      Further, 
      the pair $(\alpha_i,\beta_i)$ realizes a homotopy $H_i$-equivalence between $c_i$ and $d_i$,
      as naturality of composition implies
      $k_i (\beta\alpha) = \beta_i\alpha_i$.
      
      Assembling the above data, 
      we see that for $(a_i;b_i) = (d_i;c_i)$, $(c_i;d_i)$, $(c_i;c_i)$, or $(d_i;d_i)$,
      the $\lambda$-indexed diagrams $i \mapsto \O_{f c}(a_i;b_i)^{H_i}$
      have an action by $\Stab(\ksi)$,
      and hence so do their tensor products $\otimes_\lambda \O_{f c}(a_i;b_i)^{H_i}$.
      Thus, for any graph subgroup $\Gamma \leq \Stab(\ksi) \leq G \times \Sigma_n$,
      we have composition maps
      \begin{equation}
            \O_{f c}(\ksi)^{\Gamma} \otimes \left(
                  \bigotimes\limits_\lambda \O_{f c}(c_i;c_i)
            \right)^{\Gamma}
            \longrightarrow
            \left(
                  \O_{f c}(\ksi) \otimes \bigotimes\limits_\lambda \O_{f c}(c_i;c_i)
            \right)^\Gamma
            \longrightarrow
            \O_{f c}(\ksi)^\Gamma
      \end{equation}
      and similarly for any $(c_i;d_i)$, $(d_i;c_i)$, $(d_i;d_i)$,
      replacing $\ksi$ with $\theta$ when required.
      
      Now, consider the following maps:
      \begin{equation}
            \begin{tikzcd}[row sep = tiny]
                  \bigotimes_\lambda \alpha_i: 1_\V \simeq 1_\V^{\otimes n} \longrightarrow \bigotimes_\lambda \O_{f c}(c_i;d_i)
                  \\ % --------------------------------------------------
                  \bigotimes_\lambda \beta_i: 1_\V \simeq 1_\V^{\otimes n} \longrightarrow \bigotimes_\lambda \O_{f c}(d_i;c_i)
                  \\ % --------------------------------------------------
                  H_c: \bigotimes\limits_\lambda \mathbb C \xrightarrow{k_i H_{\beta\alpha,1}}
                  \bigotimes\limits_\lambda \O_{f c}(c_i;c_i)^{H_i} \longrightarrow
                  \bigotimes\limits_\lambda \O_{f c}(c_i;c_i)
                  \\ % --------------------------------------------------
                  H_d: \bigotimes\limits_\lambda \mathbb C \xrightarrow{k_i H_{\alpha\beta,1}}
                  \bigotimes\limits_\lambda \O_{f c}(d_i;d_i)^{H_i} \longrightarrow
                  \bigotimes\limits_\lambda \O_{f c}(d_i;d_i)
                  \\ % --------------------------------------------------
                  (\alpha_i)_{\**} : 
                  \O_{f c}(\theta)
                  \cong
                  \O_{f c}(\theta) \otimes 1_\V^{\otimes n} \xrightarrow{1 \otimes\ \bigotimes_i \alpha_i}
                  \O_{f c}(\theta) \otimes \bigotimes_i \O_{f c}(c_i;d_i) \xrightarrow{\circ}
                  \O_{f c}(\ksi)
                  \\ % --------------------------------------------------
                  (\beta_i)_{\**} :
                  \O_{f c}(\ksi)
                  \cong
                  \O_{f c}(\ksi) \otimes 1_\V^{\otimes n} \xrightarrow{1 \otimes\ \bigotimes_i \beta_i}
                  \O_{f c}(\ksi) \otimes \bigotimes_i \O_{f c}(d_i;c_i) \xrightarrow{\circ}
                  \O_{f c}(\theta)
                  \\ % --------------------------------------------------
                  (\beta_i)_{\**}(\alpha_i)_{\**}
                  \\ % --------------------------------------------------
                  (\alpha_i)_{\**}(\beta_i)_{\**}.
            \end{tikzcd}
      \end{equation}
      We claim these are all $\Stab(\ksi)$-equivariant
      (via the natural actions, and acting on $\otimes_\lambda \mathbb C$ via the projection $\Stab(\ksi) \to \Sigma_{|\ksi|}$).
      We will just show this holds for the first map, as the rest are analogous or consequences:
      for any $(k,\pi) \in \Stab(\ksi)$,
      \begin{equation}
            \label{AB_STAR_EQ}
            (k, \pi) . \otimes_i(\alpha_i)
            =
            \otimes_i (k \alpha_{\pi^{-1}(i)})
            =
            \otimes_i (k k_{\pi^{-1}i} \alpha_1)
            =
            \otimes_i (k k_i h_i \alpha_1)
            =
            \otimes_i (k_i \alpha_1)
            =
            \otimes_i (\alpha_i);
      \end{equation}
      % Thirdly, since $\V% $ has Cartesian fixed-points, we see that for any graph subgroup $\Gamma \leq \Stab(\ksi)$
      % we have a string of isomorphisms
      % \begin{equation}
      %       \label{CART_FIXED_AB_EQ}
      %       \begin{tikzcd}
      %             \O(\ksi)^\Gamma \otimes \O(a_1;b_1)^{H_{\Gamma(1)}}
      %             \arrow[r, "\tilde\Delta", "\simeq"']
      %             &
      %             \O(\ksi)^\Gamma \otimes \left(
      %                   \bigotimes\limits_\lambda \O(a_i; b_i)
      %             \right)^\Gamma
      %             \arrow[r, "\simeq"]
      %             &
      %             \left(
      %                   \O(\ksi) \otimes \bigotimes\limits_\lambda \O(a_i;b_i)
      %             \right)^\Gamma,
      %       \end{tikzcd}
      % \end{equation}
      % where
      % $(a_i;b_i) \in \set{(c_i;d_i), (c_i;c_i), (d_i;d_i)}$,
      % $H_{\Gamma(1)}$ is the projection onto $G$ of those $\gamma \in \Gamma$ which fix $1 \in \underline{n}$,
      % and $\tilde\Delta$ is the ``twisted diagonal''
      % \begin{equation}
      %       \O(a_1;b_1) \xrightarrow{\Delta} \bigotimes_\lambda \O(a_1;b_1) \xrightarrow{\otimes k_i} \bigotimes_\lambda \O(a_i;b_i).
      % \end{equation}
      %  Hence these maps descend to $\Gamma$ fixed points for any graph subgroup $\Gamma \leq \Stab(\ksi)$.
      Thus, 
      Remarks \ref{HK_EQUIV_REM}, \ref{CYL_REM}, and \ref{ASSEM_HOM_REM} with Lemma \ref{ASSEM_HOM_LEM}
      yield an induced homotopy
      \begin{equation}
            (\beta_i)_{\**}(\alpha_i)_{\**} \sim id_{\O_{f c}(\ksi)^\Gamma}
      \end{equation}
      in $\V$ for all graph subgroups $\Gamma \leq \Stab(\ksi)$:
      \begin{equation}
            % \begin{tikzcd}
            %       1_\V \amalg 1_\V \arrow[rr, "{((\beta_i)_{\**}(\alpha_i)_{\**}, id)}"] \arrow[d]
            %       &&
            %       \V(\O(\ksi)^\Gamma, \O(\ksi)^\Gamma)
            %       \\
            %       \mathbb C \arrow[r, "H_{\beta\alpha,1}"]
            %       &
            %       \O(c_1;c_1)^{H_{\Gamma(1)}} \arrow[r, "\tilde\Delta"]
            %       &
            %       \left(
            %             \bigotimes_\lambda \O(c_i;c_i)
            %       \right)^\Gamma
            %       \arrow[u, "{(-)_{\**}}"]
            % \end{tikzcd}
            \begin{tikzcd}[column sep = small]
                  \O_{f c}(\ksi)^\Gamma \otimes (1_\V \amalg 1_\V) \arrow[d, tail] \arrow[rr, "{((\beta_i)_{\**}(\alpha_i)_{\**}, id)}"]
                  &&
                  % &
                  \O_{f c}(\ksi)^\Gamma
                  \\                  
                  % \mathbb C \arrow[r, "\Delta"]
                  % &
                  \O_{f c}(\ksi)^\Gamma \otimes \left(\bigotimes\limits_\lambda \mathbb C\right)^{\Stab(\ksi)}
                  \arrow[r, "H"]
                  &
                  \O_{f c}(\ksi)^\Gamma \otimes \left(\bigotimes\limits_\lambda \O_{f c}(c_i;c_i)\right)^{\Stab(\ksi)} \arrow[r]
                  &
                  \O_{f c}(\ksi)^\Gamma \otimes \left(\bigotimes\limits_\lambda \O_{f c}(c_i;c_i)\right)^{\Gamma} \arrow[u, "\circ"']                  
                  % &
                  % \mathbb C \arrow[r, "H_{\alpha\beta,1}"']
                  % &
                  % (\O(d;d)^{H_\Gamma})_f \arrow[r]
                  % &
                  % (\O(d;d)_f)^{H_\Gamma} \arrow[u, "{(-)_{\**}}"],
            \end{tikzcd}
      \end{equation}
      and similarly $(\alpha_i)_{\**}(\beta_i)_{\**} \sim id_{\O_{f c}(\theta)^\Gamma}$.
      Hence
      $(\alpha_i)_{\**}: \O_{f c}(\ksi)^\Gamma \leftrightarrow \O_{f c}(\theta)^\Gamma: (\beta_i)_{\**}$
      are inverse isomorphisms in $\Ho(V)$ for all $\Gamma$, again by Corollar \ref{HTPIC_ISO_COR}.

      The rest of the proof follows as for Proposition \ref{CAV_4.14_PROP1}.
      % Thus, in particular, $(\alpha_i)_{\**}$ is a weak equiavlence in $\V^{\Stab(\ksi)}_{\mathrm{Gr}_\ksi}$.     
      % Composing with the functorial fibrant replacement weak equivalences yields our zig-zag
      % \begin{equation}
      %       \O(\ksi) \xrightarrow{\simeq} \O_f(\ksi) \xrightarrow{\simeq} \O_f(\ksi_c^d) \xleftarrow{\simeq} \O(\ksi_c^d).
      % \end{equation}      
      
      % Finally, if $F: \O \to \P$ (with $\P$ also fibrant),
      % then the images of $(\alpha_i)_{\**}$ and $(\beta_i)_{\**}$ yield
      % homotopy equivalences $\P(\F(\ksi))^\Gamma \to \P(\F(\theta))^\Gamma$ for any graph subgroup $\Gamma \leq \Stab(\ksi)$,
      % as desired.
\end{proof}

We can now prove the main result of this subsection.
 
\begin{proposition}
      [{c.f. \cite[4.15]{Cav14}, \cite[2.13]{BM13}}]
      \label{CAV_4.15_PROP}
      \label{2OUTOF3_PROP}
      Suppose $\V$ is right proper and has cofibrant symmetric pushout powers.
      Then the class of weak $\F$-equivalences in $\mathsf{Op}^G(\V)$ satisfies the 2-out-of-3 condition.
\end{proposition}
\begin{proof}
      Let $\O \xrightarrow{F} \P \xrightarrow{L} \Q$ be a composition of maps in $\mathsf{Op}^G(\V)$.
      If $F$ and $L$ are weak $\F$-equivalences,
      the composite is obviously a local weak $\F$-equivalence:
      $\O(\ksi)^\Gamma \xrightarrow{\sim} \P(F(\ksi))^\Gamma \xrightarrow{\sim} \Q(LF(\ksi))^\Gamma$.
      Moreover, as functors preserve equivalences of colors, $L F$ is essentially surjective by Lemma \ref{CAV_4.10_LEM}. 
      % Moreover, concatination of $\V$-intervals collapses the consecutive essential surjectivity diagrams on the left
      % onto the one on the right.
      % \begin{equation}
      %       \begin{tikzcd}[row sep = tiny]
      %             \1 \arrow[rr, dashed, "a"] \arrow[dr]
      %             &&
      %             j^{\**}\O^H \arrow[dd]
      %             \\
      %             & \J \arrow[dr, dashed]
      %             &&
      %             \1 \arrow[rr, "a", dashed] \arrow[dr]
      %             &&
      %             j^{\**}\O^H \arrow[dd]
      %             \\
      %             \1 \arrow[rr, dashed, "b"] \arrow[ur] \arrow[dr]
      %             &&
      %             j^{\**} \P^H \arrow[dd]
      %             &&
      %             \J \** \J' \arrow[dr, dashed]
      %             \\
      %             & \J' \arrow[dr, dashed]
      %             &&
      %             \1 \arrow[rr, "c"] \arrow[ur]
      %             &&
      %             j^{\**}\Q^H
      %             \\
      %             \1 \arrow[ur] \arrow[rr, "c"]
      %             &&
      %             j^{\**} \Q^H
      %       \end{tikzcd}
      % \end{equation}
      
      If $L$ and $FL$ are weak $\F$-equivalences,
      then $F$ is a local weak $\F$-equivalence by 2-out-of-3 in each $\V^{\Stab(\ksi)}_{\F_\ksi}$.
      Moreover, if $b \in \P^H$, then by Remark \ref{ESS_SUR_REM}, there exists $a \in \mathfrak C(\O)^H$ such that
      $LF(a)$ and $L(b)$ are (virtually) $H$-equivalent.
      Lemma \ref{REF_VIRT_LEM} then implies $F(a)$ and $b$ are virtually $H$-equivalent, 
      and since $\V$ is right proper, Lemma \ref{RIGHTPROPER_LEM} implies they are $H$-equivalent.

      Lastly, suppose $F$ and $LF$ are weak $\F$-equivalences.
      It is immediate that $L$ is essentially surjective.
      Now, given a signature $\theta = (d_1,\ldots,d_n;d_0$) in $\C(\P)$,
      let $\Lambda = \lambda_1 \amalg \dots \amalg \lambda_r$ denote the partition of $\underline{n}$
      where $i < j$ are in the same class iff there exists $k_{i,j} \in G$ such that $d_j = k_{i,j} \cdot d_i$.
      Define $R' \subseteq \underline{n}$ to be the subset of minimal representatives in each class,
      $R = R' \amalg \set{0}$,
      and $H_r$ the stabilizer in $G$ of $c_r$ for each $r \in R$.
      Moreover, fix choices of $k_{r,j}$. 
      
      By the essential surjectivity of $F$, for all $r \in R$ there exist $c_r \in \C(\O)^{H_r}$ such that
      $F(c_r)$ is (homotopy) $H_r$-equivalent to $d_r$.
      %
      We extend the set $\set{c_r}_{r\in R}$ to a signature $(c_1,\ldots, c_n;c_0)$
      by defining $c_j = k_{r,j} \cdot c_r$.
      % THESE ARE NOT INDEPENDENT of choice, but good enough
      % Consequently, $F(c_i)$ is homotopy equivalent to $d_i$ via $k_{r,i}\gamma_r$,
      % where $\gamma_r$ realizes the homotopy equivalence between $F(c_r)$ and $d_r$ for $i \in \lambda_r$.
      This yields a diagram of the form
      \begin{equation}
            \label{TWOOFTHREE_EQ}
            \begin{tikzcd}
                  \O(c_1,\ldots, c_n;c_0) \arrow[r, "(1)"]
                  &
                  \P(F(c_1),\ldots, F(c_n); F(c_0)) \arrow[d,dash, "(3)"] \arrow[r, "(2)"]
                  &
                  \Q(LF(c_1),\ldots, LF(c_n);LF(c_0)) \arrow[d, dash, "(4)"]
                  \\
                  &
                  \P(d_1,\ldots, d_n;d_0) \arrow[r, "(5)"]
                  &
                  \Q(L(d_1),\ldots, L(d_n); L(d_0)).
            \end{tikzcd}
      \end{equation}
      $(1)$ is a weak equivalence in $\V^{\Stab(\ksi)}_{\F_\ksi}$ by assumption, and
      $(2)$ is a weak-equivalence in $\V^{\Stab(\ksi)}_{\F_\ksi}$ by 2-out-of-3 here.
      $(3)$ and $(4)$ are zig-zags of weak equivalences in $\V^{\Stab(\theta)}_{\mathrm{Gr}_\theta}$ by iterating applications of
      Propositions \ref{CAV_4.14_PROP1} and \ref{CAV_4.14_PROP2},
      as each application only changes the colors in a single partition class.
      As these zig-zags are functorial, the above diagram commutes.
      %
      Finally, $\Stab(\ksi) \geq \Stab(\theta)$ by the same calculation as in \eqref{STAB_KT_EQ}.
      Thus $(5)$ is a weak equivalence in $\V^{\Stab(\theta)}_{\F_\theta}$ by 2-out-of-3, and hence
      $L$ is a local weak $\F$-equivalence, as desired.
\end{proof}



% ------------------------------- DWYER-KAN DESCRIPTION -----------------------------

\subsection{Dwyer-Kan equivalences}

We would like to recognize our weak $\F$-equivalences slightly differently.

\begin{definition}
      $F: \O \to \P$ in $\Op^G(\V)$ is a \textit{Dwyer-Kan} (or \textit{DK}) \textit{$\F$-equivalence} if
      $F$ is a local weak $\F$-equivalence and
      $j^{\**}\pi_0(F^H)$ is essentially surjective for all $H \leq G$.
\end{definition}

\begin{remark}
      If $\V$ has diagonals, then $F \in \Op^G(\V)$ is a $DK$-$\F$-equivalence iff
      $F$ is a local weak $\F$-equivalence such that 
      the associated map of \textit{$\F$-genuine equivariant operads} under the composite
      \begin{equation}
            \Op^G(\V) \to \Op_\F(\V) \xrightarrow{\pi_0} \Op_\F(\Set) 
      \end{equation}
      is an equivalence.
      This will be explored further, along with colored genuine equivariant operads, in a sequal.
\end{remark}

\begin{proposition}
      \label{WE_ARE_DK_PROP}
      Weak $\F$-equivalences in the sense of Theorem \ref{MODEL_THM} are DK-$\F$-equivalences.
\end{proposition}
\begin{proof}
      By Lemma \ref{VIR_HTPY_LEM}, essential surjectivity implies $\pi_0$-essential surjectivity. 
\end{proof}


For the reverse direction (cf. \cite[\S 2]{BM13}), we need to show that
homotopy equivalences are all virtual equivalences.
This requires another condition on $\V$, namely that the homotopy equivalences all satisfy a coherence condition.

\begin{definition}
      Recall the category $\mathbb A \in \Cat^{\set{0,1}}(\V)$ which detects arrows.
      A cofibration $\mathbb A \to \J$ in $\Cat^{\set{0,1}}(\V)$ into a $\V$-interval is called \textit{natural} if
      it fits into a commuting diagram
      \begin{equation}
            \begin{tikzcd}
                  \mathbb A \arrow[d, tail] \arrow[r]
                  &
                  \I \arrow[d, "\sim"]
                  \\
                  \J \arrow[r, "\sim"']
                  &
                  \I_f.
            \end{tikzcd}
      \end{equation}

      A homotopy equivalence between two objects in a $\V$-category $\mathcal C$ is called \textit{coherent} if
      the detecting map $\alpha: \mathbb A \to \mathcal C_f$ factors along a natural cofibration
      \begin{equation}
            \begin{tikzcd}
                  \mathbb A \arrow[r, "\alpha"] \arrow[d, tail, dashed]
                  &
                  \mathcal C_f
                  \\
                  \J \arrow[ur, dashed]
            \end{tikzcd}
      \end{equation}
      The $\V$-category $\mathcal C$ satisfies the \textit{coherence axiom} if all homotopy equivalences are coherent.
\end{definition}

\begin{proposition}
      If $\V$ is right proper with cofibrant unit satisfying the coherence axiom, then
      DK-$\F$-equivalences are weak $\F$-equivalences in $\Op^G(\V)$.
\end{proposition}
\begin{proof}
      It suffices to show that any homotopy $H$-equivalence between objects in $\P$ is in fact an $H$-equivalence.
      The coherence axiom implies that all homotopy $H$-equivalences are virtual $H$-equivalences, and then
      right properness and Lemma \ref{RIGHTPROPER_LEM} imply these are actual $H$-equivalences.
\end{proof}

\begin{remark}
      We note that this is a \textit{non-equivariant} condition on $\V$.
\end{remark}



% -------------------- Satisfying the coherence axiom - nothing new --------------------
\subsubsection{Satisfying the coherence axiom - nothing new}
\todo[inline]{come back: blend in to the above better, maybe don't try to add anything}
\begin{lemma}
      [{\cite[Prop 2.24]{BM13}}]
      If $\V$ satisfies transfer for operads and
      is right proper with cofibrant unit,
      then $\V$ satisfies the coherence axiom.
\end{lemma}
\begin{proof}
      This is proved in \textit{loc cite} using the $W$-construction $W(H,\I) \simeq W_!J$.
      \todo[inline]{very unsatisfying.}
\end{proof}

This has been known for a while for $\sSet$ (implied by Dwyer-Kan, Begner).

\begin{proposition}
      [{cf. \cite[\S 1]{Joy02}}]
      $(\sSet, \times)$       % or $(\sSet_{\**}, \wedge)$ then
      satisfies the coherence axiom.
\end{proposition}
\begin{proof}
      Let $\mathcal C_f \in \Cat(\sSet)$ be locally fibrant, and
      suppose $\alpha: \mathbb A \to \mathcal C_f$ realizes a homotopy equivalence.
      Then $h c N \alpha: \Delta[1] \to h c N \mathcal C$ realizes a quasi-isomorphism (as $\Ho(h c N (-)) \simeq \pi_0(-)$).
      Since $W_!N \I$ is a $\V$-interval,
      it suffices to show that $\Delta[1] \to N \I$ is a trivial cofibration.
      This is obvious a weak equivalence as both are contractible,
      and (by e.g. \cite[Lemma 0.15]{Rie}) is cellular on outer horn inclusions.
\end{proof}

\begin{remark}
      There are other known examples.
      In particular,   
      if $\V$ satisfies the Lurie's \textit{Invertibility Hypothesis} \cite[A.3.2.12]{Lur09},
      then $\V$ satisfies the coherence axiom.

      Moreover, \cite{Law16} shows that it suffices to show
      \begin{enumerate*}
      \item $\V$ is cofibrantly generated and locally presentable
      \item every object in $\V$ is cofibrant, and
      \item weak equivalences are closed under filtered colimits.
      \end{enumerate*}
\end{remark}


\subsection{Examples}




% ---------------------------------------- EXAMPLES ------------------------------

\begin{example}
      $(\Top,\times)$:
      cellularity in \cite{Pia91},
      cofibrant symmetric pushout powers in {\color{red} SOURCE},
      right proper and generating set of intervals follows from all objects being fibrant (and \cite[Lemma 2.1]{BM13} for the latter).
\end{example}

\todo[inline]{come back: compare cofibrant symmetric pushout powers with}
\begin{itemize}
\item freely powered of Lurie
\item commutative monoid axiom of White-Yau
\item symmetric h-monoidal of Pavlov-Scholbach
\end{itemize}


% \begin{example}
%       $R$ a commutative ring containing the rational numbers,
%       $\mathcal A$ the abelian category of projective $R$-modules,
%       and consider $Ch(\mathcal A)$ with projective model strucutre:

%       \begin{itemize}
%       \item cellularity in \cite{Ste16},
%       \item cofibrant symmetric pushout powers implied by ``freely powered'', proved in \cite[Prop 7.1.4/7]{Lur17},
%       \item right proper?
%       \item generating set of intervals?
%       \item cofibrant unit?            
%       \end{itemize}

%       More generally, $\mathcal C$ locally presentable quasi-abelian category,
%       $R$ a commutative monoid object in $\mathcal C$ containing the rational numbers,
%       and consider $dg_R(\mathcal C)$ with the projective model structure \cite[Prop 2.12]{Wal15};
%       this also satisfies cofibrant symmetric pushout powers by \cite[Prop 3.4]{Wal15}.
% \end{example}








%%%%%%%%%%%%%%%%%%%%%%%%%%%%%%%%%%%%%%%%%%%%%%%%%%%%%%%%%%%%%%%%%%%%%%%
\newpage

\section{In $\mathsf{dSet}_G$}

\begin{definition}
      Define the \textit{genuine operadic nerve} $N: \Op_G \to \dSet_G$ by
      \begin{equation}
            N\P(T) = \Hom_{\Op_G}(T, \P)
      \end{equation}
      where we think of $T$ as the operad $T \in \Op^G \into \Op_G$. 
\end{definition}

\begin{remark}
      We note that $N\P \in (SCI)^{\boxslash !}$,
      as $T \in \Op_G$ is a free $\mathbb F_G$-algebra on its vertices.
\end{remark}

\begin{remark}
      We can rephrase the definition of being an $\mathbb F_G$-algebra in terms of $N\P$.
      For $\P \in \Sym_G$ a $G$-symmetric sequence,
      a genuine $G$-operad structure on $\P$ is given by:
      \begin{itemize}
      \item Composition Maps: $ $\\
            maps 
            $N\P(T) \to \P(\mathsf{lr}(T))$
            for all $T \in \Omega_G$.
      \item Naturality under restriction and conjugation: $ $\\
            maps $N\P(T_1) \to N\P(T_0)$
            for all quotient maps $T_0 \to T_1$ in $\Omega_{G,0}$,
            such that the following commutes:
            \begin{equation}
                  \begin{tikzcd}
                        N\P(T_1) \arrow[r] \arrow[d]
                        &
                        \P(\mathsf{lr}(T_1)) \arrow[d]
                        \\
                        N\P(T_0) \arrow[r]
                        &
                        \P(\mathsf{lr}(T_0)).
                  \end{tikzcd}
            \end{equation}
      \item Associativity under $\mathbb F_G$: $ $\\
            maps $N\P(T_1) \to N\P(T_0)$
            for all planar tall maps $T_0 \to T_1$ in $\Omega_G^t$,
            such that the analogus diagram (with the right vertical map the identity) commutes.\footnote{
              As in \cite{BP17}, we note that ``associativity'' under $\mathbb F_G$ includes both
              the usual notion of associativity of our composition maps,
              but also unitality;
              this is recorded here by the fact that degeneracies are always planar tall.}
      \end{itemize}
\end{remark}

The above reflects the following result.

\begin{proposition}
      $\Op_G$ is equivalent to the subcategory of $\mathsf{dSet_G}$ spanned by those $X$ such that
      \begin{enumerate}
      \item $X(H/H) = \set{\**}$ for all $H \leq G$.
      \item $X(T) \cong \otimes_{T_v \in V(T)}X(T_v)$. 
      \end{enumerate}
\end{proposition}
\begin{proof}
      The fact that $N\P \in (SCI_{\F})^{\boxslash !}$ is immediate, as remarked above.

      For the reverse direction, we will follow the construction of the homotopy operad as in \cite[\S 6]{MW09},
      replacing their use of inner horn inclusions with \textit{orbital} inner $G$-horn inclusions,
      to show that any $X \in (OHI)^{\boxslash !}$ is in the image of $N$; 
      the result will then follow from \cite[HYPER PROP]{BP18}.

      In fact, interpreting all of their pictures are as \textit{orbital} representations of $G$-trees yields that,
      for all $C \in \Sigma_G$
      \begin{itemize}
      \item $\sim_{G e}$ is an equivalence relation on $X(C)$ for all $Ge \in E_G(C)$.
      \item The relations $\sim_{G e}$ and $\sim_{G e'}$ are equal for all $e,e'\in E(C)$.
      \item $[h] \circ [f] = [h \circ f]$ yields a well-defined composition map.
      \end{itemize}
      \todo[inline]{naturality, associativity of composition}
\end{proof}



\newpage

\subsection{The homotopy genuine equivariant operad}


Our goal in this section is to build,
for each $G$-$\infty$-operad $X \in \mathsf{dSet}^G$,
the associated homotopy genuine equivariant operad
$\mathsf{ho} (X)$,
which we will describe as an object in
$\mathsf{dSet}_G$
satisfying a strict Segal condition.


We start with some notation. 
Given a multiset $I$ of edges of a tree $T \in \Omega$
(formally, $I$ is a function 
$I \colon \boldsymbol{E}(T) \to \mathbb{N}_0$),
we write $\sigma^I T \in \Omega$
for the tree obtained by degenerating $T$ once for each edge in $I$.
More explicitly, $\sigma^I T$ is the unique tree such that there is a planar degeneracy
$\pi \colon \sigma^I T \to T$
such that $|\pi^{-1}(e)| = I(e) + 1$.
Moreover,
note that if $T\in \Omega_G$ is a $G$-tree, 
then $\sigma^{I} T \in \Omega_{G}$
can be defined if $I$ is $G$-equivariant
(formally, this means that the multiset $I$ is a composite
$\boldsymbol{E}(T) \to \boldsymbol{E}_G(T)
\to \mathbb{N}_0$).

Our main interest will be in degeneracies of $G$-corollas. Recall that, up to isomorphism, 
a $G$-corolla $C \in \Sigma_G$ is determined the number $0 \leq k$ of leaf orbits
and isotropy subgroups
$H_i \leq H_0 \leq G$ for $0 \leq i \leq k$,
where $H_0$ is the isotropy of a (chosen) root edge.
Pictorially, such a $G$-corolla has the orbital representation given on the left below,
but in this section we will find it more convenient to label edge orbits using coset notation as on the right below,
so that $[e_i] = G e_i$ denotes the $G$-orbit of $e_i$.
\[
\begin{tikzpicture}
[grow=up,auto,level distance=2.3em,every node/.style = {font=\footnotesize},dummy/.style={circle,draw,inner sep=0pt,minimum size=1.75mm}]
	\node at (0,0) [font=\normalsize]{$C$}
		child{node [dummy] {}
			child{
			edge from parent node [swap,near end] {$G/H_k$} node [name=Kn] {}}
			child{
			edge from parent node [near end] {$G/H_1$}
node [name=Kone,swap] {}}
		edge from parent node [swap] {$G/H_0$}
		};
		\draw [dotted,thick] (Kone) -- (Kn) ;
	\node at (5,0) [font=\normalsize]{$C$}
		child{node [dummy] {}
			child{
			edge from parent node [swap,near end] {$[e_k]$} node [name=Kn] {}}
			child{
			edge from parent node [near end] {$[e_1]$}
node [name=Kone,swap] {}}
		edge from parent node [swap] {$[e_0]$}
		};
		\draw [dotted,thick] (Kone) -- (Kn) ;
\end{tikzpicture}
\]
We will then abbreviate $\sigma^i C = \sigma^{[e_i]} C$, and write $e_i$, $e_i'$ for the two edges of $\sigma^i C $ that degenerate the edge $e_i$ of $C$,
with $e_i$ denoting the inner edge and $e'_i$ the outer
edge.
\[
\begin{tikzpicture}
[grow=up,auto,level distance=3em,
every node/.style = {font=\footnotesize},
dummy/.style={circle,draw,inner sep=0pt,minimum size=1.75mm}]
	\node at (0,0) [font=\normalsize]{$\sigma^0 C$}
		child{node [dummy] {}
			child{node [dummy] {}
				child{
				edge from parent node [swap,near end] {$[e_k]$} node [name=Kn] {}}
				child{
				edge from parent node [near end] {$[e_1]$}
node [name=Kone,swap] {}}
			edge from parent node [swap] {$[e_0]$}}
		edge from parent node [swap] {$[e'_0]$}
		};
		\draw [dotted,thick] (Kone) -- (Kn) ;
	\node at (5,0) [font=\normalsize]{$\sigma^i C$}
		child{node [dummy] {}
			child{
			edge from parent node [swap,near end] {$[e_k]$} node [near start,inner sep=1pt,name=Kn] {}}
			child[level distance=3.4em]{node [dummy] {}
				child[level distance=2.7em]{
				edge from parent node [swap] {$[e'_i]$}
}
			edge from parent node [near end,swap] {$[e_i]$}
node [near start,inner sep=1pt,name=Kone,swap] {}
node [near start,inner sep=1pt,name=Kone1] {}}
			child{
			edge from parent node [near end] {$[e_1]$}
node [swap] {}
node [near start,inner sep=1pt,name=Kn1,swap]{}}
		edge from parent node [swap] {$[e_0]$}
		};
		\draw [dotted,thick] (Kone) -- (Kn) ;
		\draw [dotted,thick] (Kone1) -- (Kn1) ;
\end{tikzpicture}
\]
$\sigma^i C$ then has an orbital inner face
$\sigma^i C - [e_i]$ obtained by removing $[e_i]$
as well as an orbital outer face obtained by removing $e'_i$,
which we denote $\sigma^i C - [e'_i]$.
Moreover, note that we have natural identifications
$C = \sigma^i C - [e_i]$,
$C = \sigma^i C - [e'_i]$.

In what follows, we will find it convenient to simplify notation by denoting maps $\Omega[T] \to X$,
where $T \in \Omega_G$ and $X \in \mathsf{dSet}^G$,
simply as $T \to X$.


\begin{definition}
	Let $X \in \mathsf{dSet}^G$ be a $G$-$\infty$-operad and $C$ a $G$-corolla with edge orbits
	$[e_0],\cdots,[e_k]$.
	
	Given two operations 
	$f,g\colon C \rightrightarrows X$,
	we write $f \sim_i g$ if there exists a map
	$H \colon \sigma^i C \to X$ such that
\begin{itemize}
\item $f$ equals the restriction $H|_{\sigma^i C-[e'_i]}$;
\item $g$ equals the restriction $H|_{\sigma^i C-[e_i]}$;
\item the restriction $H|_{\sigma^i [e_i]}$
is the degeneracy $\sigma^i [e_i] \to [e_i] \to C \to X$.
\end{itemize}
\end{definition}


\begin{remark}\label{HOMOTBOUND REM}
	Note that if $f \sim_i g$ then it must be
	$f|_{\partial C} = g|_{\partial C}$.
\end{remark}


\begin{example}\label{EQUIVSIM EX}
	Let $G = \mathbb{Z}_{/2} = \{\pm 1\}$
	and consider the $G$-corolla with orbital and expanded representations as given on the left below.
\[
\begin{tikzpicture}
[grow=up,auto,level distance=2.3em,every node/.style = {font=\footnotesize},dummy/.style={circle,draw,inner sep=0pt,minimum size=1.75mm}]
	\node at (0,0) [font=\normalsize]{$C$}
		child{node [dummy] {}
			child{
			edge from parent node [swap] {$G \cdot e$}
node [name=Kone,swap] {}}
		edge from parent node [swap] {$G/G \cdot r$}
		};
	\node at (3,0) [font=\normalsize]{$C$}
		child{node [dummy] {}
			child{
			edge from parent node [swap,near end] {$-e$} node [name=Kn] {}}
			child{
			edge from parent node [near end] {$e$}
node [name=Kone,swap] {}}
		edge from parent node [swap] {$r$}
		};
	\node at (7,0) [font=\normalsize]{$\sigma^{\{e,-e\}} C$}
		child{node [dummy] {}
			child{node [dummy] {}
				child{
				edge from parent node [swap] {$G \cdot e'$}
node [swap] {}}
			edge from parent node [swap] {$G \cdot e$}
node [swap] {}}
		edge from parent node [swap] {$G/G \cdot r$}
		};
	\node at (10,0) [font=\normalsize]{$\sigma^{\{e,-e\}} C$}
		child{node [dummy] {}
			child{node [dummy] {}
				child{
				edge from parent node [swap] {$-e'$} node {}}
			edge from parent node [swap,near end] {$-e$} node {}}
			child{node [dummy] {}
				child{
				edge from parent node {$e'$}
node [swap] {}}
			edge from parent node [near end] {$e$}
node [swap] {}}
		edge from parent node [swap] {$r$}
		};
\end{tikzpicture}
\]
$C$ then has a single leaf $G$-edge orbit $[e] = G \cdot e$, so that for
$f,g \colon C \to X$ it is
$f \sim_1 g$
if there exists a 
$H \colon \sigma^{\{e,-e\}}C \to X$
such that 
\begin{equation}\label{EQUIVHOMOT EQ}
	f = H|_{\sigma^{\{e,-e\}}C - \{e',-e'\}}
\qquad
	g = H|_{\sigma^{\{e,-e\}}C - \{e,-e\}}
\qquad
	H_{\sigma^e e}, H|_{\sigma^{-e}-e} \text{ are degenerate}.
\end{equation}
It is worthwhile to compare this equivariant relation with the relations obtained if one forgets the $G$-actions. Indeed, while \eqref{EQUIVHOMOT EQ} implicitly assumes that all of $f,g,H$ are $G$-equivariant,
by omitting that assumption one can reinterpret 
\eqref{EQUIVHOMOT EQ}
as defining a relation
$f \sim_{[e]} g$ between not necessarily $G$-equivariant maps $f,g \colon C \to X$.

A priori, $\sim_{[e]}$ relation differs from the 
non-equivariant 
$\sim_{e}$ and $\sim_{-e}$
relations obtained by regarding $C$ as a non-equivariant corolla.
However, for $f,g,H$ as in \eqref{EQUIVHOMOT EQ} one has
\begin{equation}\label{EQUIVSIM EQ}
f = H|_{\sigma^{\{e,-e\}}C - \{e',-e'\}}
\sim_e H|_{\sigma^{\{e,-e\}}C - \{e,-e'\}}
\sim_{-e} H|_{\sigma^{\{e,-e\}}C - \{e,-e\}} =g
\end{equation}
so that, by Lemma \ref{EQUIVI LEM}(b) below one has that
$f \sim_{[e]} g$ in fact implies
$f \sim_{e} g$. Moreover, the converse statement follows immediately by using degeneracies.

More generally, similar considerations show that the $\sim$ relations are compatible with restricting the $G$-actions.
\end{example}


\begin{lemma}\label{EQUIVI LEM}
	Let $X \in \mathsf{dSet}^G$ be a $G$-$\infty$-operad and $C$ a $G$-corolla with edge orbits
	$[e_0],\cdots,[e_k]$. Then:
\begin{itemize}
	\item[(a)] each of the relations $\sim_i$ is an equivalence relation;
	\item[(b)] all the equivalence relations $\sim_i$ coincide.
\end{itemize}
\end{lemma}

\begin{proof}
	We first address (a). 
	
	For the reflexive condition, one can take $H$ to be the degeneracy
	$\sigma^i C \xrightarrow{\sigma^i} C \xrightarrow{f} X$.
	
	For the symmetry and transitive conditions, consider the tree
	$\sigma^{ii} C$, which degenerates $[e_i]$ twice.
\[
\begin{tikzpicture}
[grow=up,auto,level distance=3em,
every node/.style = {font=\footnotesize},
dummy/.style={circle,draw,inner sep=0pt,minimum size=1.75mm}]
	\node at (0,0) [font=\normalsize]{$\sigma^{ii} C$}
		child{node [dummy] {}
			child{
			edge from parent node [swap,near end] {$[e_k]$} node [near start,inner sep=1pt,name=Kn] {}}
			child[level distance=3.4em]{node [dummy] {}
				child[level distance=2.7em]{node [dummy] {}
					child[level distance=2.7em]{
					edge from parent node [swap] {$[e''_i]$}
}
				edge from parent node [swap] {$[e'_i]$}
}
			edge from parent node [near end,swap] {$[e_i]$}
node [near start,inner sep=1pt,name=Kone,swap] {}
node [near start,inner sep=1pt,name=Kone1] {}}
			child{
			edge from parent node [near end] {$[e_1]$}
node [swap] {}
node [near start,inner sep=1pt,name=Kn1,swap]{}}
		edge from parent node [swap] {$[e_0]$}
		};
		\draw [dotted,thick] (Kone) -- (Kn) ;
		\draw [dotted,thick] (Kone1) -- (Kn1) ;
\end{tikzpicture}
\]
Suppose $f \sim_i g$, and let 
$H \colon \sigma^{i} C \to X$ be the associated homotopy.
Define a map 
$\bar{H} \colon \Lambda^{[e_i]}_o[\sigma^{ii} C] \to X$ by
\[
	\bar{H}|_{\sigma^{ii}C - [e''_i]} = H,
		\qquad
	\bar{H}|_{\sigma^{ii}C - [e'_i]} = f \circ \sigma^i,
		\qquad
	\bar{H}|_{\sigma^{ii} [e_i]} = 
	f|_{[e_i]} \circ \sigma^{ii} =
	g|_{[e_i]} \circ \sigma^{ii}.
\]
Since the orbital inner horn inclusion
$\bar{H} \colon \Lambda^{[e_i]}_o[\sigma^{ii} C] \to \Omega[C]$
is $G$-inner anodyne,
$\bar{H}$ admits an extension $\widetilde{H} \colon \sigma^{ii}C \to X$.
The restriction $\bar{H}|_{\sigma^{ii}C - [e_i]}$ then provides the homotopy exhibiting $g \sim_i f$, and symmetry of $\sim_i$ follows.

Next, suppose $f \sim_i g$ and $g \sim_i h$, and let 
$H \colon \sigma^{i} C \to X$ and
$K \colon \sigma^{i} C \to X$ be be the associated homotopies.
Define a map 
$\bar{H} \colon \Lambda^{[e'_i]}_o[\sigma^{ii} C] \to X$ by
\[
	\bar{H}|_{\sigma^{ii}C - [e''_i]} = H,
		\qquad
	\bar{H}|_{\sigma^{ii}C - [e_i]} = K,
		\qquad
	\bar{H}|_{\sigma^{ii} [e_i]} = 
	f|_{[e_i]} \circ \sigma^{ii} =
	g|_{[e_i]} \circ \sigma^{ii} =
	h|_{[e_i]} \circ \sigma^{ii}.
\]
$\bar{H}$ again admits an extension $\widetilde{H} \colon \sigma^{ii}C \to X$, and the restriction $\bar{H}|_{\sigma^{ii}C - [e'_i]}$
provides the homotopy exhibiting $f \sim_i g$, so that transitivity of $\sim_i$.

We next turn to (b). Consider the tree $\sigma^{ij} C$ which degenerates $C$ once along each of $[e_i]$ and $[e_j]$.
\[
\begin{tikzpicture}
[grow=up,auto,level distance=2.75em,
every node/.style = {font=\footnotesize},
dummy/.style={circle,draw,inner sep=0pt,minimum size=1.75mm}]
	\node at (0,0) [font=\normalsize]{$\sigma^{ij} C$}
		child{node [dummy] {}
			child{
			edge from parent node [swap,near end] {$[e_k]$} node [near start,inner sep=1pt,name=Kn] {}}
			child[level distance=3.4em,sibling distance=2em]{node [dummy] {}
				child[level distance=2.7em]{
				edge from parent node [swap] {$[e'_j]$}
}
			edge from parent node [very near end,swap] {$[e_j]$}
node [near start,inner sep=1pt,name=Kone,swap] {}
node [inner sep=1pt,name=Kn2] {}}
			child[level distance=3.4em,sibling distance=2em]{node [dummy] {}
				child[level distance=2.7em]{
				edge from parent node {$[e'_i]$}
}
			edge from parent node [very near end] {$[e_i]$}
node [inner sep=1pt,name=Kone2,swap] {}
node [near start,inner sep=1pt,name=Kone1] {}}
			child{
			edge from parent node [near end] {$[e_1]$}
node [swap] {}
node [near start,inner sep=1pt,name=Kn1,swap]{}}
		edge from parent node [swap] {$[e_0]$}
		};
		\draw [dotted,thick] (Kn) -- (Kone) ;
		\draw [dotted,thick] (Kone1) -- (Kn1) ;
		\draw [dotted,thick] (Kone2) -- (Kn2) ;
\end{tikzpicture}
\]
Suppose $f \sim_i g$ with $H \colon \sigma^{i} C \to X$ the associated homotopy.
Define a map 
$\bar{H} \colon \Lambda^{[e_i]}_o[\sigma^{ij} C] \to X$ by
\[
	\bar{H}|_{\sigma^{ij}C - [e'_j]} = H,
		\qquad
	\bar{H}|_{\sigma^{ij}C - [e_j]} = f \circ \sigma^i,
		\qquad
	\bar{H}|_{\sigma^{ij}C - [e'_i]} = f \circ \sigma^j.
\]
Yet again, $\bar{H}$ admits an extension $\widetilde{H} \colon \sigma^{ij}C \to X$, and the restriction $\bar{H}|_{\sigma^{ij}C - [e_i]}$
provides a homotopy exhibiting $g \sim_j f$. (b) now follows.
\end{proof}

In light of Lemma \ref{EQUIVI LEM},
given $f,g \rightrightarrows C \to X$ with 
$C$ a $G$-corolla and $X$ a $G$-$\infty$-operad,
we will henceforth write $f \sim g$ whenever $f \sim_i g$ for some (and thus all) $i$.
We now extend the $\sim$ relation.

\begin{definition}\label{XTENDSIM DEF}
	Let $T \in \Omega_G$ be a $G$-tree
	and $X \in \mathsf{dSet}^G$ be a 
	$G$-$\infty$-operad.
	
	Given dendrices $x,y\colon T \to X$ we write
	$x \sim y$ if there are equivalences of restrictions
	$x|_{T_v} \sim y|_{T_v}$ for all $G$-vertices
	$v \in \boldsymbol{V}_G(T)$.
	
	Further, we define $\mathsf{ho}(X)(T) = X(T)/\sim$.
\end{definition}

\begin{proposition}
Let $X \in \mathsf{dSet}^G$ be a $G$-$\infty$-operad. Then the assignment 
		$T \mapsto \mathsf{ho}(X)(T)$
		is a contravariant functor in $T \in \Omega_G$, i.e.
		$\mathsf{ho}(X)\in \mathsf{dSet}_G$.
\end{proposition}


\begin{proof}
	It suffices to show that the $\sim$ equivalence relations are compatible with the generating classes of maps in $\Omega_G$, namely
	degeneracies, inner faces, outer faces, and quotient maps.
	
	The cases of degeneracies and outer faces are obvious. In the case of quotients, 
	since any quotient $\bar{T} \to T$ of $G$-trees induces quotients on $G$-vertices, it suffices to consider the case of a quotient
	$\bar{C} \xrightarrow{\pi} C$ of $G$-corollas.
	But it is then straightforward to check that if a homotopy exhibiting $f \sim_0 g$ also induces a homotopy exhibiting 
	$f \circ \pi \sim_0 g \circ \pi$
	(notably, the same needs not be true for the relations $f \sim_i g$ when $0<i$, 
	in which case the exhibiting homotopy 
	may instead exhibit a string of relations 
	$f \circ \pi \sim \cdots \sim g \circ \pi$
	as in \eqref{EQUIVSIM EQ}).

It remains to address the most interesting case,
that inner faces. Since inner faces can be factored as composites of inner faces that collapse a singe inner edge orbit,
it suffices to consider the case of faces
$D \to T$ where $T$ has a single edge edge orbit.
I.e. we can assume that there are $G$-corollas
$C_1$, $C_2$ such that 
$T = C_1 \amalg_{[e_i]} C_2$ and
$D = T - [e_i]$, as illustrated below.
\[
\begin{tikzpicture}
[grow=up,auto,level distance=3em,
every node/.style = {font=\footnotesize},
dummy/.style={circle,draw,inner sep=0pt,minimum size=1.75mm}]
	\node at (0,0) [font=\normalsize]{$C_1$}
		child{node [dummy] {}
			child{
			edge from parent node [swap,near end] {} node [near start,inner sep=1pt,name=Kn] {}}
			child[level distance=3.4em]{node {}
			edge from parent node [near end,swap] {$[e_i]$}
node [near start,inner sep=1pt,name=Kone,swap] {}
node [near start,inner sep=1pt,name=Kone1] {}}
			child{
			edge from parent node [near end] {}
node [swap] {}
node [near start,inner sep=1pt,name=Kn1,swap]{}}
		edge from parent node [swap] {$[e_0]$}
		};
		\draw [dotted,thick] (Kone) -- (Kn) ;
		\draw [dotted,thick] (Kone1) -- (Kn1) ;
	\node at (4,0) [font=\normalsize]{$C_2$}
		child{node [dummy] {}
			child{
			edge from parent node [swap,near end] {} node [name=Kn] {}}
			child{
			edge from parent node [near end] {}
node [name=Kone,swap] {}}
		edge from parent node [swap] {$[e_i]$}
		};
		\draw [dotted,thick] (Kone) -- (Kn) ;
	\node at (9,0) [font=\normalsize]{$T$}
		child{node [dummy] {}
			child{
			edge from parent node [swap,near end] {} node [near start,inner sep=1pt,name=Kn] {}}
			child[level distance=3.4em]{node [dummy] {}
				child{
				edge from parent node [swap,near end] {} node [name=Kn2] {}}
				child{
				edge from parent node [near end] {}
node [name=Kone2,swap] {}}
			edge from parent node [near end,swap] {$[e_i]$}
node [near start,inner sep=1pt,name=Kone,swap] {}
node [near start,inner sep=1pt,name=Kone1] {}}
			child{
			edge from parent node [near end] {}
node [swap] {}
node [near start,inner sep=1pt,name=Kn1,swap]{}}
		edge from parent node [swap] {$[e_0]$}
		};
		\draw [dotted,thick] (Kone) -- (Kn) ;
		\draw [dotted,thick] (Kone1) -- (Kn1) ;
		\draw [dotted,thick] (Kone2) -- (Kn2) ;
\end{tikzpicture}
\]
The claim is now that if
$x,y \colon T \to X$ are such that
$x|_{C_1} \sim y|_{C_1}$ and
$x|_{C_2} \sim y|_{C_2}$
then it is also 
$x|_{D} \sim y|_{D}$.
This will follow from the next two claims:
\begin{itemize}
\item[(i)] if $x,y \colon T \to X$ are such that
$x|_{C_1} = y|_{C_1}$ and
$x|_{C_2} = y|_{C_2}$
then $x|_{D} \sim y|_{D}$;
\item[(ii)]
given $x \colon T \to X$, $f\colon C_1 \to X$ and
$g \colon C_2 \to X$ such that
$f \sim x|_{C_1}$, $g \sim x|_{C_2}$,
there exists
$y \colon T \to X$ such that
$y|_{C_1} = f$, $y|_{C_2} = g$ and
$y|_D = x|_D$.
\end{itemize}
To show (i) and (ii), consider the degeneracies
$\sigma^0 T$ and $\sigma^i T$ pictured below.
\[
\begin{tikzpicture}
[grow=up,auto,level distance=3em,
every node/.style = {font=\footnotesize},
dummy/.style={circle,draw,inner sep=0pt,minimum size=1.75mm}]
	\node at (0,0) [font=\normalsize]{$\sigma^0 T$}
		child{node [dummy] {}
		child{node [dummy] {}
			child{
			edge from parent node [swap,near end] {} node [near start,inner sep=1pt,name=Kn] {}}
			child[level distance=3.4em]{node [dummy] {}
				child{
				edge from parent node [swap,near end] {} node [name=Kn2] {}}
				child{
				edge from parent node [near end] {}
node [name=Kone2,swap] {}}
			edge from parent node [near end,swap] {$[e_i]$}
node [near start,inner sep=1pt,name=Kone,swap] {}
node [near start,inner sep=1pt,name=Kone1] {}}
			child{
			edge from parent node [near end] {}
node [swap] {}
node [near start,inner sep=1pt,name=Kn1,swap]{}}
		edge from parent node [swap] {$[e_0]$}}
		edge from parent node [swap] {$[e'_0]$}
		};
		\draw [dotted,thick] (Kone) -- (Kn) ;
		\draw [dotted,thick] (Kone1) -- (Kn1) ;
		\draw [dotted,thick] (Kone2) -- (Kn2) ;
	\node at (6,0) [font=\normalsize]{$\sigma^i T$}
		child{node [dummy] {}
			child{
			edge from parent node [swap,near end] {} node [near start,inner sep=1pt,name=Kn] {}}
			child[level distance=3.4em]{node [dummy] {}
			child{node [dummy] {}
				child{
				edge from parent node [swap,near end] {} node [name=Kn2] {}}
				child{
				edge from parent node [near end] {}
node [name=Kone2,swap] {}}
			edge from parent node [swap] {$[e'_i]$}}
			edge from parent node [near end,swap] {$[e_i]$}
node [near start,inner sep=1pt,name=Kone,swap] {}
node [near start,inner sep=1pt,name=Kone1] {}}
			child{
			edge from parent node [near end] {}
node [swap] {}
node [near start,inner sep=1pt,name=Kn1,swap]{}}
		edge from parent node [swap] {$[e_0]$}
		};
		\draw [dotted,thick] (Kone) -- (Kn) ;
		\draw [dotted,thick] (Kone1) -- (Kn1) ;
		\draw [dotted,thick] (Kone2) -- (Kn2) ;
\end{tikzpicture}
\]
Given $x,y$ as in (i), one can now build a map
$H \colon \Lambda_o^{[e_i]}[\sigma^0 T] \to X$ by
\[
	H|_{\sigma^0 T - [e_0]} = x,
\qquad
	H|_{\sigma^0 T - [e'_0]} = y,
\qquad
	H|_{\sigma^0 C_1} = 
	x|_{C_1} \circ \sigma^0 = 
	y|_{C_1} \circ \sigma^0.
\]
Letting $\widetilde{H}\colon \sigma^0 T \to X$
be an extension of $H$,
the restriction $H|_{\sigma^0 T - [e_i]}$
provides the desired homotopy 
$x|_{D} \sim y|_{D}$, showing (i).


Lastly, let $x,f,g$ be as in (ii), 
and let
$K \colon \sigma^i C_1 \to X$ exhibit the relation
$f \sim_i x|_{C_1}$
and 
$ \bar{K} \colon \sigma^i C_2 \to X$
exhibit the relation
$x|_{C_2} \sim_i g$ (note the reversed order).
Now build the map
$H \colon \Lambda_o^{[e'_i]}[\sigma^i T] \to X$ by
\[
	H|_{\sigma^i T - [e_i]} = x,
\qquad
	H|_{\sigma^i C_1} = K,
\qquad
	H|_{\sigma^i C_2} = \bar{K}.
\]
Again letting 
$\widetilde{H} \colon \sigma^i T \to X$,
the restriction 
$\widetilde{H}|_{\sigma^i T - [e'_i]}$
provides the required $y \colon T \to X$,
showing (ii) and finishing the proof.
\end{proof}


\begin{corollary}
Let $X \in \mathsf{dSet}^G$ be a $G$-$\infty$-operad. Then
	\begin{itemize}
	\item[(a)] $\mathsf{ho}(X)\in \mathsf{dSet}_G$ is a genuine equivariant operad, i.e. it satisfies the strict right lifting condition against the Segal core inclusions
	$Sc[T] \to \Omega[T]$ for $T \in \Omega_G$;
	\item[(b)] the quotient map
	$\gamma_{\**}X \to \mathsf{ho}(X)$ is the universal map from $\gamma_{\**}X$ to a genuine equivariant operad.
	\end{itemize}
\end{corollary}

\begin{proof}
	Note first that by Remark \ref{HOMOTBOUND REM}
	any map 	$Sc[T] \to \mathsf{ho}(X)$ admits a factorization 
	$Sc[T] \to \gamma_{\**}X \xrightarrow{q} \mathsf{ho}(X)$.
	
	The right lifting property for $\mathsf{ho}(X)$
	against the maps $Sc[T] \to \Omega[T]$
	is then automatic from the lifting property for $X$.

	For strictness,	
	note that Definition \ref{XTENDSIM DEF}
	can be reinterpreted as saying that
	$x,y \colon \Omega[T] \rightrightarrows X$
	give rise to the same point of 
	$\mathsf{ho}(X)$, i.e. 
	the composites 
	$\Omega[T] \rightrightarrows X \xrightarrow{q}
	\mathsf{ho}(X)$ coincide, 
	iff the composites 
	$Sc[T] \to \Omega[T] \rightrightarrows X \xrightarrow{q}
	\mathsf{ho}(X)$ coincide, showing strictness, and thus (a).
		
	For (b), since $\mathsf{ho}(X)$ is a quotient of
	$\gamma_{\**} X$, it suffices to show that any map
	from $F \colon \gamma_{\**}X \to Y$ with $Y$ a genuine equivariant operad must also enforce the $\sim$ relation.
	For a $G$-corolla $C$ and
	$f,g\colon C \to X$ such that 
	$H \colon \sigma^i C \to X$ exhibits
	$f \sim_i g$, 
	the strict lifting condition for $Y$
	shows that the maps
	$F\circ H \colon \sigma^i C \to Y$,
	$f \circ \sigma^i \colon \sigma^i C \to Y$
	must coincide, and thus that
	$F(f)=F(g)$.
	The further claim that $F$ respects equivalences
	of general dendrices $x,y\colon T \rightrightarrows X$
	is immediate from Definition \ref{XTENDSIM DEF}.
\end{proof}


\newpage




\section{Scratchwork}

\subsection{Colored simplicial tensors and cotensors}



\[
\begin{tikzcd}
	K \otimes f^{\**} P \ar{r} \ar{ddd}&
	K \otimes f^{\**} \left( (K \otimes P)^K \right) \ar{r}{\simeq} \ar{d}&
	K \otimes \left( f^{\**} (K \otimes P) \right)^K \ar{r} \ar{d} &
	f^{\**} (K \otimes P) \ar{ddd}
\\
	&
	K \otimes f^{\**} \left( (L \otimes P)^K \right) \ar{d}
	\ar{r}{\simeq} &
	K \otimes \left( f^{\**} (L \otimes P) \right)^K
	\ar{d}
\\
	&
	L \otimes f^{\**} \left( (L \otimes P)^K \right)
	\ar{r}{\simeq} &
	L \otimes \left( f^{\**} (L \otimes P) \right)^K
\\
	L \otimes f^{\**} P \ar{r} &
	L \otimes f^{\**} \left( (L \otimes P)^L \right) \ar{r}{\simeq} \ar{u} &
	L \otimes \left( f^{\**} (L \otimes P) \right)^L \ar{r}&
	f^{\**} (L \otimes P)
\end{tikzcd}
\]



\[
\begin{tikzcd}
	K \otimes f^{\**} P \ar{r} \ar{d}&
	K \otimes f^{\**} \left( (K \otimes P)^K \right) \ar{r}{\simeq} &
	K \otimes \left( f^{\**} (K \otimes P) \right)^K \ar{r}  &
	f^{\**} (K \otimes P) \ar{ddd}
\\
	K \otimes \left( (L \otimes f^{\**} P) \right)^L \ar{d} &
	K \otimes \left( (L \otimes f^{\**} \left( (K \otimes P)^K \right)) \right)^L &
\\
	K \otimes \left( (L \otimes f^{\**} P) \right)^K \ar{d} &
	K \otimes \left( (L \otimes f^{\**} \left( (K \otimes P)^K \right)) \right)^K
\\
	L \otimes f^{\**} P \ar{r} &
	L \otimes f^{\**} \left( (L \otimes P)^L \right) \ar{r}{\simeq} &
	L \otimes \left( f^{\**} (L \otimes P) \right)^L \ar{r}&
	f^{\**} (L \otimes P)
\end{tikzcd}
\]




\[
\begin{tikzcd}
	f^{\**} P \ar{rr} \ar{rd} \ar{dd} & &
	f^{\**} \left( (K \otimes P)^K \right) \ar{d} &
	\left( f^{\**} (K \otimes P) \right)^K \ar{l}[swap]{\simeq} \ar{ddd}
\\
	&
	f^{\**} \left( (L \otimes P)^L \right) \ar{r} &
	f^{\**} \left( (L \otimes P)^K \right) 
\\
	\left( L \otimes f^{\**} P \right)^L \ar{r} \ar{d} &
	\left( f^{\**}( L \otimes P ) \right)^L \ar{u}{\simeq} \ar{rrd} 
\\
	\left( L \otimes f^{\**} P \right)^K \ar{rrr} &&&
	\left( f^{\**}( L \otimes P ) \right)^K \ar{uul}[swap]{\simeq}
\end{tikzcd}
\]


\newpage


\subsection{Semi-cofibrantly generated}


The following codifies a formal argument implicit in the proof of \cite[Thm. 7.19]{CM13b}.

\begin{definition}
Given a set $J$ of maps that admit the small object argument, we say that $X \in \mathcal{M}$ is \textit{$J$-fibrant} if $X \to \**$ has the right lifting property against maps in $J$.

Further, given $D$ a class of maps in $\mathcal{M}$,
we write $D_{J\text{-fib}} \subseteq D$ to denote 
the subclass of maps whose target is $J$-fibrant.
\end{definition}

\begin{lemma}\label{SEMICOF LEM}
	Let $\mathcal{M}$ be a model category with $(C,W,F)$
	the corresponding classes of cofibrations, weak equivalences and fibrations. 
	Further, $J$ be a set of maps admitting the small object argument and such that:
\begin{itemize}
	\item[(i)] $J \subseteq C \cap W$;
	\item[(ii)] 
	$\left(J^{\boxslash} \cap W \right)_{J\text{-fib}}
	\subseteq \left( F \cap W \right)_{J\text{-fib}}$.
\end{itemize}
Then one further has that:
\begin{itemize}
	\item[(a)]
	$\left(\prescript{\boxslash}{}{\left(J^{\boxslash}\right)}\right)_{J\text{-fib}}
	= 
	\left( C \cap W \right)_{J\text{-fib}}$;
	\item[(b)]
	$\left(J^{\boxslash} \right)_{J\text{-fib}}
	= F_{J\text{-fib}}$.
\end{itemize}
\end{lemma}

\begin{remark}
Rephrasing (b), one has that the fibrant objects of $\mathcal{M}$ are precisely the $J$-fibrant objects
and thus that the fibrations between fibrant objects are precisely the $J$-fibrations.
\end{remark}

\begin{proof}
	To check (a), recalling first that 
	$\prescript{\boxslash}{}{\left(J^{\boxslash}\right)}$
	is the saturation of $J$, one has that (i) in fact implies 
	$\prescript{\boxslash}{}{\left(J^{\boxslash}\right)}
		\subseteq C \cap W $.
	For the converse direction, given a trivial cofibration
	$A \to Y$ with $J$-fibrant target,
	form the factorization 
	$A \to X \to Y$ as a 
	$J$-cofibration followed by a $J$-fibration. 
	By the first direction the map $A\to X$ is a weak equivalence, and thus by 2-out-of-3 so is $X \to Y$.
	But then by (ii) the map $X \to Y$ is a trivial fibration, so that the lifting below exists,
	showing that $A \to Y$ is a retract of $A \to X$, and thus also in the saturation $\prescript{\boxslash}{}{\left(J^{\boxslash}\right)}$, 
	as desired.
\[
\begin{tikzcd}
	A \ar[>->]{r}{J} \ar[>->]{d}[swap]{\sim}&
	X \ar[->>]{d}{J}
%& &
%	A \ar[>->]{r}{\sim} \ar[>->]{d}[swap]{\sim}&
%	Y \ar[->>]{d}{\sim}
\\
	Y \ar[equal]{r} \ar[dashed]{ru} & Y
%& &
%	X \ar[equal]{r} \ar[dashed]{ru} & X
\end{tikzcd}
\]

To check (b), one direction is again immediate from (i),
since $J^{\boxslash} \supseteq (C \cap W)^{\boxslash} = F$.
For the converse direction, it suffices to show that 
a $J$-fibration $X\to Y$ with $J$-fibrant target has the right lifting property against trivial cofibrations, as on the left diagram below.
After factoring the bottom horizontal map as a $J$-cofibration followed by a $J$-fibration as on the right diagram, it suffices to shows that a lift $B' \to X$ exists.
But since $B'$ is $J$-fibrant, this follows from (a), which shows that the composite $A \to B \to B'$ is a $J$-cofibration.
\[
\begin{tikzcd}
	A \ar{r} \ar[>->]{d}[swap]{\sim}&
	X \ar[->>]{d}{J}
&&
	A \ar{rr} \ar[>->]{d}[swap]{\sim}&&
	X \ar[->>]{d}{J}
\\
	B \ar{r} \ar[dashed]{ru} & Y
&&
	B \ar[>->]{r}[swap]{J} &
	B' \ar[->>]{r}[swap]{J} \ar[dashed]{ru}
	& Y
\end{tikzcd}
\]
\end{proof}

\begin{remark}
	Analyzing the proof above, one is free to replace the class of fibrant objects with any other class that is compatible with $J$-fibrations, in the sense that if 
	$X \to Y$ is a $J$-fibration and $Y$ is in the class, then so is $X$.
\end{remark}





\subsection{Formalizing some stuff}

The following is a reformalized proof of \cite[Thm. 8.14]{CM13b}.


\begin{proposition}
The (right) derived composite functors in the following diagram commute up to a zigzag of weak equivalences. 
\[
\begin{tikzcd}
	\mathsf{PreOp} \ar{d}[swap]{\gamma^{\**}}&
	\mathsf{sOp} \ar{l}[swap]{N} \ar{d}{hcN}
\\
	\mathsf{sdSet} &
	\mathsf{dSet} \ar{l}{c_{!}}
\end{tikzcd}
\]
\end{proposition}

Note that though $\gamma^{\**}$ and $c_{!}$ are left Quillen, they both preserve all equivalences, 
so that one needs only perform fibrant replacements in 
$\mathsf{sOp}$.

\begin{proof}
	Recall that, given an object $X$ in a model category $\mathcal{M}$, a simplicial frame of $X$ is a fibrant replacement
	$c_!(X) \to \widetilde{X}(\bullet)$ of the constant 
	simplcial object $c_!(X)$ in the Reedy model structure on $\mathcal{M}^{\Delta^{op}}$.
	Moreover, if $X$ was already fibrant one is free to assume that $\widetilde{X}(0) = X$.
	
	Let $\mathcal{O} \in \mathsf{sOp}$ be fibrant, 
	choose a (functorial) fibrant simplicial frame
	$\widetilde{\mathcal{O}}(\bullet) \in \mathsf{sOp}^{\Delta^{op}}$, where we assume $\widetilde{\mathcal{O}} (0) = \mathcal{O}$.
	Next, let 
	$\gamma^{\**} N \widetilde{\mathcal{O}}(\bullet) 
	\to \widetilde{Q}(\bullet)$
	be a Reedy fibrant replacement in  
	$\mathsf{sdSet}^{\Delta^{op}}$.
	
	We claim that the following is a zigzag of weak equivalences in $\mathsf{sdSet}$.
\begin{equation}\label{BIGZIG EQ}
	\gamma^{\**} N \mathcal{O} \xrightarrow{\sim}
	\widetilde{Q}(0) \xrightarrow{\sim}
	\delta^{\**} \widetilde{Q} \xleftarrow{\sim}
	\widetilde{Q}_0 \xleftarrow{\sim}
	\left(\gamma^{\**} N \widetilde{\mathcal{O}}\right)_0
	\xrightarrow{\sim}
	hcN \widetilde{\mathcal{O}} \xleftarrow{\sim}
	c_{!} hcN \mathcal{O}
\end{equation}
That the first map is an equivalence is obvious from definition of $\widetilde{Q}$ and the assumption $\widetilde{\mathcal{O}}(0) = \mathcal{O}$.

For the second and third maps, note first that $\widetilde{Q}$ is homotopically constant, in the sense that all structure maps $\widetilde{Q}(m) \to \widetilde{Q}(m')$
are equivalences.
Moreover, since the levels $\widetilde{Q}$ are fibrant in 
$\mathsf{sdSet}$, this implies that these are simplicial equivalences, i.e. that for each tree $T \in \Omega$
the evaluations 
$\widetilde{Q}(T)(m) \to \widetilde{Q}(T)(m')$
are Kan equivalences in $\mathsf{sSet}$.
But since $\widetilde{Q}(T) \in \mathsf{sSet}^{\Delta^{op}}$ is itself Reedy fibrant, this shows that it is in fact joint Reedy fibrant, so that one has Kan equivalences 
$\widetilde{Q}(T)(0) \xrightarrow{\sim}
\delta^{\**} \widetilde{Q}(T) \xleftarrow{\sim}
\widetilde{Q}_0(T)$, showing that the second and third maps in \eqref{BIGZIG EQ} are indeed weak equivalences.

For the fourth equivalence, note that one can write
\[\widetilde{Q}_0(T) = 
\mathsf{Hom}_{\mathsf{sdSet}}(\Omega[T],\widetilde{Q})=
\mathsf{Hom}_{\mathsf{PreOp}}(\Omega[T],\gamma_{\**}\widetilde{Q})\]
\[
\left(\gamma^{\**} N \widetilde{\mathcal{O}}\right)_0(T) = 
\mathsf{Hom}_{\mathsf{sdSet}}(\Omega[T],\gamma^{\**} N \widetilde{\mathcal{O}})=
\mathsf{Hom}_{\mathsf{PreOp}}(\Omega[T], N \widetilde{\mathcal{O}})
\]
The claim now follows by noting that
$N \mathcal{O} \to \gamma_{\**} \widetilde{Q}$
is an equivalence of Reedy fibrant objects on 
$\mathsf{PreOp}^{\Delta^{op}}$ (over the tame model structure) and that $\Omega(T)$ is tame cofibrant. 

For the fifth equivalence, note that
\[
\left(\gamma^{\**} N \widetilde{\mathcal{O}}\right)_0(T) = 
\mathsf{Hom}_{\mathsf{PreOp}}(\Omega[T], N \widetilde{\mathcal{O}}) =
\mathsf{Hom}_{\mathsf{sOp}}(\Omega(T),  \widetilde{\mathcal{O}})
\]
\[
\left(hcN \widetilde{\mathcal{O}} \right)(T) = 
\mathsf{Hom}_{\mathsf{sOp}}(W_!(T),  \widetilde{\mathcal{O}})
\]
so that the required claim follows since 
$\widetilde{\mathcal{O}}$ is Reedy fibrant and
$W_!(T) \to \Omega(T)$ is a weak equivalence of cofibrant operads.

Lastly, for the last map, one needs simply note that
$c_! hcN \mathcal{O} = hcN c_! \mathcal{O}$, so that the required claim follows since 
$c_! \mathcal{O} \to \widetilde{O}$
is a levelwise equivalence of levelwise fibrant operads
and $hcN$ is right Quillen.
\end{proof}




\begin{lemma} \label{INTER_LEM}
Let $\mathcal{O} \in \mathsf{sOp}$ and let
$g \colon x \to y$ be an equivalence in $\mathcal{O}$.

Then there exists a countable, cofibrant and contractile $H \in \mathsf{sOp}_{\{0,1\}}$ 
and a map 
$\varphi \colon H \to \mathcal{O}$
such that 
$g$ is in the image of $\varphi$. 
\end{lemma}


\begin{proof}
	We start by considering the case where $\mathcal{O}$ is locally fibrant.
	
	$g$ can be regarded as a map
	$[1] \xrightarrow{g} \mathcal{O}$,
	and one thus likewise gets a map
	$\Delta[1] \xrightarrow{g}  hcN \mathcal{O}$
	which is an equivalence in the 
	$\infty$-category $hcN \mathcal{O}$,
	so that one can find a (non-unique) factorization
	$\Delta[1] \to J \to hcN \mathcal{O}$
	which by adjunction yields a factorization
	$[1]=W_!\Delta[1] \to W_! J \to \mathcal{O}$,
	which establishes the desired claim 
	since $W_! J$ is contractible due to 
	Example \ref{WJ EX}.
	
	For a general $\mathcal{O}$, 
	consider first a local fibrant replacement
	$F \colon \mathcal{O} \to \mathcal{O}'$.
	One can hence find a map 
	$W_! J \to \mathcal{O}'$ such that
	$F(g)$ is in its image. 
	We now factor this map as
	$W_! J \xrightarrow{\sim} H \to \mathcal{O}'$
	where the second map is a local fibration.
	
	One can now form a pullback
\[
\begin{tikzcd}
	\tilde{H} \ar{r} \ar{d} & H' \ar{d}
\\
	\mathcal{O} \ar{r} & \mathcal{O}'
\end{tikzcd}
\]
where $\tilde{H}$ is seen to be contractible since
$\mathsf{sSet}$ is right proper.
	A priori, $\tilde{H}$ will need not be countable nor cofibrant, but this is easily rectified:
	indeed one can show that any countable subcomplex of $\tilde{H}$ is contained in a contractible countable subomplex, 
	yielding a countable contractible subcomplex whose image in $\mathcal{O}$ contains $g$. Lastly, performing a cofibrant replacement of that complex finishes the proof.
\end{proof}





\begin{example}\label{WJ EX}
	Let $J = N \widetilde{[1]}$ be the nerve of the contractible groupoid on two objects.
	
	Then there is an identification
\[
	W_{!} J \simeq \mathbb{F}^{\bullet} \widetilde{[1]}
\]
	where $\mathbb{F}$ denotes the (unital) free operad monad.

	To see this, we start by describing
	$\mathbb{F}^{\bullet} \widetilde{[1]}$.
	Writing $f \colon 0 \to 1$ and 
	$g \colon  1\to 0$ for the non-identity arrows in 
	$\widetilde{[1]}$ (so that $g=f^{-1}$), the $0$-simplices of $\mathbb{F}^{\bullet} \widetilde{[1]}$
	are the alternating words
	$f,g,fg,gf,fgf,gfg,fgfg,gfgf,\cdots$
	in the letters $f$, $g$.
	More generally
	$n$-simplices are given by equipping such alternating words with ``$n$ nested layers of brackets''
	(so that, for example, 
	$\left((f)(gf)\right) 
	\left( (gf) \right)$
	encodes a $2$-simplex).
	Alternatively, given an alternating word of length $l$, such bracketings are encoded by a flag of subsets
	$F_1 \subseteq F_2 \subseteq \cdots 
	\subseteq F_n \subseteq \{1,\cdots,l-1\}$.
	
	To describe $W_{!} J$, we apply the explicit description of the $W_{!} (-)$ construction given in 
	\cite{DS11}.
	Following \cite[Cor. 4.8]{DS11}, the $n$-simplices of $W_{!} J$ are uniquely encoded by a map
\begin{equation}\label{NECMAP EQ}
	N = \Delta^{k_1} \vee \Delta^{k_2} 
	\vee \cdots \vee \Delta^{k_r} 
	\to J 
\end{equation}
which is totally nondegenerate (this means that all simplices $\Delta^{k_i} \to J$ are nondegenerate)
together with a flag of subsets
	$\boldsymbol{J}(N) = 
	G_0 \subseteq 
	G_1 \subseteq \cdots \subseteq
	G_{n-1} \subseteq \boldsymbol{E}^{\mathsf{i}}(N)$.
	Noting that the nondegenerate simplices of $J$ (other than the points $0,1$)
	are themselves identified with alternating words
	$f,g,fg,gf,fgf,gfg,fgfg,gfgf,\cdots$,
	one sees that so is the map \eqref{NECMAP EQ}.
	Therefore, we see that a $n$-simplex of 
	$W_{!} J$ is uniquely, determined by some alternating word of some size $l$ together with a flag  
	$G_0 \subseteq 
	G_1 \subseteq \cdots \subseteq
	G_{n-1} \subseteq \boldsymbol{E}^{\mathsf{i}}(N)
	=\{1,\cdots,l-1\}$, since 
	$G_0$ suffices to recover the domain of 
	\eqref{NECMAP EQ}.
	
	This shows that $W_{!} J$ $\mathbb{F}^{\bullet} \widetilde{[1]}$ indeed have the same simplices.
	The fact that the simplicial operators coincide can be readily checked explicitly with the most interesting case is that of the top differential $d_n$, which in either case is induced by multiplication in 
	$\widetilde{[1]}$ 
	(that this is the case for $W_{!} J$ follows from the description of the simplicial operators in 
	\cite[Cor. 4.4]{DS11} together with the description of the ``flanking'' procedure in \cite[Lemma 4.5]{DS11}).
\end{example}




\subsection{Extra lifts for $\infty$-categories}


\begin{lemma}
The inclusion 
\[[0,1,2] \cup [0,2,3,\cdots,n] \cup 
\Lambda^0[0,1,3,\cdots,n]
 \to \Lambda^{0,2}[n]\]
is built cellularly from inclusions
$\Lambda^0[k] \to \Delta[k]$ with $k<n$.
Moreover, all such cells send $[0,1]$ to $[0,1]$.
\end{lemma}

\begin{proof}
Since $[0,2,3,\cdots,n]$ is in the domain, all missing faces must contain $1$.
Moreover, since the smallest face not containing $2$
that is not in $\Lambda^0[0,1,3,\cdots,n]$ is $[1,3,\cdots,n]$,
which is also the smallest face not in $\Lambda^{0,2}[n]$,
we see that all missing faces must contain $2$ as well.

It now suffices to check that $0$ is characteristic with respect to the missing faces, i.e.
that $12\underline{a}$ is missing iff $012\underline{a}$ is missing, and this is now obvious. 
\end{proof}

\begin{remark}
	The map $\Lambda^{0,2}[n] \to \Delta[n]$ is inner anodyne ($2$ is characteristic).
	
	This observation, together with the precious lemma, are the technical core of the observation that lifts
\[
	\begin{tikzcd}
	\Lambda^{0}[n] \ar{d}  \ar{r} & X
\\
	\Delta[n] \ar[dashed]{ru}
	\end{tikzcd}
\]
exist when $X$ is an $\infty$-category and $[0,1]$ is mapped to an equivalence in $X$.
\end{remark}




\subsection{TBD}


\begin{lemma}
Suppose that a subcategory $\Xi$ has subcategories 
$\Xi^-,\Xi^+$ which contain the isomorphisms and satisfy the unique factorization up to unique isomorphism axiom.

Write $\mathsf{Arr}(\Xi)$ for the arrow category of $\Xi$ and 
$\mathsf{Arr}^{-}(\Xi), \mathsf{Arr}^{+}(\Xi)$
for the full subcategories whose objects are arrows in $\Xi^-,\Xi^+$. 
Then $\mathsf{Arr}^{-}(\Xi)$ (resp. $\mathsf{Arr}^{+}(\Xi)$)
is initial (resp. terminal) in $\mathsf{Arr}(\Xi)$.
\end{lemma}



\begin{proof}
Given $f \in \mathsf{Arr}(\Xi)$, we need to show that there exist diagrams as on the left, and moreover that all such diagrams are connected. 
Existence is immediate from the factorization assumption . Moreover, its is straightforward from the ``uniqueness up to isomorphism'' that all connections are connected.
But by factoring the left vertical map in the diagram below, we now see that all such diagrams are connected to a factorization.
\[
\begin{tikzcd}
	\bullet \ar{r}{f} \ar{d}& 
	\bullet \ar{d}
\\
	\bullet \ar{r}[swap]{+} &
	\bullet
\end{tikzcd}
\]
%
%\[
%\begin{tikzcd}
%	\bullet \ar{r}{-} \ar{dd} \ar{rd}[swap]{-} &
%	\bullet \ar{rrr}{+} \ar{rrd}{-} &&&
%	\bullet \ar{dd}
%\\
%	&
%	\bullet \ar{rrd}[swap]{+}  && 
%	\bullet \ar{rd}{+} \ar[dashed]{ll}[swap]{\simeq}
%\\
%	\bullet 	\ar{rrr}[swap]{-} &&&
%	\bullet \ar{r}[swap]{+} & 
%	\bullet
%\end{tikzcd}
%\]
\end{proof}



\begin{lemma}\label{REDUCELAN LEM}
	Suppose that $F \colon \mathcal{C} \to \mathcal{D}$ is a functor in 
	$\mathsf{Cat}^G$.
Then the following square commutes up to natural isomorphism
\[
\begin{tikzcd}[column sep=50pt]
	\mathcal{V}^{G \ltimes \mathcal{C}} 
	\ar{r}{\mathsf{Lan}_{G \ltimes \mathcal{C} \to G \ltimes \mathcal{D}}} \ar{d}[swap]{\mathsf{fgt}}&
	\mathcal{V}^{G \ltimes \mathcal{D}} \ar{d}{\mathsf{fgt}}
\\
	\mathcal{V}^{\mathcal{C}} 
	\ar{r}[swap]{\mathsf{Lan}_{\mathcal{C} \to\mathcal{D}}} &
	\mathcal{V}^{\mathcal{D}}
\end{tikzcd}
\]
\end{lemma}


\begin{proof}
For each $d \in \mathcal{D}$ (recall that $\mathcal{D}$ and $G \ltimes \mathcal{D}$ have the same objects) one has an obvious inclusion 
$\mathcal{C} \downarrow d \to G\ltimes \mathcal{C} \downarrow d$.
Moreover, for each object of $G\ltimes \mathcal{C} \downarrow d$
there is a unique $g \in G$ such that the object is described as a composite
$F(c) \xrightarrow{g} g F(c) \to d$,
were $g F(c) \to d$ can be regarded as an object of $\mathcal{C} \downarrow d$.
One thus has a retraction 
$G \ltimes \mathcal{C} \downarrow d \to \mathcal{C} \downarrow d$
showing that $\mathcal{C} \downarrow d$ is terminal in
$G \ltimes \mathcal{C} \downarrow d$
and finishing the proof. 
\end{proof}



\subsection{Fixed color lemmas}

Given a set $\mathfrak{C}$ of colors,
write $\Sigma_{\mathfrak{C}}$ for the groupoid of corollas with edges labeled by colors in $\mathfrak{C}$.

If, in addition, $\mathfrak{C}$ is a $G$-set, 
we write $G \ltimes \Sigma_{\mathfrak{C}}^{op}$ for the larger groupoid obtained via the associated Grothendieck construction.


Writing
$\mathsf{Sym}^{G,\mathfrak{C}} = 
\mathsf{Set}^{G \ltimes \Sigma_{\mathfrak{C}}^{op}}$,
we have a natural monad $\mathbb{F}$ on
$\mathsf{Sym}^{G,\mathfrak{C}}$
whose algebra category we denote by 
$\mathsf{Op}^{G,\mathfrak{C}}$.


\begin{notation}
	Given a $G$-equivariant function 
	$f \colon \mathfrak{C} \to \mathfrak{D}$
	we write
\[
	f_{\**} \colon 
	\mathsf{Sym}^{G,\mathfrak{C}}
	\rightleftarrows
	\mathsf{Sym}^{G,\mathfrak{D}}
	\colon f^{\**}
\qquad
	\check{f}_{\**} \colon 
	\mathsf{Op}^{G,\mathfrak{C}}
	\rightleftarrows
	\mathsf{Op}^{G,\mathfrak{D}}
	\colon f^{\**}
\]
for the standard adjunctions (note the need to distinguish notations for the left adjoints).
\end{notation}


\begin{example}\label{GCORMPA EX}
Given a $G$-corolla $C \in \Sigma_G$, we write $\partial C$ for the set of edges of $C$, which is naturally identified with the set of objects of the associated $G$-operad
$\Omega(C) \in \mathsf{Op}^G$.

One can then regard $\Omega(C) \in \mathsf{Op}^{G,\partial C}$ and, moreover, $\Omega(C)$ is in fact the free operad over the symmetric sequence obtained by removing the units of $\Omega(C)$,
which we denote by
$\Omega'(C) \in \mathsf{Sym}^{G,\partial C}$.

Given the non-equivariant decomposition
$C = C_1 \amalg \cdots \amalg C_k$
with $C_i \in \Sigma$, 
one can naturally regard the $C_i$ as objects of $\Sigma_{\partial C}$.
In fact, one then has an identification
\begin{equation}\label{SOMEIDEN EQ}
	\Omega'(C) \simeq 
	\Sigma_{\partial C}[C_1] \amalg \cdots \amalg \Sigma_{\partial C}[C_k]
\end{equation}
where $\Sigma_{\partial C}[-]$ denotes the representable presheaf in 
$\mathsf{Set}^{\Sigma_{\partial C}^{op}}$.
This claim requires some justification, since a priori the right hand side of \eqref{SOMEIDEN EQ} is an object in $\mathsf{Set}^{\Sigma_{\partial C}^{op}}$,
rather than in $\mathsf{Set}^{G \ltimes \Sigma_{\partial C}^{op}}$,
i.e. we need to describe the action of the additional action arrows
$D \xrightarrow{g} gD$ on this presheaf.
This action is given by the following diagram, where the vertical $g$ arrows simply act on labels, and all horizontal arrows are shuffle arrows (i.e. arrows in $\Sigma_{\partial C}$).
The diagonal $C_g$ arrow corresponds to the structural $G$-action on $C$. It is then straightforward to check that there is a unique dashed shuffle $\tau_g$ as indicated
\[
\begin{tikzcd}
	D \ar{d}[swap]{g} \ar{r}{\sigma} & C_i \ar{rd}{C_g} \ar{d}[swap]{g}
\\
	g D \ar{r}[swap]{g \sigma} & g C_i \ar[dashed]{r}{\simeq}[swap]{\tau_g} & C_{g i}
\end{tikzcd}
\]
and one defines $g_{\**}\colon \Sigma_{\partial C}[C_i] \to \Sigma_{\partial C}[C_{gi}]$
via $\sigma \mapsto \tau_g \circ (g \sigma)$.

Moreover, letting $f \colon \partial C \to \mathfrak{C}$
be a map of colors, 
one obtains $\mathfrak{C}$-corollas $C_i^{f} \in \Sigma_{\mathfrak{C}}$
by coloring each edge $e\in \partial C$ by $f(e) \in \mathfrak{C}$, resulting in a generalized identification 
\begin{equation}\label{SOMEIDENGEN EQ}
	f_{\**} \Omega'(C) \simeq 
	\Sigma_{\mathfrak{C}}[C_1^f] \amalg \cdots \amalg \Sigma_{\mathfrak{C}}[C^f_k]
\end{equation}
Indeed, \eqref{SOMEIDENGEN EQ} follows from the observation
that the Kan extension 
$\mathsf{Lan}_{G \ltimes \Sigma_{\partial C} \to G \ltimes \Sigma_{\mathfrak{C}}}$ 
coincides, after forgetting with the $G$-action arrows,
with the Kan extension
$\mathsf{Lan}_{\Sigma_{\partial C} \to \Sigma_{\mathfrak{C}}}$
(cf. Lemma \ref{REDUCELAN LEM}).
\end{example}


\begin{definition}
	Given a $\mathfrak{C}$-corolla $C$, 
	a subgroup 
	$\Gamma \leq \mathsf{Aut}_{G \ltimes \Sigma_{\mathfrak{C}}^{op}}(C)$
	is called a \textit{$G$-graph subgroup} if
	$\Gamma \cap \mathsf{Aut}_{\Sigma_{\mathfrak{C}}^{op}}(C) = \**$.
	
	We write $\mathcal{F}^{\Gamma} = \{\mathcal{F}^{\Gamma}_C\}$
	for the collection of families of $G$-graph subgroups.
	
	A $G$-$\mathfrak{C}$-symmetric sequence
	$X \in \mathsf{Sym}^{G,\mathfrak{C}}$
	is called $\Sigma$-cofibrant if each level
	$X(C)$ is $\mathcal{F}^{\Gamma}_C$-cofibrant.
\end{definition}


\begin{remark}
	Write a $\mathfrak{C}$-corolla as $C^f \in \Sigma_{\mathfrak{C}}$,
	where $C \in \Sigma$ is the underlying corolla and
	$f\colon \partial C \to \mathfrak{C}$
	is the coloring.
	A $G$-graph subgroup 
	$\Gamma \leq \mathsf{Aut}_{G \ltimes \Sigma_{\mathfrak{C}}^{op}}(C^f)$ is, under the map 
	$G \ltimes \Sigma_{\mathfrak{C}}^{op} \to
	G \times \Sigma^{op}$,
	identified with a 
	$G$-graph subgroup of 
	$G \times \mathsf{Aut}_{\Sigma^{op}}(C)$,
	i.e., with the graph of a partial antihomomorphism
\[
	G^{op} \geq H^{op} \xrightarrow{(-)^{-1}} H 
	\xrightarrow{\tau_{(-)}} \mathsf{Aut}_{\Sigma^{op}}(C)
\]
	which is subject to the requirement
\[
	f(\tau_h(e)) = h f(e).
\]
Using the $\tau$ automorphisms one can then
\begin{inparaenum}
\item[(i)] equip $C$ with a $H$-action,
so that one can regard $C \in \Sigma^H \subseteq \Sigma_H$;
\item[(ii)] extend $\Sigma_{\mathfrak{C}}[C^f]$ to 
an object in $\mathsf{Set}^{H \ltimes \Sigma_{\mathfrak{C}}}$
by defining the action of the $H$-action arrows $h$ via 
$\sigma \mapsto \tau_{h} \circ (h \sigma)$;
\item[(iii)]
following Example \ref{GCORMPA EX}, one thus has an identification
\[
	f_{\**} \Omega'(C) \simeq \Sigma_{\mathfrak{C}}[C^f]
\]
of objects in $\mathsf{Set}^{H \ltimes \Sigma_{\mathfrak{C}}}$ and therefore an identification 
\[
	f_{\**} \left( G \cdot_H \Omega'(C) \right) \simeq 
	G\cdot_H \Sigma_{\mathfrak{C}}[C^f]
\]
of objects in $\mathsf{Set}^{G \ltimes \Sigma_{\mathfrak{C}}}$.
\end{inparaenum}
\end{remark}


\begin{example}
Let $G = \mathbb{Z}_{/2} = \{\pm 1\}$ and 
$\mathfrak{C} = \{\mathfrak{a}, -\mathfrak{a}, \mathfrak{b}\}$ where we implicitly have
$-\mathfrak{b} = \mathfrak{b}$.
Consider the two $\mathfrak{C}$-corollas 
$C,D \in \Sigma_{\mathfrak{C}}$ below.
\begin{equation}
	\begin{tikzpicture}[auto,grow=up, level distance = 2.2em,
	every node/.style={font=\scriptsize,inner sep = 2pt}]%
		\tikzstyle{level 2}=[sibling distance=3em]%
			\node at (0,0) [font = \normalsize] {$C$}%	
				child{node [dummy] {}%
					child{node {}%
					edge from parent node [swap] {$-\mathfrak{a}$}}%
					child[level distance = 2.9em]{node {}%
					edge from parent node [swap,	near end] {$\mathfrak{b}$}}%
					child[level distance = 2.9em]{node {}%
					edge from parent node [near end] {$\mathfrak{b}$}}%
					child{node {}%
					edge from parent node  {$\mathfrak{a}$}}%
				edge from parent node [swap] {$\mathfrak{b}$}};%
			\node at (7,0) [font = \normalsize] {$D$}%	
				child{node [dummy] {}%
					child{node {}%
					edge from parent node [swap] {$-\mathfrak{a}$}}%
					child[level distance = 2.9em]{node {}%
					edge from parent node [swap,	near end] {$-\mathfrak{a}$}}%
					child[level distance = 2.9em]{node {}%
					edge from parent node [near end] {$\mathfrak{a}$}}%
					child{node {}%
					edge from parent node  {$\mathfrak{a}$}}%
				edge from parent node [swap] {$\mathfrak{b}$}};%
	\end{tikzpicture}%
\end{equation}%
Each of $C,D$ admit exactly two non-trivial $G$-graph subgroups,
which are encoded by the $\mathbb{Z}_{/2}$-actions on the underlying corollas depicted below.
\begin{equation}
	\begin{tikzpicture}[auto,grow=up, level distance = 2.2em,
	every node/.style={font=\scriptsize,inner sep = 2pt}]%
		\tikzstyle{level 2}=[sibling distance=3em]%
			\node at (-1.6,0) [font = \normalsize] {$C_1$}%	
				child{node [dummy] {}%
					child{node {}%
					edge from parent node [swap] {$-a$}}%
					child[level distance = 2.9em]{node {}%
					edge from parent node [swap,	near end] {$c\phantom{b}$}}%
					child[level distance = 2.9em]{node {}%
					edge from parent node [near end] {$b$}}%
					child{node {}%
					edge from parent node  {$a$}}%
				edge from parent node [swap] {$r$}};%
			\node at (1.6,0) [font = \normalsize] {$C_2$}%	
				child{node [dummy] {}%
					child{node {}%
					edge from parent node [swap] {$-a$}}%
					child[level distance = 2.9em]{node {}%
					edge from parent node [swap,	near end] {$-b$}}%
					child[level distance = 2.9em]{node {}%
					edge from parent node [near end] {$b$}}%
					child{node {}%
					edge from parent node  {$a$}}%
				edge from parent node [swap] {$r$}};%
			\node at (5.4,0) [font = \normalsize] {$D_1$}%	
				child{node [dummy] {}%
					child{node {}%
					edge from parent node [swap] {$-a$}}%
					child[level distance = 2.9em]{node {}%
					edge from parent node [swap,	near end] {$-b$}}%
					child[level distance = 2.9em]{node {}%
					edge from parent node [near end] {$b$}}%
					child{node {}%
					edge from parent node  {$a$}}%
				edge from parent node [swap] {$r$}};%
			\node at (8.6,0) [font = \normalsize] {$D_2$}%	
				child{node [dummy] {}%
					child{node {}%
					edge from parent node [swap] {$-b$}}%
					child[level distance = 2.9em]{node {}%
					edge from parent node [swap,	near end] {$-a$}}%
					child[level distance = 2.9em]{node {}%
					edge from parent node [near end] {$b$}}%
					child{node {}%
					edge from parent node  {$a$}}%
				edge from parent node [swap] {$r$}};%
	\end{tikzpicture}%
\end{equation}%

\end{example}


The following is the analogue of \cite[Prop. 3.2]{CM13b}

\begin{proposition}\label{KEYPR PROP}
Suppose that $\mathcal{O} \in \mathsf{Op}^{G,\mathfrak{C}}$
is $\Sigma$-cofibrant.
Further, let $C \in \Sigma_G$ be any $G$-corolla and consider 
a pushout in $\mathsf{Op}^{G}$ of the form
\begin{equation}\label{PUSHOUTPROP EQ}
\begin{tikzcd}
	\partial \Omega(C) \ar{r} \ar{d} & \mathcal{O} \ar{d}
\\
	\Omega(C) \ar{r} & \mathcal{P}.
\end{tikzcd}
\end{equation}
Then the induced map
\begin{equation}\label{ANODYNE MAP}
	\Omega[C] \amalg_{\partial \Omega[C]} N\mathcal{O} \to N\mathcal{P}
\end{equation}
is $G$-inner anodyne.
\end{proposition}

\begin{proof}
The desired claim that \eqref{ANODYNE MAP}
is $G$-inner anodyne will follow by applying the  
\textit{characteristic edge lemma} \cite[Lemma 3.4]{BP18},
but we first need some preliminary discussion. 

Let us write $f \colon \partial C \to \mathfrak{C}$
for the induced map of colors.
The first step is to rewrite \eqref{PUSHOUTPROP EQ} as a pushout diagram in $\mathsf{Op}^{G,\mathfrak{C}}$, which can be done by applying $\check{f}_{\**}$
to the leftmost objects in \eqref{PUSHOUTPROP EQ}.
Since
\[
	\check{f}_{\**} \Omega(C) \simeq 
	\check{f}_{\**} \left( \mathbb{F} \Omega'(C) \right) \simeq 
	\mathbb{F} \left(f_{\**}  \Omega'(C) \right)
\]
one has that, writing $C \simeq G \cdot_H C_{\star}$ one has the alternative pushout in $\mathsf{Op}^{G,\mathfrak{C}}$
\begin{equation}
\begin{tikzcd}
	\mathbb{F} ( \emptyset ) \ar{r} \ar{d} & \mathcal{O} \ar{d}
\\
	\mathbb{F} \left( 
	G \cdot_H \Sigma_{\mathfrak{C}}[C^f_{\star}] \right) \ar{r} & \mathcal{P}.
\end{tikzcd}
\end{equation}
Writing $B = G \cdot_H \Sigma_{\mathfrak{C}}[C^f_{\star}]$, one then has
\begin{equation}\label{PUSHOPPR EQ}
	\mathcal{P}(C) = 
	\coprod_{
	[T] \in \mathsf{Iso}
	\left( \Omega_{\mathfrak{C}}^a \downarrow_{\mathsf{r}} C \right)
	}
	\left(
		\prod_{v \in V^{ac}(T)} \mathcal{O}(T_v)
	\times
		\prod_{v \in V^{in}(T)} B(T_v)
	\right)
	\cdot_{\mathsf{Aut}_{\Omega^a_{\mathfrak{C}}}(T)} \mathsf{Aut}_{\Sigma_{\mathfrak{C}}}(C)
\end{equation}

We now discuss the dendrices of $N \mathcal{P}$. Firstly, recall that, by the strict Segal condition characterization of nerves \cite[Cor. 2.7]{CM13a},
a dendrex $\Omega[T] \to N \mathcal{P}$
is uniquely specified by the tree $T \in \Omega$ together with a choice of operations
$\{p_v \in \mathcal{P}(T_v)\}_{v \in \boldsymbol{V}(T)}$.
Noting that \eqref{PUSHOPPR EQ} implies that the canonical map (of sequences) 
$\mathcal{O} \amalg B \to \mathcal{P}$
is a monomorphism, 
we will say a dendrex $(T,\{p_v\})$ is \textit{elementary}
if all operations $p_v$ are in $\mathcal{O} \amalg B$.
Additionally, an elementary dendrex $(T,\{p_v\})$ is called \textit{alternating} if $T$ is an alternating tree and 
$p_v$ is in $\mathcal{O}$ (resp. $B$) if
$v \in \boldsymbol{V}(T)$ is active (resp. inert).

Given an elementary dendrex $(T,\{p_v\})$ and a map of trees 
$\varphi \colon S \to T$
we will need to know when 
$\varphi^{\**}(T,\{p_v\})$
is again elementary.
Since all maps in $\Omega$ are, uniquely up to isomorphism,
factored as a degeneracy followed by an inner face followed by an outer face, it suffices to discuss each of those cases.
It is straightforward to check that 
$\varphi^{\**}(T,\{p_v\})$ 
is elementary whenever $\varphi$ is a degeneracy or an outer face.
For an inner face
$\varphi \colon T-D \to T$,
noting that a partial composite $p \circ_i q$ of non-unit operations is in $\mathcal{O} \amalg B$ iff both operations are in $\mathcal{O}$,
one sees that $\varphi^{\**}(T,\{p_v\})$ is elementary iff
for each $d \in D$ the adjacent vertices of $T$ are either both labeled by operations of $\mathcal{P}$ or one of them is labeled by an identity.
An elementary dendrex is called \textit{reduced} if it has no such edges. 
In other words, an elementary dendrex is reduced iff none of its inner faces are reduced, so that, in particular, all elementary dendrices admit at least one reduced inner face (note that specifying such a face is somewhat subtle: even when a dendrex is non-degenerate, it may not be enough to collapse edges connecting $\mathcal{O}$ vertices, since this may possibly introduce new identity vertices, resulting in a degenerate vertex).
Note that reduced dendrices are necessarily non-degenerate, but not vice versa.
In fact, a non-degenerate dendrex is reduced iff it has a degeneracy which is an alternating dendrex. 


In what follows, we will find it convenient to work with elementary dendrices that have been suitably ``planarized''.
To do so, fix a subset
$\mathcal{O}^{\mathsf{st}} \amalg B^{\mathsf{st}} 
\subset
\coprod_{C \in \Sigma_{\mathfrak{C}}}
\mathcal{O}(C) \amalg B(C)$
of coset representatives for the $\Sigma$-action,
which we call \textit{standard} representatives.
An elementary dendrex $(T,\{p_v\})$ is then called \textit{standard}
if all $p_v$ are standard (i.e. in 
$\mathcal{O}^{\mathsf{st}} \amalg B^{\mathsf{st}}$).
Moreover, since both $\mathcal{O}$ and $B$ are $\Sigma$-cofibrant/$\Sigma$-free sequences (the former by assumption),
for each elementary simplex $(T,\{p_v\})$,
there is a unique (replanarization) isomorphism
$\varphi \colon T' \to T$ such that
$(T',\{\varphi^{\**}p_v\})$ is standard,
and we write 
$\mathsf{st}(T,\{p_v\}) = \varphi^{\**} (T,\{p_v\}) = (T',\{\varphi^{\**}_v p_v\})$
to denote this.


Note that it now follows from \eqref{PUSHOPPR EQ} that,
for each operation $p \in \mathcal{P}(C)$,
there exists a unique standard alternating dendrex
$b'_p \colon \Omega[T'_p] \to N \mathcal{P}$ and isomorphism 
$C \simeq T'_p - \boldsymbol{E}^{\mathsf{i}}(T'_p)$
such that the composite
\begin{equation}\label{STANDELDE EQ}
	\Omega[C] \simeq
	\Omega[T'_p - \boldsymbol{E}^{\mathsf{i}}(T'_p)] \to 
	\Omega[T'_p] \xrightarrow{b'_p}
	N \mathcal{P}
\qquad
	\Omega[C] \simeq
	\Omega[T_p - \boldsymbol{E}^{\mathsf{i}}(T_p)] \to 
	\Omega[T_p] \xrightarrow{b_p}
	N \mathcal{P}
\end{equation}
is $p$. In fact, due to the correspondence between alternating elementary dendrices and reduced elementary dendrices, 
the analogous claim  also holds 
for the corresponding non-degenerate dendrex 
$b_p \colon \Omega[T_p] \to N \mathcal{P}$.


We can now finally discuss how to apply \cite[Lemma 3.4]{BP18}.

Firstly, we need to identify a $G$-poset $I$ and dendrices 
$b_i \colon \Omega[U_i] \to N \mathcal{P}$ for $i \in I$.
Firstly, the underlying set of $I$ is the set of 
non-degenerate standard dendrices of $\mathcal{P}$,
which we abbreviate as
$i = (U_i,\{p_v^i\})$.
The dendrex $b_i \colon \Omega[U_i] \to N \mathcal{P}$ is then tautological, being $i$ itself, but it will preferable to use distinct notations for $i \in I$ and
$b_i \in N \mathcal{P} (U_i)$.
Given $i,j \in I$, we write $i \leq j$ if exists a (in general not  planar) face map
$\varphi \colon U_i \to U_j$
such that $b_i = \varphi^{\**}(b_j)$.
Note that by the uniqueness of standardizations $\varphi$ can only be an isomorphism if $i=j$, showing that $\leq$ indeed satisfies anti-symmetry.
Lastly, we define the $G$-action on $I$ via
\[b_{g i} = \mathsf{st} (g b_i).\]
When reading this formula, note that
$g b_i \in \mathcal{P}(U_i)$
(since this uses the $G$-action
on $\mathcal{P}$),
while $b_{g i} \in \mathcal{P}(U_{g i})$,
where $U_{g i}$ comes with the unique isomorphism
$U_{g i} \xrightarrow{g^{-1}} U_i$ which standardizes $b g_i$.

Lastly, the characteristic edge sets 
$\Xi^i \subseteq \boldsymbol{E}^{\mathsf{i}}(U_i)$ consist of those inner edges such that at least one of the adjacent vertices is mapped to an operation in $B$. Note that, by the discussion above, for an inner face of $\varphi \colon U_i - D \to U_i$
the dendrex $\varphi^{\**}(b_i)$ is elementary iff
$D \cap \Xi^i = \emptyset$.

We now note that it is in fact 
$N \mathcal{P} = 
A \cup \bigcup_{i \in I} b_i\left(\Omega[U_i]\right)$ with $A = B \cup N\O$.
Indeed, given an arbitrary (non-elementary) non-degenerate dendrex
$(S,\{p_v\}_{v \in \boldsymbol{V}(T)})$, 
the trees $T_{p_v}$ from \eqref{STANDELDE EQ}
can be regarded as a $S$-substitution datum, which after assembled
yields a non-degenerate standard dendrex 
$\Omega[T] \xrightarrow{b_{\{p_v\}}} N \mathcal{P}$ 
whose image contains $(S,\{p_v\}_{v})$.

We now check the characteristic conditions in \cite[Lemma 3.4]{BP18}.

(Ch0.2) is straightforward.

For (Ch1), since outer faces of standard dendrices are again standard, one needs only consider the case 
$\bar{V}=U_i$, or else $\bar{V}$ would be in some $U_j$ for $j<i$.
But if $\bar{V}=U_i$, the assumption in (Ch1) states that
$\Xi^i = \emptyset$, so that $i$ must either be a dendrex where all vertices map to $\mathcal{O}$, i.e. $b_i \in N \mathcal{O}$,
or $U_i$ is a corolla whose vertex maps to $B$, i.e. 
$b_i \in B$.
In either case, one has
$b_i \in A = B \cup N\mathcal{O}$, and (Ch1) follows.

To check both (Ch2) and (Ch3), observe first that
$V \hookrightarrow U_i$ will automatically be in $A_{<i}$ if either 
$\bar{V} \neq U_i$ or $V = U_i - D$ with 
$D \not \subseteq \Xi_i$
(since in either case $V$ would be in some $U_j$ with $j<i$),
so that one needs only consider the case 
$V=U_i$.

(Ch2) then follows vacuously, since, except in the trivial cases where $\Xi^i = \emptyset$, the dendrex $b_i(U_i - \Xi^i)$ always contains at least one vertex not in $\mathcal{O} \amalg B$.

For (Ch3), we argue that if 
$b_i(U_i - \Xi^i) \in b_j \left( \Omega[U_j] \right)$
then in fact $i\leq j$.
Writing $\bar{U}_i = U_i - \Xi^i$, the hypothesis is that
\[
\begin{tikzcd}
	\bar{U}_i \ar{d} \ar{r}{\bar{\varphi}} & U_j \ar{r}{b_j} & N \mathcal{P}
\\
	U_i \ar[dashed]{ru}[swap]{\varphi}
\end{tikzcd}
\]
there is a face map $\bar{\varphi}$ as above such that
$b_i(\bar{U}_i) = \bar{\varphi}^{\**}(b_j)$, and the goal is to build $\varphi$ such that $b_i = \varphi^{\**}(b_j)$.
Let $w \in \boldsymbol{V}(\bar{U}_i)$ be a vertex and 
let $p_w$ be the corresponding operation of $\mathcal{P}$.
Then the outer fact $(U_i)_{w}$ is precisely $T_{p_w}$ from
\eqref{STANDELDE EQ}.
On the other hand, letting
$(U_j)_w - D_w \hookrightarrow (U_j)_w$
be any choice of reduced inner face, 
one has that this too is $T_{p_w}$, at least up to a replanarization isomorphism, i.e. one has isomorphims
$(U_i)_{w} \simeq (U_j)_w - D_w$,
compatible with the restrictions of $b_i$, $b_j$.
But then combining these isomorphisms yields the desired $\varphi$,
and (Ch3) follows.

Lastly, we show (Ch0.1).
Given any non-degenerate dendrex
$\Omega[V] \xrightarrow{c} N \mathcal{P}$
by applying \eqref{STANDELDE EQ} to each individual operation
(together with an ``assembly of substitution data'' argument)
one obtains that there exists a unique
non-degenerate standard dendrex 
$b_c \colon \Omega[U_c] \to N \mathcal{P}$,
edge subset $D_c \subseteq \Xi^c$ and isomorphism
$V \simeq U_c - D_c$
such that $b$ equals the composite
\begin{equation}\label{STANDELDEGER EQ}
	V \simeq U_c - D_c
	\hookrightarrow U_c
	\xrightarrow{b_c} N \mathcal{P}
\end{equation}
Recall now that, by the preliminary argument for (Ch2) and (Ch3), 
the non-degenerate dendrices not in 
$b_i^{-1}(A_{< i})$ are precisely the replanarizations of the faces 
$U_i - D$ with $D \subseteq \Xi^i$.
But the uniqueness of the data in \eqref{STANDELDEGER EQ}
implies that all such replanarizations of the $U_i - D$
do indeed have distinct images in $N \mathcal{P}$,
thus establishing (Ch0.1) and finishing the proof.
\end{proof}


\begin{remark}
	In general, injectivity of the map
	$b_i \colon \Omega[U_i] \to N \mathcal{P}$
	will fail in $b_i^{-1}(A_{< i})$.
	Indeed, in general two edges/vertices of $U_i$
	may be assigned the same color/operation of $\mathcal{P}$.
	In fact, injectivity may in general fail even for large outer faces.
\end{remark}


{\color{blue} Bla poset not finite but still projective/injective, which is enough}


\begin{remark}
In addition to \eqref{PUSHOPPR EQ}, 
one also has the alternative formula
\begin{equation}\label{PUSHOPPRG EQ}
	\mathcal{P}(C) = 
	\coprod_{
	[T] \in \mathsf{Iso}
	\left( G \ltimes \Omega_{\mathfrak{C}}^a \downarrow_{\mathsf{r}} C \right)
	}
	\left(
		\prod_{v \in V^{ac}(T)} \mathcal{O}(T_v)
	\times
		\prod_{v \in V^{in}(T)} B(T_v)
	\right)
	\cdot_{\mathsf{Aut}_{G \ltimes \Omega^a_{\mathfrak{C}}}(T)} \mathsf{Aut}_{G \ltimes \Sigma_{\mathfrak{C}}}(C)
\end{equation}
which replaces the roles of 
$\Omega_{\mathfrak{C}}^a$, $\Sigma_{\mathfrak{C}}$
with 
$G \ltimes \Omega_{\mathfrak{C}}^a$,
$G \ltimes \Sigma_{\mathfrak{C}}$.
The connection between the two formulas is given by Lemma \ref{REDUCELAN LEM},
though some care is needed.
Namely, \eqref{PUSHOPPRG EQ} generally features fewer coproduct summands but this is compensated by the inductions
$(-) \cdot_{\mathsf{Aut}_{G \ltimes \Omega^a_{\mathfrak{C}}}(T)} \mathsf{Aut}_{G \ltimes \Sigma_{\mathfrak{C}}}(C)$,
which produce more terms than the 
$(-) \cdot_{\mathsf{Aut}_{\Omega^a_{\mathfrak{C}}}(T)} \mathsf{Aut}_{\Sigma_{\mathfrak{C}}}(C)$
inductions.
\end{remark}



\subsection{Tame time}

\begin{definition}
	The \textit{colored tensor product} 
\[
\begin{tikzcd}[row sep = 0, column sep = 40pt]
	\mathsf{PreOp}^G \times \mathsf{sSet} \ar{r}{(-)\otimes_{\mathfrak{C}}(-)} &
	\mathsf{PreOp}^G
\end{tikzcd}
\]
is defined by $(X \otimes_{\mathfrak{C}} K)(T) = X(T) \times K$
whenever $T$ is a non-linear tree (equivalently, 
$\mathsf{Hom}_{\Omega}(T,\eta)=\emptyset$) and
is defined by the following pushout when $T=[n]$ is linear.
\[
\begin{tikzcd}
	X(\eta) \times K \ar{r} \ar{d} \arrow[dr, phantom, "\ulcorner", very near start]  &
	X(\eta) \ar{d}
\\
	X([n]) \times K \ar{r} & 
	(X \otimes_{\mathfrak{C}} K)([n]) 
\end{tikzcd}
\]
\end{definition}

\begin{remark}
More concisely, $X \otimes_{\mathfrak{C}} K$ is defined by the pushout
\[
\begin{tikzcd}
	\left(\mathsf{sk}_{\eta}X \right) \times K \ar{r} \ar{d} \arrow[dr, phantom, "\ulcorner", very near start]  &
	\mathsf{sk}_{\eta}X \ar{d}
\\
	X \times K \ar{r} & 
	X \otimes_{\mathfrak{C}} K 
\end{tikzcd}
\]
\end{remark}


\begin{remark}
For fixed $K \in \mathsf{sSet}$, the functor
$(-) \otimes_{\mathfrak{C}} K
\colon \mathsf{PreOp}^G \to \mathsf{PreOp}^G$
preserves all colimits (indeed, it is not hard to build the right adjoint explicitly).

However, for a fixed $X \in \mathsf{PreOp}^G$,
the functor 
$X \otimes_{\mathfrak{C}} (-)
\colon \mathsf{sSet} \to \mathsf{PreOp}^G$
does not preserve all colimits.
In particular, this functor can not preserve coproducts since, writing 
$\mathcal{C} = X(\eta)$ for the $G$-set of objects of $X$,
the image of $X \otimes_{\mathfrak{C}} (-)$ is entirely contained in the subcategory
$\mathsf{PreOp}^{G,\mathfrak{C}} \subset
\mathsf{PreOp}^G$
of preoperads with $G$-set of objects $\mathfrak{C}$ and maps which are the identity on objects. 
Instead, one has that the functor 
\[
X \otimes_{\mathfrak{C}} (-) \colon
\mathsf{sSet} \to \mathsf{PreOp}^{G,\mathfrak{C}}
\]
does preserve colimits. 
In practice, this means that some standard arguments concerning tensor products can only be applied after adjusting the objects of the relevant preoperads
({\color{red} see later}).
\end{remark}


\begin{remark}\label{COLORTENSGAM REM}
Let $X \to Y$ be any map in $\mathsf{PreOp}^G$
which is the identity on colors and 
$K \in \mathsf{sSet}$. Then the squares below are pushout squares.
Moreover, whenever $K$ is connected the rightmost horizontal maps are isomorphisms.
\[
\begin{tikzcd}
	X \times K \ar{r} \ar{d} 
	\arrow[dr, phantom, "\ulcorner", very near start] &
	\gamma_! \left( X \times K \right) \ar{r} \ar{d} 
	\arrow[dr, phantom, "\ulcorner", very near start] &
	X \otimes_{\mathfrak{C}} K \ar{d}
\\
	Y \times K \ar{r} &
	\gamma_! \left( Y \times K \right) \ar{r} &
	Y \otimes_{\mathfrak{C}} K
\end{tikzcd}
\]
\end{remark}


\begin{definition}
	Let $f \colon \mathfrak{C} \to \mathfrak{D}$
	be a map of $G$-sets (of colors).
	We define adjoint functors
\[
	f_{!} \colon
	\mathsf{PreOp}^{G,\mathfrak{C}}
\rightleftarrows
	\mathsf{PreOp}^{G,\mathfrak{D}}
	\colon f^{\**}
\]
via the pushout and pullback squares
(note that $\mathsf{sk}_{\eta} f_! A$ depends only on 
$\mathfrak{C}$ while 
$\mathsf{csk}_{\eta} f^{\**} X$ depends only on
$\mathfrak{D}$)
\[
\begin{tikzcd}
	\mathsf{sk}_{\eta} A \ar{r} \ar{d} \arrow[dr, phantom, "\ulcorner", very near start]  &
	\mathsf{sk}_{\eta} f_! A \ar{d}
&&
	f^{\**} X \ar{r} \ar{d} &
	X \ar{d}
\\
	A \ar{r} & 
	f_! A
&&
	\mathsf{csk}_{\eta} f^{\**} X \ar{r} & 
	\mathsf{csk}_{\eta} X
	\arrow[ul, phantom, "\lrcorner", very near start]
\end{tikzcd}
\]
\end{definition}


\begin{definition}
	A $G$-preoperad $X \in \mathsf{PreOp}^G$ is called a \textit{$G$-Segal operad} if, 
	for each $G$-tree $T$,
	the natural map 
	$X\left( \Omega[T] \right) \to 
	X \left( Sc[T] \right)$
	is a Kan equivalence.
\end{definition}

\begin{notation}
Given a $G$-Segal operad $X$ and $G$-corolla $C$, 
$X(\partial \Omega[C])$ is a discrete simplicial set whose elements
are the $C$-profiles $(c_1,\cdots,c_n;c_0)$ of $X$.
The map $X(\Omega[C]) \to X(\partial \Omega[C])$ hence yields
a coproduct decomposition 
\[
X(\Omega[C]) \simeq \coprod_{C\text{-profiles }(c_1,\cdots,c_n;c_0)}
X(c_1,\cdots,c_n;c_0)
\]
\end{notation}


\begin{remark}\label{SEOPDK REM}
Given a $G$-Segal operad $X$, consider a dendroidal Reedy fibrant replacement $X \to \tilde{X}$ such that $X(\eta) \simeq \tilde{X}(\eta)$. 
This means that all maps 
$X(\Omega[T]) \to \tilde{X}(\Omega[T])$ are Kan equivalences,
and moreover, by the following pullback diagram
\[
\begin{tikzcd}
	Z (Sc[T]) \ar{r} \ar{d} &
	\prod_{v \in \boldsymbol{V}_G(T)} Z
	(\Omega[T_v]) \ar{d}
\\
	\prod_{(G/H_i \cdot e_i) \in \boldsymbol{E}_G(T)} 
	\mathfrak{C}^{H_i} \ar{r}  &
	\prod_{v \in \boldsymbol{V}_G(T)}
	\prod_{(G/H_i \cdot e_i) \in \boldsymbol{E}_G(T_v)} 
	\mathfrak{C}^{H_i} 
	\arrow[ul, phantom, "\lrcorner", very near start]
\end{tikzcd}
\]
so are the maps $X(Sc[T]) \to \tilde{X}(Sc[T])$.
This shows that $\tilde{X}$ is also a Segal operad, 
and thus a fibrant object in $\mathsf{PreOp}^G$.

Furthermore, the Kan equivalences 
$X(\Omega[C]) \to \tilde{X}(\Omega[C])$
induce Kan equivalences 
$X(c_1,\cdots,c_n;c_0) \to \tilde{X}(c_1,\cdots,c_n;c_0)$.
It follows that the complete equivalences between Segal operads are precisely the Dwyer-Kan equivalences. 
\end{remark}


\begin{remark}\label{SLIMOD REM}
Noting that for every fibrant 
$\tilde{X} \in \mathsf{PreOp}^G$
any equivalence in $\tilde{X}$ is in the image of a map
$J \to \tilde{X}$, 
a slight modification of the proof of Lemma \ref{INTER_LEM}
shows that for any Segal operad $X$
any equivalence in $X$ is in the image of a countable, contractible
$I \in \mathsf{PreOp}^G$
such that $\eta \amalg \eta \to I$
is a tame cofibration.
\end{remark}




\begin{theorem}
	There is a model structure on 
	$\mathsf{PreOp}^G$,
	called the \textbf{tame model structure},
	such that:
\begin{itemize}
	\item the weak equivalences are the complete equivalences (i.e. detected by inclusion into 
	$\mathsf{sdSet}^G$);
	\item generating cofibrations are given by the maps
	\begin{itemize}
		\item[(TC1)] $G/H \cdot \left(\emptyset \to\Omega[\eta]\right)$ for $H\leq G$;
		\item[(TC2)] $\Omega[C] \otimes_{\mathfrak{C}} \left(\partial \Delta[n] \to \Delta[n]\right)$ for $C \in \Sigma_G$, $n \geq 0$;
		\item[(TC3)] 
$\left( Sc[T] \to \Omega[T] \right) 
\square_{\mathfrak{C}} 
\left(\partial \Delta[n] \to \Delta[n]\right)$ for $T \in \Omega_G$, $n \geq 0$.
	\end{itemize}
\end{itemize}
Furthermore, one has generating anodyne cofibrations the maps
\begin{itemize}
	\item[(TA1)] $G/H \cdot 
	\left(\Omega[\eta] \to I \right)$ for $H \leq G$,
	and $\Omega[\eta] \to I$ a weak equivalence in $\mathsf{PreOp}$ such that $I(\eta) = \{0,1\}$, $\Omega[\eta] \amalg \Omega[\eta] \to I$ is a tame cofibration, and $I$ is countable;
	\item[(TA2)] $\Omega[C] \otimes_{\mathfrak{C}}\left(\Lambda^i[n] \to \Delta[n]\right)$ for $C \in \Sigma_G$, $0 \leq i \leq n$;
	\item[(TA3)] 
$\left( Sc[T] \to \Omega[T] \right) 
\square_{\mathfrak{C}} 
\left(\partial \Delta[n] \to \Delta[n]\right)$ for $T \in \Omega_G$, $n \geq 0$.
	\end{itemize}
\end{theorem}


\begin{proof}
	The existence of the model structure will follow by applying J. Smith's theorem \cite[Thm. 1.7]{Bek00}. Conditions c0 and c2 therein are inherited from $\mathsf{sdSet}^G$
	and the technical ``solution set condition'' c3 follows from
	\cite[Prop. 1.15]{Bek00} since weak equivalences are accessible, being the preimage by $\gamma^{\**}$ if the weak equivalences in 
	$\mathsf{sdSet}^G$ 
	(see \cite[Cor. A.2.6.5]{Lur09} and \cite[Cor. A.2.6.6]{Lur09}).
	
	For c1, we must show that any map $X \to Y$ with the right lifting property against (TC1), (TC2), (TC3) is a weak equivalence.
	Writing $f \colon \mathfrak{C} \to \mathfrak{D}$ for the underlying map of colors,
	consider the factorization $X \to f^{\**}Y \to Y$.
	Note that since maps out of (TC1) depend only on objects and both of (TC2) and (TC3) consist of maps which are identities on objects,
	$X \to Y$ will have the right lifting property against (TC1) iff 
	$f^{\**} X \to Y$ does
	and the right lifting property against 
	(TC2) and (TC3) iff $X \to f^{\**}Y$ does.
	
Note now that $f^{\**} Y \to Y$ has the right lifting proper against all maps 
	$\left(\partial \Omega[T] \to \Omega[T] \right) \times \Delta[n]$.
	Indeed, if $T \simeq G/H \cdot \eta$ is a stick, this is precisely the lifting condition agains (TC1), and otherwise it follows automatically since $\left(\partial \Omega[T] \to \Omega[T] \right) \times \Delta[n]$ is the identity on objects.
	Therefore, the levels 
	$\left(f^{\**} Y \right)_n \to Y_n$ are trivial fibrations in 
	$\mathsf{dSet}^G$, showing that 
	$f^{\**} Y \to Y$ is a dendroidal equivalence, 
	and thus a complete equivalence. 
	
	Since the maps in both of (TC2) and (TC3) are the identify on objects, $X \to Y$ has the right lifting property against these maps iff $X \to f^{\**}Y$ does.
The lifting property against (TC2) then says that the maps
$X(\Omega[C]) \to f^{\**} Y (\Omega[C])$
are trivial Kan fibrations for all $G$-corollas $C \in \Sigma_G$,
and thus so are the maps
$X(Sc[T]) \to f^{\**} Y (Sc[T])$ for all $G$-trees $T \in \Omega_G$.
But it then follows from the lifting property against
(TC3) that the maps 
$X(\Omega[T]) \to f^{\**} Y (\Omega[T])$
are trivial Kan fibrations for all $G$-trees,
showing that $X \to f^{\**} Y$ is a simplicial equivalence, and thus a complete equivalence. 
\[
\begin{tikzcd}
	X(\Omega[T]) \ar{r} \ar[->>]{d}{\sim} &
	X(Sc[T]) \ar{r} \ar[->>]{d}{\sim} &
	\prod_{v \in \boldsymbol{V}_G(T)} X(\Omega[T_v])
	\ar[->>]{d}{\sim}
\\
	f^{\**} Y(\Omega[T]) \ar{r} &
	f^{\**} Y(Sc[T]) \ar{r} \ar{d} &
	\prod_{v \in \boldsymbol{V}_G(T)} f^{\**} Y
	(\Omega[T_v]) \ar{d}
	\arrow[ul, phantom, "\lrcorner", very near start]
\\
	&
	\prod_{(G/H_i \cdot e_i) \in \boldsymbol{E}_G(T)} 
	\mathfrak{C}^{H_i} \ar{r}  &
	\prod_{v \in \boldsymbol{V}_G(T)}
	\prod_{(G/H_i \cdot e_i) \in \boldsymbol{E}_G(T_v)} 
	\mathfrak{C}^{H_i} 
	\arrow[ul, phantom, "\lrcorner", very near start]
\end{tikzcd}
\]
This completes the proof of c1, establishing the existence of the tame model structure.

We now turn to the ``further'' claim considering the claimed generating anodyne cofibrations, i.e., 
we wish to show that the maps in 
(TA1), (TA2), (TA3) satisfy the conditions in
Lemma \ref{SEMICOF LEM}.

We first check condition (i).
The case of maps in (TC1) is tautological.
Since $\Lambda^{i}[n]$ is connected, 
the maps in (TA2) have the form
$\gamma_{!} 
\left( \Omega[C] \times
\left( \Lambda^i[n] \to \Delta[n] \right) \right)$,
and are thus weak equivalences thanks to the pushouts
in Remark \ref{COLORTENSGAM REM}.
As for (TA3), it follows from Remark \ref{COLORTENSGAM REM}
that the maps
$\left( Sc[T] \to \Omega[T] \right) \otimes \partial \Delta[n]$
and 
$\left( Sc[T] \to \Omega[T] \right) \otimes \Delta[n]$
are trivial cofibrations, so that the claim follows from a standard pushout and 2-out-of-3 argument.

We now turn to condition (ii).
The lifting condition against (TA3) says that $J$-fibrant objects are such that the maps $X(\Omega[T]) \to X(Sc[T])$
are trivial fibrations, and thus that such $X$ are Segal operads.
Therefore, by Remark \ref{SEOPDK REM} it suffices to check that $J$-fibrations between Segal operads which are also DK equivalences are in fact trivial fibrations, i.e. that they have the right lifting property against the maps in (TC1),(TC2),(TC3).
Given $X \to Y$ a $J$-fibration with $J$-fibrant $Y$,
the lifting property against (TC3) is tautological since 
(TC3) equals (TA3).
Next, the lifting property against (TA2) says that the maps
$X(\Omega[T]) \to f^{\**} Y(\Omega[T])$
are Kan fibrations, and the DK condition says that these are Kan equivalences,
so that we conclude that such maps have the right lifting property against (TC2).
Lastly, given any lifting problem against a map in (TC1),
essential surjectivity and Remark \ref{SLIMOD REM}
produce a lifting problem against a map in (TA1) which has a solution, providing a solution to the original problem.
\end{proof}


{\color{red} To show that maps in (TA3) are normal cofibrations one can use a pushout of projective cofibrant cubes argument.}


The following results are adapted from \cite{JT07} (see Proposition 7.15 therein). 


\begin{proposition}
	A cofibration $A \to B$ is a weak equivalence iff it has the left lifting property against all fibrations between fibrant objects.
\end{proposition}

\begin{proof}
	Let $B \xrightarrow{\sim} \tilde{B}$ be a fibrant replacement and
	let $A \xrightarrow{\sim} \tilde{A} \twoheadrightarrow \tilde{B}$
	be a factorization of the composite $A \to \tilde{B}$ 
	as a trivial cofibration followed by a fibration.
	One then has a lift in the diagram
\[
\begin{tikzcd}
	A \ar{r}{\sim} \ar[>->]{d} & \tilde{A} \ar[->>]{d}
\\
	B \ar{r}{\sim} \ar[dashed]{ru} & \tilde{B}
\end{tikzcd}
\]
where the top and bottom horizontal maps are weak equivalences. 
But then the 2-out-of-6 property for weak equivalences says that all maps are weak equivalences.
\end{proof}


\begin{corollary}\label{SIMPLQUILL COR}
An adjunction 
\[
F \colon \mathcal{C}
	\rightleftarrows
\mathcal{D} \colon G
\]
between model categories is a Quillen adjunction
provided that $F$ preserves cofibrations
and $G$ preserves fibrations between fibrant objects.
\end{corollary}


\begin{lemma}
	Let $A \to B$ be a tame cofibration in $\mathsf{PreOp}^G$, 
	$\mathcal{O} \in \mathsf{sOp}^G$ a $\Sigma$-cofibrant 
	$G$-operad,
	and consider a pushout diagram in $\mathsf{sOp}^G$ of the form
\[
\begin{tikzcd}
	\tau A \ar{r} \ar{d} & \mathcal{O} \ar{d}
\\
	\tau B \ar{r} & \mathcal{P}
\end{tikzcd}
\]
	Then $\mathcal{O} \to \mathcal{P}$ is a $\Sigma$-cofibration and 
\begin{equation}\label{UNITEQUIV EQ}
B \amalg_{A} N \mathcal{O}
	\to 
N \mathcal{P}
\end{equation}
is a weak equivalence.
\end{lemma}

\begin{proof}
	We first consider the case where $A\to B$ is in one of (TC1),(TC2),(TC3). 
	
	The (TC1) case is immediate, 
	since $\mathcal{O} \to \mathcal{O} \amalg G/H \cdot \Omega(\eta)$ is a $\Sigma$-cofibration and
	\eqref{UNITEQUIV EQ}
	is the isomorphism
	$N\mathcal{O} \amalg G/H\cdot \Omega[\eta] \simeq 
	N\left( \mathcal{O} \amalg G/H \cdot \Omega(\eta) \right)$.

	The (TC3) case is also straightforward:
	since $\tau A \to \tau B$ is an isomorphism, one can take 
	$\mathcal{O}=\mathcal{P}$, so that 
	\eqref{UNITEQUIV EQ} becomes a section of the map
	$N \mathcal{O} \to B \amalg_{A} N \mathcal{O}$, which is a trivial cofibration (it is a pushout of $A \to B$),
	and 2-out-of-3 hence implies that \eqref{UNITEQUIV EQ} is a weak equivalence.

	The most interesting case is then (TC2), 
	in which case it is well known that 
	$\mathcal{O} \to \mathcal{P}$ is a $\Sigma$-cofibration and
	each of the levels
$(B \amalg_{A} N \mathcal{O})_n
	\to 
(N \mathcal{P})_n$
for $n \geq 0$
is an equivalence in $\mathsf{dSet}^G$ by (an iteration of)
Proposition \ref{KEYPR PROP}, 
showing that \eqref{UNITEQUIV EQ} is in fact a dendroidal equivalence, and thus also a complete equivalence.

	We now turn to the case of $A \to B$ a general cofibration between cofibrant objects.
	As usual, $A \to B$ is a retract of a transfinite composition of pushouts of generating cofibrations.
	Since the conclusions of the result are invariant under retracts,
	we are free to assume that $A \to B$ is a transfinite composite
\[
A = A_0 \to A_1 \to A_2 \to \cdots \to A_{\beta} \to 
colim_{\beta < \kappa} A_{\beta} = B.
\]
where each map $A_{\beta} \to A_{\beta +1}$ is a pushout of a map in one of (TC1),(TC2),(TC3).

Defining $\mathcal{O}_{\beta}$ by replacing $A \to B$ with $A \to A_{\beta}$ in the pushout,
$\mathcal{O} \to \mathcal{P}$ becomes the transfinite composite of the maps $\mathcal{O}_{\beta} \to \mathcal{O}_{\beta + 1}$
and \eqref{UNITEQUIV EQ} becomes
$
colim_{\beta < \kappa} \left( 
N \mathcal{O} \amalg_{N \tau A} N \tau A_{\beta}
	\to 
N \mathcal{O}_{\beta}
\right)
$.
It thus suffices to show, by induction on $\beta < \kappa$, 
that the maps $\mathcal{O}_{\beta} \to \mathcal{O}_{\beta + 1}$ are $\Sigma$-cofibrations and that the maps 
$N \mathcal{O} \amalg_{N \tau A} N \tau A_{\beta}
	\to 
N \mathcal{O}_{\beta}$
are weak equivalences
(that this last condition suffices follows since
filtered colimits of weak equivalences in $\mathsf{PreOp}^G$ are weak equivalences ({\color{red} add this})).
Consider now the following diagrams.
\[
\begin{tikzcd}
	\tau A \ar{r} \ar{d} & \mathcal{O} \ar{d}
&&
	A_{\beta} \amalg_{A} N \mathcal{O}
	\ar[>->]{r} \ar{d}[swap]{\sim} &
	A_{\beta+1} \amalg_{A} N \mathcal{O}
	\ar{d}[swap]{\sim}
\\
	\tau A_{\beta} \ar{r} \ar{d} & \mathcal{O}_{\beta} \ar{d}
&&
	N \mathcal{O}_{\beta} \ar[>->]{r} &
	A_{\beta+1} \amalg_{A_{\beta}} N \mathcal{O}_{\beta} \ar{d}
\\
	\tau A_{\beta + 1} \ar{r} & \mathcal{O}_{\beta + 1}
&&
	&
	N \mathcal{O}_{\beta+1}
\end{tikzcd}
\]
The induction hypothesis states that
$\mathcal{O} \to \mathcal{O}_{\beta}$ is a $\Sigma$-cofibration and that the map
$A_{\beta} \amalg_A N \mathcal{O} \to \mathcal{O}_{\beta}$ is a weak equivalence.
Therefore, $\mathcal{O}_{\beta}$ is $\Sigma$-cofibrant 
and the both vertical maps marked $\sim$ in the rightmost diagram above are weak equivalences 
(this uses the fact that $\mathsf{PreOp}^G$ is left proper),
and thus the induction step will follow provided that the result holds for
the map $A_{\beta} \to A_{\beta + 1}$ and $\mathcal{O}_{\beta}$.
But $A_{\beta} \to A_{\beta + 1}$ is assumed to be a pushout of a map in (TC1),(TC2),(TC3), in which case the result is already known, and thus noting that the result is clearly invariant under pushouts finishes the proof.
\end{proof}

Setting $A = \emptyset $, $\mathcal{O}= \emptyset$ in the previous result yields the following.

\begin{corollary}\label{KEYEQUIV COR}
	If $B \in \mathsf{PreOp}^G$ is tame cofibrant, then 
	$B \to N \tau B$ is a weak equivalence.
\end{corollary}

\begin{proposition}
The adjunction
\[
	\tau \colon \mathsf{PreOp}^G_{\text{tame}}
		\rightleftarrows 
	\mathsf{sOp}^G \colon N
\]
is a Quillen equivalence.
\end{proposition}


\begin{proof}
Firstly, note that $N$ preserves and detects weak equivalences.
Indeed, this follows since all objects in the image of $N$ are Segal operads, so that by Remark \ref{SEOPDK REM} a map in the image of $N$ is a weak equivalence iff it is a Dwyer-Kan equivalence.

Next, we show that this is a Quillen adjunction using Corollary \ref{SIMPLQUILL COR}.
The claim that $\tau$ preserves cofibrations follows since
$\tau$ sends the maps in (TC1) and (TC2) to generating cofibrations of $\mathsf{sOp}^G$ and the maps in (TC3) to isomorphisms.
For the claim that $N$ preserves fibrations between fibrant,
we use a somewhat indirect argument
(though we note that a direct argument is also possible,
by showing that fibrations between fibrant objects in $\mathsf{PreOp}^G$
also satisfy a ``local fibration plus isofibration'' description).
By Corollary \ref{KEYEQUIV COR} and 2-out-of-3, 
one has that for any trivial cofibration between cofibrant objects
$A \to B$, the map $N \tau A \to N \tau B$ is a weak equivalence, and thus so is $\tau A \to \tau B$.
This shows that $\tau$ sends all maps in (TA1),(TA2),(TA3)
to trivial cofibrations, and since these maps detect fibrations between fibrant objects in $\mathsf{PreOp}^G$, 
the standard adjunction argument shows that 
$N$ indeed preserves fibrations between fibrant objects.

For the Quillen equivalence claim, 
let $B \in \mathsf{PreOp}^G$ be tame cofibrant and
$\mathcal{O} \in \mathsf{sOp}^G$ be fibrant.
We must show that the leftmost map below is a weak equivalence iff its adjoint, which is the rightmost composite, is.
\[
	\tau B \to \mathcal{O},
\qquad
	B \xrightarrow{\sim} N \tau B \to N \mathcal{O}
\]
This is immediate from Corollary \ref{KEYEQUIV COR}
and the fact that $N$ preserves and detects weak equivalences.
\end{proof}


\begin{proposition}
	The adjunction 
$W_! \colon \mathsf{dSet}^G 
	\rightleftarrows 
\mathsf{sOp}^G \colon hcN$
	is a Quillen adjunction.
\end{proposition}

{\color{red} HERE}

\begin{proof}
	We again apply Corollary \ref{SIMPLQUILL COR}.
	For the claim that $W_!$ preserves cofibrations,
	it suffices to show this for the generating cofibrations
	$G\cdot_H \left( \partial \Omega[U] \to \Omega[U] \right)$ for $U \in \Omega^H$.
	But this follows since 
	$G \cdot_H \left(W_! \partial \Omega[U] \to W_! \Omega[U] \right)$
	is a pushout of the map
\[
	G \cdot_H \Omega[U - \boldsymbol{E}^{\mathsf{i}}(U)]
\otimes
	\left(
	\partial \left( \Delta[1]^{\times \boldsymbol{E}^{\mathsf{i}}(U) } \right) 
		\to
	\Delta[1]^{\times \boldsymbol{E}^{\mathsf{i}}(U) }
	\right).
\]
Similarly, the map
	$G \cdot_H \left(W_! \Lambda^E[U] \to W_! \Omega[U] \right)$
is a pushout 

%\[
%\begin{tikzcd}
%	G \cdot_H 
%	\Omega[U - \boldsymbol{E}^{\mathsf{i}}(U)] \otimes 
%	\partial \left( \Delta[1]^{\times \boldsymbol{E}^{\mathsf{i}}(U) } \right) 
%	\ar{r} \ar{d} &
%	G \cdot_H W_! \partial \Omega[U] \ar{d}
%\\
%	G \cdot_H 
%	\Omega[U - \boldsymbol{E}^{\mathsf{i}}(U)] \otimes 
%	\Delta[1]^{\times \boldsymbol{E}^{\mathsf{i}}(U) } \ar{r}	&
%	G \cdot_H W_! \Omega[U]
%\end{tikzcd}
%\]

{\color{red} HERE}
	
\end{proof}




\section{Coloring via spans}


\subsection{Preliminaries}


Recall \cite[Prop. 2.7]{BP17}
that if
$\pi \colon \mathcal{E} \to \mathcal{B}$ is 
a split Grothendieck fibration, 
then so is the map of functor categories
$\mathcal{E}^{\mathcal{C}} \to \mathcal{B}^{\mathcal{C}}$
for any category $\mathcal{C}$.
Explicitly, given functors $E \colon \mathcal{C} \to \mathcal{E}$,
$B',B \colon \mathcal{C} \to \mathcal{B}$
such that $B=\pi \circ E$
and a natural transformation
$\varphi \colon B' \Rightarrow B$,
the pullback functor
$\varphi^{\**} E \colon \mathcal{C} \to \mathcal{E}$
is described on objects by
$\left(\varphi^{\**} E\right) (c) =
\left(\varphi(c)\right)^{\**} (E(c))$
and on arrows $f\colon c \to \bar{c}$
as the unique dashed arrow in the leftmost diagram below which makes that diagram commute and lifts $B'(f)$. 
\[
\begin{tikzcd}
	\varphi^{\**}E(c) \ar{r} 
	\ar[dashed]{d}[swap]{\varphi^{\**}E (f)} &
	E(c) \ar{d}{E(f)}
&&
	B'(c) \ar{d}[swap]{B'(f)} \ar{r} &
	B(c) \ar{d}{B(f)}
\\
	\varphi^{\**}E(\bar{c}) \ar{r} &
	E(\bar{c}) 
&&
	B'(\bar{c}) \ar{r} &
	B(\bar{c})
\end{tikzcd}
\]
Given a split Grothendieck fibration 
$\pi^{\**} \colon \mathcal{E} \to \mathcal{B}$,
we now define a pullback functor on weak right spans
\begin{equation}\label{WSPANPULL EQ}
\pi^{\**} \colon
\mathsf{WSpan}^r(\**,\mathcal{B})
	\to
\mathsf{WSpan}^r(\**,\mathcal{E})
\end{equation}
as follows.

On objects, i.e. functors $B\colon \mathcal{C} \to \mathcal{B}$, one simply sets 
$\pi^{\**}(\colon \mathcal{C} \to \mathcal{B})=
(\mathcal{C} \times_{\mathcal{B}} \mathcal{E}
\to \mathcal{E})
$.
On arrows, which are pairs 
$(i,\varphi \colon B_2 \circ i \Rightarrow B_1)$
as in the diagram below
\begin{equation}
\begin{tikzcd}[row sep = tiny, column sep = 35pt]
	\mathcal{C}_1 \arrow{dr}[name=U]{B_1} \arrow{dd}[swap]{i}
\\
	& \mathcal{B}
\\
	|[alias=V]| \mathcal{C}_2 \arrow{ur}[swap]{B_2}
\arrow[Rightarrow, from=V, to=U,shorten >=0.25cm,shorten <=0.25cm
,swap,"\varphi"
]
\end{tikzcd}
\end{equation}
and writing 
$\pi \colon \mathcal{C}_i \times_{\mathcal{B}} \mathcal{E}
\to \mathcal{C}_i$
and
$E_i \colon \mathcal{C}_i \times_{\mathcal{B}} \mathcal{E}
\to \mathcal{E}$
for the projections, one sets
\[
\pi^{\**} (i,\varphi)=
\left(
	\left( i \pi,
	\left( \varphi \pi \right)^{\**} E_1 \right),
	\left( \varphi \pi \right)^{\**} E_1 \Rightarrow E_1
\right)
\]
or, put another way, 
$\pi^{\**}(i,\phi)$
is characterized as the unique choice of dashed data in the following diagram
\[
\begin{tikzcd}[column sep = small, row sep = small]
	\mathcal{C}_1 \times_{\mathcal{B}} \mathcal{E} 
	\ar{rrrrr}[name=toE]{E_1} \ar[dashed]{rd} \ar{dd}
	&&&
	&&
	\mathcal{E}  \ar{dd}
\\
	&
	|[alias=DBE]|
	\mathcal{C}_2 \times_{\mathcal{B}} \mathcal{E} \ar{rrrru}[swap]{E_2}
\\
	\mathcal{C}_1 \ar{rrrrr}[name=toB]{B_1} \ar{rd} 
	&&&
	&&
	\mathcal{B} 
\\
	&
	|[alias=D]| \mathcal{C}_2 \ar{rrrru}[swap]{B_2}
\arrow[Rightarrow, from=DBE, to=toE, shorten <=0.15cm,shorten >=0.15cm,dashed
%,swap,"\pi_i"
]
	\arrow[Rightarrow, from=D, to=toB, shorten <=0.15cm,shorten >=0.15cm,swap,"\varphi"]
	\arrow[from=DBE, to=D, crossing over]
\end{tikzcd}
\]
such that the side faces commute, the top natural transformation consists of pushout arrows for $\pi \colon \mathcal{E} \to \mathcal{B}$, and the total diagram commutes, in the sense that the two composite natural transformations $B_2 i \pi \Rightarrow \pi E_1$ coincide.
Associativity and unitality of $\pi^{\**}$ are straightforward.



\begin{remark}\label{SPANLIM REM}
Given a diagram 
$J \xrightarrow{j \mapsto (B_j \colon \mathcal{C}_j \to \mathcal{B})}
\mathsf{WSpan}^r(\**,\mathcal{B})$
and a cone over it, 
i.e. an object
$(B \colon \mathcal{C} \to \mathcal{B}) \in \mathsf{WSpan}^r(\**,\mathcal{B})$
together with compatible maps 
$(i_j,\varphi_j) \colon B \to B_j$,
this will be a limit in 
$\mathsf{WSpan}^r(\**,\mathcal{B})$
provided that
\[
	\mathcal{C} = \lim_{j \in J} \mathcal{C}_j
\qquad
	B = colim_{j \in J} B_j i_j
\]
where the limit takes place in 
$\mathsf{Cat}$
and the colimit in 
$\mathsf{Fun}(\mathcal{C}, \mathcal{B})$.
\end{remark}


\subsection{Edge functors}

\begin{notation}
An edge $(i,\varphi)\colon B_1 \to B_2$ in $\mathsf{WSpan}^r(\**,\mathcal{B})$
will be called a 
\textit{$1$-arrow}
if $\varphi = id_{B_1}$,
i.e. if the arrow exhibits the commutative diagram
$B_1 = B_2 i$.
\end{notation}


Recall the identification
\[
\Omega^t \simeq |\Omega^{\bullet}|
\]
of the category $\Omega^t$ of trees and tall maps with the realization of the simplicial object in categories with $n$-th level the planar $n$-strings $\Omega^{n}$.
It then follows that the target functors
$\Omega^n \to \Omega^t$
given by 
$(T_0 \to T_1 \to \cdots \to T_n)
\mapsto T_n$
define a simplicial object in 
$\mathsf{WSpan}^r(\**,\Omega^t)$
where all simplicial operators other than the top faces $d_n$ are $1$-arrows in
$\mathsf{WSpan}^r(\**,\Omega^t)$.

Furthermore, letting 
$\boldsymbol{E} \colon \Omega^{t} \to \mathsf{F}$
be the edge functor, sending each tree to the underlying set of edges, 
one obtains a simplicial object in
$\mathsf{WSpan}^r(\**,\mathsf{F})$,
whose levels we will denote by
$\boldsymbol{E}\colon \Omega^n \to \mathsf{F}$.
Note that by definition
$\boldsymbol{E}(T_0 \to T_1 \to \cdots \to T_n)=
\boldsymbol{E}(T_n)$.
Next, define a functor
\begin{equation}\label{SIGMAFUN EQ}
\begin{tikzcd}[row sep = 0]
	\mathsf{WSpan}^r(\**,\mathsf{F}) \ar{r}{\Sigma \wr (-)} &
	\mathsf{WSpan}^r(\**,\mathsf{F})
\\
	\mathcal{C} \to \mathsf{Fin}
	\ar[mapsto]{r} &
	\Sigma \wr \mathcal{C} \to
	\Sigma \wr \mathsf{Fin} \xrightarrow{\amalg}
	\mathsf{Fin}
\end{tikzcd}
\end{equation}
We have now extended the categories
$\Omega^n$ and $\Sigma \wr \Omega^n$
and the simplicial arrows $d_i$, $s_i$
to objects and arrows in 
$\mathsf{WSpan}^r(\**,\mathsf{F})$.
We next discuss how to likewise extend the vertex functors
$V \colon \Omega^n \to \Sigma \wr \Omega^{n-1}$.
The key case is that of $n=1$, in which case we have a natural transformation
\[
\begin{tikzcd}
	\Omega^1 \ar{rr}{\boldsymbol{E}} \ar{d}[swap]{\boldsymbol{V}}&&
	|[alias=Ft]|
	\mathsf{F} \ar[equal]{d}
\\
	|[alias=SOm]|
	\Sigma \wr \Omega^0 \ar{r}[swap]{\boldsymbol{E}} &
	\Sigma \wr \mathsf{F} \ar{r}[swap]{\amalg} &
	\mathsf{F}
\arrow[Rightarrow, from=SOm, to=Ft,shorten >=0.25cm,shorten <=0.25cm
%,swap,"\varphi"
]
\end{tikzcd}
\]
which, for each $T_0 \to T_1$ is the obvious map
$\coprod_{v \in \boldsymbol{V}(T_0)}
\boldsymbol{E}(T_{1,v}) \to
\boldsymbol{E}(T_1)$
induced by the maps
$T_{1,v} \to T_1$. 
For $n \neq 1$, one can either mimic this formula or, alternatively,
note that if the diagrams 
\begin{equation}\label{VASEDGE EQ}
\begin{tikzcd}
	\Omega^n \ar{r}{\boldsymbol{V}} 
	\ar{d}[swap]{d^{1,\cdots,n-1}} &
	\Sigma \wr \Omega^{n-1}
	\ar{d}{d^{0,\cdots,n-2}}
&&
	\Omega^0 \ar{r}{\boldsymbol{V}} 
	\ar{d}[swap]{s^0} &
	\Sigma \wr \Sigma
	\ar{d}{s^{-1}}
\\
	\Omega^1 \ar{r}{\boldsymbol{V}} &
	\Sigma \wr \Omega^0
&&
	\Omega^1 \ar{r}{\boldsymbol{V}} &
	\Sigma \wr \Omega^0
\end{tikzcd}
\end{equation}
are to remain commutative when regarded as diagrams in $\mathsf{WSpan}^r(\**,\mathsf{F})$, 
the fact that the vertical arrows are $1$-arrows
implies that the natural transformation component of the bottom $\boldsymbol{V}$ functors immediately determines the natural transformation compoment of the top $\boldsymbol{V}$ functors.

To simply notation, in what follows we will represent diagrams in
$\mathsf{WSpan}^r(\**,\mathsf{F})$ simply by their category component.

\begin{proposition}
Let $1 \leq i \leq n$. Then the squares 
\[
\begin{tikzcd}
	\Omega^n \ar{r}{\boldsymbol{V}} 
	\ar{d}[swap]{d^{i}} &
	\Sigma \wr \Omega^{n-1}
	\ar{d}{d^{i-1}}
\\
	\Omega^{n-1} \ar{r}{\boldsymbol{V}} &
	\Sigma \wr \Omega^{n-2}
\end{tikzcd}
\]
are pullback squares when regarded as squares in 
$\mathsf{WSpan}^r(\**,\mathsf{F})$.
\end{proposition}

\begin{proof}
We need to check the conditions in 
Remark \ref{SPANLIM REM}.
The fact that the underlying diagrams of categories are pullback diagrams is already known.
Furthermore, when $i<n$ the vertical arrows are $1$-arrows and it is hence obvious that square of functors $\Omega^n \to \mathsf{F}$ is a pullback square (it is a square where two opposite sides are identities).
For the case $i=n$, by using combining the simplicial identities and the $i<n$ case, one reduces to checking the case $i=n=1$.
The required claim is then that for each $1$-string
$T_0 \to T_1$ there is a natural isomorphism
\[
	\boldsymbol{E}(T_1) \simeq
	colim\left(
	\boldsymbol{E}(T_0) \leftarrow 
	\coprod_{v \in \boldsymbol{V}(T_0)} \boldsymbol{E}(T_{0,v}) \to 
	\coprod_{v \in \boldsymbol{V}(T_0)} \boldsymbol{E}(T_{1,v})
	\right)
\]
and this is clear from the discussion of substitution data.
\end{proof}


\subsection{Vertex functors}

In the previous section we discussed how to equip the categories $\Omega^n$,
and the functors 
$d^i \colon \Omega^n \to \Omega^{n-1}$,
$s^j \colon \Omega^n \to \Omega^{n+1}$,
$V \colon \Omega^n \to \Sigma \wr \Omega^{n-1}$
with ``edge data'' by extending them to 
spans over $\mathsf{Fin}$.
In this section we discuss a variant which extends these categories and functors (except for the top face maps $d^n$) to spans encoding ``vertex data''.

Firstly, we discuss some preliminaries.
Given a split Grothendieck fibration
$\mathcal{E} \to \mathcal{B}$ we write
$\mathsf{WSpan}^r(\**,\mathcal{E} \to \mathcal{B})
\subseteq
\mathsf{WSpan}^r(\**,\mathcal{E})$
for the subcategory with the same objects but with arrows only those natural transformations whose constituent arrows are pullbacks for $\mathcal{E} \to \mathcal{B}$.

We will now extend the categories $\Omega^n$,
and arrows 
$d^i$ (for $i<n$),
$s^j$ and $V$
to objects and arrows in
$\mathsf{WSpan}^r(\**,\Sigma \wr \Sigma \to \Sigma)$,
where $\Sigma\wr \Sigma \to \Sigma$
is the natural split Grothendieck construction.

Adapting \eqref{SIGMAFUN EQ} we now have a functor
\begin{equation}\label{SIGMAFUN2 EQ}
\begin{tikzcd}[row sep = 0]
	\mathsf{WSpan}^r(\**,\Sigma\wr \Sigma \to \Sigma) \ar{r}{\Sigma \wr (-)} &
	\mathsf{WSpan}^r(\**,\Sigma\wr \Sigma \to \Sigma)
\\
	\mathcal{C} \to \Sigma\wr \Sigma
	\ar[mapsto]{r} &
	\Sigma \wr \mathcal{C} \to
	\Sigma \wr \Sigma \wr \Sigma \xrightarrow{\amalg}
	\Sigma \wr \Sigma
\end{tikzcd}
\end{equation}
Combining this $\Sigma \wr (-)$ functor with the vertex functors
$V \colon \Omega^n \to \Sigma \wr \Omega^{n-1}$
one obtains, by induction on $n$,
functors
$V^n \colon \Omega^n \to \Sigma \wr \Sigma$.
In other words, one has extensions of the categories 
$\Omega^n$, $\Sigma \wr \Omega^n$ to objects in 
$\mathsf{WSpan}^r(\**,\Sigma \wr \Sigma \to \Sigma)$
in such a way that the vertex functors
$V \colon \Omega^n \to \Sigma \wr \Omega^{n-1}$
become $1$-arrows of $\mathsf{WSpan}^r(\**,\Sigma \wr \Sigma \to \Sigma)$.

Next, we discuss how to extend the simplicial operators. Firstly, for $d^0 \colon \Omega^1 \to \Omega^0$, this is given by the permutation isomorphism below.
\begin{equation}\label{1STPI EQ}
\begin{tikzcd}
	\Omega^1 \ar{rr}{d^0} \ar{d}[swap]{\boldsymbol{V}}&&
	|[alias=Ft]|
	\Omega^0 \ar{d}{\boldsymbol{V}}
\\
	|[alias=SOm]|
	\Sigma \wr \Omega^0 \ar{r}[swap]{\boldsymbol{V}} &
	\Sigma \wr \Sigma \wr \Sigma \ar{r}[swap]{\amalg} &
	\Sigma \wr \Sigma
\arrow[Leftrightarrow, from=SOm, to=Ft,shorten >=0.25cm,shorten <=0.25cm
,swap,"\pi"
]
\end{tikzcd}
\end{equation}

Before defining the higher simplicial operators, note that since the maps 
$\Sigma \wr d^{0,\cdots,n}
\colon \Sigma \wr \Omega^{n} \to \Sigma \wr \Sigma$
are maps of split Grothendieck fibrations over $\Sigma$,
\eqref{WSPANPULL EQ}
generalizes to give a pullback functor
\begin{equation}
\left( \Sigma \wr d^{0,\cdots,n} \right)^{\**} \colon
\mathsf{WSpan}^r(\**,\Sigma \wr \Sigma \to \Sigma)
	\to
\mathsf{WSpan}^r(\**,\Sigma \wr \Omega^{n} \to \Sigma)
\end{equation}

The remaining simplicial operators are then defined by combining two inductive cases.

First, since the functors $\boldsymbol{V}$ are $1$-arrows,
the arrows $d^i \colon \Omega^{i+1} \to \Omega^i$
in $\mathsf{WSpan}^r(\**,\Sigma \wr \Sigma \to \Sigma)$
are determined by demanding that the squares
\begin{equation}\label{RMCOMPART EQ}
\begin{tikzcd}
	\Omega^{i+1} \ar{r}{\boldsymbol{V}} 
	\ar{d}[swap]{d^{i}} &
	\Sigma \wr \Omega^{i}
	\ar{d}{d^{i-1}}
\\
	\Omega^{i} \ar{r}{\boldsymbol{V}} &
	\Sigma \wr \Omega^{i-1}
\end{tikzcd}
\end{equation}
remain commutative when regarded as diagrams in
$\mathsf{WSpan}^r(\**,\Sigma \wr \Sigma \to \Sigma)$.

Second, the operators
$d^i \colon \Omega^n \to \Omega^{n-1}$
for $n>i+1$ are defined by the diagram 
\begin{equation}\label{DIDEF EQ}
\begin{tikzcd}
	\Omega^n \ar{rr}{\boldsymbol{V}^{i+1}}[name=Vi1]{} \ar{d}[swap]{d^i} &&
	|[alias=Ft]|
	\Sigma \wr \Omega^{n-i-2} \ar{r}{\boldsymbol{V}^{n-i-2}} &
	\Sigma \wr \Sigma \wr \Sigma \ar{r}{\amalg} &
	\Sigma \wr \Sigma
\\
	|[alias=SOm]|
	\Omega^{n-1} \ar{rru}[swap]{\boldsymbol{V}^i}
\arrow[Leftrightarrow, from=SOm, to=Vi1,shorten >=0.25cm,shorten <=0.25cm
,swap,"\pi"
]
\end{tikzcd}
\end{equation}
where the leftmost section is obtained by applying
$\left(\Sigma \wr d^{0,\cdots,n-i-2} \right)^{\**}$
to \eqref{1STPI EQ}
to the arrow $d^i \colon \Omega^{i+1} \to \Omega^i$ in 
$\mathsf{WSpan}^r(\**,\Sigma \wr \Sigma \to \Sigma)$.

\begin{remark}\label{RMCOM REM}
	More generally, the squares below are also commutative squares 
	in $\mathsf{WSpan}^r(*,\Sigma \wr \Sigma \to \Sigma)$
\begin{equation}\label{RMCOM EQ}
\begin{tikzcd}
	\Omega^n \ar{r}{\boldsymbol{V}} 
	\ar{d}[swap]{d^{i}} &
	\Sigma \wr \Omega^{n-1}
	\ar{d}{d^{i-1}}
\\
	\Omega^{n-1} \ar{r}{\boldsymbol{V}} &
	\Sigma \wr \Omega^{n-2}
\end{tikzcd}
\end{equation}
Indeed, the since the composites
\begin{equation}
\mathsf{WSpan}^r(\**,\Sigma \wr \Sigma \to \Sigma)
	\xrightarrow{\left( \Sigma \wr d^{0,\cdots,n} \right)^{\**}}
\mathsf{WSpan}^r(\**,\Sigma \wr \Omega^{n} \to \Sigma)
	\xrightarrow{\Sigma \wr (-)}
\mathsf{WSpan}^r(\**,\Sigma \wr \Omega^{n} \to \Sigma)
\end{equation}
\begin{equation}
\mathsf{WSpan}^r(\**,\Sigma \wr \Sigma \to \Sigma)
	\xrightarrow{\Sigma \wr (-)}
\mathsf{WSpan}^r(\**,\Sigma \wr \Sigma \to \Sigma)
	\xrightarrow{\left( \Sigma \wr d^{0,\cdots,n} \right)^{\**}}
\mathsf{WSpan}^r(\**,\Sigma \wr \Omega^{n} \to \Sigma)
\end{equation}
are naturally isomorphic it follows that the composite 
\begin{equation}
\begin{tikzcd}
	\Omega^n \ar{d}[swap]{d^i} \ar{r}{\boldsymbol{V}} &
	\Sigma \wr \Omega^{n-1} \ar{rr}{\boldsymbol{V}^{i}}[name=Vi1]{} \ar{d}[swap]{d^{i-1}} &&
	|[alias=Ft]|
	\Sigma \wr \Sigma \wr \Omega^{n-i-2} \ar{r}{\boldsymbol{V}^{n-i-2}} &
	\Sigma \wr \Sigma \wr \Sigma \wr \Sigma \ar{r}{\amalg} &
	\Sigma \wr \Sigma \wr \Sigma \ar{r}{\amalg} &
	\Sigma \wr \Sigma
\\
	\Omega^{n-1} \ar{r}{\boldsymbol{V}} &
	|[alias=SOm]|
	\Sigma \wr \Omega^{n-2} \ar{rru}[swap]{\boldsymbol{V}^{i-1}}
\arrow[Leftrightarrow, from=SOm, to=Vi1,shorten >=0.25cm,shorten <=0.25cm
,swap,"\pi"
]
\end{tikzcd}
\end{equation}
indeed computes the natural transformation component of 
$d^i \colon \Omega^n \to \Omega^{n-1}$.
\end{remark}

To check the simplicial identities,
the lowest identity $d^0 d^1 = d^0 d^0$ for $n=2$ is the equality of the usual diagrams
\begin{equation}
\begin{tikzcd}
	\Omega^2 \ar{d}[swap]{d^0} \ar{r}{\boldsymbol{V}}&
	\Sigma \wr \Omega^1 \ar{r}{\boldsymbol{V}} &
	|[alias=Ft1]|
	\Sigma \wr \Sigma \wr \Omega^0 \ar{d}{\amalg} \ar{r}{\boldsymbol{V}} &
	\Sigma \wr \Sigma \wr \Sigma \wr \Sigma
	\ar{d}{\amalg}
\\
	|[alias=SOm1]|
	\Omega^1 \ar{rr}{\boldsymbol{V}} \ar{d}[swap]{d^0} &&
	\Sigma \wr \Omega^0 \ar{r}{\boldsymbol{V}} &
	|[alias=Ft]|
	\Sigma \wr \Sigma \wr \Sigma
	\ar{d}{\amalg}
\\
	|[alias=SOm]|
	\Omega^0 \ar{rrr}[swap]{\boldsymbol{V}} &&&
	\Sigma \wr \Sigma
\arrow[Leftrightarrow, from=SOm, to=Ft,shorten >=0.25cm,shorten <=0.25cm
,swap,"\pi"
]
\arrow[Leftrightarrow, from=SOm1, to=Ft1,shorten >=0.25cm,shorten <=0.25cm
,swap,"\pi"
]
\end{tikzcd}
\end{equation}
and
\begin{equation}
\begin{tikzcd}
	\Omega^2 \ar{d}[swap]{d^1} \ar{r}{\boldsymbol{V}} &
	\Sigma \wr \Omega^1 \ar{r}{\boldsymbol{V}} \ar{d}[swap]{d^0} &
	\Sigma \wr \Sigma \wr \Omega^0 \ar{r}{\boldsymbol{V}} &
	|[alias=Ft1]|
	\Sigma \wr \Sigma \wr \Sigma \wr \Sigma
	\ar{d}{\amalg}
\\
	\Omega^1 \ar{r}{\boldsymbol{V}} \ar{d}[swap]{d^0} &
	|[alias=SOm1]|
	\Sigma \wr \Omega^0 \ar{rr}{\boldsymbol{V}} &&
	|[alias=Ft]|
	\Sigma \wr \Sigma \wr \Sigma
	\ar{d}{\amalg}
\\
	|[alias=SOm]|
	\Omega^0 \ar{rrr}[swap]{\boldsymbol{V}} &&&
	\Sigma \wr \Sigma
\arrow[Leftrightarrow, from=SOm, to=Ft,shorten >=0.25cm,shorten <=0.25cm
,swap,"\pi"
]
\arrow[Leftrightarrow, from=SOm1, to=Ft1,shorten >=0.25cm,shorten <=0.25cm
,swap,"\pi"
]
\end{tikzcd}
\end{equation}


For the other simplicial identities,
stating that for $0\leq i < j < n$, then
$d^i d^j\colon \Omega^n \to \Omega^{n-2}$ and
$d^{j-1} d^i \colon \Omega^n \to \Omega^{n-2}$ coincide,
the cases with $j<n-1$ follow inductively on $n$
from the definition in \eqref{DIDEF EQ}
while the $0<i$ cases similarly follow inductively
from Remark \ref{RMCOM REM}.
We are left with the case $i=0$, $j=n-1$, 
which will likewise follow inductively.
Indeed, consider the following diagram
\begin{equation}\label{MOREIND EQ}
\begin{tikzcd}
	\Omega^n \ar{r}{d^0} \ar{d}[swap]{d^{n-1}} &
	\Omega^{n-1} \ar{r}{d^0} \ar{d}{d^{n-2}} &
	\Omega^{n-2} \ar{d}{d^{n-3}}
\\
	\Omega^{n-1} \ar{r}[swap]{d^0} &
	\Omega^{n-2} \ar{r}[swap]{d^0} &
	\Omega^{n-3}
\end{tikzcd}
\end{equation}
We want to show that the left square commutes as a diagram in 
$\mathsf{WSpan}^r(\**,\Sigma \wr \Sigma \to \Sigma)$.
Since we already know that the underlying diagram in $\mathsf{Cat}$ commutes, and the natural transformation components of all $d^i$ are isomorphisms (so that \eqref{MOREIND EQ} induces a diagram in 
$\mathsf{Fun}(\Omega^n, \Sigma \wr \Sigma)$ 
with the same shape and where all arrows are isomorphisms), 
it suffices to show that the rightmost and total squares 
in \eqref{MOREIND EQ} commute as diagrams in 
$\mathsf{WSpan}^r(\**,\Sigma \wr \Sigma \to \Sigma)$.
The commutativity of the the right square is direct from the induction hypothesis, and the commutativity of the total square follows similarly from the string of identities 
(as maps in $\mathsf{WSpan}^r(\**,\Sigma \wr \Sigma \to \Sigma)$)
\[
d^0 d^0 d^{n-1} = d^0 d^1 d^{n-1} =
d^0 d^{n-2} d^1 = d^{n-3} d^0 d^1 = d^{n-3} d^0 d^0
\]
which all hold due to the induction hypothesis.



\subsection{Edge and vertex compatibilities}

In this section we discuss the compatibilities between the edge and vertex functors of the previous sections.

Firstly, we note that the functors
$\boldsymbol{V}^n \colon \Omega^n \to \Sigma \wr \Sigma$
can be extended to spans in 
$\mathsf{WSpan}^r(\**,\mathsf{F})$
via the following diagram.
\begin{equation}\label{EVCOMP EQ}
\begin{tikzcd}
	\Omega^{n+1} \ar{r}{\boldsymbol{V}} 
	\ar{rrdd}[name=arr1t]{} &
	\Sigma \wr \Omega^{n} \ar{r}{\boldsymbol{V}^{n}}[name=Vi1]{} 
	\ar{rd}[name=arrt]{}[swap,name=arrb]{} &
	|[alias=Sigma3]|
	\Sigma \wr \Sigma \wr \Sigma \ar{r}{\amalg} \ar{d} &
	\Sigma \wr \Sigma \ar{ldd}
\\
	&&
	|[alias=SOm]|
	\Sigma \wr \mathsf{F} \ar{d}
\\
	&&
	\mathsf{F}
\arrow[Rightarrow, from=Sigma3, to=arrt, shorten >=0.1cm, shorten <=0.1cm
%,swap,"\pi"
]
\arrow[Rightarrow, from=arrb, to=arr1t, shorten >=0.1cm, shorten <=0.1cm
%,swap,"\pi"
]
\end{tikzcd}
\end{equation}



\begin{proposition}
For $0 \leq i \leq n$ the composite 
\begin{equation}\label{DIVN1 EQ}
\begin{tikzcd}
	|[alias=Sigma3]|
	\Omega^{n+1} \ar{rr}{\boldsymbol{V}^{n+1}}[swap,name=Vn1]{} \ar{d}[swap]{d^i} &&
	\Sigma \wr \Sigma \ar{lld}[name=arrb]{} \ar{lldd}
\\
	|[alias=Omegan]|
	\Omega^n \ar{d}
\\
	|[alias=F]|
	\mathsf{F}
\arrow[Rightarrow, from=Vn1, to=Omegan, shorten >=0.1cm, shorten <=0.1cm
%,swap,"\pi"
]
\arrow[Rightarrow, from=arrb, to=F, shorten >=0.1cm, shorten <=0.1cm
%,swap,"\pi"
]
\end{tikzcd}
\end{equation}
matches the arrow
$\boldsymbol{V}^{n+1} \colon \Omega^{n+1} \to \Sigma \wr \Sigma$
in $\mathsf{WSpan}^r(\**,\mathsf{F})$.
\end{proposition}


\begin{proof}
The base case is that of $n=1$, $i=0$, which states that the composite natural transformations in the following two diagrams coincide,
and this is readily verified explicitly.
\begin{equation}
\begin{tikzcd}
	\Omega^{1} \ar{r}{\boldsymbol{V}} \ar{d}[swap]{d^0}
	&
	\Sigma \wr \Omega^{0} \ar{r}{\boldsymbol{V}}{} 
	\ar{rd}[name=arrt]{}[swap,name=arrb]{} &
	|[alias=Sigma3]|
	\Sigma \wr \Sigma \wr \Sigma \ar{r}{\amalg} \ar{d} &
	\Sigma \wr \Sigma \ar{ldd}
&
	\Omega^{1} \ar{r}{\boldsymbol{V}} \ar{d}[swap]{d^0}
	&
	\Sigma \wr \Omega^{0} \ar{r}{\boldsymbol{V}}[name=Vi1]{} 
	&
	\Sigma \wr \Sigma \wr \Sigma \ar{r}{\amalg} &
	\Sigma \wr \Sigma \ar{ldd}
\\
	\Omega^0 \ar{rrd}[name=arr1t]{} &&
	|[alias=SOm]|
	\Sigma \wr \mathsf{F} \ar{d} &
&
	\Omega^0 \ar{rrru}[name=diagu]{}[swap,name=diagb]{} \ar{rrd}[name=OF]{} &&
	 &
\\
	&&
	\mathsf{F} &
&
	&&
	\mathsf{F} &
\arrow[Rightarrow, from=Sigma3, to=arrt, shorten >=0.1cm, shorten <=0.1cm
%,swap,"\pi"
]
\arrow[Rightarrow, from=arrb, to=arr1t, shorten >=0.1cm, shorten <=0.1cm
%,swap,"\pi"
]
\arrow[Rightarrow, from=Vi1, to=diagu, shorten >=0.1cm, shorten <=0.1cm
%,swap,"\pi"
]
\arrow[Rightarrow, from=diagb, to=OF, shorten >=0.1cm, shorten <=0.1cm
%,swap,"\pi"
]
\end{tikzcd}
\end{equation}
The other cases are verified inductively.
Thanks to the simplicial identities 
$d^i d^j = d^{j-1} d^i$ for $i<j$, 
one reduces to the case of
$d^n \colon \Omega^{n+1} \to \Omega^n$.

Recall now that $\boldsymbol{V}^{n+1}$ is defined by \eqref{EVCOMP EQ}, and that by the commutativity of \eqref{VASEDGE EQ}, we can rewrite \eqref{EVCOMP EQ} as given by the left diagram below.
\begin{equation}
\begin{tikzcd}
	\Omega^{n+1} \ar{r}{\boldsymbol{V}} 
	\ar{d}[swap]{d^n} &
	\Sigma \wr \Omega^{n} \ar{r}{\boldsymbol{V}^{n}}[name=Vi1]{} 
	\ar{d}[swap]{d^{n-1}} &
	|[alias=Sigma3]|
	\Sigma \wr \Sigma \wr \Sigma \ar{r}{\amalg} \ar{dd} &
	\Sigma \wr \Sigma \ar{ldd}
&
	\Omega^{n+1} \ar{r}{\boldsymbol{V}} 
	\ar{d}[swap]{d^n} &
	\Sigma \wr \Omega^{n} \ar{r}{\boldsymbol{V}^{n}}[swap,name=Vi12]{} 
	\ar{d}[swap]{d^{n-1}} &
	|[alias=Sigma32]|
	\Sigma \wr \Sigma \wr \Sigma \ar{r}{\amalg} \ar{dd} &
	\Sigma \wr \Sigma \ar{ldd}
\\
	\Omega^n \ar{rrdd}[name=arr1t]{} 
	\ar{r}{\boldsymbol{V}} &
	\Sigma \wr \Omega^{n-1} \ar{rd}[name=arrt]{}[swap,name=arrb]{} &&
&
	\Omega^n \ar{rrdd}[name=arr1t2]{} 
	\ar{r}{\boldsymbol{V}} &
	|[alias=SOn1]|
	\Sigma \wr \Omega^{n-1} \ar{rd}[name=arrt2]{}[swap,name=arrb2]{}
	\ar{ru}
\\
	&&
	|[alias=SOm]|
	\Sigma \wr \mathsf{F} \ar{d} &
&
	&&
	|[alias=SOm]|
	\Sigma \wr \mathsf{F} \ar{d} &
\\
	&&
	\mathsf{F} &
&
	&&
	\mathsf{F} &
\arrow[Rightarrow, from=Sigma3, to=arrt, shorten >=0.1cm, shorten <=0.1cm
%,swap,"\pi"
]
\arrow[Rightarrow, from=arrb, to=arr1t, shorten >=0.1cm, shorten <=0.1cm
%,swap,"\pi"
]
\arrow[Rightarrow, from=Sigma32, to=arrt2, shorten >=0.1cm, shorten <=0.1cm
%,swap,"\pi"
]
\arrow[Rightarrow, from=arrb2, to=arr1t2, shorten >=0.1cm, shorten <=0.1cm
%,swap,"\pi"
]
\arrow[Leftrightarrow, from=Vi12, to=SOn1, shorten >=0.1cm, shorten <=0.1cm
%,swap,"\pi"
]
\end{tikzcd}
\end{equation}
The left diagram above then matches the right diagram due to the induction hypothesis.
Lastly, the right diagram matches \eqref{DIVN1 EQ}
by combining 
\eqref{EVCOMP EQ}
with the commutative squares \eqref{RMCOM EQ}.
\end{proof}



\section{2-categories of spans}

{\color{red} HERE}

\subsection{$2$-overcategories}


\begin{definition}
Let $\mathcal{C}$ be a $2$-category and $c \in \mathcal{C}$ an object.
We write $\mathcal{C} \downarrow^r c$ for the $2$-category such that:
\begin{itemize}
	\item objects are the arrows $\beta \colon b \to c$;
	\item a $1$-arrow from 
	$\beta \colon b \to c$
	to
	$\beta' \colon b' \to c$ 
	is a pair $(f,\phi)$
	formed by a $1$-arrow $f\colon b \to b'$ and a $2$-arrow
	$\phi \colon \beta' f \Rightarrow \beta$
		\begin{equation}
		\begin{tikzcd}[row sep = tiny, column sep = 35pt]
			b \arrow{dr}[name=U]{\beta} \arrow{dd}[swap]{f}
		\\
			& c
		\\
			|[alias=V]| b' \arrow{ur}[swap]{\beta'}
		\arrow[Rightarrow, from=V, to=U,shorten >=0.25cm,shorten <=0.25cm
		,swap,"\phi"
		]
		\end{tikzcd}
		\end{equation}
	\item a $2$-arrow from $(f,\phi)$ to $(f',\phi')$ is a $2$-arrow $\varphi \colon f \to f'$ such that
	$\phi' \circ \varphi \beta' = \phi$.
		\begin{equation}
		\begin{tikzcd}[column sep = 50pt]
			b \arrow{dr}[name=U]{\beta} 
			\arrow[bend right]{dd}[swap]{f}[name=F]{}
			\arrow[bend left]{dd}{f'}[swap,name=FF]{}
			&
		&
			b \arrow{dr}[name=U2]{\beta} 
			\arrow[bend right]{dd}[swap]{f}
			&
		\\
			& c
		&
			& c
		\\
			b' \arrow{ur}[swap]{\beta'}[near start, name=V]{}
			&
		&
			|[alias=V2]| b' \arrow{ur}[swap]{\beta'}
			&
		\arrow[Rightarrow, from=V, to=U,shorten >=0.25cm,shorten <=0.25cm
		,swap,"\phi'"
		]
		\arrow[Rightarrow, from=F, to=FF,shorten >=0.0cm,shorten <=0.0cm
		,swap,"\varphi"
		]
		\arrow[Rightarrow, from=V2, to=U2,shorten >=0.25cm,shorten <=0.25cm
		,swap,"\phi"
		]
		\end{tikzcd}
		\end{equation}
\end{itemize}
\end{definition}

\begin{example}
When $\mathcal{C}$ is a $1$-category, then 
$\mathcal{C} \downarrow^r c$ is the usual overcategory.
\end{example}

\begin{example}
When $\mathcal{C} = \mathsf{Cat}$ is the $2$-category of categories, 
then the underlying $1$-category of
the $2$-category
$\mathsf{Cat} \downarrow^r \mathcal{B}$
coincides with the category of weak right spans
$\mathsf{WSpan}^r(\**,\mathcal{B})$.
\end{example}


We will find it useful to discuss iterations of the 
$2$-overcategory construction described above.
Given a $1$-arrow in $\mathsf{Cat}$ (i.e. a functor) $\mathcal{B} \to \mathsf{F}$, we write
$\mathsf{Cat} \downarrow^r (\mathcal{B} \to \mathsf{F})$
as an abbreviation for the iterate $2$-overcategory
$\left(\mathsf{Cat} \downarrow^r \mathsf{F} \right) \downarrow^r (\mathcal{B} \to \mathsf{F}) $.
Unpacking definitions, the objects of 
$\mathsf{Cat} \downarrow^r (\mathcal{B} \to \mathsf{F})$
are quadruples $(\mathcal{C}, B, F, \gamma)$ as below,
\begin{equation}
	\begin{tikzcd}[row sep = tiny, column sep = 35pt]
		&
		|[alias=V]|
		\mathcal{B} \arrow{dd} 
	\\
		\mathcal{C} \arrow{ur}{B} 
		\arrow{dr}[swap,name=U]{F} & 
	\\
		&
		\mathsf{F}
	\arrow[Rightarrow, from=V, to=U,shorten >=0.25cm,shorten <=0.25cm
	,"\gamma"
	]
	\end{tikzcd}
\end{equation}
a $1$-arrow from 
$(\mathcal{C}, B, F, \gamma)$ to 
$(\mathcal{C}', B', F', \gamma')$
is a triple $(f,\varphi, \epsilon)$ as below such that then composite natural transformations in the two diagrams coincide,
\begin{equation}
\begin{tikzcd}[column sep = 50pt,row sep = 50pt]
	|[alias=VV]|
	\mathcal{C}' \ar{r}{B'} &
	|[alias=V]|
	\mathcal{B} \ar{d} &
&
	|[alias=VVVV]|
	\mathcal{C}' \ar{r}{B'} \ar{rd}[swap,near end]{F'}[name=UUU]{} &
	|[alias=VVV]|
	\mathcal{B} \ar{d} &
\\
	\mathcal{C} \ar{r}[swap]{F}[swap,name=U]{} \ar{u}{f} \ar{ru}[near start,swap]{B}[name=UU]{}&
	\mathsf{F} &
&
	\mathcal{C} \ar{r}[swap]{F}[name=UUUU]{} \ar{u}{f} &
	\mathsf{F} &
	\arrow[Rightarrow, from=V, to=U, shorten >=0.25cm,shorten <=0.25cm
	,"\gamma"
	]
	\arrow[Rightarrow, from=VV, to=UU, shorten >=0.15cm,shorten <=0.15cm,
	swap,"\varphi"
	]
	\arrow[Rightarrow, from=VVV, to=UUU, shorten >=0.15cm,shorten <=0.15cm
	,"\gamma'"
	]
	\arrow[Rightarrow, from=VVVV, to=UUUU, shorten >=0.25cm,shorten <=0.25cm,
	swap
	,"\epsilon"
	]
\end{tikzcd}
\end{equation}
and a natural transformation from  
$(f,\varphi, \epsilon)$ to
$(f',\varphi', \epsilon)'$
is a natural transformation 
$\rho \colon f \to f'$ such that
$\varphi' \circ \rho B' = \varphi$ and
$\epsilon' \circ \rho F' = \epsilon$


{\color{red} HERE} 

\bibliography{biblio}{}





\bibliographystyle{alpha}



\end{document}