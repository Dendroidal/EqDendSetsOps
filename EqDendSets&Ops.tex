\documentclass[a4paper,10pt
,draft
]{article}%


\usepackage[hidelinks]{hyperref}
\hypersetup{
  % colorlinks,
  final,
  pdftitle={Equivariant Dendroidal Segal Spaces},
  pdfauthor={Bonventre, P. and Pereira, L. A.},
  % pdfsubject={Your subject here},
  % pdfkeywords={keyword1, keyword2},
  linktoc=page
}
\usepackage[open=false]{bookmark}

\usepackage[draft]{showkeys}

\input{commands.tex}%




%-------- TIKZ -----------------------------------------
\usepackage{tikz}%
\usetikzlibrary{matrix,arrows,decorations.pathmorphing,
cd,patterns,calc}
\tikzset{%
  treenode/.style = {shape=rectangle, rounded corners,%
                     draw, align=center,%
                     top color=white, bottom color=blue!20},%
  root/.style     = {treenode, font=\Large, bottom color=red!30},%
  env/.style      = {treenode, font=\ttfamily\normalsize},%
  dummy/.style    = {circle,draw,inner sep=0pt,minimum size=2mm}%
}%

\usetikzlibrary[decorations.pathreplacing]
% \usetikzlibrary{external}\tikzexternalize
% \makeatletters
% \renewcommand{\todo}[2][]{\tikzexternaldisable\@todo[#1]{#2}\tikzexternalenable}

% \makeatother


% -------- COMMANDS ON DRAFT--------------------------

\usepackage{ifdraft}
\ifdraft{
  \color[RGB]{63,63,63}
  % \pagecolor[rgb]{0.5,0.5,0.5}
  \pagecolor[RGB]{220,220,204}
  % \color[rgb]{1,1,1}
}


\usepackage{todonotes}%[obeyDraft]


% -------- Reference Numbering
% \numberwithin{equation}{section}%
% \numberwithin{figure}{section} 


% ------- Author Info -----------------------

\author{Peter Bonventre, Lu\'is A. Pereira}%
\title{Untitled}%
\date{\today}


% ------- Commands --------------------------------
\newcommand{\mycircled}[2][none]{%
  \mathbin{
    \tikz[baseline=(a.base)]\node[draw,circle,inner sep=-1.5pt, outer sep=0pt,fill=#1](a){\ensuremath #2\strut};
  }
}
\newcommand{\owr}{\mycircled{\wr}}
\newcommand{\UV}{\underline{\mathcal V}}
\renewcommand{\phi}{\varphi}
\newcommand{\UC}{\underline{\mathfrak C}}


%----- Document ---------------------------------

\begin{document}

\maketitle

\begin{abstract}
      Things and stuff
\end{abstract}



\tableofcontents




% ----------STUFF-----------------------------------------



\section{Overview}


\[
	\begin{tikzcd}
		\mathsf{PreOp}^G
\\
		\mathsf{sdSet}^G \ar{r}[swap]{(-)_0} \ar{u}{\gamma_{\**}} &
		\mathsf{dSet}^G
	\end{tikzcd}
\]






\section{Coloured Operads}


\subsection{Non-Equivariant Coloured Operads}


Fix a closed symmetric monoidal category $\V$.

\begin{definition}
      Fix a set $\mathfrak C$ of \textit{colours}.
      A tuple
      $\ksi = (c_1, \ldots, c_n; c_0) \in \mathfrak C^{\times n} \times \mathfrak C$
      is called a \textit{signature} of $\mathfrak C$, and let $|\ksi|$ denote the length $n$
      (so $\ksi \in \mathfrak C^{\times |\ksi| + 1}$).
      
      A \textit{$\mathfrak C$-coloured operad} in $\V$ consists of the following data:
      \begin{enumerate}%\itemsep-4pt
      \item An object $\O(\ksi) \in \V$ for each signature $\ksi$.
      \item For each $c \in \mathfrak C$, a \textit{unit} $1_c \in \O(c;c)$.
      \item For any signature $\ksi \in \mathfrak C^{\times n+1}$ and $\sigma \in \Sigma_n$, a map $\O(\ksi) \to \O(\sigma \cdot \ksi)$,
            where $\Sigma_n$ acts on the left of $\mathfrak C^{\times n+1}$ by acting on the first $n$ coordinates.
            Explicitly, this is a map
            \begin{equation}
                  \O(c_1, \ldots, c_n; c_0) \xrightarrow{\sigma} \O(c_{\sigma^{-1}1}, \ldots, c_{\sigma^{-1}n}; c_0).
            \end{equation}
      \item For any compatible signatures $\ksi = (c_1, \ldots, c_n; c_0)$, $\ksi_i = (c_{i,1}, \ldots, c_{i,m_i}; c_i)$, a \textit{composition} map
            \begin{equation}
                  \O(\ksi) \times \O(\ksi_1) \times \ldots \times \O(\ksi_n) \to \O(c_{1,1}, \ldots, c_{n,m_n}; c_0)
            \end{equation}
      \end{enumerate}
      subject to all the compatibilities you'd expect.

      A map of $\mathfrak C$-coloured operads is a compatible collection of maps
      $\set{\O(\ksi) \to \O'(\ksi)}_{\ksi}$.
      
      Let $\Op^{\mathfrak C}(\V)$ denote the category of $\mathfrak C$-coloured operads in $\V$.
\end{definition}

\begin{definition}
      Given a map $f: \mathfrak C' \to \mathfrak C$ and a $\mathfrak C$-coloured operad $\O$,
      there is a natural $\mathfrak C'$-coloured operad $f^*(\O)$, where
      \begin{equation}
            f^{\**}(\O)(c'_1, \ldots, c'_n; c'_0) = \O(f(c'_1), \ldots, f(c'_n); f(c'_0)).
      \end{equation}

      A \textit{map of coloured operads} $\O' \to \O$ is given by the data of a map of colours $f: \mathfrak C' \to \mathfrak C$,
      and a map of $\mathfrak C'$-coloured operads $\O' \to f^*(\O)$.
      
      Let $\Op(\V)$ denote the category of coloured operads in $\V$.
\end{definition}

\begin{remark}
      The category $\Op(\V)$ is isomorphic to the Grothendieck construction on the functor
      \begin{equation}
            \begin{tikzcd}[row sep = tiny]
                  \mathsf F \arrow[r] & \mathsf{Cat}
                  \\
                  \mathfrak C \arrow[r, mapsto] & \Op^{\mathfrak C}(\V).
            \end{tikzcd}
      \end{equation}
\end{remark}

\begin{notation}
      In previous work \todo{cite}, $\Op(\V)$ has been used to denote \textit{single-coloured} operads specifically; that is, $\set{\**}$-coloured operads.
      For this article, we will write these as $\Op^{\set{\**}}(\V)$. 
\end{notation}


\subsection{Equivariant Coloured Operads}

\begin{definition}
      The category $\Op^G(\V)$ of  \textit{$G$-coloured operads} in $\V$ is the category of
      $G$-objects in $\Op(\V)$.
\end{definition}


\begin{remark}
      Unpacking this definition, we see $\O \in \Op^G(\V)$ consists of the following data:
      \begin{enumerate}
      \item A $G$-set $\mathfrak C$ of colours.
      \item For each signature $\ksi$ of $\mathfrak C$, an object $\O(\ksi) \in \V$.
      \item For each signature $\ksi \in \mathfrak C^{\times n+1}$ and $(g,\sigma) \in G\times \Sigma_n$, a map
            $\O(\ksi) \to \O((g,\sigma)\cdot \ksi)$,
            where $G$ acts on $\mathfrak C^{\times n+1}$ diagonally (across all $n+1$ coordinates), and $\Sigma_n$ acts on the first $n$.
      \item For each $c \in \mathfrak C$, a \textit{unit} $1_c \in \O(c;c)^{G_c}$, where $G_c$ is the stabilizer of $c$.
      \item For compatible signatures $\ksi$, $\ksi_1$, $\ldots$, $\ksi_n$, \textit{composition maps}
            \begin{equation}
                  \O(\ksi) \otimes \O(\ksi_1) \otimes \ldots \otimes \O(\ksi_n) \to \O(\ksi \circ (\ksi_1, \ldots, \ksi_n)),
            \end{equation}
      \end{enumerate}
      such that composition is
      compatible with the $G$-action on each component as well as the appropriate actions of $\Sigma$,
      and is unital and associative. 
\end{remark}

\begin{remark}
      Unlike in the single-coloured case, this is \textit{not} the same as coloured operads in $\V^G$.
      Indeed, objects in $\Op(\V^G)$ have a $G$-fixed set of colours, and each level $\O(\ksi)$ is a full $G$-set
      (though only a partial $\Sigma_{|\ksi|}$-set).
\end{remark}

\begin{definition}
      Given a $G$-set $\mathfrak C$, let $\Op^{G,\mathfrak C}(\V)$ denote the category of \textit{$\mathfrak C$-coloured operads} and maps which are the identity on colours.

      Parallel to the non-equivariant case, $\Op^G(\V)$ is isomorphic to the Grothendieck construction on the functor
            \begin{equation}
            \begin{tikzcd}[row sep = tiny]
                  \mathsf F^G \arrow[r] & \mathsf{Cat}
                  \\
                  \mathfrak C \arrow[r, mapsto] & \Op^{G,\mathfrak C}(\V).
            \end{tikzcd}
      \end{equation}
      
\end{definition}

\subsubsection{Categorical Description}

\begin{definition}
      Given a $G$-set $X$, let $B_XG$ denote the \textit{translation category} of $X$,
      with object set $X$ and morphisms $g: x \to g\cdot x$ for all pairs $(g,x) \in G \times X$.

      We will denote $B_{\set{\**}}G$ by $\mathsf G$.
\end{definition}

\begin{remark}
      We observe that we have a natural diagonal map
      \begin{equation}
            F \times \mathsf G \into \mathsf F \wr \mathsf G,
      \end{equation}
      and so for any functor $F: \mathcal C \to \mathsf F$, we have an induced functor
      $F: \mathcal C \times \mathsf G \to \mathsf F \wr \mathsf G$. 
\end{remark}

\begin{definition}
      Let $\mathfrak C \Sigma$ be the category
      \begin{equation}
            \mathfrak C \Sigma = \coprod\limits_{n\geq 0} B_{\mathfrak C^{\times n} \times \mathfrak C}(G \times \Sigma_n).
      \end{equation}
      We note that $\mathrm{Ob}(\mathfrak C \Sigma)$ is precisely the set of \textit{signatures} in $\mathfrak C$.
      Further, we observe that this is equivalent to the pullback
      \begin{equation}
            \label{CSIGMA_EQ}
            \begin{tikzcd}
                  \mathfrak C \Sigma \arrow[d] \arrow[r, "E"]
                  &
                  \mathsf F \wr B_{\mathfrak C}G \arrow[d]
                  \\
                  \Sigma \times \mathsf G \arrow[r, "E"]
                  &
                  \mathsf F \wr \mathsf G
            \end{tikzcd}
      \end{equation}
      where $E: \Omega \to \mathsf F$ sends a tree to its set of edges.
      \todo[inline]{$B_{\mathfrak C}G = G \ltimes \mathfrak C$}
      
      More generally, let $\mathfrak C \Omega$ be the pullback
      \begin{equation}
            \label{COMEGA_EQ}
            \begin{tikzcd}
                  \mathfrak C \Omega \arrow[d] \arrow[r, "E"]
                  &
                  \mathsf F \wr B_{\mathfrak C}G \arrow[d]
                  \\
                  \Omega \times \mathsf G \arrow[r, "E"]
                  &
                  \mathsf F \wr \mathsf G
            \end{tikzcd}
      \end{equation}

      We have a natural inclusion of categories $\mathfrak C \Sigma \into \mathfrak C \Omega$.
      Moreover, we will called elements of these categories
      \textit{coloured trees} (or \textit{coloured corollas}),
      and denote them by $(T,\mathfrak c)$, where $\mathfrak c: E(T) \to \mathfrak C$ is a map of sets.
\end{definition}

\begin{remark}
      Unpacking definitions, we see that a map $(T, \mathfrak c) \to (S, \mathfrak d)$ is given by
      a map $f: T \to S$ in $\Omega$ and an element $g\in G$,
      such that $g.\mathfrak c(e) = \mathfrak d(f(e))$ for all $e \in E(T)$.
      \begin{equation}
            \begin{tikzcd}
                  E(T) \arrow[r, "f"] \arrow[d, "\mathfrak c"']
                  &
                  E(S) \arrow[d, "\mathfrak d"]
                  \\
                  \mathfrak C \arrow[r, "g"]
                  &
                  \mathfrak C
            \end{tikzcd}
      \end{equation}
      
      In particular, we have maps of the form
      \begin{equation}
            g = (id, g): (T, E(T) \to \mathfrak C) \to (T, E(T) \to \mathfrak C \xrightarrow{g \cdot} \mathfrak C). 
      \end{equation}
\end{remark}

\begin{remark}
      $\mathfrak C \Omega$ is equivalent to the
      Grothendieck construction on the functor
      \begin{equation}
            \begin{tikzcd}[row sep = tiny]
                  \Omega^{op} \times G \arrow[r]
                  &
                  \mathsf{Cat}
                  \\
                  T \arrow[r, mapsto]
                  &
                  \mathsf{Fun}(E(T), \mathfrak C).
            \end{tikzcd}
      \end{equation}
      \todo[inline]{compare with genuine case: RHS equals $\mathsf{Fun}(\Phi(E(T)), \mathfrak C) = \mathsf{Fun}(\Phi(E(G \cdot T)), \mathfrak C))$}
      and a similar result holds for $\mathfrak C \Sigma$. 
\end{remark}

\begin{remark}
      Note that we can replace the $G$-set $\mathfrak C$ with a \textit{coefficient system} $\underline{\mathfrak C}$,
      substituting the rectangle of pullbacks below for \eqref{COMEGA_EQ}
      \begin{equation}
            \begin{tikzcd}
                  \underline{\mathfrak C}\Omega \arrow[d] \arrow[r, "E"]
                  &
                  \mathsf F \wr B_{\mathfrak C(G/e)}G \arrow[r] \arrow[d]
                  &
                  \mathsf F \wr \underline{\mathfrak C} \arrow[d]
                  \\
                  \Omega \times G \arrow[r, "E"]
                  &
                  \mathsf F \wr G \arrow[r]
                  &
                  \mathsf F \wr O_G
            \end{tikzcd}
      \end{equation}
      with $\mathfrak C(G/e) \into \underline{\mathfrak C}$ and $G \into O_G$ the natural inclusions.
      \todo[inline]{compare $B_{\mathfrak C(G/e)}G = G \ltimes \mathfrak C(G/e)$ and $\underline{\mathfrak C} = O_G \ltimes \underline{\mathfrak C}$.}
      In this case, $\underline{\mathfrak C}\Omega = \mathfrak C(G/e)\Omega$.
\end{remark}

Many of the natural functors around $\Omega$ and $\Sigma$ have generalizations to the coloured setting,
which can be built through a straightforward use of the universal property of pullbacks.

\begin{definition}
      We have a natural \textit{vertex} functor
      $V: \mathfrak C \Omega \to \Sigma \wr \mathfrak C \Sigma$,
      as colourings of a tree restrict to colourings of each vertex corolla.

      Similarly, there is a \textit{leaf-root} funct or
      $\mathsf{lr}: \mathfrak C \Omega \to \mathfrak C \Sigma$,
      where the colouring of $\mathsf{lr}(T)$ is a restrict of the colouring of $T$.
\end{definition}

\begin{definition}
      The category $\Sym^{G,\mathfrak C}$ of \textit{symmetric $(G,\mathfrak C)$-sequences} is
      the category of functors $X: \mathfrak C \Sigma^{op} \to \V$.
\end{definition}

\begin{definition}
      Given $X \in \Sym^{G, \mathfrak C}$, let $\mathbb F^{\mathfrak C} X$ denote the left Kan extension below.
      \begin{equation} 
           \begin{tikzcd}
                  \mathfrak C \Omega^{op}
                  \arrow[d, "\mathsf{lr}"']
                  \arrow[r, "V"]
                  &
                  (\Sigma \wr \mathfrak C \Sigma)^{op} \arrow[r, "X"]
                  \arrow[dl, Rightarrow]
                  &
                  (\Sigma \wr \V^{op})^{op} \arrow[r, "\otimes"]
                  &
                  \V
                  \\
                  \mathfrak C \Sigma^{op} \arrow[urrr, "\Lan = \mathbb F^{\mathfrak C} X"']
            \end{tikzcd}
      \end{equation}
\end{definition}


\subsection{Single-Coloured Operads}
We first show that this generalizes the free single-coloured operad monad.
When $\mathfrak C = \set{\**}$, we have
$\mathfrak C \Omega = \Omega \times G$, and similarly
$\mathfrak C \Sigma = \Sigma \times G$.

\begin{notation}
      Given $X \in \mathsf{Cat}(\C, \mathsf{Fun}(\mathcal D, \V))$,
      let $\tilde X$ denote the adjoint functor in the isomorphic category $\mathsf{Cat}(\C \times \mathcal D, \V)$.
\end{notation}

\begin{lemma}
      \label{SPAN_LAN_LEM}
      Conisder the two spans below.
      \begin{equation}
            \begin{tikzcd}
                  \C \arrow[d, "p"] \arrow[r, "X"]
                  &
                  \mathsf{Fun}(\mathcal D, \V)
                  &&
                  \C \times \mathcal D \arrow[d, "p \times \mathsf{id}"] \arrow[r, "\tilde X"]
                  &
                  \V
                  \\
                  \mathcal E
                  &
                  &&
                  \mathcal E \times \mathcal D
            \end{tikzcd}
      \end{equation}
      
      Then $\Lan_p X$ is adjoint to $\Lan_{p \times \mathsf{id}} \tilde X$. 
\end{lemma}
\begin{proof}
      We have
      \begin{align}
        \widetilde{\Lan_p X}(e,d)
        &= (\Lan_p X(e))(d)
          = \left(
          \colim\limits_{\substack{ \C \downarrow e \\ p(c) \to e}} X(c)
        \right)(d)
        = \colim\limits_{\substack{ \C \downarrow e \\ p(c) \to e}}(X(c)(d))
        = \colim\limits_{\substack{ \C \downarrow e \\ p(c) \to e}}(\tilde X(c,d))\\
        &= \colim\limits_{\substack{ \C \times \set{d} \downarrow (e,d) \\ p(c) \to e}}(\tilde X(c,d))
        \cong \colim\limits_{\substack{ \C \times \mathcal D \downarrow (e,d) \\ (p(c),d') \to (e,d)}}(\tilde X(c,d'))
        = \Lan_{p \times \mathsf{id}}\tilde X(c,d),
      \end{align}
      where the isomorphism holds by a straightforward finality argument.
      On maps, a similar argument holds.
\end{proof}

\begin{notation}[\cite{BP17}]
      Let $\mathbb F'$ denote the \textit{free single-coloured operad monad} on $\V$, given by the left Kan extension of the following diagram.
      \begin{equation}
            \begin{tikzcd}
                  \Omega^{op}
                  \arrow[d, "\mathsf{lr}"']
                  \arrow[r, "V"]
                  &
                  (\Sigma \wr \Sigma)^{op} \arrow[r, "X"]
                  \arrow[dl, Rightarrow]
                  &
                  (\Sigma \wr \V^{op})^{op} \arrow[r, "\otimes"]
                  &
                  \V
                  \\
                  \Sigma^{op} \arrow[urrr, "\Lan = \mathbb F' X"']
            \end{tikzcd}
      \end{equation}
\end{notation}

\begin{proposition}
      $\mathbb F^{\set{\**}}$ is a monad, and moreover
      the category of $\mathbb F^{\set{\**}}$-algebras in $\mathsf{Fun}(\Sigma \times G, \V)$ is equivalent to
      the category of $\mathbb F'$-algebras in $\mathsf{Fun}(\Sigma, \V^G)$.
\end{proposition}
\begin{proof}
      Let $\tau: \tilde X \mapsto X$ denote the isomorphism of categories
      $\mathsf{Fun}(\Sigma \times G, \V) \xrightarrow{\tau} \mathsf{Fun}(\Sigma, \V^G)$.
      Then $\mathbb F^{\set{\**}} = \tau^{-1} \mathbb F' \tau$ by \ref{SPAN_LAN_LEM}, and so
      $\mathbb F^{\set{\**}}$ is in fact a monad, and the
      the isomorphism lifts to an isomorphism on the category of algebras.
\end{proof}


\subsection{General Case}

\begin{theorem}
      For every $G$-set $\mathfrak C$, $\mathbb F^{\mathfrak C}$ is a monad, with category of algebras given by $\Op^{G,\mathfrak C}(\V)$. 
\end{theorem}
\begin{proof}
      \todo[inline]{This will be a corollary of Genuine Coloured stuff}
\end{proof}






\newpage

\section{Coloured Genuine Equivariant Operads}

Throughout this section, we will abuse notation, and refer to
a coefficient system and its associated (Grothendieck) category over $O_G$ by the same name.

Idea: we have a \textit{coefficient system} $\UC$ of colours, and
a \textit{signature} will consist of a tuple $\ksi = (x_1, \dots, x_n;x_0)$
with $x_i \in \mathfrak C(G/H_i)$ for subgroups $H_i \leq H_0 \leq G$.

\subsection{Coloured $G$-Trees}

\newcommand{\CS}{\underline{\mathfrak C} \Sigma_G}
\newcommand{\CO}{\underline{\mathfrak C} \Omega_G}


\begin{definition}
      The \textit{edge orbit} functor $E_G: \Omega_G \to \mathsf F \wr O_G$ sends a $G$-tree $T$ to the tuple $(E_G(T), (G/G_e)_{Ge \in E_G(T)})$
      with $G_e$ denoting $\mathrm{Stab}_G(e)$, and
      where we have canonical representatives for elements in $E_G(T)$ by choosing
      $e \in Ge$ minimal with respect to the planar structure on $T$.
\end{definition}

\begin{definition}
      Let $\underline{\mathfrak C}$ be a $G$-coefficient system of sets.
      Then the category $\underline{\mathfrak C}\Omega_G$ of \textit{$\underline{\mathfrak C}$-coloured $G$-trees}
      is defined to be the pullback below.
      \begin{equation}
            \label{COMEGA_G_EQ}
            \begin{tikzcd}
                  \CO \arrow[d] \arrow[r]
                  &
                  \mathsf F \wr \underline{\mathfrak C} \arrow[d]
                  \\
                  \Omega_G \arrow[r, "E_G"]
                  &
                  \mathsf F \wr O_G
            \end{tikzcd}
      \end{equation}
      The category $\CS$ of $\underline{\mathfrak C}$-coloured corollas is the subcategory defined similarly,
      with $\Omega_G$ replaced with $\Sigma_G$.
\end{definition}


Explicitly, objects of $\CO$ are pairs $(T, \mathfrak c)$ of
a $G$-tree $T$ and
a map $\mathfrak c: E_G(T) \to \underline{\mathfrak C}$ over $O_G$.
That is, each orbit of edges $Ge$ (with $e$ minimal)
is assigned a ``colour'' $\mathfrak c(Ge) \in \underline{\mathfrak C}(G/G_{e})$.
Morphisms $(T, \mathfrak c) \to (S, \mathfrak d)$
are given by maps of trees $\phi: T \to S$ such that, for every edge orbit $Ge$ of $T$, we have
\begin{equation}
      \mathfrak c(Ge) = \phi_{e}^{\**}g_e^{\**}\mathfrak d(Gf),
\end{equation}
where $\phi_{e}: G / G_{e} \to G / G_{\phi(e)}$ is the map in $O_G$ induced by $\phi$,
and $\phi(e) = g_e f$ for $f \in Gf \in E_G(S)$ minimal; as $g_e$ is unique modulo $G_f$, $g_e^{\**}$ is well-defined.


\begin{remark}
      { \color{blue} % -------------------- ALTERNATIVELY: ----------------------------------------
        Alternatively,
        consider the Grothendieck construction on the functor
        \begin{equation}
              \begin{tikzcd}[row sep = tiny]
                    \mathsf F^{G,op} \arrow[r]
                    &
                    \mathsf{Set}
                    \\
                    A \arrow[r, mapsto]
                    &
                    \Set^{O_G^{op}}(\Phi(A), \underline{\mathfrak C}),
              \end{tikzcd}
        \end{equation}
        where $\Phi: \Set^G \to \Set^{O_G^{op}}$ sends a $G$-set $X$ to its fixed-point system $G/H \mapsto X^H$.
        We will denote this by $\mathsf F^G \wr \underline{\mathfrak C}$.
        Then $\CO$ is also isomorphic to the pullback
        \begin{equation}
              \begin{tikzcd}
                    \CO \arrow[d] \arrow[r]
                    &
                    \mathsf F^G \wr \underline{\mathfrak C} \arrow[d]
                    \\
                    \Omega_G \arrow[r, "E"]
                    &
                    \mathsf F^G.
              \end{tikzcd}
        \end{equation}
        \todo[inline]{We note that the class of morphisms in $\mathsf F^G$ in the image of $E$ (restricted to $\Omega_G^0$)
          are those isomorphic to an adjunction counit $G \cdot_H A|_H \to A$.}
        In this case, a colouring is a map $\mathfrak c: \Phi E(T) \to \mathfrak C$ of coefficient systems,
        and morphisms are maps $\phi: T \to S$ such that
        $\mathfrak c(G/H,e) = \mathfrak d(G/H,e)$ for all $e \in E(T)^H$.
        \begin{equation}
              \begin{tikzcd}
                    \Phi A \arrow[rr, "f"] \arrow[dr, "\mathfrak c"']
                    &&
                    \Phi B \arrow[dl, "\mathfrak d"]
                    \\
                    &
                    \underline{\mathfrak C}
              \end{tikzcd}
        \end{equation}
        It is easy to show this is equivalent to requiring that
        $\mathfrak c(G/G_e,e) = \phi_e^{\**} \mathfrak d(G/G_{\phi(e)}, \phi(e))$.
      }
      %%%%%%%%%%%%% COLOR: BLUE ----------------------------------------
      \todo[inline]{figure out whether first or ``alternatively'' is more useful as the chosen construction}

      
      Similarly, $\underline{\mathfrak C}\Omega_G$ is isomorphic to the Grothendieck construction on the functor
      \begin{equation}
            \begin{tikzcd}[row sep = tiny]
                  \Omega_G^{op} \arrow[r]
                  &
                  \mathsf{Cat}
                  \\
                  T \arrow[r, mapsto]
                  &
                  \Set^{O_G^{op}}(\Phi(E(T)), \underline{\mathfrak C}),
            \end{tikzcd}
      \end{equation}
      
      $\CS$ can be defined similarly, with the relevant sources restricted to
      $\Sigma_G \subseteq \Omega_G$. 
\end{remark}


\begin{remark}
      $\underline{\mathfrak C}\Omega_G$ is also a root fibration ---
      that is, a split Grothendieck fibration over the orbit category.
      \todo[inline]{cite reading material}
      Formally, as $\mathsf F \wr (-)$ and pullbacks preserve such fibrations, and these are compatible under composition,
      this follows from the natural maps $\underline{\mathfrak C}\Omega_G \to \Omega_G \to O_G$.
      Explicitly, $\underline{\mathfrak C}\Omega_G(G/H)$ has as objects those pairs $(T,\mathfrak c)$ such that
      $T \simeq G \cdot_H T_*$ for $T_{\**} \in \Omega^H$.
      % In this case, we have a canonical isomorphism
      % $E_H(T_{\**}) \to E_G(T) \simeq E_H(T_{\**})$, and we have a natural factorization
      % \begin{equation}
      %       \begin{tikzcd}
      %             E_G(T) \simeq E_H(T_{\**}) \arrow[r, "\mathfrak c"] \arrow[dr]
      %             &
      %             \mathfrak C|_{H} \arrow[r, hookrightarrow] \arrow[d]
      %             &
      %             \mathfrak C \arrow[d]
      %             \\
      %             &
      %             O_H \arrow[r, hookrightarrow]
      %             &
      %             O_G.
      %       \end{tikzcd}
      % \end{equation}
      Maps $\phi:(T,\mathfrak c) \to (S, \mathfrak d)$ in each fiber are called \textit{root-fixed}:
      as maps in $\Omega_G$, they are \textit{rooted} ($G r_T \to G r_S$ is a planar isomorphism),
      and moreover $\mathfrak c(Gr_T) = \mathfrak d(G r_S)$.
      
      Given $q: G/H \to G/K$ in the orbit category,
      the chosen Cartesian maps are the induced root pullback maps $q: q^{\**}T \to T$ on $G$-trees,
      with the colouring of $q^{\**}T$ defined as follows:
      for $b\in E(q^{\**}T)$, minimal in it's $G$-orbit, we have $q(b) = g a$ for some $g \in G$ and $a \in E(T)$ minimal in its orbit.
      Moreover, as this $g$ is unique modulo $G_a$, we have that there is a well-defined map $g_*: G/G_{q(b)} \to G/G_a$,
      and as $q$ induces a unique map $q_b^{\**}: G/G_b \to G/G_{q(b)}$, we have
      \begin{equation}
            (q^{\**}\mathfrak c)([b]) q^{\**}g^{\**}\mathfrak c([a]).
      \end{equation}

      {\color{blue} % ---------- ALTERNATIVELY --------------------
        Alternatively, on $\Phi E(q^{\**} T)$, we have
        $(q^{\**}\mathfrak c)(G/H, b) = q_b^{\**}(\mathfrak c(G/H, q(b)))$.
      } % --------------------------------------------------
\end{remark}

\begin{remark}
      We note that any \textit{planar} map of coloured $G$-trees is always \textit{colour-fixed}, in that
      $\mathfrak c(Ge) = \mathfrak d(G \phi (e))$ for all $Ge \in E_G(T)$.
\end{remark}

\begin{remark}
      A \textit{quotient} map in $\UC \Omega_G$ is any morphism such that the underlying map in $\Omega_G$ is a quotient.
\end{remark}

% \begin{remark}
%       Replacing the bottom-left corner in \eqref{COMEGA_G_EQ} with $\Omega^G$ changes the pullback to the category
%       $\underline{\mathfrak C}\Omega^G$ of ``$\underline{\mathfrak C}$-coloured trees with $G$-action''.
%       \todo[inline]{leaf-root for later?}
%       Any coloured $G$-tree $(T, \mathfrak c)$ with $T = (T_a)_{a \in A}$ has that each $T_a$ is a
%       ``$\underline{\mathfrak C|_{G_a}}$-coloured tree with $G_a$-action.''
% \end{remark}

We have natural inclusions on the left
\begin{equation}
      \begin{tikzcd}
            \underline{\mathfrak C}\Sigma \arrow[d, "\iota"] \arrow[r]
            &
            \underline{\mathfrak C}\Omega \arrow[d]
            &&
            \Sigma \times G \arrow[d] \arrow[r]
            &
            \Omega \times G \arrow[d]
            \\
            \underline{\mathfrak C}\Sigma_G \arrow[r]
            &
            \underline{\mathfrak C}\Omega_G
            &&
            \Sigma_G \arrow[r]
            &
            \Omega_G
      \end{tikzcd}
\end{equation}
which forget to the uncoloured inclusions on the right.
Specifically, $U \mapsto G \cdot U$ and, as $E_G(G \cdot U) = E(U)$, the associated colouring map is simply $\mathfrak c$ again.
On morphisms, $(\phi,g)$ maps to $(\phi)_{G} \circ g$.

\subsection{Planar Strings and Stuff}

\todo[inline]{Need to strike a balance between what to show explicitly, and what to just state.
  \S 3.4 and \S 4 from \cite{BP17} extend almost formally, though phrasing it as such...

  We still have natural span $\UC\Sigma_G \leftarrow \UC\Omega_G^0 \to \mathsf F_s \wr \UC\Sigma_G$, such that
  the left arrow is a map of rooted fibrations.
  This, plus whats already in \S 3.4 and \S 4, may be enough to just formally push through.}

Generalizing \cite[Remark 3.78]{BP17}
\todo[inline]{otherwise, have to force on the non-equivariant trees the correct isotropy of their colours. If not, we just see $\Phi\mathfrak C(G/e)$, and not the whole coefficient system.}
\begin{definition}
      Given $(T,\mathfrak c) \in \underline{\mathfrak C}\Omega_G$, a
      \textit{planar (resp. rooted) $T$-substitution datum} is a tuple
      $((U_{v_{Ge}}, \mathfrak c_{v_{Ge}}))_{v_{Ge} \in V_G(T)}$ of $\underline{\mathfrak C}$-coloured $G$-trees along with
      planar (resp. rooted colour-fixed) tall maps
      $T_{v_{Ge}} \to U_{v_{Ge}}$.

      A map of planar (resp. rooted) $T$-substitution data $(U_{v_{Ge}}) \to (V_{v_{Ge}})$ is a compatible tuple of planar (resp. rooted colour-fixed) tall maps $(U_{v_{Ge}} \to V_{v_{Ge}})$.
      Let $\mathsf{Sub}_p(T)$ and $\mathsf{Sub}(T)$ denote the categories of planar (resp. rooted) $T$-substituion datum.
\end{definition}

\begin{lemma}[{cf. \cite[Prop. 3.41]{BP17}}]
      Let $(T,\mathfrak c) \in \underline{\mathfrak C}\Omega_G$ be a $\underline{\mathfrak C}$-coloured $G$-tree.
      There are isomorphisms of categories
      \begin{equation}
            \label{SUB_EQUIV_EQ}
            \begin{tikzcd}[row sep = 4pt]
                  \mathsf{Sub}_p(T) \arrow[r, shift left]
                  &
                  (T, \mathfrak c) \downarrow \underline{\mathfrak C}\Omega_G^{pt}
                  \arrow[l, shift left]
                  &
                  \mathsf{Sub}(T) \arrow[r, shift left]
                  &
                  (T, \mathfrak c) \downarrow \underline{\mathfrak C}\Omega_G^{r}
                  \arrow[l, shift left]                  
                  \\
                  (U_{v_{Ge}}) \arrow[r, mapsto]
                  &
                  ((T, \mathfrak c) \to \colim_{Sc_G(T)}U_{(-)}).
                  &
                  (U_{v_{Ge}}) \arrow[r, mapsto]
                  &
                  ((T, \mathfrak c) \to \colim_{Sc_G(T)}U_{(-)}).
            \end{tikzcd}
      \end{equation}
      where $\underline{\mathfrak C}\Omega_G^{pt}, \underline{\mathfrak C}\Omega_G^{r}$ are the categories of planar tall (resp. rooted) maps under $(T, \mathfrak c)$. 
\end{lemma}
\begin{proof}
      This follows as in \cite[Prop. 3.41]{BP17}, going by induction on $n=|V_G(T)|$.
      Let $(U_T,\mathfrak c_{U_T})$ denote the colimit, if it exists.
      If $n$ is 0 or 1, $T$ is terminal in $Sc_G(T)$, and the colouring $\mathfrak c_{U_T}$ is just $\mathfrak c$.
      Otherwise, we have a decomposition $T = R \amalg_{Ge} S$ with
      the planar ordering on $Ge$ in $R$, $S$, and $T$ the same,
      $E_G(T) = E_G(R) \amalg_{Ge} E_G(S)$
      $\mathfrak c_{R} = \mathfrak c|_{E_G(R)}$,
      $\mathfrak c_{S} = \mathfrak c|_{E_G(S)}$,
      such that
      the existance of $U_T$ and $\mathfrak c_{U_T}$ follow from the existance of the pushout below in $\underline{\mathfrak C}\Omega_G^{pt,cf}$.
      \begin{equation}
            \begin{tikzcd}
                  (\eta_{Ge}, \mathfrak c) \arrow[d] \arrow[r]
                  &
                  (U_S, \mathfrak c_{U_S}) \arrow[d, dashed]
                  \\
                  (U_R, \mathfrak c_{U_R}) \arrow[r, dashed]
                  &
                  (U_T, \mathfrak c_{U_T})
            \end{tikzcd}
      \end{equation}
      By induction, $U_S$, $U_R$, $\mathfrak c_{U_S}$, $\mathfrak c_{U_R}$ exist
      (with unique choices such that $(U_{v_{Ge}}, \mathfrak c_{U_{v_{Ge}}}) \into(U_R, \mathfrak c_{U_R})$ is planar [and colour-fixed]).
      Forgetting colours, this is an equivariant grafting diagram, and hence the $G$-tree $U_T$ exists.
      Moreover, we have $E_G(U_T) = E_G(U_S) \amalg_{Ge} E_G(U_R)$, and so we have a well-defined colouring
      \begin{equation}
            \mathfrak c_{U_T}(Gf) =
            \begin{cases}
                  \mathfrak c_{U_R}(Gf) \qquad \qquad & Gf \in E_G(R) \\
                  \mathfrak c_{U_S}(Gf) & Gf \in E_G(S)
            \end{cases}
      \end{equation}
      since the overlap $Ge$ is in $T$, and hence it is dictated that $\mathfrak c_{U_T}(Ge) = \mathfrak c (Ge)$.
\end{proof}

\begin{lemma}[{cf. \cite[Lemma 3.63]{BP17}}]
      $\underline{\mathfrak C}\Omega_G^0 \to \mathsf F_s \wr \underline{\mathfrak C}\Sigma_G$
      sends root pullbacks to pullbacks over $\mathsf F_s \wr O_G$.
\end{lemma}
\begin{proof}
      Exactly as in \textit{loc cite}, with the additional note that
      the colouring of $\psi^{\**}T$ is precisely such that each $(\psi^{\**}T)_{v_{Ge}} \to T_{v_{G\phi(e)}}$
      is a pullback in $\UC \Sigma_G$.
\end{proof}

\begin{definition}
      The category $\UC\Omega_G^n$ of \textit{coloured planar $n$-strings} is the category
      whoses objects are strings
      \begin{equation}
            (T_0,\mathfrak c_0)
            \xrightarrow{\phi_1} (T_1, \mathfrak c_1)
            \xrightarrow{\phi_2} \ldots
            \xrightarrow{\phi_n} (T_n, \mathfrak c_n)
      \end{equation}
      where $(T_i, \mathfrak c_i) \in \UC\Omega_G$ and the $\phi_i$ are all coloured planar tall maps,
      while arrows are commutative diagrams of quotient maps.
\end{definition}

\begin{remark}
      We observe
      \begin{enumerate}
      \item $\UC\Omega_G^\bullet \to \UC\Sigma_G$ is an augmented simplicial object in categories.
      \item $\UC\Omega_G^n \to O_G$ is a root fibration.
      \item We have a vertex functor $V_G: \UC\Omega_G^{n+1} \to \mathsf F_s \wr \UC\Omega_G^n$ by
            \begin{equation}
                  \large(
                  (T_0,\mathfrak c_0)
                  \to (T_1, \mathfrak c_1)
                  \to \dots
                  \to (T_n, \mathfrak c_n)
                  \large)
                  \mapsto
                  \large(
                  (T_{1,v_{Ge}},\mathfrak c_1)
                  \to \dots \to
                  (T_{n,v_{Ge}}, \mathfrak c_n)
                  \large)_{v_{Ge} \in V_G(T_0)}
            \end{equation}
            where we write abusively denote by $T_{i,v_{Ge}}$ the $G$-tree $(T_{i,\bar\phi_i(f)^\uparrow \leq \bar\phi_i(f)})_{f \in Ge}$
            and by $\mathfrak c_i$ the restriction to any of its sub-$G$-trees.

            Alternatively, regarding the source above as a string of $n-1$ arrows in
            $(T_0, \mathfrak c_0) \downarrow \UC\Omega_G^{pt}$,
            the image under $V_G$ can be recognized as the inverse image under \eqref{SUB_EQUIV_EQ}.
      \end{enumerate}
\end{remark}

\begin{proposition}[{cf. \cite[Prop 3.82]{BP17}}]
      For any $n \geq 0$, the commutative diagram
      \begin{equation}
            \begin{tikzcd}
                  \UC\Omega_G^n \arrow[d, "d_{1,\dots,n}"'] \arrow[r, "V_G"]
                  &
                  \mathsf F_s \wr \UC\Omega_G^{n-1} \arrow[d, "\mathsf F \wr d_{0,\dots,n-1}"]
                  \\
                  \UC\Omega_G^0 \arrow[r, "V_G"']
                  &
                  \mathsf F_s \wr \UC\Sigma_G
            \end{tikzcd}
      \end{equation}
      is a pullback diagram in $\Cat$.
\end{proposition}
\begin{proof}
      % This follows as in \textit{loc cite}, as the colourings descend to all sub-$G$-trees. 
\end{proof}

\begin{proposition}
      [{cf. \cite[Lemma 4.28]{BP17}}]
      $N_{\mathfrak C}$ on spans preserves right Kan extensions over $\mathsf F \wr \mathcal A \downarrow \mathsf F \wr \UC\Sigma_G$.
\end{proposition}
\begin{proof}
      Stuff
\end{proof}

Similarly, \cite[Prop 3.47, 3.90, 4.12, 4.15, 4.26, 4.30]{BP17} naturally generalized to the coloured-setting,
replacing all instances of $\Omega_G^n$ or $\Sigma_G$ with $\UC\Omega_G^n$ and $\UC\Sigma_G$.
In particular, this yields the following definitions and proposition.
\begin{definition}[{cf. \cite[Defn 4.3]{BP17}}]
      Let $\mathsf{WSpan}^l(\mathcal C, \mathcal D)$ (resp. $\mathsf{WSpan}^r(\mathcal C, \mathcal D)$)
      denote the category of \textit{left (resp. right) weak spans}, with objects
      \begin{equation}
            \mathcal C \xleftarrow{k} \mathcal A \xrightarrow{X} \mathcal D
      \end{equation}
      and arrows those diagrams as on the left (resp. right) below
      \begin{equation}
            \begin{tikzcd}[row sep = tiny]
                  & \mathcal A_1 \arrow[dr, "X_1", ""'{name=U}] \arrow[dl, "k_1"'] \arrow[dd, "i"']
                  &
                  &&
                  &
                  \mathcal A_1 \arrow[dr, "X_1", ""'{name=A}] \arrow[dl, "k_1"'] \arrow[dd, "i"']
                  \\
                  \mathcal C
                  &&
                  \mathcal D
                  &&
                  \mathcal C
                  &&
                  \mathcal D
                  \\
                  & |[alias=V]| \mathcal A_2 \arrow[ur, "X_2"'] \arrow[ul, "k_2"]
                  &
                  &&
                  &
                  |[alias=B]| \mathcal A_2 \arrow[ur, "X_2"'] \arrow[ul, "k_2"]
                  \arrow[Rightarrow, from = U, to = V]
                  \arrow[Rightarrow, from = A, to = B]
            \end{tikzcd}
      \end{equation}
      denoted by $(i,\phi): (k_1,X_1) \to (k_2,X_2)$, with composition defined in the natural way.      
\end{definition}

\todo[inline]{recall adjunctions with $\mathsf{Lan}$ and $\mathsf{Ran}$, canonical op-isos, etc}

\begin{definition}[{cf. \cite[Defn 4.16]{BP17}}]
      Suppose $\V$ is a symmetric monoidal category with diagonals \todo{recall}.
      We define an endofunctor $N_{\UC}$ on $\mathsf{WSpan}^r(\UC\Sigma_G, \V^{op})$
      by letting $N_{\UC}(\UC\Sigma_G \leftarrow \mathcal A \to \V^{op})$ be given by the span
      \begin{equation}
            \begin{tikzcd}
                  \UC\Omega_G^0 \wr \mathcal A \arrow[d] \arrow[r, "V_G"]
                  &
                  \mathsf F \wr \mathcal A \arrow[d] \arrow[r]
                  &
                  \mathsf F \wr \V^{op} \arrow[r, "\otimes^{op}"]
                  &
                  \V^{op}
                  \\
                  \UC\Omega_G^0 \arrow[r, "V_G"'] \arrow[d]
                  &
                  \mathsf F \wr \UC\Sigma_G
                  \\
                  \UC\Sigma_G
            \end{tikzcd}
      \end{equation}
      where the given square is a pullback, and on arrows in the natural way.

      Moreover, we have a multiplication $\mu: N_{\UC} \circ N_{\UC} \Rightarrow N_{\UC}$ given by the natural isomorphism
      \begin{equation}\label{MULTDEFSPAN EQ}
            \begin{tikzcd}
                  \UC\Sigma_G \ar[equal]{d}&
                  \UC\Omega_{G}^1 \wr A \ar{r}{V_G} \ar{d}[swap]{d_{0}} \ar{l}&
                  \Fin \wr \UC\Omega_{G}^0 \wr A \ar{r}{\Fin \wr V_G} &
                  |[alias=FFOmega]| \Fin^{\wr 2} \wr A \ar{d}{\sigma^0} \ar{r} &
                  \Fin^{\wr 2} \wr \mathcal{V}^{op} \ar{d}{\sigma^0} \ar{r}{\otimes^{op}} &
                  \Fin \wr \mathcal{V}^{op} \ar{r}{\otimes^{op}} &
                  |[alias=dog]|
                  \mathcal{V}^{op} \ar[equal]{d}
                  \\
                  \UC\Sigma_G &
                  |[alias=Omega]|\UC\Omega_{G}^{0} \wr A \ar{rr}[swap]{V_G} \ar{l}&&
                  \Fin \wr A \ar{r} &
                  |[alias=cat]|
                  \Fin \wr \mathcal{V}^{op} \ar{rr}[swap]{\otimes^{op}} &&
                  \mathcal{V}^{op}
                  \arrow[Leftrightarrow, from=FFOmega, to=Omega,shorten <=0.15cm,,shorten >=0.15cm,"\pi_0"]
                  \arrow[Leftrightarrow, from=dog, to=cat,shorten <=0.15cm,,shorten >=0.15cm,"\alpha"]
            \end{tikzcd}
      \end{equation}
      and a unit $\eta: id \Rightarrow N_{\UC}$ give by the strictly commuting diagram
      \begin{equation}\label{UNITSPAN EQ}
            \begin{tikzcd}
                  \UC\Sigma_G \ar[equal]{d} &
                  A \ar{l} \ar{d}[swap]{s_{-1}} \ar[equal]{r} &
                  A \ar{d}{\delta^0} \ar{r} &
                  \mathcal{V}^{op} \ar{d}{\delta^0} \ar[equal]{r}&
                  \mathcal{V}^{op} \ar[equal]{d}
                  \\
                  \UC\Sigma_G &
                  \UC\Omega_{G}^{0} \wr A \ar{l} \ar{r}[swap]{V_G}&
                  \Fin \wr A \ar{r} &
                  \Fin \wr \mathcal{V}^{op} \ar{r}[swap]{\otimes^{op}} &
                  \mathcal{V}^{op}.
            \end{tikzcd}
      \end{equation}	
\end{definition}

\begin{proposition}
      [{cf. \cite[Prop 4.19]{BP17}}]
      $(N_{\UC},\mu,\eta)$ is a monad on $\mathsf{WSpan}^r(\UC\UC\Sigma_G, \V^{op})$.
\end{proposition}


\begin{definition}
      The \textit{genuine $\UC$-coloured operad monad} is the monad
      $\mathbb F_{G,\UC}$ on $\Sym_{G, \UC}(\V) = \mathsf{Fun}(\UC\Sigma_G^{op}, \V)$ given by
      \begin{equation}
            \mathbb F_{G,\UC} = \Lan \circ N_{\UC} \circ \iota
      \end{equation}
      with multiplication and unit given by
      \begin{equation}
            \mathsf{Lan} \circ N_{\UC} \circ \iota \circ
            \mathsf{Lan} \circ N_{\UC} \circ \iota
            \overset{\simeq}{\Leftarrow}
            \mathsf{Lan} \circ N_{\UC} \circ  N_{\UC} \circ \iota
            \Rightarrow
            \mathsf{Lan} \circ N_{\UC} \circ \iota
      \end{equation}
      \begin{equation}
            id \overset{\simeq}{\Leftarrow} \mathsf{Lan} \circ \iota
            \Rightarrow
            \mathsf{Lan} \circ N_{\UC} \circ \iota.
      \end{equation}
      We will write $\Op_{G,\UC}(\V)$ for the category 
      $\mathsf{Alg}_{\mathbb{F}_{G,\UC}}(\mathsf{Sym}_{G,\UC}(\mathcal{V}))$ of \textit{genuine $\UC$-coloured operads}.
\end{definition}


\subsection{Genuine $\mathfrak C$-coloured operads}


\todo[inline]{Come back : Something about profiles.}
\todo[inline]{come back: Combine with above}

\begin{remark}
      Given $X\in \dSet_G$ with $X(\eta_{G/H}) = \mathfrak C(G/H)$, we have that
      $\UC\Sigma_G$ is equal to the category of \textit{profiles} $\partial\Omega[C] \to X$,
      where $C$ ranges over all of $\Sigma_G$.
      \todo[inline]{come back}
\end{remark}

\subsection{Comparison with $\mathfrak C$-coloured operads}

Given $(T = (T_i)_I, \mathfrak c) \in \UC\Omega_G$, we define
$\mathfrak c_i: E(T_i) \to \mathfrak C(G/e)$ by
\begin{equation}
      \mathfrak c_i(e) = g^{\**}q_e^{\**}(\mathfrak c[f]),
\end{equation}
where
$e \in Gf$ (with $f$ minimal in the planar structure on $T$),
$g\in G$ minimal such that $g e = f$,
$q: G \to r(T)$ the unique quotient map preserving minimal elements,
and $q_e:G/G_e \to G / G_{q(e)}$ the induced map.

Then $(T_i, \mathfrak c_i) \in \UC\Omega$, and moreover
$i \mapsto (T_i, \mathfrak c_i)$ yields a well-defined functor $B_{I}G \to \UC\Omega$.


\begin{remark}
      The colouring $\mathfrak c_i$ is \textit{almost} the composite
      \begin{equation}
            E(T_i) \to E_{G_i}(T_i) \xrightarrow{\simeq} E_G(T) \to \mathfrak C \to G \ltimes \mathfrak C(G/e)
      \end{equation}
      where $G_i$ is the stabilizer in $G$ of $T_i$, and
      $E_{G_i}(T_i) \to E_G(T)$ is the canonical isomorphism sending
      $e{G_i} \to Gf$
      with $f \in Ge$ minimal.
      However, this composite does not record the ``twisting'' action by the element $g_e$.
\end{remark}


With that, we have the formula
\begin{equation}
      \begin{tikzcd}
            \iota_{\**}Y(T,\mathfrak c) =
            \left(
                  \prod_I Y(T_i, \mathfrak c_i)
            \right)^G.
      \end{tikzcd}
\end{equation}

\begin{remark}[{cf. \cite[Rem 4.35]{BP17}}]
      Equivalently, the essential image of $\iota_{\**}$ are those sheaves $X \in \Sym_{G, \mathfrak C}(\V)$ such that
      the canonical map
      \begin{equation}
            X(C,\mathfrak c) \xrightarrow{\simeq} X(q^{\**}(C, \mathfrak c))^\Gamma
      \end{equation}
      is an isomorphism, where $q: G \to r(C)$ is the unique map preserving the minimal element, and
      $\Gamma \leq \mathsf{Aut}(q^{\**}(C,\mathfrak c))$ the subgroup preserving the quotient map $q^{\**}C \to C$
      under precomposition.
\end{remark}

\begin{remark}
      Alternatively, $\mathfrak c_i$ is the composite
      \begin{equation}
            E(T_i) \to E(T) \to \mathfrak C(G/e).
      \end{equation}
\end{remark}


\todo[inline]{Come BACK}

DO STUFF.






























%%%%%%%%%%%%%%%%%%%%%%%%%%%%%%%%%%%%%%%%%%%%%%%%%%%%%%%%%%%%%%%%%%%%%%%
\newpage

\section{In $\mathsf{dSet}_G$}

\begin{definition}
      Define the \textit{genuine operadic nerve} $N: \Op_G \to \dSet_G$ by
      \begin{equation}
            N\P(T) = \Hom_{\Op_G}(T, \P)
      \end{equation}
      where we think of $T$ as the operad $T \in \Op^G \into \Op_G$. 
\end{definition}

\begin{remark}
      We note that $N\P \in (SCI)^{\boxslash !}$,
      as $T \in \Op_G$ is a free $\mathbb F_G$-algebra on its vertices.
\end{remark}

\begin{remark}
      We can rephrase the definition of being an $\mathbb F_G$-algebra in terms of $N\P$.
      For $\P \in \Sym_G$ a $G$-symmetric sequence,
      a genuine $G$-operad structure on $\P$ is given by:
      \begin{itemize}
      \item Composition Maps: $ $\\
            maps 
            $N\P(T) \to \P(\mathsf{lr}(T))$
            for all $T \in \Omega_G$.
      \item Naturality under restriction and conjugation: $ $\\
            maps $N\P(T_1) \to N\P(T_0)$
            for all quotient maps $T_0 \to T_1$ in $\Omega_{G,0}$,
            such that the following commutes:
            \begin{equation}
                  \begin{tikzcd}
                        N\P(T_1) \arrow[r] \arrow[d]
                        &
                        \P(\mathsf{lr}(T_1)) \arrow[d]
                        \\
                        N\P(T_0) \arrow[r]
                        &
                        \P(\mathsf{lr}(T_0)).
                  \end{tikzcd}
            \end{equation}
      \item Associativity under $\mathbb F_G$: $ $\\
            maps $N\P(T_1) \to N\P(T_0)$
            for all planar tall maps $T_0 \to T_1$ in $\Omega_G^t$,
            such that the analogus diagram (with the right vertical map the identity) commutes.\footnote{
              As in \cite{BP17}, we note that ``associativity'' under $\mathbb F_G$ includes both
              the usual notion of associativity of our composition maps,
              but also unitality;
              this is recorded here by the fact that degeneracies are always planar tall.}
      \end{itemize}
\end{remark}

The above reflects the following result.

\begin{proposition}
      $\Op_G$ is equivalent to the subcategory of $\mathsf{dSet_G}$ spanned by those $X$ such that
      \begin{enumerate}
      \item $X(H/H) = \set{\**}$ for all $H \leq G$.
      \item $X(T) \cong \otimes_{T_v \in V(T)}X(T_v)$. 
      \end{enumerate}
\end{proposition}
\begin{proof}
      The fact that $N\P \in (SCI_G)^{\boxslash !}$ is immediate, as remarked above.

      For the reverse direction, we will follow the construction of the homotopy operad as in \cite[\S 6]{MW09},
      replacing their use of inner horn inclusions with \textit{orbital} inner $G$-horn inclusions,
      to show that any $X \in (OHI)^{\boxslash !}$ is in the image of $N$; 
      the result will then follow from \cite[HYPER PROP]{BP18}.

      In fact, interpreting all of their pictures are as \textit{orbital} representations of $G$-trees yields that,
      for all $C \in \Sigma_G$
      \begin{itemize}
      \item $\sim_{G e}$ is an equivalence relation on $X(C)$ for all $Ge \in E_G(C)$.
      \item The relations $\sim_{G e}$ and $\sim_{G e'}$ are equal for all $e,e'\in E(C)$.
      \item $[h] \circ [f] = [h \circ f]$ yields a well-defined composition map. \todo[inline]{come back}
      \end{itemize}
      \todo[inline]{need to show naturality, check associativity of composition}
\end{proof}



\newpage



\section{Scratchwork}

\subsection{Colored simplicial tensors and cotensors}



\[
\begin{tikzcd}
	K \otimes f^{\**} P \ar{r} \ar{ddd}&
	K \otimes f^{\**} \left( (K \otimes P)^K \right) \ar{r}{\simeq} \ar{d}&
	K \otimes \left( f^{\**} (K \otimes P) \right)^K \ar{r} \ar{d} &
	f^{\**} (K \otimes P) \ar{ddd}
\\
	&
	K \otimes f^{\**} \left( (L \otimes P)^K \right) \ar{d}
	\ar{r}{\simeq} &
	K \otimes \left( f^{\**} (L \otimes P) \right)^K
	\ar{d}
\\
	&
	L \otimes f^{\**} \left( (L \otimes P)^K \right)
	\ar{r}{\simeq} &
	L \otimes \left( f^{\**} (L \otimes P) \right)^K
\\
	L \otimes f^{\**} P \ar{r} &
	L \otimes f^{\**} \left( (L \otimes P)^L \right) \ar{r}{\simeq} \ar{u} &
	L \otimes \left( f^{\**} (L \otimes P) \right)^L \ar{r}&
	f^{\**} (L \otimes P)
\end{tikzcd}
\]



\[
\begin{tikzcd}
	K \otimes f^{\**} P \ar{r} \ar{d}&
	K \otimes f^{\**} \left( (K \otimes P)^K \right) \ar{r}{\simeq} &
	K \otimes \left( f^{\**} (K \otimes P) \right)^K \ar{r}  &
	f^{\**} (K \otimes P) \ar{ddd}
\\
	K \otimes \left( (L \otimes f^{\**} P) \right)^L \ar{d} &
	K \otimes \left( (L \otimes f^{\**} \left( (K \otimes P)^K \right)) \right)^L &
\\
	K \otimes \left( (L \otimes f^{\**} P) \right)^K \ar{d} &
	K \otimes \left( (L \otimes f^{\**} \left( (K \otimes P)^K \right)) \right)^K
\\
	L \otimes f^{\**} P \ar{r} &
	L \otimes f^{\**} \left( (L \otimes P)^L \right) \ar{r}{\simeq} &
	L \otimes \left( f^{\**} (L \otimes P) \right)^L \ar{r}&
	f^{\**} (L \otimes P)
\end{tikzcd}
\]




\[
\begin{tikzcd}
	f^{\**} P \ar{rr} \ar{rd} \ar{dd} & &
	f^{\**} \left( (K \otimes P)^K \right) \ar{d} &
	\left( f^{\**} (K \otimes P) \right)^K \ar{l}[swap]{\simeq} \ar{ddd}
\\
	&
	f^{\**} \left( (L \otimes P)^L \right) \ar{r} &
	f^{\**} \left( (L \otimes P)^K \right) 
\\
	\left( L \otimes f^{\**} P \right)^L \ar{r} \ar{d} &
	\left( f^{\**}( L \otimes P ) \right)^L \ar{u}{\simeq} \ar{rrd} 
\\
	\left( L \otimes f^{\**} P \right)^K \ar{rrr} &&&
	\left( f^{\**}( L \otimes P ) \right)^K \ar{uul}[swap]{\simeq}
\end{tikzcd}
\]


\newpage


\subsection{Semi-cofibrantly generated}


The following codifies a formal argument implicit in the proof of \cite[Thm. 7.19]{CM13b}.

\begin{definition}
Given a set $J$ of maps that admit the small object argument, we say that $X \in \mathcal{M}$ is \textit{$J$-fibrant} if $X \to \**$ has the right lifting property against maps in $J$.

Further, given $D$ a class of maps in $\mathcal{M}$,
we write $D_{J\text{-fib}} \subseteq D$ to denote 
the subclass of maps whose target is $J$-fibrant.
\end{definition}

\begin{lemma}
	Let $\mathcal{M}$ be a model category with $(C,W,F)$
	the corresponding classes of cofibrations, weak equivalences and fibrations. 
	Further, $J$ be a set of maps admitting the small object argument and such that:
\begin{itemize}
	\item[(i)] $J \subseteq C \cap W$;
	\item[(ii)] 
	$\left(J^{\boxslash} \cap W \right)_{J\text{-fib}}
	\subseteq \left( F \cap W \right)_{J\text{-fib}}$.
\end{itemize}
Then one further has that:
\begin{itemize}
	\item[(a)]
	$\left(\prescript{\boxslash}{}{\left(J^{\boxslash}\right)}\right)_{J\text{-fib}}
	= 
	\left( C \cap W \right)_{J\text{-fib}}$;
	\item[(b)]
	$\left(J^{\boxslash} \right)_{J\text{-fib}}
	= F_{J\text{-fib}}$.
\end{itemize}
\end{lemma}

\begin{remark}
Rephrasing (b), one has that the fibrant objects of $\mathcal{M}$ are precisely the $J$-fibrant objects
and thus that the fibrations between fibrant objects are precisely the $J$-fibrations.
\end{remark}

\begin{proof}
	To check (a), recalling first that 
	$\prescript{\boxslash}{}{\left(J^{\boxslash}\right)}$
	is the saturation of $J$, one has that (i) in fact implies 
	$\prescript{\boxslash}{}{\left(J^{\boxslash}\right)}
		\subseteq C \cap W $.
	For the converse direction, given a trivial cofibration
	$A \to Y$ with $J$-fibrant target,
	form the factorization 
	$A \to X \to Y$ as a 
	$J$-cofibration followed by a $J$-fibration. 
	By the first direction the map $A\to X$ is a weak equivalence, and thus by 2-out-of-3 so is $X \to Y$.
	But then by (ii) the map $X \to Y$ is a trivial fibration, so that the lifting below exists,
	showing that $A \to Y$ is a retract of $A \to X$, and thus also in the saturation $\prescript{\boxslash}{}{\left(J^{\boxslash}\right)}$, 
	as desired.
\[
\begin{tikzcd}
	A \ar[>->]{r}{J} \ar[>->]{d}[swap]{\sim}&
	X \ar[->>]{d}{J}
%& &
%	A \ar[>->]{r}{\sim} \ar[>->]{d}[swap]{\sim}&
%	Y \ar[->>]{d}{\sim}
\\
	Y \ar[equal]{r} \ar[dashed]{ru} & Y
%& &
%	X \ar[equal]{r} \ar[dashed]{ru} & X
\end{tikzcd}
\]

To check (b), one direction is again immediate from (i),
since $J^{\boxslash} \supseteq (C \cap W)^{\boxslash} = F$.
For the converse direction, it suffices to show that 
a $J$-fibration $X\to Y$ with $J$-fibrant target has the right lifting property against trivial cofibrations, as on the left diagram below.
After factoring the bottom horizontal map as a $J$-cofibration followed by a $J$-fibration as on the right diagram, it suffices to shows that a lift $B' \to X$ exists.
But since $B'$ is $J$-fibrant, this follows from (a), which shows that the composite $A \to B \to B'$ is a $J$-cofibration.
\[
\begin{tikzcd}
	A \ar{r} \ar[>->]{d}[swap]{\sim}&
	X \ar[->>]{d}{J}
&&
	A \ar{rr} \ar[>->]{d}[swap]{\sim}&&
	X \ar[->>]{d}{J}
\\
	B \ar{r} \ar[dashed]{ru} & Y
&&
	B \ar[>->]{r}[swap]{J} &
	B' \ar[->>]{r}[swap]{J} \ar[dashed]{ru}
	& Y
\end{tikzcd}
\]
\end{proof}

\begin{remark}
	Analyzing the proof above, one is free to replace the class of fibrant objects with any other class that is compatible with $J$-fibrations, in the sense that if 
	$X \to Y$ is a $J$-fibration and $Y$ is in the class, then so is $X$.
\end{remark}





\subsection{Formalizing some stuff}

The following is a reformalized proof of \cite[Thm. 8.14]{CM13b}.


\begin{proposition}
The (right) derived composite functors in the following diagram commute up to a zigzag of weak equivalences. 
\[
\begin{tikzcd}
	\mathsf{PreOp} \ar{d}[swap]{\gamma^{\**}}&
	\mathsf{sOp} \ar{l}[swap]{N} \ar{d}{hcN}
\\
	\mathsf{sdSet} &
	\mathsf{dSet} \ar{l}{c_{!}}
\end{tikzcd}
\]
\end{proposition}

Note that though $\gamma^{\**}$ and $c_{!}$ are left Quillen, they both preserve all equivalences, 
so that one needs only perform fibrant replacements in 
$\mathsf{sOp}$.

\begin{proof}
	Recall that, given an object $X$ in a model category $\mathcal{M}$, a simplicial frame of $X$ is a fibrant replacement
	$c_!(X) \to \widetilde{X}(\bullet)$ of the constant 
	simplcial object $c_!(X)$ in the Reedy model structure on $\mathcal{M}^{\Delta^{op}}$.
	Moreover, if $X$ was already fibrant one is free to assume that $\widetilde{X}(0) = X$.
	
	Let $\mathcal{O} \in \mathsf{sOp}$ be fibrant, 
	choose a (functorial) fibrant simplicial frame
	$\widetilde{\mathcal{O}}(\bullet) \in \mathsf{sOp}^{\Delta^{op}}$, where we assume $\widetilde{\mathcal{O}} (0) = \mathcal{O}$.
	Next, let 
	$\gamma^{\**} N \widetilde{\mathcal{O}}(\bullet) 
	\to \widetilde{Q}(\bullet)$
	be a Reedy fibrant replacement in  
	$\mathsf{sdSet}^{\Delta^{op}}$.
	
	We claim that the following is a zigzag of weak equivalences in $\mathsf{sdSet}$.
\begin{equation}\label{BIGZIG EQ}
	\gamma^{\**} N \mathcal{O} \xrightarrow{\sim}
	\widetilde{Q}(0) \xrightarrow{\sim}
	\delta^{\**} \widetilde{Q} \xleftarrow{\sim}
	\widetilde{Q}_0 \xleftarrow{\sim}
	\left(\gamma^{\**} N \widetilde{\mathcal{O}}\right)_0
	\xrightarrow{\sim}
	hcN \widetilde{\mathcal{O}} \xleftarrow{\sim}
	c_{!} hcN \mathcal{O}
\end{equation}
That the first map is an equivalence is obvious from definition of $\widetilde{Q}$ and the assumption $\widetilde{\mathcal{O}}(0) = \mathcal{O}$.

For the second and third maps, note first that $\widetilde{Q}$ is homotopically constant, in the sense that all structure maps $\widetilde{Q}(m) \to \widetilde{Q}(m')$
are equivalences.
Moreover, since the levels $\widetilde{Q}$ are fibrant in 
$\mathsf{sdSet}$, this implies that these are simplicial equivalences, i.e. that for each tree $T \in \Omega$
the evaluations 
$\widetilde{Q}(T)(m) \to \widetilde{Q}(T)(m')$
are Kan equivalences in $\mathsf{sSet}$.
Buy since $\widetilde{Q}(T) \in \mathsf{sSet}^{\Delta^{op}}$ is itself Reedy fibrant, this shows that it is in fact joint Reedy fibrant, so that one has Kan equivalences 
$\widetilde{Q}(T)(0) \xrightarrow{\sim}
\delta^{\**} \widetilde{Q}(T) \xleftarrow{\sim}
\widetilde{Q}_0(T)$, showing that the second and third maps in \eqref{BIGZIG EQ} are indeed weak equivalences.

For the fourth equivalence, note that one can write
\[\widetilde{Q}_0(T) = 
\mathsf{Hom}_{\mathsf{sdSet}}(\Omega[T],\widetilde{Q})=
\mathsf{Hom}_{\mathsf{PreOp}}(\Omega[T],\gamma_{\**}\widetilde{Q})\]
\[
\left(\gamma^{\**} N \widetilde{\mathcal{O}}\right)_0(T) = 
\mathsf{Hom}_{\mathsf{sdSet}}(\Omega[T],\gamma^{\**} N \widetilde{\mathcal{O}})=
\mathsf{Hom}_{\mathsf{PreOp}}(\Omega[T], N \widetilde{\mathcal{O}})
\]
The claim now follows by noting that
$N \mathcal{O} \to \gamma_{\**} \widetilde{Q}$
is an equivalence of Reedy fibrant objects on 
$\mathsf{PreOp}^{\Delta^{op}}$ (over the tame model structure) and that $\Omega(T)$ is tame cofibrant. 

For the fifth equivalence, note that
\[
\left(\gamma^{\**} N \widetilde{\mathcal{O}}\right)_0(T) = 
\mathsf{Hom}_{\mathsf{PreOp}}(\Omega[T], N \widetilde{\mathcal{O}}) =
\mathsf{Hom}_{\mathsf{sOp}}(\Omega(T),  \widetilde{\mathcal{O}})
\]
\[
\left(hcN \widetilde{\mathcal{O}} \right)(T) = 
\mathsf{Hom}_{\mathsf{sOp}}(W_!(T),  \widetilde{\mathcal{O}})
\]
so that the required claim follows since 
$\widetilde{\mathcal{O}}$ is Reedy fibrant and
$W_!(T) \to \Omega(T)$ is a weak equivalence of cofibrant operads.

Lastly, for the last map, one needs simply note that
$c_! hcN \mathcal{O} = hcN c_! \mathcal{O}$, so that the required claim follows since 
$c_! \mathcal{O} \to \widetilde{O}$
is a levelwise equivalence of levelwise fibrant operads
and $hcN$ is right Quillen.
\end{proof}


{\color{red} Maybe revise Lemmas A.21 and A.22, and the proof of Proposition 5.28 in \cite{BP18} to discuss evaluation of Reedy cofibrations/fibrations}



\bibliography{biblio}{}



\bibliographystyle{alpha}



\end{document}