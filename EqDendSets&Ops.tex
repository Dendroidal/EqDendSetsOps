\documentclass[a4paper,10pt
,draft
]{article}%

\usepackage[hidelinks]{hyperref}
\hypersetup{
  % colorlinks,
  final,
  pdftitle={Equivariant Dendroidal Segal Spaces},
  pdfauthor={Bonventre, P. and Pereira, L. A.},
  % pdfsubject={Your subject here},
  % pdfkeywords={keyword1, keyword2},
  linktoc=page
}
%\usepackage[open=false]{bookmark}

\input{commands.tex}%


%-------- Tikz ---------------------------

\usepackage{tikz}%
\usetikzlibrary{matrix,arrows,decorations.pathmorphing,
cd,patterns,calc}
\tikzset{%
  treenode/.style = {shape=rectangle, rounded corners,%
                     draw, align=center,%
                     top color=white, bottom color=blue!20},%
  root/.style     = {treenode, font=\Large, bottom color=red!30},%
  env/.style      = {treenode, font=\ttfamily\normalsize},%
  dummy/.style    = {circle,draw,inner sep=0pt,minimum size=2mm}%
}%

\usetikzlibrary[decorations.pathreplacing]



% ---- Commands on draft --------

\usepackage{ifdraft}
\ifdraft{
  \color[RGB]{63,63,63}
  % \pagecolor[rgb]{0.5,0.5,0.5}
  \pagecolor[RGB]{220,220,204}
  % \color[rgb]{1,1,1}
}


\usepackage[draft]{showkeys}
\usepackage{todonotes}%[obeyDraft]


% ----- Labels Changed? --------

\makeatletter

\def\@testdef #1#2#3{%
  \def\reserved@a{#3}\expandafter \ifx \csname #1@#2\endcsname
  \reserved@a  \else
  \typeout{^^Jlabel #2 changed:^^J%
    \meaning\reserved@a^^J%
    \expandafter\meaning\csname #1@#2\endcsname^^J}%
  \@tempswatrue \fi}

\makeatother


% ---- Commands --------

\newcommand{\mycircled}[2][none]{%
  \mathbin{
    \tikz[baseline=(a.base)]\node[draw,circle,inner sep=-1.5pt, outer sep=0pt,fill=#1](a){\ensuremath #2\strut};
cf.  }
}
\newcommand{\owr}{\mycircled{\wr}}
\newcommand{\UV}{\underline{\mathcal V}}
\renewcommand{\phi}{\varphi}
\newcommand{\UC}{\underline{\mathfrak C}}

\renewcommand{\F}{\mathcal F}
\newcommand{\I}{\mathbb I}
\newcommand{\J}{\mathbb J}
\newcommand{\Q}{\mathcal Q}
\renewcommand{\1}{\ensuremath{\mathbb{id}}}
\renewcommand{\hat}{\widehat}


% ---- Title --------

\title{To be determined}%

\author{Peter Bonventre, Lu\'is A. Pereira}%

\date{\today}


% ---- Document body --------

\begin{document}

\maketitle

\begin{abstract}
      Things and stuff
\end{abstract}

\tableofcontents


\section{Overview}


\[
	\begin{tikzcd}
		\mathsf{PreOp}^G
\\
		\mathsf{sdSet}^G \ar{r}[swap]{(-)_0} \ar{u}{\gamma_{\**}} &
		\mathsf{dSet}^G
	\end{tikzcd}
\]


\section{Preliminaries}

\subsection{Wreath products and Grothendieck fibrations}

First, we record a basic result.

\begin{lemma}
      \label{PB_GR_EQ}
      If $\mathcal C \to \mathcal D$ is a Grothendieck fibration, then the pullback
      \begin{equation}
            \begin{tikzcd}
                  \mathcal A \arrow[d] \arrow[r]
                  &
                  \mathcal C \arrow[d]
                  \\
                  \mathcal B \arrow[r]
                  &
                  \mathcal D
            \end{tikzcd}
      \end{equation}
      is isomorphic to the Grothendieck construction on
      \begin{equation}
            \label{PB_GR_EQ}
            % \begin{tikzcd}[row sep = tiny]
            %       \mathcal B^{op} \arrow[r]
            %       &
            %       \mathsf{Cat}
            %       \\
            %       b \arrow[r, mapsto]
            %       &
            %       \mathcal C_{F(b)},
            % \end{tikzcd}
            \mathcal B^{op} \longto \Cat,
            \qquad \qquad
            b \longmapsto \mathcal C_{F(b)},
      \end{equation}
      where $\mathcal C_{d}$ is the fiber over $d \in \mathcal D$.
\end{lemma}
\begin{proof}
      The fact that the map \eqref{PB_GR_EQ} is a functor follows from $\mathcal C \to \mathcal D$ being a fibration;
      the rest follows by unpacking definitions.
\end{proof}

% -------------------- Wreath Products --------------------

Now, recall the notation $\mathsf F \wr \mathcal C$ for a category $\mathcal C$.

\begin{remark}
      \label{WR_DIAG_REM}
      We observe that we have a natural diagonal map
      % \begin{equation}
      $
      F \times \mathcal C \into \mathsf F \wr \mathcal C,
      $
      % \end{equation}
      for any category $\mathcal C$,
      and thus for any functor $F: \mathcal D \to \mathsf F$, we have an induced functor
      $F: \mathcal D \times \mathcal C \to \mathsf F \wr \mathcal C$. 
\end{remark}

\begin{definition}[{cf. \cite[Defn 4.3]{BP17}}]
      Let $\mathsf{WSpan}^l(\mathcal C, \mathcal D)$ (resp. $\mathsf{WSpan}^r(\mathcal C, \mathcal D)$)
      denote the category of \textit{left (resp. right) weak spans}, with objects
      \begin{equation}
            \mathcal C \xleftarrow{k} \mathcal A \xrightarrow{X} \mathcal D
      \end{equation}
      and arrows those diagrams as on the left (resp. right) below
      \begin{equation}
            \begin{tikzcd}[row sep = tiny]
                  & \mathcal A_1 \arrow[dr, "X_1", ""'{name=U}] \arrow[dl, "k_1"'] \arrow[dd, "i"']
                  &
                  &&
                  &
                  \mathcal A_1 \arrow[dr, "X_1", ""'{name=A}] \arrow[dl, "k_1"'] \arrow[dd, "i"']
                  \\
                  \mathcal C
                  &&
                  \mathcal D
                  &&
                  \mathcal C
                  &&
                  \mathcal D
                  \\
                  & |[alias=V]| \mathcal A_2 \arrow[ur, "X_2"'] \arrow[ul, "k_2"]
                  &
                  &&
                  &
                  |[alias=B]| \mathcal A_2 \arrow[ur, "X_2"'] \arrow[ul, "k_2"]
                  \arrow[Rightarrow, from = U, to = V]
                  \arrow[Rightarrow, from = B, to = A]
            \end{tikzcd}
      \end{equation}
      denoted by $(i,\phi): (k_1,X_1) \to (k_2,X_2)$, with composition defined in the natural way.      
\end{definition}


\begin{remark}      
      We observe that there is a functor $\mathsf F \wr \mathcal C \longto \mathsf{WSpan}^l(\**, \mathcal C)$,
      sending $(A, (c_a))$ to the obvious map $A \to \mathcal C$ (thinking of $A$ as a discrete category).

      Furthermore, we have a pair of maps
      \begin{equation}
            \mathsf{fgt}: \mathsf{Fib}(\mathcal C) \leftrightarrows \mathsf{WSpan}^l(\**, \mathcal C) : \widehat{(-)}
      \end{equation}
      which form an adjunction when restricted to $\mathsf{Span}(\**, \mathcal C) = \mathsf{Cat} \downarrow \mathcal C$.
      Given $p: \mathcal D \to \mathcal C$, define $\hat{\mathcal D}$ to be the Grothendieck construction on the functor
      \begin{equation}
            % \begin{tikzcd}[row sep = tiny]
            %       \mathcal D \arrow[r]
            %       &
            %       \mathsf{Cat}
            %       \\
            %       d \arrow[r, mapsto]
            %       &
            %       \mathcal C \downarrow p(a).
            % \end{tikzcd}
            \mathcal D \longto \Cat,
            \qquad \qquad
            d \longmapsto \mathcal C \downarrow p(a).
      \end{equation}
      Explicitly, objects are pairs $(d, c \xrightarrow{\alpha} p(d))$,
      and maps are pairs $(d \xrightarrow{f} d', c \xrightarrow{g} c')$ such that the diagram below commutes.
      \begin{equation}
            \begin{tikzcd}[row sep = small]
                  c \arrow[r, "g"] \arrow[d, "\alpha"']
                  &
                  c' \arrow[dd, "{\alpha'}"]
                  \\
                  p(d) \arrow[d, "{p(f)}"]
                  \\
                  p(d') \arrow[r, equal]
                  &
                  p(d')
            \end{tikzcd}
      \end{equation}
      This has the natural structure of a fibration over $\mathcal C$,
      sending $(a, c \to p(a))$ to $c$, with
      Cartesian arrows defined by precomposition.
      % 
      Moreover, for any fibration $E$ over $\mathcal C$ and map of weak spans
      \begin{equation}
            \begin{tikzcd}[row sep = tiny]
                  \mathcal D \arrow[dr, "p"] \arrow[dd, "F"']
                  \\
                  & \mathcal C
                  \\
                  \mathcal E \arrow[ur, "q"']
            \end{tikzcd}
      \end{equation}
      there is a natural map $\hat{\mathcal D} \to E$ in $\mathsf{Fib}(\mathcal C)$, defined by
      \begin{equation}
            (d, c \xrightarrow{\alpha} p(d)) \mapsto \alpha^{\**} \Phi_d^{\**} F(d).
      \end{equation}
\end{remark}




{\color{blue}
  Alternatively, consider the Grothendieck construction on the functor
  \begin{equation}
        \label{FG_GR_EQ}
        % \begin{tikzcd}[row sep = tiny]
        %       \mathsf F^{G,op} \arrow[r]
        %       &
        %       \mathsf{Set}
        %       \\
        %       A \arrow[r, mapsto]
        %       &
        %       \Set^{O_G^{op}}(\Phi(A), \underline{\mathfrak C}),
        % \end{tikzcd}      
        \mathsf F^{G,op} \longto \Set,
        \qquad \qquad
        A \longmapsto \Set^{O_G^{op}}(\Phi(A), \underline{\mathfrak C}),
  \end{equation}
  where $\Phi: \Set^G \to \Set^{O_G^{op}}$ sends a $G$-set $X$ to its fixed-point system $G/H \mapsto X^H$.
  We will denote this by $\mathsf F^G \wr \underline{\mathfrak C}$.
} % ---------- COLOR BLUE ----------

{\color{red}
  Another option:
  given any category $\mathcal C$, define $\mathsf F^G \wr \mathcal C$
  as the Grothendieck construction
  \begin{equation}
        % \begin{tikzcd}[row sep = tiny]
        %       \mathsf F^{G,op} \arrow[r]
        %       &
        %       \mathsf{Set}
        %       \\
        %       A \arrow[r, mapsto]
        %       &
        %       \mathcal C^{G \ltimes A}.
        % \end{tikzcd}
        \mathsf F^{G,op} \longto \Set,
        \qquad \qquad
        A \longmapsto \mathcal C^{G \ltimes A}.
  \end{equation}
} % ---------- COLOR RED ----------

 






\subsection{Homotopy in a general model category}


%{\color{OliveGreen}
We recall the following about homotopies in general model categories.
\begin{definition}
      For any $A \in \V$, a \textit{cylinder} object for $A$ is an object $\mathbb C(A)$ equipped with a factorization of the fold map
      \begin{equation}
            \begin{tikzcd}
                  A \amalg A \arrow[r, tail]
                  &
                  \mathbb C(A) \arrow[r, "\simeq"]
                  &
                  A
            \end{tikzcd}
      \end{equation}
      into a cofibration followed by a weak equivalence.
      
      For the tensor unit $1_\V$, we write $\mathbb C = \mathbb C(1_\V)$, and call this simply a \textit{cylinder} in $\V$.
      % A cylinder for $A$ is called \textit{good} if the second map is a trivial fibration.
      
      A (left) \textit{homotopy} between maps $f,g: A \to B$ in $\V$ is a map $H_{fg}: \mathbb C(A) \to \V(A,B)$ such that
      the diagram below commutes.
      \begin{equation}
            \begin{tikzcd}[row sep = tiny]
                  A \amalg A \arrow[rr, "{(f,g)}"] \arrow[dr]
                  &&
                  B
                  \\
                  &
                  \mathbb C(A) \arrow[ur, "H_{fg}"']
            \end{tikzcd}
      \end{equation}
\end{definition}

% \begin{remark}
%       If $B$ is fibrant, we may lift any homotopy to a homotopy out of a good cylinder,
%       using the functorial factorization
%       $\mathbb C(A) \overset{\sim}{\rightarrowtail} \mathbb C'(A) \xrightarrow{\sim}{\twoheadrightarrow} A$.
% \end{remark}

\begin{remark}
      \label{CYL_REM}
      For $\mathbb C$ a cylinder in $\V$,
      if $A \in \V$ is cofibrant, then $A \otimes \mathbb C$ is a cylinder object for $A$,
      as the fold map can be written
      \begin{equation}
            A \amalg A \simeq A \otimes (1_\V \amalg 1_\V) \rightarrowtail A \otimes \mathbb C \xrightarrow{\sim} A
      \end{equation}
      as $A \otimes (-)$ preserves cofibrations, as well as weak equivalences between cofibrant objects (by Ken Brown's lemma).
\end{remark}

These cylinder objects provide another description of the mapping sets in the homotopy category of $\V$.

\begin{proposition}       [{\cite[1.2.10]{Hov99}}]
      If $A$ is fibrant and $B$ cofibrant, then
      homotopy is an equivalence relation $\sim$ on $\V(A,B)$.
      Moreover, 
      $\Ho(V)(A,B) = \V(A_f, B_c)/\sim$.
\end{proposition}

We can use these cylinder objects to extend the notion of homotopy to $\V$-categories or $\V$-operads.

\begin{definition}
      \label{HTPY_DEFN}
      Given $\mathcal C \in \Cat(\V)$, define $\pi_0(\mathcal C)$ to be the (unenriched) category with
      the same objects as $\mathcal C$, and $\pi_0(\mathcal C)(c,d) = \Ho(\V)(1_\V, \mathcal C(c,d))$.

      We say maps $f,g \in \mathcal C(c,d)$ are \textit{homotopic}
      if the representing maps $f,g: 1_\V \to \mathcal C(c,d)$ are homotopic in $\V$.
      If $\mathcal C$ is fibrant, this is equivalent to $[f] = [g]$ in $\pi_0\mathcal C$.

      We say maps $f,g \in \O(c;d)$ are \textit{homotopic} if they are homotopic in $j^{\**}\O$. 
\end{definition}

\begin{lemma}
      If $\O \in \Op^{G, \mathfrak C}(\V)$ is cofibrant,
      and $f,g: 1_\V \to \O(c;d)$ are homotopic, then for any $\mathfrak C(\O)$ profile $\ksi$ such that either
      \begin{enumerate}[label = (\roman*)]
      \item the target of $\ksi$ is $c$, or
      \item $d$ is in the source of $\ksi$,
      \end{enumerate}
      the maps
      \begin{equation}
            f_{\**}, g_{\**}: \O(\ksi) \simeq \O(\ksi) \otimes 1_\V \to \O(\ksi) \otimes \O(c;d) \xrightarrow{\circ} \O(\ksi_c^d)
      \end{equation}
      are homotopic,
      where $\ksi_c^d$ is the profile created from $\ksi$ by replacing $c$ with $d$.
\end{lemma}
\begin{proof}
      This follows from Remark \ref{CYL_REM} and the following commuting diagram.
      \begin{equation}
            \begin{tikzcd}
                  \O(\ksi) \otimes (1_\V \amalg 1_\V) \arrow[r, "{(f_{\**}, g_{\**})}"] \arrow[d, tail]
                  &
                  \O(\ksi_c^d)
                  \\
                  \O(\ksi) \otimes \mathbb C \arrow[r, "H_{fg}"]
                  &
                  \O(c;d) \arrow[u, "{(-)_{\**}}"]
            \end{tikzcd}
      \end{equation}
\end{proof}

The following is immediate.
\begin{corollary}
      \label{HTPIC_ISO_COR}
      Let $\O$ be locally bifibrant in $\Op(\V)$. 
      If $f$ is homotopic to the identity on a color $c$, then for any profile $\ksi$ containing $c$, the map
      $f_{\**}: \O(\ksi) \to \O(\ksi)$
     is an isomorphism in the homotopy category $\Ho(\V)$.
\end{corollary}


% ------------------------------ ASSEMBLING HOMOTOPIES ------------------------------


\begin{definition}[{cf. \cite[Defn 6.16]{BP17}}]
        We say a symmetric monoidal model category $\V$ has \textit{cofibrant symmetric pushout powers} if
        for all (trivial) cofibrations $f$, the pushout product power $f^{\square n}$
        is a $\Sigma_n$-genuine (trivial) cofibration in $\V^{\Sigma_n}$. 
\end{definition}

We may assemble homotopies in the following manner.

\begin{lemma}
      \label{ASSEM_HOM_LEM}
      Suppose $\V$ has cofibrant symmetric pushout powers.
      If $\mathbb C$ is a cylinder, then so is each $\left(\mathbb C^{\otimes n}\right)^{K}$ for all $K \leq \Sigma_n$.
\end{lemma}
\begin{proof}
      % It suffices to show there exist
      % $1_\V \amalg 1_\V \rightarrowtail (\mathbb C^{\otimes n})^K$
      % and
      % $(\mathbb C^{\otimes n})^K \xrightarrow{\sim} (1_\V^{\otimes n})^K \simeq 1_\V$
      % (where the coherence axioms imply that $1_\V^{\otimes n}$ always has a trivial $\Sigma_n$-action).
      % 
      We consider each structure map separately.
      
      We note that, as $\otimes$ commutes with colimits,
      $(1_\V \amalg 1_\V)^{\otimes n} \simeq \coprod_{\chi} 1_{\V,\chi}$
      with $\chi$ running over all set maps $\underline{n} \to \set{0,1}$,
      and $\Sigma_n$ acting by pre-composition on $\chi$.
      Now, we have the composite
      \begin{equation}
%            \begin{tikzcd}
            1_\V \amalg 1_\V = 1_{\V, 0} \amalg 1_{\V, 1}
            \simeq
            \left((1_\V \amalg 1_\V)^{\otimes n}\right)
            \longrightarrow
            (1_\V \amalg 1_\V)^{\otimes n}
            \longrightarrow
            \mathbb C^{\otimes n}
%            \end{tikzcd}
      \end{equation}
      where $i: \underline{n} \to \set{i} \into \set{0,1}$ is the constant map,
      and the second arrow is a genuine $\Sigma_n$-cofibration
      as we may attach each $\Sigma_n$-orbit $\Sigma_n \chi$ individually via maps $\varnothing \to \amalg_{\V,\sigma\chi}$
      (and we've already attached the stable orbits).

      Now,
      we note that $(\mathbb C \to 1_\V)^{\otimes n}$ is a weak equivalence by induction using Ken Brown's lemma,
      as $\mathbb C^{\otimes n}$ is cofibrant,
      $\mathbb C^{\otimes n} \to 1_\V^{\otimes n}$ is a map between cofibrant objects,
      and $\mathbb C \otimes (-)$ preserves all trivial cofibrations.
      % 
      Let $Q(n) \to \mathbb C^{\otimes n}$ denote $(1_\V \to \mathbb C)^{\square n}$,
      which is a genuine trivial cofibration in $\V^{\Sigma_n}$ by the assumption on $\V$.
      Consider the pushout and induced maps in $\V^{\Sigma_n}$ below
      \begin{equation}
            \begin{tikzcd}
                  Q(n) \arrow[r, tail, "\sim"] \arrow[d, tail, "\sim"']
                  &
                  \mathbb C^{\otimes n} \arrow[d, tail, "\sim"] \arrow[drr, bend left, "\sim"]
                  \\
                  \mathbb C^{\otimes n} \arrow[r, tail, "\sim"] \arrow[rrr, bend right, "\sim"]
                  &
                  P \arrow[r, tail, dashed, "\sim"]
                  &
                  P' \arrow[r, two heads, "\sim", dashed]
                  &
                  1_\V^{\otimes n}
            \end{tikzcd}
      \end{equation}
      where we have factored the unique map $P \to 1_\V^{\otimes n}$ into a cofibration and fibration.

      Then, as (the proof of) \cite[Prop 6.3]{BP17} shows that $(-)^H$ preserves pushouts over genuine cofibrations
      as well as genuine \textit{trivial} cofibrations,
      and since $(-)^H$ preserves trivial fibrations by construction,
      we have the string of equivalences in $\V$ below for any $K \leq \Sigma_n$
      \begin{equation}
            \begin{tikzcd}
                  (\mathbb C^{\otimes n})^K \arrow[r, tail, "\sim"]
                  &
                  P^K \arrow[r, tail, "\sim"]
                  &
                  P'^K \arrow[r, two heads, "\sim"]
                  &
                  (1_\V^{\otimes n})^K \simeq 1_\V.
            \end{tikzcd}
      \end{equation}
\end{proof}

\begin{remark}
      \label{ASSEM_HOM_REM}
      The inclusions $i_k: 1_\V \to \mathbb C$, $k \in \set{0,1}$ factors through $\mathbb C^{\otimes n}$ as the composite below
      \begin{equation}
            1_\V
            \xrightarrow{i_k}
            \mathbb C^{\otimes n}
            \xrightarrow{\sim}
            1_\V^{\otimes j-1} \otimes \mathbb C \otimes 1_\V^{\otimes n-j}
            \xrightarrow{\simeq}
            \mathbb C
      \end{equation}
      for any $j \in \set{n}$.
      Indeed, the following diagram commutes since cylinders factor the fold map $\nabla$.
      \begin{equation}
            \begin{tikzcd}
                  1_\V \amalg 1_\V \arrow[r]
                  &
                  \coprod_\chi 1_{\V, \chi} \arrow[r, "\simeq"] \arrow[d, "\nabla \otimes id \otimes \nabla"]
                  &
                  (1_\V \amalg 1_\V)^{\otimes n} \arrow[r] \arrow[d, "\nabla \otimes id \otimes \nabla"]
                  &
                  \mathbb C^{\otimes n} \arrow[d]
                  \\
                  &
                  1_{\V, j\mapsto 0} \amalg 1_{\V, j \mapsto 1} \arrow[r, "\simeq"]
                  &
                  1_\V^{\otimes j-1} \otimes (1_\V \amalg 1_\V) \otimes 1_\V^{n-j} \arrow[r]
                  &
                  1_\V^{\otimes j-1} \otimes \mathbb C \otimes 1_\V^{n-j}
            \end{tikzcd}
      \end{equation}
      
      Thus, for any homotopy $\mathbb C^{\otimes n} \to \mathcal C$,
      there are $n$ associated homotopies $x_{i,0} \sim x_{i,1}$,
      and the inclusion $1_{\V, \chi} \to \mathbb C^{\otimes n}$ ``sees'' the tuple of objects $(x_{i,\chi(i)})$.
\end{remark}

% ------------------------------ diagonals for cylinder objects ----------

% \begin{definition}
%       The category $\V$ is said to \textit{have diagonals for cylinder objects} if
%       for any cylinder object $\mathbb C$ and $n \geq 0$ there exists a map
%       \begin{equation}
%             \Delta: \mathbb C \to \left(\mathbb C^{\otimes n}\right)^{\Sigma_n}
%       \end{equation}
%       such that the following diagram commutes.
%       \begin{equation}
%             \begin{tikzcd}
%                   1_\V \amalg 1_\V \arrow[r] \arrow[d]
%                   &
%                   \left((1_\V \amalg 1_\V)^{\otimes n}\right)^{\Sigma_n} \arrow[d]
%                   \\
%                   \mathbb C \arrow[r, "\Delta"]
%                   &
%                   \left(\mathbb C^{\otimes n}\right)^{\Sigma_n}
%             \end{tikzcd}
%       \end{equation}
% \end{definition}

% This implies the diagram below commutes for all $k \in \underline{n}$ (since $\mathbb C$ factors the fold map).
% \begin{equation}
%       \begin{tikzcd}
%             1_\V \arrow[rrr, "i_k"] \arrow[d, "i_k"]
%             &&&
%             \mathbb C
%             % &&&
%             % 1_\V \arrow[lll, "i_1"'] \arrow[d, "i_1"]
%             \\
%             \mathbb C \arrow[r, "\Delta"]
%             &
%             \mathbb C^{\otimes n} \arrow[r]
%             &
%             1_\V^{\otimes k-1} \otimes \mathbb C \otimes 1_\V^{\otimes n-k} \arrow[r, "\simeq"]
%             &
%             \mathbb C \arrow[u, equal]
%             % &
%             % \mathbb C \otimes 1_\V^{\otimes n-1} \arrow[l, "\simeq"']
%             % &
%             % \mathbb C^{\otimes n} \arrow[l]
%             % &
%             % \mathbb C \arrow[l, "\Delta"']
%       \end{tikzcd}
% \end{equation}

% It suffices to show the map $\mathbb C^{\otimes n} \to 1_\V^{\otimes n} \simeq 1_\V$ is a trivial $\Sigma_n$-fibration, as a lift of
% \begin{equation}
%       \begin{tikzcd}
%             1_\V \amalg 1_\V \arrow[r] \arrow[d, tail]
%             &
%             \left((1_\V \amalg 1_\V)^{\otimes n}\right)^{\Sigma_n} \arrow[r]
%             &
%             \left(\mathbb C^{\otimes n}\right)^{\Sigma_n} \arrow[d]
%             \\
%             \mathbb C \arrow[rr, two heads, "\simeq"]
%             &&
%             1_\V \simeq 1_\V^{\otimes n} \simeq \left(1_\V^{\otimes n}\right)^{\Sigma_n}
%       \end{tikzcd}
% \end{equation}
% would satisfy these properties.
% \todo[inline]{come back: this need not happen. It may only be a weak equivalence.}
% 
% } % END OF OLIVE GREEN


\section{Colored Operads}


\subsection{Non-Equivariant Colored Operads}

Fix a closed symmetric monoidal category $(\V, \otimes)$.

\begin{definition}
      Fix a set $\mathfrak C$ of \textit{colors}.
      A tuple
      $\ksi = (c_1, \ldots, c_n; c_0) \in \mathfrak C^{\times n} \times \mathfrak C$
      is called a \textit{signature} of $\mathfrak C$.
      A \textit{$\mathfrak C$-colored operad} $\mathcal{O}$ in $\V$ consists of the following data:
      \begin{enumerate}[label = (\arabic*)]
      \item An object $\O(\ksi) \in \V$ for each signature $\ksi$.
      \item For each $c \in \mathfrak C$, a \textit{unit} $1_c \in \O(c;c)$.                        
      \item For any signature $\ksi \in \mathfrak C^{\times n+1}$ and $\sigma \in \Sigma_n$, a map $\O(\xi) \to \O(\sigma \cdot \xi)$,
            where $\Sigma_n$ acts on the left of $\mathfrak C^{\times n+1}$ by acting on the first $n$ coordinates.
            Explicitly, this is a map
            \begin{equation}
                  \O(c_1, \ldots, c_n; c_0) \xrightarrow{\sigma} \O(c_{\sigma^{-1}1}, \ldots, c_{\sigma^{-1}n}; c_0).
            \end{equation}
      \item For any compatible signatures $\xi = (c_1, \ldots, c_n; c_0)$, $\xi_i = (c_{1}^i, \ldots, c_{m_i}^i; c_i)$, a \textit{composition} map
            \begin{equation}
                  \O(\xi) \otimes \O(\xi_1) \otimes \ldots \otimes \O(\xi_n) \to \O(\xi \circ (\xi_i))
            \end{equation}
      \end{enumerate}
      where $\xi \circ (\xi_i) = (c_1^1,c_2^1,\dots,c_{m_n}^{n}; c_0)$,
      subject to the associativity, unitality, and $\Sigma$-equivariant compatibilities one would expect.      
      
      A map of $\mathfrak C$-colored operads is a compatible collection of maps
      $\set{\O(\xi) \to \O'(\xi)}_{\xi}$.
      
      Let $\Op^{\mathfrak C}(\V)$ denote the category of $\mathfrak C$-colored operads in $\V$.
\end{definition}

\begin{definition}
      \label{OP_MAP_DEFN}
      Given a map of $G$-sets $f: \mathfrak C' \to \mathfrak C$ and a $\mathfrak C$-colored operad $\O$,
      there is a natural $\mathfrak C'$-colored operad $f^*(\O)$, where
      \begin{equation}
            f^{\**}(\O)(\xi') = \O(f(\xi')),
      \end{equation}

      A \textit{map of colored operads} $\O' \to \O$ is given by the data of a map of colors $f: \mathfrak C' \to \mathfrak C$,
      and a map of $\mathfrak C'$-colored operads $\O' \to f^*(\O)$.
      
      Let $\Op(\V)$ denote the category of colored operads in $\V$.
\end{definition}

\begin{remark}
      (Colored) operads are also known as \textit{symmetric multicategories}.
\end{remark}

\begin{remark}
      The category $\Op(\V)$ is equal to the Grothendieck construction on the functor
      \begin{equation}
            % \begin{tikzcd}[row sep = tiny]
            %       \mathsf F^{op} \arrow[r] & \mathsf{Cat}
            %       \\
            %       \mathfrak C \arrow[r, mapsto] & \Op^{\mathfrak C}(\V).
            % \end{tikzcd}
            \mathsf F^{op} \longto \mathsf{Cat},
            \qquad \qquad
            \mathfrak C \longmapsto \Op^{\mathfrak C}(\V).
      \end{equation}
\end{remark}




\subsection{Equivariant Colored Operads}

Let $G$ be a finite group, with a fixed (random) total ordering.
\begin{notation}
      [{cf. \cite{BP17}}]
      Recall that $\mathsf F$ denotes the category of \textit{finite sets} equipped with a total order and set maps,
      or more accurately any full subcategory where the only ordered isomorphisms are the identity.
      
      Moreover, let $\mathsf O_G \into \mathsf F^G$ denote the full subcategory of \textit{transitive} $G$-sets.
      In particular, we note that the oribts $G/H$ are well-defined (using the chosen total order on $G$,
      and the ``minimal representative'' total order on $G/H$).
\end{notation}

\begin{notation}
      For any $G$-set $A$ and $a \in A$, let $G_a$ denote the stabilizer $\Stab_G(a)$ of $a$ in $G$.
\end{notation}

We will now move into the world of $G$-equivarinat colored operads.

\begin{definition}
      The category $\Op^G(\V)$ of  \textit{$G$-colored operads} in $\V$ is the category of
      $G$-objects in $\Op(\V)$.
\end{definition}

% Unpacking this definition, we see $\O \in \Op^G(\V)$ consists of the following data:
% \begin{enumerate}[label = (\arabic*)]
%       \setcounter{enumi}{-1}
% \item A $G$-set $\mathfrak C$ of colors.
% \item For each signature $\xi$ of $\mathfrak C$, an object $\O(\xi) \in \V$.
% \item where $G$ acts on $\mathfrak C^{\times n+1}$ diagonally (across all $n+1$ coordinates), and $\Sigma_n$ acts on the first $n$.
% \item For each $c \in \mathfrak C$, a \textit{unit} $1_c \in \O(c;c)^{G_c}$.
% \item For compatible signatures $\xi$, $\xi_1$, $\ldots$, $\xi_n$, \textit{composition maps}
%       \begin{equation}
%             \O(\xi) \otimes \O(\xi_1) \otimes \ldots \otimes \O(\xi_n) \to \O(\xi \circ (\xi_i)),
%       \end{equation}
% \end{enumerate}
% such that composition is
% compatible with the $G$-action on each component as well as the appropriate actions of $\Sigma$,
% and is unital and associative. 

We would like to unpack this definition, and first do so in general.

% -------------------- Equivariant Grothendieck Constructions --------------------

Suppose $E = \mathcal D \ltimes \mathcal C_{(-)}$ is the Grothendieck construction on the functor
\begin{equation}
      % \begin{tikzcd}[row sep = tiny]
      %       \mathcal D^{op} \arrow[r]
      %       &
      %       \mathsf{Cat}
      %       \\
      %       d \arrow[r, mapsto]
      %       &
      %       \mathcal C_{d}.
      % \end{tikzcd}
      \mathcal D^{op} \longto \Cat,
      \qquad \qquad
      d \longmapsto \mathcal C_d.
\end{equation}

A $G$-action on an element $(d,c_d)$ is given by compatible maps
$d \xrightarrow{g} d$ and
$c_d \xrightarrow{g_c} g^{\**}c_d$,
and $G$-maps between such objects
are pairs of maps $(d \xrightarrow{f} d, c_d \xrightarrow{f_c} f^{\**}c_{d'})$ which commute with the action.
Equivalently, and more concretely,
an object in $E^G$ is given by a pair $(d, \underline{c}_d)$ where
$d \in \mathcal D^G$ and
$\underline{c}_d$ is a $G$-indexed diagram $E G^{op} \to \mathcal C_d$,
such that $\underline{c}_d(g) = g_{d}^{\**}\underline{c}_d(e)$.
A $G$-map in this context is given by a pair $(f, \Phi_c)$ where
$f: d \to d'$ in $\mathcal D^G$ and
$\Phi: \underline{c}_d \Rightarrow f^{\**}\underline{c}_{d'}: E G^{op} \to \mathcal C_d$
(where we note that $f^{\**}\underline{c}_{d'}$ is a functor since $f$ is a $G$-map).

Put another way, we have the following.
\begin{lemma}
      \label{G_GR_LEM}
      The category of $G$-objects
      $\left(\mathcal D \ltimes \mathcal C_{(-)}\right)^G$
      is given by the Grothendieck construction on the functor
      \begin{equation}
            % \begin{tikzcd}[row sep = tiny]
            %       \mathcal D^{G,op} \arrow[r]
            %       &
            %       \Cat
            %       \\
            %       d \arrow[r, mapsto]
            %       &
            %       \Fun^{G^{op}}(E G^{op}, \mathcal C_d)
            % \end{tikzcd}
            \mathcal D^{G,op} \longrightarrow \Cat,
            \qquad \qquad
            d \longmapsto \Fun^{G^{op}}(E G^{op}, \mathcal C_d)
      \end{equation}
      where $\Fun^{G^{op}}$ denotes the category of $G^{op}$-functors and $G^{op}$-natural transformations,
      and $\mathcal C_d$ has a $G^{op}$ action induced by the $G$-action on $d$.
\end{lemma}
% ----------------------------------------------------------------------


Thus, for $E = \Op(\V)$, we see that an object $\O \in \Op^G(\V)$ consists of
a $G$-set $\mathfrak C$ of colors, and a compatible $G$-indexed diagram of $\mathfrak C$-colored operads.
Explicitly, $\O$ is the following data:
\begin{enumerate}[label = (\arabic*), start = 0]
\item A $G$-set $\mathfrak C = \mathfrak C(\O)$ of colors.
\item For each signature $\ksi$, an object $\O(\ksi) \in \V$.
\item \label{GUNIT_LBL} 
      For each $c \in \ksi$, a \textit{unit} $1_c \in \O(c;c)$.
\item \label{SACTION_LBL}
      For all signatures $\ksi \in \mathfrak C^{\times n + 1}$ and $\sigma \in \Sigma_n$, maps $\O(\ksi) \to \O(\sigma \cdot \ksi)$,
      which are unital and associative.
\item \label{COMP_LBL}
      For all compatible signatures $\ksi \in \mathfrak C^{\times n+1}$, $\ksi_i$,
      \textit{composition maps} $\O(\ksi) \otimes \O(\ksi_1) \otimes \dots \otimes \O(\ksi_n) \to \O(\ksi \circ (\ksi_i))$,
      which are unital, associative, and appropriately $\Sigma$-equivariant.
\item \label{GACTION_LBL}
      For all $g \in G$, maps $\O(\ksi) \to g^{\**}\O(\ksi) = \O(g \cdot \ksi)$
      (with $G$ acting on $\mathfrak C^{\times n+1}$ diagonally)
      which are unital and associative, and which commute with the composition maps
      (if $\ksi, (\ksi_i)$ are compatible, then so are $g \ksi, (g \ksi_i)$).
\end{enumerate}
Synthesizing, we may combine \ref{SACTION_LBL} and \ref{GACTION_LBL} into
\begin{enumerate}
\item[($3'$)] For all signatures $\xi \in \mathfrak C^{\times n+1}$ and $(g,\sigma) \in G\times \Sigma_n$, maps
      $\O(\xi) \to \O((g,\sigma)\cdot \xi)$
      which are unital and associative.
\end{enumerate}

and replace \ref{GUNIT_LBL} and \ref{COMP_LBL} with
\begin{enumerate}
\item[($2'$)] For each $c \in \mathfrak C$, a $G_c$-fixed unit $1_c \in \O(c;c)^{G_c}$.
\item[($4'$)] For all compatible signatures $\ksi, (\ksi_i)$,
      composition maps $\O(\ksi) \otimes \bigotimes_i \O(\ksi_i) \to \O(\ksi \circ (\ksi_i))$
      which are unital, assocative, $G$-equivariant, and approriately $\Sigma$-equivariant.
\end{enumerate}

From Lemma \ref{G_GR_LEM}, we can see that
maps of $G$-operads $\O \to \P$ are given by a pair $(f, F)$ where
$f: \mathfrak C(\O) \to \mathfrak C(\P)$ is a map of $G$-sets, and
$F: \O \to f^{\**}\P$ is a map in $\Op^{\mathfrak C(\O)}(\V)$ that commutes with the $G$-action
(or, equivalently, so that the $g^{\**}F$ assemble into a map of $G$-indexed diagrams).

Moreover, for any $G$-set $\mathfrak C$ let $\Op^{G, \mathfrak C}(\V) \subseteq \Op^G(\V)$
denote the subcategory of $\mathfrak C$-colored operads and maps which are the identity on colors.
Then 
\begin{equation}
      \Fun^{G^{op}}(E G^{op}, \Op^{\mathfrak C}(\V)) \simeq \Op^{G, \mathfrak C}(\V),
\end{equation}
and thus Lemma \ref{G_GR_LEM} implies the following.
\begin{lemma}
      $\Op^G(\V)$ is isomorphic to the Grothendieck construction on the functor
      \begin{equation}
            % \begin{tikzcd}[row sep = tiny]
            %       \mathsf (F^G)^{op} \arrow[r] & \mathsf{Cat}
            %       \\
            %       \mathfrak C \arrow[r, mapsto] & \Op^{G,\mathfrak C}(\V).
            % \end{tikzcd}
            \mathsf F^{G,op} \longto \Cat,
            \qquad \qquad
            \mathfrak C \longmapsto \Op^{G, \mathfrak C}(\V).
      \end{equation}
\end{lemma}


\begin{remark}
	Unlike in the single-colored case, $\Op^G(\V)$ does \textit{not} coincide with the category of colored operads in $\V^G$.
	Indeed, objects in $\Op(\V^G)$ have a fixed $G$-set of colors,
        and each level $\O(\xi)$ has an action by the full group $G$
	(though only a partial action by $\Sigma_{|\xi|}$).
\end{remark}


\subsubsection{Categorical description: colored trees}
\label{COMEGA_SEC}

Given a $G$-set $A$, let $G \ltimes A$ denote the Grothendieck construction on the functor $G^{op} \to \Set$
defining the action on $A$,
with object set $A$ and morphisms $g: a \to g\cdot a$ for all pairs $(g,a) \in G \times A$
\footnote{In the literature, $G \ltimes A$ is often called the \textit{translation category} of $A$, and denoted $B_A G$.}.
Further, if $A$ is a $G \times \Sigma$-set, then $G \ltimes (\Sigma \ltimes A) \simeq \Sigma \ltimes (G \ltimes A)$.

We now categorify the set of $\mathfrak C$-signatures.
\begin{definition}
      Let $\Sigma_{\mathfrak C}$ denote the \textit{$\mathfrak C$-symmetric category}
      and $G \ltimes \Sigma_{\mathfrak C}$ the \textit{$(G,\mathfrak C)$-symmetric category}, given as below.      
      \begin{equation}
            \Sigma_{\mathfrak C} = \coprod_{n \geq 0} \Sigma_n \ltimes \mathfrak C^{\times n} = \Sigma^{+1} \wr \mathfrak C
            \qquad \qquad
            G \ltimes \Sigma_{\mathfrak C} = \coprod\limits_{n\geq 0} (G \times \Sigma_n) \ltimes \mathfrak C^{\times n}.
      \end{equation}
      
      We observe that this is equivalent to the pullback on the left below,
      where $E: \Sigma \to \mathsf F$ sends $\underline{n}$ to $\underline{n+1}$.
      \begin{equation}
            \label{COMEGA_B_EQ}
            \begin{tikzcd}
                  G \ltimes \Sigma_{\mathfrak C} \arrow[d] \arrow[r, "E"]
                  &
                  \mathsf F \wr (G \ltimes {\mathfrak C}) \arrow[d]
                  & %
                  \Omega_{\mathfrak C} \arrow[d] \arrow[r, "E"]
                  &
                  \mathsf F \wr \mathfrak C \arrow[d]
                  & %
                  G \ltimes \Omega_{\mathfrak C} \arrow[d] \arrow[r, "E"]
                  &
                  \mathsf F \wr (G \ltimes {\mathfrak C}) \arrow[d]
                  \\
                  \Sigma \times G \arrow[r, "E"]
                  &
                  \mathsf F \wr G
                  & %
                  \Omega \arrow[r, "E"]
                  &
                  F
                  & %
                  \Omega \times G \arrow[r, "E"]
                  &
                  \mathsf F \wr G
            \end{tikzcd}
      \end{equation}
      More generally, let $\Omega_{\mathfrak C}$ and $G \ltimes \Omega_{\mathfrak C}$ be the middle and right pullbacks above,
      where $E: \Omega \to \mathsf F$ sends a tree $U$ to its set $E(U)$ of edges.

      We have a natural inclusion of categories $G \ltimes \Sigma_{\mathfrak C} \into G \ltimes \Omega_{\mathfrak C}$,
      and as such we will called elements of these categories
      \textit{colored trees} (or \textit{colored corollas}),
      and denote them by $(U,\mathfrak c)$, where $\mathfrak c: E(U) \to \mathfrak C$ is a map of sets.
\end{definition}

\begin{remark}
      When $G = \set{e}$ and $\mathfrak C = \set{*}$, $G \ltimes \Sigma_{\mathfrak C} = \Sigma$.
\end{remark}

Unpacking definitions, we see that a map $(U, \mathfrak c) \to (V, \mathfrak d)$ is given by
a map $f: U \to V$ in $\Omega$ and an element $g\in G$,
such that $g.\mathfrak c(e) = \mathfrak d(f(e))$ for all $e \in E(U)$.
\begin{equation}
      \begin{tikzcd}
            E(U) \arrow[r, "f"] \arrow[d, "\mathfrak c"']
            &
            E(V) \arrow[d, "\mathfrak d"]
            \\
            \mathfrak C \arrow[r, "g"]
            &
            \mathfrak C
      \end{tikzcd}
\end{equation}

In particular, we have maps of the form
\begin{equation}
      g = (id, g): (U, E(U) \to \mathfrak C) \to (U, E(U) \to \mathfrak C \xrightarrow{g \cdot} \mathfrak C). 
\end{equation}

\begin{remark}
      Following Lemma \ref{PB_GR_EQ}, it is straightforward that
      $G \ltimes \Omega_{\mathfrak C}$ is equal to the
      Grothendieck construction on the functor
      \begin{equation}
            \label{COMEGA_B_GR_EQ}
            \begin{tikzcd}[row sep = tiny]
                  \Omega^{op} \times G^{op} \arrow[r]
                  &
                  \mathsf{Set}
                  \\
                  U \arrow[r, mapsto]
                  &
                  \Hom_{\mathsf{Cat}}(E(U), G \ltimes \mathfrak C)
                  \\
                  (\phi: U \to V, g) \arrow[r, mapsto]
                  &
                  (E(U) \xrightarrow{\phi} E(V) \xrightarrow{\mathfrak c} \mathfrak C \xrightarrow{g^{-1}} \mathfrak C)
            \end{tikzcd}
      \end{equation}
      % \todo[inline]{compare with genuine case: RHS equals
      %   $\Hom(\Phi(E(U)), \Phi(\mathfrak C))$
      %   equals
      %   $\Hom(\Phi(E(G \cdot U)), \mathfrak C_{fr}))$}
      and a similar result holds for $G \ltimes \Sigma_{\mathfrak C}$. 
\end{remark}


% -------------------- Replace $\mathfrak C$ with a coefficient system --------------------
\begin{remark}
      Note that we can replace the $G$-set $\mathfrak C$ with a
      \textit{coefficient system} $\underline{\mathfrak C}$ in multiple ways,
      defining categories $\Omega_{\UC}$.
      In each case, the $G$-set definitions are recovered by replacing $\UC$ with $O_G \ltimes \Phi(\mathfrak C)$,
      and the resulting categories are always equal to $G \ltimes \Omega_{\mathfrak C(G/e)}$.

      % --------------------------
      
      First, we can substitute \eqref{COMEGA_B_EQ} for the rectangle of pullbacks below
      \begin{equation}
            \label{COMEGA_OG_EQ}
            \begin{tikzcd}
                  \underline{\mathfrak C}\Omega \arrow[d] \arrow[r, "E"]
                  &
                  \mathsf F \wr (G \ltimes \mathfrak C(G/e)) \arrow[r] \arrow[d]
                  &
                  \mathsf F \wr (O_G \ltimes \UC) \arrow[d]
                  \\
                  \Omega \times G \arrow[r, "E"]
                  &
                  \mathsf F \wr G \arrow[r]
                  &
                  \mathsf F \wr O_G
            \end{tikzcd}
      \end{equation}
      (that the right square sans $\mathsf F \wr (-)$ is a pullback is straightforward).
      % with $\mathfrak C(G/e) \into \underline{\mathfrak C}$, $G \ltimes \mathfrak C(G/e) \to O_G \ltimes \mathfrak C(G/e)$, and $G \into O_G$ the natural inclusions.

      Analogously to \eqref{COMEGA_B_GR_EQ}, this is equal to the Grothendieck construction on the functor
      \begin{equation}
            \label{COMEGA_OG_GR_EQ}
            \begin{tikzcd}[row sep = tiny]
                  \Omega^{op} \times G^{op} \arrow[r]
                  &
                  \mathsf{Set}
                  \\
                  U \arrow[r, mapsto]
                  &
                  \Hom_{\mathsf{Cat} \downarrow O_G}((G \cdot E(U))/G, O_G \ltimes \UC)
                  =
                  \Hom_{\mathsf{Cat}}(E(U), \mathfrak C(G/e)).
            \end{tikzcd}
      \end{equation}

      % ------------------------------------------------------------

      {\color{blue} % -------------------- ALTERNATIVELY ------------------------------
      Alternatively (cf. \eqref{FG_GR_EQ}),
      $\UC \Omega$ is isomorphic to the rectangle of pullbacks
      \begin{equation}
            \label{COMEGA_FG_EQ}
            \begin{tikzcd}
                  \Omega_{\UC} \arrow[d] \arrow[r]
                  &
                  \mathsf F \wr (G \ltimes {\mathfrak C(G/e)}) \arrow[r] \arrow[d]
                  &
                  \mathsf F^G \wr \UC \arrow[d]
                  \\
                  \Omega \times G \arrow[r]
                  &
                  \mathsf F \wr G \arrow[r]
                  &
                  \mathsf F^G,
            \end{tikzcd}
      \end{equation}
      where the map $F \wr G \to F^G$ sends objects $A$ to $G \cdot A$ and
      maps $(\phi, (g_a))$ to $G \cdot A \to G \cdot B$, $(h, a) \mapsto (h g_a^{-1}, \phi(a))$.
      % bottom arrow sends $U$ to $G \cdot E(U)$, and
      % on arrows sends $(\phi,g)$ to
      % \begin{equation}
      %       \begin{tikzcd}[row sep = tiny]
      %             G \cdot E(V) \arrow[r, "\phi \times {g^{-1}}"]
      %             &
      %             G \cdot E(U)
      %             \\
      %             (h,s) \arrow[r, mapsto]
      %             &
      %             {(hg^{-1}, \phi (s))}.
      %       \end{tikzcd}
      % \end{equation}
      Further, this is equal to the Grothendieck construction (cf. \eqref{COMEGA_B_GR_EQ})
      \begin{equation}
            \label{COMEGA_FG_GR_EQ}
            \begin{tikzcd}[row sep = tiny]
                  \Omega^{op} \times G^{op} \arrow[r]
                  &
                  \mathsf{Set}
                  \\
                  U \arrow[r, mapsto]
                  &
                  \Hom_{\mathsf{Set}^{O_G^{op}}}(\Phi(G \cdot E(U)), \UC)
            \end{tikzcd}
      \end{equation}
      In this context, a map $\phi: (U,\mathfrak c) \to (V, \mathfrak d)$ between colored trees is itself colored iff
      the triangle below commutes.
      \begin{equation}
            \label{COLOR_MAP_FG_EQ}
            \begin{tikzcd}[row sep = tiny]
                  \Phi(G \cdot E(T)) \arrow[rr, "\phi"] \arrow[dr, "\mathfrak c"']
                  &&
                  \Phi(G \cdot E(V)) \arrow[dl, "\mathfrak d"]
                  \\
                  &
                  \UC
            \end{tikzcd}
      \end{equation}
      } % -------------------- ALTERNATIVELY: END ------------------------------
\end{remark}

\begin{definition}
      The category of \textit{symmetric $(G,\mathfrak C)$-sequences} in $\V$, denoted $\Sym^{G,\mathfrak C}(\V)$, is
      the category of functors $X: G \ltimes \Sigma_{\mathfrak C}^{op} \to \V$.
\end{definition}

\begin{remark}
      \label{COLOR_CHANGE_REM}
      As the set of objects in $G \ltimes \Sigma_{\mathfrak C}$ is precisely the set of $\mathfrak C$-signatures,
      every $\mathfrak C$-colored operad has an underlying symmetric $(G,\mathfrak C)$-sequence.
      
      Moreover, extending Defintion \ref{OP_MAP_DEFN}, we have that any map of $G$-sets
      $F: \mathfrak C \to \mathfrak C'$ induces a pair of adjoints
      \begin{equation}
            \label{COLOR_CHANGE_EQ}
            \begin{tikzcd}
                  \Op^{G, \mathfrak C'}(\V) \arrow[r, shift right, "F^{\**}"'] % \arrow[d, "\mathsf{fgt}"']
                  &
                  \Op^{G, \mathfrak C}(\V) \arrow[l, shift right, "F_!"'] % \arrow[d, "\mathsf{fgt}"]
                  \\
                  \Sym^{G, \mathfrak C'}(\V) \arrow[r, shift right, "F^{\**}"']
                  &
                  \Sym^{G, \mathfrak C}(\V) \arrow[l, shift right, "F_{!!}"'].
            \end{tikzcd}
      \end{equation}
      where we highlight that $F^{\**}$ commutes with the forgetful functor down to sequences, but $F_!$ does not.
      
      We also record the straightforward observation that if a map $F: \O_1 \to \O_2$ is color-fixed, then
      a commuting square (resp. lifting diagram, pullback) as in the middle below is
      equivalent to such squares on the left and right.
      \begin{equation}
            \label{COLOR_SQ_EQ}
            \begin{tikzcd}
                  a_! \O_1 \arrow[d, "a_! F"'] \arrow[r, "a"]
                  &
                  \O \arrow[d, "p"]
                  &&
                  \O_1 \arrow[d, "F"] \arrow[r, "a"]
                  &
                  \O \arrow[d, "p"]
                  &&
                  \O_1 \arrow[d, "F"] \arrow[r, "a"]
                  &
                  a^{\**} \O \arrow[d]
                  \\
                  a_! \O_2 \arrow[r]
                  &
                  p^{\**} \P
                  &&
                  \O_2 \arrow[r]
                  &
                  \P
                  &&
                  \O_2 \arrow[r]
                  &
                  a^{\**} p^{\**} \P
            \end{tikzcd}
      \end{equation}
\end{remark}

\begin{remark}
      We also have a inclusion-forgetful adjunction
      \begin{equation}
            \label{JSTAR_CAT_EQ}
            \begin{tikzcd}
                  \Op^{G, \mathfrak C}(\V) \arrow[r, shift right, "j^{\**}"'] \arrow[d, "{(-)^H}"']
                  &
                  \Cat^{G, \mathfrak C}(\V) \arrow[l, shift right, "j_!"'] \arrow[d, "{(-)^H}"']
                  \\
                  \Op^{\mathfrak C^H}(\V)  \arrow[r, shift right, "j^{\**}"']
                  &
                  \Cat^{\mathfrak C^H}(\V) \arrow[l, shift right, "j_!"']
            \end{tikzcd}
      \end{equation}
      and we see that $j^{\**}$ commutes with taking $H$-fixed points.
\end{remark}

Now, many of the natural functors around $\Omega$ and $\Sigma$ have generalizations to the colored setting,
which can be built through a straightforward use of the universal property of pullbacks.

\begin{definition}
      We have a natural \textit{vertex} functor
      $V: G \ltimes \Omega_{\mathfrak C} \to \Sigma \wr (G \ltimes \Sigma_{\mathfrak C})$,
      as colorings of a tree restrict to colorings of each vertex corolla.

      Similarly, there is a \textit{leaf-root} functor
      $\mathsf{lr}: G \ltimes \Omega_{\mathfrak C} \to G \ltimes \Sigma_{\mathfrak C}$,
      where the coloring of $\mathsf{lr}(T)$ is a restrict of the coloring of $T$.
\end{definition}


\todo[inline]{come back - fix narative}

Moreover, the algebraic structure on these operads is determined by a monad on symmetric sequences.

\begin{definition}
      Given $X \in \Sym^{G, \mathfrak C}$, let $\mathbb F^{\mathfrak C} X$ denote the left Kan extension below.
      \begin{equation} 
           \begin{tikzcd}
                  G \ltimes \Omega_{\mathfrak C}^{op}
                  \arrow[d, "\mathsf{lr}"']
                  \arrow[r, "V"]
                  &
                  (\Sigma \wr (G \ltimes \Sigma_{\mathfrak C}))^{op} \arrow[r, "X"]
                  \arrow[dl, Rightarrow]
                  &
                  (\Sigma \wr \V^{op})^{op} \arrow[r, "\otimes"]
                  &
                  \V
                  \\
                  (G \ltimes \Sigma_{\mathfrak C})^{op} \arrow[urrr, "\Lan = \mathbb F^{\mathfrak C} X"']
            \end{tikzcd}
      \end{equation}
\end{definition}

\subsection{Single-Colored Operads}
We first show that this generalizes the free single-colored operad monad.

Note that when $\mathfrak C = \set{\**}$, we have
$G \ltimes \Omega_{\mathfrak C} = \Omega \times G$ and 
$G \ltimes \Sigma_{\mathfrak C} = \Sigma \times G$.

\begin{notation}
      Given a functor $X : \C \to \mathsf{Fun}(\mathcal D, \V)$,
      let $\tilde X$ denote the adjoint functor $\tilde X: \C \times \mathcal D \to \V$.
\end{notation}

\begin{lemma}
      \label{SPAN_LAN_LEM}
      Conisder the two spans below.
      \begin{equation}
            \begin{tikzcd}
                  \C \arrow[d, "p"] \arrow[r, "X"]
                  &
                  \mathsf{Fun}(\mathcal D, \V)
                  &&
                  \C \times \mathcal D \arrow[d, "p \times \mathsf{id}"] \arrow[r, "\tilde X"]
                  &
                  \V
                  \\
                  \mathcal E
                  &
                  &&
                  \mathcal E \times \mathcal D
            \end{tikzcd}
      \end{equation}
      
      Then $\Lan_p X$ is adjoint to $\Lan_{p \times \mathsf{id}} \tilde X$. 
\end{lemma}
\begin{proof}
      Using the pointwise description of the Kan extension, we have
      \begin{align}
        \widetilde{\Lan_p X}(e,d)
        &= (\Lan_p X(e))(d)
          = \left(
          \colim\limits_{\substack{ \C \downarrow e \\ p(c) \to e}} X(c)
        \right)(d)
        = \colim\limits_{\substack{ \C \downarrow e \\ p(c) \to e}}(X(c)(d))
        = \colim\limits_{\substack{ \C \downarrow e \\ p(c) \to e}}(\tilde X(c,d))\\
        &= \colim\limits_{\substack{ \C \times \set{d} \downarrow (e,d) \\ p(c) \to e}}(\tilde X(c,d))
        \cong \colim\limits_{\substack{ \C \times \mathcal D \downarrow (e,d) \\ (p(c),d') \to (e,d)}}(\tilde X(c,d'))
        = \Lan_{p \times \mathsf{id}}\tilde X(c,d),
      \end{align}
      where the isomorphism holds by a straightforward finality argument.
      On maps, a similar argument holds.
\end{proof}

\begin{notation}[\cite{BP17}]
      Let $\mathbb F'$ denote the \textit{free single-colored operad monad} on $\V$, given by the left Kan extension of the following diagram.
      \begin{equation}
            \begin{tikzcd}
                  \Omega^{op}
                  \arrow[d, "\mathsf{lr}"']
                  \arrow[r, "V"]
                  &
                  (\Sigma \wr \Sigma)^{op} \arrow[r, "X"]
                  \arrow[dl, Rightarrow]
                  &
                  (\Sigma \wr \V^{op})^{op} \arrow[r, "\otimes"]
                  &
                  \V
                  \\
                  \Sigma^{op} \arrow[urrr, "\Lan = \mathbb F' X"']
            \end{tikzcd}
      \end{equation}
\end{notation}

\begin{proposition}
      \label{TEST_PROP}
      $\mathbb F^{\set{\**}}$ is a monad, and moreover
      the category of $\mathbb F^{\set{\**}}$-algebras in $\mathsf{Fun}(\Sigma \times G, \V)$ is equivalent to
      the category of $\mathbb F'$-algebras in $\mathsf{Fun}(\Sigma, \V^G)$.
\end{proposition}
\begin{proof}
      Let $\tau: \tilde X \mapsto X$ denote the isomorphism of categories
      $\mathsf{Fun}(\Sigma \times G, \V) \xrightarrow{\tau} \mathsf{Fun}(\Sigma, \V^G)$.
      Then $\mathbb F^{\set{\**}} = \tau^{-1} \mathbb F' \tau$ by \ref{SPAN_LAN_LEM}, and so
      $\mathbb F^{\set{\**}}$ is in fact a monad, and the
      the isomorphism lifts to an isomorphism on the category of algebras.
\end{proof}

The general case will be given in Proposition \ref{FC_MONAD_PROP}.





\newpage

\section{Colored Genuine Equivariant Operads}

Throughout this section, we will abuse notation, and refer to
a coefficient system and its associated (Grothendieck) category over $O_G$ by the same name.

Idea: we have a \textit{coefficient system} $\UC$ of colors, and
a \textit{signature} will consist of a tuple $\xi = (x_1, \dots, x_n;x_0)$
with $x_i \in \mathfrak C(G/H_i)$ for subgroups $H_i \leq H_0 \leq G$
(or, equivalently, each $x_i$ is secretly in fact a whole $G$-diagram of objects).

\subsection{Colored $G$-Trees}

Extending \S \ref{COMEGA_SEC}, we make the following definitions.

\begin{definition}
      \label{EG_DEFN}
      For $T \in \Omega_G$, let $E_G(T)$ denote the set of \textit{edge orbits} $E(T)/G$.
      The \textit{edge orbit} functor $E_G: \Omega_G \to \mathsf F \wr \mathsf O_G$ sends a $G$-tree $T$ to
      % the tuple $(E_G(T), (G/G_e)_{Ge \in E_G(T)})$,
      % where we have chosen canonical representatives for elements in $E_G(T)$ by choosing
      % $e \in Ge$ minimal with respect to the planar structure on $T$.
      % Given $\phi: T \to S$, $G t \in E_G(T)$,
      % the $G t$ component of the image $E_G(\phi)$ is given the map
      % $(g_t)_{\**}: G/G_t \to G/G_s$, $h G_t \mapsto h g_t G_s$, where
      % $s$ in minimal in $G \phi(t)$ and
      % $g_t \in G$ such that $\phi(t) = g_t s$
      % (as $g_t$ is unique modulo $G_s$, the map $(g_t)_{\**}$ is well-defined).
      the tuple $(E_G(T), (Gt))$.
\end{definition}

\begin{definition}[{cf. \eqref{COMEGA_OG_EQ}}]
      Let $\underline{\mathfrak C}$ be a $G$-coefficient system of sets.
      Then the category $\underline{\mathfrak C}\Omega_G$ (resp. its subcategory $\UC \Sigma_G$)
      of \textit{$\underline{\mathfrak C}$-colored $G$-trees}
      (resp. \textit{$\UC$-colored $G$-corollas})
      is defined to be the pullback on the right (left) below.
      \begin{equation}
            \label{COMEGAG_OG_EQ}
            \begin{tikzcd}
                  \UC \Sigma_G \arrow[d] \arrow[r]
                  &
                  \mathsf F \wr \UC \arrow[d]
                  &&
                  \UC \Omega_G \arrow[d] \arrow[r]
                  &
                  \mathsf F \wr \underline{\mathfrak C} \arrow[d]
                  \\
                  \Sigma_G \arrow[r, "E_G"]
                  &
                  \mathsf F \wr \mathsf O_G
                  &&
                  \Omega_G \arrow[r, "E_G"]
                  &
                  \mathsf F \wr \mathsf O_G
            \end{tikzcd}
      \end{equation}
\end{definition}

Objects of $\UC \Omega_G$ are pairs $(T, \mathfrak c)$ of
a $G$-tree $T$ and
a functor $\mathfrak c: E_G(T) \to \underline{\mathfrak C}$ over $\mathsf O_G$.
Explicitly, each orbit of edges $G t$ % (with $t$ minimal)
is assigned a color $\mathfrak c(G t) \in \underline{\mathfrak C}(G t)$. % $\underline{\mathfrak C}(G/G_{t})$.
Morphisms $(T, \mathfrak c) \to (S, \mathfrak d)$ 
are given by maps of trees $\phi: T \to S$ such that, for every edge orbit $G t$ of $T$, we have
\begin{equation}
      \mathfrak c(G t) = \phi_{G t}^{\**} \mathfrak d(G \phi(t))
      % \mathfrak c(G t) = g_t^{\**} \mathfrak d(G s),
      %\phi_{e}^{\**}g_e^{\**}\mathfrak d(Gf),
\end{equation}
with $\phi_{G t}: G t \to G \phi(t)$.
% with $g_t \in G$ defined as in Definition \ref{EG_DEFN}.
% where $\phi_{e}: G / G_{e} \to G / G_{\phi(e)}$ is the map in $\mathsf O_G$ induced by $\phi$,
% and $\phi(e) = g_e f$ for $f \in Gf \in E_G(S)$ minimal; as $g_e$ is unique modulo $G_f$, $g_e^{\**}$ is well-defined.


\begin{remark}
      In analogue to \eqref{COMEGA_OG_GR_EQ}, we note that $\UC \Omega_G$ is equivalent to the Grothendieck construction
      \begin{equation}
            \label{COMEGAG_OG_GR_EQ}
            \begin{tikzcd}[row sep = tiny]
                  \Omega_G^{op} \arrow[r]
                  &
                  \mathsf{Set}
                  \\
                  T \arrow[r, mapsto]
                  &
                  \Hom_{\Cat \downarrow O_G}(E_G(T), \UC)
            \end{tikzcd}
      \end{equation}
      with the functor defined on arrows $\phi: T \to S$ by sending $\mathfrak c: E_G(S) \to \UC$ to the map
      $E_G(T) \to \UC$, $Gt \to \phi_{Gt}^{\**}\mathfrak c(G\phi(t))$. 
\end{remark}

\begin{remark}      
      Equivalently, we have $\Hom_{\Cat \downarrow O_G}(E_G(T), \UC) = \Hom_{\mathsf{Fib} (\mathsf O_G)}(\hat E_G(T), \UC)$,
      where $\hat E_G(T) \xrightarrow{p} \mathsf O_G$ is the Grothendieck fibration
      $(d, c \xrightarrow{\alpha} p(d)) \mapsto c$.
      The (standard) coloring of $G t$ is recovered by $\mathfrak c(G t, id_{G t})$, and
      a triangle of fibrations on the left below over $\mathsf O_G$
      implies that the right triangle commutes.
      \begin{equation}
            \begin{tikzcd}[column sep = small]
                  \hat E_G(T) \arrow[rr, "\phi"] \arrow[dr, "\mathfrak c"']
                  &&
                  \hat E_G(S) \arrow[dl, "\mathfrak d"]
                  &
                  (G t, id) \arrow[rr, mapsto] \arrow[dr, mapsto]
                  &&
                  (G \phi(t), \phi_{G t}) \arrow[dl, mapsto]
                  \\
                  &
                  \UC
                  &
                  &
                  &
                  \mathfrak c(G t) = \phi_{G t}^{\**} \mathfrak d(G \phi(t)).
            \end{tikzcd}
      \end{equation}
\end{remark}

\begin{remark}
      { \color{blue} % -------------------- ALTERNATIVELY: ----------------------------------------
        Alternatively, (cf. \eqref{COMEGA_FG_EQ})
        $\UC\Omega_G$ is isomorphic to the pullback below.
        \footnote{We note that the class of morphisms in $\mathsf F^G$ in the image of $E$, when restricted to $\Omega_G^0$
          are those isomorphic to an adjunction counit $G \cdot_H A|_H \to A$.}
        \begin{equation}
              \label{COMEGAG_FG_EQ}
              \begin{tikzcd}
                    \UC \Omega_G \arrow[d] \arrow[r]
                    &
                    \mathsf F^G \wr \underline{\mathfrak C} \arrow[d]
                    \\
                    \Omega_G \arrow[r, "E"]
                    &
                    \mathsf F^G.
              \end{tikzcd}
        \end{equation}
        
        Similarly to \eqref{COMEGA_B_GR_EQ}, $\underline{\mathfrak C}\Omega_G$ is isomorphic to the Grothendieck construction on the functor
        \begin{equation}
              \begin{tikzcd}[row sep = tiny]
                    \Omega_G^{op} \arrow[r]
                    &
                    \mathsf{Set}
                    \\
                    T \arrow[r, mapsto]
                    &
                    \Hom_{\Set^{\mathsf O_G^{op}}}(\Phi(E(T)), \underline{\mathfrak C}),
              \end{tikzcd}
        \end{equation}
        In particular, we note that
        $\Hom_{\mathsf{Cat}}(E(T), B_{\mathfrak C} G) = \Hom_{\Set^{\mathsf O_G^{op}}}(\Phi(E(T)), \Phi(\mathfrak C))$. 
        
        
        As in \eqref{COMEGA_FG_EQ}, a coloring is a map $\mathfrak c: \Phi E(T) \to \mathfrak C$ of coefficient systems,
        and morphisms are maps $\phi: T \to S$ which are compatible on colorings,
        in that the triangle below  (cf. \eqref{COLOR_MAP_FG_EQ}) commutes.
        \begin{equation}
              \begin{tikzcd}[row sep = tiny]
                    \Phi E(T) \arrow[rr, "f"] \arrow[dr, "\mathfrak c"']
                    &&
                    \Phi E(S) \arrow[dl, "\mathfrak d"]
                    \\
                    &
                    \underline{\mathfrak C}
              \end{tikzcd}
        \end{equation}
        It is easy to show this is equivalent to requiring that
        % $\mathfrak c(G/G_t,t) = \phi_t^{\**} \mathfrak d(G/G_{\phi(t)}, \phi(t))$,
        % where $\phi_t: G/G_t \to G/G_{\phi(t)}$ is the unique map preserving the coset of the unit.
        $\mathfrak c(G t, t) = \phi_{G t}^{\**} \mathfrak d(G \phi(t), \phi(t))$. 
        
        $\UC \Sigma_G$ can be defined similarly, with the relevant sources restricted to
        $\Sigma_G \subseteq \Omega_G$. 
      }
      %%%%%%%%%%%%% COLOR: BLUE ----------------------------------------
\end{remark}


\begin{remark}
      $\underline{\mathfrak C}\Omega_G$ is also naturally a root fibration, 
      that is, a split Grothendieck fibration over the orbit category.

      Formally, as $\mathsf F \wr (-)$ and pullbacks preserve such fibrations, and these are compatible under composition,
      this follows from the natural maps $\underline{\mathfrak C}\Omega_G \to \Omega_G \to \mathsf O_G$.
      Explicitly, $\underline{\mathfrak C}\Omega_G(I)$ has as objects those pairs $(T,\mathfrak c)$ such that
      % $T \simeq G \cdot_H T_*$ for $T_{\**} \in \Omega^H$.
      $r(T) = I$. 
      % In this case, we have a canonical isomorphism
      % $E_H(T_{\**}) \to E_G(T) \simeq E_H(T_{\**})$, and we have a natural factorization
      % \begin{equation}
      %       \begin{tikzcd}
      %             E_G(T) \simeq E_H(T_{\**}) \arrow[r, "\mathfrak c"] \arrow[dr]
      %             &
      %             \mathfrak C|_{H} \arrow[r, hookrightarrow] \arrow[d]
      %             &
      %             \mathfrak C \arrow[d]
      %             \\
      %             &
      %             O_H \arrow[r, hookrightarrow]
      %             &
      %             \mathsf O_G.
      %       \end{tikzcd}
      % \end{equation}
      Maps $\phi:(T,\mathfrak c) \to (S, \mathfrak d)$ in each fiber are called \textit{root-fixed}:
      as maps in $\Omega_G$, they are \textit{rooted} ($r(T) \to r(S)$ is a planar isomorphism),
      and moreover $\mathfrak c(r(T)) = \mathfrak d(r(S))$.
      
      Given $q: I \to J$ in the orbit category,
      the chosen Cartesian maps are the induced root pullback maps $q: q^{\**}T \to T$ on $G$-trees,
      with the coloring $q^{\**}\mathfrak c$ of $q^{\**}T$ defined by
      $(q^{\**}\mathfrak c)(G t) = q_{G t}^{\**}(G \phi(t))$.
      % for $b\in E(q^{\**}T)$, minimal in it's $G$-orbit, we have $q(b) = g a$ for some $g \in G$ and $a \in E(T)$ minimal in its orbit.
      % Moreover, as this $g$ is unique modulo $G_a$, we have that there is a well-defined map $g_*: G/G_{q(b)} \to G/G_a$,
      % and as $q$ induces a unique map $q_b^{\**}: G/G_b \to G/G_{q(b)}$, we have
      % \begin{equation}
      %       (q^{\**}\mathfrak c)(Gb) q_b^{\**}g^{\**}\mathfrak c(Ga).
      % \end{equation}

      {\color{blue} % ---------- ALTERNATIVELY --------------------
        Alternatively, on $\Phi E(q^{\**} T)$, we have
        $(q^{\**}\mathfrak c)(G/H, s) = q_b^{\**}(\mathfrak c(G/H, q(s)))$.
      } % --------------------------------------------------
\end{remark}

\begin{remark}
      We note that any \textit{planar} map of colored $G$-trees is always \textit{color-fixed}, in that
      $\mathfrak c(Ge) = \mathfrak d(G \phi (e))$ for all $Ge \in E_G(T)$.
\end{remark}

\begin{remark}
      A \textit{quotient} map in $\UC \Omega_G$ is any morphism such that the underlying map in $\Omega_G$ is a quotient.
\end{remark}

We have natural inclusions on the left
\begin{equation}
      \begin{tikzcd}
            \underline{\mathfrak C}\Sigma \arrow[d, "\iota"] \arrow[r]
            &
            \underline{\mathfrak C}\Omega \arrow[d]
            &&
            \Sigma \times G \arrow[d] \arrow[r]
            &
            \Omega \times G \arrow[d]
            \\
            \underline{\mathfrak C}\Sigma_G \arrow[r]
            &
            \underline{\mathfrak C}\Omega_G
            &&
            \Sigma_G \arrow[r]
            &
            \Omega_G
      \end{tikzcd}
\end{equation}
which forget to the uncolored inclusions on the right.
Specifically, $U \mapsto G \cdot U$ and, as $E_G(G \cdot U) = E(U)$, the associated coloring map is simply $\mathfrak c$ again.
On morphisms, $(\phi,g)$ maps to $g \cdot (\phi)_{G}$.

\begin{remark}
      Given $X\in \dSet_G$ with $X(\eta_{G/H}) = \mathfrak C(G/H)$, we have that
      $\UC\Sigma_G$ is equal to the category of \textit{profiles} $\partial\Omega[C] \to X$,
      where $C$ ranges over all of $\Sigma_G$.
      % come back: contextualize
\end{remark}


\subsection{Planar Strings and Stuff}

% come back

Main observation:
\begin{remark}
      Given a map $\phi: T \to S$ of $G$-trees such that $S$ is $\UC$-colored, there is a unique coloring on $T$ such that
      $\phi$ is a map of colored trees.
      Moreover, any factorization $T \to S' \to S$ in $\Omega_G$ of a map of colored trees $T \to S$
      comes equipped with a canonical coloring so that the factorization is in fact in $\UC\Omega_G$.

      As an upshot, the vast majority of the operations and constructions in \cite[\S 3-5]{BP17}
      follow through immediately to the colored setting,
      once we establish the requisit categorical characterizations of the objects in question.

      Later, when considering the model structures, the coloring will similarly not directly interact with
      any of the arguments in \textit{loc cite} as, in particular,
      the coloring has no effect on the breakdown into ``active'' and ``inert'' vertices
      (with the exception that in the partition product description of $\Aut((T_v, \mathfrak c)_{v \in V_G^{ac}(T)})$,
      the partition is not the particular of the underlying $G$-tree).
      
      In this section, we highlight some of those.
\end{remark}

\todo[inline]{Need to strike a balance between what to show explicitly, and what to just state.
  \S 3.4 and \S 4 from \cite{BP17} extend almost formally, though phrasing it as such...

  We still have natural span $\UC\Sigma_G \leftarrow \UC\Omega_G^0 \to \mathsf F_s \wr \UC\Sigma_G$, such that
  the left arrow is a map of rooted fibrations.
  This, plus whats already in \S 3.4 and \S 4, may be enough to just formally push through.}

Generalizing \cite[Remark 3.78]{BP17}
\footnote{We do this as opposed to the original definition: unless we somehow force the ``correct isotropy'' of color on the non-equivariant trees, we just see $\Phi\mathfrak C(G/e)$, and not the whole coefficient system.}
\begin{definition}
      Given $(T,\mathfrak c) \in \underline{\mathfrak C}\Omega_G$, a
      \textit{planar (resp. rooted) $T$-substitution datum} is a tuple
      $((U_{v_{Ge}}, \mathfrak c_{v_{Ge}}))_{v_{Ge} \in V_G(T)}$ of $\underline{\mathfrak C}$-colored $G$-trees along with
      planar (resp. rooted color-fixed) tall maps
      $T_{v_{Ge}} \to U_{v_{Ge}}$.

      A map of planar (resp. rooted) $T$-substitution data $(U_{v_{Ge}}) \to (V_{v_{Ge}})$ is a compatible tuple of planar (resp. rooted color-fixed) tall maps $(U_{v_{Ge}} \to V_{v_{Ge}})$.
      Let $\mathsf{Sub}_p(T)$ and $\mathsf{Sub}(T)$ denote the categories of planar (resp. rooted) $T$-substituion datum.
\end{definition}

\begin{lemma}[{cf. \cite[Prop. 3.41]{BP17}}]
      Let $(T,\mathfrak c) \in \underline{\mathfrak C}\Omega_G$ be a $\underline{\mathfrak C}$-colored $G$-tree.
      There are isomorphisms of categories
      \begin{equation}
            \label{SUB_EQUIV_EQ}
            \begin{tikzcd}[row sep = 4pt]
                  \mathsf{Sub}_p(T) \arrow[r, shift left]
                  &
                  (T, \mathfrak c) \downarrow \underline{\mathfrak C}\Omega_G^{pt}
                  \arrow[l, shift left]
                  &
                  \mathsf{Sub}(T) \arrow[r, shift left]
                  &
                  (T, \mathfrak c) \downarrow \underline{\mathfrak C}\Omega_G^{r}
                  \arrow[l, shift left]                  
                  \\
                  (U_{v_{Ge}}) \arrow[r, mapsto]
                  &
                  ((T, \mathfrak c) \to \colim_{Sc_G(T)}U_{(-)}).
                  &
                  (U_{v_{Ge}}) \arrow[r, mapsto]
                  &
                  ((T, \mathfrak c) \to \colim_{Sc_G(T)}U_{(-)}).
            \end{tikzcd}
      \end{equation}
      where $\underline{\mathfrak C}\Omega_G^{pt}, \underline{\mathfrak C}\Omega_G^{r}$ are the categories of planar tall (resp. rooted) maps under $(T, \mathfrak c)$. 
\end{lemma}
\begin{proof}
      This follows as in \cite[Prop. 3.41]{BP17}, going by induction on $n=|V_G(T)|$.
      Let $(U_T,\mathfrak c_{U_T})$ denote the colimit, if it exists.
      If $n$ is 0 or 1, $T$ is terminal in $Sc_G(T)$, and the coloring $\mathfrak c_{U_T}$ is just $\mathfrak c$.
      Otherwise, we have a decomposition $T = R \amalg_{Ge} S$ with
      the planar ordering on $Ge$ in $R$, $S$, and $T$ the same,
      $E_G(T) = E_G(R) \amalg_{Ge} E_G(S)$
      $\mathfrak c_{R} = \mathfrak c|_{E_G(R)}$,
      $\mathfrak c_{S} = \mathfrak c|_{E_G(S)}$,
      such that
      the existence of $U_T$ and $\mathfrak c_{U_T}$ follow from the existence of the pushout below in $\underline{\mathfrak C}\Omega_G^{pt}$.
      \begin{equation}
            \begin{tikzcd}
                  (\eta_{Ge}, \mathfrak c) \arrow[d] \arrow[r]
                  &
                  (U_S, \mathfrak c_{U_S}) \arrow[d, dashed]
                  \\
                  (U_R, \mathfrak c_{U_R}) \arrow[r, dashed]
                  &
                  (U_T, \mathfrak c_{U_T})
            \end{tikzcd}
      \end{equation}
      By induction, $U_S$, $U_R$, $\mathfrak c_{U_S}$, $\mathfrak c_{U_R}$ exist
      (with unique choices such that $(U_{v_{Ge}}, \mathfrak c_{U_{v_{Ge}}}) \into(U_R, \mathfrak c_{U_R})$ is planar).
      Forgetting colors, this is an equivariant grafting diagram, and hence the $G$-tree $U_T$ exists.
      Moreover, we have $E_G(U_T) = E_G(U_S) \amalg_{Ge} E_G(U_R)$, and so we have a well-defined coloring
      \begin{equation}
            \mathfrak c_{U_T}(Gf) =
            \begin{cases}
                  \mathfrak c_{U_R}(Gf) \qquad \qquad & Gf \in E_G(R) \\
                  \mathfrak c_{U_S}(Gf) & Gf \in E_G(S)
            \end{cases}
      \end{equation}
      since the overlap $Ge$ is in $T$, and hence it is dictated that $\mathfrak c_{U_T}(Ge) = \mathfrak c (Ge)$.
\end{proof}

\begin{lemma}[{cf. \cite[Lemma 3.63]{BP17}}]
      $\underline{\mathfrak C}\Omega_G^0 \to \mathsf F_s \wr \underline{\mathfrak C}\Sigma_G$
      sends root pullbacks to pullbacks over $\mathsf F_s \wr \mathsf O_G$.
\end{lemma}
\begin{proof}
      Exactly as in \textit{loc cite}, with the additional note that
      the coloring of $\psi^{\**}T$ is precisely such that each $(\psi^{\**}T)_{v_{Ge}} \to T_{v_{G\phi(e)}}$
      is a pullback in $\UC \Sigma_G$.
\end{proof}

\begin{definition}
      The category $\UC\Omega_G^n$ of \textit{colored planar $n$-strings} is the category
      whoses objects are strings
      \begin{equation}
            (T_0,\mathfrak c_0)
            \xrightarrow{\phi_1} (T_1, \mathfrak c_1)
            \xrightarrow{\phi_2} \ldots
            \xrightarrow{\phi_n} (T_n, \mathfrak c_n)
      \end{equation}
      where $(T_i, \mathfrak c_i) \in \UC\Omega_G$ and the $\phi_i$ are all colored planar tall maps,
      while arrows are commutative diagrams of quotient maps.
\end{definition}

\begin{remark}
      We observe
      \begin{enumerate}
      \item $\UC\Omega_G^\bullet \to \UC\Sigma_G$ is an augmented simplicial object in categories.
      \item $\UC\Omega_G^n \to \mathsf O_G$ is a root fibration.
      \item We have a vertex functor $V_G: \UC\Omega_G^{n+1} \to \mathsf F_s \wr \UC\Omega_G^n$ by
            \begin{equation}
                  \left(
                  (T_0,\mathfrak c_0)
                  \to (T_1, \mathfrak c_1)
                  \to \dots
                  \to (T_n, \mathfrak c_n)
                  \large)
                  \mapsto
                  \large(
                  (T_{1,v_{Ge}},\mathfrak c_1)
                  \to \dots \to
                  (T_{n,v_{Ge}}, \mathfrak c_n)
                  \right)_{v_{Ge} \in V_G(T_0)}
            \end{equation}
            where we write abusively denote by $T_{i,v_{Ge}}$ the $G$-tree $(T_{i,\bar\phi_i(f)^\uparrow \leq \bar\phi_i(f)})_{f \in Ge}$
            and by $\mathfrak c_i$ the restriction to any of its sub-$G$-trees.

            Alternatively, regarding the source above as a string of $n-1$ arrows in
            $(T_0, \mathfrak c_0) \downarrow \UC\Omega_G^{pt}$,
            the image under $V_G$ can be recognized as the inverse image under \eqref{SUB_EQUIV_EQ}.
      \end{enumerate}
\end{remark}

\begin{proposition}[{cf. \cite[Prop 3.82]{BP17}}]
      For any $n \geq 0$, the commutative diagram
      \begin{equation}
            \begin{tikzcd}
                  \UC\Omega_G^n \arrow[d, "d_{1,\dots,n}"'] \arrow[r, "V_G"]
                  &
                  \mathsf F_s \wr \UC\Omega_G^{n-1} \arrow[d, "\mathsf F \wr d_{0,\dots,n-1}"]
                  \\
                  \UC\Omega_G^0 \arrow[r, "V_G"']
                  &
                  \mathsf F_s \wr \UC\Sigma_G
            \end{tikzcd}
      \end{equation}
      is a pullback diagram in $\Cat$.
\end{proposition}
\begin{proof}
      This follows as in \textit{loc cite}, as the colorings descend to all sub-$G$-trees.
\end{proof}

% \begin{proposition}
%       [{cf. \cite[Lemma 4.28]{BP17}}]
%       $N_{\mathfrak C}$ on spans preserves right Kan extensions over $\mathsf F \wr \mathcal A \downarrow \mathsf F \wr \UC\Sigma_G$.
% \end{proposition}
% \begin{proof}
%       Also follows from non-equivariant proof.
% \end{proof}

Similarly, \cite[Prop 3.47, 3.90, 4.12, 4.15, 4.26, 4.28, 4.30]{BP17} naturally generalized to the colored-setting,
replacing all instances of $\Omega_G^n$ or $\Sigma_G$ with $\UC\Omega_G^n$ and $\UC\Sigma_G$.
In particular, this yields the following definitions and proposition.


% recall adjunctions with $\mathsf{Lan}$ and $\mathsf{Ran}$, canonical op-isos, etc}

\begin{definition}[{cf. \cite[Defn 4.16]{BP17}}]
      Suppose $\V$ is a symmetric monoidal category with diagonals.
      We define an endofunctor $N_{\UC}$ on $\mathsf{WSpan}^r(\UC\Sigma_G, \V^{op})$
      by letting $N_{\UC}(\UC\Sigma_G \leftarrow \mathcal A \to \V^{op})$ be given by the span
      \begin{equation}
            \begin{tikzcd}
                  \UC\Omega_G^0 \wr \mathcal A \arrow[d] \arrow[r, "V_G"]
                  &
                  \mathsf F \wr \mathcal A \arrow[d] \arrow[r]
                  &
                  \mathsf F \wr \V^{op} \arrow[r, "\otimes^{op}"]
                  &
                  \V^{op}
                  \\
                  \UC\Omega_G^0 \arrow[r, "V_G"'] \arrow[d]
                  &
                  \mathsf F \wr \UC\Sigma_G
                  \\
                  \UC\Sigma_G
            \end{tikzcd}
      \end{equation}
      where the given square is a pullback, and on arrows in the natural way.

      Moreover, we have a multiplication $\mu: N_{\UC} \circ N_{\UC} \Rightarrow N_{\UC}$ given by the natural isomorphism
      \begin{equation}\label{MULTDEFSPAN EQ}
            \begin{tikzcd}
                  \UC\Sigma_G \ar[equal]{d}&
                  \UC\Omega_{G}^1 \wr A \ar{r}{V_G} \ar{d}[swap]{d_{0}} \ar{l}&
                  \Fin \wr \UC\Omega_{G}^0 \wr A \ar{r}{\Fin \wr V_G} &
                  |[alias=FFOmega]| \Fin^{\wr 2} \wr A \ar{d}{\sigma^0} \ar{r} &
                  \Fin^{\wr 2} \wr \mathcal{V}^{op} \ar{d}{\sigma^0} \ar{r}{\otimes^{op}} &
                  \Fin \wr \mathcal{V}^{op} \ar{r}{\otimes^{op}} &
                  |[alias=dog]|
                  \mathcal{V}^{op} \ar[equal]{d}
                  \\
                  \UC\Sigma_G &
                  |[alias=Omega]|\UC\Omega_{G}^{0} \wr A \ar{rr}[swap]{V_G} \ar{l}&&
                  \Fin \wr A \ar{r} &
                  |[alias=cat]|
                  \Fin \wr \mathcal{V}^{op} \ar{rr}[swap]{\otimes^{op}} &&
                  \mathcal{V}^{op}
                  \arrow[Leftrightarrow, from=FFOmega, to=Omega,shorten <=0.15cm,,shorten >=0.15cm,"\pi_0"]
                  \arrow[Leftrightarrow, from=dog, to=cat,shorten <=0.15cm,,shorten >=0.15cm,"\alpha"]
            \end{tikzcd}
      \end{equation}
      and a unit $\eta: id \Rightarrow N_{\UC}$ give by the strictly commuting diagram
      \begin{equation}\label{UNITSPAN EQ}
            \begin{tikzcd}
                  \UC\Sigma_G \ar[equal]{d} &
                  A \ar{l} \ar{d}[swap]{s_{-1}} \ar[equal]{r} &
                  A \ar{d}{\delta^0} \ar{r} &
                  \mathcal{V}^{op} \ar{d}{\delta^0} \ar[equal]{r}&
                  \mathcal{V}^{op} \ar[equal]{d}
                  \\
                  \UC\Sigma_G &
                  \UC\Omega_{G}^{0} \wr A \ar{l} \ar{r}[swap]{V_G}&
                  \Fin \wr A \ar{r} &
                  \Fin \wr \mathcal{V}^{op} \ar{r}[swap]{\otimes^{op}} &
                  \mathcal{V}^{op}.
            \end{tikzcd}
      \end{equation}	
\end{definition}

\begin{proposition}
      [{cf. \cite[Prop 4.19]{BP17}}]
      $(N_{\UC},\mu,\eta)$ is a monad on $\mathsf{WSpan}^r(\UC\Sigma_G, \V^{op})$.
\end{proposition}
\begin{proof}
      Nothing new.
\end{proof}


\begin{definition}
      The \textit{genuine $\UC$-colored operad monad} is the monad
      $\mathbb F_{G,\UC}$ on $\Sym_{G, \UC}(\V) = \mathsf{Fun}(\UC\Sigma_G^{op}, \V)$ given by
      \begin{equation}
            \mathbb F_{G,\UC} = \Lan \circ N_{\UC} \circ \iota
      \end{equation}
      with multiplication and unit given by
      \begin{equation}
            \mathsf{Lan} \circ N_{\UC} \circ \iota \circ
            \mathsf{Lan} \circ N_{\UC} \circ \iota
            \overset{\simeq}{\Leftarrow}
            \mathsf{Lan} \circ N_{\UC} \circ  N_{\UC} \circ \iota
            \Rightarrow
            \mathsf{Lan} \circ N_{\UC} \circ \iota
      \end{equation}
      \begin{equation}
            id \overset{\simeq}{\Leftarrow} \mathsf{Lan} \circ \iota
            \Rightarrow
            \mathsf{Lan} \circ N_{\UC} \circ \iota.
      \end{equation}
      We will write $\Op_{G,\UC}(\V)$ for the category 
      $\mathsf{Alg}_{\mathbb{F}_{G,\UC}}(\mathsf{Sym}_{G,\UC}(\mathcal{V}))$ of \textit{genuine $\UC$-colored operads}.
\end{definition}



\subsection{Comparison with $\mathfrak C$-colored operads}

We have an inclusion
\begin{equation}
      \begin{tikzcd}[row sep = tiny]
            \UC\Omega \arrow[r, hookrightarrow]
            &
            \UC\Omega_G
            \\
            (U, \mathfrak c) \arrow[r, mapsto]
            &
            (G \cdot U, \mathfrak c: E_G(G \cdot U) \to \UC)
            &
            \mbox{\color{blue} alternatively, $\mathfrak c: \Phi(G \cdot E(U)) \to \UC$}
      \end{tikzcd}
\end{equation}
sending maps $(\phi, g)$ to $g \cdot \phi$.
{\color{blue} Alternatively, sending $(\phi,g)$ to $((h,t) \mapsto (hg^{-1}, \phi(t)))$.}

We have component maps in the opposite direction.
Given $(T = (T_i)_I, \mathfrak c) \in \UC\Omega_G$, we define
$\mathfrak c_i: E(T_i) \to \mathfrak C(G)$ by
\begin{equation}
      \mathfrak c_i(t) = q_t^{\**}(\mathfrak c (G t)),
\end{equation}
where
$q_t: G \to Gt$ sending $e \mapsto t$. 
% $t \in Gf$ (with $f$ minimal in the planar structure on $T$),
% $g_t\in G$ such that $t = g_t \cdot f$ (well-defined up to $G_f$),
% and $(q_t)_{\**}: G/e \to G/G_f$ the map $x \mapsto g G_f$.

% \begin{remark} % this is still true, but less important.
%       The coloring $\mathfrak c_i$ is \textit{almost} the composite
%       \begin{equation}
%             E(T_i) \to E_{G_i}(T_i) \xrightarrow{\simeq} E_G(T) \to \mathfrak C \to G \ltimes \mathfrak C(G/e)
%       \end{equation}
%       where $G_i$ is the stabilizer in $G$ of $T_i$, and
%       $E_{G_i}(T_i) \to E_G(T)$ is the canonical isomorphism sending
%       $e{G_i} \to Gf$
%       with $f \in Ge$ minimal.
%       However, this composite does not record the ``twisting'' action by the element $g_e$.
% \end{remark}


{\color{blue} % ------------------------------ CoLOR BLUE --------------------
  Alternatively, $\mathfrak c_i$ is the composite
  \begin{equation}
        E(T_i) \to E(T) \xrightarrow{\mathfrak c_{G/e}} \mathfrak C(G/e).
  \end{equation}
} % ------------------------------ COLOR BLUE ------------------------------

Then the map $B_I \to \UC\Omega$, $i \mapsto (T_i, \mathfrak c_i)$ is a functor
(as the inclusion {\color{blue} and $\mathfrak c_{G/e}$} are $G$-equivariant),
and we have
\begin{equation}
      \iota_{\**}Y(T, \mathfrak c) =
      \left(
            \prod_I Y(T_i, \mathfrak c_i)
      \right)^G.
\end{equation}



\begin{remark}[{cf. \cite[Rem 4.35]{BP17}}]
      Equivalently, the essential image of $\iota_{\**}$ are those sheaves $X \in \Sym_{G, \mathfrak C}(\V)$ such that
      the canonical map
      \begin{equation}
            X(C,\mathfrak c) \xrightarrow{\simeq} X(q^{\**}(C, \mathfrak c))^\Gamma
      \end{equation}
      is an isomorphism, where $q: G \to r(C)$ is the unique map preserving the minimal element, and
      $\Gamma \leq \mathsf{Aut}(q^{\**}(C,\mathfrak c))$ the subgroup preserving the quotient map $q^{\**}C \to C$
      under precomposition.
\end{remark}

\todo[inline]{come back:
  same retraction from free guys,
  same interpretation of essentially image of $\iota_{\**}$ and $\iota_!$,
  $\iota_{\**}\iota^{\**} \to id$ is still an iso on free guys
}


\cite[Prop 4.38]{BP17} follows as before.
In particular:
\begin{proposition}
      \label{FC_MONAD_PROP}
      For every coefficient system $\UC$,
      $\mathbb F^{\mathfrak C}$ is a monad, with category of algebras $\Op^{G,\mathfrak C}(\V)$.
      
      Moreover, $\iota_{!}$ embeds $\Op^{G, \UC}(\V)$ as the full subcategory of $\Op_{G, \UC}(\V)$
      of those objects whose underlying presheaf lives in the subcategory $\Sym^{G, \mathfrak C}(\V)$ of $\Sym_{G, \mathfrak C}(\V)$. 
\end{proposition}

\begin{remark}
      As in \cite{BP17}, the above inclusion allows for $\mathbb F^{G, \UC}$ and $\mathbb F_{G, \UC}$ to be considered in tandem.
      However, homtopically, we want to compare $\Op^{G, \UC}(\V)$ and $\Op_{G, \UC}(\V)$ via the more interesting inclusion
      $\iota_{\**}$.
\end{remark}




\subsection{Free extensions}

% come back

Category of labeled planar colored strings, as before.
As most of the construction/category theory is on the level of strings (or formally due to pullback constructions),
``coloring'' does not affect the proofs of \cite[Prop 5.30,5.41]{BP17}.

\begin{proposition}
      [{cf. \cite[Prop 5.48]{BP17}}]
      For any $(T, \mathfrak c) \in \UC\Omega_G^e$ there exists a unqiue $\mathsf{lr}_{\mathcal P}(T, \mathfrak c) \in \UC\hat\Omega_G^e$
      equipped with a unique planar (hence color-fixed) label map in $\UC\Omega_G^e$
      $\mathsf{lr}_{\mathcal P}(T, \mathfrak c) \to (T, \mathfrak c)$.      
      Furthermore, $\mathsf{lr}_{\mathcal P}$ extends to a right retraction
      $\mathsf{lr}_{\mathcal P}: \UC\Omega_G^e \to \UC\hat{\Omega}_G^e$.
\end{proposition}
\begin{proof}
      The underlying $G$-tree of $\mathsf{lr}_{\mathcal P}(T, \mathfrak c)$ is simply $\mathsf{lr}_{\mathcal P}(T)$, and since
      the edges of this tree are a $G$-subset of $E(T)$, the coloring descends uniquely.       
\end{proof}


\begin{proposition}
      [{cf. \cite[Prop 5.37]{BP17}}]
      $\Ran$ behaves well.
\end{proposition}
\begin{proof}
      The proof of this result in the Appendix of \cite{BP17} requires no assumptions on $\Sigma_G$.
      In particular, the proof follows when replacing $\Sigma_G$ with $\UC\Sigma_G$.
\end{proof}

\begin{corollary}
      $\mathcal P[u] = \mathcal P \hat\amalg_{\mathbb F_{\UC}X}\mathbb F_{\UC}Y
      \simeq
      \Lan_{\left(\UC\hat\Omega_G^e \to \UC\Sigma_G\right)^{op}}\tilde N^{(\mathcal P, X, Y)}$.
\end{corollary}


\begin{remark}
      For $T,S \in \UC\Omega_G$, any factorization $T \to T' \to S$ in $\Omega_G$ uniquely extends to one in $\UC\Omega_G$,
      as there is a canonical coloring on $T'$, namely $\Phi E(T') \to \Phi E(S) \to \UC$. 
\end{remark}

The above gives \cite[Lem 5.57]{BP17}.

\begin{proposition}
      [{cf. \cite[Prop 5.66]{BP17}}]
      For each $\UC$-profile $(C,\mathfrak c) \in \UC\Sigma_G$,
      we have the following pushout in $\V^{\mathrm{Aut}(C, \mathfrak c)}$
      \begin{equation}\label{FILTRATION_LAN_LEVEL}
            \begin{tikzcd}
                  \coprod\limits_{[T, \mathfrak d] \in \mathsf{Iso}
                    \left((C,\mathfrak c) \downarrow_{\mathsf{r}} \UC\Omega_G^a[k]\right)}
                  \left(
                        \bigotimes\limits_{v \in V_{G}^{ac}(T)}\P(T_v, \mathfrak d_v) \otimes
                        Q_T^{in}[u]
                  \right)
                  \mathop{\otimes}\limits_{\mathsf{Aut}(T, \mathfrak d)} \mathsf{Aut}(C, \mathfrak c)
                  \arrow[r] \arrow[d] &
                  \P_{k-1}(C, \mathfrak c) \arrow[d] 
                  \\
                  \coprod\limits_{[T, \mathfrak d] \in \mathsf{Iso}
                    \left((C, \mathfrak c) \downarrow_{\mathsf{r}} \UC\Omega_G^a[k]\right)}
                  \left(
                        \bigotimes\limits_{v \in V_{G}^{ac}(T)}\P(T_v, \mathfrak d_v) \otimes
                        \bigotimes\limits_{v \in V_{G}^{in}(T)} Y(T_v, \mathfrak d_v)
                  \right)
                  \underset{\mathsf{Aut}(T, \mathfrak d)}{\otimes} \mathsf{Aut}(C, \mathfrak c)
                  \arrow[r] &
                  \P_k(C, \mathfrak c)
            \end{tikzcd}
      \end{equation}
      where $ V_{G}^{ac}(T)$, $V_{G}^{in}(T)$ denote the active and inert vertices of the underlying $G$-tree $T \in \Omega_G^a[k]$,
      and $Q_T^{in}[u]$ is the domain 
      of the iterated pushout product
      \begin{equation}
            \underset{v \in V_G^{in}(T)}
            {\mathlarger{\mathlarger{\mathlarger{\square}}}}u(T_v, \mathfrak d_v)
            \colon
            Q_T^{in}[u] \to
            \bigotimes\limits_{v \in V_{G}^{in}(T)} Y(T_v, \mathfrak d_v).
      \end{equation}
\end{proposition}
\begin{proof}
      The same proof as before works.
\end{proof}

\subsection{Model structures}


\begin{proposition}
      Let $\V$ be a cocomplete model category will cellular fixed points.
      Now suppose $\mathcal D$ is a groupoid, and let $\F_d$ be a family of subgroups of $\mathsf{Aut}(d)$ for each $d \in \mathcal D$.
      Then the category of diagrams $\V^{\mathcal D}$ has an \textit{$\F$-model structure} $\V^\D_\F$, where
      a map $f: X \to Y$ is a
      weak equivalence (resp. fibration) iff $f(d): X(d) \to Y(d)$ is so in $\V^{\mathsf{Aut}(d)}_{\F_d}$ for each $d \in \mathcal D$.
\end{proposition}
\begin{proof}
      This is the model structure transferred along the adjunction
      \begin{equation}
            \begin{tikzcd}
                  \V^\D \leftrightarrows
                  \mathop{\prod}\limits_{d \in \D}\V^{\mathsf{Aut}(d)}_{\F_d}
            \end{tikzcd}
      \end{equation}
      which exists by a straightforward exercise adapting and combining the proofs of
      \cite[Thm 11.6.1]{Hir03} and \cite[Prop 2.6]{Ste16}.
\end{proof}

\begin{example}
      Let $\V$ be a cocomplete model category with cellular fixed points,
      $\mathfrak C$ a $G$-set, and $\F = \set{\F_n}$ a collection of families $\F_n$ of graph subgroups of $G \times \Sigma_n$.
      Then $\Sym^{G,\mathfrak C}(\V) = \V^{\UC\Sigma^{op}}$ has an $\F$-model structure
      \begin{equation}
            \Sym^{G,\mathfrak C}_\F(\V) = \prod_n \V^{B_{\mathfrak C^{\times n+1}}(G \times \Sigma_n)}_{\F_n},
      \end{equation}
      where
      $(\F_n)_{\xi}$ is all $\Gamma \in \F_n$ such that $\Gamma \leq \mathsf{Aut}(\xi)$.
      % $\V^{B_{\mathfrak C^{\times n+1}}(G \times \Sigma_n)}_{\F_n}$ has the model structure lifted from the adjunction to
      % $\prod_{(c_i)}\V^{\mathsf{Aut}(c_1,\dots,c_n; c_0)}_{\F_{(c_i)}}$ where
      % $\F_{(c_i)}$ is the set of all $\Gamma \in \F_n$ such that $\Gamma \leq \mathsf{Aut}(c_1,\dots, c_n;c_0)$.
\end{example}

On the other hand, let $\Sym_{\F, \mathfrak C}(\V)$ denote the category $\V^{\UC\Sigma_\F}$ with the projective model structure.

% \begin{lemma}
%       \label{EXMAIN_LEM}
%       Let $\V$ be {\color{red} GOOD}.
%       Let $\P \in \Sym_{G, \mathfrak C}(\V)$ be level genuine cofibrant, and
%       $u: X\ to Y$ in $\Sym_{G, \mathfrak C}(\V)$ be a level genuine cofibration.
%       Then, for each $(T, \mathfrak c) \in \UC \Omega_G^a [k]$, and writing $C = \mathsf{lr}(T)$, the map
%       \begin{equation}
%             \left(
%                   \bigotimes_{v \in V_G^{ac}(T)}\P(T_v, \mathfrak c) \otimes
%                   {\mathlarger{\mathlarger{\mathlarger{\square}}}}u(T_v, \mathfrak d_v)
%             \right)
%             \mathop{\otimes}\limits_{\Aut(T, \mathfrak c)}\Aut(C, \mathfrak c)
%       \end{equation}
%       is a genuine cofibration in $\V^{\Aut(C, \mathfrak c)}_{\mbox{gen}}$,
%       which is trivial if $k \geq 1$ and $u$ is trivial.
% \end{lemma}
% \begin{proof}
%       This follows from \cite[Prop 6.24]{BP17} analogously as \cite[Lemma 5.72]{BP17} did.
% \end{proof}

\begin{theorem}
      \label{THM1_C}
      Let $(\V,\otimes)$ denote {\color{red} THINGS}.
      For each $G$-set $\mathfrak C$,
      there exist model structures on $\Op^{G, \mathfrak C}(\V)$ such that
      $\O \to \O'$ is a weak equivalence (resp. fibration) if the maps
      \begin{equation}
            \O(\xi)^\Gamma \to \O'(\xi)^\Gamma
      \end{equation}
      are weak equivalences (resp. fibrations) in $\V$ for all
      $\mathfrak C$-signatures $\xi$ and
      graph subgroups $\Gamma \leq \Stab(\xi)$.

      {\color{red} More generally \ldots}
\end{theorem}


\begin{theorem}
      \label{THM2_C}
      Let $(\V,\otimes)$ denote {\color{red} THINGS}.
      For each $G$-set $\mathfrak C$,
      the projective model structure on $\Op_{G, \mathfrak C}(\V)$ exists.
      Explicitly, a map
      $\P \to \P'$ is a weak equivalence (resp. fibration) if the maps
      \begin{equation}
            \P(C,\mathfrak c) \to \P'(C, \mathfrak c)
      \end{equation}
      are weak equivalences (resp. fibrations) in $\V$ for all
      $(C,\mathfrak c) \in \UC\Sigma_G$.

      {\color{red} More generally \ldots}
\end{theorem}

\begin{proof}[{proof of Theorems \ref{THM1_C} and \ref{THM2_C}}]
      Follow exactly as in \cite{BP17},
      replacing the use of \cite[{(5.67) and Lemma 5.72}]{BP17} with
      the colored analogues, whose proofs are identical to the single-colored cases.
      In particular, the model structures are those lifted across the adjunctions below.
      \begin{equation}
            \label{MAINPFADJ EQ}
            \begin{tikzcd}[column sep =5em]
                  \mathop{\prod}\limits_{(C, \mathfrak c) \in \UC\Sigma_\F}
                  \mathcal{V}^{\mathsf{Aut}(C, \mathfrak c)}_{\text{proj}}
                  \ar[shift left=1.5]{r}
                  &
                  \mathsf{Sym}_{\F, \mathfrak C}(\mathcal{V}) 
                  \arrow[l, shift left=1.5, "\left(\text{ev}_{(C, \mathfrak c)}(\minus)\right)"] 
                  \arrow[r, shift left=1.5,swap,"\mathbb{F}_{G, \mathfrak C}"']
                  &
                  \mathsf{Op}_{\F, \mathfrak C}(\mathcal{V})
                  \ar[shift left=1.5]{l}
                  \\ % ---------- NEXT ROW ----------
                  \mathop{\prod}\limits_{\xi \in \UC\Sigma}
                  \mathcal{V}^{\mathsf{Aut}(\xi)}_{\F_\xi}
                  \ar[shift left=1.5]{r}
                  &
                  \mathsf{Sym}^{G, \mathfrak C}_\F(\mathcal{V}) 
                  \arrow[l, shift left=1.5, "\left(\text{ev}_{\xi}(\minus)\right)"] 
                  \arrow[r, shift left=1.5,swap,"\mathbb{F}^{G, \mathfrak C}"']
                  &
                  \mathsf{Op}^{G, \mathfrak C}_\F(\mathcal{V})
                  \ar[shift left=1.5]{l}
            \end{tikzcd}
      \end{equation}      
\end{proof}

The following is immediate.
\begin{corollary}
      \label{COLOR_CHANGE_Q_COR}
      The change of color adjunctions from \eqref{COLOR_CHANGE_EQ},
      and the forgetful functor $j^{\**}$ from \eqref{JSTAR_CAT_EQ}
      form Quillen adjunctions.
\end{corollary}
% \begin{proof}
%       For any $F: \mathfrak C \to \mathfrak C'$, $F_{\**}$ clearly preserves (trivial) fibrations.
% \end{proof}

\begin{remark}
      In particular, this implies that $j^{\**}$ commutes with fibrant replacement.
\end{remark}


\subsection{Assembly}

Goal: $\Op^{G,\mathfrak C}(\V)$ assemble into $\Op(\V)^G$.







%%%%%%% ------------------------------------------------------------
\newpage

\subsection{Aside: Comparisons and cofibrancy}

\todo[inline]{come back}


\begin{definition}
      Given a coefficient system $\UC$, we define the category $\UC O_{\F_n}$ to be the Grothendieck construction of the functor
      \begin{equation}
            \begin{tikzcd}[row sep = tiny]
                  O_{\F_n}^{op} \arrow[r]
                  &
                  \Set
                  \\
                  G \times \Sigma_n / \Gamma \arrow[r, mapsto]
                  &
                  \Hom_{\Set^{\mathsf O_G^{op}}}(\Phi(G \cdot_\Gamma \underline{n+1}), \mathfrak C)
            \end{tikzcd}
      \end{equation}
      where, for $\phi = \phi_\Gamma: H \to \Sigma_n$ the unique homomorphism such that $\Gamma = \Gamma(\phi)$,
      the (right) action of $\Gamma$ on $G \cdot \underline{n+1}$ is given by
      $(g,i).(h,\phi(h)) = (gh, \phi(h)^{-1}(i))$.
\end{definition}

Equivalently, objects are given by transitive sets $G \times \Sigma_n / \Gamma$
equipped with a $\UC$-coloring on their associated $G$-corollas $C_\Gamma$.

\begin{remark}
      We record the hom-sets in $\UC O_{\F_n}^{op}$ for later use. We have that
      \mbox{$\UC O_{\F_n}^{op}((G \times \Sigma_n/\Gamma, \mathfrak c), (G \times \Sigma_n / \Lambda, \mathfrak d))$}
      is given by those cosets $(x,\tau)\Gamma$ in $G \times \Sigma_n / \Gamma$ such that
      \begin{enumerate}
      \item $(x^{-1},\tau^{-1})\Lambda(x,\tau) \subseteq \Gamma$
            (equivalently, $(x,\tau)\Gamma \in (G \times \Sigma_n/\Lambda)^\Gamma$).
      \item $\mathfrak d(G/H, [g,i]) = \mathfrak c(G/H, [g x, \tau^{-1}(i)]$
            for all $(G/H, [g,i]) \in \Phi(G \cdot_{\Lambda} \underline{n+1}$.
      \end{enumerate}
\end{remark}

\begin{notation}
      For $\UC$ a coefficient system, we will write $B_{\UC^{\times n+1}}$ to mean
      $B_{\UC(G/e)^{\times n+1}}(G \times \Sigma_n)$.
\end{notation}

We have a natural map
\begin{equation}
      \begin{tikzcd}[row sep = tiny]
            B_{\UC^{\times n+1}} \arrow[r]
            &
            \UC O_{\F_n}
            \\
            (c_1,\dots,c_n;c_0) \arrow[r, mapsto]
            &
            (G \times \Sigma_n/e, (g,i) \mapsto g.c_i).
      \end{tikzcd}
\end{equation}

This induces a pair of adjunction
\begin{equation}
      \begin{tikzcd}[column sep =5em]
            \V^{\UC O_{\F_n}^{op}} \arrow[r, "\iota^{\**}"']
            &
            \V^{B_{\UC^{\times n+1}}^{op}}
            \arrow[l, bend right, "\iota_{!}"']
            \arrow[l, bend left, "\iota_{\**}"]
      \end{tikzcd}
\end{equation}

We have explicit formula for $\iota_{\**}$ and $\iota_!$, given by
\begin{align*}
  (\iota_{\**} X)(G \times \Sigma_n/\Gamma, \mathfrak c) &= X(c_1,\dots, c_n;c_0)^\Gamma\\
  (\iota_! X)(G \times \Sigma_n/\Gamma, \mathfrak c) &=
                                                       {\begin{cases}
                                                               X(c_1,\dots,c_n;c_0) \qquad \qquad & G \cdot_\Gamma \underline{n+1} \mbox{ is $G$-free} \\
                                                               \varnothing & \mbox{otherwise,}
                                                       \end{cases}}
\end{align*}
with  $c_i := \mathfrak c(G/e, [e, i])$.

We equip $\V^{\UC O_{\F_n}^{op}}$ with the projective model structure.
\begin{lemma}
      This is a Quillen adjunction.
\end{lemma}
\begin{proof}
      One can check that $\epsilon$ is a natural isomorphsim
      \begin{equation}
            X(c_1,\dots,c_n; c_0)^{\set{e}} \simeq X(c_1,\dots,c_n; c_0).
      \end{equation}
      Moreover, 
      \begin{equation}
            \iota_{\**}\iota^{\**} Y (G \times \Sigma_n / \Gamma, \mathfrak c)
            =
            Y(G \times \Sigma_n / e, (g,i) \mapsto g.\mathfrak c(G/e, [e,i]))^\Gamma,
      \end{equation}
      (where the coloring on the right hand side is given by the composite
      $\Phi(G \cdot \underline{n+1}) \xrightarrow{q} \Phi(G \cdot_\Gamma \underline{n+1})$)
      where $q: G \times \Sigma_n \to G \times \Sigma_n / \Gamma$
      preserves the minimal element\todo{sends $[e]$ to $[e]$},
      \todo[inline]{or just $Y(q^{\**}(-))$, with $\UC O_{\F_n}$ a fibration over $\mathsf O_G$}
      and $\eta$ is the natural map.      
      
      We have that $\iota_{\**}$ is right Quillen by construction of the model structures.

      COME BACK
\end{proof}

We record that $\iota_{\**}$ again fully-faithful since $\epsilon$ is an iso.

\begin{proposition}
      The above adjunction is an equivalence if and only if $\mathfrak C$ is a $G$-set.
\end{proposition}
\begin{proof}
      Following the proof in \cite{Ste16}, it suffices to compare the following evaluations.
      \begin{align*}
        (\UC O_{\F_n}^{op})((G \times \Sigma_n/\Gamma, \mathfrak c), (G \times \Sigma_n / \Lambda, \mathfrak d) \cdot A
        \\
        \iota_{\**}\iota^{\**}(\UC O_{\F_n}^{op})((G \times \Sigma_n/\Gamma, \mathfrak c), - ) \cdot A)(G \times \Sigma_n / \Lambda, \mathfrak d)        
      \end{align*}
      The elements in the top are those $(x,\tau)$ such that
      \mbox{$\mathfrak c(G/H, [g x, \tau^{-1}(i)]) = \mathfrak d(G/H, [g,i])$}
      while the bottom are those $(x,\tau)$ with the weaker constraint
      \mbox{$\mathfrak c(G/e, [g x, \tau^{-1}(i)]) = \mathfrak d(G/e, [g,i])$}.
      Unless $\UC$ is a $G$-set, we don't even know if we have a map of colorings on levels other than $G/e$,
      let alone whether they satisfy this particular condition.
      \todo[inline]{this is not a complete/full/reasonable argument for the ``if'' direction}.
\end{proof}





% ------------------------------------------------------------
\newpage

\section{Model structure for all colors} 
\renewcommand{\C}{\mathfrak C}

\textbf{Fix a cofibrantly generated model category $\V$.}
We generalize and synthesize \cite{BM13}, \cite{Cav14}, and \cite{CM13b} to assemble the collection of
``color-fixed'' ($\F$)-model structures into a single model structure on $\Op^G(\V)$.


% \begin{definition}
%       We say $\V$ \textit{admits (semi)-transfer for categories/operads} if for any set $\mathfrak C$,
%       the categories $\Cat^{\mathfrak C}(\V)$, $\Op^{\mathfrak C}(\V)$
%       may be equipped with the canonical (semi)-model structure.
% \end{definition}

% \begin{remark}
%       If $\V$ satisfies the \textit{monoid axiom} of \cite{SS00}
%       or has a monoidal fibrant replacement functor and a comonoidal Hopf interval object \cite{BM03},
%       then $\V$ admits transfers for categories and operads by \cite{Mur11,Mur14} and \cite{BM03}, respectively.
% \end{remark}

As in the previous section, we fix an indexing system $\F = \set{\F_n}$ of families $\F_n$ of graph subgroups of $G \times \Sigma_n$,
with the maximal family denoted $\mathrm{Gr} = \set{\mathrm{Gr}_n}$.
We say an indexing family $\F$ \textit{has units} if
$\F_1$ contains all graph subgroups of the form $H \leq G \times \Sigma_1$.
In particular, \cite[Remark 4.50]{BP17}) implies that any \textit{weak indexing system} has units.

\begin{definition}
      We say $\V$ has \textit{cellular fixed points} if
      \todo[inline]{come back}
\end{definition}

\begin{remark}
      \label{LEVEL_COF_REM}
      An immediate consequence of cellularity is that for any genuine cofibration $f \in \V^G_{gen}$,
      $f^H$ is a cofibration in $\V$ for all $H \leq G$.
\end{remark}

\begin{definition}
      Given $\V$ with cellular fixed points, we say
      $\V$ \textit{admits (semi)-transfers for $G$-categories/operads} if
      for any $G$-set $\mathfrak C$ and indexing collection $\F$,
      the categories $\Cat^{G, \mathfrak C}(\V)$, $\Op^{G, \mathfrak C}(\V)$
      may be equipped with the canonical $\F$-(semi)-model structure.
\end{definition}

\begin{example}
      By Theorem \ref{THM1_C}, $(\sSet, \times)$ and $(\sSet_{\**}, \wedge)$ admit transfers for $G$-operads,
      and any $(\V, \otimes)$ satisfying the hypotheses of said theorem admits semi-transfers for $G$-operads.
\end{example}

\begin{example}
      \cite[Theorem 3.1]{GW17} shows that $(\Top, \times)$ admits transfers for $G$-operads when $\mathfrak C = \set{*}$.
\end{example}

\textbf{\color{OliveGreen} In this section, further assume that $\V$
  has cellular fixed points and admits (semi)-transfer for $G$-categories and $G$-operads}.

\begin{remark}
      \label{JSTAR_REM}        
      Extending \eqref{JSTAR_CAT_EQ}, we have another inclusion-forgetful adjunction,      
      \begin{equation}
            \begin{tikzcd}
                  \mathsf{Op}^G(\V) \arrow[d, "(-)^H"']
                  \arrow[r, shift right, "j^*"']
                  &
                  \mathsf{Cat}^G(\V) \arrow[l, shift right, swap, "j_!"] \arrow[d, "(-)^H"]
                  \\
                  \Op(\V) \arrow[r, shift right, "j^*"']
                  &
                  \Cat(\V) \arrow[l, shift right, swap, "j_!"]
            \end{tikzcd}
      \end{equation}
      and again $j^{\**}$ commutes with $H$-fixed points and fibrant replacement.
\end{remark}

\begin{definition}
      We highlight three particular $\V$-categories.
      We let $1_\V$ and $\varnothing$ denote the unit object and initial object of $\V$, respectively.
      \begin{enumerate}[label = (\roman*)]
      \item Let $\I$ be the $\V$-category that detects isomorphisms: it has objects $\set{0,1}$,
            with $\I(0,0) = \I(0,1) = \I(1,0) = \I(1,1) = 1_\V$.
      \item Let $\mathbb A$ be the $\V$-category that detects arrows: it has objects $\set{0,1}$,
            with $\mathbb A(0,0) = \mathbb A(0,1) = \mathbb A(1,1) = 1_\V$, and $\mathbb A(1,0) = \varnothing$.
      \item Let $\1$ be the $\V$-category that detects objects: it has a single object $\set{\**}$, with $\1(\**, \**) = 1_\V$.
      \end{enumerate}
\end{definition}

\begin{definition}
      A {\em $\V$-interval} is a cofibrant object in $\Cat^{\set{0,1}}(\V)$ (with the transfered model structure)
      equipped with a weak equivalence $\J \to \I_f$.
      A set $\mathcal{G}$ of $\V$-intervals is {\em generating} if all $\V$-intervals $\J$ can be obtained
      as a retract of a trivial extension of an element in $\mathcal{G}$ in $\Cat^{\set{0,1}}(\V)$:
      \begin{equation}
            \begin{tikzcd}
                  \mathbb{G} \arrow[r,rightarrowtail, "\sim"]
                  &
                  \mathbb{K} \arrow[r,yshift=-.3em, "r"']
                  &
                  \mathbb{J} \arrow[l,yshift=.3em, "i"']
            \end{tikzcd}
      \end{equation}
\end{definition}

\begin{definition}
      \label{PL_ES_DEFN}
      We say a functor $F: \mathcal C \to \mathcal D$ in $\Cat(\V)$ is
      \begin{enumerate}[label = (\roman*)]
      \item \textit{path-lifting}
            if it has the right lifting property against all maps of the form
            $\1 \to \J$
            where $\J$ is a $\V$-interval.
      \item \textit{essentially surjective}
            if for any object $d: \1 \to \mathcal D$,
            there is an object $c: \1 \to \mathcal C$
            and a map $\J \to \mathcal D$ out of a $\V$-interval fitting in to the commuting diagram below.
            \begin{equation}
                  \begin{tikzcd}
                        \1 \arrow[rr, dashed, "c"] \arrow[dr, "i_0"]
                        &&
                        \mathcal C \arrow[dd, "F"]
                        \\
                        &
                        \J \arrow[dr, dashed]
                        \\
                        \1 \arrow[ur, " i_1"] \arrow[rr,"b"]
                        &&
                        \mathcal D
                  \end{tikzcd}
            \end{equation}
      \end{enumerate}
\end{definition}

\begin{definition}
      \label{MODEL_DEFN}
      We call a map $F: \O \to \P$ in $\mathsf{Op}^G(\V)$
      \begin{itemize}
      \item a {\em local $\F$-fibration} (resp. {\em local weak $\F$-equivalence}) if
            $F(\xi): \O(\xi)\to \P(F(\xi))$
            is a fibration (resp. weak equivalence) in $\V^{\Stab(\xi)}_{\F_\xi}$ for all $\xi\in \C(\O)^{\times n+1}$ and all $n$.
      \item a {\em local trivial $\F$-fibration} if both a local $\F$-fibration and a local weak $\F$-equivalence.
      \item {\em essentially surjective} (resp. {\em path lifting}) if $j^*F^H$ is essentially surjective (resp. path lifting) in $\Cat(\V)$ for all $H\leq G$.
      \item a {\em $\F$-fibration} if both path-lifting and a local $\F$-fibration.
      \item a {\em weak $\F$-equivalence} if both essentially surjective and a local weak $\F$-equivalence.
      % \item a \textit{DK-$\F$-equivalence} if a local weak $\F$-equivalence such that
      %       $\pi_0 j^{\**}F^H$ (cf. Definition \ref{HTPY_DEFN}) is essentially surjective.
      \item a \textit{(trivial) $\F$-cofibration} if it has the left lifting property against all trivial $\F$-fibrations (resp. $\F$-fibrations).
      \end{itemize}
\end{definition}

% \begin{remark}
%       If $\V$ has diagonals, then $F \in \Op^G(\V)$ is a $DK$-$\F$-equivalence iff
%       $F$ is a local weak $\F$-equivalence such that 
%       the associated map of \textit{$\F$-genuine equivariant operads} under the composite
%       \begin{equation}
%             \Op^G(\V) \to \Op_\F(\V) \xrightarrow{\pi_0} \Op_\F(\Set) 
%       \end{equation}
%       is an equivalence.
% \end{remark}

% \begin{remark}
%       Trivial $\F$-fibrations are precisely local $\F$-fibrations which are surjective on objects.
%       Thus $\F$-cofibrations are $f: \O \to \P$ such that
%       each $\mathfrak C(\O)^H \to \mathfrak C(\P)^H$ is injective and
%       $f_!\O \to \P$ is an $\F$-cofibration in $\Op^{G, \mathfrak C(\P)}(\V)$.
% \end{remark}


\begin{theorem}
      \label{MODEL_THM}
      Fix an indexing family $\F = \set{\F_n}$ with units.
      Let $(\V, \otimes)$, denote either $(\sSet, \times)$ or $(\sSet_{\**}, \wedge)$.
      Then there exists a cofibrantly generated model structure on the category $\Op^G(\V)$,
      denoted $\Op^G_\F(\V)$, with
      weak $\F$-equivalences, $\F$-fibrations, and $\F$-cofibrations defines as in Definition \ref{MODEL_DEFN}.
           
      Moreover, analogous semi-model category structures $\Op^G_\F(\V)$ exist
      provided that $(\V, \otimes)$:
      \begin{enumerate}[label = (\roman*)]\itemsep-4pt
      \item is a cofibrantly generated model category,
      \item is a closed monoidal model category with cofibrant unit
            \footnote{Cofibrant unit also needed for \ref{J-CELL_PROP}.},
      \item has cellular fixed-point functors,
      \item \label{CSPP_LBL} has cofibrant symmetric pushout powers,
            \footnote{Also needed for Lemma \ref{CAV_4.14_PROP2}.},            
            % --------------------
      \item \label{RP_LBL} is right proper
            \footnote{Needed for \ref{RIGHTPROPER_LEM}, \ref{CAV_4.15_PROP}.},
      \item \label{GENSET_LBL} has a set $\mathbb{G}$ of generating $\V$-intervals
            \footnote{Needed so we have a \textit{set} of generating trivial cofibrations},
      % \item \label{TCWE_LBL} has that the class of genuine weak equivalences in $\mathsf{Op}^G(\V)$ is closed under transfinite compositions
      %       \footnote{Needed for \ref{J-CELL_PROP}}.
      \end{enumerate}
\end{theorem}
\begin{proof}
      In the first case, we have that the model category $\Op^{G,\mathfrak C}_\F(\sSet)$ exists
      for any $G$-set $\mathfrak C$ and indexing family $\F$ by Theorem \ref{THM1_C},
      while in the second case, conditions $(i)$ -- \ref{CSPP_LBL} are sufficient to construct the
      semi-model category $\Op^{G, \mathfrak C}_\F(\V)$ from said theorem.
      
      After this difference, the proofs of the two cases are identical, as
      \ref{RP_LBL} $\sSet$ is right proper
      by Lemma \ref{INTER_LEM} and e.g. \cite[Prop 2.1.5]{Cis06} or \cite[Lemma 1.12]{BM13}, and
      \ref{GENSET_LBL} $\sSet$ has a generating set of intervals
      by e.g. \cite[Lemma 1.12]{BM13}.
      % \ref{TCWE_LBL} the class of genuine weak equivalences in $\mathsf{Op}^G(\sSet)$ is closed under transfinite compositions
      % by an argument analogous to \cite[Lemma 1.24]{CM13b}.
      % Now, we note that condition \ref{TCWE_LBL} proves the analogous statement for any $\F$,
      % since the transfinite composite of local $\F$-equivalences is a local $\F$-equivalence.

      Since $\mathsf{Op}^G(\V)$ is complete and cocomplete, it thus suffices to prove (following \cite{Hov98} Theorem 2.1.19) that:
      \begin{enumerate}[label = (\arabic*)]
      \item the class of weak $\F$-equivalences has the 2-out-of-3 property and is closed under retracts;
      \item the domains of $I_{\F}$ (resp. $J_{\F}$) are small relative to $I_{\F}$-cell (resp. $J_{\F}$-cell);
      \item $I_{\F}$-inj $= W\cap J_{\F}$-inj;
      \item $J_{\F}$-cell $\subseteq W\cap I_{\F}$-cof.
      \end{enumerate}
      (1) follows from Proposition \ref{CAV_4.15_PROP} and the fact that if $L$ is a retract of $F$, $L^H$ is a retract of $F^H$.
      (2) follows since colimits in $\mathsf{Op}^G(\V)$ are created in $\Op(\V)$, and it holds non-equivariantly.
      (3) follows from Lemma \ref{CAV_4.8}.
      (4) follows from Lemma \ref{POINT_4_LEMMA} and Proposition \ref{J-CELL_PROP}.
\end{proof}

% \begin{remark}
%       If we could show via some other method that $(\V, \otimes)$ satisfied actual transfer for $G$-operads, then
%       conditions \ref{CSPP_LBL} -- \ref{TCWE_LBL} would imply that the $\F$-model structure existed on $\Op^G(\V)$. 
% \end{remark}

\begin{remark}
      This recovers the main results of \cite{BM13, Cav14} for $G = \set{e}$. 
\end{remark}



We spend the rest of this section proving the results needed in the proof of Theorem \ref{MODEL_THM},
beginning with a description of the sets of generating (trivial) cofibrations.

Fix a graph subgroup $\Gamma \in \F_n$ of $G \times \Sigma_n$, and $X \in \V^\Gamma$.
We now construct the ``free operad with stabilizer $\Gamma$ generated by $X$''.
Consider the $G$-set of colors $\mathfrak C_\Gamma := G \cdot_\Gamma \underline{n+1}$,
and let $\xi_0$ denote the signature $([e,1],[e,2],\dots,[e,n];[e,0])$.
We define the $(G,\mathfrak C_\Gamma)$-symmetric sequence
\begin{equation}
      C_\Gamma[X](\xi) =
      \begin{cases}
            (g,\sigma)^{\**} X \qquad \qquad & \xi = (g,\sigma).\xi_0
            \\
            \varnothing & \mbox{otherwise,}
      \end{cases}
\end{equation}
where $g \in G$ and $\sigma \in \Sigma_n$ are chosen to be the \textit{minimal} elements in those groups with this property,
and $\varnothing$ is the initial object in $\V$.

Let $\mathbb F_\Gamma[X]$ denote free operad $\mathbb F^{\mathfrak C_\Gamma} C_\Gamma[X]$.
It is straightforward that the operad $\mathbb F_\Gamma[X]$ has the universal property
\begin{equation}
      \Hom_{\Op^G(\V)}(\mathbb F_\Gamma[X], \O) = \mathop\prod\limits_{\xi \in (\mathfrak C(\O)^{\times n+1})^\Gamma}\Hom_{\V^\Gamma}(X, \O(\xi)).
\end{equation}

Define $I_{\F,loc}$ and $J_{\F, loc}$ to be the sets
\begin{align*}
  \set{\mathbb F_\Gamma[\Gamma/\Gamma \cdot i]}, \qquad \qquad \set{\mathbb F_\Gamma[\Gamma/\Gamma \cdot j]},
\end{align*}
where $\Gamma$ runs over all graph subgroups of $G \times \Sigma_n$ in $\F_n$,
and $i$ (resp. $j$) runs over all generating (trivial) cofibrations in $\V$.

The universal property makes the following immediate.
\begin{corollary}[{cf. \cite[\S 4.2]{Cav14}, \cite[1.16]{CM13b}}]
      $\O \to \O'$ is a local (trivial) $\F$-fibration iff
      $\O \to \O'$ has the right lifting property against $J_{\F, loc}$ (resp. $I_{\F, loc}$).
\end{corollary}

Now, define
\begin{equation}
      I_{\F}:= I_{\F, loc} \mathbin{\cup} \set{\varnothing \to G/H \cdot \1}_{H\leq G},
      \qquad \qquad
      J_{\F} := J_{\F, loc} \mathbin{\cup} \set{G/H \cdot (\1 \to \J)}_{H\leq G,\ \J\in\mathbb{G}}
\end{equation}
where $\1$ defined as in Definition \ref{PL_ES_DEFN}, and $\mathbb{G}$ is a generating set of $\V$-intervals. 

\begin{lemma}
      [{cf. \cite[4.8]{Cav14}, \cite[2.3]{BM13}, \cite[1.18]{CM13b}}]
      \label{CAV_4.8}
      Suppose $\V$ has a generating set of intervals.
      A map $F$ in $\mathsf{Op}^G(\V)$ is a trivial $\F$-fibration
      iff
      $F$ is a local trivial $\F$-fibration such that $F^H$ is surjective on $H$-fixed colors for all $H\leq G$
      iff
      $F$ has the right lifting property against $I_{\F}$.
\end{lemma}
\begin{proof}
      The second iff is immediate by the construction of $I_{\F}$.
      For the first, we have by definition that
      $F$ is a trivial $\F$-fibration
      iff
      it is a local trivial $\F$-fibration such that $j^*F^H$ is path-lifting and essentially surjective for all $H\leq G$.
      \cite[2.4]{BM13} completes the proof. 
      % Moreover, right lifting against $I_{\F, loc}$ is identical to being a local trivial $\F$-fibration, while
      % lifting against $\varnothing \to G/H\otimes \1$ precisely say that $F^H$ is surjective on colors;
      % combining these observations yields the result.
\end{proof}

\begin{lemma}
      [{cf. \cite[1.20]{CM13b}, \cite[\S 4.3]{Cav14}}]
      $F$ has right lifting against $J_{\F}$ iff $F$ is an $\F$-fibration.
\end{lemma}
\begin{proof}
      Again, lifting against $J_{\F, loc}$ is identical to being a local $\F$-fibration, while lifting against $G/H \cdot (\1 \to \J)$
      is equivalent to $F^H$ lifting against $\1 \to \J$.
\end{proof}

\begin{lemma}
      [{cf. \cite[1.19]{CM13b}}]
      \label{POINT_4_LEMMA}
      $J_{\F}\mbox{-cof} \subseteq I_{\F}\mbox{-cof}$; that is, trivial cofibrations are cofibrations.
\end{lemma}
\begin{proof}
      % It suffices to show that if $F$ has (right) lifting against $I_\F$, it has lifting aginst $J_{\F}$.
      Clearly a local trivial $\F$-fibration is a local $\F$-fibration.
      On the other hand, by locality and \eqref{COLOR_SQ_EQ},
      any cofibration in $\mathsf{Op}^{G, \mathfrak C}(\V)$ for any $G$-set $\C$
      is a cofibration when considered in $\mathsf{Op}^G(\V)$.
      Thus, since $G/H \cdot (\1 \to \1 \amalg \1)$ is a pushout of $G/H \cdot(\varnothing \to \1)$
      and hence is in $I_{\F}\mbox{-cof}$, the composite
      \begin{equation}
            \begin{tikzcd}
                  G/H \cdot \1 \arrow[r, rightarrowtail]
                  &
                  G/H \cdot (\1 \amalg \1) \arrow[r, rightarrowtail]
                  &
                  G/H \cdot \J 
            \end{tikzcd}
      \end{equation}
      is in $I_{\F}\mbox{-cof}$.
      Thus $J_\F \subseteq I_\F\mbox{-cof}$, implying the result.
\end{proof}

\subsection{Trivial cofibrations are weak equivalences}

Similar to many cases in the literature, the two most difficult steps in the proof of Theorem \ref{MODEL_THM} are showing that
$J_\F$-cells are weak equivalences, and that weak equivalences satisfy 2-out-of-3.
The next two subsectiosn deal with these issues, in that order.

\begin{lemma}
      \label{TRANSCOMP_ES_LEM}
      Suppose $\V$ is a monoidal model category.
      The transfinite composition of essentially surjective maps is essentially surjective.
\end{lemma}
\begin{proof}
      Since taking fixed points commutes with filtered colimits, they commute with transfinite composition,
      and hence by \cite[4.17]{Cav14}, we are done.
\end{proof}

\begin{lemma}
      \label{COF_TC_LEM}
      Suppose $\V$ admits (semi)-transfer for operads.
      Then local $\mathrm{Gr}$-cofibrations and  trivial $\mathrm{Gr}$-cofibrations (with cofibrant source) are closed under transfinite composition.
\end{lemma}
\begin{proof}
      Explicitly, suppose we have a chain of maps
      \begin{equation}
            \O_0 \xrightarrow{f_0} \O_1 \xrightarrow{f_1} \dots \xrightarrow{f_\alpha} \O_{\alpha+1} \xrightarrow{f_{\alpha+1}} \dots
      \end{equation}
      in $\Op^G(\V)$, and let $F_\alpha$ denote the compoite $\O_0 \to \O_\alpha$.
      Then we claim that
      if each map of the form on the left in \eqref{COF_TC_EQ}
      is a (trivial) $\mathrm{Gr}$-cofibrations in $\V^{\Stab(\theta)}_{\F_\theta}$
      for all $\mathfrak C(\O_\alpha)$-signatures $\theta$, then
      the composite map on the right in \eqref{COF_TC_EQ}
      is a (trivial) $\mathrm{Gr}$-cofibration in $\V^{\Stab(\ksi)}_{\F_\ksi}$
      for all $\mathfrak C(\O)$-signatures $\ksi$.     
       \begin{equation}
            \label{COF_TC_EQ}
            \O_\alpha(\theta) \to \O_{\alpha+1}(F_\alpha(\theta)),
            \qquad  \qquad
            \O(\ksi) \to \O_\alpha(F_\alpha(\ksi))
      \end{equation}
      %
      Indeed, if $\theta = f_{\alpha-1}(\ksi)$,
      $\Stab(\ksi) \leq \Stab(\theta)$,
      and the restriction map $\V^{\Stab(\theta)}_{\mathrm{Gr}_\theta} \to \V^{\Stab(\ksi)}_{\mathrm{Gr}_\ksi}$
      preserves (trivial) cofibrations (cf. \cite[Prop 6.6]{BP17},
      and thus the $\Stab(\theta)$-genuine (trivial) cofibration $f_\alpha(\theta)$ 
      restricts to a $\Stab(\ksi)$-genuine (trivial) cofibration $f_\alpha(f_{\alpha-1}(\ksi))$.
      %
      Thus, we have a string of (trivial) $\F_\ksi$-cofibrations in $\V^{\Stab(\ksi)}_{\mathrm{Gr}_\ksi}$
      \begin{equation}
            \O_0(\ksi) \rightarrowtail \O_1(F_1(\ksi)) \rightarrowtail \O_2(F_2(\ksi)) \rightarrowtail \dots
            \rightarrowtail \O_\alpha(F_\alpha(\ksi)) \rightarrowtail \dots
      \end{equation}
      As cofibrations and trivial cofibrations in $\V^{\Stab(\ksi)}_{\mathrm{Gr}_\ksi}$ (with cofibrant source if semi-admissible)
      are closed under transfinite composition, the result is proven.
\end{proof}

The following follows from the cellular description of cofibrations in $\Op^{G, \mathfrak C}(\V)$.
\begin{lemma}
      \label{LOCAL_COF_LEM}
      Suppose $\V$ admits transfer for operads, and suppose
      $f: \O \to \P$ in $\Op^{G, \mathfrak C}(\V)$ is a (trivial) $\mathrm{Gr}$-cofibration.
      Then it is a local (trivial) $\mathrm{Gr}$-cofibration.

      If $\V$ only admits semi-transfer, the result holds when $\O$ is level $\mathrm{Gr}$-cofibrant itself.
\end{lemma}

\begin{proposition}
      [{c.f. \cite[4.20]{Cav14}}]
      \label{J-CELL_PROP}
      Suppose the unit $1_\V \in \V$ is cofibrant.
      % and weak $\F$-equivalences are closed under transfinite compositions.
      If $\V$ admits (semi)-transfer for operads, then relative $J_{\F}$-cells (with cofibrant source) are weak equivalences.
\end{proposition}
\begin{proof}
      We begin with the case that $\V$ admits transfer for operads.
           
      Now, as essentially surjective maps are closed under transfinite composition by Lemma \ref{TRANSCOMP_ES_LEM},
      it suffices by Lemma \ref{COF_TC_LEM} to prove that the pushout of a map $j \in J_{\F}$ is a local weak $\F$-equivalence
      which is a $\mathrm{Gr}$-cofibration (which in particular follows from being a local trivial $\F$-cofibration).

      Firstly, if $j = \mathbb F_\Gamma[\Gamma/\Gamma \cdot i] \in J_{\F, loc}$, 
      we are considering the pushout of a trivial cofibration in $\mathsf{Op}^{G,\mathfrak C(\Q)}(\V)$ (cf. \eqref{COLOR_SQ_EQ}).
      By the existance of the transferred model structure, this is again a trivial cofibration.
      Moreover, it is a local weak $\F$-equivalence which is the identity on colors,
      hence a trivial $\F$-cofibration,
      and hence a local trivial $\F$-cofibration by Lemma \ref{LOCAL_COF_LEM}. 
      
      Secondly, supppose $j$ is of the form $G/H \cdot (\1 \to \J)$ for $\J$ a $\V$-interval.
      We split this pushout into a composition of two pushouts
      \begin{equation}
            \begin{tikzcd}
                  G/H \cdot \1 \arrow[r, "a"] \arrow[d, "G/H \cdot \phi"']
                  % \arrow[dr,phantom, yshift=.1em, xshift=.5em, "\lrcorner" near end]
                  &
                  \O \arrow[d,"\phi'"]
                  \\
                  G/H \cdot \J_{\set{0}} \arrow[r] \arrow[d, "G/H \cdot \psi"']
                  % \arrow[dr,phantom, yshift=.1em, xshift=.5em, "\lrcorner" near end]
                  &
                  \O' \arrow[d,"\psi'"]
                  \\
                  G/H \cdot \J \arrow[r]
                  &
                  \P
            \end{tikzcd}
      \end{equation}
      where $\J_{\set{0}}$ is the full subcategory of $\J$ spanned by the object $0$.
      It suffices to show both $\psi'$ and $\phi'$ are local trivial $\F$-cofibrations which are essentially surjective on fixed points. 

      We first consider the bottom pushout.
      We know that $\psi$ is injective on colors and a local isomorphism in $\Op(\V)$,
      and hence so is $G/H \cdot \psi$ in $\Op^G(\V)$.
      Since colimits are created non-equivariantly, and equivariant isomorphisms are detected by invertible equivariant maps,
      \cite[Prop B.22]{Cav14} implies that $\psi'$ is also a local isomorphism in $\Op^G(\V)$,
      so in particular a local trivial $\F$-cofibration.
      
      Moreover, we observe that $\C(\P) = \C(\O') \amalg (G/H \times \set{1})$.
      Thus, if $x \in \C(\P)^K$ for $K \leq G$ is in $\C(\O')$, we have essential surjectivity trivially,
      as shown on the left below in \eqref{J-CELL_EQ},
      where $\I_c \to \I$ is a cofibrant replacement in $\Op^{\set{0,1}}(\V)$.
      %
      \begin{equation}
            \label{J-CELL_EQ}
            \begin{tikzcd}
                  \1 \arrow[rrr, "x"] \arrow[dr, " i_0"]
                  &&&
                  (\O')^K \arrow[dd, "\psi'"]
                  &[15pt] % ----------
                  \1 \arrow[r, "0"] \arrow[dr, "i_0"']
                  &
                  \J_{\set{0}} \arrow[d] \arrow[r, "g"]
                  &
                  (\O')^K \arrow[d, "{\psi'}"]
                  \\
                  &
                  \I_c \arrow[r]
                  &
                  \I \arrow[dr, "x"]
                  &
                  & % ----------
                  &
                  \J \arrow[r, "g"]
                  &
                  \P^K \arrow[d, equal]
                  \\
                  \1 \arrow[ur, " i_1"] \arrow[rrr,"x"]
                  &&&
                  (\P)^K
                  & % ----------
                  \1 \arrow[rr, "g \cdot 1"] \arrow[ur, "i_1"]
                  &&
                  \P^K
            \end{tikzcd}
      \end{equation}
      %
      If instead $x  = g \cdot 1 \in (G/H \cdot 1)^K \subseteq \mathfrak C(\P)^K$,
      the pushout square yields the diagram on the right above in \eqref{J-CELL_EQ},
      where the maps $\J_{\set{0}} \xrightarrow{g} (\O')^K$, $\J \xrightarrow{g} \P^K$ are adjoint to the composites
      \begin{equation}
            G/K \cdot \J_{\set{0}} \xrightarrow{g} G/H \cdot \J_{\set{0}} \longto \O',
            \qquad \qquad
            G/K \cdot \J \xrightarrow{g} G/H \cdot \J \longto \P
      \end{equation}
      (using that $(G/H)^K \simeq \Hom(G/K, G/H)$).
      % Lastly, if we consider (any element in the orbit of) the new object $1\in \C(\P)^H$,
      % there is an associated object $0 \in \C(\O')^H$ such that the essentially surjectivity diagram
      % factors through the pushout diagram for $\psi$:
      % \begin{equation}
      %       \begin{tikzcd}
      %             G/H \cdot \1 \arrow[r,"0"] \arrow[dr, "G/H \cdot i_0"']
      %             &
      %             G/H \cdot \J_{\set{0}} \arrow[r] \arrow[d]
      %             &
      %             \O' \arrow[d, "\psi'"]
      %             \\
      %             &
      %             G/H \cdot \J \arrow[r]
      %             &
      %             \P \arrow[d, equal]
      %             \\
      %             G/H \cdot \1 \arrow[ur, "G/H \cdot i_1"] \arrow[rr, "1"]
      %             &&
      %             \P.
      %       \end{tikzcd}
      % \end{equation}
      Hence $\psi'$ is essentially surjective and a local trival $\F$-cofibration.

      Now, consider the top pushout. \eqref{COLOR_SQ_EQ} again implies that this pushout is created in $\Op^{G, \mathfrak C(X)}(\V)$.
      In particular, this implies $\phi'$ is bijective on objects, and hence essentially surjective.
      Further, since $1_\V$ is cofibrant in $\V$, \cite[Thm. 1.15]{BM13} implies that $\J_{\set 0}$ is cofibrant in $\Op^{\**}(\V)$,
      and since $\1$ is the initial object here, $\phi$ is a trivial cofibration here.
      Thus $a_! (G/H \cdot \phi)$ is a trivial $\F$-cofibration in $\Op^{G, \mathfrak C(X)}(\V)$ by Corollary \ref{COLOR_CHANGE_Q_COR},
      {(as $G/H \cdot \phi$ is one in $\Op^{G, G/H}(\V)$,
        since $\O \to \P$ a trivial $\F$-fibration in $\Op^{G, \mathfrak C}(\V)$
        implies $j^{\**}\O^H \to \j^{\**}\P^H$ is one in $\Cat^{\mathfrak C^H}e(\V)$)}.
      Hence $\phi'$ is a trivial $\F$-cofibration in $\mathsf{Op}^{G,\C(X)}(\V)$,
      and thus is a (local) trivial $\F$-cofibration in $\Op^G(\V)$.

      Since both $\phi'$ and $\psi'$ are essentially surjective local trivial $\F$-cofibrations in $\mathsf{Op}^G(\V)$, the result is proved.

      In the case where $\V$ only admits semi-transfer, the extra assumption that $\O$ is cofibrant is sufficient to run an identical argument
      (as the utilized pushouts remain weak equivalences in the underlying semi-model structure).
\end{proof}


\subsection{2-out-of-3}

For the proof of 2-out-of-3 in Proposition \ref{2OUTOF3_PROP}, we will need to convert between
``essential surjectivity'' and ``local equivalence'' information. 
To begin, we recall some equivalence relations on objects in a $\V$-category \cite{Cav14, BM13}:
\begin{definition}
      Given $\mathcal{C}$ in  $\Cat(\V)$ and $a,b\in\mathrm{Ob}(\mathcal C)$, we say $a$ and $b$ are
      \begin{itemize}
      \item {\em equivalent} if there exists a map $\gamma: \J \to \mathcal C$ such that
            $\gamma i_0 = a$, $\gamma i_1 = b$
            for some $\V$-interval $\J$;
      \item {\em virtually equivalent} if $a$ and $b$ are equivalent in some fibrant replacement
            $\mathcal C_f$ of $\mathcal C$ in $\Cat^{\mathrm{Ob}(\mathcal C)}(\V)$;
      \item {\em homotopy equivalent} if $a$ and $b$ are isomorphic in the unenriched category $\pi_0 \mathcal C_f$
            for some fibrant replacement $\mathcal C_f$ of $\mathcal C$;
            equivalently, if there exist maps
            $\alpha: 1_\V \to \mathcal C_f(a,b)$ and $\beta: 1_\V\to \mathcal C_f(b,a)$ such that
            $\beta\alpha$ and $\alpha\beta$ are homotopic (cf. Definition \ref{HTPY_DEFN})
            to the identity arrows
            $1_V\to \mathcal C_f(a,a)$ and $1_V \to \mathcal C_f(b,b)$, respectively.
      \end{itemize}
\end{definition}

Equivariantly, we have the following:
\begin{definition}
      Fix $\mathcal{C}\in \Cat^G(\V)$, $a,b\in \mathrm{Ob}(\mathcal{C})$, and $H \leq G$.
      We say $a$ and $b$ are 
      \textit{(virtually, homotopy) $H$-equivalent}
      if they are (resp. virtually, homotopy) equivalent in $\mathcal{C}^H$;
      % Two options for virtually equivalent:
      % (i) they are virtually equivalent in $\mathcal C^H$
      % (ii) they are $H$-equivalent in some fibrant replacement $\mathcal C_f$ of $\mathcal C$
      % in $\Cat^{G, \mathrm{Ob}(\mathcal C)}(\V)$.
\end{definition}



\begin{remark}
      \label{HK_EQUIV_REM}
      We note that if $K \leq H \leq G$, then (virtually, homotopy) $H$-equivalent implies (virtually, homotopy) $K$-equivalent
      as we have inclusions of categories
      $j^{\**}(\O^H) \to j^{\**}(\O^K)$
      (resp. by functoriality of fibrant replacement,
      $j^{\**}(\O^H)_f \to j^{\**}(\O^K)_f$,
      $\pi_0(j^{\**}(\O^H)_f) \to \pi_0(j^{\**}(\O^K)_f)$).
      % Further, if $\V$ has a fibrant replacement functor that commutes with taking fixed points for any subgroup of $G$,     
      % % (in which case the two definitions of virtually $H$-equivalent coincide)
      % then virtually (resp. homotopy) $H$-equivalent implies virtually (homotopy) $K$-equivalent,
      % as we would have an inclusion of categories
      % Moreover, this would imply that $a$ and $b$ are virtually $H$-equivalent iff
      % they are $H$-equivalent in some fibrant replacement $\mathcal C_f$ of $\mathcal C$ in $\Op^{G, \mathrm{Ob}(\mathcal C)}(\V)$.
\end{remark}

\begin{remark}
      \label{HVIRT_REM}
      If $a,b\in \C$ are virtually $H$-equivalent (which could more accurately be called ``$H$-virtually equivalent''),
      they are in fact \textit{virtually} $H$-equivalent:
      there exists a lift
      \begin{equation}
            \label{FIBFIX_LIFT_EQ}
            \begin{tikzcd}
                  \mathcal C^H \arrow[d, tail, "\sim"'] \arrow[r, "\sim"]
                  &
                  (\mathcal C_f)^H
                  \\
                  (\mathcal C^H)_f \arrow[ur, dashed, "\sim"]
            \end{tikzcd}
      \end{equation}
      and thus any equivalence in $(\mathcal C^H)_f$ induces an equivalence in $(\mathcal C_f)^H$. 
\end{remark}

\begin{definition}
      For a $G$-operad $\O\in \mathsf{Op}^G(\V)$ and $a,b\in \C(\O)$, we say $a$ and $b$ are
      {\em (virtually, homotopy) $H$-equivalent}
      if they are so in $j^*\O$. 
\end{definition}

\begin{remark}
      \label{ESS_SUR_REM}
      Unpacking definitions, we see
      $F: \O \to \P$ in $\Op^G(\V)$ is essentially surjective iff
      for any $b \in \P^H$ there exists $a \in \O^H$ such that $F(a)$ and $b$ are $H$-equivalent.
\end{remark}

\begin{notation}
      As fixed points $(-)^\Gamma$ and fibrant replacement $(-)_f$ need not commute, we will write
      \begin{equation}
            \O_f(\ksi)^\Gamma = (\O_f(\ksi))^\Gamma,
            \qquad
            \O^\Gamma(\ksi)_f = (\O(\ksi)^{\Gamma})_f.
      \end{equation}
\end{notation}



The following two lemmas follow directly from the proofs of their non-equivariant counterparts:
\begin{lemma}
      [{cf. \cite[4.10]{Cav14}}]
      \label{CAV_4.10_LEM}
      $H$-equivalence and virtual $H$-equivalence define equivalence relations on $\C(\O)^H$.
\end{lemma}
% \begin{proof}
%       {\color{OliveGreen}
%         Follows exactly as in \textit{loc cite}; either version of virtual $H$-equivalence works.

%         $ $
        
%         Indeed,
%         symmetry follows from the transposition isomorphism $\tau^{\**}\J \to \J$.
        
%         Reflexivity follows from the composition $\I_c \to \I \to \mathcal C^H$,
%         \todo{either version (i) or (ii) works here}
%         $\mathcal C_f^H$
%         of cofibrant replacement followed by the map realizing the identity map on $a$.
        
%         Transitivity follows from the amalgamation of interval objects \cite[Cor. 1.16]{BM13}
%         by the following two claims.
%         First, for any maps of $G$-sets $f: A \to \mathfrak C(\O)$,
%         we have a canonical ``identity'' map $f^{\**}\O \to \O$.
%         Second, a chase through the adjunctions yields that
%         any pair of maps $h: \J \to \mathcal C$ and $h': \J' \to \mathcal C$
%         induces a map $h \** h' : \J \** \J' \to \mathcal C$ such that      
%         $(h \** h') i_0 = h i_0$ and $(h \** h') i_1 = h' i_1$.
%       }
%   \end{proof}


\begin{lemma}
      [{cf. \cite[4.12]{Cav14}, \cite[2.10]{BM13}}]
      \label{RIGHTPROPER_LEM}
      If $\V$ is right proper, then two colors are virtually $H$-equivalent iff they are $H$-equivalent. 
\end{lemma}
% \begin{proof}
%       {\color{OliveGreen}
%         Need: virtual $H$-equivalent (i). Then it is an immediate consequence of \textit{loc cite}.
%       }
% \end{proof}

We elaborate on the following two lemmas from \cite{BM13}.

\begin{lemma}
      [{cf. \cite[4.13]{Cav14}, \cite[2.11]{BM13}}]
      \label{VIR_HTPY_LEM}
      Virtually $H$-equivalent colors are homotopy $H$-equivalent. 
\end{lemma}
\begin{proof}
      {\color{OliveGreen} We reframe and complete \cite[2.10]{BM13} (they were missing \eqref{J11_CYL_EQ}).}
      Let $\mathcal C_f$ denote $j^{\**}(\O^H)_f$.
      Suppose $H: \J \to \mathcal C_f(x,y)$ realizes a virtual equivalence between $x$ and $y$.
      Let's factor the equipped map $\J \xrightarrow{\sim} \I_f$ (a weak equivalence in $\Cat^{\set{0,1}}(\V)$)
      as a trivial cofibration and trivial fibration
      $\J \xrightarrow{\sim} \J' \xrightarrow{\sim} \I_f$,
      and then $H$ extends to $\J'$ since $\mathcal C_f$ is fibrant.

      Thus we have lifts $f_{01}$ and $f_{10}$ in the diagrams below,
      where all designations on arrows are as maps in $\Cat^{\set{0,1}}(\V)$.
      \begin{equation}
            \begin{tikzcd}
                  &
                  \J \arrow[r, "H"] \arrow[d, tail, "\sim"]
                  &
                  \mathcal C_f
                  &&
                  \J \arrow[r, "H"] \arrow[d, tail, "\sim"]
                  &
                  \mathcal C_f
                  \\
                  \1 \amalg \1 \arrow[r] \arrow[d, tail]
                  &
                  \J' \arrow[ur, dashed, "H'"'] \arrow[d, two heads, "\sim"]
                  &&
                  \1 \amalg \1 \arrow[r] \arrow[d, tail]
                  &
                  \J' \arrow[ur, dashed, "H'"'] \arrow[d, two heads, "\sim"]
                  \\
                  \mathbb A \arrow[r] \arrow[ur, dashed, "f_{01}"]
                  &
                  \I_f
                  &&
                  \mathbb A^{op} \arrow[r] \arrow[ur, dashed, "f_{10}"]
                  &
                  \I_f
            \end{tikzcd}
      \end{equation}
      This provides maps
      $\alpha = H'_{01} f_{01}: 1_\V \to \mathcal C_f(x,y)$
      and
      $\beta = H'_{10} f_{10}: 1_\V \to \mathcal C_f(y,x)$.
      By construction, the composite $\alpha\beta$ (resp. $\beta\alpha$) factors through $\J'(1,1)$ (resp. $\J'(0,0)$)
      by a map we denote $f_1$ (resp. $f_0$).
      \begin{equation}
            \begin{tikzcd}
                  1_\V \arrow[r, "\simeq"] \arrow[drr, "f_1"']
                  &
                  1_\V \otimes 1_\V \arrow[r, "f_{01} \otimes f_{10}"]
                  &
                  \J'(0,1) \otimes \J'(1,0) \arrow[d, "\circ"] \arrow[r, "H'_{01} \otimes H'_{10}"]
                  &
                  \mathcal C_f(x,y) \otimes \mathcal C_f(y,x) \arrow[d, "\circ"]
                  \\
                  &&
                  \J'(1,1) \arrow[r, "H'_{11}"]
                  &
                  \mathcal C_f(y,y)
            \end{tikzcd}
      \end{equation}
      
      Now, if $\mathbb C$ is any cylinder in $\V$, we have a lift in the following diagram, for $i \in \set{0,1}$.
      \begin{equation}
            \label{J11_CYL_EQ}
            \begin{tikzcd}
                  1_\V \amalg 1_\V \arrow[rr, "{(id, f_i)}"] \arrow[d, tail]
                  &&
                  \J'(i,i) \arrow[d, "\sim", two heads]
                  \\
                  \mathbb C \arrow[r] \arrow[urr, dashed]
                  &
                  1_\V \arrow[r]
                  &
                  (1_\V)_f
              \end{tikzcd}
        \end{equation}
        Thus $f_1$ (resp. $f_0$) factors through a cylinder,
        implying $\alpha\beta$ (resp. $\beta\alpha)$ is homotopic to $id_y$ (resp. $id_x$),
        and hence that the objects $x,y\in \mathcal C_f$ are homotopy equivalent.
  \end{proof}
  
\begin{lemma}
      [{cf. \cite[4.11]{Cav14}, \cite[2.9]{BM13}}]
      \label{REF_VIRT_LEM}
      Suppose the indexing family $\F$ has units.
      Local weak $\F$-equivalences reflect $H$-equivalences.
      Explicitly,
      fixing $H \leq G$, 
      $a_0,a_1 \in \C(\O)^H$, and $F: \O \to \P$ in $\mathsf{Op}^G(\V)$ a local weak $\F$-equivalence,
      if $F(a_0)$ and $F(a_1)$ are virtually $H$-equivalent then so are $a_0$ and $a_1$.
\end{lemma}
\begin{proof}
      % {\color{OliveGreen}
      %   Need: virtual $H$-equivalent (i).
      % }
      Since $\F$ has units, $\F_1$ contains all $H \leq G \times \Sigma_1$ (cf. \cite[Remark 4.50]{BP17}), and thus
      $j^{\**}F^H: j^{\**}\O^H \to j^{\**}\P^H$ is a local weak equivalence in $\Cat(\V)$.
      %
      As in \cite{BM13}, we may build a fibrant replacement of $j^{\**}F^H$ which is a local trivial fibration in $\Cat(\V)$.
      \begin{equation}
            \begin{tikzcd}
                 j^{\**} \O^H \arrow[r, "\sim_l"] \arrow[d, dashed, "\sim"']
                  &
                  F^{\**} (j^{\**}\P^H) \arrow[r, "\simeq_l", two heads] \arrow[d, "\sim"']
                  &
                  j^{\**} \P^H \arrow[d, "\sim"]
                  \\
                  j^{\**} (\O^H)_f \arrow[r, dashed, two heads]
                  &
                  F^{\**} (j^{\**}(\P^H)_f) \arrow[r, "\simeq_l", two heads]
                  &
                  j^{\**}(\P^H)_f
            \end{tikzcd}
      \end{equation}
      (where all designations $(-)_l$ on arrows denote ``local'').
      Thus we have a lift on the left in \eqref{REF_VIRT_EQ},
      where the bottom arrow realizes the virtual $H$-equivalence between $F(a_0)$ and $F(a_1)$,
      as such a lift in $\Cat(\V)$ is equivalent to a lift in $\Cat^{\set{0,1}}(\V)$
      after pulling the right vertical arrow back along the the inclusion
      $a: \set{0,1} \to \set{a_0,a_1} \into \mathfrak C(\O)^H$ as on the right (cf. \eqref{COLOR_SQ_EQ}),
      and this lift exists by locality.
      \begin{equation}
            \label{REF_VIRT_EQ}
            \begin{tikzcd}
                  \1 \amalg \1 \arrow[r, "{(a_0,a_1)}"] \arrow[d, tail, "{(i_0, i_1)}"']
                  &
                  j^{\**}(\O^H)_f \arrow[d, two heads, "\sim_l"]
                  &&
                  \1 \amalg \1 \arrow[r, "{(a_0,a_1)}"] \arrow[d, tail, "{(i_0, i_1)}"']
                  &
                  a^{\**}j^{\**}(\O^H)_f \arrow[d, two heads, "\sim"]
                  \\
                  \J \arrow[r] \arrow[ur, dashed]
                  &
                  j^{\**}(\P^H)_f
                  &&
                  \J \arrow[r] \arrow[ur, dashed]
                  &
                  a^{\**} F^{\**} j^{\**}(\P^H)_f.
            \end{tikzcd}
      \end{equation}
\end{proof}

We can now prove the following two similar lemmas, generalizing \cite[Prop. 4.14]{Cav14} and \cite[Prop. 2.12]{BM13},
which show that in many cases, a homotopy equivalence of colors induces a homotopy equivalence of objects.

\begin{lemma}
      \label{CAV_4.14_PROP1}
      % Suppose $\V$ has \textit{Cartesian fixed points} \cite[Def 6.27]{BP17}, and
      % a fibrant replacement functor that commutes with taking fixed points for all subgroups of $G$.
      Let $\O \in \Op^G(\V)$,
      $c,d \in \mathfrak C(\O)$, 
      $H = \Stab(c)$, and suppose $c$ and $d$ are homotopy $H$-equivalent.
      %
      Then, for any $\mathfrak C(\O)$-profile $\ksi = (c_1,\dots, c_n;c)$,
      there exists a zig-zag of weak equivalences in $\V^{\Stab(\ksi)}_{\mathrm{Gr}_\ksi}$
      between $\O(\ksi)$ and $\O(\ksi_c^d)$,
      where $\ksi_c^d = (c_1,\dots, c_n;d)$.

      Finally, any functor $F: \O \to \P$ induces a functorial zig-zag of weak equivalences in $\V^{\Stab(\ksi)}_{\mathrm{Gr}_\ksi}$
      between $\P(F(\ksi))$ and $\P(\F(\ksi_c^d))$.
\end{lemma}
\begin{proof}
      For $\P = \O$, $\O_f$, or $\O_{f c} = (\O_f)_c$,
      and $(a;b) = (c;c)$, $(c;d)$, $(d;c)$, or $(d;d)$,
      we see that $\P(a;b) \in \V$ has a $\Stab(\ksi)$-action
      via the projection $\Stab(\ksi) \to H$.
      Further, naturality of composition implies that the maps
      \begin{equation}
            \alpha_{\**}: \P(\ksi) \to \P(\ksi_c^d),
            \qquad
            \beta_{\**}: \P(\ksi_c^d) \to \P(\ksi)
      \end{equation}
      and their composites are $\Stab(\ksi)$-equivariant.
      Thus we have natural composition maps
      \begin{equation}
            \P(\ksi)^\Gamma \otimes \P(c;d)^{H_\Gamma} \longrightarrow
            (\P(\ksi) \otimes \P(c;d))^\Gamma \longrightarrow
            \P(\ksi_c^d)^\Gamma
      \end{equation}
      for any graph subgroup $\Gamma \leq \Stab(\ksi) \leq G \times \Sigma_n$,
      where $H_\Gamma$ is the projection of $\Gamma$ onto $G$,
      and similarly when replacing $(c;d)$ with $(d;c)$, $(c;c)$, $(d;d)$,
      and $\ksi$ by $\theta$ when required.
      
      By assumption we have maps
      $\alpha: 1_\V \to \O^H(c;d)_f$ and $\beta: 1_\V \to \O^H(d;c)_f$
      representing the homotopy $H$-equivalence,
      with associated homotopies $H_{\beta\alpha,1}$ and $H_{\alpha\beta,1}$.
      Additionally, for any graph subgroup $\Gamma$, we have lifts of these homotopies, e.g.
      \begin{equation}
            \label{OFC_HOM_LIFT}
            \begin{tikzcd}
                  &&& \O_{f c}(c;c)^{H_\Gamma} \arrow[d, two heads, "\sim"]
                  \\
                  \mathbb C \arrow[r, "H_{\beta\alpha,1}"'] \arrow[urrr, dashed, shift left, "H_{\beta\alpha,1}"]
                  &
                  \O^H(c;c)_f \arrow[r]
                  &
                  \O^{H_\Gamma}(c;c)_f \arrow[r]
                  &
                  \O_f(c;c)^{H_\Gamma}
            \end{tikzcd}
      \end{equation}
      (using Remark \ref{HK_EQUIV_REM} and \eqref{FIBFIX_LIFT_EQ}), compatible with similar lifts of $\alpha$ and $\beta$.
      %
      Thus we have an induced homotopy $\beta\alpha \sim id_{\O_{f c}(\ksi)^{\Gamma}}$ in $\V$
      \begin{equation}
            \begin{tikzcd}
                  \O_{f c}(\ksi)^{\Gamma} \otimes (1_\V \amalg 1_\V) \arrow[d, tail] \arrow[r, "{((\beta\alpha)_{\**}, id)}"]
                  &
                  \O_{f c}(\ksi)^\Gamma
                  % &
                  % 1_\V \amalg 1_\V \arrow[d, tail] \arrow[rr, "{((\alpha\beta)_{\**}, id)}"]
                  % &&
                  % \V((\O(\ksi_c^d)_f)^\Gamma, (\O(\ksi_c^d)_f)^\Gamma)
                  \\                  
                  \O_{f c}(\ksi)^\Gamma \otimes \mathbb C \arrow[r, "H_{\beta\alpha,1}"]
                  &
                  \O_{f c}(\ksi)^\Gamma \otimes \O_{f c}(c;c)^{H_\Gamma} \arrow[u, "\circ"]
                  % &
                  % \mathbb C \arrow[r, "H_{\alpha\beta,1}"']
                  % &
                  % (\O(d;d)^{H_\Gamma})_f \arrow[r]
                  % &
                  % (\O(d;d)_f)^{H_\Gamma} \arrow[u, "{(-)_{\**}}"],
            \end{tikzcd}
      \end{equation}
      and similarly $\alpha\beta \sim id_{\O_{f c}(\ksi_c^d)^\Gamma}$.
      Hence, as both $\O_{f c}(\ksi)^\Gamma$ and $\O_{f c}(\ksi_c^d)^\Gamma$ are bifibrant by Remark \ref{LEVEL_COF_REM},
      $\alpha_{\**}: \O_{f c}(\ksi)^\Gamma \leftrightarrow \O_{f c}(\ksi_c^d)^\Gamma : \beta_{\**}$ are inverse isomorphisms in $\Ho(\V)$ for all $\Gamma$.
      %
      Thus, in particular, $\alpha_{\**}: \O_{f c}(\ksi) \to \O_{f c}(\ksi_c^d)$ is a weak equivalence in $\V^{\Stab(\ksi)}_{\mathrm{Gr}_\ksi}$.
      Composing with the functorial (co)fibrant replacement weak equivalences yields our zig-zag of $\mathrm{Gr}_\ksi$-equivalences
      \begin{equation}
            \O(\ksi) \xrightarrow{\simeq} \O_f(\ksi) \xleftarrow{\sim} \O_{f c}(\ksi)
            \xrightarrow{\simeq}
            \O_{f c}(\ksi_c^d) \xrightarrow{\sim} \O_f(\ksi_c^d) \xleftarrow{\simeq} \O(\ksi_c^d).
      \end{equation}
      
      Finally, any $F: \O \to \P$ induces a map $F_{f c}: \O_{f c} \to \P_{f c}$, and
      the images of $\alpha_{\**}$ and $\beta_{\**}$ yield
      homotopy equivalences $\P_{f c}(F(\ksi))^\Gamma \to \P_{f c}(F(\ksi_c^d))^\Gamma$ for any graph subgroup $\Gamma \leq \Stab(\ksi)$,
      as desired.
\end{proof}



\begin{lemma}
      % [{c.f. \cite[4.14]{Cav14}}]
      \label{CAV_4.14_PROP2}
      Suppose $\V$ has cofibrant symmetric pushout powers.
      Let $\O \in \Op^G(\V)$, and suppose
      $c_i$ and $d_j$ are homotopy $H$-equivalent with $H \leq \Stab(c_i) \leq \Stab(d_j)$;
      without loss of generality we may assume $i = j = 1$.

      Then, for any $\mathfrak C(\O)$-profile $\ksi = (c_1,\dots,c_n;c)$,
      there exists a zig-zag of weak equivalences between
      $\O(\ksi)$ and $\O(\theta)$
      in $\V^{\Stab(\xi)}_{\mathrm{Gr}_\xi}$,
      where
      $\theta = (d_1,\ldots, d_n; c)$, with the colors $d_i$ defined as follows:
            
      Let $\lambda \subseteq \underline{n} = \set{1,2,\ldots, n}$ denote
      the set of all $i$ such that $c_i = k_i \cdot c_1$ for some $k_i\in G$,
      and fix choices of such $k_i$.
      If $i \notin \lambda$, let $k_i$ denote the identity element of $G$.
      % \todo[inline]{if $H = \Stab{c_1}$, then $k_i$ are well-defined in $k_i H$.}
      Then, for all $i \in \underline{n}$, we define
      \begin{equation}
            \label{DCOLORS_EQ}
            d_i =
            \begin{cases}
                  k_i \cdot d_1 \qquad \qquad & i \in \lambda
                  \\
                  c_i & \mbox{otherwise.}
            \end{cases}
      \end{equation}
      
      Finally, any functor $F: \O \to \P$ induces a functorial zig-zag of weak equivalences in $\V^{\Stab(\ksi)}_{\mathrm{Gr}_\ksi}$
      between $\P(F(\ksi))$ and $\P(F(\theta))$.
\end{lemma}
\begin{proof}
      Firstly, we claim that $\Stab_{G \times \Sigma_n}(\ksi) \subseteq \Stab_{G \times \Sigma_n}(\theta)$.
      To that end, suppose $(k,\pi)\in \Stab_{G\times \Sigma_n}(\ksi)$, so that $k \cdot c_{\pi^{-1}(i)} = c_i$ for all $i$.
      Thus $\pi$ must act on $\lambda$ and $\underline{n} \setminus \lambda$ independently,
      so for $i \in \lambda$ we have % $k \cdot c_{\pi^{-1}(i)} = c_i$, or
      $k k_{\pi^{-1}(i)} \cdot c_1 = k_i \cdot c_1$, and so
      $k_i^{-1} k k_{\pi^{-1}(i)} =:h_i \in \Stab(c_1) \leq \Stab(d_1)$. Hence 
      \begin{equation}
            \label{STAB_KT_EQ}
            k \cdot d_{\pi^{-1}(i)} = k k_{\pi^{-1}(i)} \cdot d_1 = k_i h_r \cdot d_1 = k_i \cdot d_1 = d_i,
      \end{equation}
      as desired.
      On the other hand, if $i \not \in \lambda$, then
      $k \cdot d_{\pi^{-1}(i)} = k \cdot c_{\pi^{-1}(i)} = c_i = d_i$.
      
      Now, by assumption, we have maps
      $\alpha: 1_\V \to \O^H(c_1;d_1)_f$ and $\beta: 1_\V \to \O^H(d_1;c_1)_f$
      representing the homotopy $H$-equivalence,
      with associated homotopies $H_{\beta\alpha,1}$ and $H_{\alpha\beta,1}$.
      %
      Chose a set of maps (cf. \eqref{FIBFIX_LIFT_EQ})
      \begin{equation}
            \O^H(a_1;b_1)_f \to \O_f(a_1;b_1)^H,
            \qquad \qquad
            (a_1;b_1) \in  \set{(d_1;c_1), (c_1;d_1), (c_1;c_1), (d_1;d_1)},
      \end{equation}
      and use these as in \eqref{OFC_HOM_LIFT} to compatibly lift
      $\alpha_1$, $\beta_1$, their composites, and their associated homotopies to $\O_{f c}(-)^H$.
      % These provide compatable maps for all $i \in \lambda$
      % \begin{equation}
      %       \begin{tikzcd}
      %             \O^{H_i}(a_i;b_i)_f
      %             &
      %             \O^H(a_1;b_1)_f \arrow[l, "\simeq", "k_i"'] \arrow[r]
      %             &
      %             \O_f(a_1;b_1)^H \arrow[r, "\simeq"', "k_i"]
      %             &
      %             \O_f(a_i;b_i)^{H_i}.
      %       \end{tikzcd}
      % \end{equation}   
      % 
      For each $i \in \underline{n}$, define
      \begin{equation}
            \label{WEAKEQCOLORS_EQ}
            H_i =
            \begin{cases}
                  k_i H k_i^{-1} \qquad & i \in \lambda
                  \\
                  H & \mbox{else,}
            \end{cases}
            \qquad
            \qquad 
            \alpha_i =
            \begin{cases}
                  1_\V \xrightarrow{\alpha} O_{f c}(c_1;d_1)^H \xrightarrow{k_i} \O_{f c}(c_i;d_i)^{H_i} \qquad & i \in \lambda
                  \\
                  1_\V \xrightarrow{id} \O_{f c}(c_i;d_i)^H & \mbox{else},
            \end{cases}
      \end{equation}
      and $\beta_i$ similarly.
      Note that all the $H_i$, $\alpha_i$, and $\beta_i$ are independent of the choice of $k_i\in k_i H$.
      Further, 
      the pair $(\alpha_i,\beta_i)$ realizes a homotopy $H_i$-equivalence between $c_i$ and $d_i$,
      as naturality of composition implies
      $k_i (\beta\alpha) = \beta_i\alpha_i$.
      
      Assembling the above data, 
      we see that for $(a_i;b_i) = (d_i;c_i)$, $(c_i;d_i)$, $(c_i;c_i)$, or $(d_i;d_i)$,
      the $\lambda$-indexed diagrams $i \mapsto \O_{f c}(a_i;b_i)^{H_i}$
      have an action by $\Stab(\ksi)$,
      and hence so do their tensor products $\otimes_\lambda \O_{f c}(a_i;b_i)^{H_i}$.
      Thus, for any graph subgroup $\Gamma \leq \Stab(\ksi) \leq G \times \Sigma_n$,
      we have composition maps
      \begin{equation}
            \O_{f c}(\ksi)^{\Gamma} \otimes \left(
                  \bigotimes\limits_\lambda \O_{f c}(c_i;c_i)
            \right)^{\Gamma}
            \longrightarrow
            \left(
                  \O_{f c}(\ksi) \otimes \bigotimes\limits_\lambda \O_{f c}(c_i;c_i)
            \right)^\Gamma
            \longrightarrow
            \O_{f c}(\ksi)^\Gamma
      \end{equation}
      and similarly for any $(c_i;d_i)$, $(d_i;c_i)$, $(d_i;d_i)$,
      replacing $\ksi$ with $\theta$ when required.
      
      Now, consider the following maps:
      \begin{equation}
            \begin{tikzcd}[row sep = tiny]
                  \bigotimes_\lambda \alpha_i: 1_\V \simeq 1_\V^{\otimes n} \longrightarrow \bigotimes_\lambda \O_{f c}(c_i;d_i)
                  \\ % --------------------------------------------------
                  \bigotimes_\lambda \beta_i: 1_\V \simeq 1_\V^{\otimes n} \longrightarrow \bigotimes_\lambda \O_{f c}(d_i;c_i)
                  \\ % --------------------------------------------------
                  H_c: \bigotimes\limits_\lambda \mathbb C \xrightarrow{k_i H_{\beta\alpha,1}}
                  \bigotimes\limits_\lambda \O_{f c}(c_i;c_i)^{H_i} \longrightarrow
                  \bigotimes\limits_\lambda \O_{f c}(c_i;c_i)
                  \\ % --------------------------------------------------
                  H_d: \bigotimes\limits_\lambda \mathbb C \xrightarrow{k_i H_{\alpha\beta,1}}
                  \bigotimes\limits_\lambda \O_{f c}(d_i;d_i)^{H_i} \longrightarrow
                  \bigotimes\limits_\lambda \O_{f c}(d_i;d_i)
                  \\ % --------------------------------------------------
                  (\alpha_i)_{\**} : 
                  \O_{f c}(\theta)
                  \cong
                  \O_{f c}(\theta) \otimes 1_\V^{\otimes n} \xrightarrow{1 \otimes\ \bigotimes_i \alpha_i}
                  \O_{f c}(\theta) \otimes \bigotimes_i \O_{f c}(c_i;d_i) \xrightarrow{\circ}
                  \O_{f c}(\ksi)
                  \\ % --------------------------------------------------
                  (\beta_i)_{\**} :
                  \O_{f c}(\ksi)
                  \cong
                  \O_{f c}(\ksi) \otimes 1_\V^{\otimes n} \xrightarrow{1 \otimes\ \bigotimes_i \beta_i}
                  \O_{f c}(\ksi) \otimes \bigotimes_i \O_{f c}(d_i;c_i) \xrightarrow{\circ}
                  \O_{f c}(\theta)
                  \\ % --------------------------------------------------
                  (\beta_i)_{\**}(\alpha_i)_{\**}
                  \\ % --------------------------------------------------
                  (\alpha_i)_{\**}(\beta_i)_{\**}.
            \end{tikzcd}
      \end{equation}
      We claim these are all $\Stab(\ksi)$-equivariant
      (via the natural actions, and acting on $\otimes_\lambda \mathbb C$ via the projection $\Stab(\ksi) \to \Sigma_{|\ksi|}$).
      We show this holds for just the first map, as the rest are analogous or consequences:
      for any $(k,\pi) \in \Stab(\ksi)$,
      \begin{equation}
            \label{AB_STAR_EQ}
            (k, \pi) . \otimes_i(\alpha_i)
            =
            \otimes_i (k \alpha_{\pi^{-1}(i)})
            =
            \otimes_i (k k_{\pi^{-1}i} \alpha_1)
            =
            \otimes_i (k k_i h_i \alpha_1)
            =
            \otimes_i (k_i \alpha_1)
            =
            \otimes_i (\alpha_i);
      \end{equation}
      % Thirdly, since $\V% $ has Cartesian fixed-points, we see that for any graph subgroup $\Gamma \leq \Stab(\ksi)$
      % we have a string of isomorphisms
      % \begin{equation}
      %       \label{CART_FIXED_AB_EQ}
      %       \begin{tikzcd}
      %             \O(\ksi)^\Gamma \otimes \O(a_1;b_1)^{H_{\Gamma(1)}}
      %             \arrow[r, "\tilde\Delta", "\simeq"']
      %             &
      %             \O(\ksi)^\Gamma \otimes \left(
      %                   \bigotimes\limits_\lambda \O(a_i; b_i)
      %             \right)^\Gamma
      %             \arrow[r, "\simeq"]
      %             &
      %             \left(
      %                   \O(\ksi) \otimes \bigotimes\limits_\lambda \O(a_i;b_i)
      %             \right)^\Gamma,
      %       \end{tikzcd}
      % \end{equation}
      % where
      % $(a_i;b_i) \in \set{(c_i;d_i), (c_i;c_i), (d_i;d_i)}$,
      % $H_{\Gamma(1)}$ is the projection onto $G$ of those $\gamma \in \Gamma$ which fix $1 \in \underline{n}$,
      % and $\tilde\Delta$ is the ``twisted diagonal''
      % \begin{equation}
      %       \O(a_1;b_1) \xrightarrow{\Delta} \bigotimes_\lambda \O(a_1;b_1) \xrightarrow{\otimes k_i} \bigotimes_\lambda \O(a_i;b_i).
      % \end{equation}
      %  Hence these maps descend to $\Gamma$ fixed points for any graph subgroup $\Gamma \leq \Stab(\ksi)$.
      Thus, 
      Remarks \ref{HK_EQUIV_REM}, \ref{CYL_REM}, and \ref{ASSEM_HOM_REM} with Lemma \ref{ASSEM_HOM_LEM}
      yield an induced homotopy
      \begin{equation}
            (\beta_i)_{\**}(\alpha_i)_{\**} \sim id_{\O_{f c}(\ksi)^\Gamma}
      \end{equation}
      in $\V$ for all graph subgroups $\Gamma \leq \Stab(\ksi)$:
      \begin{equation}
            % \begin{tikzcd}
            %       1_\V \amalg 1_\V \arrow[rr, "{((\beta_i)_{\**}(\alpha_i)_{\**}, id)}"] \arrow[d]
            %       &&
            %       \V(\O(\ksi)^\Gamma, \O(\ksi)^\Gamma)
            %       \\
            %       \mathbb C \arrow[r, "H_{\beta\alpha,1}"]
            %       &
            %       \O(c_1;c_1)^{H_{\Gamma(1)}} \arrow[r, "\tilde\Delta"]
            %       &
            %       \left(
            %             \bigotimes_\lambda \O(c_i;c_i)
            %       \right)^\Gamma
            %       \arrow[u, "{(-)_{\**}}"]
            % \end{tikzcd}
            \begin{tikzcd}[column sep = small]
                  \O_{f c}(\ksi)^\Gamma \otimes (1_\V \amalg 1_\V) \arrow[d, tail] \arrow[rr, "{((\beta_i)_{\**}(\alpha_i)_{\**}, id)}"]
                  &&
                  % &
                  \O_{f c}(\ksi)^\Gamma
                  \\                  
                  % \mathbb C \arrow[r, "\Delta"]
                  % &
                  \O_{f c}(\ksi)^\Gamma \otimes \left(\bigotimes\limits_\lambda \mathbb C\right)^{\Stab(\ksi)}
                  \arrow[r, "H"]
                  &
                  \O_{f c}(\ksi)^\Gamma \otimes \left(\bigotimes\limits_\lambda \O_{f c}(c_i;c_i)\right)^{\Stab(\ksi)} \arrow[r]
                  &
                  \O_{f c}(\ksi)^\Gamma \otimes \left(\bigotimes\limits_\lambda \O_{f c}(c_i;c_i)\right)^{\Gamma} \arrow[u, "\circ"']                  
                  % &
                  % \mathbb C \arrow[r, "H_{\alpha\beta,1}"']
                  % &
                  % (\O(d;d)^{H_\Gamma})_f \arrow[r]
                  % &
                  % (\O(d;d)_f)^{H_\Gamma} \arrow[u, "{(-)_{\**}}"],
            \end{tikzcd}
      \end{equation}
      and similarly $(\alpha_i)_{\**}(\beta_i)_{\**} \sim id_{\O_{f c}(\theta)^\Gamma}$.
      Hence
      $(\alpha_i)_{\**}: \O_{f c}(\ksi)^\Gamma \leftrightarrow \O_{f c}(\theta)^\Gamma: (\beta_i)_{\**}$
      are inverse isomorphisms in $\Ho(V)$ for all $\Gamma$.

      The rest of the proof follows as for Proposition \ref{CAV_4.14_PROP1}.
      % Thus, in particular, $(\alpha_i)_{\**}$ is a weak equiavlence in $\V^{\Stab(\ksi)}_{\mathrm{Gr}_\ksi}$.     
      % Composing with the functorial fibrant replacement weak equivalences yields our zig-zag
      % \begin{equation}
      %       \O(\ksi) \xrightarrow{\simeq} \O_f(\ksi) \xrightarrow{\simeq} \O_f(\ksi_c^d) \xleftarrow{\simeq} \O(\ksi_c^d).
      % \end{equation}      
      
      % Finally, if $F: \O \to \P$ (with $\P$ also fibrant),
      % then the images of $(\alpha_i)_{\**}$ and $(\beta_i)_{\**}$ yield
      % homotopy equivalences $\P(\F(\ksi))^\Gamma \to \P(\F(\theta))^\Gamma$ for any graph subgroup $\Gamma \leq \Stab(\ksi)$,
      % as desired.
\end{proof}

We can now prove the main result of this subsection.
 
\begin{proposition}
      [{c.f. \cite[4.15]{Cav14}, \cite[2.13]{BM13}}]
      \label{CAV_4.15_PROP}
      \label{2OUTOF3_PROP}
      Suppose $\V$ is right proper and has cofibrant symmetric pushout powers.
      Then the class of weak $\F$-equivalences in $\mathsf{Op}^G(\V)$ satisfies the 2-out-of-3 condition.
\end{proposition}
\begin{proof}
      Let $\O \xrightarrow{F} \P \xrightarrow{L} \Q$ be a composition of maps in $\mathsf{Op}^G(\V)$.
      If $F$ and $L$ are weak $\F$-equivalences,
      the composite is obviously a local weak $\F$-equivalence:
      $\O(\ksi)^\Gamma \xrightarrow{\sim} \P(F(\ksi))^\Gamma \xrightarrow{\sim} \Q(LF(\ksi))^\Gamma$.
      Moreover, as functors preserve equivalences of colors, $L F$ is essentially surjective by Lemma \ref{CAV_4.10_LEM}. 
      % Moreover, concatination of $\V$-intervals collapses the consecutive essential surjectivity diagrams on the left
      % onto the one on the right.
      % \begin{equation}
      %       \begin{tikzcd}[row sep = tiny]
      %             \1 \arrow[rr, dashed, "a"] \arrow[dr]
      %             &&
      %             j^{\**}\O^H \arrow[dd]
      %             \\
      %             & \J \arrow[dr, dashed]
      %             &&
      %             \1 \arrow[rr, "a", dashed] \arrow[dr]
      %             &&
      %             j^{\**}\O^H \arrow[dd]
      %             \\
      %             \1 \arrow[rr, dashed, "b"] \arrow[ur] \arrow[dr]
      %             &&
      %             j^{\**} \P^H \arrow[dd]
      %             &&
      %             \J \** \J' \arrow[dr, dashed]
      %             \\
      %             & \J' \arrow[dr, dashed]
      %             &&
      %             \1 \arrow[rr, "c"] \arrow[ur]
      %             &&
      %             j^{\**}\Q^H
      %             \\
      %             \1 \arrow[ur] \arrow[rr, "c"]
      %             &&
      %             j^{\**} \Q^H
      %       \end{tikzcd}
      % \end{equation}
      
      If $L$ and $FL$ are weak $\F$-equivalences,
      then $F$ is a local weak $\F$-equivalence by 2-out-of-3 in each $\V^{\Stab(\ksi)}_{\F_\ksi}$.
      Moreover, if $b \in \P^H$, then by Remark \ref{ESS_SUR_REM}, there exists $a \in \mathfrak C(\O)^H$ such that
      $LF(a)$ and $L(b)$ are (virtually) $H$-equivalent.
      Lemma \ref{REF_VIRT_LEM} then implies $F(a)$ and $b$ are virtually $H$-equivalent, 
      and since $\V$ is right proper, Lemma \ref{RIGHTPROPER_LEM} implies they are $H$-equivalent.

      Lastly, suppose $F$ and $LF$ are weak $\F$-equivalences.
      It is immediate that $L$ is essentially surjective.
      Now, given a signature $\theta = (d_1,\ldots,d_n;d_0$) in $\C(\P)$,
      let $\Lambda = \lambda_1 \amalg \dots \amalg \lambda_r$ denote the partition of $\underline{n}$
      where $i < j$ are in the same class iff there exists $k_{i,j} \in G$ such that $d_j = k_{i,j} \cdot d_i$.
      Define $R' \subseteq \underline{n}$ to be the subset of minimal representatives in each class,
      $R = R' \amalg \set{0}$,
      and $H_r$ the stabilizer in $G$ of $c_r$ for each $r \in R$.
      Moreover, fix choices of $k_{r,j}$. 
      
      By the essential surjectivity of $F$, for all $r \in R$ there exist $c_r \in \C(\O)^{H_r}$ such that
      $F(c_r)$ is (homotopy) $H_r$-equivalent to $d_r$.
      %
      We extend the set $\set{c_r}_{r\in R}$ to a signature $(c_1,\ldots, c_n;c_0)$
      by defining $c_j = k_{r,j} \cdot c_r$.
      % THESE ARE NOT INDEPENDENT of choice, but good enough
      % Consequently, $F(c_i)$ is homotopy equivalent to $d_i$ via $k_{r,i}\gamma_r$,
      % where $\gamma_r$ realizes the homotopy equivalence between $F(c_r)$ and $d_r$ for $i \in \lambda_r$.
      This yields a diagram of the form
      \begin{equation}
            \label{TWOOFTHREE_EQ}
            \begin{tikzcd}
                  \O(c_1,\ldots, c_n;c_0) \arrow[r, "(1)"]
                  &
                  \P(F(c_1),\ldots, F(c_n); F(c_0)) \arrow[d,dash, "(3)"] \arrow[r, "(2)"]
                  &
                  \Q(LF(c_1),\ldots, LF(c_n);LF(c_0)) \arrow[d, dash, "(4)"]
                  \\
                  &
                  \P(d_1,\ldots, d_n;d_0) \arrow[r, "(5)"]
                  &
                  \Q(L(d_1),\ldots, L(d_n); L(d_0)).
            \end{tikzcd}
      \end{equation}
      $(1)$ is a weak equivalence in $\V^{\Stab(\ksi)}_{\F_\ksi}$ by assumption, and
      $(2)$ is a weak-equivalence in $\V^{\Stab(\ksi)}_{\F_\ksi}$ by 2-out-of-3 here.
      $(3)$ and $(4)$ are zig-zags of weak equivalences in $\V^{\Stab(\theta)}_{\mathrm{Gr}_\theta}$ by iterating applications of
      Propositions \ref{CAV_4.14_PROP1} and \ref{CAV_4.14_PROP2},
      as each application only changes the colors in a single partition class.
      As these zig-zags are functorial, the above diagram commutes.
      %
      Finally, $\Stab(\ksi) \geq \Stab(\theta)$ by the same calculation as in \eqref{STAB_KT_EQ}.
      Thus $(5)$ is a weak equivalence in $\V^{\Stab(\theta)}_{\F_\theta}$ by 2-out-of-3, and hence
      $L$ is a local weak $\F$-equivalence, as desired.
\end{proof}



% ------------------------------- DWYER-KAN DESCRIPTION -----------------------------

\subsection{Dwyer-Kan equivalences}

We would like to recognize our weak $\F$-equivalences slightly differently.

\begin{definition}
      $F: \O \to \P$ in $\Op^G(\V)$ is a \textit{Dwyer-Kan} (or \textit{DK}) \textit{$\F$-equivalence} if
      $F$ is a local weak $\F$-equivalence and
      $j^{\**}\pi_0(F^H)$ is essentially surjective for all $H \leq G$.
\end{definition}

\begin{remark}
      If $\V$ has diagonals, then $F \in \Op^G(\V)$ is a $DK$-$\F$-equivalence iff
      $F$ is a local weak $\F$-equivalence such that 
      the associated map of \textit{$\F$-genuine equivariant operads} under the composite
      \begin{equation}
            \Op^G(\V) \to \Op_\F(\V) \xrightarrow{\pi_0} \Op_\F(\Set) 
      \end{equation}
      is an equivalence.
\end{remark}

\begin{proposition}
      \label{WE_ARE_DK_PROP}
      Weak $\F$-equivalences in the sense of Theorem \ref{MODEL_THM} are DK-$\F$-equivalences.
\end{proposition}
\begin{proof}
      By Lemma \ref{VIR_HTPY_LEM}, essential surjectivity implies $\pi_0$-essential surjectivity. 
\end{proof}


For the reverse direction (cf. \cite[\S 2]{BM13}), we need to show that
homotopy equivalences are all virtual equivalences.
This requires another condition on $\V$, namely that the homotopy equivalences all satisfy a coherence condition.

\begin{definition}
      Recall the category $\mathbb A \in \Cat^{\set{0,1}}(\V)$ which detects arrows.
      A cofibration $\mathbb A \to \J$ in $\Cat^{\set{0,1}}(\V)$ into a $\V$-interval is called \textit{natural} if
      it fits into a commuting diagram
      \begin{equation}
            \begin{tikzcd}
                  \mathbb A \arrow[d, tail] \arrow[r]
                  &
                  \I \arrow[d, "\sim"]
                  \\
                  \J \arrow[r, "\sim"']
                  &
                  \I_f.
            \end{tikzcd}
      \end{equation}

      A homotopy equivalence between two objects in a $\V$-category $\mathcal C$ is called \textit{coherent} if
      the detecting map $\alpha: \mathbb A \to \mathcal C_f$ factors along a natural cofibration
      \begin{equation}
            \begin{tikzcd}
                  \mathbb A \arrow[r, "\alpha"] \arrow[d, tail, dashed]
                  &
                  \mathcal C_f
                  \\
                  \J \arrow[ur, dashed]
            \end{tikzcd}
      \end{equation}
      The $\V$-category $\mathcal C$ satisfies the \textit{coherence axiom} if all homotopy equivalences are coherent.
\end{definition}

\begin{proposition}
      If $\V$ is right proper with cofibrant unit satisfying the coherence axiom, then
      DK-$\F$-equivalences are weak $\F$-equivalences in $\Op^G(\V)$.
\end{proposition}
\begin{proof}
      It suffices to show that any homotopy $H$-equivalence between objects in $\P$ is in fact an $H$-equivalence.
      The coherence axiom implies that all homotopy $H$-equivalences are virtual $H$-equivalences, and then
      right properness and Lemma \ref{RIGHTPROPER_LEM} imply these are actual $H$-equivalences.
\end{proof}




% -------------------- Satisfying the coherence axiom - nothing new --------------------
\subsubsection{Satisfying the coherence axiom - nothing new}
\todo[inline]{come back: blend in to the above better, maybe don't try to add anything}
\begin{lemma}
      [{\cite[Prop 2.24]{BM13}}]
      If $\V$ satisfies transfer for operads and
      is right proper with cofibrant unit,
      then $\V$ satisfies the coherence axiom.
\end{lemma}
\begin{proof}
      This is proved in \textit{loc cite} using the $W$-construction $W(H,\I) \simeq W_!J$.
      \todo[inline]{very unsatisfying.}
\end{proof}

This has been known for a while for $\sSet$ (implied by Dwyer-Kan, Begner).

\begin{proposition}
      [{cf. \cite[\S 1]{Joy02}}]
      $(\sSet, \times)$       % or $(\sSet_{\**}, \wedge)$ then
      satisfies the coherence axiom.
\end{proposition}
\begin{proof}
      Let $\mathcal C_f \in \Cat(\sSet)$ be locally fibrant, and
      suppose $\alpha: \mathbb A \to \mathcal C_f$ realizes a homotopy equivalence.
      Then $h c N \alpha: \Delta[1] \to h c N \mathcal C$ realizes a quasi-isomorphism (as $\Ho(h c N (-)) \simeq \pi_0(-)$).
      Since $W_!N \I$ is a $\V$-interval,
      it suffices to show that $\Delta[1] \to N \I$ is a trivial cofibration.
      This is obvious a weak equivalence as both are contractible,
      and (by e.g. \cite[Lemma 0.15]{Rie}) is cellular on outer horn inclusions.
\end{proof}

\begin{remark}
      There are other known examples.
      In particular,   
      if $\V$ satisfies the Lurie's \textit{Invertibility Hypothesis} \cite[A.3.2.12]{Lur09},
      then $\V$ satisfies the coherence axiom.

      Moreover, \cite{Law16} shows that it suffices to show
      \begin{enumerate*}
      \item $\V$ is cofibrantly generated and locally presentable
      \item every object in $\V$ is cofibrant, and
      \item weak equivalences are closed under filtered colimits.
      \end{enumerate*}
\end{remark}


\subsection{Examples}




% ---------------------------------------- EXAMPLES ------------------------------

\begin{example}
      $(\Top,\times)$:
      cellularity in \cite{Pia91},
      cofibrant symmetric pushout powers in {\color{red} SOURCE},
      right proper and generating set of intervals follows from all objects being fibrant (and \cite[Lemma 2.1]{BM13} for the latter).
\end{example}

\todo[inline]{come back: compare cofibrant symmetric pushout powers with}
\begin{itemize}
\item freely powered of Lurie
\item commutative monoid axiom of White-Yau
\item symmetric h-monoidal of Pavlov-Scholbach
\end{itemize}


% \begin{example}
%       $R$ a commutative ring containing the rational numbers,
%       $\mathcal A$ the abelian category of projective $R$-modules,
%       and consider $Ch(\mathcal A)$ with projective model strucutre:

%       \begin{itemize}
%       \item cellularity in \cite{Ste16},
%       \item cofibrant symmetric pushout powers implied by ``freely powered'', proved in \cite[Prop 7.1.4/7]{Lur17},
%       \item right proper?
%       \item generating set of intervals?
%       \item cofibrant unit?            
%       \end{itemize}

%       More generally, $\mathcal C$ locally presentable quasi-abelian category,
%       $R$ a commutative monoid object in $\mathcal C$ containing the rational numbers,
%       and consider $dg_R(\mathcal C)$ with the projective model structure \cite[Prop 2.12]{Wal15};
%       this also satisfies cofibrant symmetric pushout powers by \cite[Prop 3.4]{Wal15}.
% \end{example}













%%%%%%%%%%%%%%%%%%%%%%%%%%%%%%%%%%%%%%%%%%%%%%%%%%%%%%%%%%%%%%%%%%%%%%%
\newpage

\section{The homotopy genuine equivariant operad}


Our goal in this section is to build,
for each $G$-$\infty$-operad $X \in \mathsf{dSet}^G$,
the associated homotopy genuine equivariant operad
$\mathsf{ho} (X)$,
which we will describe as an object in
$\mathsf{dSet}_G$
satisfying a strict Segal condition.


We start with some notation. 
Given a multiset $I$ of edges of a tree $T \in \Omega$
(formally, $I$ is a function 
$I \colon \boldsymbol{E}(T) \to \mathbb{N}_0$),
we write $\sigma^I T \in \Omega$
for the tree obtained by degenerating $T$ once for each edge in $I$.
More explicitly, $\sigma^I T$ is the unique tree such that there is a planar degeneracy
$\pi \colon \sigma^I T \to T$
such that $|\pi^{-1}(e)| = I(e) + 1$.
Moreover,
note that if $T\in \Omega_G$ is a $G$-tree, 
then $\sigma^{I} T \in \Omega_{G}$
can be defined if $I$ is $G$-equivariant
(formally, this means that the multiset $I$ is a composite
$\boldsymbol{E}(T) \to \boldsymbol{E}_G(T)
\to \mathbb{N}_0$).

Our main interest will be in degeneracies of $G$-corollas. Recall that, up to isomorphism, 
a $G$-corolla $C \in \Sigma_G$ is determined the number $0 \leq k$ of leaf orbits
and isotropy subgroups
$H_i \leq H_0 \leq G$ for $0 \leq i \leq k$,
where $H_0$ is the isotropy of a (chosen) root edge.
Pictorially, such a $G$-corolla has the orbital representation given on the left below,
but in this section we will find it more convenient to label edge orbits using coset notation as on the right below,
so that $[e_i] = G e_i$ denotes the $G$-orbit of $e_i$.
\[
\begin{tikzpicture}
[grow=up,auto,level distance=2.3em,every node/.style = {font=\footnotesize},dummy/.style={circle,draw,inner sep=0pt,minimum size=1.75mm}]
	\node at (0,0) [font=\normalsize]{$C$}
		child{node [dummy] {}
			child{
			edge from parent node [swap,near end] {$G/H_k$} node [name=Kn] {}}
			child{
			edge from parent node [near end] {$G/H_1$}
node [name=Kone,swap] {}}
		edge from parent node [swap] {$G/H_0$}
		};
		\draw [dotted,thick] (Kone) -- (Kn) ;
	\node at (5,0) [font=\normalsize]{$C$}
		child{node [dummy] {}
			child{
			edge from parent node [swap,near end] {$[e_k]$} node [name=Kn] {}}
			child{
			edge from parent node [near end] {$[e_1]$}
node [name=Kone,swap] {}}
		edge from parent node [swap] {$[e_0]$}
		};
		\draw [dotted,thick] (Kone) -- (Kn) ;
\end{tikzpicture}
\]
We will then abbreviate $\sigma^i C = \sigma^{[e_i]} C$, and write $e_i$, $e_i'$ for the two edges of $\sigma^i C $ that degenerate the edge $e_i$ of $C$,
with $e_i$ denoting the inner edge and $e'_i$ the outer
edge.
\[
\begin{tikzpicture}
[grow=up,auto,level distance=3em,
every node/.style = {font=\footnotesize},
dummy/.style={circle,draw,inner sep=0pt,minimum size=1.75mm}]
	\node at (0,0) [font=\normalsize]{$\sigma^0 C$}
		child{node [dummy] {}
			child{node [dummy] {}
				child{
				edge from parent node [swap,near end] {$[e_k]$} node [name=Kn] {}}
				child{
				edge from parent node [near end] {$[e_1]$}
node [name=Kone,swap] {}}
			edge from parent node [swap] {$[e_0]$}}
		edge from parent node [swap] {$[e'_0]$}
		};
		\draw [dotted,thick] (Kone) -- (Kn) ;
	\node at (5,0) [font=\normalsize]{$\sigma^i C$}
		child{node [dummy] {}
			child{
			edge from parent node [swap,near end] {$[e_k]$} node [near start,inner sep=1pt,name=Kn] {}}
			child[level distance=3.4em]{node [dummy] {}
				child[level distance=2.7em]{
				edge from parent node [swap] {$[e'_i]$}
}
			edge from parent node [near end,swap] {$[e_i]$}
node [near start,inner sep=1pt,name=Kone,swap] {}
node [near start,inner sep=1pt,name=Kone1] {}}
			child{
			edge from parent node [near end] {$[e_1]$}
node [swap] {}
node [near start,inner sep=1pt,name=Kn1,swap]{}}
		edge from parent node [swap] {$[e_0]$}
		};
		\draw [dotted,thick] (Kone) -- (Kn) ;
		\draw [dotted,thick] (Kone1) -- (Kn1) ;
\end{tikzpicture}
\]
$\sigma^i C$ then has an orbital inner face
$\sigma^i C - [e_i]$ obtained by removing $[e_i]$
as well as an orbital outer face obtained by removing $e'_i$,
which we denote $\sigma^i C - [e'_i]$.
Moreover, note that we have natural identifications
$C = \sigma^i C - [e_i]$,
$C = \sigma^i C - [e'_i]$.

In what follows, we will find it convenient to simplify notation by denoting maps $\Omega[T] \to X$,
where $T \in \Omega_G$ and $X \in \mathsf{dSet}^G$,
simply as $T \to X$.


\begin{definition}
	Let $X \in \mathsf{dSet}^G$ be a $G$-$\infty$-operad and $C$ a $G$-corolla with edge orbits
	$[e_0],\cdots,[e_k]$.
	
	Given two operations 
	$f,g\colon C \rightrightarrows X$,
	we write $f \sim_i g$ if there exists a map
	$H \colon \sigma^i C \to X$ such that
\begin{itemize}
\item $f$ equals the restriction $H|_{\sigma^i C-[e'_i]}$;
\item $g$ equals the restriction $H|_{\sigma^i C-[e_i]}$;
\item the restriction $H|_{\sigma^i [e_i]}$
is the degeneracy $\sigma^i [e_i] \to [e_i] \to C \to X$.
\end{itemize}
\end{definition}


\begin{remark}\label{HOMOTBOUND REM}
	Note that if $f \sim_i g$ then it must be
	$f|_{\partial C} = g|_{\partial C}$.
\end{remark}


\begin{example}\label{EQUIVSIM EX}
	Let $G = \mathbb{Z}_{/2} = \{\pm 1\}$
	and consider the $G$-corolla with orbital and expanded representations as given on the left below.
\[
\begin{tikzpicture}
[grow=up,auto,level distance=2.3em,every node/.style = {font=\footnotesize},dummy/.style={circle,draw,inner sep=0pt,minimum size=1.75mm}]
	\node at (0,0) [font=\normalsize]{$C$}
		child{node [dummy] {}
			child{
			edge from parent node [swap] {$G \cdot e$}
node [name=Kone,swap] {}}
		edge from parent node [swap] {$G/G \cdot r$}
		};
	\node at (3,0) [font=\normalsize]{$C$}
		child{node [dummy] {}
			child{
			edge from parent node [swap,near end] {$-e$} node [name=Kn] {}}
			child{
			edge from parent node [near end] {$e$}
node [name=Kone,swap] {}}
		edge from parent node [swap] {$r$}
		};
	\node at (7,0) [font=\normalsize]{$\sigma^{\{e,-e\}} C$}
		child{node [dummy] {}
			child{node [dummy] {}
				child{
				edge from parent node [swap] {$G \cdot e'$}
node [swap] {}}
			edge from parent node [swap] {$G \cdot e$}
node [swap] {}}
		edge from parent node [swap] {$G/G \cdot r$}
		};
	\node at (10,0) [font=\normalsize]{$\sigma^{\{e,-e\}} C$}
		child{node [dummy] {}
			child{node [dummy] {}
				child{
				edge from parent node [swap] {$-e'$} node {}}
			edge from parent node [swap,near end] {$-e$} node {}}
			child{node [dummy] {}
				child{
				edge from parent node {$e'$}
node [swap] {}}
			edge from parent node [near end] {$e$}
node [swap] {}}
		edge from parent node [swap] {$r$}
		};
\end{tikzpicture}
\]
$C$ then has a single leaf $G$-edge orbit $[e] = G \cdot e$, so that for
$f,g \colon C \to X$ it is
$f \sim_1 g$
if there exists a 
$H \colon \sigma^{\{e,-e\}}C \to X$
such that 
\begin{equation}\label{EQUIVHOMOT EQ}
	f = H|_{\sigma^{\{e,-e\}}C - \{e',-e'\}}
\qquad
	g = H|_{\sigma^{\{e,-e\}}C - \{e,-e\}}
\qquad
	H_{\sigma^e e}, H|_{\sigma^{-e}-e} \text{ are degenerate}.
\end{equation}
It is worthwhile to compare this equivariant relation with the relations obtained if one forgets the $G$-actions. Indeed, while \eqref{EQUIVHOMOT EQ} implicitly assumes that all of $f,g,H$ are $G$-equivariant,
by omitting that assumption one can reinterpret 
\eqref{EQUIVHOMOT EQ}
as defining a relation
$f \sim_{[e]} g$ between not necessarily $G$-equivariant maps $f,g \colon C \to X$.

A priori, $\sim_{[e]}$ relation differs from the 
non-equivariant 
$\sim_{e}$ and $\sim_{-e}$
relations obtained by regarding $C$ as a non-equivariant corolla.
However, for $f,g,H$ as in \eqref{EQUIVHOMOT EQ} one has
\begin{equation}\label{EQUIVSIM EQ}
f = H|_{\sigma^{\{e,-e\}}C - \{e',-e'\}}
\sim_e H|_{\sigma^{\{e,-e\}}C - \{e,-e'\}}
\sim_{-e} H|_{\sigma^{\{e,-e\}}C - \{e,-e\}} =g
\end{equation}
so that, by Lemma \ref{EQUIVI LEM}(b) below one has that
$f \sim_{[e]} g$ in fact implies
$f \sim_{e} g$. Moreover, the converse statement follows immediately by using degeneracies.

More generally, similar considerations show that the $\sim$ relations are compatible with restricting the $G$-actions.
\end{example}


\begin{lemma}\label{EQUIVI LEM}
	Let $X \in \mathsf{dSet}^G$ be a $G$-$\infty$-operad and $C$ a $G$-corolla with edge orbits
	$[e_0],\cdots,[e_k]$. Then:
\begin{itemize}
	\item[(a)] each of the relations $\sim_i$ is an equivalence relation;
	\item[(b)] all the equivalence relations $\sim_i$ coincide.
\end{itemize}
\end{lemma}

\begin{proof}
	We first address (a). 
	
	For the reflexive condition, one can take $H$ to be the degeneracy
	$\sigma^i C \xrightarrow{\sigma^i} C \xrightarrow{f} X$.
	
	For the symmetry and transitive conditions, consider the tree
	$\sigma^{ii} C$, which degenerates $[e_i]$ twice.
\[
\begin{tikzpicture}
[grow=up,auto,level distance=3em,
every node/.style = {font=\footnotesize},
dummy/.style={circle,draw,inner sep=0pt,minimum size=1.75mm}]
	\node at (0,0) [font=\normalsize]{$\sigma^{ii} C$}
		child{node [dummy] {}
			child{
			edge from parent node [swap,near end] {$[e_k]$} node [near start,inner sep=1pt,name=Kn] {}}
			child[level distance=3.4em]{node [dummy] {}
				child[level distance=2.7em]{node [dummy] {}
					child[level distance=2.7em]{
					edge from parent node [swap] {$[e''_i]$}
}
				edge from parent node [swap] {$[e'_i]$}
}
			edge from parent node [near end,swap] {$[e_i]$}
node [near start,inner sep=1pt,name=Kone,swap] {}
node [near start,inner sep=1pt,name=Kone1] {}}
			child{
			edge from parent node [near end] {$[e_1]$}
node [swap] {}
node [near start,inner sep=1pt,name=Kn1,swap]{}}
		edge from parent node [swap] {$[e_0]$}
		};
		\draw [dotted,thick] (Kone) -- (Kn) ;
		\draw [dotted,thick] (Kone1) -- (Kn1) ;
\end{tikzpicture}
\]
Suppose $f \sim_i g$, and let 
$H \colon \sigma^{i} C \to X$ be the associated homotopy.
Define a map 
$\bar{H} \colon \Lambda^{[e_i]}_o[\sigma^{ii} C] \to X$ by
\[
	\bar{H}|_{\sigma^{ii}C - [e''_i]} = H,
		\qquad
	\bar{H}|_{\sigma^{ii}C - [e'_i]} = f \circ \sigma^i,
		\qquad
	\bar{H}|_{\sigma^{ii} [e_i]} = 
	f|_{[e_i]} \circ \sigma^{ii} =
	g|_{[e_i]} \circ \sigma^{ii}.
\]
Since the orbital inner horn inclusion
$\bar{H} \colon \Lambda^{[e_i]}_o[\sigma^{ii} C] \to \Omega[C]$
is $G$-inner anodyne,
$\bar{H}$ admits an extension $\widetilde{H} \colon \sigma^{ii}C \to X$.
The restriction $\bar{H}|_{\sigma^{ii}C - [e_i]}$ then provides the homotopy exhibiting $g \sim_i f$, and symmetry of $\sim_i$ follows.

Next, suppose $f \sim_i g$ and $g \sim_i h$, and let 
$H \colon \sigma^{i} C \to X$ and
$K \colon \sigma^{i} C \to X$ be be the associated homotopies.
Define a map 
$\bar{H} \colon \Lambda^{[e'_i]}_o[\sigma^{ii} C] \to X$ by
\[
	\bar{H}|_{\sigma^{ii}C - [e''_i]} = H,
		\qquad
	\bar{H}|_{\sigma^{ii}C - [e_i]} = K,
		\qquad
	\bar{H}|_{\sigma^{ii} [e_i]} = 
	f|_{[e_i]} \circ \sigma^{ii} =
	g|_{[e_i]} \circ \sigma^{ii} =
	h|_{[e_i]} \circ \sigma^{ii}.
\]
$\bar{H}$ again admits an extension $\widetilde{H} \colon \sigma^{ii}C \to X$, and the restriction $\bar{H}|_{\sigma^{ii}C - [e'_i]}$
provides the homotopy exhibiting $f \sim_i g$, so that transitivity of $\sim_i$.

We next turn to (b). Consider the tree $\sigma^{ij} C$ which degenerates $C$ once along each of $[e_i]$ and $[e_j]$.
\[
\begin{tikzpicture}
[grow=up,auto,level distance=2.75em,
every node/.style = {font=\footnotesize},
dummy/.style={circle,draw,inner sep=0pt,minimum size=1.75mm}]
	\node at (0,0) [font=\normalsize]{$\sigma^{ij} C$}
		child{node [dummy] {}
			child{
			edge from parent node [swap,near end] {$[e_k]$} node [near start,inner sep=1pt,name=Kn] {}}
			child[level distance=3.4em,sibling distance=2em]{node [dummy] {}
				child[level distance=2.7em]{
				edge from parent node [swap] {$[e'_j]$}
}
			edge from parent node [very near end,swap] {$[e_j]$}
node [near start,inner sep=1pt,name=Kone,swap] {}
node [inner sep=1pt,name=Kn2] {}}
			child[level distance=3.4em,sibling distance=2em]{node [dummy] {}
				child[level distance=2.7em]{
				edge from parent node {$[e'_i]$}
}
			edge from parent node [very near end] {$[e_i]$}
node [inner sep=1pt,name=Kone2,swap] {}
node [near start,inner sep=1pt,name=Kone1] {}}
			child{
			edge from parent node [near end] {$[e_1]$}
node [swap] {}
node [near start,inner sep=1pt,name=Kn1,swap]{}}
		edge from parent node [swap] {$[e_0]$}
		};
		\draw [dotted,thick] (Kn) -- (Kone) ;
		\draw [dotted,thick] (Kone1) -- (Kn1) ;
		\draw [dotted,thick] (Kone2) -- (Kn2) ;
\end{tikzpicture}
\]
Suppose $f \sim_i g$ with $H \colon \sigma^{i} C \to X$ the associated homotopy.
Define a map 
$\bar{H} \colon \Lambda^{[e_i]}_o[\sigma^{ij} C] \to X$ by
\[
	\bar{H}|_{\sigma^{ij}C - [e'_j]} = H,
		\qquad
	\bar{H}|_{\sigma^{ij}C - [e_j]} = f \circ \sigma^i,
		\qquad
	\bar{H}|_{\sigma^{ij}C - [e'_i]} = f \circ \sigma^j.
\]
Yet again, $\bar{H}$ admits an extension $\widetilde{H} \colon \sigma^{ij}C \to X$, and the restriction $\bar{H}|_{\sigma^{ij}C - [e_i]}$
provides a homotopy exhibiting $g \sim_j f$. (b) now follows.
\end{proof}

In light of Lemma \ref{EQUIVI LEM},
given $f,g \rightrightarrows C \to X$ with 
$C$ a $G$-corolla and $X$ a $G$-$\infty$-operad,
we will henceforth write $f \sim g$ whenever $f \sim_i g$ for some (and thus all) $i$.
We now extend the $\sim$ relation.

\begin{definition}\label{XTENDSIM DEF}
	Let $T \in \Omega_G$ be a $G$-tree
	and $X \in \mathsf{dSet}^G$ be a 
	$G$-$\infty$-operad.
	
	Given dendrices $x,y\colon T \to X$ we write
	$x \sim y$ if there are equivalences of restrictions
	$x|_{T_v} \sim y|_{T_v}$ for all $G$-vertices
	$v \in \boldsymbol{V}_G(T)$.
	
	Further, we define $\mathsf{ho}(X)(T) = X(T)/\sim$.
\end{definition}

\begin{proposition}
Let $X \in \mathsf{dSet}^G$ be a $G$-$\infty$-operad. Then the assignment 
		$T \mapsto \mathsf{ho}(X)(T)$
		is a contravariant functor in $T \in \Omega_G$, i.e.
		$\mathsf{ho}(X)\in \mathsf{dSet}_G$.
\end{proposition}


\begin{proof}
	It suffices to show that the $\sim$ equivalence relations are compatible with the generating classes of maps in $\Omega_G$, namely
	degeneracies, inner faces, outer faces, and quotient maps.
	
	The cases of degeneracies and outer faces are obvious. In the case of quotients, 
	since any quotient $\bar{T} \to T$ of $G$-trees induces quotients on $G$-vertices, it suffices to consider the case of a quotient
	$\bar{C} \xrightarrow{\pi} C$ of $G$-corollas.
	But it is then straightforward to check that if a homotopy exhibiting $f \sim_0 g$ also induces a homotopy exhibiting 
	$f \circ \pi \sim_0 g \circ \pi$
	(notably, the same needs not be true for the relations $f \sim_i g$ when $0<i$, 
	in which case the exhibiting homotopy 
	may instead exhibit a string of relations 
	$f \circ \pi \sim \cdots \sim g \circ \pi$
	as in \eqref{EQUIVSIM EQ}).

It remains to address the most interesting case,
that inner faces. Since inner faces can be factored as composites of inner faces that collapse a singe inner edge orbit,
it suffices to consider the case of faces
$D \to T$ where $T$ has a single edge edge orbit.
I.e. we can assume that there are $G$-corollas
$C_1$, $C_2$ such that 
$T = C_1 \amalg_{[e_i]} C_2$ and
$D = T - [e_i]$, as illustrated below.
\[
\begin{tikzpicture}
[grow=up,auto,level distance=3em,
every node/.style = {font=\footnotesize},
dummy/.style={circle,draw,inner sep=0pt,minimum size=1.75mm}]
	\node at (0,0) [font=\normalsize]{$C_1$}
		child{node [dummy] {}
			child{
			edge from parent node [swap,near end] {} node [near start,inner sep=1pt,name=Kn] {}}
			child[level distance=3.4em]{node {}
			edge from parent node [near end,swap] {$[e_i]$}
node [near start,inner sep=1pt,name=Kone,swap] {}
node [near start,inner sep=1pt,name=Kone1] {}}
			child{
			edge from parent node [near end] {}
node [swap] {}
node [near start,inner sep=1pt,name=Kn1,swap]{}}
		edge from parent node [swap] {$[e_0]$}
		};
		\draw [dotted,thick] (Kone) -- (Kn) ;
		\draw [dotted,thick] (Kone1) -- (Kn1) ;
	\node at (4,0) [font=\normalsize]{$C_2$}
		child{node [dummy] {}
			child{
			edge from parent node [swap,near end] {} node [name=Kn] {}}
			child{
			edge from parent node [near end] {}
node [name=Kone,swap] {}}
		edge from parent node [swap] {$[e_i]$}
		};
		\draw [dotted,thick] (Kone) -- (Kn) ;
	\node at (9,0) [font=\normalsize]{$T$}
		child{node [dummy] {}
			child{
			edge from parent node [swap,near end] {} node [near start,inner sep=1pt,name=Kn] {}}
			child[level distance=3.4em]{node [dummy] {}
				child{
				edge from parent node [swap,near end] {} node [name=Kn2] {}}
				child{
				edge from parent node [near end] {}
node [name=Kone2,swap] {}}
			edge from parent node [near end,swap] {$[e_i]$}
node [near start,inner sep=1pt,name=Kone,swap] {}
node [near start,inner sep=1pt,name=Kone1] {}}
			child{
			edge from parent node [near end] {}
node [swap] {}
node [near start,inner sep=1pt,name=Kn1,swap]{}}
		edge from parent node [swap] {$[e_0]$}
		};
		\draw [dotted,thick] (Kone) -- (Kn) ;
		\draw [dotted,thick] (Kone1) -- (Kn1) ;
		\draw [dotted,thick] (Kone2) -- (Kn2) ;
\end{tikzpicture}
\]
The claim is now that if
$x,y \colon T \to X$ are such that
$x|_{C_1} \sim y|_{C_1}$ and
$x|_{C_2} \sim y|_{C_2}$
then it is also 
$x|_{D} \sim y|_{D}$.
This will follow from the next two claims:
\begin{itemize}
\item[(i)] if $x,y \colon T \to X$ are such that
$x|_{C_1} = y|_{C_1}$ and
$x|_{C_2} = y|_{C_2}$
then $x|_{D} \sim y|_{D}$;
\item[(ii)]
given $x \colon T \to X$, $f\colon C_1 \to X$ and
$g \colon C_2 \to X$ such that
$f \sim x|_{C_1}$, $g \sim x|_{C_2}$,
there exists
$y \colon T \to X$ such that
$y|_{C_1} = f$, $y|_{C_2} = g$ and
$y|_D = x|_D$.
\end{itemize}
To show (i) and (ii), consider the degeneracies
$\sigma^0 T$ and $\sigma^i T$ pictured below.
\[
\begin{tikzpicture}
[grow=up,auto,level distance=3em,
every node/.style = {font=\footnotesize},
dummy/.style={circle,draw,inner sep=0pt,minimum size=1.75mm}]
	\node at (0,0) [font=\normalsize]{$\sigma^0 T$}
		child{node [dummy] {}
		child{node [dummy] {}
			child{
			edge from parent node [swap,near end] {} node [near start,inner sep=1pt,name=Kn] {}}
			child[level distance=3.4em]{node [dummy] {}
				child{
				edge from parent node [swap,near end] {} node [name=Kn2] {}}
				child{
				edge from parent node [near end] {}
node [name=Kone2,swap] {}}
			edge from parent node [near end,swap] {$[e_i]$}
node [near start,inner sep=1pt,name=Kone,swap] {}
node [near start,inner sep=1pt,name=Kone1] {}}
			child{
			edge from parent node [near end] {}
node [swap] {}
node [near start,inner sep=1pt,name=Kn1,swap]{}}
		edge from parent node [swap] {$[e_0]$}}
		edge from parent node [swap] {$[e'_0]$}
		};
		\draw [dotted,thick] (Kone) -- (Kn) ;
		\draw [dotted,thick] (Kone1) -- (Kn1) ;
		\draw [dotted,thick] (Kone2) -- (Kn2) ;
	\node at (6,0) [font=\normalsize]{$\sigma^i T$}
		child{node [dummy] {}
			child{
			edge from parent node [swap,near end] {} node [near start,inner sep=1pt,name=Kn] {}}
			child[level distance=3.4em]{node [dummy] {}
			child{node [dummy] {}
				child{
				edge from parent node [swap,near end] {} node [name=Kn2] {}}
				child{
				edge from parent node [near end] {}
node [name=Kone2,swap] {}}
			edge from parent node [swap] {$[e'_i]$}}
			edge from parent node [near end,swap] {$[e_i]$}
node [near start,inner sep=1pt,name=Kone,swap] {}
node [near start,inner sep=1pt,name=Kone1] {}}
			child{
			edge from parent node [near end] {}
node [swap] {}
node [near start,inner sep=1pt,name=Kn1,swap]{}}
		edge from parent node [swap] {$[e_0]$}
		};
		\draw [dotted,thick] (Kone) -- (Kn) ;
		\draw [dotted,thick] (Kone1) -- (Kn1) ;
		\draw [dotted,thick] (Kone2) -- (Kn2) ;
\end{tikzpicture}
\]
Given $x,y$ as in (i), one can now build a map
$H \colon \Lambda_o^{[e_i]}[\sigma^0 T] \to X$ by
\[
	H|_{\sigma^0 T - [e_0]} = x,
\qquad
	H|_{\sigma^0 T - [e'_0]} = y,
\qquad
	H|_{\sigma^0 C_1} = 
	x|_{C_1} \circ \sigma^0 = 
	y|_{C_1} \circ \sigma^0.
\]
Letting $\widetilde{H}\colon \sigma^0 T \to X$
be an extension of $H$,
the restriction $H|_{\sigma^0 T - [e_i]}$
provides the desired homotopy 
$x|_{D} \sim y|_{D}$, showing (i).


Lastly, let $x,f,g$ be as in (ii), 
and let
$K \colon \sigma^i C_1 \to X$ exhibit the relation
$f \sim_i x|_{C_1}$
and 
$ \bar{K} \colon \sigma^i C_2 \to X$
exhibit the relation
$x|_{C_2} \sim_i g$ (note the reversed order).
Now build the map
$H \colon \Lambda_o^{[e'_i]}[\sigma^i T] \to X$ by
\[
	H|_{\sigma^i T - [e_i]} = x,
\qquad
	H|_{\sigma^i C_1} = K,
\qquad
	H|_{\sigma^i C_2} = \bar{K}.
\]
Again letting 
$\widetilde{H} \colon \sigma^i T \to X$,
the restriction 
$\widetilde{H}|_{\sigma^i T - [e'_i]}$
provides the required $y \colon T \to X$,
showing (ii) and finishing the proof.
\end{proof}


\begin{corollary}
Let $X \in \mathsf{dSet}^G$ be a $G$-$\infty$-operad. Then
	\begin{itemize}
	\item[(a)] $\mathsf{ho}(X)\in \mathsf{dSet}_G$ is a genuine equivariant operad, i.e. it satisfies the strict right lifting condition against the Segal core inclusions
	$Sc[T] \to \Omega[T]$ for $T \in \Omega_G$;
	\item[(b)] the quotient map
	$\gamma_{\**}X \to \mathsf{ho}(X)$ is the universal map from $\gamma_{\**}X$ to a genuine equivariant operad.
	\end{itemize}
\end{corollary}

\begin{proof}
	Note first that by Remark \ref{HOMOTBOUND REM}
	any map 	$Sc[T] \to \mathsf{ho}(X)$ admits a factorization 
	$Sc[T] \to \gamma_{\**}X \xrightarrow{q} \mathsf{ho}(X)$.
	
	The right lifting property for $\mathsf{ho}(X)$
	against the maps $Sc[T] \to \Omega[T]$
	is then automatic from the lifting property for $X$.

	For strictness,	
	note that Definition \ref{XTENDSIM DEF}
	can be reinterpreted as saying that
	$x,y \colon \Omega[T] \rightrightarrows X$
	give rise to the same point of 
	$\mathsf{ho}(X)$, i.e. 
	the composites 
	$\Omega[T] \rightrightarrows X \xrightarrow{q}
	\mathsf{ho}(X)$ coincide, 
	iff the composites 
	$Sc[T] \to \Omega[T] \rightrightarrows X \xrightarrow{q}
	\mathsf{ho}(X)$ coincide, showing strictness, and thus (a).
		
	For (b), since $\mathsf{ho}(X)$ is a quotient of
	$\gamma_{\**} X$, it suffices to show that any map
	from $F \colon \gamma_{\**}X \to Y$ with $Y$ a genuine equivariant operad must also enforce the $\sim$ relation.
	For a $G$-corolla $C$ and
	$f,g\colon C \to X$ such that 
	$H \colon \sigma^i C \to X$ exhibits
	$f \sim_i g$, 
	the strict lifting condition for $Y$
	shows that the maps
	$F\circ H \colon \sigma^i C \to Y$,
	$f \circ \sigma^i \colon \sigma^i C \to Y$
	must coincide, and thus that
	$F(f)=F(g)$.
	The further claim that $F$ respects equivalences
	of general dendrices $x,y\colon T \rightrightarrows X$
	is immediate from Definition \ref{XTENDSIM DEF}.
\end{proof}


\newpage




\section{Scratchwork}

\subsection{Colored simplicial tensors and cotensors}



\[
\begin{tikzcd}
	K \otimes f^{\**} P \ar{r} \ar{ddd}&
	K \otimes f^{\**} \left( (K \otimes P)^K \right) \ar{r}{\simeq} \ar{d}&
	K \otimes \left( f^{\**} (K \otimes P) \right)^K \ar{r} \ar{d} &
	f^{\**} (K \otimes P) \ar{ddd}
\\
	&
	K \otimes f^{\**} \left( (L \otimes P)^K \right) \ar{d}
	\ar{r}{\simeq} &
	K \otimes \left( f^{\**} (L \otimes P) \right)^K
	\ar{d}
\\
	&
	L \otimes f^{\**} \left( (L \otimes P)^K \right)
	\ar{r}{\simeq} &
	L \otimes \left( f^{\**} (L \otimes P) \right)^K
\\
	L \otimes f^{\**} P \ar{r} &
	L \otimes f^{\**} \left( (L \otimes P)^L \right) \ar{r}{\simeq} \ar{u} &
	L \otimes \left( f^{\**} (L \otimes P) \right)^L \ar{r}&
	f^{\**} (L \otimes P)
\end{tikzcd}
\]



\[
\begin{tikzcd}
	K \otimes f^{\**} P \ar{r} \ar{d}&
	K \otimes f^{\**} \left( (K \otimes P)^K \right) \ar{r}{\simeq} &
	K \otimes \left( f^{\**} (K \otimes P) \right)^K \ar{r}  &
	f^{\**} (K \otimes P) \ar{ddd}
\\
	K \otimes \left( (L \otimes f^{\**} P) \right)^L \ar{d} &
	K \otimes \left( (L \otimes f^{\**} \left( (K \otimes P)^K \right)) \right)^L &
\\
	K \otimes \left( (L \otimes f^{\**} P) \right)^K \ar{d} &
	K \otimes \left( (L \otimes f^{\**} \left( (K \otimes P)^K \right)) \right)^K
\\
	L \otimes f^{\**} P \ar{r} &
	L \otimes f^{\**} \left( (L \otimes P)^L \right) \ar{r}{\simeq} &
	L \otimes \left( f^{\**} (L \otimes P) \right)^L \ar{r}&
	f^{\**} (L \otimes P)
\end{tikzcd}
\]




\[
\begin{tikzcd}
	f^{\**} P \ar{rr} \ar{rd} \ar{dd} & &
	f^{\**} \left( (K \otimes P)^K \right) \ar{d} &
	\left( f^{\**} (K \otimes P) \right)^K \ar{l}[swap]{\simeq} \ar{ddd}
\\
	&
	f^{\**} \left( (L \otimes P)^L \right) \ar{r} &
	f^{\**} \left( (L \otimes P)^K \right) 
\\
	\left( L \otimes f^{\**} P \right)^L \ar{r} \ar{d} &
	\left( f^{\**}( L \otimes P ) \right)^L \ar{u}{\simeq} \ar{rrd} 
\\
	\left( L \otimes f^{\**} P \right)^K \ar{rrr} &&&
	\left( f^{\**}( L \otimes P ) \right)^K \ar{uul}[swap]{\simeq}
\end{tikzcd}
\]


\newpage


\subsection{Semi-cofibrantly generated}


The following codifies a formal argument implicit in the proof of \cite[Thm. 7.19]{CM13b}.

\begin{definition}
Given a set $J$ of maps that admit the small object argument, we say that $X \in \mathcal{M}$ is \textit{$J$-fibrant} if $X \to \**$ has the right lifting property against maps in $J$.

Further, given $D$ a class of maps in $\mathcal{M}$,
we write $D_{J\text{-fib}} \subseteq D$ to denote 
the subclass of maps whose target is $J$-fibrant.
\end{definition}

\begin{lemma}\label{SEMICOF LEM}
	Let $\mathcal{M}$ be a model category with $(C,W,F)$
	the corresponding classes of cofibrations, weak equivalences and fibrations. 
	Further, $J$ be a set of maps admitting the small object argument and such that:
\begin{itemize}
	\item[(i)] $J \subseteq C \cap W$;
	\item[(ii)] 
	$\left(J^{\boxslash} \cap W \right)_{J\text{-fib}}
	\subseteq \left( F \cap W \right)_{J\text{-fib}}$.
\end{itemize}
Then one further has that:
\begin{itemize}
	\item[(a)]
	$\left(\prescript{\boxslash}{}{\left(J^{\boxslash}\right)}\right)_{J\text{-fib}}
	= 
	\left( C \cap W \right)_{J\text{-fib}}$;
	\item[(b)]
	$\left(J^{\boxslash} \right)_{J\text{-fib}}
	= F_{J\text{-fib}}$.
\end{itemize}
\end{lemma}

\begin{remark}
Rephrasing (b), one has that the fibrant objects of $\mathcal{M}$ are precisely the $J$-fibrant objects
and thus that the fibrations between fibrant objects are precisely the $J$-fibrations.
\end{remark}

\begin{proof}
	To check (a), recalling first that 
	$\prescript{\boxslash}{}{\left(J^{\boxslash}\right)}$
	is the saturation of $J$, one has that (i) in fact implies 
	$\prescript{\boxslash}{}{\left(J^{\boxslash}\right)}
		\subseteq C \cap W $.
	For the converse direction, given a trivial cofibration
	$A \to Y$ with $J$-fibrant target,
	form the factorization 
	$A \to X \to Y$ as a 
	$J$-cofibration followed by a $J$-fibration. 
	By the first direction the map $A\to X$ is a weak equivalence, and thus by 2-out-of-3 so is $X \to Y$.
	But then by (ii) the map $X \to Y$ is a trivial fibration, so that the lifting below exists,
	showing that $A \to Y$ is a retract of $A \to X$, and thus also in the saturation $\prescript{\boxslash}{}{\left(J^{\boxslash}\right)}$, 
	as desired.
\[
\begin{tikzcd}
	A \ar[>->]{r}{J} \ar[>->]{d}[swap]{\sim}&
	X \ar[->>]{d}{J}
%& &
%	A \ar[>->]{r}{\sim} \ar[>->]{d}[swap]{\sim}&
%	Y \ar[->>]{d}{\sim}
\\
	Y \ar[equal]{r} \ar[dashed]{ru} & Y
%& &
%	X \ar[equal]{r} \ar[dashed]{ru} & X
\end{tikzcd}
\]

To check (b), one direction is again immediate from (i),
since $J^{\boxslash} \supseteq (C \cap W)^{\boxslash} = F$.
For the converse direction, it suffices to show that 
a $J$-fibration $X\to Y$ with $J$-fibrant target has the right lifting property against trivial cofibrations, as on the left diagram below.
After factoring the bottom horizontal map as a $J$-cofibration followed by a $J$-fibration as on the right diagram, it suffices to shows that a lift $B' \to X$ exists.
But since $B'$ is $J$-fibrant, this follows from (a), which shows that the composite $A \to B \to B'$ is a $J$-cofibration.
\[
\begin{tikzcd}
	A \ar{r} \ar[>->]{d}[swap]{\sim}&
	X \ar[->>]{d}{J}
&&
	A \ar{rr} \ar[>->]{d}[swap]{\sim}&&
	X \ar[->>]{d}{J}
\\
	B \ar{r} \ar[dashed]{ru} & Y
&&
	B \ar[>->]{r}[swap]{J} &
	B' \ar[->>]{r}[swap]{J} \ar[dashed]{ru}
	& Y
\end{tikzcd}
\]
\end{proof}

\begin{remark}
	Analyzing the proof above, one is free to replace the class of fibrant objects with any other class that is compatible with $J$-fibrations, in the sense that if 
	$X \to Y$ is a $J$-fibration and $Y$ is in the class, then so is $X$.
\end{remark}





\subsection{Formalizing some stuff}

The following is a reformalized proof of \cite[Thm. 8.14]{CM13b}.


\begin{proposition}
The (right) derived composite functors in the following diagram commute up to a zigzag of weak equivalences. 
\[
\begin{tikzcd}
	\mathsf{PreOp} \ar{d}[swap]{\gamma^{\**}}&
	\mathsf{sOp} \ar{l}[swap]{N} \ar{d}{hcN}
\\
	\mathsf{sdSet} &
	\mathsf{dSet} \ar{l}{c_{!}}
\end{tikzcd}
\]
\end{proposition}

Note that though $\gamma^{\**}$ and $c_{!}$ are left Quillen, they both preserve all equivalences, 
so that one needs only perform fibrant replacements in 
$\mathsf{sOp}$.

\begin{proof}
	Recall that, given an object $X$ in a model category $\mathcal{M}$, a simplicial frame of $X$ is a fibrant replacement
	$c_!(X) \to \widetilde{X}(\bullet)$ of the constant 
	simplcial object $c_!(X)$ in the Reedy model structure on $\mathcal{M}^{\Delta^{op}}$.
	Moreover, if $X$ was already fibrant one is free to assume that $\widetilde{X}(0) = X$.
	
	Let $\mathcal{O} \in \mathsf{sOp}$ be fibrant, 
	choose a (functorial) fibrant simplicial frame
	$\widetilde{\mathcal{O}}(\bullet) \in \mathsf{sOp}^{\Delta^{op}}$, where we assume $\widetilde{\mathcal{O}} (0) = \mathcal{O}$.
	Next, let 
	$\gamma^{\**} N \widetilde{\mathcal{O}}(\bullet) 
	\to \widetilde{Q}(\bullet)$
	be a Reedy fibrant replacement in  
	$\mathsf{sdSet}^{\Delta^{op}}$.
	
	We claim that the following is a zigzag of weak equivalences in $\mathsf{sdSet}$.
\begin{equation}\label{BIGZIG EQ}
	\gamma^{\**} N \mathcal{O} \xrightarrow{\sim}
	\widetilde{Q}(0) \xrightarrow{\sim}
	\delta^{\**} \widetilde{Q} \xleftarrow{\sim}
	\widetilde{Q}_0 \xleftarrow{\sim}
	\left(\gamma^{\**} N \widetilde{\mathcal{O}}\right)_0
	\xrightarrow{\sim}
	hcN \widetilde{\mathcal{O}} \xleftarrow{\sim}
	c_{!} hcN \mathcal{O}
\end{equation}
That the first map is an equivalence is obvious from definition of $\widetilde{Q}$ and the assumption $\widetilde{\mathcal{O}}(0) = \mathcal{O}$.

For the second and third maps, note first that $\widetilde{Q}$ is homotopically constant, in the sense that all structure maps $\widetilde{Q}(m) \to \widetilde{Q}(m')$
are equivalences.
Moreover, since the levels $\widetilde{Q}$ are fibrant in 
$\mathsf{sdSet}$, this implies that these are simplicial equivalences, i.e. that for each tree $T \in \Omega$
the evaluations 
$\widetilde{Q}(T)(m) \to \widetilde{Q}(T)(m')$
are Kan equivalences in $\mathsf{sSet}$.
But since $\widetilde{Q}(T) \in \mathsf{sSet}^{\Delta^{op}}$ is itself Reedy fibrant, this shows that it is in fact joint Reedy fibrant, so that one has Kan equivalences 
$\widetilde{Q}(T)(0) \xrightarrow{\sim}
\delta^{\**} \widetilde{Q}(T) \xleftarrow{\sim}
\widetilde{Q}_0(T)$, showing that the second and third maps in \eqref{BIGZIG EQ} are indeed weak equivalences.

For the fourth equivalence, note that one can write
\[\widetilde{Q}_0(T) = 
\mathsf{Hom}_{\mathsf{sdSet}}(\Omega[T],\widetilde{Q})=
\mathsf{Hom}_{\mathsf{PreOp}}(\Omega[T],\gamma_{\**}\widetilde{Q})\]
\[
\left(\gamma^{\**} N \widetilde{\mathcal{O}}\right)_0(T) = 
\mathsf{Hom}_{\mathsf{sdSet}}(\Omega[T],\gamma^{\**} N \widetilde{\mathcal{O}})=
\mathsf{Hom}_{\mathsf{PreOp}}(\Omega[T], N \widetilde{\mathcal{O}})
\]
The claim now follows by noting that
$N \mathcal{O} \to \gamma_{\**} \widetilde{Q}$
is an equivalence of Reedy fibrant objects on 
$\mathsf{PreOp}^{\Delta^{op}}$ (over the tame model structure) and that $\Omega(T)$ is tame cofibrant. 

For the fifth equivalence, note that
\[
\left(\gamma^{\**} N \widetilde{\mathcal{O}}\right)_0(T) = 
\mathsf{Hom}_{\mathsf{PreOp}}(\Omega[T], N \widetilde{\mathcal{O}}) =
\mathsf{Hom}_{\mathsf{sOp}}(\Omega(T),  \widetilde{\mathcal{O}})
\]
\[
\left(hcN \widetilde{\mathcal{O}} \right)(T) = 
\mathsf{Hom}_{\mathsf{sOp}}(W_!(T),  \widetilde{\mathcal{O}})
\]
so that the required claim follows since 
$\widetilde{\mathcal{O}}$ is Reedy fibrant and
$W_!(T) \to \Omega(T)$ is a weak equivalence of cofibrant operads.

Lastly, for the last map, one needs simply note that
$c_! hcN \mathcal{O} = hcN c_! \mathcal{O}$, so that the required claim follows since 
$c_! \mathcal{O} \to \widetilde{O}$
is a levelwise equivalence of levelwise fibrant operads
and $hcN$ is right Quillen.
\end{proof}




\begin{lemma} \label{INTER_LEM}
Let $\mathcal{O} \in \mathsf{sOp}$ and let
$g \colon x \to y$ be an equivalence in $\mathcal{O}$.

Then there exists a countable, cofibrant and contractile $H \in \mathsf{sOp}_{\{0,1\}}$ 
and a map 
$\varphi \colon H \to \mathcal{O}$
such that 
$g$ is in the image of $\varphi$. 
\end{lemma}


\begin{proof}
	We start by considering the case where $\mathcal{O}$ is locally fibrant.
	
	$g$ can be regarded as a map
	$[1] \xrightarrow{g} \mathcal{O}$,
	and one thus likewise gets a map
	$\Delta[1] \xrightarrow{g}  hcN \mathcal{O}$
	which is an equivalence in the 
	$\infty$-category $hcN \mathcal{O}$,
	so that one can find a (non-unique) factorization
	$\Delta[1] \to J \to hcN \mathcal{O}$
	which by adjunction yields a factorization
	$[1]=W_!\Delta[1] \to W_! J \to \mathcal{O}$,
	which establishes the desired claim 
	since $W_! J$ is contractible due to 
	Example \ref{WJ EX}.
	
	For a general $\mathcal{O}$, 
	consider first a local fibrant replacement
	$F \colon \mathcal{O} \to \mathcal{O}'$.
	One can hence find a map 
	$W_! J \to \mathcal{O}'$ such that
	$F(g)$ is in its image. 
	We now factor this map as
	$W_! J \xrightarrow{\sim} H \to \mathcal{O}'$
	where the second map is a local fibration.
	
	One can now form a pullback
\[
\begin{tikzcd}
	\tilde{H} \ar{r} \ar{d} & H' \ar{d}
\\
	\mathcal{O} \ar{r} & \mathcal{O}'
\end{tikzcd}
\]
where $\tilde{H}$ is seen to be contractible since
$\mathsf{sSet}$ is right proper.
	A priori, $\tilde{H}$ will need not be countable nor cofibrant, but this is easily rectified:
	indeed one can show that any countable subcomplex of $\tilde{H}$ is contained in a contractible countable subomplex, 
	yielding a countable contractible subcomplex whose image in $\mathcal{O}$ contains $g$. Lastly, performing a cofibrant replacement of that complex finishes the proof.
\end{proof}





\begin{example}\label{WJ EX}
	Let $J = N \widetilde{[1]}$ be the nerve of the contractible groupoid on two objects.
	
	Then there is an identification
\[
	W_{!} J \simeq \mathbb{F}^{\bullet} \widetilde{[1]}
\]
	where $\mathbb{F}$ denotes the (unital) free operad monad.

	To see this, we start by describing
	$\mathbb{F}^{\bullet} \widetilde{[1]}$.
	Writing $f \colon 0 \to 1$ and 
	$g \colon  1\to 0$ for the non-identity arrows in 
	$\widetilde{[1]}$ (so that $g=f^{-1}$), the $0$-simplices of $\mathbb{F}^{\bullet} \widetilde{[1]}$
	are the alternating words
	$f,g,fg,gf,fgf,gfg,fgfg,gfgf,\cdots$
	in the letters $f$, $g$.
	More generally
	$n$-simplices are given by equipping such alternating words with ``$n$ nested layers of brackets''
	(so that, for example, 
	$\left((f)(gf)\right) 
	\left( (gf) \right)$
	encodes a $2$-simplex).
	Alternatively, given an alternating word of length $l$, such bracketings are encoded by a flag of subsets
	$F_1 \subseteq F_2 \subseteq \cdots 
	\subseteq F_n \subseteq \{1,\cdots,l-1\}$.
	
	To describe $W_{!} J$, we apply the explicit description of the $W_{!} (-)$ construction given in 
	\cite{DS11}.
	Following \cite[Cor. 4.8]{DS11}, the $n$-simplices of $W_{!} J$ are uniquely encoded by a map
\begin{equation}\label{NECMAP EQ}
	N = \Delta^{k_1} \vee \Delta^{k_2} 
	\vee \cdots \vee \Delta^{k_r} 
	\to J 
\end{equation}
which is totally nondegenerate (this means that all simplices $\Delta^{k_i} \to J$ are nondegenerate)
together with a flag of subsets
	$\boldsymbol{J}(N) = 
	G_0 \subseteq 
	G_1 \subseteq \cdots \subseteq
	G_{n-1} \subseteq \boldsymbol{E}^{\mathsf{i}}(N)$.
	Noting that the nondegenerate simplices of $J$ (other than the points $0,1$)
	are themselves identified with alternating words
	$f,g,fg,gf,fgf,gfg,fgfg,gfgf,\cdots$,
	one sees that so is the map \eqref{NECMAP EQ}.
	Therefore, we see that a $n$-simplex of 
	$W_{!} J$ is uniquely, determined by some alternating word of some size $l$ together with a flag  
	$G_0 \subseteq 
	G_1 \subseteq \cdots \subseteq
	G_{n-1} \subseteq \boldsymbol{E}^{\mathsf{i}}(N)
	=\{1,\cdots,l-1\}$, since 
	$G_0$ suffices to recover the domain of 
	\eqref{NECMAP EQ}.
	
	This shows that $W_{!} J$ $\mathbb{F}^{\bullet} \widetilde{[1]}$ indeed have the same simplices.
	The fact that the simplicial operators coincide can be readily checked explicitly with the most interesting case is that of the top differential $d_n$, which in either case is induced by multiplication in 
	$\widetilde{[1]}$ 
	(that this is the case for $W_{!} J$ follows from the description of the simplicial operators in 
	\cite[Cor. 4.4]{DS11} together with the description of the ``flanking'' procedure in \cite[Lemma 4.5]{DS11}).
\end{example}




\subsection{Extra lifts for $\infty$-categories}


\begin{lemma}
The inclusion 
\[[0,1,2] \cup [0,2,3,\cdots,n] \cup 
\Lambda^0[0,1,3,\cdots,n]
 \to \Lambda^{0,2}[n]\]
is built cellularly from inclusions
$\Lambda^0[k] \to \Delta[k]$ with $k<n$.
Moreover, all such cells send $[0,1]$ to $[0,1]$.
\end{lemma}

\begin{proof}
Since $[0,2,3,\cdots,n]$ is in the domain, all missing faces must contain $1$.
Moreover, since the smallest face not containing $2$
that is not in $\Lambda^0[0,1,3,\cdots,n]$ is $[1,3,\cdots,n]$,
which is also the smallest face not in $\Lambda^{0,2}[n]$,
we see that all missing faces must contain $2$ as well.

It now suffices to check that $0$ is characteristic with respect to the missing faces, i.e.
that $12\underline{a}$ is missing iff $012\underline{a}$ is missing, and this is now obvious. 
\end{proof}

\begin{remark}
	The map $\Lambda^{0,2}[n] \to \Delta[n]$ is inner anodyne ($2$ is characteristic).
	
	This observation, together with the precious lemma, are the technical core of the observation that lifts
\[
	\begin{tikzcd}
	\Lambda^{0}[n] \ar{d}  \ar{r} & X
\\
	\Delta[n] \ar[dashed]{ru}
	\end{tikzcd}
\]
exist when $X$ is an $\infty$-category and $[0,1]$ is mapped to an equivalence in $X$.
\end{remark}




\subsection{TBD}


\begin{lemma}
Suppose that a subcategory $\Xi$ has subcategories 
$\Xi^-,\Xi^+$ which contain the isomorphisms and satisfy the unique factorization up to unique isomorphism axiom.

Write $\mathsf{Arr}(\Xi)$ for the arrow category of $\Xi$ and 
$\mathsf{Arr}^{-}(\Xi), \mathsf{Arr}^{+}(\Xi)$
for the full subcategories whose objects are arrows in $\Xi^-,\Xi^+$. 
Then $\mathsf{Arr}^{-}(\Xi)$ (resp. $\mathsf{Arr}^{+}(\Xi)$)
is initial (resp. terminal) in $\mathsf{Arr}(\Xi)$.
\end{lemma}



\begin{proof}
Given $f \in \mathsf{Arr}(\Xi)$, we need to show that there exist diagrams as on the left, and moreover that all such diagrams are connected. 
Existence is immediate from the factorization assumption . Moreover, its is straightforward from the ``uniqueness up to isomorphism'' that all connections are connected.
But by factoring the left vertical map in the diagram below, we now see that all such diagrams are connected to a factorization.
\[
\begin{tikzcd}
	\bullet \ar{r}{f} \ar{d}& 
	\bullet \ar{d}
\\
	\bullet \ar{r}[swap]{+} &
	\bullet
\end{tikzcd}
\]
%
%\[
%\begin{tikzcd}
%	\bullet \ar{r}{-} \ar{dd} \ar{rd}[swap]{-} &
%	\bullet \ar{rrr}{+} \ar{rrd}{-} &&&
%	\bullet \ar{dd}
%\\
%	&
%	\bullet \ar{rrd}[swap]{+}  && 
%	\bullet \ar{rd}{+} \ar[dashed]{ll}[swap]{\simeq}
%\\
%	\bullet 	\ar{rrr}[swap]{-} &&&
%	\bullet \ar{r}[swap]{+} & 
%	\bullet
%\end{tikzcd}
%\]
\end{proof}



\begin{lemma}\label{REDUCELAN LEM}
	Suppose that $F \colon \mathcal{C} \to \mathcal{D}$ is a functor in 
	$\mathsf{Cat}^G$.
Then the following square commutes up to natural isomorphism
\[
\begin{tikzcd}[column sep=50pt]
	\mathcal{V}^{G \ltimes \mathcal{C}} 
	\ar{r}{\mathsf{Lan}_{G \ltimes \mathcal{C} \to G \ltimes \mathcal{D}}} \ar{d}[swap]{\mathsf{fgt}}&
	\mathcal{V}^{G \ltimes \mathcal{D}} \ar{d}{\mathsf{fgt}}
\\
	\mathcal{V}^{\mathcal{C}} 
	\ar{r}[swap]{\mathsf{Lan}_{\mathcal{C} \to\mathcal{D}}} &
	\mathcal{V}^{\mathcal{D}}
\end{tikzcd}
\]
\end{lemma}


\begin{proof}
For each $d \in \mathcal{D}$ (recall that $\mathcal{D}$ and $G \ltimes \mathcal{D}$ have the same objects) one has an obvious inclusion 
$\mathcal{C} \downarrow d \to G\ltimes \mathcal{C} \downarrow d$.
Moreover, for each object of $G\ltimes \mathcal{C} \downarrow d$
there is a unique $g \in G$ such that the object is described as a composite
$F(c) \xrightarrow{g} g F(c) \to d$,
were $g F(c) \to d$ can be regarded as an object of $\mathcal{C} \downarrow d$.
One thus has a retraction 
$G \ltimes \mathcal{C} \downarrow d \to \mathcal{C} \downarrow d$
showing that $\mathcal{C} \downarrow d$ is terminal in
$G \ltimes \mathcal{C} \downarrow d$
and finishing the proof. 
\end{proof}



\subsection{Fixed color lemmas}

Given a set $\mathfrak{C}$ of colors,
write $\Sigma_{\mathfrak{C}}$ for the groupoid of corollas with edges labeled by colors in $\mathfrak{C}$.

If, in addition, $\mathfrak{C}$ is a $G$-set, 
we write $G \ltimes \Sigma_{\mathfrak{C}}^{op}$ for the larger groupoid obtained via the associated Grothendieck construction.


Writing
$\mathsf{Sym}^{G,\mathfrak{C}} = 
\mathsf{Set}^{G \ltimes \Sigma_{\mathfrak{C}}^{op}}$,
we have a natural monad $\mathbb{F}$ on
$\mathsf{Sym}^{G,\mathfrak{C}}$
whose algebra category we denote by 
$\mathsf{Op}^{G,\mathfrak{C}}$.


\begin{notation}
	Given a $G$-equivariant function 
	$f \colon \mathfrak{C} \to \mathfrak{D}$
	we write
\[
	f_{\**} \colon 
	\mathsf{Sym}^{G,\mathfrak{C}}
	\rightleftarrows
	\mathsf{Sym}^{G,\mathfrak{D}}
	\colon f^{\**}
\qquad
	\check{f}_{\**} \colon 
	\mathsf{Op}^{G,\mathfrak{C}}
	\rightleftarrows
	\mathsf{Op}^{G,\mathfrak{D}}
	\colon f^{\**}
\]
for the standard adjunctions (note the need to distinguish notations for the left adjoints).
\end{notation}


\begin{example}\label{GCORMPA EX}
Given a $G$-corolla $C \in \Sigma_G$, we write $\partial C$ for the set of edges of $C$, which is naturally identified with the set of objects of the associated $G$-operad
$\Omega(C) \in \mathsf{Op}^G$.

One can then regard $\Omega(C) \in \mathsf{Op}^{G,\partial C}$ and, moreover, $\Omega(C)$ is in fact the free operad over the symmetric sequence obtained by removing the units of $\Omega(C)$,
which we denote by
$\Omega'(C) \in \mathsf{Sym}^{G,\partial C}$.

Given the non-equivariant decomposition
$C = C_1 \amalg \cdots \amalg C_k$
with $C_i \in \Sigma$, 
one can naturally regard the $C_i$ as objects of $\Sigma_{\partial C}$.
In fact, one then has an identification
\begin{equation}\label{SOMEIDEN EQ}
	\Omega'(C) \simeq 
	\Sigma_{\partial C}[C_1] \amalg \cdots \amalg \Sigma_{\partial C}[C_k]
\end{equation}
where $\Sigma_{\partial C}[-]$ denotes the representable presheaf in 
$\mathsf{Set}^{\Sigma_{\partial C}^{op}}$.
This claim requires some justification, since a priori the right hand side of \eqref{SOMEIDEN EQ} is an object in $\mathsf{Set}^{\Sigma_{\partial C}^{op}}$,
rather than in $\mathsf{Set}^{G \ltimes \Sigma_{\partial C}^{op}}$,
i.e. we need to describe the action of the additional action arrows
$D \xrightarrow{g} gD$ on this presheaf.
This action is given by the following diagram, where the vertical $g$ arrows simply act on labels, and all horizontal arrows are shuffle arrows (i.e. arrows in $\Sigma_{\partial C}$).
The diagonal $C_g$ arrow corresponds to the structural $G$-action on $C$. It is then straightforward to check that there is a unique dashed shuffle $\tau_g$ as indicated
\[
\begin{tikzcd}
	D \ar{d}[swap]{g} \ar{r}{\sigma} & C_i \ar{rd}{C_g} \ar{d}[swap]{g}
\\
	g D \ar{r}[swap]{g \sigma} & g C_i \ar[dashed]{r}{\simeq}[swap]{\tau_g} & C_{g i}
\end{tikzcd}
\]
and one defines $g_{\**}\colon \Sigma_{\partial C}[C_i] \to \Sigma_{\partial C}[C_{gi}]$
via $\sigma \mapsto \tau_g \circ (g \sigma)$.

Moreover, letting $f \colon \partial C \to \mathfrak{C}$
be a map of colors, 
one obtains $\mathfrak{C}$-corollas $C_i^{f} \in \Sigma_{\mathfrak{C}}$
by coloring each edge $e\in \partial C$ by $f(e) \in \mathfrak{C}$, resulting in a generalized identification 
\begin{equation}\label{SOMEIDENGEN EQ}
	f_{\**} \Omega'(C) \simeq 
	\Sigma_{\mathfrak{C}}[C_1^f] \amalg \cdots \amalg \Sigma_{\mathfrak{C}}[C^f_k]
\end{equation}
Indeed, \eqref{SOMEIDENGEN EQ} follows from the observation
that the Kan extension 
$\mathsf{Lan}_{G \ltimes \Sigma_{\partial C} \to G \ltimes \Sigma_{\mathfrak{C}}}$ 
coincides, after forgetting with the $G$-action arrows,
with the Kan extension
$\mathsf{Lan}_{\Sigma_{\partial C} \to \Sigma_{\mathfrak{C}}}$
(cf. Lemma \ref{REDUCELAN LEM}).
\end{example}


\begin{definition}
	Given a $\mathfrak{C}$-corolla $C$, 
	a subgroup 
	$\Gamma \leq \mathsf{Aut}_{G \ltimes \Sigma_{\mathfrak{C}}^{op}}(C)$
	is called a \textit{$G$-graph subgroup} if
	$\Gamma \cap \mathsf{Aut}_{\Sigma_{\mathfrak{C}}^{op}}(C) = \**$.
	
	We write $\mathcal{F}^{\Gamma} = \{\mathcal{F}^{\Gamma}_C\}$
	for the collection of families of $G$-graph subgroups.
	
	A $G$-$\mathfrak{C}$-symmetric sequence
	$X \in \mathsf{Sym}^{G,\mathfrak{C}}$
	is called $\Sigma$-cofibrant if each level
	$X(C)$ is $\mathcal{F}^{\Gamma}_C$-cofibrant.
\end{definition}


\begin{remark}
	Write a $\mathfrak{C}$-corolla as $C^f \in \Sigma_{\mathfrak{C}}$,
	where $C \in \Sigma$ is the underlying corolla and
	$f\colon \partial C \to \mathfrak{C}$
	is the coloring.
	A $G$-graph subgroup 
	$\Gamma \leq \mathsf{Aut}_{G \ltimes \Sigma_{\mathfrak{C}}^{op}}(C^f)$ is, under the map 
	$G \ltimes \Sigma_{\mathfrak{C}}^{op} \to
	G \times \Sigma^{op}$,
	identified with a 
	$G$-graph subgroup of 
	$G \times \mathsf{Aut}_{\Sigma^{op}}(C)$,
	i.e., with the graph of a partial antihomomorphism
\[
	G^{op} \geq H^{op} \xrightarrow{(-)^{-1}} H 
	\xrightarrow{\tau_{(-)}} \mathsf{Aut}_{\Sigma^{op}}(C)
\]
	which is subject to the requirement
\[
	f(\tau_h(e)) = h f(e).
\]
Using the $\tau$ automorphisms one can then
\begin{inparaenum}
\item[(i)] equip $C$ with a $H$-action,
so that one can regard $C \in \Sigma^H \subseteq \Sigma_H$;
\item[(ii)] extend $\Sigma_{\mathfrak{C}}[C^f]$ to 
an object in $\mathsf{Set}^{H \ltimes \Sigma_{\mathfrak{C}}}$
by defining the action of the $H$-action arrows $h$ via 
$\sigma \mapsto \tau_{h} \circ (h \sigma)$;
\item[(iii)]
following Example \ref{GCORMPA EX}, one thus has an identification
\[
	f_{\**} \Omega'(C) \simeq \Sigma_{\mathfrak{C}}[C^f]
\]
of objects in $\mathsf{Set}^{H \ltimes \Sigma_{\mathfrak{C}}}$ and therefore an identification 
\[
	f_{\**} \left( G \cdot_H \Omega'(C) \right) \simeq 
	G\cdot_H \Sigma_{\mathfrak{C}}[C^f]
\]
of objects in $\mathsf{Set}^{G \ltimes \Sigma_{\mathfrak{C}}}$.
\end{inparaenum}
\end{remark}


\begin{example}
Let $G = \mathbb{Z}_{/2} = \{\pm 1\}$ and 
$\mathfrak{C} = \{\mathfrak{a}, -\mathfrak{a}, \mathfrak{b}\}$ where we implicitly have
$-\mathfrak{b} = \mathfrak{b}$.
Consider the two $\mathfrak{C}$-corollas 
$C,D \in \Sigma_{\mathfrak{C}}$ below.
\begin{equation}
	\begin{tikzpicture}[auto,grow=up, level distance = 2.2em,
	every node/.style={font=\scriptsize,inner sep = 2pt}]%
		\tikzstyle{level 2}=[sibling distance=3em]%
			\node at (0,0) [font = \normalsize] {$C$}%	
				child{node [dummy] {}%
					child{node {}%
					edge from parent node [swap] {$-\mathfrak{a}$}}%
					child[level distance = 2.9em]{node {}%
					edge from parent node [swap,	near end] {$\mathfrak{b}$}}%
					child[level distance = 2.9em]{node {}%
					edge from parent node [near end] {$\mathfrak{b}$}}%
					child{node {}%
					edge from parent node  {$\mathfrak{a}$}}%
				edge from parent node [swap] {$\mathfrak{b}$}};%
			\node at (7,0) [font = \normalsize] {$D$}%	
				child{node [dummy] {}%
					child{node {}%
					edge from parent node [swap] {$-\mathfrak{a}$}}%
					child[level distance = 2.9em]{node {}%
					edge from parent node [swap,	near end] {$-\mathfrak{a}$}}%
					child[level distance = 2.9em]{node {}%
					edge from parent node [near end] {$\mathfrak{a}$}}%
					child{node {}%
					edge from parent node  {$\mathfrak{a}$}}%
				edge from parent node [swap] {$\mathfrak{b}$}};%
	\end{tikzpicture}%
\end{equation}%
Each of $C,D$ admit exactly two non-trivial $G$-graph subgroups,
which are encoded by the $\mathbb{Z}_{/2}$-actions on the underlying corollas depicted below.
\begin{equation}
	\begin{tikzpicture}[auto,grow=up, level distance = 2.2em,
	every node/.style={font=\scriptsize,inner sep = 2pt}]%
		\tikzstyle{level 2}=[sibling distance=3em]%
			\node at (-1.6,0) [font = \normalsize] {$C_1$}%	
				child{node [dummy] {}%
					child{node {}%
					edge from parent node [swap] {$-a$}}%
					child[level distance = 2.9em]{node {}%
					edge from parent node [swap,	near end] {$c\phantom{b}$}}%
					child[level distance = 2.9em]{node {}%
					edge from parent node [near end] {$b$}}%
					child{node {}%
					edge from parent node  {$a$}}%
				edge from parent node [swap] {$r$}};%
			\node at (1.6,0) [font = \normalsize] {$C_2$}%	
				child{node [dummy] {}%
					child{node {}%
					edge from parent node [swap] {$-a$}}%
					child[level distance = 2.9em]{node {}%
					edge from parent node [swap,	near end] {$-b$}}%
					child[level distance = 2.9em]{node {}%
					edge from parent node [near end] {$b$}}%
					child{node {}%
					edge from parent node  {$a$}}%
				edge from parent node [swap] {$r$}};%
			\node at (5.4,0) [font = \normalsize] {$D_1$}%	
				child{node [dummy] {}%
					child{node {}%
					edge from parent node [swap] {$-a$}}%
					child[level distance = 2.9em]{node {}%
					edge from parent node [swap,	near end] {$-b$}}%
					child[level distance = 2.9em]{node {}%
					edge from parent node [near end] {$b$}}%
					child{node {}%
					edge from parent node  {$a$}}%
				edge from parent node [swap] {$r$}};%
			\node at (8.6,0) [font = \normalsize] {$D_2$}%	
				child{node [dummy] {}%
					child{node {}%
					edge from parent node [swap] {$-b$}}%
					child[level distance = 2.9em]{node {}%
					edge from parent node [swap,	near end] {$-a$}}%
					child[level distance = 2.9em]{node {}%
					edge from parent node [near end] {$b$}}%
					child{node {}%
					edge from parent node  {$a$}}%
				edge from parent node [swap] {$r$}};%
	\end{tikzpicture}%
\end{equation}%

\end{example}


The following is the analogue of \cite[Prop. 3.2]{CM13b}

\begin{proposition}\label{KEYPR PROP}
Suppose that $\mathcal{O} \in \mathsf{Op}^{G,\mathfrak{C}}$
is $\Sigma$-cofibrant.
Further, let $C \in \Sigma_G$ be any $G$-corolla and consider 
a pushout in $\mathsf{Op}^{G}$ of the form
\begin{equation}\label{PUSHOUTPROP EQ}
\begin{tikzcd}
	\partial \Omega(C) \ar{r} \ar{d} & \mathcal{O} \ar{d}
\\
	\Omega(C) \ar{r} & \mathcal{P}.
\end{tikzcd}
\end{equation}
Then the induced map
\begin{equation}\label{ANODYNE MAP}
	\Omega[C] \amalg_{\partial \Omega[C]} N\mathcal{O} \to N\mathcal{P}
\end{equation}
is $G$-inner anodyne.
\end{proposition}

\begin{proof}
The desired claim that \eqref{ANODYNE MAP}
is $G$-inner anodyne will follow by applying the  
\textit{characteristic edge lemma} \cite[Lemma 3.4]{BP18},
but we first need some preliminary discussion. 

Let us write $f \colon \partial C \to \mathfrak{C}$
for the induced map of colors.
The first step is to rewrite \eqref{PUSHOUTPROP EQ} as a pushout diagram in $\mathsf{Op}^{G,\mathfrak{C}}$, which can be done by applying $\check{f}_{\**}$
to the leftmost objects in \eqref{PUSHOUTPROP EQ}.
Since
\[
	\check{f}_{\**} \Omega(C) \simeq 
	\check{f}_{\**} \left( \mathbb{F} \Omega'(C) \right) \simeq 
	\mathbb{F} \left(f_{\**}  \Omega'(C) \right)
\]
one has that, writing $C \simeq G \cdot_H C_{\star}$ one has the alternative pushout in $\mathsf{Op}^{G,\mathfrak{C}}$
\begin{equation}
\begin{tikzcd}
	\mathbb{F} ( \emptyset ) \ar{r} \ar{d} & \mathcal{O} \ar{d}
\\
	\mathbb{F} \left( 
	G \cdot_H \Sigma_{\mathfrak{C}}[C^f_{\star}] \right) \ar{r} & \mathcal{P}.
\end{tikzcd}
\end{equation}
Writing $B = G \cdot_H \Sigma_{\mathfrak{C}}[C^f_{\star}]$, one then has
\begin{equation}\label{PUSHOPPR EQ}
	\mathcal{P}(C) = 
	\coprod_{
	[T] \in \mathsf{Iso}
	\left( \Omega_{\mathfrak{C}}^a \downarrow_{\mathsf{r}} C \right)
	}
	\left(
		\prod_{v \in V^{ac}(T)} \mathcal{O}(T_v)
	\times
		\prod_{v \in V^{in}(T)} B(T_v)
	\right)
	\cdot_{\mathsf{Aut}_{\Omega^a_{\mathfrak{C}}}(T)} \mathsf{Aut}_{\Sigma_{\mathfrak{C}}}(C)
\end{equation}

We now discuss the dendrices of $N \mathcal{P}$. Firstly, recall that, by the strict Segal condition characterization of nerves \cite[Cor. 2.7]{CM13a},
a dendrex $\Omega[T] \to N \mathcal{P}$
is uniquely specified by the tree $T \in \Omega$ together with a choice of operations
$\{p_v \in \mathcal{P}(T_v)\}_{v \in \boldsymbol{V}(T)}$.
Noting that \eqref{PUSHOPPR EQ} implies that the canonical map (of sequences) 
$\mathcal{O} \amalg B \to \mathcal{P}$
is a monomorphism, 
we will say a dendrex $(T,\{p_v\})$ is \textit{elementary}
if all operations $p_v$ are in $\mathcal{O} \amalg B$.
Additionally, an elementary dendrex $(T,\{p_v\})$ is called \textit{alternating} if $T$ is an alternating tree and 
$p_v$ is in $\mathcal{O}$ (resp. $B$) if
$v \in \boldsymbol{V}(T)$ is active (resp. inert).

Given an elementary dendrex $(T,\{p_v\})$ and a map of trees 
$\varphi \colon S \to T$
we will need to know when 
$\varphi^{\**}(T,\{p_v\})$
is again elementary.
Since all maps in $\Omega$ are, uniquely up to isomorphism,
factored as a degeneracy followed by an inner face followed by an outer face, it suffices to discuss each of those cases.
It is straightforward to check that 
$\varphi^{\**}(T,\{p_v\})$ 
is elementary whenever $\varphi$ is a degeneracy or an outer face.
For an inner face
$\varphi \colon T-D \to T$,
noting that a partial composite $p \circ_i q$ of non-unit operations is in $\mathcal{O} \amalg B$ iff both operations are in $\mathcal{O}$,
one sees that $\varphi^{\**}(T,\{p_v\})$ is elementary iff
for each $d \in D$ the adjacent vertices of $T$ are either both labeled by operations of $\mathcal{P}$ or one of them is labeled by an identity.
An elementary dendrex is called \textit{reduced} if it has no such edges. 
In other words, an elementary dendrex is reduced iff none of its inner faces are reduced, so that, in particular, all elementary dendrices admit at least one reduced inner face (note that specifying such a face is somewhat subtle: even when a dendrex is non-degenerate, it may not be enough to collapse edges connecting $\mathcal{O}$ vertices, since this may possibly introduce new identity vertices, resulting in a degenerate vertex).
Note that reduced dendrices are necessarily non-degenerate, but not vice versa.
In fact, a non-degenerate dendrex is reduced iff it has a degeneracy which is an alternating dendrex. 


In what follows, we will find it convenient to work with elementary dendrices that have been suitably ``planarized''.
To do so, fix a subset
$\mathcal{O}^{\mathsf{st}} \amalg B^{\mathsf{st}} 
\subset
\coprod_{C \in \Sigma_{\mathfrak{C}}}
\mathcal{O}(C) \amalg B(C)$
of coset representatives for the $\Sigma$-action,
which we call \textit{standard} representatives.
An elementary dendrex $(T,\{p_v\})$ is then called \textit{standard}
if all $p_v$ are standard (i.e. in 
$\mathcal{O}^{\mathsf{st}} \amalg B^{\mathsf{st}}$).
Moreover, since both $\mathcal{O}$ and $B$ are $\Sigma$-cofibrant/$\Sigma$-free sequences (the former by assumption),
for each elementary simplex $(T,\{p_v\})$,
there is a unique (replanarization) isomorphism
$\varphi \colon T' \to T$ such that
$(T',\{\varphi^{\**}p_v\})$ is standard,
and we write 
$\mathsf{st}(T,\{p_v\}) = \varphi^{\**} (T,\{p_v\}) = (T',\{\varphi^{\**}_v p_v\})$
to denote this.


Note that it now follows from \eqref{PUSHOPPR EQ} that,
for each operation $p \in \mathcal{P}(C)$,
there exists a unique standard alternating dendrex
$b'_p \colon \Omega[T'_p] \to N \mathcal{P}$ and isomorphism 
$C \simeq T'_p - \boldsymbol{E}^{\mathsf{i}}(T'_p)$
such that the composite
\begin{equation}\label{STANDELDE EQ}
	\Omega[C] \simeq
	\Omega[T'_p - \boldsymbol{E}^{\mathsf{i}}(T'_p)] \to 
	\Omega[T'_p] \xrightarrow{b'_p}
	N \mathcal{P}
\qquad
	\Omega[C] \simeq
	\Omega[T_p - \boldsymbol{E}^{\mathsf{i}}(T_p)] \to 
	\Omega[T_p] \xrightarrow{b_p}
	N \mathcal{P}
\end{equation}
is $p$. In fact, due to the correspondence between alternating elementary dendrices and reduced elementary dendrices, 
the analogous claim  also holds 
for the corresponding non-degenerate dendrex 
$b_p \colon \Omega[T_p] \to N \mathcal{P}$.


We can now finally discuss how to apply \cite[Lemma 3.4]{BP18}.

Firstly, we need to identify a $G$-poset $I$ and dendrices 
$b_i \colon \Omega[U_i] \to N \mathcal{P}$ for $i \in I$.
Firstly, the underlying set of $I$ is the set of 
non-degenerate standard dendrices of $\mathcal{P}$,
which we abbreviate as
$i = (U_i,\{p_v^i\})$.
The dendrex $b_i \colon \Omega[U_i] \to N \mathcal{P}$ is then tautological, being $i$ itself, but it will preferable to use distinct notations for $i \in I$ and
$b_i \in N \mathcal{P} (U_i)$.
Given $i,j \in I$, we write $i \leq j$ if exists a (in general not  planar) face map
$\varphi \colon U_i \to U_j$
such that $b_i = \varphi^{\**}(b_j)$.
Note that by the uniqueness of standardizations $\varphi$ can only be an isomorphism if $i=j$, showing that $\leq$ indeed satisfies anti-symmetry.
Lastly, we define the $G$-action on $I$ via
\[b_{g i} = \mathsf{st} (g b_i).\]
When reading this formula, note that
$g b_i \in \mathcal{P}(U_i)$
(since this uses the $G$-action
on $\mathcal{P}$),
while $b_{g i} \in \mathcal{P}(U_{g i})$,
where $U_{g i}$ comes with the unique isomorphism
$U_{g i} \xrightarrow{g^{-1}} U_i$ which standardizes $b g_i$.

Lastly, the characteristic edge sets 
$\Xi^i \subseteq \boldsymbol{E}^{\mathsf{i}}(U_i)$ consist of those inner edges such that at least one of the adjacent vertices is mapped to an operation in $B$. Note that, by the discussion above, for an inner face of $\varphi \colon U_i - D \to U_i$
the dendrex $\varphi^{\**}(b_i)$ is elementary iff
$D \cap \Xi^i = \emptyset$.

We now note that it is in fact 
$N \mathcal{P} = 
A \cup \bigcup_{i \in I} b_i\left(\Omega[U_i]\right)$.
Indeed, given an arbitrary (non-elementary) non-degenerate dendrex
$(S,\{p_v\}_{v \in \boldsymbol{V}(T)})$, 
the trees $T_{p_v}$ from \eqref{STANDELDE EQ}
can be regarded as a $S$-substitution datum, which after assembled
yields a non-degenerate standard dendrex 
$\Omega[T] \xrightarrow{b_{\{p_v\}}} N \mathcal{P}$ 
whose image contains $(S,\{p_v\}_{v})$.

We now check the characteristic conditions in \cite[Lemma 3.4]{BP18}.

(Ch0.2) is straightforward.

For (Ch1), since outer faces of standard dendrices are again standard, one needs only consider the case 
$\bar{V}=U_i$, or else $\bar{V}$ would be in some $U_j$ for $j<i$.
But if $\bar{V}=U_i$, the assumption in (Ch1) states that
$\Xi^i = \emptyset$, so that $i$ must either be a dendrex where all vertices map to $\mathcal{O}$, i.e. $b_i \in N \mathcal{O}$,
or $U_i$ is a corolla its vertex maps to $B$, i.e. 
$b_i \in B$.
In either case, one has
$b_i \in A = B \cup N\mathcal{O}$, and (Ch1) follows.

To check both (Ch2) and (Ch3), observe first that
$V \hookrightarrow U_i$ will automatically be in $A_{<i}$ if either 
$\bar{V} \neq U_i$ or $T = U_i - D$ with 
$D \not \subseteq \Xi_i$
(since in either case $U_i$ would be in some $U_j$ with $j<i$),
so that one needs only consider the case 
$V=U_i$.

(Ch2) then follows since, except in the trivial cases where $\Xi^i = \emptyset$, the dendrex $b_i(U_i - \Xi^i)$ always contains at least one vertex not in $\mathcal{O} \amalg B$.

For (Ch3), we argue that if 
$b_i(U_i - \Xi^i) \in b_j \left( \Omega[U_j] \right)$
then in fact $i\leq j$.
Writing $\bar{U}_i = U_i - \Xi^i$, the hypothesis is that
\[
\begin{tikzcd}
	\bar{U}_i \ar{d} \ar{r}{\bar{\varphi}} & U_j \ar{r}{b_j} & N \mathcal{P}
\\
	U_i \ar[dashed]{ru}[swap]{\varphi}
\end{tikzcd}
\]
there is a face map $\bar{\varphi}$ as above such that
$b_i(\bar{U}_i) = \bar{\varphi}^{\**}(b_j)$, and the goal is to build $\varphi$ such that $b_i = \varphi^{\**}(b_j)$.
Let $w \in \boldsymbol{V}(\bar{U}_i)$ be a vertex and 
let $p_w$ be the corresponding operation of $\mathcal{P}$.
Then the outer fact $(U_i)_{w}$ is precisely $T_{p_w}$ from
\eqref{STANDELDE EQ}.
On the other hand, letting
$(U_j)_w - D_w \hookrightarrow (U_j)_w$
be any choice of reduced inner face, 
one has that this too is $T_{p_w}$, at least up to a replanarization isomorphism, i.e. one has isomorphims
$(U_i)_{w} \simeq (U_j)_w - D_w$,
compatible with the restrictions of $b_i$, $b_j$.
But then combining these isomorphisms yields the desired $\varphi$,
and (Ch3) follows.

Lastly, we show (Ch0.1).
Given any non-degenerate dendrex
$\Omega[V] \xrightarrow{c} N \mathcal{P}$
by applying \eqref{STANDELDE EQ} to each individual operation
(together with an ``assembly of substitution data'' argument)
one obtains that there exists a unique
non-degenerate standard dendrex 
$b_c \colon \Omega[U_c] \to N \mathcal{P}$,
edge subset $D_c \subseteq \Xi^c$ and isomorphism
$V \simeq U_c - D_c$
such that $b$ equals the composite
\begin{equation}\label{STANDELDEGER EQ}
	V \simeq U_c - D_c
	\hookrightarrow U_c
	\xrightarrow{b_c} N \mathcal{P}
\end{equation}
Recall now that, by the preliminary argument for (Ch2) and (Ch3), 
the non-degenerate dendrices not in 
$b_i^{-1}(A_{< i})$ are precisely the replanarizations of the faces 
$U_i - D$ with $D \subseteq \Xi^i$.
But the uniqueness of the data in \eqref{STANDELDEGER EQ}
implies that all such replanarizations of the $U_i - D$
do indeed have distinct images in $N \mathcal{P}$,
thus establishing (Ch0.1) and finishing the proof.
\end{proof}


\begin{remark}
	In general, injectivity of the map
	$b_i \colon \Omega[U_i] \to N \mathcal{P}$
	will fail in $b_i^{-1}(A_{< i})$.
	Indeed, in general two edges/vertices of $U_i$
	may be assigned the same color/operation of $\mathcal{P}$.
	In fact, injectivity may in general fail even for large outer faces.
\end{remark}


{\color{blue} Bla poset not finite but still projective/injective, which is enough}


\begin{remark}
In addition to \eqref{PUSHOPPR EQ}, 
one also has the alternative formula
\begin{equation}\label{PUSHOPPRG EQ}
	\mathcal{P}(C) = 
	\coprod_{
	[T] \in \mathsf{Iso}
	\left( G \ltimes \Omega_{\mathfrak{C}}^a \downarrow_{\mathsf{r}} C \right)
	}
	\left(
		\prod_{v \in V^{ac}(T)} \mathcal{O}(T_v)
	\times
		\prod_{v \in V^{in}(T)} B(T_v)
	\right)
	\cdot_{\mathsf{Aut}_{G \ltimes \Omega^a_{\mathfrak{C}}}(T)} \mathsf{Aut}_{G \ltimes \Sigma_{\mathfrak{C}}}(C)
\end{equation}
which replaces the roles of 
$\Omega_{\mathfrak{C}}^a$, $\Sigma_{\mathfrak{C}}$
with 
$G \ltimes \Omega_{\mathfrak{C}}^a$,
$G \ltimes \Sigma_{\mathfrak{C}}$.
The connection between the two formulas is given by Lemma \ref{REDUCELAN LEM},
though some care is needed.
Namely, \eqref{PUSHOPPRG EQ} generally features fewer coproduct summands but this is compensated by the inductions
$(-) \cdot_{\mathsf{Aut}_{G \ltimes \Omega^a_{\mathfrak{C}}}(T)} \mathsf{Aut}_{G \ltimes \Sigma_{\mathfrak{C}}}(C)$,
which produce more terms than the 
$(-) \cdot_{\mathsf{Aut}_{\Omega^a_{\mathfrak{C}}}(T)} \mathsf{Aut}_{\Sigma_{\mathfrak{C}}}(C)$
inductions.
\end{remark}



\subsection{Tame time}

\begin{definition}
	The \textit{colored tensor product} 
\[
\begin{tikzcd}[row sep = 0, column sep = 40pt]
	\mathsf{PreOp}^G \times \mathsf{sSet} \ar{r}{(-)\otimes_{\mathfrak{C}}(-)} &
	\mathsf{PreOp}^G
\end{tikzcd}
\]
is defined by $(X \otimes_{\mathfrak{C}} K)(T) = X(T) \times K$
whenever $T$ is a non-linear tree (equivalently, 
$\mathsf{Hom}_{\Omega}(T,\eta)=\emptyset$) and
is defined by the following pushout when $T=[n]$ is linear.
\[
\begin{tikzcd}
	X(\eta) \times K \ar{r} \ar{d} \arrow[dr, phantom, "\ulcorner", very near start]  &
	X(\eta) \ar{d}
\\
	X([n]) \times K \ar{r} & 
	(X \otimes_{\mathfrak{C}} K)([n]) 
\end{tikzcd}
\]
\end{definition}

\begin{remark}
More concisely, $X \otimes_{\mathfrak{C}} K$ is defined by the pushout
\[
\begin{tikzcd}
	\left(\mathsf{sk}_{\eta}X \right) \times K \ar{r} \ar{d} \arrow[dr, phantom, "\ulcorner", very near start]  &
	\mathsf{sk}_{\eta}X \ar{d}
\\
	X \times K \ar{r} & 
	X \otimes_{\mathfrak{C}} K 
\end{tikzcd}
\]
\end{remark}


\begin{remark}
For fixed $K \in \mathsf{sSet}$, the functor
$(-) \otimes_{\mathfrak{C}} K
\colon \mathsf{PreOp}^G \to \mathsf{PreOp}^G$
preserves all colimits (indeed, it is not hard to build the right adjoint explicitly).

However, for a fixed $X \in \mathsf{PreOp}^G$,
the functor 
$X \otimes_{\mathfrak{C}} (-)
\colon \mathsf{sSet} \to \mathsf{PreOp}^G$
does not preserve all colimits.
In particular, this functor can not preserve coproducts since, writing 
$\mathcal{C} = X(\eta)$ for the $G$-set of objects of $X$,
the image of $X \otimes_{\mathfrak{C}} (-)$ is entirely contained in the subcategory
$\mathsf{PreOp}^{G,\mathfrak{C}} \subset
\mathsf{PreOp}^G$
of preoperads with $G$-set of objects $\mathfrak{C}$ and maps which are the identity on objects. 
Instead, one has that the functor 
\[
X \otimes_{\mathfrak{C}} (-) \colon
\mathsf{sSet} \to \mathsf{PreOp}^{G,\mathfrak{C}}
\]
does preserve colimits. 
In practice, this means that some standard arguments concerning tensor products can only be applied after adjusting the objects of the relevant preoperads
({\color{red} see later}).
\end{remark}


\begin{remark}\label{COLORTENSGAM REM}
Let $X \to Y$ be any map in $\mathsf{PreOp}^G$
which is the identity on colors and 
$K \in \mathsf{sSet}$. Then the squares below are pushout squares.
Moreover, whenever $K$ is connected the rightmost horizontal maps are isomorphisms.
\[
\begin{tikzcd}
	X \times K \ar{r} \ar{d} 
	\arrow[dr, phantom, "\ulcorner", very near start] &
	\gamma_! \left( X \times K \right) \ar{r} \ar{d} 
	\arrow[dr, phantom, "\ulcorner", very near start] &
	X \otimes_{\mathfrak{C}} K \ar{d}
\\
	Y \times K \ar{r} &
	\gamma_! \left( Y \times K \right) \ar{r} &
	Y \otimes_{\mathfrak{C}} K
\end{tikzcd}
\]
\end{remark}


\begin{definition}
	Let $f \colon \mathfrak{C} \to \mathfrak{D}$
	be a map of $G$-sets (of colors).
	We define adjoint functors
\[
	f_{!} \colon
	\mathsf{PreOp}^{G,\mathfrak{C}}
\rightleftarrows
	\mathsf{PreOp}^{G,\mathfrak{D}}
	\colon f^{\**}
\]
via the pushout and pullback squares
(note that $\mathsf{sk}_{\eta} f_! A$ depends only on 
$\mathfrak{C}$ while 
$\mathsf{csk}_{\eta} f^{\**} X$ depends only on
$\mathfrak{D}$)
\[
\begin{tikzcd}
	\mathsf{sk}_{\eta} A \ar{r} \ar{d} \arrow[dr, phantom, "\ulcorner", very near start]  &
	\mathsf{sk}_{\eta} f_! A \ar{d}
&&
	f^{\**} X \ar{r} \ar{d} &
	X \ar{d}
\\
	A \ar{r} & 
	f_! A
&&
	\mathsf{csk}_{\eta} f^{\**} X \ar{r} & 
	\mathsf{csk}_{\eta} X
	\arrow[ul, phantom, "\lrcorner", very near start]
\end{tikzcd}
\]
\end{definition}


\begin{definition}
	A $G$-preoperad $X \in \mathsf{PreOp}^G$ is called a \textit{$G$-Segal operad} if, 
	for each $G$-tree $T$,
	the natural map 
	$X\left( \Omega[T] \right) \to 
	X \left( Sc[T] \right)$
	is a Kan equivalence.
\end{definition}

\begin{notation}
Given a $G$-Segal operad $X$ and $G$-corolla $C$, 
$X(\partial \Omega[C])$ is a discrete simplicial set whose elements
are the $C$-profiles $(c_1,\cdots,c_n;c_0)$ of $X$.
The map $X(\Omega[C]) \to X(\partial \Omega[C])$ hence yields
a coproduct decomposition 
\[
X(\Omega[C]) \simeq \coprod_{C\text{-profiles }(c_1,\cdots,c_n;c_0)}
X(c_1,\cdots,c_n;c_0)
\]
\end{notation}


\begin{remark}\label{SEOPDK REM}
Given a $G$-Segal operad $X$, consider a dendroidal Reedy fibrant replacement $X \to \tilde{X}$ such that $X(\eta) \simeq \tilde{X}(\eta)$. 
This means that all maps 
$X(\Omega[T]) \to \tilde{X}(\Omega[T])$ are Kan equivalences,
and moreover, by the following pullback diagram
\[
\begin{tikzcd}
	Z (Sc[T]) \ar{r} \ar{d} &
	\prod_{v \in \boldsymbol{V}_G(T)} Z
	(\Omega[T_v]) \ar{d}
\\
	\prod_{(G/H_i \cdot e_i) \in \boldsymbol{E}_G(T)} 
	\mathfrak{C}^{H_i} \ar{r}  &
	\prod_{v \in \boldsymbol{V}_G(T)}
	\prod_{(G/H_i \cdot e_i) \in \boldsymbol{E}_G(T_v)} 
	\mathfrak{C}^{H_i} 
	\arrow[ul, phantom, "\lrcorner", very near start]
\end{tikzcd}
\]
so are the maps $X(Sc[T]) \to \tilde{X}(Sc[T])$.
This shows that $\tilde{X}$ is also a Segal operad, 
and thus a fibrant object in $\mathsf{PreOp}^G$.

Furthermore, the Kan equivalences 
$X(\Omega[C]) \to \tilde{X}(\Omega[C])$
induce Kan equivalences 
$X(c_1,\cdots,c_n;c_0) \to \tilde{X}(c_1,\cdots,c_n;c_0)$.
It follows that the complete equivalences between Segal operads are precisely the Dwyer-Kan equivalences. 
\end{remark}


\begin{remark}\label{SLIMOD REM}
Noting that for every fibrant 
$\tilde{X} \in \mathsf{PreOp}^G$
any equivalence in $\tilde{X}$ is in the image of a map
$J \to \tilde{X}$, 
a slight modification of the proof of Lemma \ref{INTER_LEM}
shows that for any Segal operad $X$
any equivalence in $X$ is in the image of a countable, contractible
$I \in \mathsf{PreOp}^G$
such that $\eta \amalg \eta \to I$
is a tame cofibration.
\end{remark}




\begin{theorem}
	There is a model structure on 
	$\mathsf{PreOp}^G$,
	called the \textbf{tame model structure},
	such that:
\begin{itemize}
	\item the weak equivalences are the complete equivalences (i.e. detected by inclusion into 
	$\mathsf{sdSet}^G$);
	\item generating cofibrations are given by the maps
	\begin{itemize}
		\item[(TC1)] $G/H \cdot \left(\emptyset \to\Omega[\eta]\right)$ for $H\leq G$;
		\item[(TC2)] $\Omega[C] \otimes_{\mathfrak{C}} \left(\partial \Delta[n] \to \Delta[n]\right)$ for $C \in \Sigma_G$, $n \geq 0$;
		\item[(TC3)] 
$\left( Sc[T] \to \Omega[T] \right) 
\square_{\mathfrak{C}} 
\left(\partial \Delta[n] \to \Delta[n]\right)$ for $T \in \Omega_G$, $n \geq 0$.
	\end{itemize}
\end{itemize}
Furthermore, one has generating anodyne cofibrations the maps
\begin{itemize}
	\item[(TA1)] $G/H \cdot 
	\left(\Omega[\eta] \to I \right)$ for $H \leq G$,
	and $\Omega[\eta] \to I$ a weak equivalence in $\mathsf{PreOp}$ such that $I(\eta) = \{0,1\}$, $\Omega[\eta] \amalg \Omega[\eta] \to I$ is a tame cofibration, and $I$ is countable;
	\item[(TA2)] $\Omega[C] \otimes_{\mathfrak{C}}\left(\Lambda^i[n] \to \Delta[n]\right)$ for $C \in \Sigma_G$, $0 \leq i \leq n$;
	\item[(TA3)] 
$\left( Sc[T] \to \Omega[T] \right) 
\square_{\mathfrak{C}} 
\left(\partial \Delta[n] \to \Delta[n]\right)$ for $T \in \Omega_G$, $n \geq 0$.
	\end{itemize}
\end{theorem}


\begin{proof}
	The existence of the model structure will follow by applying J. Smith's theorem \cite[Thm. 1.7]{Bek00}. Conditions c0 and c2 therein are inherited from $\mathsf{sdSet}^G$
	and the technical ``solution set condition'' c3 follows from
	\cite[Prop. 1.15]{Bek00} since weak equivalences are accessible, being the preimage by $\gamma^{\**}$ if the weak equivalences in 
	$\mathsf{sdSet}^G$ 
	(see \cite[Cor. A.2.6.5]{Lur09} and \cite[Cor. A.2.6.6]{Lur09}).
	
	For c1, we must show that any map $X \to Y$ with the right lifting property against (TC1), (TC2), (TC3) is a weak equivalence.
	Writing $f \colon \mathfrak{C} \to \mathfrak{D}$ for the underlying map of colors,
	consider the factorization $X \to f^{\**}Y \to Y$.
	Note that since maps out of (TC1) depend only on objects and both of (TC2) and (TC3) consist of maps which are identities on objects,
	$X \to Y$ will have the right lifting property against (TC1) iff 
	$f^{\**} X \to Y$ does
	and the right lifting property against 
	(TC2) and (TC3) iff $X \to f^{\**}Y$ does.
	
Note now that $f^{\**} Y \to Y$ has the right lifting proper against all maps 
	$\left(\partial \Omega[T] \to \Omega[T] \right) \times \Delta[n]$.
	Indeed, if $T \simeq G/H \cdot \eta$ is a stick, this is precisely the lifting condition agains (TC1), and otherwise it follows automatically since $\left(\partial \Omega[T] \to \Omega[T] \right) \times \Delta[n]$ is the identity on objects.
	Therefore, the levels 
	$\left(f^{\**} Y \right)_n \to Y_n$ are trivial fibrations in 
	$\mathsf{dSet}^G$, showing that 
	$f^{\**} Y \to Y$ is a dendroidal equivalence, 
	and thus a complete equivalence. 
	
	Since the maps in both of (TC2) and (TC3) are the identify on objects, $X \to Y$ has the right lifting property against these maps iff $X \to f^{\**}Y$ does.
The lifting property against (TC2) then says that the maps
$X(\Omega[C]) \to f^{\**} Y (\Omega[C])$
are trivial Kan fibrations for all $G$-corollas $C \in \Sigma_G$,
and thus so are the maps
$X(Sc[T]) \to f^{\**} Y (Sc[T])$ for all $G$-trees $T \in \Omega_G$.
But it then follows from the lifting property against
(TC3) that the maps 
$X(\Omega[T]) \to f^{\**} Y (\Omega[T])$
are trivial Kan fibrations for all $G$-trees,
showing that $X \to f^{\**} Y$ is a simplicial equivalence, and thus a complete equivalence. 
\[
\begin{tikzcd}
	X(\Omega[T]) \ar{r} \ar[->>]{d}{\sim} &
	X(Sc[T]) \ar{r} \ar[->>]{d}{\sim} &
	\prod_{v \in \boldsymbol{V}_G(T)} X(\Omega[T_v])
	\ar[->>]{d}{\sim}
\\
	f^{\**} Y(\Omega[T]) \ar{r} &
	f^{\**} Y(Sc[T]) \ar{r} \ar{d} &
	\prod_{v \in \boldsymbol{V}_G(T)} f^{\**} Y
	(\Omega[T_v]) \ar{d}
	\arrow[ul, phantom, "\lrcorner", very near start]
\\
	&
	\prod_{(G/H_i \cdot e_i) \in \boldsymbol{E}_G(T)} 
	\mathfrak{C}^{H_i} \ar{r}  &
	\prod_{v \in \boldsymbol{V}_G(T)}
	\prod_{(G/H_i \cdot e_i) \in \boldsymbol{E}_G(T_v)} 
	\mathfrak{C}^{H_i} 
	\arrow[ul, phantom, "\lrcorner", very near start]
\end{tikzcd}
\]
This completes the proof of c1, establishing the existence of the tame model structure.

We now turn to the ``further'' claim considering the claimed generating anodyne cofibrations, i.e., 
we wish to show that the maps in 
(TA1), (TA2), (TA3) satisfy the conditions in
Lemma \ref{SEMICOF LEM}.

We first check condition (i).
The case of maps in (TC1) is tautological.
Since $\Lambda^{i}[n]$ is connected, 
the maps in (TA2) have the form
$\gamma_{!} 
\left( \Omega[C] \times
\left( \Lambda^i[n] \to \Delta[n] \right) \right)$,
and are thus weak equivalences thanks to the pushouts
in Remark \ref{COLORTENSGAM REM}.
As for (TA3), it follows from Remark \ref{COLORTENSGAM REM}
that the maps
$\left( Sc[T] \to \Omega[T] \right) \otimes \partial \Delta[n]$
and 
$\left( Sc[T] \to \Omega[T] \right) \otimes \Delta[n]$
are trivial cofibrations, so that the claim follows from a standard pushout and 2-out-of-3 argument.

We now turn to condition (ii).
The lifting condition against (TA3) says that $J$-fibrant objects are such that the maps $X(\Omega[T]) \to X(Sc[T])$
are trivial fibrations, and thus that such $X$ are Segal operads.
Therefore, by Remark \ref{SEOPDK REM} it suffices to check that $J$-fibrations between Segal operads which are also DK equivalences are in fact trivial fibrations, i.e. that they have the right lifting property against the maps in (TC1),(TC2),(TC3).
Given $X \to Y$ a $J$-fibration with $J$-fibrant $Y$,
the lifting property against (TC3) is tautological since 
(TC3) equals (TA3).
Next, the lifting property against (TA2) says that the maps
$X(\Omega[T]) \to f^{\**} Y(\Omega[T])$
are Kan fibrations, and the DK condition says that these are Kan equivalences,
so that we conclude that such maps have the right lifting property against (TC2).
Lastly, given any lifting problem against a map in (TC1),
essential surjectivity and Remark \ref{SLIMOD REM}
produce a lifting problem against a map in (TA1) which has a solution, providing a solution to the original problem.
\end{proof}


{\color{red} To show that maps in (TA3) are normal cofibrations one can use a pushout of projective cofibrant cubes argument.}


The following results are adapted from \cite{JT07} (see Proposition 7.15 therein). 


\begin{proposition}
	A cofibration $A \to B$ is a weak equivalence iff it has the left lifting property against all fibrations between fibrant objects.
\end{proposition}

\begin{proof}
	Let $B \xrightarrow{\sim} \tilde{B}$ be a fibrant replacement and
	let $A \xrightarrow{\sim} \tilde{A} \twoheadrightarrow \tilde{B}$
	be a factorization of the composite $A \to \tilde{B}$ 
	as a trivial cofibration followed by a fibration.
	One then has a lift in the diagram
\[
\begin{tikzcd}
	A \ar{r}{\sim} \ar[>->]{d} & \tilde{A} \ar[->>]{d}
\\
	B \ar{r}{\sim} \ar[dashed]{ru} & \tilde{B}
\end{tikzcd}
\]
where the top and bottom horizontal maps are weak equivalences. 
But then the 2-out-of-6 property for weak equivalences says that all maps are weak equivalences.
\end{proof}


\begin{corollary}\label{SIMPLQUILL COR}
An adjunction 
\[
F \colon \mathcal{C}
	\rightleftarrows
\mathcal{D} \colon G
\]
between model categories is a Quillen adjunction
provided that $F$ preserves cofibrations
and $G$ preserves fibrations between fibrant objects.
\end{corollary}


\begin{lemma}
	Let $A \to B$ be a tame cofibration in $\mathsf{PreOp}^G$, 
	$\mathcal{O} \in \mathsf{sOp}^G$ a $\Sigma$-cofibrant 
	$G$-operad,
	and consider a pushout diagram in $\mathsf{sOp}^G$ of the form
\[
\begin{tikzcd}
	\tau A \ar{r} \ar{d} & \mathcal{O} \ar{d}
\\
	\tau B \ar{r} & \mathcal{P}
\end{tikzcd}
\]
	Then $\mathcal{O} \to \mathcal{P}$ is a $\Sigma$-cofibration and 
\begin{equation}\label{UNITEQUIV EQ}
B \amalg_{A} N \mathcal{O}
	\to 
N \mathcal{P}
\end{equation}
is a weak equivalence.
\end{lemma}

\begin{proof}
	We first consider the case where $A\to B$ is in one of (TC1),(TC2),(TC3). 
	
	The (TC1) case is immediate, 
	since $\mathcal{O} \to \mathcal{O} \amalg G/H \cdot \Omega(\eta)$ is a $\Sigma$-cofibration and
	\eqref{UNITEQUIV EQ}
	is the isomorphism
	$N\mathcal{O} \amalg G/H\cdot \Omega[\eta] \simeq 
	N\left( \mathcal{O} \amalg G/H \cdot \Omega(\eta) \right)$.

	The (TC3) case is also straightforward:
	since $\tau A \to \tau B$ is an isomorphism, one can take 
	$\mathcal{O}=\mathcal{P}$, so that 
	\eqref{UNITEQUIV EQ} becomes a section of the map
	$N \mathcal{O} \to B \amalg_{A} N \mathcal{O}$, which is a trivial cofibration (it is a pushout of $A \to B$),
	and 2-out-of-3 hence implies that \eqref{UNITEQUIV EQ} is a weak equivalence.

	The most interesting case is then (TC2), 
	in which case it is well known that 
	$\mathcal{O} \to \mathcal{P}$ is a $\Sigma$-cofibration and
	each of the levels
$(B \amalg_{A} N \mathcal{O})_n
	\to 
(N \mathcal{P})_n$
for $n \geq 0$
is an equivalence in $\mathsf{dSet}^G$ by (an iteration of)
Proposition \ref{KEYPR PROP}, 
showing that \eqref{UNITEQUIV EQ} is in fact a dendroidal equivalence, and thus also a complete equivalence.

	We now turn to the case of $A \to B$ a general cofibration between cofibrant objects.
	As usual, $A \to B$ is a retract of a transfinite composition of pushouts of generating cofibrations.
	Since the conclusions of the result are invariant under retracts,
	we are free to assume that $A \to B$ is a transfinite composite
\[
A = A_0 \to A_1 \to A_2 \to \cdots \to A_{\beta} \to 
colim_{\beta < \kappa} A_{\beta} = B.
\]
where each map $A_{\beta} \to A_{\beta +1}$ is a pushout of a map in one of (TC1),(TC2),(TC3).

Defining $\mathcal{O}_{\beta}$ by replacing $A \to B$ with $A \to A_{\beta}$ in the pushout,
$\mathcal{O} \to \mathcal{P}$ becomes the transfinite composite of the maps $\mathcal{O}_{\beta} \to \mathcal{O}_{\beta + 1}$
and \eqref{UNITEQUIV EQ} becomes
$
colim_{\beta < \kappa} \left( 
N \mathcal{O} \amalg_{N \tau A} N \tau A_{\beta}
	\to 
N \mathcal{O}_{\beta}
\right)
$.
It thus suffices to show, by induction on $\beta < \kappa$, 
that the maps $\mathcal{O}_{\beta} \to \mathcal{O}_{\beta + 1}$ are $\Sigma$-cofibrations and that the maps 
$N \mathcal{O} \amalg_{N \tau A} N \tau A_{\beta}
	\to 
N \mathcal{O}_{\beta}$
are weak equivalences
(that this last condition suffices follows since
filtered colimits of weak equivalences in $\mathsf{PreOp}^G$ are weak equivalences ({\color{red} add this})).
Consider now the following diagrams.
\[
\begin{tikzcd}
	\tau A \ar{r} \ar{d} & \mathcal{O} \ar{d}
&&
	A_{\beta} \amalg_{A} N \mathcal{O}
	\ar[>->]{r} \ar{d}[swap]{\sim} &
	A_{\beta+1} \amalg_{A} N \mathcal{O}
	\ar{d}[swap]{\sim}
\\
	\tau A_{\beta} \ar{r} \ar{d} & \mathcal{O}_{\beta} \ar{d}
&&
	N \mathcal{O}_{\beta} \ar[>->]{r} &
	A_{\beta+1} \amalg_{A_{\beta}} N \mathcal{O}_{\beta} \ar{d}
\\
	\tau A_{\beta + 1} \ar{r} & \mathcal{O}_{\beta + 1}
&&
	&
	N \mathcal{O}_{\beta+1}
\end{tikzcd}
\]
The induction hypothesis states that
$\mathcal{O} \to \mathcal{O}_{\beta}$ is a $\Sigma$-cofibration and that the map
$A_{\beta} \amalg_A N \mathcal{O} \to \mathcal{O}_{\beta}$ is a weak equivalence.
Therefore, $\mathcal{O}_{\beta}$ is $\Sigma$-cofibrant 
and the both vertical maps marked $\sim$ in the rightmost diagram above are weak equivalences 
(this uses the fact that $\mathsf{PreOp}^G$ is left proper),
and thus the induction step will follow provided that the result holds for
the map $A_{\beta} \to A_{\beta + 1}$ and $\mathcal{O}_{\beta}$.
But $A_{\beta} \to A_{\beta + 1}$ is assumed to be a pushout of a map in (TC1),(TC2),(TC3), in which case the result is already known, and thus noting that the result is clearly invariant under pushouts finishes the proof.
\end{proof}

Setting $A = \emptyset $, $\mathcal{O}= \emptyset$ in the previous result yields the following.

\begin{corollary}\label{KEYEQUIV COR}
	If $B \in \mathsf{PreOp}^G$ is tame cofibrant, then 
	$B \to N \tau B$ is a weak equivalence.
\end{corollary}

\begin{proposition}
The adjunction
\[
	\tau \colon \mathsf{PreOp}^G_{\text{tame}}
		\rightleftarrows 
	\mathsf{sOp}^G \colon N
\]
is a Quillen equivalence.
\end{proposition}


\begin{proof}
Firstly, note that $N$ preserves and detects weak equivalences.
Indeed, this follows since all objects in the image of $N$ are Segal operads, so that by Remark \ref{SEOPDK REM} a map in the image of $N$ is a weak equivalence iff it is a Dwyer-Kan equivalence.

Next, we show that this is a Quillen adjunction using Corollary \ref{SIMPLQUILL COR}.
The claim that $\tau$ preserves cofibrations follows since
$\tau$ sends the maps in (TC1) and (TC2) to generating cofibrations of $\mathsf{sOp}^G$ and the maps in (TC3) to isomorphisms.
For the claim that $N$ preserves fibrations between fibrant,
we use a somewhat indirect argument
(though we note that a direct argument is also possible,
by showing that fibrations between fibrant objects in $\mathsf{PreOp}^G$
also satisfy a ``local fibration plus isofibration'' description).
By Corollary \ref{KEYEQUIV COR} and 2-out-of-3, 
one has that for any trivial cofibration between cofibrant objects
$A \to B$, the map $N \tau A \to N \tau B$ is a weak equivalence, and thus so is $\tau A \to \tau B$.
This shows that $\tau$ sends all maps in (TA1),(TA2),(TA3)
to trivial cofibrations, and since these maps detect fibrations between fibrant objects in $\mathsf{PreOp}^G$, 
the standard adjunction argument shows that 
$N$ indeed preserves fibrations between fibrant objects.

For the Quillen equivalence claim, 
let $B \in \mathsf{PreOp}^G$ be tame cofibrant and
$\mathcal{O} \in \mathsf{sOp}^G$ be fibrant.
We must show that the leftmost map below is a weak equivalence iff its adjoint, which is the rightmost composite, is.
\[
	\tau B \to \mathcal{O},
\qquad
	B \xrightarrow{\sim} N \tau B \to N \mathcal{O}
\]
This is immediate from Corollary \ref{KEYEQUIV COR}
and the fact that $N$ preserves and detects weak equivalences.
\end{proof}


\begin{proposition}
	The adjunction 
$W_! \colon \mathsf{dSet}^G 
	\rightleftarrows 
\mathsf{sOp}^G \colon hcN$
	is a Quillen adjunction.
\end{proposition}

{\color{red} HERE}

\begin{proof}
	We again apply Corollary \ref{SIMPLQUILL COR}.
	For the claim that $W_!$ preserves cofibrations,
	it suffices to show this for the generating cofibrations
	$G\cdot_H \left( \partial \Omega[U] \to \Omega[U] \right)$ for $U \in \Omega^H$.
	But this follows since 
	$G \cdot_H \left(W_! \partial \Omega[U] \to W_! \Omega[U] \right)$
	is a pushout of the map
\[
	G \cdot_H \Omega[U - \boldsymbol{E}^{\mathsf{i}}(U)]
\otimes
	\left(
	\partial \left( \Delta[1]^{\times \boldsymbol{E}^{\mathsf{i}}(U) } \right) 
		\to
	\Delta[1]^{\times \boldsymbol{E}^{\mathsf{i}}(U) }
	\right).
\]
Similarly, the map
	$G \cdot_H \left(W_! \Lambda^E[U] \to W_! \Omega[U] \right)$
is a pushout 

%\[
%\begin{tikzcd}
%	G \cdot_H 
%	\Omega[U - \boldsymbol{E}^{\mathsf{i}}(U)] \otimes 
%	\partial \left( \Delta[1]^{\times \boldsymbol{E}^{\mathsf{i}}(U) } \right) 
%	\ar{r} \ar{d} &
%	G \cdot_H W_! \partial \Omega[U] \ar{d}
%\\
%	G \cdot_H 
%	\Omega[U - \boldsymbol{E}^{\mathsf{i}}(U)] \otimes 
%	\Delta[1]^{\times \boldsymbol{E}^{\mathsf{i}}(U) } \ar{r}	&
%	G \cdot_H W_! \Omega[U]
%\end{tikzcd}
%\]

{\color{red} HERE}
	
\end{proof}




\section{Coloring via spans}


\subsection{Preliminaries}


Recall \cite[Prop. 2.7]{BP17}
that if
$\pi \colon \mathcal{E} \to \mathcal{B}$ is 
a split Grothendieck fibration, 
then so is the map of functor categories
$\mathcal{E}^{\mathcal{C}} \to \mathcal{B}^{\mathcal{C}}$
for any category $\mathcal{C}$.
Explicitly, given functors $E \colon \mathcal{C} \to \mathcal{E}$,
$B',B \colon \mathcal{C} \to \mathcal{B}$
such that $B=\pi \circ E$
and a natural transformation
$\varphi \colon B' \Rightarrow B$,
the pullback functor
$\varphi^{\**} E \colon \mathcal{C} \to \mathcal{E}$
is described on objects by
$\left(\varphi^{\**} E\right) (c) =
\left(\varphi(c)\right)^{\**} (E(c))$
and on arrows $f\colon c \to \bar{c}$
as the unique dashed arrow in the leftmost diagram below which makes that diagram commute and lifts $B'(f)$. 
\[
\begin{tikzcd}
	\varphi^{\**}E(c) \ar{r} 
	\ar[dashed]{d}[swap]{\varphi^{\**}E (f)} &
	E(c) \ar{d}{E(f)}
&&
	B'(c) \ar{d}[swap]{B'(f)} \ar{r} &
	B(c) \ar{d}{B(f)}
\\
	\varphi^{\**}E(\bar{c}) \ar{r} &
	E(\bar{c}) 
&&
	B'(\bar{c}) \ar{r} &
	B(\bar{c})
\end{tikzcd}
\]
Given a split Grothendieck fibration 
$\pi^{\**} \colon \mathcal{E} \to \mathcal{B}$,
we now define a pullback functor on weak right spans
\begin{equation}\label{WSPANPULL EQ}
\pi^{\**} \colon
\mathsf{WSpan}^r(\**,\mathcal{B})
	\to
\mathsf{WSpan}^r(\**,\mathcal{E})
\end{equation}
as follows.

On objects, i.e. functors $B\colon \mathcal{C} \to \mathcal{B}$, one simply sets 
$\pi^{\**}(\colon \mathcal{C} \to \mathcal{B})=
(\mathcal{C} \times_{\mathcal{B}} \mathcal{E}
\to \mathcal{E})
$.
On arrows, which are pairs 
$(i,\varphi \colon B_2 \circ i \Rightarrow B_1)$
as in the diagram below
\begin{equation}
\begin{tikzcd}[row sep = tiny, column sep = 35pt]
	\mathcal{C}_1 \arrow{dr}[name=U]{B_1} \arrow{dd}[swap]{i}
\\
	& \mathcal{B}
\\
	|[alias=V]| \mathcal{C}_2 \arrow{ur}[swap]{B_2}
\arrow[Rightarrow, from=V, to=U,shorten >=0.25cm,shorten <=0.25cm
,swap,"\varphi"
]
\end{tikzcd}
\end{equation}
and writing 
$\pi \colon \mathcal{C}_i \times_{\mathcal{B}} \mathcal{E}
\to \mathcal{C}_i$
and
$E_i \colon \mathcal{C}_i \times_{\mathcal{B}} \mathcal{E}
\to \mathcal{E}$
for the projections, one sets
\[
\pi^{\**} (i,\varphi)=
\left(
	\left( i \pi,
	\left( \varphi \pi \right)^{\**} E_1 \right),
	\left( \varphi \pi \right)^{\**} E_1 \Rightarrow E_1
\right)
\]
or, put another way, 
$\pi^{\**}(i,\phi)$
is characterized as the unique choice of dashed data in the following diagram
\[
\begin{tikzcd}[column sep = small, row sep = small]
	\mathcal{C}_1 \times_{\mathcal{B}} \mathcal{E} 
	\ar{rrrrr}[name=toE]{E_1} \ar[dashed]{rd} \ar{dd}
	&&&
	&&
	\mathcal{E}  \ar{dd}
\\
	&
	|[alias=DBE]|
	\mathcal{C}_2 \times_{\mathcal{B}} \mathcal{E} \ar{rrrru}[swap]{E_2}
\\
	\mathcal{C}_1 \ar{rrrrr}[name=toB]{B_1} \ar{rd} 
	&&&
	&&
	\mathcal{B} 
\\
	&
	|[alias=D]| \mathcal{C}_2 \ar{rrrru}[swap]{B_2}
\arrow[Rightarrow, from=DBE, to=toE, shorten <=0.15cm,shorten >=0.15cm,dashed
%,swap,"\pi_i"
]
	\arrow[Rightarrow, from=D, to=toB, shorten <=0.15cm,shorten >=0.15cm,swap,"\varphi"]
	\arrow[from=DBE, to=D, crossing over]
\end{tikzcd}
\]
such that the side faces commute, the top natural transformation consists of pushout arrows for $\pi \colon \mathcal{E} \to \mathcal{B}$, and the total diagram commutes, in the sense that the two composite natural transformations $B_2 i \pi \Rightarrow \pi E_1$ coincide.
Associativity and unitality of $\pi^{\**}$ are straightforward.



\begin{remark}\label{SPANLIM REM}
Given a diagram 
$J \xrightarrow{j \mapsto (B_j \colon \mathcal{C}_j \to \mathcal{B})}
\mathsf{WSpan}^r(\**,\mathcal{B})$
and a cone over it, 
i.e. an object
$(B \colon \mathcal{C} \to \mathcal{B}) \in \mathsf{WSpan}^r(\**,\mathcal{B})$
together with compatible maps 
$(i_j,\varphi_j) \colon B \to B_j$,
this will be a limit in 
$\mathsf{WSpan}^r(\**,\mathcal{B})$
provided that
\[
	\mathcal{C} = \lim_{j \in J} \mathcal{C}_j
\qquad
	B = colim_{j \in J} B_j i_j
\]
where the limit takes place in 
$\mathsf{Cat}$
and the colimit in 
$\mathsf{Fun}(\mathcal{C}, \mathcal{B})$.
\end{remark}


\subsection{Edge functors}

\begin{notation}
An edge $(i,\varphi)\colon B_1 \to B_2$ in $\mathsf{WSpan}^r(\**,\mathcal{B})$
will be called a 
\textit{$1$-arrow}
if $\varphi = id_{B_1}$,
i.e. if the arrow exhibits the commutative diagram
$B_1 = B_2 i$.
\end{notation}


Recall the identification
\[
\Omega^t \simeq |\Omega^{\bullet}|
\]
of the category $\Omega^t$ of trees and tall maps with the realization of the simplicial object in categories with $n$-th level the planar $n$-strings $\Omega^{n}$.
It then follows that the target functors
$\Omega^n \to \Omega^t$
given by 
$(T_0 \to T_1 \to \cdots \to T_n)
\mapsto T_n$
define a simplicial object in 
$\mathsf{WSpan}^r(\**,\Omega^t)$
where all simplicial operators other than the top faces $d_n$ are $1$-arrows in
$\mathsf{WSpan}^r(\**,\Omega^t)$.

Furthermore, letting 
$\boldsymbol{E} \colon \Omega^{t} \to \mathsf{F}$
be the edge functor, sending each tree to the underlying set of edges, 
one obtains a simplicial object in
$\mathsf{WSpan}^r(\**,\mathsf{F})$,
whose levels we will denote by
$\boldsymbol{E}\colon \Omega^n \to \mathsf{F}$.
Note that by definition
$\boldsymbol{E}(T_0 \to T_1 \to \cdots \to T_n)=
\boldsymbol{E}(T_n)$.
Next, define a functor
\begin{equation}\label{SIGMAFUN EQ}
\begin{tikzcd}[row sep = 0]
	\mathsf{WSpan}^r(\**,\mathsf{F}) \ar{r}{\Sigma \wr (-)} &
	\mathsf{WSpan}^r(\**,\mathsf{F})
\\
	\mathcal{C} \to \mathsf{Fin}
	\ar[mapsto]{r} &
	\Sigma \wr \mathcal{C} \to
	\Sigma \wr \mathsf{Fin} \xrightarrow{\amalg}
	\mathsf{Fin}
\end{tikzcd}
\end{equation}
We have now extended the categories
$\Omega^n$ and $\Sigma \wr \Omega^n$
and the simplicial arrows $d_i$, $s_i$
to objects and arrows in 
$\mathsf{WSpan}^r(\**,\mathsf{F})$.
We next discuss how to likewise extend the vertex functors
$V \colon \Omega^n \to \Sigma \wr \Omega^{n-1}$.
The key case is that of $n=1$, in which case we have a natural transformation
\[
\begin{tikzcd}
	\Omega^1 \ar{rr}{\boldsymbol{E}} \ar{d}[swap]{\boldsymbol{V}}&&
	|[alias=Ft]|
	\mathsf{F} \ar[equal]{d}
\\
	|[alias=SOm]|
	\Sigma \wr \Omega^0 \ar{r}[swap]{\boldsymbol{E}} &
	\Sigma \wr \mathsf{F} \ar{r}[swap]{\amalg} &
	\mathsf{F}
\arrow[Rightarrow, from=SOm, to=Ft,shorten >=0.25cm,shorten <=0.25cm
%,swap,"\varphi"
]
\end{tikzcd}
\]
which, for each $T_0 \to T_1$ is the obvious map
$\coprod_{v \in \boldsymbol{V}(T_0)}
\boldsymbol{E}(T_{1,v}) \to
\boldsymbol{E}(T_1)$
induced by the maps
$T_{1,v} \to T_1$. 
For $n \neq 1$, one can either mimic this formula or, alternatively,
note that if the diagrams 
\[
\begin{tikzcd}
	\Omega^n \ar{r}{\boldsymbol{V}} 
	\ar{d}[swap]{d^{1,\cdots,n-1}} &
	\Sigma \wr \Omega^{n-1}
	\ar{d}{d^{0,\cdots,n-2}}
&&
	\Omega^0 \ar{r}{\boldsymbol{V}} 
	\ar{d}[swap]{s^0} &
	\Sigma \wr \Sigma
	\ar{d}{s^{-1}}
\\
	\Omega^1 \ar{r}{\boldsymbol{V}} &
	\Sigma \wr \Omega^0
&&
	\Omega^1 \ar{r}{\boldsymbol{V}} &
	\Sigma \wr \Omega^0
\end{tikzcd}
\]
are to remain commutative when regarded as diagrams in $\mathsf{WSpan}^r(\**,\mathsf{F})$, 
the fact that the vertical arrows are $1$-arrows
implies that the natural transformation component of the bottom $\boldsymbol{V}$ functors immediately determines the natural transformation compoment of the top $\boldsymbol{V}$ functors.

To simply notation, in what follows we will represent diagrams in
$\mathsf{WSpan}^r(\**,\mathsf{F})$ simply by their category component.

\begin{proposition}
Let $1 \leq i \leq n$. Then the squares 
\[
\begin{tikzcd}
	\Omega^n \ar{r}{\boldsymbol{V}} 
	\ar{d}[swap]{d^{i}} &
	\Sigma \wr \Omega^{n-1}
	\ar{d}{d^{i-1}}
\\
	\Omega^{n-1} \ar{r}{\boldsymbol{V}} &
	\Sigma \wr \Omega^{n-2}
\end{tikzcd}
\]
are pullback squares when regarded as squares in 
$\mathsf{WSpan}^r(\**,\mathsf{F})$.
\end{proposition}

\begin{proof}
We need to check the conditions in 
Remark \label{SPANLIM REM}.
The fact that the underlying diagrams of categories are pullback diagrams is already known.
Furthermore, when $i<n$ the vertical arrows are $1$-arrows and it is hence obvious that square of functors $\Omega^n \to \mathsf{F}$ is a pullback square (it is a square where two opposite sides are identities).
For the case $i=n$, by using combining the simplicial identities and the $i<n$ case, one reduces to checking the case $i=n=1$.
The required claim is then that for each $1$-string
$T_0 \to T_1$ there is a natural isomorphism
\[
	\boldsymbol{E}(T_1) \simeq
	colim\left(
	\boldsymbol{E}(T_0) \leftarrow 
	\coprod_{v \in \boldsymbol{V}(T_0)} \boldsymbol{E}(T_{0,v}) \to 
	\coprod_{v \in \boldsymbol{V}(T_0)} \boldsymbol{E}(T_{1,v})
	\right)
\]
and this is clear from the discussion of substitution data.
\end{proof}


\subsection{Vertex functors}

In the previous section we discussed how to equip the categories $\Omega^n$,
and the functors 
$d^i \colon \Omega^n \to \Omega^{n-1}$,
$s^j \colon \Omega^n \to \Omega^{n+1}$,
$V \colon \Omega^n \to \Sigma \wr \Omega^{n-1}$
with ``edge data'' by extending them to 
spans over $\mathsf{Fin}$.
In this section we discuss a variant which extends these categories and functors (except for the top face maps $d^n$) to spans encoding ``vertex data''.

Firstly, we discuss some preliminaries.
Given a split Grothendieck fibration
$\mathcal{E} \to \mathcal{B}$ we write
$\mathsf{WSpan}^r(\**,\mathcal{E} \to \mathcal{B})
\subseteq
\mathsf{WSpan}^r(\**,\mathcal{E})$
for the subcategory with the same objects but with arrows only those natural transformations whose constituent arrows are pullbacks for $\mathcal{E} \to \mathcal{B}$.

We will now extend the categories $\Omega^n$,
and arrows 
$d^i$ (for $i<n$),
$s^j$ and $V$
to objects and arrows in
$\mathsf{WSpan}^r(\**,\Sigma \wr \Sigma \to \Sigma)$,
where $\Sigma\wr \Sigma \to \Sigma$
is the natural split Grothendieck construction.

Adapting \eqref{SIGMAFUN EQ} we now have a functor
\begin{equation}\label{SIGMAFUN2 EQ}
\begin{tikzcd}[row sep = 0]
	\mathsf{WSpan}^r(\**,\Sigma\wr \Sigma \to \Sigma) \ar{r}{\Sigma \wr (-)} &
	\mathsf{WSpan}^r(\**,\Sigma\wr \Sigma \to \Sigma)
\\
	\mathcal{C} \to \Sigma\wr \Sigma
	\ar[mapsto]{r} &
	\Sigma \wr \mathcal{C} \to
	\Sigma \wr \Sigma \wr \Sigma \xrightarrow{\amalg}
	\Sigma \wr \Sigma
\end{tikzcd}
\end{equation}
Combining this $\Sigma \wr (-)$ functor with the vertex functors
$V \colon \Omega^n \to \Sigma \wr \Omega^{n-1}$
one obtains, by induction on $n$,
functors
$V^n \colon \Omega^n \to \Sigma \wr \Sigma$.
In other words, one has extensions of the categories 
$\Omega^n$, $\Sigma \wr \Omega^n$ to objects in 
$\mathsf{WSpan}^r(\**,\Sigma \wr \Sigma \to \Sigma)$
in such a way that the vertex functors
$V \colon \Omega^n \to \Sigma \wr \Omega^{n-1}$
become $1$-arrows of $\mathsf{WSpan}^r(\**,\Sigma \wr \Sigma \to \Sigma)$.

Next, we discuss how to extend the simplicial operators. Firstly, for $d^0 \colon \Omega^1 \to \Omega^0$, this is given by the permutation isomorphism below.
\begin{equation}\label{1STPI EQ}
\begin{tikzcd}
	\Omega^1 \ar{rr}{d^0} \ar{d}[swap]{\boldsymbol{V}}&&
	|[alias=Ft]|
	\Omega^0 \ar{d}{\boldsymbol{V}}
\\
	|[alias=SOm]|
	\Sigma \wr \Omega^0 \ar{r}[swap]{\boldsymbol{V}} &
	\Sigma \wr \Sigma \wr \Sigma \ar{r}[swap]{\amalg} &
	\Sigma \wr \Sigma
\arrow[Leftrightarrow, from=SOm, to=Ft,shorten >=0.25cm,shorten <=0.25cm
,swap,"\pi"
]
\end{tikzcd}
\end{equation}
Next, to define 
$d^0 \colon \Omega^n \to \Omega^{n-1}$ for a general $n\geq 1$,
note that since
$\Sigma \wr d^{0,\cdots,n}
\colon \Sigma \wr \Omega^{n} \to \Sigma \wr \Sigma$
is a map of split Grothendieck fibrations over $\Sigma$,
\eqref{WSPANPULL EQ}
generalizes to give a pullback functor
\begin{equation}
\left( \Sigma \wr d^{0,\cdots,n} \right)^{\**} \colon
\mathsf{WSpan}^r(\**,\Sigma \wr \Sigma \to \Sigma)
	\to
\mathsf{WSpan}^r(\**,\Sigma \wr \Omega^{n} \to \Sigma)
\end{equation}
so that the desired natural transformation for
$d^0 \colon \Omega^n \to \Omega^{n-1}$
is given by the diagram
\begin{equation}
\begin{tikzcd}
	\Omega^n \ar{rr}{d^0} \ar{d}[swap]{\boldsymbol{V}}&&
	|[alias=Ft]|
	\Omega^{n-1} \ar{d}{\boldsymbol{V}}
\\
	|[alias=SOm]|
	\Sigma \wr \Omega^{n-1} \ar{r}[swap]{\boldsymbol{V}} &
	\Sigma \wr \Sigma \wr \Omega^{n-2} \ar{r}[swap]{\amalg} &
	\Sigma \wr \Omega^{n-2} \ar{r}[swap]{\Sigma \wr V^{n-2}} &
	\Sigma \wr \Sigma \wr \Sigma \ar{r}[swap]{\amalg} &
	\Sigma \wr \Sigma
\arrow[Leftrightarrow, from=SOm, to=Ft,shorten >=0.25cm,shorten <=0.25cm
,swap,"\pi"
]
\end{tikzcd}
\end{equation}
where the leftmost section is obtained by applying
$\left(\Sigma \wr d^{0,\cdots,n} \right)^{\**}$
to \eqref{1STPI EQ}.
Lastly,
since the vertex functors $\boldsymbol{V}$ are $1$-arrows,
the operators 
$d^i \colon \Omega^n \to \Omega^{n-1}$
for $0<i<n$
(and similarly the degeneracies $s^j$)
are determined by demanding that the squares
\begin{equation}\label{RMCOM EQ}
\begin{tikzcd}
	\Omega^n \ar{r}{\boldsymbol{V}} 
	\ar{d}[swap]{d^{i}} &
	\Sigma \wr \Omega^{n-1}
	\ar{d}{d^{i-1}}
\\
	\Omega^{n-1} \ar{r}{\boldsymbol{V}} &
	\Sigma \wr \Omega^{n-2}
\end{tikzcd}
\end{equation}
remain commutative when regarded as diagrams in
$\mathsf{WSpan}^r(\**,\Sigma \wr \Sigma \to \Sigma)$.
To check the simplicial identities,
the identity $d^0 d^1 = d^0 d^0$ for $n=2$ is the equality of the usual diagrams
\begin{equation}
\begin{tikzcd}
	\Omega^2 \ar{d}[swap]{d^0} \ar{r}{\boldsymbol{V}}&
	\Sigma \wr \Omega^1 \ar{r}{\boldsymbol{V}} &
	|[alias=Ft1]|
	\Sigma \wr \Sigma \wr \Omega^0 \ar{d}{\amalg} \ar{r}{\boldsymbol{V}} &
	\Sigma \wr \Sigma \wr \Sigma \wr \Sigma
	\ar{d}{\amalg}
\\
	|[alias=SOm1]|
	\Omega^1 \ar{rr}{\boldsymbol{V}} \ar{d}[swap]{d^0} &&
	\Sigma \wr \Omega^0 \ar{r}{\boldsymbol{V}} &
	|[alias=Ft]|
	\Sigma \wr \Sigma \wr \Sigma
	\ar{d}{\amalg}
\\
	|[alias=SOm]|
	\Omega^0 \ar{rrr}[swap]{\boldsymbol{V}} &&&
	\Sigma \wr \Sigma
\arrow[Leftrightarrow, from=SOm, to=Ft,shorten >=0.25cm,shorten <=0.25cm
,swap,"\pi"
]
\arrow[Leftrightarrow, from=SOm1, to=Ft1,shorten >=0.25cm,shorten <=0.25cm
,swap,"\pi"
]
\end{tikzcd}
\end{equation}
and
\begin{equation}
\begin{tikzcd}
	\Omega^2 \ar{d}[swap]{d^1} \ar{r}{\boldsymbol{V}} &
	\Sigma \wr \Omega^1 \ar{r}{\boldsymbol{V}} \ar{d}[swap]{d^0} &
	\Sigma \wr \Sigma \wr \Omega^0 \ar{r}{\boldsymbol{V}} &
	|[alias=Ft1]|
	\Sigma \wr \Sigma \wr \Sigma \wr \Sigma
	\ar{d}{\amalg}
\\
	\Omega^1 \ar{r}{\boldsymbol{V}} \ar{d}[swap]{d^0} &
	|[alias=SOm1]|
	\Sigma \wr \Omega^0 \ar{rr}{\boldsymbol{V}} &&
	|[alias=Ft]|
	\Sigma \wr \Sigma \wr \Sigma
	\ar{d}{\amalg}
\\
	|[alias=SOm]|
	\Omega^0 \ar{rrr}[swap]{\boldsymbol{V}} &&&
	\Sigma \wr \Sigma
\arrow[Leftrightarrow, from=SOm, to=Ft,shorten >=0.25cm,shorten <=0.25cm
,swap,"\pi"
]
\arrow[Leftrightarrow, from=SOm1, to=Ft1,shorten >=0.25cm,shorten <=0.25cm
,swap,"\pi"
]
\end{tikzcd}
\end{equation}
while $d^0 d^1 = d^0 d^0$ equalities for other $n$
then follow by using the 
$\left(\Sigma \wr d^{0,\cdots,n}\right)^{\**}$
functors.
Other simplicial identities involving $d^0$ then follow inductively from these identities ({\color{red} flesh this out}) 
and, lastly, the simplicial identities not involving $d^0$ follow directly by induction and the squares \eqref{RMCOM EQ}.

{\color{red} HERE}





\bibliography{biblio}{}





\bibliographystyle{alpha}



\end{document}