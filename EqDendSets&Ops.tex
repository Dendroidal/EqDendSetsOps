\documentclass[a4paper,10pt
,draft
]{article}%

\usepackage[hidelinks]{hyperref}
\hypersetup{
  % colorlinks,
  final,
  pdftitle={Equivariant Dendroidal Segal Spaces},
  pdfauthor={Bonventre, P. and Pereira, L. A.},
  % pdfsubject={Your subject here},
  % pdfkeywords={keyword1, keyword2},
  linktoc=page
}
\usepackage[open=false]{bookmark}

\input{commands.tex}%


%-------- Tikz ---------------------------

\usepackage{tikz}%
\usetikzlibrary{matrix,arrows,decorations.pathmorphing,
cd,patterns,calc}
\tikzset{%
  treenode/.style = {shape=rectangle, rounded corners,%
                     draw, align=center,%
                     top color=white, bottom color=blue!20},%
  root/.style     = {treenode, font=\Large, bottom color=red!30},%
  env/.style      = {treenode, font=\ttfamily\normalsize},%
  dummy/.style    = {circle,draw,inner sep=0pt,minimum size=2mm}%
}%

\usetikzlibrary[decorations.pathreplacing]




% -------- Commands on draft -------------

\usepackage{ifdraft}
\ifdraft{
  \color[RGB]{63,63,63}
  % \pagecolor[rgb]{0.5,0.5,0.5}
  \pagecolor[RGB]{220,220,204}
  % \color[rgb]{1,1,1}
}


\usepackage[draft]{showkeys}
\usepackage{todonotes}%[obeyDraft]


% ----- What Labels Changed? --------------------------------
\makeatletter

\def\@testdef #1#2#3{%
  \def\reserved@a{#3}\expandafter \ifx \csname #1@#2\endcsname
  \reserved@a  \else
  \typeout{^^Jlabel #2 changed:^^J%
    \meaning\reserved@a^^J%
    \expandafter\meaning\csname #1@#2\endcsname^^J}%
  \@tempswatrue \fi}

\makeatother


% ------- Commands --------------------
\newcommand{\mycircled}[2][none]{%
  \mathbin{
    \tikz[baseline=(a.base)]\node[draw,circle,inner sep=-1.5pt, outer sep=0pt,fill=#1](a){\ensuremath #2\strut};
  }
}
\newcommand{\owr}{\mycircled{\wr}}
\newcommand{\UV}{\underline{\mathcal V}}
\renewcommand{\phi}{\varphi}
\newcommand{\UC}{\underline{\mathfrak C}}

\renewcommand{\F}{\mathcal F}
\newcommand{\I}{\mathbb I}
\newcommand{\J}{\mathbb J}
\renewcommand{\1}{\ensuremath{\mathbb{id}}}


% ------ Personal Info ---------------
\title{Colored Genuine Operads}%

\author{Peter Bonventre and Lu\'is Pereria}

\date{\today}





% ------ Document body -----------

\begin{document}

\maketitle

\begin{abstract}
      Things and stuff
\end{abstract}

\tableofcontents


\section{Overview}


\[
	\begin{tikzcd}
		\mathsf{PreOp}^G
\\
		\mathsf{sdSet}^G \ar{r}[swap]{(-)_0} \ar{u}{\gamma_{\**}} &
		\mathsf{dSet}^G
	\end{tikzcd}
\]


\section{Preliminaries}

Recall the notation $\mathsf F \wr \mathcal C$ for a category $\mathcal C$.

Similarly, consider the Grothendieck construction on the functor
\begin{equation}
      \label{FG_GR_EQ}
      \begin{tikzcd}[row sep = tiny]
            \mathsf F^{G,op} \arrow[r]
            &
            \mathsf{Set}
            \\
            A \arrow[r, mapsto]
            &
            \Set^{O_G^{op}}(\Phi(A), \underline{\mathfrak C}),
      \end{tikzcd}
\end{equation}
where $\Phi: \Set^G \to \Set^{O_G^{op}}$ sends a $G$-set $X$ to its fixed-point system $G/H \mapsto X^H$.
We will denote this by $\mathsf F^G \wr \underline{\mathfrak C}$.

\section{Colored Operads}


\subsection{Non-Equivariant Colored Operads}

Fix a closed symmetric monoidal category $\V$.

\begin{definition}
      Fix a set $\mathfrak C$ of \textit{colors}.
      A tuple
      $\ksi = (c_1, \ldots, c_n; c_0) \in \mathfrak C^{\times n} \times \mathfrak C$
      is called a \textit{signature} of $\mathfrak C$.
      A \textit{$\mathfrak C$-colored operad} $\mathcal{O}$ in $\V$ consists of the following data:
      \begin{enumerate}[label = (\arabic*)]
      \item An object $\O(\ksi) \in \V$ for each signature $\ksi$.
      \item For each $c \in \mathfrak C$, a \textit{unit} $1_c \in \O(c;c)$.
      \item For any signature $\ksi \in \mathfrak C^{\times n+1}$ and $\sigma \in \Sigma_n$, a map $\O(\xi) \to \O(\sigma \cdot \xi)$,
            where $\Sigma_n$ acts on the left of $\mathfrak C^{\times n+1}$ by acting on the first $n$ coordinates.
            Explicitly, this is a map
            \begin{equation}
                  \O(c_1, \ldots, c_n; c_0) \xrightarrow{\sigma} \O(c_{\sigma^{-1}1}, \ldots, c_{\sigma^{-1}n}; c_0).
            \end{equation}
      \item For any compatible signatures $\xi = (c_1, \ldots, c_n; c_0)$, $\xi_i = (c_{1}^i, \ldots, c_{m_i}^i; c_i)$, a \textit{composition} map
            \begin{equation}
                  \O(\xi) \times \O(\xi_1) \times \ldots \times \O(\xi_n) \to \O(\xi \circ (\xi_i))
            \end{equation}
      \end{enumerate}
      where $\xi \circ (\xi_i) = (c_1^1,c_2^1,\dots,c_{m_n}^{n}; c_0)$,
      subject to the associativity, unitality, and $\Sigma$-equivariant compatibilities one would expect.      
      
      A map of $\mathfrak C$-colored operads is a compatible collection of maps
      $\set{\O(\xi) \to \O'(\xi)}_{\xi}$.
      
      Let $\Op^{\mathfrak C}(\V)$ denote the category of $\mathfrak C$-colored operads in $\V$.
\end{definition}

\begin{definition}
      \label{OP_MAP_DEFN}
      Given a map of $G$-sets $f: \mathfrak C' \to \mathfrak C$ and a $\mathfrak C$-colored operad $\O$,
      there is a natural $\mathfrak C'$-colored operad $f^*(\O)$, where
      \begin{equation}
            f^{\**}(\O)(\xi') = \O(f(\xi')),
      \end{equation}

      A \textit{map of colored operads} $\O' \to \O$ is given by the data of a map of colors $f: \mathfrak C' \to \mathfrak C$,
      and a map of $\mathfrak C'$-colored operads $\O' \to f^*(\O)$.
      
      Let $\Op(\V)$ denote the category of colored operads in $\V$.
\end{definition}

\begin{remark}
      (Colored) operads are also known as \textit{symmetric multicategories}.
\end{remark}

\begin{remark}
      The category $\Op(\V)$ is isomorphic to the Grothendieck construction on the functor
      \begin{equation}
            \begin{tikzcd}[row sep = tiny]
                  \mathsf F^{op} \arrow[r] & \mathsf{Cat}
                  \\
                  \mathfrak C \arrow[r, mapsto] & \Op^{\mathfrak C}(\V).
            \end{tikzcd}
      \end{equation}
\end{remark}




\subsection{Equivariant Colored Operads}

Let $G$ be a finite group, with a fixed (random) total ordering.
\begin{notation}
      [{cf. \cite{BP17}}]
      Recall that $\mathsf F$ denotes the category of \textit{finite sets} equipped with a total order and set maps,
      or more accurately any full subcategory where the only ordered isomorphisms are the identity.
      
      Moreover, let $\mathsf O_G \into \mathsf F^G$ denote the full subcategory of \textit{transitive} $G$-sets.
      In particular, we note that the oribts $G/H$ are well-defined (using the chosen total order on $G$,
      and the ``minimal representative'' total order on $G/H$).
\end{notation}

\begin{notation}
      For any $G$-set $A$ and $a \in A$, let $G_a$ denote the stabilizer $\Stab_G(a)$ of $a$ in $G$.
\end{notation}

\begin{definition}
      The category $\Op^G(\V)$ of  \textit{$G$-colored operads} in $\V$ is the category of
      $G$-objects in $\Op(\V)$.
\end{definition}

Unpacking this definition, we see $\O \in \Op^G(\V)$ consists of the following data:
\begin{enumerate}[label = (\arabic*)]
\item A $G$-set $\mathfrak C$ of colors.
\item For each signature $\xi$ of $\mathfrak C$, an object $\O(\xi) \in \V$.
\item For each signature $\xi \in \mathfrak C^{\times n+1}$ and $(g,\sigma) \in G\times \Sigma_n$, a map
      $\O(\xi) \to \O((g,\sigma)\cdot \xi)$,
      where $G$ acts on $\mathfrak C^{\times n+1}$ diagonally (across all $n+1$ coordinates), and $\Sigma_n$ acts on the first $n$.
\item For each $c \in \mathfrak C$, a \textit{unit} $1_c \in \O(c;c)^{G_c}$.
\item For compatible signatures $\xi$, $\xi_1$, $\ldots$, $\xi_n$, \textit{composition maps}
      \begin{equation}
            \O(\xi) \otimes \O(\xi_1) \otimes \ldots \otimes \O(\xi_n) \to \O(\xi \circ (\xi_i)),
      \end{equation}
\end{enumerate}
such that composition is
compatible with the $G$-action on each component as well as the appropriate actions of $\Sigma$,
and is unital and associative. 


\begin{remark}
	Unlike in the single-colored case, $\Op^G(\V)$ does \textit{not} coincide with the category of colored operads in $\V^G$.
	Indeed, objects in $\Op(\V^G)$ have a fixed $G$-set of colors,
        and each level $\O(\xi)$ has an action by the full group $G$
	(though only a partial action by $\Sigma_{|\xi|}$).
\end{remark}

\begin{definition}
      Given a $G$-set $\mathfrak C$, let $\Op^{G,\mathfrak C}(\V)$ denote the category of \textit{$\mathfrak C$-colored operads} and maps which are the identity on colors.
      
      Parallel to the non-equivariant case, $\Op^G(\V)$ is isomorphic to the Grothendieck construction on the functor
      \begin{equation}
            \begin{tikzcd}[row sep = tiny]
                  \mathsf (F^G)^{op} \arrow[r] & \mathsf{Cat}
                  \\
                  \mathfrak C \arrow[r, mapsto] & \Op^{G,\mathfrak C}(\V).
            \end{tikzcd}
      \end{equation}
\end{definition}


\subsubsection{Categorical description: colored trees}
\label{COMEGA_SEC}

\begin{definition}
      Given a $G$-set $X$, let $B_XG$ denote the \textit{translation category} of $X$,
      with object set $X$ and morphisms $g: x \to g\cdot x$ for all pairs $(g,x) \in G \times X$.
\end{definition}

\begin{remark}
      We observe that we have a natural diagonal map
%      \begin{equation}
      $
      F \times G \into \mathsf F \wr G,
      $
      % \end{equation}
      and so for any functor $F: \mathcal C \to \mathsf F$, we have an induced functor
      $F: \mathcal C \times G \to \mathsf F \wr G$. 
\end{remark}

\begin{definition}
      Let $\mathfrak C \Sigma$ denote the \textit{$(G,\mathfrak C)$-symmetric category}, given by
      \begin{equation}
            \mathfrak C \Sigma = \coprod\limits_{n\geq 0} B_{\mathfrak C^{\times n} \times \mathfrak C}(G \times \Sigma_n).
      \end{equation}
      
      We note that $\mathrm{Ob}(\mathfrak C \Sigma)$ is precisely the set of \textit{signatures} in $\mathfrak C$.
      We observe that this is equivalent to the pullback on the left below.
      \begin{equation}
            \label{COMEGA_B_EQ}
            \begin{tikzcd}
                  \mathfrak C \Sigma \arrow[d] \arrow[r, "E"]
                  &
                  \mathsf F \wr B_{\mathfrak C}G \arrow[d]
                  &&
                  \mathfrak C \Omega \arrow[d] \arrow[r, "E"]
                  &
                  \mathsf F \wr B_{\mathfrak C}G \arrow[d]
                  \\
                  \Sigma \times G \arrow[r, "E"]
                  &
                  \mathsf F \wr G
                  &&
                  \Omega \times G \arrow[r, "E"]
                  &
                  \mathsf F \wr G
            \end{tikzcd}
      \end{equation}
      where $E: \Sigma \to \mathsf F$ sends $\underline{n}$ to $\underline{n+1}$.
      \todo[inline]{$B_{\mathfrak C}G = G \ltimes \mathfrak C$}
      More generally, let $\mathfrak C \Omega$ be the pullback on the right above,
      where $E: \Omega \to \mathsf F$ sends a tree $U$ to its set $E(U)$ of edges.

      We have a natural inclusion of categories $\mathfrak C \Sigma \into \mathfrak C \Omega$,
      and as such we will called elements of these categories
      \textit{colored trees} (or \textit{colored corollas}),
      and denote them by $(U,\mathfrak c)$, where $\mathfrak c: E(U) \to \mathfrak C$ is a map of sets.
\end{definition}

\begin{remark}
      When $G = \set{e}$ and $\mathfrak C = \set{*}$, $\mathfrak C \Sigma = \Sigma$.
\end{remark}

Unpacking definitions, we see that a map $(U, \mathfrak c) \to (V, \mathfrak d)$ is given by
a map $f: U \to V$ in $\Omega$ and an element $g\in G$,
such that $g.\mathfrak c(e) = \mathfrak d(f(e))$ for all $e \in E(U)$.
\begin{equation}
      \begin{tikzcd}
            E(U) \arrow[r, "f"] \arrow[d, "\mathfrak c"']
            &
            E(V) \arrow[d, "\mathfrak d"]
            \\
            \mathfrak C \arrow[r, "g"]
            &
            \mathfrak C
      \end{tikzcd}
\end{equation}

In particular, we have maps of the form
\begin{equation}
      g = (id, g): (U, E(U) \to \mathfrak C) \to (U, E(U) \to \mathfrak C \xrightarrow{g \cdot} \mathfrak C). 
\end{equation}

\begin{remark}
      $\mathfrak C \Omega$ is equal to the
      Grothendieck construction on the functor
      \begin{equation}
            \label{COMEGA_B_GR_EQ}
            \begin{tikzcd}[row sep = tiny]
                  \Omega^{op} \times G^{op} \arrow[r]
                  &
                  \mathsf{Set}
                  \\
                  U \arrow[r, mapsto]
                  &
                  \Hom_{\mathsf{Cat}}(E(U), B_{\mathfrak C}G)
                  \\
                  (\phi: U \to V, g) \arrow[r, mapsto]
                  &
                  (E(U) \xrightarrow{\phi} E(V) \xrightarrow{\mathfrak c} \mathfrak C \xrightarrow{g^{-1}} \mathfrak C)
            \end{tikzcd}
      \end{equation}
      % \todo[inline]{compare with genuine case: RHS equals
      %   $\Hom(\Phi(E(U)), \Phi(\mathfrak C))$
      %   equals
      %   $\Hom(\Phi(E(G \cdot U)), \mathfrak C_{fr}))$}
      and a similar result holds for $\mathfrak C \Sigma$. 
\end{remark}

\begin{remark}
      Note that we can replace the $G$-set $\mathfrak C$ with a
      \textit{coefficient system} $\underline{\mathfrak C}$ in multiple ways,
      defining categories $\UC\Omega$.
      In each case, the $G$-set definitions are recovered by replacing $\UC$ with $\Phi(\mathfrak C)$,
      and the resulting categories are always equal to $\mathfrak C(G/e)\Omega$.

      % ------------------------------------------------------------
      
      First, we can substitute \eqref{COMEGA_B_EQ} for the rectangle of pullbacks below
      \begin{equation}
            \label{COMEGA_OG_EQ}
            \begin{tikzcd}
                  \underline{\mathfrak C}\Omega \arrow[d] \arrow[r, "E"]
                  &
                  \mathsf F \wr B_{\mathfrak C(G/e)}G \arrow[r] \arrow[d]
                  &
                  \mathsf F \wr \underline{\mathfrak C} \arrow[d]
                  \\
                  \Omega \times G \arrow[r, "E"]
                  &
                  \mathsf F \wr G \arrow[r]
                  &
                  \mathsf F \wr O_G
            \end{tikzcd}
      \end{equation}
      with $\mathfrak C(G/e) \into \underline{\mathfrak C}$ and $G \into O_G$ the natural inclusions.
      \todo[inline]{compare $B_{\mathfrak C(G/e)}G = G \ltimes \mathfrak C(G/e)$ and $\underline{\mathfrak C} = O_G \ltimes \underline{\mathfrak C}$.}
      Analogously to \eqref{COMEGA_B_GR_EQ}, this is equal to the Grothendieck construction on the functor
      \begin{equation}
            \label{COMEGA_OG_GR_EQ}
            \begin{tikzcd}[row sep = tiny]
                  \Omega^{op} \times G^{op} \arrow[r]
                  &
                  \mathsf{Set}
                  \\
                  U \arrow[r, mapsto]
                  &
                  \Hom_{\mathsf{Cat} \downarrow O_G}((G \cdot E(U))/G, \UC).
            \end{tikzcd}
      \end{equation}

      % ------------------------------------------------------------

      {\color{blue} % -------------------- ALTERNATIVELY ------------------------------
      Alternatively (cf. \eqref{FG_GR_EQ}),
      $\UC \Omega$ is isomorphic to the rectangle of pullbacks
      \begin{equation}
            \label{COMEGA_FG_EQ}
            \begin{tikzcd}
                  \UC \Omega \arrow[d] \arrow[r]
                  &
                  \mathsf F \wr B_{\mathfrak C(G/e)}G \arrow[r] \arrow[d]
                  &
                  \mathsf F^G \wr \UC \arrow[d]
                  \\
                  \Omega \times G \arrow[r]
                  &
                  \mathsf F \wr G \arrow[r]
                  &
                  \mathsf F^G,
            \end{tikzcd}
      \end{equation}
      where the map $F \wr G \to F^G$ sends objects $A$ to $G \cdot A$ and
      maps $(\phi, (g_a))$ to $G \cdot A \to G \cdot B$, $(h, a) \mapsto (h g_a^{-1}, \phi(a))$.
      % bottom arrow sends $U$ to $G \cdot E(U)$, and
      % on arrows sends $(\phi,g)$ to
      % \begin{equation}
      %       \begin{tikzcd}[row sep = tiny]
      %             G \cdot E(V) \arrow[r, "\phi \times {g^{-1}}"]
      %             &
      %             G \cdot E(U)
      %             \\
      %             (h,s) \arrow[r, mapsto]
      %             &
      %             {(hg^{-1}, \phi (s))}.
      %       \end{tikzcd}
      % \end{equation}
      Further, this is equal to the Grothendieck construction (cf. \eqref{COMEGA_B_GR_EQ})
      \begin{equation}
            \label{COMEGA_FG_GR_EQ}
            \begin{tikzcd}[row sep = tiny]
                  \Omega^{op} \times G^{op} \arrow[r]
                  &
                  \mathsf{Set}
                  \\
                  U \arrow[r, mapsto]
                  &
                  \Hom_{\mathsf{Set}^{O_G^{op}}}(\Phi(G \cdot E(U)), \UC)
            \end{tikzcd}
      \end{equation}
      In this context, a map $\phi: (U,\mathfrak c) \to (V, \mathfrak d)$ between colored trees is itself colored iff
      the triangle below commutes.
      \begin{equation}
            \label{COLOR_MAP_FG_EQ}
            \begin{tikzcd}[row sep = tiny]
                  \Phi(G \cdot E(T)) \arrow[rr, "\phi"] \arrow[dr, "\mathfrak c"']
                  &&
                  \Phi(G \cdot E(V)) \arrow[dl, "\mathfrak d"]
                  \\
                  &
                  \UC
            \end{tikzcd}
      \end{equation}
      } % -------------------- ALTERNATIVELY: END ------------------------------
\end{remark}

\begin{definition}
      The category of \textit{symmetric $(G,\mathfrak C)$-sequences} in $\V$, denoted $\Sym^{G,\mathfrak C}(\V)$, is
      the category of functors $X: \mathfrak C \Sigma^{op} \to \V$.
\end{definition}

\begin{remark}
      \label{COLOR_CHANGE_REM}
      We observe that the set of objects in $\mathfrak C\Sigma$ is precisely the set of $\mathfrak C$-signatures.
      Hence every $\mathfrak C$-colored operad has an underlying symmetric $(G,\mathfrak C)$-sequence.
      
      Moreover, extending Defintion \ref{OP_MAP_DEFN}, we have that any map of $G$-sets
      $F: \mathfrak C \to \mathfrak C'$ induces a pair of compatible adjoints
      \begin{equation}
            \label{COLOR_CHANGE_EQ}
            \begin{tikzcd}
                  \Op^{G, \mathfrak C'}(\V) \arrow[r, shift right, "F^{\**}"'] \arrow[d, "\mathsf{fgt}"']
                  &
                  \Op^{G, \mathfrak C}(\V) \arrow[l, shift right, "F_!"'] \arrow[d, "\mathsf{fgt}"]
                  \\
                  \Sym^{G, \mathfrak C'}(\V) \arrow[r, shift right, "F^{\**}"']
                  &
                  \Sym^{G, \mathfrak C}(\V) \arrow[l, shift right, "F_!"'].
            \end{tikzcd}
      \end{equation}

      We also record the straightforward observation that if a map $F: \O_1 \to \O_2$ is color-fixed, then
      a commuting square (resp. lifting diagram, pullback) as in the middle below is
      equivalent to such squares on the left and right.
      \begin{equation}
            \label{COLOR_SQ_EQ}
            \begin{tikzcd}
                  a_! \O_1 \arrow[d, "a_! F"'] \arrow[r, "a"]
                  &
                  \O \arrow[d, "p"]
                  &&
                  \O_1 \arrow[d, "F"] \arrow[r, "a"]
                  &
                  \O \arrow[d, "p"]
                  &&
                  \O_1 \arrow[d, "F"] \arrow[r, "a"]
                  &
                  a^{\**} \O \arrow[d]
                  \\
                  a_! \O_2 \arrow[r]
                  &
                  p^{\**} \P
                  &&
                  \O_2 \arrow[r]
                  &
                  \P
                  &&
                  \O_2 \arrow[r]
                  &
                  a^{\**} p^{\**} \P
            \end{tikzcd}
      \end{equation}
\end{remark}


Now, many of the natural functors around $\Omega$ and $\Sigma$ have generalizations to the colored setting,
which can be built through a straightforward use of the universal property of pullbacks.

\begin{definition}
      We have a natural \textit{vertex} functor
      $V: \mathfrak C \Omega \to \Sigma \wr \mathfrak C \Sigma$,
      as colorings of a tree restrict to colorings of each vertex corolla.

      Similarly, there is a \textit{leaf-root} functor
      $\mathsf{lr}: \mathfrak C \Omega \to \mathfrak C \Sigma$,
      where the coloring of $\mathsf{lr}(T)$ is a restrict of the coloring of $T$.
\end{definition}


\todo[inline]{come back - fix narative}

Moreover, the algebraic structure on these operads is determined by a monad on symmetric sequences.

\begin{definition}
      Given $X \in \Sym^{G, \mathfrak C}$, let $\mathbb F^{\mathfrak C} X$ denote the left Kan extension below.
      \begin{equation} 
           \begin{tikzcd}
                  \mathfrak C \Omega^{op}
                  \arrow[d, "\mathsf{lr}"']
                  \arrow[r, "V"]
                  &
                  (\Sigma \wr \mathfrak C \Sigma)^{op} \arrow[r, "X"]
                  \arrow[dl, Rightarrow]
                  &
                  (\Sigma \wr \V^{op})^{op} \arrow[r, "\otimes"]
                  &
                  \V
                  \\
                  \mathfrak C \Sigma^{op} \arrow[urrr, "\Lan = \mathbb F^{\mathfrak C} X"']
            \end{tikzcd}
      \end{equation}
\end{definition}

\subsection{Single-Colored Operads}
We first show that this generalizes the free single-colored operad monad.

Note that when $\mathfrak C = \set{\**}$, we have
$\mathfrak C \Omega = \Omega \times G$ and 
$\mathfrak C \Sigma = \Sigma \times G$.

\begin{notation}
      Given a functor $X : \C \to \mathsf{Fun}(\mathcal D, \V))$,
      let $\tilde X$ denote the adjoint functor $\tilde X: \C \times \mathcal D \to \V$.
\end{notation}

\begin{lemma}
      \label{SPAN_LAN_LEM}
      Conisder the two spans below.
      \begin{equation}
            \begin{tikzcd}
                  \C \arrow[d, "p"] \arrow[r, "X"]
                  &
                  \mathsf{Fun}(\mathcal D, \V)
                  &&
                  \C \times \mathcal D \arrow[d, "p \times \mathsf{id}"] \arrow[r, "\tilde X"]
                  &
                  \V
                  \\
                  \mathcal E
                  &
                  &&
                  \mathcal E \times \mathcal D
            \end{tikzcd}
      \end{equation}
      
      Then $\Lan_p X$ is adjoint to $\Lan_{p \times \mathsf{id}} \tilde X$. 
\end{lemma}
\begin{proof}
      Using the pointwise description of the Kan extension, we have
      \begin{align}
        \widetilde{\Lan_p X}(e,d)
        &= (\Lan_p X(e))(d)
          = \left(
          \colim\limits_{\substack{ \C \downarrow e \\ p(c) \to e}} X(c)
        \right)(d)
        = \colim\limits_{\substack{ \C \downarrow e \\ p(c) \to e}}(X(c)(d))
        = \colim\limits_{\substack{ \C \downarrow e \\ p(c) \to e}}(\tilde X(c,d))\\
        &= \colim\limits_{\substack{ \C \times \set{d} \downarrow (e,d) \\ p(c) \to e}}(\tilde X(c,d))
        \cong \colim\limits_{\substack{ \C \times \mathcal D \downarrow (e,d) \\ (p(c),d') \to (e,d)}}(\tilde X(c,d'))
        = \Lan_{p \times \mathsf{id}}\tilde X(c,d),
      \end{align}
      where the isomorphism holds by a straightforward finality argument.
      On maps, a similar argument holds.
\end{proof}

\begin{notation}[\cite{BP17}]
      Let $\mathbb F'$ denote the \textit{free single-colored operad monad} on $\V$, given by the left Kan extension of the following diagram.
      \begin{equation}
            \begin{tikzcd}
                  \Omega^{op}
                  \arrow[d, "\mathsf{lr}"']
                  \arrow[r, "V"]
                  &
                  (\Sigma \wr \Sigma)^{op} \arrow[r, "X"]
                  \arrow[dl, Rightarrow]
                  &
                  (\Sigma \wr \V^{op})^{op} \arrow[r, "\otimes"]
                  &
                  \V
                  \\
                  \Sigma^{op} \arrow[urrr, "\Lan = \mathbb F' X"']
            \end{tikzcd}
      \end{equation}
\end{notation}

\begin{proposition}
      \label{TEST_PROP}
      $\mathbb F^{\set{\**}}$ is a monad, and moreover
      the category of $\mathbb F^{\set{\**}}$-algebras in $\mathsf{Fun}(\Sigma \times G, \V)$ is equivalent to
      the category of $\mathbb F'$-algebras in $\mathsf{Fun}(\Sigma, \V^G)$.
\end{proposition}
\begin{proof}
      Let $\tau: \tilde X \mapsto X$ denote the isomorphism of categories
      $\mathsf{Fun}(\Sigma \times G, \V) \xrightarrow{\tau} \mathsf{Fun}(\Sigma, \V^G)$.
      Then $\mathbb F^{\set{\**}} = \tau^{-1} \mathbb F' \tau$ by \ref{SPAN_LAN_LEM}, and so
      $\mathbb F^{\set{\**}}$ is in fact a monad, and the
      the isomorphism lifts to an isomorphism on the category of algebras.
\end{proof}

The general case will be given in Proposition \ref{FC_MONAD_PROP}.





\newpage

\section{Colored Genuine Equivariant Operads}

Throughout this section, we will abuse notation, and refer to
a coefficient system and its associated (Grothendieck) category over $O_G$ by the same name.

Idea: we have a \textit{coefficient system} $\UC$ of colors, and
a \textit{signature} will consist of a tuple $\xi = (x_1, \dots, x_n;x_0)$
with $x_i \in \mathfrak C(G/H_i)$ for subgroups $H_i \leq H_0 \leq G$
(or, equivalently, each $x_i$ is secretly in fact a whole $G$-diagram of objects).

\subsection{Colored $G$-Trees}

Extending \S \ref{COMEGA_SEC}, we make the following definitions.

\begin{definition}
      \label{EG_DEFN}
      For $T \in \Omega_G$, let $E_G(T)$ denote the set of \textit{edge orbits} $E(T)/G$.
      The \textit{edge orbit} functor $E_G: \Omega_G \to \mathsf F \wr \mathsf O_G$ sends a $G$-tree $T$ to
      % the tuple $(E_G(T), (G/G_e)_{Ge \in E_G(T)})$,
      % where we have chosen canonical representatives for elements in $E_G(T)$ by choosing
      % $e \in Ge$ minimal with respect to the planar structure on $T$.
      % Given $\phi: T \to S$, $G t \in E_G(T)$,
      % the $G t$ component of the image $E_G(\phi)$ is given the map
      % $(g_t)_{\**}: G/G_t \to G/G_s$, $h G_t \mapsto h g_t G_s$, where
      % $s$ in minimal in $G \phi(t)$ and
      % $g_t \in G$ such that $\phi(t) = g_t s$
      % (as $g_t$ is unique modulo $G_s$, the map $(g_t)_{\**}$ is well-defined).
      the tuple $(E_G(T), (Gt))$.
\end{definition}

\begin{remark}
      \label{EG_OVER_OG_REM}
      One may be tempted to say that $E_G$ is in fact a functor .
      % with $E_G(T) \to \mathsf O_G$ given by $G t \mapsto G/G_t$ for $t \in G t$ minimal.
      We warn that $E_G$ is \textit{not} a functor
      $\Omega_G \to \Set \downarrow \mathsf O_G$,
      as it does not allow for the action of quotient maps.
      It is however a functor when restricted to the category $\Omega_G^{fr}$ of \textit{free} $G$-trees
      (cf. \eqref{COMEGA_FG_GR_EQ}).
      
      Instead, $E_G$ defines a functor to $\Cat \downarrow_w \mathsf O_G$ of functors and ``weakly-commuting'' triangles,
      as for all $\phi: T \to S$ there is a well-defined natural transformation as below.
      \begin{equation}
            \begin{tikzcd}
                  E_G(T) \arrow[rr, "\phi"] \arrow[dr, "\mathfrak c"', ""{name = U}]
                  &&
                  |[alias = V]| E_G(S) \arrow[dl, "\mathfrak d"]
                  \\
                  &
                  \mathsf O_G
                  % \arrow[Rightarrow, from = U, to = V, "{(g_{(\minus)})_{\**}}"]
                  \arrow[Rightarrow, from = U, to = V, "\phi_{(-)}^{\**}"'{near start}]
            \end{tikzcd}
      \end{equation}
      
      However, we may augment $E_G$ to a functor
      $\tilde E_G: \Omega_G \to \mathsf{Fib}(\mathsf \mathsf O_G)$
      to the category of fibrations over $\mathsf \mathsf O_G$.
      In particular, let $\tilde E_G(T)$ denote the Grothendieck construction on the functor
      $E_G(T) \to \Set$, $G t \mapsto \mathsf \mathsf \mathsf O_G \downarrow G t$;
      This has the natural structure of a fibration $p: \tilde E_G(T) \to \mathsf O_G$
      with $p(G t, q: I \to G t) = I$ and Cartesian arrows defined by pre-composition.
\end{remark}

\begin{definition}[{cf. \eqref{COMEGA_OG_EQ}}]
      Let $\underline{\mathfrak C}$ be a $G$-coefficient system of sets.
      Then the category $\underline{\mathfrak C}\Omega_G$ (resp. its subcategory $\UC \Sigma_G$)
      of \textit{$\underline{\mathfrak C}$-colored $G$-trees}
      (resp. \textit{$\UC$-colored $G$-corollas})
      is defined to be the pullback on the right (left) below.
      \begin{equation}
            \label{COMEGAG_OG_EQ}
            \begin{tikzcd}
                  \UC \Sigma_G \arrow[d] \arrow[r]
                  &
                  \mathsf F \wr \UC \arrow[d]
                  &&
                  \UC \Omega_G \arrow[d] \arrow[r]
                  &
                  \mathsf F \wr \underline{\mathfrak C} \arrow[d]
                  \\
                  \Sigma_G \arrow[r, "E_G"]
                  &
                  \mathsf F \wr \mathsf O_G
                  &&
                  \Omega_G \arrow[r, "E_G"]
                  &
                  \mathsf F \wr \mathsf O_G
            \end{tikzcd}
      \end{equation}
\end{definition}

Objects of $\UC \Omega_G$ are pairs $(T, \mathfrak c)$ of
a $G$-tree $T$ and
a functor $\mathfrak c: E_G(T) \to \underline{\mathfrak C}$ over $\mathsf O_G$.
Explicitly, each orbit of edges $G t$ % (with $t$ minimal)
is assigned a color $\mathfrak c(G t) \in \underline{\mathfrak C}(G t)$. % $\underline{\mathfrak C}(G/G_{t})$.
Morphisms $(T, \mathfrak c) \to (S, \mathfrak d)$ 
are given by maps of trees $\phi: T \to S$ such that, for every edge orbit $G t$ of $T$, we have
\begin{equation}
      \mathfrak c(G t) = \phi_{G t}^{\**} \mathfrak d(G \phi(t))
      % \mathfrak c(G t) = g_t^{\**} \mathfrak d(G s),
      %\phi_{e}^{\**}g_e^{\**}\mathfrak d(Gf),
\end{equation}
with $\phi_{G t}: G t \to G \phi(t)$.
% with $g_t \in G$ defined as in Definition \ref{EG_DEFN}.
% where $\phi_{e}: G / G_{e} \to G / G_{\phi(e)}$ is the map in $\mathsf O_G$ induced by $\phi$,
% and $\phi(e) = g_e f$ for $f \in Gf \in E_G(S)$ minimal; as $g_e$ is unique modulo $G_f$, $g_e^{\**}$ is well-defined.

\begin{remark}
      In analogue to \eqref{COMEGA_OG_GR_EQ}, we note that $\UC \Omega_G$ is equivalent to the Grothendieck construction
      \begin{equation}
            \label{COMEGAG_OG_GR_EQ}
            \begin{tikzcd}[row sep = tiny]
                  \Omega_G^{op} \arrow[r]
                  &
                  \mathsf{Set}
                  \\
                  T \arrow[r, mapsto]
                  &
                  \Hom_{\mathsf{Fib} (\mathsf O_G)}(\tilde E_G(T), \UC).
            \end{tikzcd}
      \end{equation}
      The (standard) coloring of $G t$ is recovered by $\mathfrak c(G t, id_{G t})$, and
      a triangle of fibrations on the left below over $\mathsf O_G$
      implies that the right triangle commutes.
      \begin{equation}
            \begin{tikzcd}[column sep = small]
                  \tilde E_G(T) \arrow[rr, "\phi"] \arrow[dr, "\mathfrak c"']
                  &&
                  \tilde E_G(S) \arrow[dl, "\mathfrak d"]
                  &
                  (G t, id) \arrow[rr, mapsto] \arrow[dr, mapsto]
                  &&
                  (G \phi(t), \phi_{G t}) \arrow[dl, mapsto]
                  \\
                  &
                  \UC
                  &
                  &
                  &
                  \mathfrak c(G t) = \phi_{G t}^{\**} \mathfrak d(G \phi(t)).
            \end{tikzcd}
      \end{equation}
\end{remark}

\begin{remark}
      { \color{blue} % -------------------- ALTERNATIVELY: ----------------------------------------
        Alternatively, (cf. \eqref{COMEGA_FG_EQ})
        $\UC\Omega_G$ is isomorphic to the pullback below.
        \footnote{We note that the class of morphisms in $\mathsf F^G$ in the image of $E$, when restricted to $\Omega_G^0$
          are those isomorphic to an adjunction counit $G \cdot_H A|_H \to A$.}
        \begin{equation}
              \label{COMEGAG_FG_EQ}
              \begin{tikzcd}
                    \UC \Omega_G \arrow[d] \arrow[r]
                    &
                    \mathsf F^G \wr \underline{\mathfrak C} \arrow[d]
                    \\
                    \Omega_G \arrow[r, "E"]
                    &
                    \mathsf F^G.
              \end{tikzcd}
        \end{equation}
        
        Similarly to \eqref{COMEGA_B_GR_EQ}, $\underline{\mathfrak C}\Omega_G$ is isomorphic to the Grothendieck construction on the functor
        \begin{equation}
              \begin{tikzcd}[row sep = tiny]
                    \Omega_G^{op} \arrow[r]
                    &
                    \mathsf{Set}
                    \\
                    T \arrow[r, mapsto]
                    &
                    \Hom_{\Set^{\mathsf O_G^{op}}}(\Phi(E(T)), \underline{\mathfrak C}),
              \end{tikzcd}
        \end{equation}
        In particular, we note that
        $\Hom_{\mathsf{Cat}}(E(T), B_{\mathfrak C} G) = \Hom_{\Set^{\mathsf O_G^{op}}}(\Phi(E(T)), \Phi(\mathfrak C))$. 
        
        
        As in \eqref{COMEGA_FG_EQ}, a coloring is a map $\mathfrak c: \Phi E(T) \to \mathfrak C$ of coefficient systems,
        and morphisms are maps $\phi: T \to S$ which are compatible on colorings,
        in that the triangle below  (cf. \eqref{COLOR_MAP_FG_EQ}) commutes.
        \begin{equation}
              \begin{tikzcd}[row sep = tiny]
                    \Phi E(T) \arrow[rr, "f"] \arrow[dr, "\mathfrak c"']
                    &&
                    \Phi E(S) \arrow[dl, "\mathfrak d"]
                    \\
                    &
                    \underline{\mathfrak C}
              \end{tikzcd}
        \end{equation}
        It is easy to show this is equivalent to requiring that
        % $\mathfrak c(G/G_t,t) = \phi_t^{\**} \mathfrak d(G/G_{\phi(t)}, \phi(t))$,
        % where $\phi_t: G/G_t \to G/G_{\phi(t)}$ is the unique map preserving the coset of the unit.
        $\mathfrak c(G t, t) = \phi_{G t}^{\**} \mathfrak d(G \phi(t), \phi(t))$. 
        
        $\UC \Sigma_G$ can be defined similarly, with the relevant sources restricted to
        $\Sigma_G \subseteq \Omega_G$. 
      }
      %%%%%%%%%%%%% COLOR: BLUE ----------------------------------------
\end{remark}


\begin{remark}
      $\underline{\mathfrak C}\Omega_G$ is also naturally a root fibration, 
      that is, a split Grothendieck fibration over the orbit category.

      Formally, as $\mathsf F \wr (-)$ and pullbacks preserve such fibrations, and these are compatible under composition,
      this follows from the natural maps $\underline{\mathfrak C}\Omega_G \to \Omega_G \to \mathsf O_G$.
      Explicitly, $\underline{\mathfrak C}\Omega_G(I)$ has as objects those pairs $(T,\mathfrak c)$ such that
      % $T \simeq G \cdot_H T_*$ for $T_{\**} \in \Omega^H$.
      $r(T) = I$. 
      % In this case, we have a canonical isomorphism
      % $E_H(T_{\**}) \to E_G(T) \simeq E_H(T_{\**})$, and we have a natural factorization
      % \begin{equation}
      %       \begin{tikzcd}
      %             E_G(T) \simeq E_H(T_{\**}) \arrow[r, "\mathfrak c"] \arrow[dr]
      %             &
      %             \mathfrak C|_{H} \arrow[r, hookrightarrow] \arrow[d]
      %             &
      %             \mathfrak C \arrow[d]
      %             \\
      %             &
      %             O_H \arrow[r, hookrightarrow]
      %             &
      %             \mathsf O_G.
      %       \end{tikzcd}
      % \end{equation}
      Maps $\phi:(T,\mathfrak c) \to (S, \mathfrak d)$ in each fiber are called \textit{root-fixed}:
      as maps in $\Omega_G$, they are \textit{rooted} ($r(T) \to r(S)$ is a planar isomorphism),
      and moreover $\mathfrak c(r(T)) = \mathfrak d(r(S))$.
      
      Given $q: I \to J$ in the orbit category,
      the chosen Cartesian maps are the induced root pullback maps $q: q^{\**}T \to T$ on $G$-trees,
      with the coloring $q^{\**}\mathfrak c$ of $q^{\**}T$ defined by
      $(q^{\**}\mathfrak c)(G t) = q_{G t}^{\**}(G \phi(t))$.
      % for $b\in E(q^{\**}T)$, minimal in it's $G$-orbit, we have $q(b) = g a$ for some $g \in G$ and $a \in E(T)$ minimal in its orbit.
      % Moreover, as this $g$ is unique modulo $G_a$, we have that there is a well-defined map $g_*: G/G_{q(b)} \to G/G_a$,
      % and as $q$ induces a unique map $q_b^{\**}: G/G_b \to G/G_{q(b)}$, we have
      % \begin{equation}
      %       (q^{\**}\mathfrak c)(Gb) q_b^{\**}g^{\**}\mathfrak c(Ga).
      % \end{equation}

      {\color{blue} % ---------- ALTERNATIVELY --------------------
        Alternatively, on $\Phi E(q^{\**} T)$, we have
        $(q^{\**}\mathfrak c)(G/H, s) = q_b^{\**}(\mathfrak c(G/H, q(s)))$.
      } % --------------------------------------------------
\end{remark}

\begin{remark}
      We note that any \textit{planar} map of colored $G$-trees is always \textit{color-fixed}, in that
      $\mathfrak c(Ge) = \mathfrak d(G \phi (e))$ for all $Ge \in E_G(T)$.
\end{remark}

\begin{remark}
      A \textit{quotient} map in $\UC \Omega_G$ is any morphism such that the underlying map in $\Omega_G$ is a quotient.
\end{remark}

We have natural inclusions on the left
\begin{equation}
      \begin{tikzcd}
            \underline{\mathfrak C}\Sigma \arrow[d, "\iota"] \arrow[r]
            &
            \underline{\mathfrak C}\Omega \arrow[d]
            &&
            \Sigma \times G \arrow[d] \arrow[r]
            &
            \Omega \times G \arrow[d]
            \\
            \underline{\mathfrak C}\Sigma_G \arrow[r]
            &
            \underline{\mathfrak C}\Omega_G
            &&
            \Sigma_G \arrow[r]
            &
            \Omega_G
      \end{tikzcd}
\end{equation}
which forget to the uncolored inclusions on the right.
Specifically, $U \mapsto G \cdot U$ and, as $E_G(G \cdot U) = E(U)$, the associated coloring map is simply $\mathfrak c$ again.
On morphisms, $(\phi,g)$ maps to $g \cdot (\phi)_{G}$.

\begin{remark}
      Given $X\in \dSet_G$ with $X(\eta_{G/H}) = \mathfrak C(G/H)$, we have that
      $\UC\Sigma_G$ is equal to the category of \textit{profiles} $\partial\Omega[C] \to X$,
      where $C$ ranges over all of $\Sigma_G$.
      \todo[inline]{come back: contextualize}
\end{remark}

\subsection{Planar Strings and Stuff}

\todo[inline]{come back}

Main observation:
\begin{remark}
      Given a map $\phi: T \to S$ of $G$-trees such that $S$ is $\UC$-colored, there is a unique coloring on $T$ such that
      $\phi$ is a map of colored trees.
      Moreover, any factorization $T \to S' \to S$ in $\Omega_G$ of a map of colored trees $T \to S$
      comes equipped with a canonical coloring so that the factorization is in fact in $\UC\Omega_G$.

      As an upshot, the vast majority of the operations and constructions in \cite[\S 3-5]{BP17}
      follow through immediately to the colored setting,
      once we establish the requisit categorical characterizations of the objects in question.

      Later, when considering the model structures, the coloring will similarly not directly interact with
      any of the arguments in \textit{loc cite} as, in particular,
      the coloring has no effect on the breakdown into ``active'' and ``inert'' vertices
      (with the exception that in the partition product description of $\Aut((T_v, \mathfrak c)_{v \in V_G^{ac}(T)})$,
      the partition is not the particular of the underlying $G$-tree).
      
      In this section, we highlight some of those.
\end{remark}

\todo[inline]{Need to strike a balance between what to show explicitly, and what to just state.
  \S 3.4 and \S 4 from \cite{BP17} extend almost formally, though phrasing it as such...

  We still have natural span $\UC\Sigma_G \leftarrow \UC\Omega_G^0 \to \mathsf F_s \wr \UC\Sigma_G$, such that
  the left arrow is a map of rooted fibrations.
  This, plus whats already in \S 3.4 and \S 4, may be enough to just formally push through.}

Generalizing \cite[Remark 3.78]{BP17}
\footnote{We do this as opposed to the original definition: unless we somehow force the ``correct isotropy'' of color on the non-equivariant trees, we just see $\Phi\mathfrak C(G/e)$, and not the whole coefficient system.}
\begin{definition}
      Given $(T,\mathfrak c) \in \underline{\mathfrak C}\Omega_G$, a
      \textit{planar (resp. rooted) $T$-substitution datum} is a tuple
      $((U_{v_{Ge}}, \mathfrak c_{v_{Ge}}))_{v_{Ge} \in V_G(T)}$ of $\underline{\mathfrak C}$-colored $G$-trees along with
      planar (resp. rooted color-fixed) tall maps
      $T_{v_{Ge}} \to U_{v_{Ge}}$.

      A map of planar (resp. rooted) $T$-substitution data $(U_{v_{Ge}}) \to (V_{v_{Ge}})$ is a compatible tuple of planar (resp. rooted color-fixed) tall maps $(U_{v_{Ge}} \to V_{v_{Ge}})$.
      Let $\mathsf{Sub}_p(T)$ and $\mathsf{Sub}(T)$ denote the categories of planar (resp. rooted) $T$-substituion datum.
\end{definition}

\begin{lemma}[{cf. \cite[Prop. 3.41]{BP17}}]
      Let $(T,\mathfrak c) \in \underline{\mathfrak C}\Omega_G$ be a $\underline{\mathfrak C}$-colored $G$-tree.
      There are isomorphisms of categories
      \begin{equation}
            \label{SUB_EQUIV_EQ}
            \begin{tikzcd}[row sep = 4pt]
                  \mathsf{Sub}_p(T) \arrow[r, shift left]
                  &
                  (T, \mathfrak c) \downarrow \underline{\mathfrak C}\Omega_G^{pt}
                  \arrow[l, shift left]
                  &
                  \mathsf{Sub}(T) \arrow[r, shift left]
                  &
                  (T, \mathfrak c) \downarrow \underline{\mathfrak C}\Omega_G^{r}
                  \arrow[l, shift left]                  
                  \\
                  (U_{v_{Ge}}) \arrow[r, mapsto]
                  &
                  ((T, \mathfrak c) \to \colim_{Sc_G(T)}U_{(-)}).
                  &
                  (U_{v_{Ge}}) \arrow[r, mapsto]
                  &
                  ((T, \mathfrak c) \to \colim_{Sc_G(T)}U_{(-)}).
            \end{tikzcd}
      \end{equation}
      where $\underline{\mathfrak C}\Omega_G^{pt}, \underline{\mathfrak C}\Omega_G^{r}$ are the categories of planar tall (resp. rooted) maps under $(T, \mathfrak c)$. 
\end{lemma}
\begin{proof}
      This follows as in \cite[Prop. 3.41]{BP17}, going by induction on $n=|V_G(T)|$.
      Let $(U_T,\mathfrak c_{U_T})$ denote the colimit, if it exists.
      If $n$ is 0 or 1, $T$ is terminal in $Sc_G(T)$, and the coloring $\mathfrak c_{U_T}$ is just $\mathfrak c$.
      Otherwise, we have a decomposition $T = R \amalg_{Ge} S$ with
      the planar ordering on $Ge$ in $R$, $S$, and $T$ the same,
      $E_G(T) = E_G(R) \amalg_{Ge} E_G(S)$
      $\mathfrak c_{R} = \mathfrak c|_{E_G(R)}$,
      $\mathfrak c_{S} = \mathfrak c|_{E_G(S)}$,
      such that
      the existence of $U_T$ and $\mathfrak c_{U_T}$ follow from the existence of the pushout below in $\underline{\mathfrak C}\Omega_G^{pt}$.
      \begin{equation}
            \begin{tikzcd}
                  (\eta_{Ge}, \mathfrak c) \arrow[d] \arrow[r]
                  &
                  (U_S, \mathfrak c_{U_S}) \arrow[d, dashed]
                  \\
                  (U_R, \mathfrak c_{U_R}) \arrow[r, dashed]
                  &
                  (U_T, \mathfrak c_{U_T})
            \end{tikzcd}
      \end{equation}
      By induction, $U_S$, $U_R$, $\mathfrak c_{U_S}$, $\mathfrak c_{U_R}$ exist
      (with unique choices such that $(U_{v_{Ge}}, \mathfrak c_{U_{v_{Ge}}}) \into(U_R, \mathfrak c_{U_R})$ is planar).
      Forgetting colors, this is an equivariant grafting diagram, and hence the $G$-tree $U_T$ exists.
      Moreover, we have $E_G(U_T) = E_G(U_S) \amalg_{Ge} E_G(U_R)$, and so we have a well-defined coloring
      \begin{equation}
            \mathfrak c_{U_T}(Gf) =
            \begin{cases}
                  \mathfrak c_{U_R}(Gf) \qquad \qquad & Gf \in E_G(R) \\
                  \mathfrak c_{U_S}(Gf) & Gf \in E_G(S)
            \end{cases}
      \end{equation}
      since the overlap $Ge$ is in $T$, and hence it is dictated that $\mathfrak c_{U_T}(Ge) = \mathfrak c (Ge)$.
\end{proof}

\begin{lemma}[{cf. \cite[Lemma 3.63]{BP17}}]
      $\underline{\mathfrak C}\Omega_G^0 \to \mathsf F_s \wr \underline{\mathfrak C}\Sigma_G$
      sends root pullbacks to pullbacks over $\mathsf F_s \wr \mathsf O_G$.
\end{lemma}
\begin{proof}
      Exactly as in \textit{loc cite}, with the additional note that
      the coloring of $\psi^{\**}T$ is precisely such that each $(\psi^{\**}T)_{v_{Ge}} \to T_{v_{G\phi(e)}}$
      is a pullback in $\UC \Sigma_G$.
\end{proof}

\begin{definition}
      The category $\UC\Omega_G^n$ of \textit{colored planar $n$-strings} is the category
      whoses objects are strings
      \begin{equation}
            (T_0,\mathfrak c_0)
            \xrightarrow{\phi_1} (T_1, \mathfrak c_1)
            \xrightarrow{\phi_2} \ldots
            \xrightarrow{\phi_n} (T_n, \mathfrak c_n)
      \end{equation}
      where $(T_i, \mathfrak c_i) \in \UC\Omega_G$ and the $\phi_i$ are all colored planar tall maps,
      while arrows are commutative diagrams of quotient maps.
\end{definition}

\begin{remark}
      We observe
      \begin{enumerate}
      \item $\UC\Omega_G^\bullet \to \UC\Sigma_G$ is an augmented simplicial object in categories.
      \item $\UC\Omega_G^n \to \mathsf O_G$ is a root fibration.
      \item We have a vertex functor $V_G: \UC\Omega_G^{n+1} \to \mathsf F_s \wr \UC\Omega_G^n$ by
            \begin{equation}
                  \left(
                  (T_0,\mathfrak c_0)
                  \to (T_1, \mathfrak c_1)
                  \to \dots
                  \to (T_n, \mathfrak c_n)
                  \large)
                  \mapsto
                  \large(
                  (T_{1,v_{Ge}},\mathfrak c_1)
                  \to \dots \to
                  (T_{n,v_{Ge}}, \mathfrak c_n)
                  \right)_{v_{Ge} \in V_G(T_0)}
            \end{equation}
            where we write abusively denote by $T_{i,v_{Ge}}$ the $G$-tree $(T_{i,\bar\phi_i(f)^\uparrow \leq \bar\phi_i(f)})_{f \in Ge}$
            and by $\mathfrak c_i$ the restriction to any of its sub-$G$-trees.

            Alternatively, regarding the source above as a string of $n-1$ arrows in
            $(T_0, \mathfrak c_0) \downarrow \UC\Omega_G^{pt}$,
            the image under $V_G$ can be recognized as the inverse image under \eqref{SUB_EQUIV_EQ}.
      \end{enumerate}
\end{remark}

\begin{proposition}[{cf. \cite[Prop 3.82]{BP17}}]
      For any $n \geq 0$, the commutative diagram
      \begin{equation}
            \begin{tikzcd}
                  \UC\Omega_G^n \arrow[d, "d_{1,\dots,n}"'] \arrow[r, "V_G"]
                  &
                  \mathsf F_s \wr \UC\Omega_G^{n-1} \arrow[d, "\mathsf F \wr d_{0,\dots,n-1}"]
                  \\
                  \UC\Omega_G^0 \arrow[r, "V_G"']
                  &
                  \mathsf F_s \wr \UC\Sigma_G
            \end{tikzcd}
      \end{equation}
      is a pullback diagram in $\Cat$.
\end{proposition}
\begin{proof}
      This follows as in \textit{loc cite}, as the colorings descend to all sub-$G$-trees.
\end{proof}

% \begin{proposition}
%       [{cf. \cite[Lemma 4.28]{BP17}}]
%       $N_{\mathfrak C}$ on spans preserves right Kan extensions over $\mathsf F \wr \mathcal A \downarrow \mathsf F \wr \UC\Sigma_G$.
% \end{proposition}
% \begin{proof}
%       Also follows from non-equivariant proof.
% \end{proof}

Similarly, \cite[Prop 3.47, 3.90, 4.12, 4.15, 4.26, 4.28, 4.30]{BP17} naturally generalized to the colored-setting,
replacing all instances of $\Omega_G^n$ or $\Sigma_G$ with $\UC\Omega_G^n$ and $\UC\Sigma_G$.
In particular, this yields the following definitions and proposition.
\begin{definition}[{cf. \cite[Defn 4.3]{BP17}}]
      Let $\mathsf{WSpan}^l(\mathcal C, \mathcal D)$ (resp. $\mathsf{WSpan}^r(\mathcal C, \mathcal D)$)
      denote the category of \textit{left (resp. right) weak spans}, with objects
      \begin{equation}
            \mathcal C \xleftarrow{k} \mathcal A \xrightarrow{X} \mathcal D
      \end{equation}
      and arrows those diagrams as on the left (resp. right) below
      \begin{equation}
            \begin{tikzcd}[row sep = tiny]
                  & \mathcal A_1 \arrow[dr, "X_1", ""'{name=U}] \arrow[dl, "k_1"'] \arrow[dd, "i"']
                  &
                  &&
                  &
                  \mathcal A_1 \arrow[dr, "X_1", ""'{name=A}] \arrow[dl, "k_1"'] \arrow[dd, "i"']
                  \\
                  \mathcal C
                  &&
                  \mathcal D
                  &&
                  \mathcal C
                  &&
                  \mathcal D
                  \\
                  & |[alias=V]| \mathcal A_2 \arrow[ur, "X_2"'] \arrow[ul, "k_2"]
                  &
                  &&
                  &
                  |[alias=B]| \mathcal A_2 \arrow[ur, "X_2"'] \arrow[ul, "k_2"]
                  \arrow[Rightarrow, from = U, to = V]
                  \arrow[Rightarrow, from = A, to = B]
            \end{tikzcd}
      \end{equation}
      denoted by $(i,\phi): (k_1,X_1) \to (k_2,X_2)$, with composition defined in the natural way.      
\end{definition}

\todo[inline]{recall adjunctions with $\mathsf{Lan}$ and $\mathsf{Ran}$, canonical op-isos, etc}

\begin{definition}[{cf. \cite[Defn 4.16]{BP17}}]
      Suppose $\V$ is a symmetric monoidal category with diagonals.
      We define an endofunctor $N_{\UC}$ on $\mathsf{WSpan}^r(\UC\Sigma_G, \V^{op})$
      by letting $N_{\UC}(\UC\Sigma_G \leftarrow \mathcal A \to \V^{op})$ be given by the span
      \begin{equation}
            \begin{tikzcd}
                  \UC\Omega_G^0 \wr \mathcal A \arrow[d] \arrow[r, "V_G"]
                  &
                  \mathsf F \wr \mathcal A \arrow[d] \arrow[r]
                  &
                  \mathsf F \wr \V^{op} \arrow[r, "\otimes^{op}"]
                  &
                  \V^{op}
                  \\
                  \UC\Omega_G^0 \arrow[r, "V_G"'] \arrow[d]
                  &
                  \mathsf F \wr \UC\Sigma_G
                  \\
                  \UC\Sigma_G
            \end{tikzcd}
      \end{equation}
      where the given square is a pullback, and on arrows in the natural way.

      Moreover, we have a multiplication $\mu: N_{\UC} \circ N_{\UC} \Rightarrow N_{\UC}$ given by the natural isomorphism
      \begin{equation}\label{MULTDEFSPAN EQ}
            \begin{tikzcd}
                  \UC\Sigma_G \ar[equal]{d}&
                  \UC\Omega_{G}^1 \wr A \ar{r}{V_G} \ar{d}[swap]{d_{0}} \ar{l}&
                  \Fin \wr \UC\Omega_{G}^0 \wr A \ar{r}{\Fin \wr V_G} &
                  |[alias=FFOmega]| \Fin^{\wr 2} \wr A \ar{d}{\sigma^0} \ar{r} &
                  \Fin^{\wr 2} \wr \mathcal{V}^{op} \ar{d}{\sigma^0} \ar{r}{\otimes^{op}} &
                  \Fin \wr \mathcal{V}^{op} \ar{r}{\otimes^{op}} &
                  |[alias=dog]|
                  \mathcal{V}^{op} \ar[equal]{d}
                  \\
                  \UC\Sigma_G &
                  |[alias=Omega]|\UC\Omega_{G}^{0} \wr A \ar{rr}[swap]{V_G} \ar{l}&&
                  \Fin \wr A \ar{r} &
                  |[alias=cat]|
                  \Fin \wr \mathcal{V}^{op} \ar{rr}[swap]{\otimes^{op}} &&
                  \mathcal{V}^{op}
                  \arrow[Leftrightarrow, from=FFOmega, to=Omega,shorten <=0.15cm,,shorten >=0.15cm,"\pi_0"]
                  \arrow[Leftrightarrow, from=dog, to=cat,shorten <=0.15cm,,shorten >=0.15cm,"\alpha"]
            \end{tikzcd}
      \end{equation}
      and a unit $\eta: id \Rightarrow N_{\UC}$ give by the strictly commuting diagram
      \begin{equation}\label{UNITSPAN EQ}
            \begin{tikzcd}
                  \UC\Sigma_G \ar[equal]{d} &
                  A \ar{l} \ar{d}[swap]{s_{-1}} \ar[equal]{r} &
                  A \ar{d}{\delta^0} \ar{r} &
                  \mathcal{V}^{op} \ar{d}{\delta^0} \ar[equal]{r}&
                  \mathcal{V}^{op} \ar[equal]{d}
                  \\
                  \UC\Sigma_G &
                  \UC\Omega_{G}^{0} \wr A \ar{l} \ar{r}[swap]{V_G}&
                  \Fin \wr A \ar{r} &
                  \Fin \wr \mathcal{V}^{op} \ar{r}[swap]{\otimes^{op}} &
                  \mathcal{V}^{op}.
            \end{tikzcd}
      \end{equation}	
\end{definition}

\begin{proposition}
      [{cf. \cite[Prop 4.19]{BP17}}]
      $(N_{\UC},\mu,\eta)$ is a monad on $\mathsf{WSpan}^r(\UC\Sigma_G, \V^{op})$.
\end{proposition}
\begin{proof}
      Nothing new.
\end{proof}


\begin{definition}
      The \textit{genuine $\UC$-colored operad monad} is the monad
      $\mathbb F_{G,\UC}$ on $\Sym_{G, \UC}(\V) = \mathsf{Fun}(\UC\Sigma_G^{op}, \V)$ given by
      \begin{equation}
            \mathbb F_{G,\UC} = \Lan \circ N_{\UC} \circ \iota
      \end{equation}
      with multiplication and unit given by
      \begin{equation}
            \mathsf{Lan} \circ N_{\UC} \circ \iota \circ
            \mathsf{Lan} \circ N_{\UC} \circ \iota
            \overset{\simeq}{\Leftarrow}
            \mathsf{Lan} \circ N_{\UC} \circ  N_{\UC} \circ \iota
            \Rightarrow
            \mathsf{Lan} \circ N_{\UC} \circ \iota
      \end{equation}
      \begin{equation}
            id \overset{\simeq}{\Leftarrow} \mathsf{Lan} \circ \iota
            \Rightarrow
            \mathsf{Lan} \circ N_{\UC} \circ \iota.
      \end{equation}
      We will write $\Op_{G,\UC}(\V)$ for the category 
      $\mathsf{Alg}_{\mathbb{F}_{G,\UC}}(\mathsf{Sym}_{G,\UC}(\mathcal{V}))$ of \textit{genuine $\UC$-colored operads}.
\end{definition}



\subsection{Comparison with $\mathfrak C$-colored operads}

We have an inclusion
\begin{equation}
      \begin{tikzcd}[row sep = tiny]
            \UC\Omega \arrow[r, hookrightarrow]
            &
            \UC\Omega_G
            \\
            (U, \mathfrak c) \arrow[r, mapsto]
            &
            (G \cdot U, \mathfrak c: E_G(G \cdot U) \to \UC)
            &
            \mbox{\color{blue} alternatively, $\mathfrak c: \Phi(G \cdot E(U)) \to \UC$}
      \end{tikzcd}
\end{equation}
sending maps $(\phi, g)$ to $g \cdot \phi$.
{\color{blue} Alternatively, sending $(\phi,g)$ to $((h,t) \mapsto (hg^{-1}, \phi(t)))$.}

We have component maps in the opposite direction.
Given $(T = (T_i)_I, \mathfrak c) \in \UC\Omega_G$, we define
$\mathfrak c_i: E(T_i) \to \mathfrak C(G)$ by
\begin{equation}
      \mathfrak c_i(t) = q_t^{\**}(\mathfrak c (G t)),
\end{equation}
where
$q_t: G \to Gt$ sending $e \mapsto t$. 
% $t \in Gf$ (with $f$ minimal in the planar structure on $T$),
% $g_t\in G$ such that $t = g_t \cdot f$ (well-defined up to $G_f$),
% and $(q_t)_{\**}: G/e \to G/G_f$ the map $x \mapsto g G_f$.

% \begin{remark} % this is still true, but less important.
%       The coloring $\mathfrak c_i$ is \textit{almost} the composite
%       \begin{equation}
%             E(T_i) \to E_{G_i}(T_i) \xrightarrow{\simeq} E_G(T) \to \mathfrak C \to G \ltimes \mathfrak C(G/e)
%       \end{equation}
%       where $G_i$ is the stabilizer in $G$ of $T_i$, and
%       $E_{G_i}(T_i) \to E_G(T)$ is the canonical isomorphism sending
%       $e{G_i} \to Gf$
%       with $f \in Ge$ minimal.
%       However, this composite does not record the ``twisting'' action by the element $g_e$.
% \end{remark}


{\color{blue} % ------------------------------ CoLOR BLUE --------------------
  Alternatively, $\mathfrak c_i$ is the composite
  \begin{equation}
        E(T_i) \to E(T) \xrightarrow{\mathfrak c_{G/e}} \mathfrak C(G/e).
  \end{equation}
} % ------------------------------ COLOR BLUE ------------------------------

Then the map $B_I \to \UC\Omega$, $i \mapsto (T_i, \mathfrak c_i)$ is a functor
(as the inclusion and $\mathfrak c_{G/e}$ are $G$-equivariant),
and we have
\begin{equation}
      \iota_{\**}Y(T, \mathfrak c) =
      \left(
            \prod_I Y(T_i, \mathfrak c_i)
      \right)^G.
\end{equation}



\begin{remark}[{cf. \cite[Rem 4.35]{BP17}}]
      Equivalently, the essential image of $\iota_{\**}$ are those sheaves $X \in \Sym_{G, \mathfrak C}(\V)$ such that
      the canonical map
      \begin{equation}
            X(C,\mathfrak c) \xrightarrow{\simeq} X(q^{\**}(C, \mathfrak c))^\Gamma
      \end{equation}
      is an isomorphism, where $q: G \to r(C)$ is the unique map preserving the minimal element, and
      $\Gamma \leq \mathsf{Aut}(q^{\**}(C,\mathfrak c))$ the subgroup preserving the quotient map $q^{\**}C \to C$
      under precomposition.
\end{remark}

\todo[inline]{come back:
  same retraction from free guys,
  same interpretation of essentially image of $\iota_{\**}$ and $\iota_!$,
  $\iota_{\**}\iota^{\**} \to id$ is still an iso on free guys
}


\cite[Prop 4.38]{BP17} follows as before.
In particular:
\begin{proposition}
      \label{FC_MONAD_PROP}
      For every coefficient system $\UC$,
      $\mathbb F^{\mathfrak C}$ is a monad, with category of algebras $\Op^{G,\mathfrak C}(\V)$.
      
      Moreover, $\iota_{!}$ embeds $\Op^{G, \UC}(\V)$ as the full subcategory of $\Op_{G, \UC}(\V)$
      of those objects whose underlying presheaf lives in the subcategory $\Sym^{G, \mathfrak C}(\V)$ of $\Sym_{G, \mathfrak C}(\V)$. 
\end{proposition}

\begin{remark}
      As in \cite{BP17}, the above inclusion allows for $\mathbb F^{G, \UC}$ and $\mathbb F_{G, \UC}$ to be considered in tandem.
      However, homtopically, we want to compare $\Op^{G, \UC}(\V)$ and $\Op_{G, \UC}(\V)$ via the more interesting inclusion
      $\iota_{\**}$.
\end{remark}




\subsection{Free extensions}

\todo[inline]{come back}

Category of labeled planar colored strings, as before.
As most of the construction/category theory is on the level of strings (or formally due to pullback constructions),
``coloring'' does not affect the proofs of \cite[Prop 5.30,5.41]{BP17}.

\begin{proposition}
      [{cf. \cite[Prop 5.48]{BP17}}]
      For any $(T, \mathfrak c) \in \UC\Omega_G^e$ there exists a unqiue $\mathsf{lr}_{\mathcal P}(T, \mathfrak c) \in \UC\hat\Omega_G^e$
      equipped with a unique planar (hence color-fixed) label map in $\UC\Omega_G^e$
      $\mathsf{lr}_{\mathcal P}(T, \mathfrak c) \to (T, \mathfrak c)$.      
      Furthermore, $\mathsf{lr}_{\mathcal P}$ extends to a right retraction
      $\mathsf{lr}_{\mathcal P}: \UC\Omega_G^e \to \UC\hat{\Omega}_G^e$.
\end{proposition}
\begin{proof}
      The underlying $G$-tree of $\mathsf{lr}_{\mathcal P}(T, \mathfrak c)$ is simply $\mathsf{lr}_{\mathcal P}(T)$, and since
      the edges of this tree are a $G$-subset of $E(T)$, the coloring descends uniquely.       
\end{proof}


\begin{proposition}
      [{cf. \cite[Prop 5.37]{BP17}}]
      $\Ran$ behaves well.
\end{proposition}
\begin{proof}
      The proof of this result in the Appendix of \cite{BP17} requires no assumptions on $\Sigma_G$.
      In particular, the proof follows when replacing $\Sigma_G$ with $\UC\Sigma_G$.
\end{proof}

\begin{corollary}
      $\mathcal P[u] = \mathcal P \hat\amalg_{\mathbb F_{\UC}X}\mathbb F_{\UC}Y
      \simeq
      \Lan_{\left(\UC\hat\Omega_G^e \to \UC\Sigma_G\right)^{op}}\tilde N^{(\mathcal P, X, Y)}$.
\end{corollary}


\begin{remark}
      For $T,S \in \UC\Omega_G$, any factorization $T \to T' \to S$ in $\Omega_G$ uniquely extends to one in $\UC\Omega_G$,
      as there is a canonical coloring on $T'$, namely $\Phi E(T') \to \Phi E(S) \to \UC$. 
\end{remark}

The above gives \cite[Lem 5.57]{BP17}.

\begin{proposition}
      [{cf. \cite[Prop 5.66]{BP17}}]
      For each $\UC$-profile $(C,\mathfrak c) \in \UC\Sigma_G$,
      we have the following pushout in $\V^{\mathrm{Aut}(C, \mathfrak c)}$
      \begin{equation}\label{FILTRATION_LAN_LEVEL}
            \begin{tikzcd}
                  \coprod\limits_{[T, \mathfrak d] \in \mathsf{Iso}
                    \left((C,\mathfrak c) \downarrow_{\mathsf{r}} \UC\Omega_G^a[k]\right)}
                  \left(
                        \bigotimes\limits_{v \in V_{G}^{ac}(T)}\P(T_v, \mathfrak d_v) \otimes
                        Q_T^{in}[u]
                  \right)
                  \mathop{\otimes}\limits_{\mathsf{Aut}(T, \mathfrak d)} \mathsf{Aut}(C, \mathfrak c)
                  \arrow[r] \arrow[d] &
                  \P_{k-1}(C, \mathfrak c) \arrow[d] 
                  \\
                  \coprod\limits_{[T, \mathfrak d] \in \mathsf{Iso}
                    \left((C, \mathfrak c) \downarrow_{\mathsf{r}} \UC\Omega_G^a[k]\right)}
                  \left(
                        \bigotimes\limits_{v \in V_{G}^{ac}(T)}\P(T_v, \mathfrak d_v) \otimes
                        \bigotimes\limits_{v \in V_{G}^{in}(T)} Y(T_v, \mathfrak d_v)
                  \right)
                  \underset{\mathsf{Aut}(T, \mathfrak d)}{\otimes} \mathsf{Aut}(C, \mathfrak c)
                  \arrow[r] &
                  \P_k(C, \mathfrak c)
            \end{tikzcd}
      \end{equation}
      where $ V_{G}^{ac}(T)$, $V_{G}^{in}(T)$ denote the active and inert vertices of the underlying $G$-tree $T \in \Omega_G^a[k]$,
      and $Q_T^{in}[u]$ is the domain 
      of the iterated pushout product
      \begin{equation}
            \underset{v \in V_G^{in}(T)}
            {\mathlarger{\mathlarger{\mathlarger{\square}}}}u(T_v, \mathfrak d_v)
            \colon
            Q_T^{in}[u] \to
            \bigotimes\limits_{v \in V_{G}^{in}(T)} Y(T_v, \mathfrak d_v).
      \end{equation}
\end{proposition}
\begin{proof}
      The same proof as before works.
\end{proof}

\subsection{Model structures}


\begin{proposition}
      Let $\V$ be a cocomplete model category will cellular fixed points.
      Now suppose $\mathcal D$ is a groupoid, and let $\F_d$ be a family of subgroups of $\mathsf{Aut}(d)$ for each $d \in \mathcal D$.
      Then the category of diagrams $\V^{\mathcal D}$ has an \textit{$\F$-model structure} $\V^\D_\F$, where
      a map $f: X \to Y$ is a
      weak equivalence (resp. fibration) iff $f(d): X(d) \to Y(d)$ is so in $\V^{\mathsf{Aut}(d)}_{\F_d}$ for each $d \in \mathcal D$.
\end{proposition}
\begin{proof}
      This is the model structure transferred along the adjunction
      \begin{equation}
            \begin{tikzcd}
                  \V^\D \leftrightarrows
                  \mathop{\prod}\limits_{d \in \D}\V^{\mathsf{Aut}(d)}_{\F_d}
            \end{tikzcd}
      \end{equation}
      which exists by a straightforward exercise adapting and combining the proofs of
      \cite[Thm 11.6.1]{Hir03} and \cite[Prop 2.6]{Ste16}.
\end{proof}

\begin{example}
      Let $\V$ be a cocomplete model category with cellular fixed points,
      $\mathfrak C$ a $G$-set, and $\F = \set{\F_n}$ a collection of families $\F_n$ of graph subgroups of $G \times \Sigma_n$.
      Then $\Sym^{G,\mathfrak C}(\V) = \V^{\UC\Sigma^{op}}$ has an $\F$-model structure
      \begin{equation}
            \Sym^{G,\mathfrak C}_\F(\V) = \prod_n \V^{B_{\mathfrak C^{\times n+1}}(G \times \Sigma_n)}_{\F_n},
      \end{equation}
      where
      $(\F_n)_{\xi}$ is all $\Gamma \in \F_n$ such that $\Gamma \leq \mathsf{Aut}(\xi)$.
      % $\V^{B_{\mathfrak C^{\times n+1}}(G \times \Sigma_n)}_{\F_n}$ has the model structure lifted from the adjunction to
      % $\prod_{(c_i)}\V^{\mathsf{Aut}(c_1,\dots,c_n; c_0)}_{\F_{(c_i)}}$ where
      % $\F_{(c_i)}$ is the set of all $\Gamma \in \F_n$ such that $\Gamma \leq \mathsf{Aut}(c_1,\dots, c_n;c_0)$.
\end{example}

On the other hand, let $\Sym_{\F, \mathfrak C}(\V)$ denote the category $\V^{\UC\Sigma_\F}$ with the projective model structure.

% \begin{lemma}
%       \label{EXMAIN_LEM}
%       Let $\V$ be {\color{red} GOOD}.
%       Let $\P \in \Sym_{G, \mathfrak C}(\V)$ be level genuine cofibrant, and
%       $u: X\ to Y$ in $\Sym_{G, \mathfrak C}(\V)$ be a level genuine cofibration.
%       Then, for each $(T, \mathfrak c) \in \UC \Omega_G^a [k]$, and writing $C = \mathsf{lr}(T)$, the map
%       \begin{equation}
%             \left(
%                   \bigotimes_{v \in V_G^{ac}(T)}\P(T_v, \mathfrak c) \otimes
%                   {\mathlarger{\mathlarger{\mathlarger{\square}}}}u(T_v, \mathfrak d_v)
%             \right)
%             \mathop{\otimes}\limits_{\Aut(T, \mathfrak c)}\Aut(C, \mathfrak c)
%       \end{equation}
%       is a genuine cofibration in $\V^{\Aut(C, \mathfrak c)}_{\mbox{gen}}$,
%       which is trivial if $k \geq 1$ and $u$ is trivial.
% \end{lemma}
% \begin{proof}
%       This follows from \cite[Prop 6.24]{BP17} analogously as \cite[Lemma 5.72]{BP17} did.
% \end{proof}

\begin{theorem}
      \label{THM1_C}
      Let $(\V,\otimes)$ denote {\color{red} THINGS}.
      For each $G$-set $\mathfrak C$,
      there exist model structures on $\Op^{G, \mathfrak C}(\V)$ such that
      $\O \to \O'$ is a weak equivalence (resp. fibration) if the maps
      \begin{equation}
            \O(\xi)^\Gamma \to \O'(\xi)^\Gamma
      \end{equation}
      are weak equivalences (resp. fibrations) in $\V$ for all
      $\mathfrak C$-signatures $\xi$ and
      graph subgroups $\Gamma \leq \Stab(\xi)$.

      {\color{red} More generally \ldots}
\end{theorem}


\begin{theorem}
      \label{THM2_C}
      Let $(\V,\otimes)$ denote {\color{red} THINGS}.
      For each $G$-set $\mathfrak C$,
      the projective model structure on $\Op_{G, \mathfrak C}(\V)$ exists.
      Explicitly, a map
      $\P \to \P'$ is a weak equivalence (resp. fibration) if the maps
      \begin{equation}
            \P(C,\mathfrak c) \to \P'(C, \mathfrak c)
      \end{equation}
      are weak equivalences (resp. fibrations) in $\V$ for all
      $(C,\mathfrak c) \in \UC\Sigma_G$.

      {\color{red} More generally \ldots}
\end{theorem}

\begin{proof}[{proof of Theorems \ref{THM1_C} and \ref{THM2_C}}]
      Follow exactly as in \cite{BP17},
      replacing the use of \cite[{(5.67) and Lemma 5.72}]{BP17} with
      the colored analogues, whose proofs are identical to the single-colored cases.
      In particular, the model structures are those lifted across the adjunctions below.
      \begin{equation}
            \label{MAINPFADJ EQ}
            \begin{tikzcd}[column sep =5em]
                  \mathop{\prod}\limits_{(C, \mathfrak c) \in \UC\Sigma_\F}
                  \mathcal{V}^{\mathsf{Aut}(C, \mathfrak c)}_{\text{proj}}
                  \ar[shift left=1.5]{r}
                  &
                  \mathsf{Sym}_{\F, \mathfrak C}(\mathcal{V}) 
                  \arrow[l, shift left=1.5, "\left(\text{ev}_{(C, \mathfrak c)}(\minus)\right)"] 
                  \arrow[r, shift left=1.5,swap,"\mathbb{F}_{G, \mathfrak C}"']
                  &
                  \mathsf{Op}_{\F, \mathfrak C}(\mathcal{V})
                  \ar[shift left=1.5]{l}
                  \\ % ---------- NEXT ROW ----------
                  \mathop{\prod}\limits_{\xi \in \UC\Sigma}
                  \mathcal{V}^{\mathsf{Aut}(\xi)}_{\F_\xi}
                  \ar[shift left=1.5]{r}
                  &
                  \mathsf{Sym}^{G, \mathfrak C}_\F(\mathcal{V}) 
                  \arrow[l, shift left=1.5, "\left(\text{ev}_{\xi}(\minus)\right)"] 
                  \arrow[r, shift left=1.5,swap,"\mathbb{F}^{G, \mathfrak C}"']
                  &
                  \mathsf{Op}^{G, \mathfrak C}_\F(\mathcal{V})
                  \ar[shift left=1.5]{l}
            \end{tikzcd}
      \end{equation}      
\end{proof}

\begin{corollary}
      \label{COLOR_CHANGE_Q_COR}
      The change of color adjunctions from \eqref{COLOR_CHANGE_EQ} form Quillen adjunctions.
\end{corollary}
\begin{proof}
      For any $F: \mathfrak C \to \mathfrak C'$, $F_{\**}$ clearly preserves (trivial) fibrations.
\end{proof}


\subsection{Comparisons and cofibrancy}

\todo[inline]{come back}

\begin{definition}
      Given a coefficient system $\UC$, we define the category $\UC O_{\F_n}$ to be the Grothendieck construction of the functor
      \begin{equation}
            \begin{tikzcd}[row sep = tiny]
                  O_{\F_n}^{op} \arrow[r]
                  &
                  \Set
                  \\
                  G \times \Sigma_n / \Gamma \arrow[r, mapsto]
                  &
                  \Hom_{\Set^{\mathsf O_G^{op}}}(\Phi(G \cdot_\Gamma \underline{n+1}), \mathfrak C)
            \end{tikzcd}
      \end{equation}
      where, for $\phi = \phi_\Gamma: H \to \Sigma_n$ the unique homomorphism such that $\Gamma = \Gamma(\phi)$,
      the (right) action of $\Gamma$ on $G \cdot \underline{n+1}$ is given by
      $(g,i).(h,\phi(h)) = (gh, \phi(h)^{-1}(i))$.
\end{definition}

Equivalently, objects are given by transitive sets $G \times \Sigma_n / \Gamma$
equipped with a $\UC$-coloring on their associated $G$-corollas $C_\Gamma$.

\begin{remark}
      We record the hom-sets in $\UC O_{\F_n}^{op}$ for later use. We have that
      \mbox{$\UC O_{\F_n}^{op}((G \times \Sigma_n/\Gamma, \mathfrak c), (G \times \Sigma_n / \Lambda, \mathfrak d))$}
      is given by those cosets $(x,\tau)\Gamma$ in $G \times \Sigma_n / \Gamma$ such that
      \begin{enumerate}
      \item $(x^{-1},\tau^{-1})\Lambda(x,\tau) \subseteq \Gamma$
            (equivalently, $(x,\tau)\Gamma \in (G \times \Sigma_n/\Lambda)^\Gamma$).
      \item $\mathfrak d(G/H, [g,i]) = \mathfrak c(G/H, [g x, \tau^{-1}(i)]$
            for all $(G/H, [g,i]) \in \Phi(G \cdot_{\Lambda} \underline{n+1}$.
      \end{enumerate}
\end{remark}

\begin{notation}
      For $\UC$ a coefficient system, we will write $B_{\UC^{\times n+1}}$ to mean
      $B_{\UC(G/e)^{\times n+1}}(G \times \Sigma_n)$.
\end{notation}

We have a natural map
\begin{equation}
      \begin{tikzcd}[row sep = tiny]
            B_{\UC^{\times n+1}} \arrow[r]
            &
            \UC O_{\F_n}
            \\
            (c_1,\dots,c_n;c_0) \arrow[r, mapsto]
            &
            (G \times \Sigma_n/e, (g,i) \mapsto g.c_i).
      \end{tikzcd}
\end{equation}

This induces a pair of adjunction
\begin{equation}
      \begin{tikzcd}[column sep =5em]
            \V^{\UC O_{\F_n}^{op}} \arrow[r, "\iota^{\**}"']
            &
            \V^{B_{\UC^{\times n+1}}^{op}}
            \arrow[l, bend right, "\iota_{!}"']
            \arrow[l, bend left, "\iota_{\**}"]
      \end{tikzcd}
\end{equation}

We have explicit formula for $\iota_{\**}$ and $\iota_!$, given by
\begin{align*}
  (\iota_{\**} X)(G \times \Sigma_n/\Gamma, \mathfrak c) &= X(c_1,\dots, c_n;c_0)^\Gamma\\
  (\iota_! X)(G \times \Sigma_n/\Gamma, \mathfrak c) &=
                                                       {\begin{cases}
                                                               X(c_1,\dots,c_n;c_0) \qquad \qquad & G \cdot_\Gamma \underline{n+1} \mbox{ is $G$-free} \\
                                                               \varnothing & \mbox{otherwise,}
                                                       \end{cases}}
\end{align*}
with  $c_i := \mathfrak c(G/e, [e, i])$.

We equip $\V^{\UC O_{\F_n}^{op}}$ with the projective model structure.
\begin{lemma}
      This is a Quillen adjunction.
\end{lemma}
\begin{proof}
      One can check that $\epsilon$ is a natural isomorphsim
      \begin{equation}
            X(c_1,\dots,c_n; c_0)^{\set{e}} \simeq X(c_1,\dots,c_n; c_0).
      \end{equation}
      Moreover, 
      \begin{equation}
            \iota_{\**}\iota^{\**} Y (G \times \Sigma_n / \Gamma, \mathfrak c)
            =
            Y(G \times \Sigma_n / e, (g,i) \mapsto g.\mathfrak c(G/e, [e,i]))^\Gamma,
      \end{equation}
      (where the coloring on the right hand side is given by the composite
      $\Phi(G \cdot \underline{n+1}) \xrightarrow{q} \Phi(G \cdot_\Gamma \underline{n+1})$
      where $q: G \times \Sigma_n \to G \times \Sigma_n / \Gamma$
      preserves the minimal element\todo{sends $[e]$ to $[e]$},
      \todo[inline]{or just $Y(q^{\**}(-))$, with $\UC O_{\F_n}$ a fibration over $\mathsf O_G$}
      and $\eta$ is the natural map.      
      
      We have that $\iota_{\**}$ is right Quillen by construction of the model structures.
\end{proof}

We record that $\iota_{\**}$ again fully-faithful since $\epsilon$ is an iso.

\begin{proposition}
      The above adjunction is an equivalence if and only if $\mathfrak C$ is a $G$-set.
\end{proposition}
\begin{proof}
      Following the proof in \cite{Ste16}, it suffices to compare the following evaluations.
      \begin{align*}
        (\UC O_{\F_n}^{op})((G \times \Sigma_n/\Gamma, \mathfrak c), (G \times \Sigma_n / \Lambda, \mathfrak d) \cdot A
        \\
        \iota_{\**}\iota^{\**}(\UC O_{\F_n}^{op})((G \times \Sigma_n/\Gamma, \mathfrak c), - ) \cdot A)(G \times \Sigma_n / \Lambda, \mathfrak d)        
      \end{align*}
      The elements in the top are those $(x,\tau)$ such that
      \mbox{$\mathfrak c(G/H, [g x, \tau^{-1}(i)]) = \mathfrak d(G/H, [g,i])$}
      while the bottom are those $(x,\tau)$ with the weaker constraint
      \mbox{$\mathfrak c(G/e, [g x, \tau^{-1}(i)]) = \mathfrak d(G/e, [g,i])$}.
      Unless $\UC$ is a $G$-set, we don't even know if we have a map of colorings on levels other than $G/e$,
      let alone whether they satisfy this particular condition.
      \todo[inline]{this is not a complete/full/reasonable argument for the ``if'' direction}.
\end{proof}



% ------------------------------------------------------------
\newpage

\section{Model structure for all colors} 
\renewcommand{\C}{\mathfrak C}

Fix a cofibrantly generated model category $\V$.

{\color{OliveGreen} In this section, assume that the canonical model structures on $\Cat(\V)$, $\Op(\V)$, and $\Op^{G,\C}(\V)$ exist.}

We follow \cite{BM13} and \cite{Cav14} to assemble the collection of ``color-fixed''
($\F$)-model structures into
a single model structure on $\Op^G(\V)$.

As in the previous section, we fix a collection $\F = \set{\F_n}$ of families $\F_n$ of graph subgroupos of $G \times \Sigma_n$

We have another free-forgetful adjunction $j^*: \mathsf{Op}^G(\V) \leftrightarrows \mathsf{Cat}^G(\V): j_!$, and note that $j^*$ commutes with taking $H$-fixed points for all $H\leq G$;
\begin{equation}
      \begin{tikzcd}
            \mathsf{Op}^G(\V) \arrow[d, "(-)^H"']
            \arrow[r, shift right, "j^*"']
            &
            \mathsf{Cat}^G(\V) \arrow[l, shift right, swap, "j_!"] \arrow[d, "(-)^H"]
            \\
            \Op(\V) \arrow[r, shift right, "j^*"']
            &
            \Cat(\V) \arrow[l, shift right, swap, "j_!"]
      \end{tikzcd}
\end{equation}


% \begin{definition}
%       For a category $\V$, a {\em Hopf interval object} in is a cofibrant Hopf object $\H\in\V$ for which there exists a factorization
%       $\begin{tikzcd} 
%             I\coprod I \arrow[r,rightarrowtail] & \H \arrow[r, twoheadrightarrow, "\simeq"] & I
%       \end{tikzcd}$ 
%       of the fold map $I$, the unit in $\V$. We can $\H$ {\em cocommutative} if its Hopf structure is. 
% \end{definition}

\begin{definition}
      % Similarly but different:
      let $\I$ be the $\V$-category with objects $\set{0,1}$ with $\I(0,0) = \I(0,1) = \I(1,0) = \I(1,1) = 1_V$.
      A {\em $\V$-interval} is a cofibrant object in $\Cat^{\set{0,1}}(\V)$ (with the transfered model structure)
      weakly equivalent to $\I$.
      A set $\mathcal{G}$ of $\V$-intervals is {\em generating} if all $\V$-intervals $\J$ can be obtained
      as a retract of a trivial extension of an element in $\mathcal{G}$ in $\Cat^{\set{0,1}}(\V)$:
      \begin{equation}
            \begin{tikzcd}
                  \mathbb{G} \arrow[r,rightarrowtail, "\simeq"]
                  &
                  \mathbb{K} \arrow[r,yshift=-.3em, "r"']
                  &
                  \mathbb{J} \arrow[l,yshift=.3em, "i"']
            \end{tikzcd}
      \end{equation}
\end{definition}

\begin{definition}
      We recall, a functor $F: \mathcal C \to \mathcal D$ in $\Cat(\V)$ is
      \begin{enumerate}
      \item \textit{path-lifting}
            if it has the right lifting property against all maps
            $\1 \to \J$
            where $\1$ is the $\V$-category representing a single object
            (i.e. $\1$ has one object, and mapping object the tensor unit $1_\V$ of $\V$),
            and $\J$ is a $\V$-interval.
      \item \textit{essentially surjective}
            if for any object $d: \1 \to \mathcal D$,
            there is an object $c: \1 \to \mathcal C$
            and a map $\J \to \mathcal D$ out of a $\V$-interval fitting in to the commuting diagram below.
            \begin{equation}
                  \begin{tikzcd}
                        \1 \arrow[rr, dashed, "c"] \arrow[dr, "i_0"]
                        &&
                        \mathcal C \arrow[dd, "F"]
                        \\
                        &
                        \J \arrow[dr, dashed]
                        \\
                        \1 \arrow[ur, " i_1"] \arrow[rr,"b"]
                        &&
                        \mathcal D
                  \end{tikzcd}
            \end{equation}
      \end{enumerate}
\end{definition}

Now and forever we further suppose that $\V$ has \textit{cellular fixed points}.

\begin{definition}
      We call a map $F: \O \to \P$ in $\mathsf{Op}^G(\V)$
      \begin{itemize}
      \item a {\em local $\F$-fibration} (resp. {\em local weak $\F$-equivalence}) if
            $F(\xi): \O(\xi)\to \P(F(\xi))$
            is a fibration (resp. weak equivalence) in $\V^{\Stab(\xi)}_{\F_\xi}$ for all $\xi\in \C(\O)^{\times n+1}$ and all $n$;
      \item a {\em local trivial $\F$-fibration} if both a local $\F$-fibration and a local weak $\F$-equivalence;
      \item {\em essentially surjective} (resp. {\em path lifting}) if $j^*F^H$ is essentially surjective (resp. path lifting) in $\Cat(\V)$ for all $H\leq G$;
      \item a {\em $\F$-fibration} if both path-lifting and a local $\F$-fibration
      \item a {\em weak $\F$-equivalence} if both essentially surjective and a local weak $\F$-equivalence.
      \end{itemize}

      Moreover, foreshadowing, we call such a map
      a \textit{(trivial) $\F$-cofibration} if it has the left lifting property against all trivial $\F$-fibrations (resp. $\F$-fibrations).
\end{definition}

\begin{theorem}
      [\cite{BM13,Cav14}]
      With $G = e$, if $\V$ satisfies (2)-(6) of the assumptions in Theorem \ref{MODEL_THEOREM}, then
      $\Cat(\V)$ and $\Op(\V)$ have a model structure where weak equivalences and fibrations are defined as above.
\end{theorem}


\subsection{Generating (Trivial) Cofibrations and Local Fibrations}

We generalize and combine efforts from \cite{CM13b, BM13, Cav14}.

Fix a graph subgroup $\Gamma \in \F_n$ of $G \times \Sigma_n$, and $X \in \V^\Gamma$.
Define $C_\Gamma[X]$ to be the ``free operad with stabilizer $\Gamma$ generated by $X$''.
Specifically, this operad has colours $\mathfrak C_\Gamma := G \cdot_\Gamma \underline{n+1}$.

Now, letting $\xi_0$ denote the signature $([e,1],[e,2],\dots,[e,n];[e,0])$,
we define the $(G,\mathfrak C_\Gamma)$-symmetric sequence
\begin{equation}
      C_\Gamma[X](\xi) =
      \begin{cases}
            (g,\sigma)^{\**} X \qquad \qquad & \xi = (g,\sigma).\xi_0
            \\
            \varnothing & \mbox{otherwise,}
      \end{cases}
\end{equation}
where $g \in G$ and $\sigma \in \Sigma_n$ are chosen to be the \textit{minimal} elements in those groups with this property,
and $\varnothing$ is the initial object in $\V$.

Let $\mathbb F_\Gamma[X]$ denote free operad $\mathbb F^{\mathfrak C_\Gamma} C_\Gamma[X]$.
It is straightforward that the operad $\mathbb F_\Gamma[X]$ has the universal property
\begin{equation}
      \Hom_{\Op^G(\V)}(\mathbb F_\Gamma[X], \O) = \mathop\prod\limits_{\xi \in (\mathfrak C(\O)^{\times n+1})^\Gamma}\Hom_{\V^\Gamma}(X, \O(\xi)).
\end{equation}

Define $I_{\F,loc}$ and $J_{\F, loc}$ to be the sets
\begin{align*}
  \set{\mathbb F_\Gamma[\Gamma/\Gamma \cdot i]}, \qquad \qquad \set{\mathbb F_\Gamma[\Gamma/\Gamma \cdot j]},
\end{align*}
where $\Gamma$ runs over all graph subgroups of $G \times \Sigma_n$ in $\F_n$,
and $i$ (resp. $j$) runs over all generating (trivial) cofibrations in $\V$.

The universal property makes the following immediate.
\begin{corollary}[{cf. \cite[\S 4.2]{Cav14}, \cite[1.16]{CM13b}}]
      $\O \to \O'$ is a local (trivial) $\F$-fibration iff
      $\O \to \O'$ has the right lifting property against $J_{\F, loc}$ (resp. $I_{\F, loc}$).
\end{corollary}

Now, define $I_{\F}:= I_{\F, loc} \mathbin{\cup} \set{\varnothing \to G/H \cdot \1}_{H\leq G}$
and
$J_{\F} := J_{\F, loc} \mathbin{\cup} \set{G/H \cdot (\1 \to \J)}_{H\leq G,\ \J\in\mathbb{G}}$
where again $\1$ is the initial $\V$-category (considered as an operad), and $\mathbb{G}$ is a generating set of $\V$-intervals. 

\begin{lemma}
      [{cf. \cite[4.8]{Cav14}, \cite[2.3]{BM13}, \cite[1.18]{CM13b}}]
      \label{CAV_4.8}
      Suppose $\V$ has a generating set of intervals.
      A map $F$ in $\mathsf{Op}^G(\V)$ is a trivial $\F$-fibration
      iff
      $F$ is a local trivial $\F$-fibration such that $F^H$ is surjective on $H$-fixed colors for all $H\leq G$
      iff
      $F$ has the right lifting property against $I_{\F}$.. 
\end{lemma}
\begin{proof}
      The second iff is immediate by the construction of $I_{\F}$.
      For the first, we have by definition that
      $F$ is a trivial $\F$-fibration
      iff
      it is a local trivial $\F$-fibration such that $j^*F^H$ is path-lifting and essentially surjective for all $H\leq G$.
      \cite[2.4]{BM13} completes the proof. 
      % Moreover, right lifting against $I_{\F, loc}$ is identical to being a local trivial $\F$-fibration, while
      % lifting against $\varnothing \to G/H\otimes \1$ precisely say that $F^H$ is surjective on colors;
      % combining these observations yields the result.
\end{proof}

\begin{lemma}
      [{cf. \cite[1.20]{CM13b}, \cite[\S 4.3]{Cav14}}]
      $F$ has right lifting against $J_{\F}$ iff $F$ is an $\F$-fibration.
\end{lemma}
\begin{proof}
      Again, lifting against $J_{\F, loc}$ is identical to being a local $\F$-fibration, while lifting against $G/H \cdot (\1 \to \J)$
      is equivalent to $F^H$ lifting against $\1 \to \J$.
\end{proof}

\begin{lemma}
      [{cf. \cite[1.19]{CM13b}}]
      \label{POINT_4_LEMMA}
      $J_{\F}\mbox{-cof} \subseteq I_{\F}\mbox{-cof}$; that is, trivial cofibrations are cofibrations.
\end{lemma}
\begin{proof}
      % It suffices to show that if $F$ has (right) lifting against $I_\F$, it has lifting aginst $J_{\F}$.
      Clearly a local trivial $\F$-fibration is a local $\F$-fibration.
      On the other hand, by locality and \eqref{COLOR_SQ_EQ},
      any cofibration in $\mathsf{Op}^{G, \mathfrak C}(\V)$ for any $G$-set $\C$
      is a cofibration when considered in $\mathsf{Op}^G(\V)$.
      Thus, since $G/H \cdot (\1 \to \1 \amalg \1)$ is a pushout of $G/H \cdot(\varnothing \to \1)$
      and hence is in $I_{\F}\mbox{-cof}$, the composite
      \begin{equation}
            \begin{tikzcd}
                  G/H \cdot \1 \arrow[r, rightarrowtail]
                  &
                  G/H \cdot (\1 \amalg \1) \arrow[r, rightarrowtail]
                  &
                  G/H \cdot \J 
            \end{tikzcd}
      \end{equation}
      is in $I_{\F}\mbox{-cof}$.
      Thus $J_\F \subseteq I_\F\mbox{-cof}$, implying the result.
\end{proof}

\subsection{Trivial cofibrations are weak equivalences}

\begin{lemma}
      \label{TRANSCOMP_ES_LEM}
      Suppose $\V$ is a monoidal model category.
      The transfinite composition of essentially surjective maps is essentially surjective.
\end{lemma}
\begin{proof}
      Since taking fixed points commutes with filtered colimits, they commute with transfinite composition,
      and hence by \cite[4.17]{Cav14}, we are done.
\end{proof}

\begin{lemma}
      [{c.f. \cite[4.20]{Cav14}}]
      \label{J-CELL_LEMMA}
      Suppose the unit $1_\V \in \V$ is cofibrant,
      and weak $\F$-equivalences are closed under transfinite compositions.
      If $\V$ admits (semi)-transfer for operads, then relative $J_{\F}$-cells (with cofibrant source) are weak equivalences.
\end{lemma}
\begin{proof}
      We first assume $\V$ admits transfer for operads.
      
      Since local weak $\F$-equivalences are closed under transfinite composition by assumption, and
      essentially surjective maps are closed under transfinite composition by Lemma \ref{TRANSCOMP_ES_LEM},
      it suffices to prove that the pushout of a map $j \in J_{\F}$ is a weak $\F$-equivalence.

      If $j \in J_{\F, loc}$, then since colimits in $\mathsf{Op}^G(\V)$ are computed in $\Op(\V)$,
      and since by \cite{Cav14} pushouts of this form can be computed fiberwise
      (that is, after shifting the operads into a single color),
      we are computing the pushout of a trivial cofibration in $\mathsf{Op}^{G,\C_{\xi}}(\V)$,
      where $\C_{\xi}$ is the $G$-set generated by the colors in the given signature $\xi$.
      By the existance of the transferred model structure, this is again a trivial cofibration.
      Hence, the pushout is a local weak $\F$-equivalence in $\mathsf{Op}^G(\V)$ which is the identity on colors,
      and hence a weak $\F$-equivalence itself.
      
      Now, supppose $j$ is of the form $G/H \cdot (\1 \to \J)$ for $\J$ a $\V$-interval.
      As in \cite{Cav14}, we can split this pushout into a composition of two pushouts
      \begin{equation}
            \begin{tikzcd}
                  G/H \cdot \1 \arrow[r, "a"] \arrow[d, "G/H \cdot \phi"']
                  % \arrow[dr,phantom, yshift=.1em, xshift=.5em, "\lrcorner" near end]
                  &
                  X \arrow[d,"\phi'"]
                  \\
                  G/H \cdot \J_{\set{0,0}} \arrow[r] \arrow[d, "G/H \cdot \psi"']
                  % \arrow[dr,phantom, yshift=.1em, xshift=.5em, "\lrcorner" near end]
                  &
                  X' \arrow[d,"\psi'"]
                  \\
                  G/H \cdot \J \arrow[r]
                  &
                  Y
            \end{tikzcd}
      \end{equation}
      where $\J_{\set{0,0}}$ is the full subcategory of $\J$ spanned by the object $0$.
      It suffices to show both $\psi'$ and $\phi'$ are local weak $\F$-equivalences which are essentially surjective on fixed points. 
      
      We first consider the bottom pushout.
      We know that $\psi$ is injective on colors and a local isomorphism in $\Op(\V)$,
      and hence so is $G/H \cdot \psi$ in $\Op^G(\V)$.
      Since colimits are created non-equivariantly, and equivariant isomorphisms are detected by invertible equivariant maps,
      \cite[Prop B.22]{Cav14} implies that $\psi'$ is also a local isomorphism in $\Op^G(\V)$,
      so in particular a local weak $\F$-equivalence.

      Moreover, we observe that $\C(Y) = \C(X') \amalg (G/H \times \set{1})$.
      Thus, if $x \in \C(Y)^K$ for $K \leq G$ is in $\C(X')$, we have essential surjectivity trivially,
      as shown below.
      \begin{equation}
            \begin{tikzcd}
                  \1 \arrow[rrr, "x"] \arrow[dr, " i_0"]
                  &&&
                  (X')^K \arrow[dd, "\psi'"]
                  \\
                  &
                  \J \arrow[r]
                  &
                  \1 \arrow[dr, "x"]
                  \\
                  \1 \arrow[ur, " i_1"] \arrow[rrr,"x"]
                  &&&
                  (Y)^K
            \end{tikzcd}
      \end{equation}
      Lastly, if we consider (any element in the orbit of) the new object $1\in \C(Y)^H$,
      there is an associated object $0 \in \C(X')^H$ such that the essentially surjectivity diagram
      factors through the pushout diagram for $\psi$:
      \begin{equation}
            \begin{tikzcd}
                  G/H \cdot \1 \arrow[r,"0"] \arrow[dr, "G/H \cdot i_0"']
                  &
                  G/H \cdot \J_{\set{0,0}} \arrow[r] \arrow[d]
                  &
                  X' \arrow[d, "\psi'"]
                  \\
                  &
                  G/H \cdot \J \arrow[r]
                  &
                  Y \arrow[d, equal]
                  \\
                  G/H \cdot \1 \arrow[ur, "G/H \cdot i_1"] \arrow[rr, "1"]
                  &&
                  Y.
            \end{tikzcd}
      \end{equation}
      Hence $\psi'$ is essentially surjective and a local weak $\F$-equivalence, thus a weak $\F$-equivalence.

      Now, consider the top pushout. \eqref{COLOR_SQ_EQ} again implies that this pushout is created in $\Op^{G, \mathfrak C(X)}(\V)$.
      In particular, this implies $\phi'$ is bijective on objects, and hence essentially surjective.
      Further, since $1_\V$ is both cofibrant and the initial object in $\Op^{\**}(\V)$,
      \cite[Thm. 1.15]{BM13} implies that $\1 \to \J_{\set{0,0}}$ is a trivial cofibration in $\Op^{\**}(\V)$.
      Thus $G/H \cdot \phi$ (resp. $a_! (G/H \cdot \phi)$) is a trivial cofibration in $\Op^{G, \**}(\V)$
      (resp. $\Op^{G, \mathfrak C(X)}(\V)$, by Corollary \ref{COLOR_CHANGE_Q_COR}),
      and hence $\phi'$ is a trivial cofibration in $\mathsf{Op}^{G,\C(X)}(\V)$,
      and thus is a local weak $\F$-equivalence in $\Op^G(\V)$.

      Hence both $\phi'$ and $\psi'$ are weak $\F$-equivalences in $\mathsf{Op}^G(\V)$, so the result is proved.

      If $\V$ only admits semi-transfer, then the assumption that $\O$ is cofibrant is sufficient to run an identical argument
      (as the utilized pushouts remain weak equivalences in the underlying semi-model structure).
\end{proof}


\subsection{2-out-of-3}

\todo[inline]{come back}

\begin{definition}
      If $\V$ has intervals, then in any $\V$-category $\mathcal C$,
      we say that two arrows  $f,g: \1 \to \mathcal C(x,y)$ are \textit{homotopic}
      if there exists a factorization of the form below, with $\mathbb J$ a $\V$-interval.
      \begin{equation}
            \begin{tikzcd}
                  1_\V \amalg 1_\V \arrow[rr, "{(f, g)}"] \arrow[dr, "{(id_0,id_0)}"']
                  &&
                  \mathcal C(x,y)
                  \\
                  &
                  \mathbb J(0,0) \arrow[ur, dashed]
            \end{tikzcd}
      \end{equation}

      Given $\mathcal C \in \Cat(\V)$, define $\pi_0(\mathcal C)$ to be the (unenriched) category with
      the same objects as $\mathcal C$, and $\pi_0(\mathcal C)(x,y) = \Ho(\V)(1_\V, \C(x,y))$.
\end{definition}

We recall some equivalence relations on objects in a $\V$-category \cite{Cav14, BM13}:
\begin{definition}
      Given $\mathcal{C}$ in  $\Cat(\V)$ and $a,b\in\mathrm{Ob}(\mathcal C)$, we say $a$ and $b$ are
      \begin{itemize}
      \item {\em equivalent} if there exists a map $\gamma: \J \to \mathcal C$ such that
            $\gamma i_0 = a$, $\gamma i_1 = b$
            for some $\V$-interval $\J$;
      \item {\em virtually equivalent} if $a$ and $b$ are equivalent in some fibrant replacement
            $\mathcal C_f$ of $\mathcal C$ in $\Cat^{\mathrm{Ob}(\mathcal C)}(\V)$;
      \item {\em homotopy equivalent} if $a$ and $b$ are isomorphic in the unenriched category $\pi_0 \mathcal C_f$
            for some $\mathcal C_f$;
            equivalently, if there exist maps
            $\alpha: 1_\V \to \mathcal C_f(a,b)$ and $\beta: 1_\V\to \mathcal C_f(b,a)$ such that
            $\beta\alpha$ and $\alpha\beta$ and homotopic\footnote{
              Recall that the factorizations in a model category give a notion of \textit{(left) homotopy} $\sim$,
              with $\Ho(\V)(X,Y) = \V(QX,RY)/\sim$.}
            to the identity arrows
            $1_V\to \mathcal C_f(a,a)$ and $1_V \to \mathcal C_f(b,b)$, respectively.
      \end{itemize}
\end{definition}

Equivariantly, we have the following:
\begin{definition}
      Given $\mathcal{C}\in \Cat^G(\V)$ and $a,b\in \mathrm{Ob}(\mathcal{C})$, we say $a$ and $b$ are
      \begin{itemize}
      \item {\em equivalent} if $\Stab_G(a) = \Stab_G(b) =: H$ and are equivalent in $\mathcal{C}^H$;
      \item {\em virtually equivalent} if they are equivalent in some fibrant replacement
            $\mathcal{C}_f$ of $\mathcal{C}$ in $\Cat^{G, \mathrm{Ob}(\mathcal C)}(\V)$;
      \item {\em homotopy equivalent} if $\Stab_G(a) = \Stab_G(b) =: H$ and they are homotopy equivalent in $\mathcal{C}^H$. 
      \end{itemize}
      For an operad $\O\in \mathsf{Op}^G(\V)$ and $a,b\in \C(\O)$, we say $a$ and $b$ are
      {\em equivalent} (resp. {\em virtually equivalent}, {\em homotopy equivalent}) if they are so in $j^*\O$. 
\end{definition}

The following three lemmas follow directly from the proofs of their non-equivariant counterparts:
\begin{lemma}
      [{cf. \cite[4.10]{Cav14}}]
      Equivalence and virtual equivalence define equivalence relations on $\C(\O)$. \qed
\end{lemma}
\begin{lemma}
      [{cf. \cite[4.13]{Cav14}, \cite[2.11]{BM13}}]
      Virtually equivalent colors are homotopy equivalent. 
\end{lemma}
\begin{lemma}
      [{cf. \cite[4.12]{Cav14}, \cite[2.10]{BM13}}]
      \label{RIGHTPROPER_LEM}
      If $\V$ is right proper, then all virtual equivalent colors are equivalent. 
\end{lemma}

\begin{lemma}
      [{cf. \cite[4.11]{Cav14}, \cite[2.9]{BM13}}]
      Any local weak $\F$-equivalence $F: \O\to \P$ in $\mathsf{Op}^G(\V)$ reflects virtual weak equivalences.
\end{lemma}
\begin{proof}
      As in the non-equivariant case, $F$ being a local weak $\F$-equivalence implies
      we have a local trivial $\F$-fibration $F': \O_f\to \P_f$,
      so in particular each $(F')^H$ is path-lifting.
      Thus, for $\Stab(a) = \Stab(b) =: H$, any virtual equivalence $\J \to \P_f^H$ between colors
      $F'(a) = F(a)$ and $F'(b) = F(b)$
      lifts to one $\J \to \O_f^H$ between $a$ and $b$
      (in particular,
      % since the fibration has source $\O_f^H$,
      the source colors $a$ and $b$ have stabilizer at least $H$, and since their images have stabilizer exactly $H$, so do they). 
\end{proof}

\begin{remark}
      We can relax the assumption on the map $\F$ above; however, this will not be needed.
\end{remark}

\subsubsection{Equivalences between levels}

We would like to generalize \cite[4.14 and 4.15]{Cav14}, which state that
$\O(\xi)$ and $\O(\xi')$ are equivalent in $\V$ if $\xi$ and $\xi'$ are related by a string of weak equivalences of colors,
as this would imply that weak $\F$-equivalences satisfy the 2-out-of-3 property.
However, the relevant notion of weak $\F$-equivalence on colours for this paper lives in $\V^{\Aut(\xi)}_{\F_\xi}$ as opposed to $\V$,
and moreover the colors can be interchanged via the action of $G$.
Thus, we will need to be flexible and change an entire orbit worth of colors in order to create the desired homotopy equivalence. 

\begin{proposition}
      [{c.f. \cite[4.14]{Cav14}}]
      \label{CAV_4.14_PROP}
      Given $\O \in \Op^G(\V)$ with colors $\C$, $\xi = (c_1,\dots,c_n;c)$ a signature in $\C$, and $K = \Stab(c)$.
      Moreover, suppose that $c_1$ and $d_1$ are homotopy equivalent.
      Then there exists a zig-zag of weak equivalences in $\V^{\Stab(\xi)}_{\F_\xi}$ between
      $\O(\xi)$ and $\O(\theta)$, where
      $\theta = (d_1,\ldots, d_n; c)$, with the colors $d_i$ defined as follows:
      
      Let $\lambda \subseteq \underline{n} = \set{1,2,\ldots, n}$ denote
      the set of all $i$ such that $c_i = k_i \cdot c_1$ for some $k_i\in K$;
      if $i \notin \lambda$, let $k_i$ denote the identity element of $G$.
      Further, for all $i \in \underline{n}$, define
      \begin{equation}
            \label{DCOLORS_EQ}
            d_i =
            \begin{cases}
                  k_i \cdot d_1 \qquad \qquad & i \in \lambda
                  \\
                  c_i & \mbox{otherwise.}
            \end{cases}
      \end{equation}

      Moreover, any functor $F: \O \to \P$ induces a functorial zig-zag of weak equivalences between
      $\P(F(\xi)$ and $\P(F(\theta))$.
\end{proposition}
\begin{proof}
      Without loss of generality, we may assume $\O$ is fibrant,
      as the fibrant replacement weak equivalences can always be added to any zig-zag.

      Denote the stabilizer of $c_1$ (and hence $d_1$) by $H$, and so we have maps
      $\alpha: 1_\V \to \O^H(c_1,d_1)$ and $\beta: 1_\V \to \O^H(d_1,c_1)$
      realizing their homotopy equivalence.
      For each $i \in \underline{n}$, define
      \begin{equation}
            \label{WEAKEQCOLORS_EQ}
            H_i =
            \begin{cases}
                  k_i H k_i^{-1} \qquad & i \in \lambda
                  \\
                  H & \mbox{else,}
            \end{cases}
            \qquad
            \qquad 
            \alpha_i =
            \begin{cases}
                  1_\V \xrightarrow{\alpha} \O^H_f(c_1;d_1) \xrightarrow{k_i} \O^H_f(c_i;d_i) \qquad & i \in \lambda
                  \\
                  1_\V \xrightarrow{id} \O^H_f(c_i;d_i) & \mbox{else},
            \end{cases}
      \end{equation}
      and $\beta_i$ similarly.
      Note that all of these --- $d_i$, $H_i$, $\alpha_i$, and $\beta_i$ --- are independent of the choice of $k_i\in k_i H$.
      Further, $\Stab(c_i) = \Stab(d_i)$, and
      the pair $(\alpha_i,\beta_i)$ realizes a homotopy equivalence between $c_i$ and $d_i$.

      We firstly claim that $\Stab_{G \times \Sigma_n}(\xi) = \Stab_{G \times \Sigma_n}(\theta)$.
      To that end, suppose $(k,\pi)\in \Stab_{G\times \Sigma_n}(\xi)$, so that $k \cdot c_{\pi^{-1}(i)} = c_i$ for all $i$.
      % We need to show $k d_{\pi^{-1}(i)} = d_i$.
      Thus $\pi$ must act on $\lambda$ and $\underline{n} \setminus \lambda$ independently,
      so for $i \in \lambda$ we have $k \cdot c_{\pi^{-1}(i)} = c_i$, or
      $k k_{\pi^{-1}(i)} \cdot c_1 = k_i \cdot c_1$, and so
      $k_i^{-1} k k_{\pi^{-1}(i)} =:h_i \in H$. Hence 
      \begin{equation}
            k \cdot d_{\pi^{-1}(i)} = k k_{\pi^{-1}(i)} \cdot d_1 = k_i h_r \cdot d_1 = k_i \cdot d_1 = d_i,
      \end{equation}
      as desired.
      On the other hand, if $i \not \in \lambda$, then
      $k \cdot d_{\pi^{-1}(i)} = k \cdot c_{\pi^{-1}(i)} = c_i = d_i$.
      The reverse direction is analogous.
      
      Now, let $\otimes \hat \beta_i$ and $\otimes \hat \alpha_i$ be the composites
      \begin{equation}
            \begin{tikzcd}[row sep = tiny]
                  \otimes \hat \beta_i :
                  % \O(c_1,\dots,c_n;c)
                  % =
                  \O(\xi)
                  \cong
                  \O(\xi) \otimes 1_\V^{\otimes n} \arrow[rr, "1 \otimes\ \bigotimes_i \beta_i"]
                  &&
                  \O(\xi) \otimes \bigotimes_i \O^{H_i}(d_i;c_i) \arrow[r,"\circ"]
                  &
                  % \O(d_1,\dots,d_n;c)
                  % =
                  \O(\theta)
                  \\
                  \otimes \hat \alpha_i : 
                  % \O(d_1,\dots,d_n;c)
                  % =
                  \O(\theta)
                  \cong
                  \O(\theta) \otimes 1_\V^{\otimes n} \arrow[rr, "1 \otimes\ \bigotimes_i \alpha_i"]
                  &&
                  \O(\theta) \otimes \bigotimes_i \O^{H_i}(c_i;d_i) \arrow[r,"\circ"]
                  &
                  % \O(c_1,\dots,c_n;c)
                  % =
                  \O(\xi).
            \end{tikzcd}
      \end{equation}

      We secondly claim that the maps $\otimes \hat \alpha_i$ and $\otimes \hat \beta_i$ descend to $\Lambda$ fixed points 
      for any subgroup $\Lambda \leq \Stab(\xi) = \Stab(\theta)$.
      Indeed, since the composition structure maps of $\O$ are natural in $G$ and $\Sigma$,
      it suffices to show that $\otimes \hat \beta_i$ is preserved by $(k,\pi)\in Stab(\xi)$.
      But we observe this directly:
      \begin{equation}
            (k, \pi) . (\otimes \hat \beta_i)
            =
            \otimes k \hat\beta_{\pi^{-1}(i)}
            =
            \otimes k k_{\pi^{-1}i} \hat \beta
            =
            \otimes k_i h_i \hat \beta
            =
            \otimes k_i \hat \beta
            =
            \otimes \hat \beta_i.
      \end{equation}
      The result for $\otimes \hat \alpha_i$ is analogous.
      
      Since all $\alpha_i$ and $\beta_i$ are homotopy equivalences,
      this second claim implies that the composites $\otimes \hat \beta_i$ and $\otimes \hat \alpha_i$
      induce isomorphisms in the homotopy category of $\V^{\Stab(\xi)}_{\F_\xi}$,
      and hence these are weak equivalences between $\O(\xi)$ and $\O(\theta)$ in $\V^{\Stab(\xi)}_{\F_\xi}$,
      as desired.

      The moreover follows exactly as in \cite{Cav14}.
\end{proof}

\begin{remark}
      \label{CAV_4.14_REM}
      The above proof also works to show an analogous result when given colors $c_j$ and $d_j$ that are homotopy equivalent.
      Furthermore, if we are given $c$ and $d$ homotopy equivalent, we may just take $\theta = (c_1,\dots, c_n;d)$,
      and the same result holds (as $K\times\Sigma_n$ acts trivially on $\O^K(d;c)$).
\end{remark}


% To that end, fix $\O_f$ fibrant in $\mathsf{Op}^G(\V)$ with colors $\C$.
% Supppose $c_1$ and $d_1$ are homotopy equivalent in $\C$, each with stabilizer $H$,

% Further, let $\xi = (c_1,c_2,\dots, c_n;c)$ be a signature in $\C(\O_f)$, with $K := \Stab(c)$.

% We define $\lambda \subseteq \underline{n} = \set{1,2,\ldots, n}$ to be
% the set of all $i$ such that $c_i = k_i \cdot c_1$ for some $k_i\in K$;
% if $i \notin \lambda$, let $k_i$ denote the identity element of $G$.
% Moreover, for all $i \in \underline{n}$, define
% \begin{equation}
%       \label{WEAKEQCOLORS_EQ}
%       H_i =
%       \begin{cases}
%             k_i H k_i^{-1} \qquad & i \in \lambda
%             \\
%             H & \mbox{else,}
%       \end{cases}
%       \qquad
%       d_i =
%       \begin{cases}
%             k_i \cdot d_1 \qquad & i \in \lambda
%             \\
%             c_i & \mbox{else},
%       \end{cases}
%       \qquad 
%       \alpha_i =
%       \begin{cases}
%             1_\V \xrightarrow{\alpha} \O^H_f(c_1;d_1) \xrightarrow{k_i} \O^H_f(c_i;d_i) \qquad & i \in \lambda
%             \\
%             1_\V \xrightarrow{id} \O^H_f(c_i;d_i) & \mbox{else},
%       \end{cases}
% \end{equation}
% and $\beta_i$ similarly.
% Note that each of these is independent of the choice of $k_i\in k_i H$.
% Further, $\Stab(c_i) = \Stab(d_i)$, and
% the pair $(\alpha_i,\beta_i)$ realizes a homotopy equivalence between $c_i$ and $d_i$.
% %as post-composition by an isomorphism remains an isomorphism in $\mathrm{Ho}\V$.
% We denote the new signature $\theta = (d_1,\cdots,d_n;c)$.

% \begin{lemma}
%       With the above notation, $\Stab_{G \times \Sigma_n}(\xi) = \Stab_{G \times \Sigma_n}(\theta)$. 
% \end{lemma}
% \begin{proof}
%       Suppose $(k,\pi)\in \Stab_{G\times \Sigma_n}(\xi)$, so that $k \cdot c_{\pi^{-1}(i)} = c_i$ for all $i$.
%       % We need to show $k d_{\pi^{-1}(i)} = d_i$.
%       Thus $\pi$ must act on $R$ and $\underline{n} \setminus R$ independently,
%       so $i \in \lambda$ we have $k \cdot c_{\pi^{-1}(i)} = c_i$, or
%       $k k_{\pi^{-1}(i)} \cdot c_1 = k_i \cdot c_1$, and so
%       $k_i^{-1} k k_{\pi^{-1}(i)} =:h_i \in H$. Hence 
%       \begin{equation}
%             k \cdot d_{\pi^{-1}(i)} = k k_{\pi^{-1}(i)} \cdot d_1 = k_i h_r \cdot d_1 = k_i \cdot d_1 = d_i,
%       \end{equation}
%       as desired.
%       On the other hand, if $i \not \in \lambda$, then
%       $k \cdot d_{\pi^{-1}(i)} = k \cdot c_{\pi^{-1}(i)} = c_i = d_i$.

%       The converse is similar.
% \end{proof}

% With the same notation as above, let $\otimes \hat \beta_i$ and $\otimes \hat \alpha_i$ be the composites
% \begin{equation}
%       \begin{tikzcd}[row sep = tiny]
%             \otimes \hat \beta_i :
%             % \O_f(c_1,\dots,c_n;c)
%             % =
%             \O_f(\xi)
%             \cong
%             \O_f(\xi) \otimes 1_\V^{\otimes n} \arrow[rr, "1 \otimes\ \bigotimes_i \beta_i"]
%             &&
%             \O_f(\xi) \otimes \bigotimes_i \O_f^{H_i}(d_i;c_i) \arrow[r,"\circ"]
%             &
%             % \O_f(d_1,\dots,d_n;c)
%             % =
%             \O_f(\theta)
%             \\
%             \otimes \hat \alpha_i : 
%             % \O_f(d_1,\dots,d_n;c)
%             % =
%             \O_f(\theta)
%             \cong
%             \O_f(\theta) \otimes 1_\V^{\otimes n} \arrow[rr, "1 \otimes\ \bigotimes_i \alpha_i"]
%             &&
%             \O_f(\theta) \otimes \bigotimes_i \O_f^{H_i}(c_i;d_i) \arrow[r,"\circ"]
%             &
%             % \O_f(c_1,\dots,c_n;c)
%             % =
%             \O_f(\xi).
%       \end{tikzcd}
% \end{equation}

% \begin{lemma}
%       The map $\otimes \hat \beta_i$ descends to $\Gamma$ fixed points for any subgroup $\Lambda\leq \Stab(\xi) = \Stab(\theta)$.
% \end{lemma}
% \begin{proof}
%       Since the composition structure maps of $\O_f$ are natural in $G$ and $\Sigma$,
%       it suffices to show that $\otimes \hat \beta_i$ is preserved by \textit{all} $(k,\pi)\in Stab(\xi)$.
%       But we observe this directly:
%       \begin{equation}
%             (k, \pi) . (\otimes \hat \beta_i)
%             =
%             \otimes k \hat\beta_{\pi^{-1}(i)}
%             =
%             \otimes k k_{\pi^{-1}i} \hat \beta
%             =
%             \otimes k_i h_i \hat \beta
%             =
%             \otimes k_i \hat \beta
%             =
%             \otimes \hat \beta_i.
%       \end{equation}
% \end{proof}

% Since all $\alpha_i$ and $\beta_i$ are homotopy equivalences,
% the above lemma implies that the composites $\otimes \hat \beta_i$ and $\otimes \hat \alpha_i$
% induce an isomorphism in the homotopy category of $\V^{\Stab(\xi)}_{\F_\xi}$,
% and hence these are weak equivalences between $\O_f(\xi)$ and $\O_f(\theta)$ in $\V^{\Stab(\xi)}_{\F_\xi}$.

% This discussion yields the following.
% \begin{corollary}
%       [{c.f. \cite[4.14]{Cav14}}]
%       \label{CAV_4.14_PROP}
%       Let $\O$ be any operad in $\mathsf{Op}^G(\V)$,
%       $\xi = (c_1,\ldots,c_n;c)$ a signature in $\C(\O)$,
%       $K = \Stab_G(c)$,
%       and suppose either $(c_j,d_j)$ or $(c,d)$ is a pair of homotopy equivalent colors.
%       Then there exists a zig-zag of weak equivalences in $\V^{\Stab(\xi)}_{\F_\xi}$ between $\O(\xi)$ and $\O(\theta)$,
%       where $\theta = (d_1,\dots,d_n;c)$ or $(c_1,\dots,c_n;d)$
%       with $d_i$ defined as in \eqref{WEAKEQCOLORS_EQ}.
%       % where $\xi' = (d_1,\ldots, d_n;d)$ with
%       % \begin{itemize}
%       % \item $R\subseteq \set{1,\ldots,n}$ those $r$ such that $c_r\in Kc_i$; in particular, choose $k_r\in K$ with $k_r c_i = c_r$ (if $i\not\in R$, let $k_i = 1$); and
%       % \item $d_i = k_i c_i$.
%       % \end{itemize}
%       Moreover, any functor $F:\O\to \P$ induces a functorial zig-zag of weak equivalences between $\P(F(\xi))$ and $\P(F(\xi'))$.
% \end{corollary}
% \begin{proof}
%       In the first case, without loss of generality we may take $j = 1$,
%       and then the result follows by the above discussion.
%       Similarly, if the homotopy equivalence between $c$ and $d$ is realized by the pair of maps $\alpha_0$ and $\beta_0$,
%       then the analogous composites $\hat \alpha_0$ and $\hat \beta_0$ descend to all fixed points
%       (as $K\times\xi_n$ acts trivially on $\O_f^K(d;c)$),
%       and hence we again have a weak equivalence
%       $\O_f(c_1,\ldots, c_n;c) \to \O_f(c_1,\ldots, c_n;d)$
%       in $\V^{\Stab{\xi}}_{\F_\xi}$.
      
%       Lastly, the fibrant replacement weak equivalences $\O(\xi)\to\O_f(\xi)$ and $\O(\xi')\to \O_f(\xi')$ complete the zig-zag.
 
%      The second statement follows identically as in the non-equivariant case found in \cite[4.14]{Cav14}.
% \end{proof}
 
\begin{proposition}
      [{c.f. \cite[4.15]{Cav14}}]
      \label{CAV_4.15_PROP}
      Suppose $\V$ is right proper.
      The class of weak $\F$-equivalences in $\mathsf{Op}^G(\V)$ satisfies the 2-out-of-3 condition.
\end{proposition}
\begin{proof}
      Essential surjectiving holds in all cases since it reduces to checking multiple instances of the non-equivariant case,
      where it holds via \cite[4.15]{Cav14}.
      Now let $\O_1 \xrightarrow{F} \O_2 \xrightarrow{L} \O_3$ be a composition of maps in $\mathsf{Op}^G(\V)$.
      If $F$ and $L$ are weak $\F$-equivalences,
      the composite is obviously a local weak $\F$-equivalence:
      $\O_1(\xi)^\Gamma \simeq \O_2(F(\xi))^\Gamma \simeq \O_3(LF(\xi))^\Gamma$.
      If $L$ and $FL$ are weak $\F$-equivalences,
      then $F$ is by 2-out-of-3 in each $\V^{\Stab(\xi)}_{\F_\xi}$.

      Lastly, suppose $F$ and $LF$ are weak $\F$-equivalences.
      Given a signature $\theta = (d_1,\ldots,d_n;d$) in $\C(\O_2)$, let $K = \Stab(d)$,
      Now, let $\Lambda = \lambda_1 \amalg \dots \amalg \lambda_r$ denote the partition of $\underline{n}$
      where $i < j$ are in the same class iff there exists $k_{i,j} \in K$ such that $d_j = k_{i,j} \cdot d_i$.
      Define $R \subseteq \underline{n}$ to be the subset of minimal representatives in each class,
      and $H_r$ the stabilizer in $G$ of $c_r$.
k
      By the essential surjectivity of $F$, there exist $c_r \in \C(\O_1)$ such that
      $\Stab(c_r) = \Stab(d_r)$ and $F(c_r)$ is equivalent, and hence homotopy equivalent, to $d_r$.
      Similarly, there exists $c\in \C(\O_1)$ such that $\Stab(c) = \Stab(d)$ with $F(c)$ and $d$ homotopy equivalent. 

      Now, we extend the set $\set{c_r}_{r\in R}$ to a signature $(c_1,\ldots, c_n;c)$ by defining $c_j = k_{r,j} \cdot c_r$
      (again, these are independent of the choice of $k_{r,j} \in k_{r,j}H_r$).

      Consequently, $F(c_i)$ is homotopy equivalent to $d_i$ via $k_{r,i}\gamma_r$,
      where $\gamma_r$ realizes the homotopy equivalence between $F(c_r)$ and $d_r$ for $i \in \lambda_r$.
      % In particular, these homotopy equivalences are coherent, as in the proof of \ref{CAV_4.14_PROP}.
      We have a diagram of the form
      \begin{equation}
            \label{TWOOFTHREE_EQ}
            \begin{tikzcd}
                  \O_1(c_1,\ldots, c_n;c) \arrow[r, "(1)"]
                  &
                  \O_2(F(c_1),\ldots, F(c_n); F(c)) \arrow[d,dash, "(3)"] \arrow[r, "(2)"]
                  &
                  \O_3(LF(c_1),\ldots, LF(c_n);LF(c)) \arrow[d, dash, "(4)"]
                  \\
                  &
                  \O_2(d_1,\ldots, d_n;d) \arrow[r, "(5)"]
                  &
                  \O_3(L(d_1),\ldots, L(d_n); L(d)).
            \end{tikzcd}
      \end{equation}
      $(1)$ is a weak equivalence in $\V^{\Stab(\xi)}_{\F_\xi}$ by assumption,
      $(2)$ is a weak-equivalence by 2-out-of-3 in $\V^{\Stab(\xi)}_{\F_\xi}$, and
      $(3)$ and $(4)$ are zig-zags of weak equivalences by iterating applications of \ref{CAV_4.14_PROP} or \ref{CAV_4.14_REM},
      as each application only changes the colours in a particular partition class.
      As these zig-zags are functorial, the above diagram commutes.
      Thus $(5)$ is a weak equivalence again by 2-out-of-3 in $\V^{\Stab(\xi)}_{\F_\xi}$, and hence
      $L$ is a local weak $\F$-equivalence, as desired.
\end{proof}


\subsection{Model structure}

The following two theorems have the same proof.

\begin{theorem}
      \label{SSET_MODEL_THM}
      Suppose $(\V, \otimes)$ is either $(\sSet, \times)$ or $(\sSet_{\**}, \wedge)$.
      Then there exists a cofibrantly generated semi-model structure on $\mathsf{Op}^G(\sSet)$ with
      $\F$-fibrations, weak $\F$-equivalences, generating $\F$-cofibrations, and generating trivial $\F$-cofibrations as described above.
\end{theorem}

\begin{theorem}
      \label{MODEL_THEOREM}
      Suppose $(\V, \otimes)$:
      \begin{enumerate}[label = (\roman*)]
      \item is a cofibrantly generated model category,
      \item is a closed monoidal model category with cofibrant unit \footnote{Cofibrant unit also needed for \ref{J-CELL_LEMMA}.}
      \item has cellular fixed-point functors,
            % which commute with a symmetric monoidal fibrant replacement functor in $\V$.
      \item has cofibrant symmetric pushout powers
            % --------------------
      \item is right proper
            \footnote{Needed for \ref{RIGHTPROPER_LEM}, \ref{CAV_4.15_PROP}.},
      \item there exists a set $\mathbb{G}$ of generating $\V$-intervals
            \footnote{Needed so we have a \textit{set} of generating trivial cofibrations}, and
      \item the class of genuine weak equivalences in $\mathsf{Op}^G(\V)$ is closed under transfinite compositions.
      \end{enumerate}
      Then, for all weak indexing systems $\F$,
      there exists a cofibrantly generated semi-model structure $\Op^G_\F(\V)$ on $\mathsf{Op}^G(\V)$ with
      $\F$-fibrations, weak $\F$-equivalences, generating $\F$-cofibrations, and generating trivial $\F$-cofibrations as described above.
\end{theorem}

\begin{proof}
      In the first case, we have the model category $\Op^{G,\mathfrak C}_\F(\sSet)$
      for any $G$-set $\mathfrak C$ and weak indexing $\F$,
      while in the second case, we only have a semi-model category $\Op^{G, \mathfrak C}_\F(\V)$.
      After this difference, the proofs are identical,
      once we establish that
      \begin{enumerate*}
      \item[$(v)$] $\sSet$ is right proper,
      \item[$(vi)$] $\sSet$ has a generating set of intervals, and
      \item[$(vii)$] the class of genuine weak equivalences in $\mathsf{Op}^G(\sSet)$ is closed under transfinite compositions.
      \end{enumerate*}
      $(v)$ follows by e.g. \cite[Prop 2.1.5]{Cis06},,
      $(vi)$ follows by e.g. \cite[Lemma 1.12]{BM13}, and
      $(vii)$ follows by an argument analogous to \cite[Lemma 1.24]{CM13b}.

      Now, we note that condition $(vii)$ proves the analogous statement for any $\F$,
      since the transfinite composite of local $\F$-equivalences is a local $\F$-equivalence.
      Since $\mathsf{Op}^G(\V)$ is complete and cocomplete, it thus suffices to prove (following \cite{Hov98} Theorem 2.1.19) that:
      \begin{enumerate}[label = (\arabic*)]
      \item the class of weak $\F$-equivalences has the 2-out-of-3 property and is closed under retracts;
      \item the domains of $I_{\F}$ (resp. $J_{\F}$) are small relative to $I_{\F}$-cell (resp. $J_{\F}$-cell);
      \item $I_{\F}$-inj $= W\cap J_{\F}$-inj;
      \item $J_{\F}$-cell $\subseteq W\cap I_{\F}$-cof.
      \end{enumerate}
      (1) follows from \ref{CAV_4.15_PROP} and the fact that if $L$ is a retract of $F$, $L^H$ is a retract of $F^H$.
      (2) follows since colimits in $\mathsf{Op}^G(\V)$ are created in $\Op(\V)$, and it holds non-equivariantly.
      (3) follows from \ref{CAV_4.8}.
      (4) follows from \ref{POINT_4_LEMMA} and \ref{J-CELL_LEMMA}.
\end{proof}


Next goal:
\begin{theorem}
      Weak equivalences are Dwyer-Kan equivalences.
\end{theorem}














%%%%%%%%%%%%%%%%%%%%%%%%%%%%%%%%%%%%%%%%%%%%%%%%%%%%%%%%%%%%%%%%%%%%%%%
\newpage

\section{In $\mathsf{dSet}_G$}

\begin{definition}
      Define the \textit{genuine operadic nerve} $N: \Op_G \to \dSet_G$ by
      \begin{equation}
            N\P(T) = \Hom_{\Op_G}(T, \P)
      \end{equation}
      where we think of $T$ as the operad $T \in \Op^G \into \Op_G$. 
\end{definition}

\begin{remark}
      We note that $N\P \in (SCI)^{\boxslash !}$,
      as $T \in \Op_G$ is a free $\mathbb F_G$-algebra on its vertices.
\end{remark}

\begin{remark}
      We can rephrase the definition of being an $\mathbb F_G$-algebra in terms of $N\P$.
      For $\P \in \Sym_G$ a $G$-symmetric sequence,
      a genuine $G$-operad structure on $\P$ is given by:
      \begin{itemize}
      \item Composition Maps: $ $\\
            maps 
            $N\P(T) \to \P(\mathsf{lr}(T))$
            for all $T \in \Omega_G$.
      \item Naturality under restriction and conjugation: $ $\\
            maps $N\P(T_1) \to N\P(T_0)$
            for all quotient maps $T_0 \to T_1$ in $\Omega_{G,0}$,
            such that the following commutes:
            \begin{equation}
                  \begin{tikzcd}
                        N\P(T_1) \arrow[r] \arrow[d]
                        &
                        \P(\mathsf{lr}(T_1)) \arrow[d]
                        \\
                        N\P(T_0) \arrow[r]
                        &
                        \P(\mathsf{lr}(T_0)).
                  \end{tikzcd}
            \end{equation}
      \item Associativity under $\mathbb F_G$: $ $\\
            maps $N\P(T_1) \to N\P(T_0)$
            for all planar tall maps $T_0 \to T_1$ in $\Omega_G^t$,
            such that the analogus diagram (with the right vertical map the identity) commutes.\footnote{
              As in \cite{BP17}, we note that ``associativity'' under $\mathbb F_G$ includes both
              the usual notion of associativity of our composition maps,
              but also unitality;
              this is recorded here by the fact that degeneracies are always planar tall.}
      \end{itemize}
\end{remark}

The above reflects the following result.

\begin{proposition}
      $\Op_G$ is equivalent to the subcategory of $\mathsf{dSet_G}$ spanned by those $X$ such that
      \begin{enumerate}
      \item $X(H/H) = \set{\**}$ for all $H \leq G$.
      \item $X(T) \cong \otimes_{T_v \in V(T)}X(T_v)$. 
      \end{enumerate}
\end{proposition}
\begin{proof}
      The fact that $N\P \in (SCI_{\F})^{\boxslash !}$ is immediate, as remarked above.

      For the reverse direction, we will follow the construction of the homotopy operad as in \cite[\S 6]{MW09},
      replacing their use of inner horn inclusions with \textit{orbital} inner $G$-horn inclusions,
      to show that any $X \in (OHI)^{\boxslash !}$ is in the image of $N$; 
      the result will then follow from \cite[HYPER PROP]{BP18}.

      In fact, interpreting all of their pictures are as \textit{orbital} representations of $G$-trees yields that,
      for all $C \in \Sigma_G$
      \begin{itemize}
      \item $\sim_{G e}$ is an equivalence relation on $X(C)$ for all $Ge \in E_G(C)$.
      \item The relations $\sim_{G e}$ and $\sim_{G e'}$ are equal for all $e,e'\in E(C)$.
      \item $[h] \circ [f] = [h \circ f]$ yields a well-defined composition map.
      \end{itemize}
      \todo[inline]{naturality, associativity of composition}
\end{proof}



\newpage



\section{Scratchwork}

\subsection{Colored simplicial tensors and cotensors}



\[
\begin{tikzcd}
	K \otimes f^{\**} P \ar{r} \ar{ddd}&
	K \otimes f^{\**} \left( (K \otimes P)^K \right) \ar{r}{\simeq} \ar{d}&
	K \otimes \left( f^{\**} (K \otimes P) \right)^K \ar{r} \ar{d} &
	f^{\**} (K \otimes P) \ar{ddd}
\\
	&
	K \otimes f^{\**} \left( (L \otimes P)^K \right) \ar{d}
	\ar{r}{\simeq} &
	K \otimes \left( f^{\**} (L \otimes P) \right)^K
	\ar{d}
\\
	&
	L \otimes f^{\**} \left( (L \otimes P)^K \right)
	\ar{r}{\simeq} &
	L \otimes \left( f^{\**} (L \otimes P) \right)^K
\\
	L \otimes f^{\**} P \ar{r} &
	L \otimes f^{\**} \left( (L \otimes P)^L \right) \ar{r}{\simeq} \ar{u} &
	L \otimes \left( f^{\**} (L \otimes P) \right)^L \ar{r}&
	f^{\**} (L \otimes P)
\end{tikzcd}
\]



\[
\begin{tikzcd}
	K \otimes f^{\**} P \ar{r} \ar{d}&
	K \otimes f^{\**} \left( (K \otimes P)^K \right) \ar{r}{\simeq} &
	K \otimes \left( f^{\**} (K \otimes P) \right)^K \ar{r}  &
	f^{\**} (K \otimes P) \ar{ddd}
\\
	K \otimes \left( (L \otimes f^{\**} P) \right)^L \ar{d} &
	K \otimes \left( (L \otimes f^{\**} \left( (K \otimes P)^K \right)) \right)^L &
\\
	K \otimes \left( (L \otimes f^{\**} P) \right)^K \ar{d} &
	K \otimes \left( (L \otimes f^{\**} \left( (K \otimes P)^K \right)) \right)^K
\\
	L \otimes f^{\**} P \ar{r} &
	L \otimes f^{\**} \left( (L \otimes P)^L \right) \ar{r}{\simeq} &
	L \otimes \left( f^{\**} (L \otimes P) \right)^L \ar{r}&
	f^{\**} (L \otimes P)
\end{tikzcd}
\]




\[
\begin{tikzcd}
	f^{\**} P \ar{rr} \ar{rd} \ar{dd} & &
	f^{\**} \left( (K \otimes P)^K \right) \ar{d} &
	\left( f^{\**} (K \otimes P) \right)^K \ar{l}[swap]{\simeq} \ar{ddd}
\\
	&
	f^{\**} \left( (L \otimes P)^L \right) \ar{r} &
	f^{\**} \left( (L \otimes P)^K \right) 
\\
	\left( L \otimes f^{\**} P \right)^L \ar{r} \ar{d} &
	\left( f^{\**}( L \otimes P ) \right)^L \ar{u}{\simeq} \ar{rrd} 
\\
	\left( L \otimes f^{\**} P \right)^K \ar{rrr} &&&
	\left( f^{\**}( L \otimes P ) \right)^K \ar{uul}[swap]{\simeq}
\end{tikzcd}
\]


\newpage


\subsection{Semi-cofibrantly generated}


The following codifies a formal argument implicit in the proof of \cite[Thm. 7.19]{CM13b}.

\begin{definition}
Given a set $J$ of maps that admit the small object argument, we say that $X \in \mathcal{M}$ is \textit{$J$-fibrant} if $X \to \**$ has the right lifting property against maps in $J$.

Further, given $D$ a class of maps in $\mathcal{M}$,
we write $D_{J\text{-fib}} \subseteq D$ to denote 
the subclass of maps whose target is $J$-fibrant.
\end{definition}

\begin{lemma}
	Let $\mathcal{M}$ be a model category with $(C,W,F)$
	the corresponding classes of cofibrations, weak equivalences and fibrations. 
	Further, $J$ be a set of maps admitting the small object argument and such that:
\begin{itemize}
	\item[(i)] $J \subseteq C \cap W$;
	\item[(ii)] 
	$\left(J^{\boxslash} \cap W \right)_{J\text{-fib}}
	\subseteq \left( F \cap W \right)_{J\text{-fib}}$.
\end{itemize}
Then one further has that:
\begin{itemize}
	\item[(a)]
	$\left(\prescript{\boxslash}{}{\left(J^{\boxslash}\right)}\right)_{J\text{-fib}}
	= 
	\left( C \cap W \right)_{J\text{-fib}}$;
	\item[(b)]
	$\left(J^{\boxslash} \right)_{J\text{-fib}}
	= F_{J\text{-fib}}$.
\end{itemize}
\end{lemma}

\begin{remark}
Rephrasing (b), one has that the fibrant objects of $\mathcal{M}$ are precisely the $J$-fibrant objects
and thus that the fibrations between fibrant objects are precisely the $J$-fibrations.
\end{remark}

\begin{proof}
	To check (a), recalling first that 
	$\prescript{\boxslash}{}{\left(J^{\boxslash}\right)}$
	is the saturation of $J$, one has that (i) in fact implies 
	$\prescript{\boxslash}{}{\left(J^{\boxslash}\right)}
		\subseteq C \cap W $.
	For the converse direction, given a trivial cofibration
	$A \to Y$ with $J$-fibrant target,
	form the factorization 
	$A \to X \to Y$ as a 
	$J$-cofibration followed by a $J$-fibration. 
	By the first direction the map $A\to X$ is a weak equivalence, and thus by 2-out-of-3 so is $X \to Y$.
	But then by (ii) the map $X \to Y$ is a trivial fibration, so that the lifting below exists,
	showing that $A \to Y$ is a retract of $A \to X$, and thus also in the saturation $\prescript{\boxslash}{}{\left(J^{\boxslash}\right)}$, 
	as desired.
\[
\begin{tikzcd}
	A \ar[>->]{r}{J} \ar[>->]{d}[swap]{\sim}&
	X \ar[->>]{d}{J}
%& &
%	A \ar[>->]{r}{\sim} \ar[>->]{d}[swap]{\sim}&
%	Y \ar[->>]{d}{\sim}
\\
	Y \ar[equal]{r} \ar[dashed]{ru} & Y
%& &
%	X \ar[equal]{r} \ar[dashed]{ru} & X
\end{tikzcd}
\]

To check (b), one direction is again immediate from (i),
since $J^{\boxslash} \supseteq (C \cap W)^{\boxslash} = F$.
For the converse direction, it suffices to show that 
a $J$-fibration $X\to Y$ with $J$-fibrant target has the right lifting property against trivial cofibrations, as on the left diagram below.
After factoring the bottom horizontal map as a $J$-cofibration followed by a $J$-fibration as on the right diagram, it suffices to shows that a lift $B' \to X$ exists.
But since $B'$ is $J$-fibrant, this follows from (a), which shows that the composite $A \to B \to B'$ is a $J$-cofibration.
\[
\begin{tikzcd}
	A \ar{r} \ar[>->]{d}[swap]{\sim}&
	X \ar[->>]{d}{J}
&&
	A \ar{rr} \ar[>->]{d}[swap]{\sim}&&
	X \ar[->>]{d}{J}
\\
	B \ar{r} \ar[dashed]{ru} & Y
&&
	B \ar[>->]{r}[swap]{J} &
	B' \ar[->>]{r}[swap]{J} \ar[dashed]{ru}
	& Y
\end{tikzcd}
\]
\end{proof}

\begin{remark}
	Analyzing the proof above, one is free to replace the class of fibrant objects with any other class that is compatible with $J$-fibrations, in the sense that if 
	$X \to Y$ is a $J$-fibration and $Y$ is in the class, then so is $X$.
\end{remark}





\subsection{Formalizing some stuff}

The following is a reformalized proof of \cite[Thm. 8.14]{CM13b}.


\begin{proposition}
The (right) derived composite functors in the following diagram commute up to a zigzag of weak equivalences. 
\[
\begin{tikzcd}
	\mathsf{PreOp} \ar{d}[swap]{\gamma^{\**}}&
	\mathsf{sOp} \ar{l}[swap]{N} \ar{d}{hcN}
\\
	\mathsf{sdSet} &
	\mathsf{dSet} \ar{l}{c_{!}}
\end{tikzcd}
\]
\end{proposition}

Note that though $\gamma^{\**}$ and $c_{!}$ are left Quillen, they both preserve all equivalences, 
so that one needs only perform fibrant replacements in 
$\mathsf{sOp}$.

\begin{proof}
	Recall that, given an object $X$ in a model category $\mathcal{M}$, a simplicial frame of $X$ is a fibrant replacement
	$c_!(X) \to \widetilde{X}(\bullet)$ of the constant 
	simplcial object $c_!(X)$ in the Reedy model structure on $\mathcal{M}^{\Delta^{op}}$.
	Moreover, if $X$ was already fibrant one is free to assume that $\widetilde{X}(0) = X$.
	
	Let $\mathcal{O} \in \mathsf{sOp}$ be fibrant, 
	choose a (functorial) fibrant simplicial frame
	$\widetilde{\mathcal{O}}(\bullet) \in \mathsf{sOp}^{\Delta^{op}}$, where we assume $\widetilde{\mathcal{O}} (0) = \mathcal{O}$.
	Next, let 
	$\gamma^{\**} N \widetilde{\mathcal{O}}(\bullet) 
	\to \widetilde{Q}(\bullet)$
	be a Reedy fibrant replacement in  
	$\mathsf{sdSet}^{\Delta^{op}}$.
	
	We claim that the following is a zigzag of weak equivalences in $\mathsf{sdSet}$.
\begin{equation}\label{BIGZIG EQ}
	\gamma^{\**} N \mathcal{O} \xrightarrow{\sim}
	\widetilde{Q}(0) \xrightarrow{\sim}
	\delta^{\**} \widetilde{Q} \xleftarrow{\sim}
	\widetilde{Q}_0 \xleftarrow{\sim}
	\left(\gamma^{\**} N \widetilde{\mathcal{O}}\right)_0
	\xrightarrow{\sim}
	hcN \widetilde{\mathcal{O}} \xleftarrow{\sim}
	c_{!} hcN \mathcal{O}
\end{equation}
That the first map is an equivalence is obvious from definition of $\widetilde{Q}$ and the assumption $\widetilde{\mathcal{O}}(0) = \mathcal{O}$.

For the second and third maps, note first that $\widetilde{Q}$ is homotopically constant, in the sense that all structure maps $\widetilde{Q}(m) \to \widetilde{Q}(m')$
are equivalences.
Moreover, since the levels $\widetilde{Q}$ are fibrant in 
$\mathsf{sdSet}$, this implies that these are simplicial equivalences, i.e. that for each tree $T \in \Omega$
the evaluations 
$\widetilde{Q}(T)(m) \to \widetilde{Q}(T)(m')$
are Kan equivalences in $\mathsf{sSet}$.
Buy since $\widetilde{Q}(T) \in \mathsf{sSet}^{\Delta^{op}}$ is itself Reedy fibrant, this shows that it is in fact joint Reedy fibrant, so that one has Kan equivalences 
$\widetilde{Q}(T)(0) \xrightarrow{\sim}
\delta^{\**} \widetilde{Q}(T) \xleftarrow{\sim}
\widetilde{Q}_0(T)$, showing that the second and third maps in \eqref{BIGZIG EQ} are indeed weak equivalences.

For the fourth equivalence, note that one can write
\[\widetilde{Q}_0(T) = 
\mathsf{Hom}_{\mathsf{sdSet}}(\Omega[T],\widetilde{Q})=
\mathsf{Hom}_{\mathsf{PreOp}}(\Omega[T],\gamma_{\**}\widetilde{Q})\]
\[
\left(\gamma^{\**} N \widetilde{\mathcal{O}}\right)_0(T) = 
\mathsf{Hom}_{\mathsf{sdSet}}(\Omega[T],\gamma^{\**} N \widetilde{\mathcal{O}})=
\mathsf{Hom}_{\mathsf{PreOp}}(\Omega[T], N \widetilde{\mathcal{O}})
\]
The claim now follows by noting that
$N \mathcal{O} \to \gamma_{\**} \widetilde{Q}$
is an equivalence of Reedy fibrant objects on 
$\mathsf{PreOp}^{\Delta^{op}}$ (over the tame model structure) and that $\Omega(T)$ is tame cofibrant. 

For the fifth equivalence, note that
\[
\left(\gamma^{\**} N \widetilde{\mathcal{O}}\right)_0(T) = 
\mathsf{Hom}_{\mathsf{PreOp}}(\Omega[T], N \widetilde{\mathcal{O}}) =
\mathsf{Hom}_{\mathsf{sOp}}(\Omega(T),  \widetilde{\mathcal{O}})
\]
\[
\left(hcN \widetilde{\mathcal{O}} \right)(T) = 
\mathsf{Hom}_{\mathsf{sOp}}(W_!(T),  \widetilde{\mathcal{O}})
\]
so that the required claim follows since 
$\widetilde{\mathcal{O}}$ is Reedy fibrant and
$W_!(T) \to \Omega(T)$ is a weak equivalence of cofibrant operads.

Lastly, for the last map, one needs simply note that
$c_! hcN \mathcal{O} = hcN c_! \mathcal{O}$, so that the required claim follows since 
$c_! \mathcal{O} \to \widetilde{O}$
is a levelwise equivalence of levelwise fibrant operads
and $hcN$ is right Quillen.
\end{proof}




\begin{lemma}
Let $\mathcal{O} \in \mathsf{sOp}$ and let
$g \colon x \to y$ be an equivalence in $\mathcal{O}$.

Then there exists a countable, cofibrant and contractile $H \in \mathsf{sOp}_{\{0,1\}}$ 
and a map 
$\varphi \colon H \to \mathcal{O}$
such that 
$g$ is in the image of $\varphi$. 
\end{lemma}


\begin{proof}
	We start by considering the case where $\mathsf{O}$ is locally fibrant.
	
	$g$ can be regarded as a map
	$[1] \xrightarrow{g} \mathcal{O}$,
	and one thus likewise gets a map
	$\Delta[1] \xrightarrow{g}  hcN \mathcal{O}$
	which is an equivalence in the 
	$\infty$-category $hcN \mathcal{O}$,
	so that one can find a (non-unique) factorization
	$\Delta[1] \to J \to hcN \mathcal{O}$
	which by adjunction yields a factorization
	$[1]=W_!\Delta[1] \to W_! J \to \mathcal{O}$,
	which establishes the desired claim 
	since $W_! J$ is contractible due to 
	Example \ref{WJ EX}.
	
	For a general $\mathcal{O}$, 
	consider first a local fibrant replacement
	$F \colon \mathcal{O} \to \mathcal{O}'$.
	One can hence find a map 
	$W_! J \to \mathcal{O}'$ such that
	$F(g)$ is in its image. 
	We now factor this map as
	$W_! J \xrightarrow{\sim} H \to \mathcal{O}'$
	where the second map is a local fibration.
	
	One can now form a pullback
\[
\begin{tikzcd}
	\tilde{H} \ar{r} \ar{d} & H' \ar{d}
\\
	\mathcal{O} \ar{r} & \mathcal{O}'
\end{tikzcd}
\]
where $\tilde{H}$ is seen to be contractible since
$\mathsf{sSet}$ is right proper.
	A priori, $\tilde{H}$ will need not be countable nor cofibrant, but this is easily rectified:
	indeed one can show that any countable subcomplex of $\tilde{H}$ is contained in a contractible countable subomplex, 
	yielding a countable contractible subcomplex whose image in $\mathcal{O}$ contains $g$. Lastly, performing a cofibrant replacement of that complex finishes the proof.
\end{proof}





\begin{example}\label{WJ EX}
	Let $J = N \widetilde{[1]}$ be the nerve of the contractible groupoid on two objects.
	
	Then there is an identification
\[
	W_{!} J \simeq \mathbb{F}^{\bullet} \widetilde{[1]}
\]
	where $\mathbb{F}$ denotes the (unital) free operad monad.

	To see this, we start by describing
	$\mathbb{F}^{\bullet} \widetilde{[1]}$.
	Writing $f \colon 0 \to 1$ and 
	$g \colon  1\to 0$ for the non-identity arrows in 
	$\widetilde{[1]}$ (so that $g=f^{-1}$), the $0$-simplices of $\mathbb{F}^{\bullet} \widetilde{[1]}$
	are the alternating words
	$f,g,fg,gf,fgf,gfg,fgfg,gfgf,\cdots$
	in the letters $f$, $g$.
	More generally
	$n$-simplices are given by equipping such alternating words with ``$n$ nested layers of brackets''
	(so that, for example, 
	$\left((f)(gf)\right) 
	\left( (gf) \right)$
	encodes a $2$-simplex).
	Alternatively, given an alternating word of length $l$, such bracketings are encoded by a flag of subsets
	$F_1 \subseteq F_2 \subseteq \cdots 
	\subseteq F_n \subseteq \{1,\cdots,l-1\}$.
	
	To describe $W_{!} J$, we apply the explicit description of the $W_{!} (-)$ construction given in 
	\cite{DS11}.
	Following \cite[Cor. 4.8]{DS11}, the $n$-simplices of $W_{!} J$ are uniquely encoded by a map
\begin{equation}\label{NECMAP EQ}
	N = \Delta^{k_1} \vee \Delta^{k_2} 
	\vee \cdots \vee \Delta^{k_r} 
	\to J 
\end{equation}
which is totally nondegenerate (this means that all simplices $\Delta^{k_i} \to J$ are nondegenerate)
together with a flag of subsets
	$\boldsymbol{J}(N) = 
	G_0 \subseteq 
	G_1 \subseteq \cdots \subseteq
	G_{n-1} \subseteq \boldsymbol{E}^{\mathsf{i}}(N)$.
	Noting that the nondegenerate simplices of $J$ (other than the points $0,1$)
	are themselves identified with alternating words
	$f,g,fg,gf,fgf,gfg,fgfg,gfgf,\cdots$,
	one sees that so is the map \eqref{NECMAP EQ}.
	Therefore, we see that a $n$-simplex of 
	$W_{!} J$ is uniquely, determined by some alternating word of some size $l$ together with a flag  
	$G_0 \subseteq 
	G_1 \subseteq \cdots \subseteq
	G_{n-1} \subseteq \boldsymbol{E}^{\mathsf{i}}(N)
	=\{1,\cdots,l-1\}$, since 
	$G_0$ suffices to recover the domain of 
	\eqref{NECMAP EQ}.
	
	This shows that $W_{!} J$ $\mathbb{F}^{\bullet} \widetilde{[1]}$ indeed have the same simplices.
	The fact that the simplicial operators coincide can be readily checked explicitly with the most interesting case is that of the top differential $d_n$, which in either case is induced by multiplication in 
	$\widetilde{[1]}$ 
	(that this is the case for $W_{!} J$ follows from the description of the simplicial operators in 
	\cite[Cor. 4.4]{DS11} together with the description of the ``flanking'' procedure in \cite[Lemma 4.5]{DS11}).
\end{example}




\subsection{Extra lifts for $\infty$-categories}


\begin{lemma}
The inclusion 
\[[0,1,2] \cup [0,2,3,\cdots,n] \cup 
\Lambda^0[0,1,3,\cdots,n]
 \to \Lambda^{0,2}[n]\]
is built cellularly from inclusions
$\Lambda^0[k] \to \Delta[k]$ with $k<n$.
Moreover, all such cells send $[0,1]$ to $[0,1]$.
\end{lemma}

\begin{proof}
Since $[0,2,3,\cdots,n]$ is in the domain, all missing faces must contain $1$.
Moreover, since the smallest face not containing $2$
that is not in $\Lambda^0[0,1,3,\cdots,n]$ is $[1,3,\cdots,n]$,
which is also the smallest face not in $\Lambda^{0,2}[n]$,
we see that all missing faces must contain $2$ as well.

It now suffices to check that $0$ is characteristic with respect to the missing faces, i.e.
that $12\underline{a}$ is missing iff $012\underline{a}$ is missing, and this is now obvious. 
\end{proof}

\begin{remark}
	The map $\Lambda^{0,2}[n] \to \Delta[n]$ is inner anodyne ($2$ is characteristic).
	
	This observation, together with the precious lemma, are the technical core of the observation that lifts
\[
	\begin{tikzcd}
	\Lambda^{0}[n] \ar{d}  \ar{r} & X
\\
	\Delta[n] \ar[dashed]{ru}
	\end{tikzcd}
\]
exist when $X$ is an $\infty$-category and $[0,1]$ is mapped to an equivalence in $X$.
\end{remark}




\subsection{TBD}


\begin{lemma}
Suppose that a subcategory $\Xi$ has subcategories 
$\Xi^-,\Xi^+$ which contain the isomorphisms and satisfy the unique factorization up to unique isomorphism axiom.

Write $\mathsf{Arr}(\Xi)$ for the arrow category of $\Xi$ and 
$\mathsf{Arr}^{-}(\Xi), \mathsf{Arr}^{+}(\Xi)$
for the full subcategories whose objects are arrows in $\Xi^-,\Xi^+$. 
Then $\mathsf{Arr}^{-}(\Xi)$ (resp. $\mathsf{Arr}^{+}(\Xi)$)
is initial (resp. terminal) in $\mathsf{Arr}(\Xi)$.
\end{lemma}



\begin{proof}
Given $f \in \mathsf{Arr}(\Xi)$, we need to show that there exist diagrams as on the left, and moreover that all such diagrams are connected. 
Existence is immediate from the factorization assumption . Moreover, its is straightforward from the ``uniqueness up to isomorphism'' that all connections are connected.
But by factoring the left vertical map in the diagram below, we now see that all such diagrams are connected to a factorization.
\[
\begin{tikzcd}
	\bullet \ar{r}{f} \ar{d}& 
	\bullet \ar{d}
\\
	\bullet \ar{r}[swap]{+} &
	\bullet
\end{tikzcd}
\]
%
%\[
%\begin{tikzcd}
%	\bullet \ar{r}{-} \ar{dd} \ar{rd}[swap]{-} &
%	\bullet \ar{rrr}{+} \ar{rrd}{-} &&&
%	\bullet \ar{dd}
%\\
%	&
%	\bullet \ar{rrd}[swap]{+}  && 
%	\bullet \ar{rd}{+} \ar[dashed]{ll}[swap]{\simeq}
%\\
%	\bullet 	\ar{rrr}[swap]{-} &&&
%	\bullet \ar{r}[swap]{+} & 
%	\bullet
%\end{tikzcd}
%\]
\end{proof}



\begin{lemma}\label{REDUCELAN LEM}
	Suppose that $F \colon \mathcal{C} \to \mathcal{D}$ is a functor in 
	$\mathsf{Cat}^G$.
Then the following square commutes up to natural isomorphism
\[
\begin{tikzcd}[column sep=50pt]
	\mathcal{V}^{G \ltimes \mathcal{C}} 
	\ar{r}{\mathsf{Lan}_{G \ltimes \mathcal{C} \to G \ltimes \mathcal{D}}} \ar{d}[swap]{\mathsf{fgt}}&
	\mathcal{V}^{G \ltimes \mathcal{D}} \ar{d}{\mathsf{fgt}}
\\
	\mathcal{V}^{\mathcal{C}} 
	\ar{r}[swap]{\mathsf{Lan}_{\mathcal{C} \to\mathcal{D}}} &
	\mathcal{V}^{\mathcal{D}}
\end{tikzcd}
\]
\end{lemma}


\begin{proof}
For each $d \in \mathcal{D}$ (recall that $\mathcal{D}$ and $G \ltimes \mathcal{D}$ have the same objects) one has an obvious inclusion 
$\mathcal{C} \downarrow d \to G\ltimes \mathcal{C} \downarrow d$.
Moreover, for each object of $G\ltimes \mathcal{C} \downarrow d$
there is a unique $g \in G$ such that the object is described as a composite
$F(c) \xrightarrow{g} g F(c) \to d$,
were $g F(c) \to d$ can be regarded as an object of $\mathcal{C} \downarrow d$.
One thus has a retraction 
$G \ltimes \mathcal{C} \downarrow d \to \mathcal{C} \downarrow d$
showing that $\mathcal{C} \downarrow d$ is terminal in
$G \ltimes \mathcal{C} \downarrow d$
and finishing the proof. 
\end{proof}



\subsection{Fixed color lemmas}

Given a set $\mathfrak{C}$ of colors,
write $\Sigma_{\mathfrak{C}}$ for the groupoid of corollas with edges labeled by colors in $\mathfrak{C}$.

If, in addition, $\mathfrak{C}$ is a $G$-set, 
we write $G \ltimes \Sigma_{\mathfrak{C}}^{op}$ for the larger groupoid obtained via the associated Grothendieck construction.


Writing
$\mathsf{Sym}^{G,\mathfrak{C}} = 
\mathsf{Set}^{G \ltimes \Sigma_{\mathfrak{C}}^{op}}$,
we have a natural monad $\mathbb{F}$ on
$\mathsf{Sym}^{G,\mathfrak{C}}$
whose algebra category we denote by 
$\mathsf{Op}^{G,\mathfrak{C}}$.


\begin{notation}
	Given a $G$-equivariant function 
	$f \colon \mathfrak{C} \to \mathfrak{D}$
	we write
\[
	f_{\**} \colon 
	\mathsf{Sym}^{G,\mathfrak{C}}
	\rightleftarrows
	\mathsf{Sym}^{G,\mathfrak{D}}
	\colon f^{\**}
\qquad
	\check{f}_{\**} \colon 
	\mathsf{Op}^{G,\mathfrak{C}}
	\rightleftarrows
	\mathsf{Op}^{G,\mathfrak{D}}
	\colon f^{\**}
\]
for the standard adjunctions (note the need to distinguish notations for the left adjoints).
\end{notation}


\begin{example}\label{GCORMPA EX}
Given a $G$-corolla $C \in \Sigma_G$, we write $\partial C$ for the set of edges of $C$, which is naturally identified with the set of objects of the associated $G$-operad
$\Omega(C) \in \mathsf{Op}^G$.

One can then regard $\Omega(C) \in \mathsf{Op}^{G,\partial C}$ and, moreover, $\Omega(C)$ is in fact the free operad over the symmetric sequence obtained by removing the units of $\Omega(C)$,
which we denote by
$\Omega'(C) \in \mathsf{Sym}^{G,\partial C}$.

Given the non-equivariant decomposition
$C = C_1 \amalg \cdots \amalg C_k$
with $C_i \in \Sigma$, 
one can naturally regard the $C_i$ as objects of $\Sigma_{\partial C}$.
In fact, one then has an identification
\begin{equation}\label{SOMEIDEN EQ}
	\Omega'(C) \simeq 
	\Sigma_{\partial C}[C_1] \amalg \cdots \amalg \Sigma_{\partial C}[C_k]
\end{equation}
where $\Sigma_{\partial C}[-]$ denotes the representable presheaf in 
$\mathsf{Set}^{\Sigma_{\partial C}^{op}}$.
This claim requires some justification, since a priori the right hand side of \eqref{SOMEIDEN EQ} is an object in $\mathsf{Set}^{\Sigma_{\partial C}^{op}}$,
rather than in $\mathsf{Set}^{G \ltimes \Sigma_{\partial C}^{op}}$,
i.e. we need to describe the action of the additional action arrows
$D \xrightarrow{g} gD$ on this presheaf.
This action is given by the following diagram, where the vertical $g$ arrows simply act on labels, and all horizontal arrows are shuffle arrows (i.e. arrows in $\Sigma_{\partial C}$).
The diagonal $C_g$ arrow corresponds to the structural $G$-action on $C$. It is then straightforward to check that there is a unique dashed shuffle $\tau_g$ as indicated
\[
\begin{tikzcd}
	D \ar{d}[swap]{g} \ar{r}{\sigma} & C_i \ar{rd}{C_g} \ar{d}[swap]{g}
\\
	g D \ar{r}[swap]{g \sigma} & g C_i \ar[dashed]{r}{\simeq}[swap]{\tau_g} & C_{g i}
\end{tikzcd}
\]
and one defines $g_{\**}\colon \Sigma_{\partial C}[C_i] \to \Sigma_{\partial C}[C_{gi}]$
via $\sigma \mapsto \tau_g \circ (g \sigma)$.

Moreover, letting $f \colon \partial C \to \mathfrak{C}$
be a map of colors, 
one obtains $\mathfrak{C}$-corollas $C_i^{f} \in \Sigma_{\mathfrak{C}}$
by coloring each edge $e\in \partial C$ by $f(e) \in \mathfrak{C}$, resulting in a generalized identification 
\begin{equation}\label{SOMEIDENGEN EQ}
	f_{\**} \Omega'(C) \simeq 
	\Sigma_{\mathfrak{C}}[C_1^f] \amalg \cdots \amalg \Sigma_{\mathfrak{C}}[C^f_k]
\end{equation}
Indeed, \eqref{SOMEIDENGEN EQ} follows from the observation
that the Kan extension 
$\mathsf{Lan}_{G \ltimes \Sigma_{\partial C} \to G \ltimes \Sigma_{\mathfrak{C}}}$ 
coincides, after forgetting with the $G$-action arrows,
with the Kan extension
$\mathsf{Lan}_{\Sigma_{\partial C} \to \Sigma_{\mathfrak{C}}}$
(cf. Lemma \ref{REDUCELAN LEM}).
\end{example}


\begin{definition}
	Given a $\mathfrak{C}$-corolla $C$, 
	a subgroup 
	$\Gamma \leq \mathsf{Aut}_{G \ltimes \Sigma_{\mathfrak{C}}^{op}}(C)$
	is called a \textit{$G$-graph subgroup} if
	$\Gamma \cap \mathsf{Aut}_{\Sigma_{\mathfrak{C}}^{op}}(C) = \**$.
	
	We write $\mathcal{F}^{\Gamma} = \{\mathcal{F}^{\Gamma}_C\}$
	for the collection of families of $G$-graph subgroups.
	
	A $G$-$\mathfrak{C}$-symmetric sequence
	$X \in \mathsf{Sym}^{G,\mathfrak{C}}$
	is called $\Sigma$-cofibrant if each level
	$X(C)$ is $\mathcal{F}^{\Gamma}_C$-cofibrant.
\end{definition}


\begin{remark}
	Write a $\mathfrak{C}$-corolla as $C^f \in \Sigma_{\mathfrak{C}}$,
	where $C \in \Sigma$ is the underlying corolla and
	$f\colon \partial C \to \mathfrak{C}$
	is the coloring.
	A $G$-graph subgroup 
	$\Gamma \leq \mathsf{Aut}_{G \ltimes \Sigma_{\mathfrak{C}}^{op}}(C^f)$ is, under the map 
	$G \ltimes \Sigma_{\mathfrak{C}}^{op} \to
	G \times \Sigma^{op}$,
	identified with a 
	$G$-graph subgroup of 
	$G \times \mathsf{Aut}_{\Sigma^{op}}(C)$,
	i.e., with the graph of a partial antihomomorphism
\[
	G^{op} \geq H^{op} \xrightarrow{(-)^{-1}} H 
	\xrightarrow{\tau_{(-)}} \mathsf{Aut}_{\Sigma^{op}}(C)
\]
	which is subject to the requirement
\[
	f(\tau_h(e)) = h f(e).
\]
Using the $\tau$ automorphisms one can then
\begin{inparaenum}
\item[(i)] equip $C$ with a $H$-action,
so that one can regard $C \in \Sigma^H \subseteq \Sigma_H$;
\item[(ii)] extend $\Sigma_{\mathfrak{C}}[C^f]$ to 
an object in $\mathsf{Set}^{H \ltimes \Sigma_{\mathfrak{C}}}$
by defining the action of the $H$-action arrows $h$ via 
$\sigma \mapsto \tau_{h} \circ (h \sigma)$;
\item[(iii)]
following Example \ref{GCORMPA EX}, one thus has an identification
\[
	f_{\**} \Omega'(C) \simeq \Sigma_{\mathfrak{C}}[C^f]
\]
of objects in $\mathsf{Set}^{H \ltimes \Sigma_{\mathfrak{C}}}$ and therefore an identification 
\[
	f_{\**} \left( G \cdot_H \Omega'(C) \right) \simeq 
	G\cdot_H \Sigma_{\mathfrak{C}}[C^f]
\]
of objects in $\mathsf{Set}^{G \ltimes \Sigma_{\mathfrak{C}}}$.
\end{inparaenum}
\end{remark}


\begin{example}
Let $G = \mathbb{Z}_{/2} = \{\pm 1\}$ and 
$\mathfrak{C} = \{\mathfrak{a}, -\mathfrak{a}, \mathfrak{b}\}$ where we implicitly have
$-\mathfrak{b} = \mathfrak{b}$.
Consider the two $\mathfrak{C}$-corollas 
$C,D \in \Sigma_{\mathfrak{C}}$ below.
\begin{equation}
	\begin{tikzpicture}[auto,grow=up, level distance = 2.2em,
	every node/.style={font=\scriptsize,inner sep = 2pt}]%
		\tikzstyle{level 2}=[sibling distance=3em]%
			\node at (0,0) [font = \normalsize] {$C$}%	
				child{node [dummy] {}%
					child{node {}%
					edge from parent node [swap] {$-\mathfrak{a}$}}%
					child[level distance = 2.9em]{node {}%
					edge from parent node [swap,	near end] {$\mathfrak{b}$}}%
					child[level distance = 2.9em]{node {}%
					edge from parent node [near end] {$\mathfrak{b}$}}%
					child{node {}%
					edge from parent node  {$\mathfrak{a}$}}%
				edge from parent node [swap] {$\mathfrak{b}$}};%
			\node at (7,0) [font = \normalsize] {$D$}%	
				child{node [dummy] {}%
					child{node {}%
					edge from parent node [swap] {$-\mathfrak{a}$}}%
					child[level distance = 2.9em]{node {}%
					edge from parent node [swap,	near end] {$-\mathfrak{a}$}}%
					child[level distance = 2.9em]{node {}%
					edge from parent node [near end] {$\mathfrak{a}$}}%
					child{node {}%
					edge from parent node  {$\mathfrak{a}$}}%
				edge from parent node [swap] {$\mathfrak{b}$}};%
	\end{tikzpicture}%
\end{equation}%
Each of $C,D$ admit exactly two non-trivial $G$-graph subgroups,
which are encoded by the $\mathbb{Z}_{/2}$-actions on the underlying corollas depicted below.
\begin{equation}
	\begin{tikzpicture}[auto,grow=up, level distance = 2.2em,
	every node/.style={font=\scriptsize,inner sep = 2pt}]%
		\tikzstyle{level 2}=[sibling distance=3em]%
			\node at (-1.6,0) [font = \normalsize] {$C_1$}%	
				child{node [dummy] {}%
					child{node {}%
					edge from parent node [swap] {$-a$}}%
					child[level distance = 2.9em]{node {}%
					edge from parent node [swap,	near end] {$c\phantom{b}$}}%
					child[level distance = 2.9em]{node {}%
					edge from parent node [near end] {$b$}}%
					child{node {}%
					edge from parent node  {$a$}}%
				edge from parent node [swap] {$r$}};%
			\node at (1.6,0) [font = \normalsize] {$C_2$}%	
				child{node [dummy] {}%
					child{node {}%
					edge from parent node [swap] {$-a$}}%
					child[level distance = 2.9em]{node {}%
					edge from parent node [swap,	near end] {$-b$}}%
					child[level distance = 2.9em]{node {}%
					edge from parent node [near end] {$b$}}%
					child{node {}%
					edge from parent node  {$a$}}%
				edge from parent node [swap] {$r$}};%
			\node at (5.4,0) [font = \normalsize] {$D_1$}%	
				child{node [dummy] {}%
					child{node {}%
					edge from parent node [swap] {$-a$}}%
					child[level distance = 2.9em]{node {}%
					edge from parent node [swap,	near end] {$-b$}}%
					child[level distance = 2.9em]{node {}%
					edge from parent node [near end] {$b$}}%
					child{node {}%
					edge from parent node  {$a$}}%
				edge from parent node [swap] {$r$}};%
			\node at (8.6,0) [font = \normalsize] {$D_2$}%	
				child{node [dummy] {}%
					child{node {}%
					edge from parent node [swap] {$-b$}}%
					child[level distance = 2.9em]{node {}%
					edge from parent node [swap,	near end] {$-a$}}%
					child[level distance = 2.9em]{node {}%
					edge from parent node [near end] {$b$}}%
					child{node {}%
					edge from parent node  {$a$}}%
				edge from parent node [swap] {$r$}};%
	\end{tikzpicture}%
\end{equation}%

\end{example}


The following is the analogue of \cite[Prop. 3.2]{CM13b}

\begin{proposition}
Suppose that $\mathcal{O} \in \mathsf{Op}^{G,\mathfrak{C}}$
is $\Sigma$-cofibrant.
Further, let $C \in \Sigma_G$ be any $G$-corolla and consider 
a pushout in $\mathsf{Op}^{G}$ of the form
\begin{equation}\label{PUSHOUTPROP EQ}
\begin{tikzcd}
	\partial \Omega(C) \ar{r} \ar{d} & \mathcal{O} \ar{d}
\\
	\Omega(C) \ar{r} & \mathcal{P}.
\end{tikzcd}
\end{equation}
Then the induced map
\[
	\Omega[C] \amalg_{\partial \Omega[C]} N\mathcal{O} \to N\mathcal{P}
\]
is $G$-inner anodyne.
\end{proposition}

\begin{proof}
The first step is to rewrite \eqref{PUSHOUTPROP EQ} as a pushout diagram in $\mathsf{Op}^{G,\mathfrak{C}}$.
Let us write $f \colon \partial C \to \mathfrak{C}$
for the induced map of colors.

{\color{red} HERE}
\end{proof}



\bibliography{biblio}{}



\bibliographystyle{alpha}



\end{document}