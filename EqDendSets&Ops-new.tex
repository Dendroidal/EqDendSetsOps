\documentclass[a4paper,10pt
,draft
]{article}%

\pdfcompresslevel=0
\pdfobjcompresslevel=0

\usepackage[pagebackref, colorlinks,citecolor=PineGreen,linkcolor=PineGreen]{hyperref} %hidelinks, 
\hypersetup{
  final,
  pdftitle={Equivariant simplicial operads and dendroidal sets},
  pdfauthor={Bonventre, P. and Pereira, L. A.},
  linktoc=page
}




\input{commands.tex}%


%-------- Tikz ---------------------------

\usepackage{tikz}%
\usetikzlibrary{matrix,arrows,decorations.pathmorphing,
cd,patterns,calc}
\tikzset{%
  treenode/.style = {shape=rectangle, rounded corners,%
                     draw, align=center,%
                     top color=white, bottom color=blue!20},%
  root/.style     = {treenode, font=\Large, bottom color=red!30},%
  env/.style      = {treenode, font=\ttfamily\normalsize},%
  dummy/.style    = {circle,draw,inner sep=0pt,minimum size=2mm}%
}%

\usetikzlibrary[decorations.pathreplacing]



% ---- Commands on draft --------

\usepackage{ifdraft}
\ifdraft{
  \color[RGB]{63,63,63}
  \pagecolor[RGB]{220,220,204}
  \usepackage{showkeys}
  \usepackage{geometry}
  \usepackage{todonotes}
}
{
  \usepackage[notref]{showkeys}
  \usepackage[margin=1in]{geometry}
  \usepackage[disable]{todonotes}
  \usepackage[fontsize=12pt]{scrextend}
}



% ----- Labels Changed? --------

\makeatletter

\def\@testdef #1#2#3{%
  \def\reserved@a{#3}\expandafter \ifx \csname #1@#2\endcsname
  \reserved@a  \else
  \typeout{^^Jlabel #2 changed:^^J%
    \meaning\reserved@a^^J%
    \expandafter\meaning\csname #1@#2\endcsname^^J}%
  \@tempswatrue \fi}

\makeatother


% ---- Commands --------

% new symbols

\newcommand{\mycircled}[2][none]{%
  \mathbin{
    \tikz[baseline=(a.base)]\node[draw,circle,inner sep=-1.5pt, outer sep=0pt,fill=#1](a){\ensuremath #2\strut};
cf.  }
}
\newcommand{\owr}{\mycircled{\wr}}

% replace symbols

% \renewcommand{\hat}{\widehat}

% random

\renewcommand{\F}{\mathcal F}
\newcommand{\Q}{\mathcal Q}

\newcommand{\lltimes}{\underline{\ltimes}}


% detecting $\V$-categories:

\newcommand{\I}{\mathbb I}
\newcommand{\J}{\mathbb J}
\renewcommand{\1}{\eta}%{\ensuremath{\mathbb{id}}}

\newcommand{\Alg}{\mathsf{Alg}}
\newcommand{\Kl}{\mathsf{Kl}}


% lazy shortcuts

\newcommand{\SC}{\Sigma_{\mathfrak C}}
\newcommand{\OC}{\Omega_{\mathfrak C}}

\newcommand{\UV}{\underline{\mathcal V}}
\newcommand{\UC}{\underline{\mathfrak C}}



\usepackage{harpoon}
\newcommand{\vect}[1]{\text{\overrightharp{\ensuremath{#1}}}}


% ---- Title --------

\title{Equivariant Segal operads, simplicial operads, and dendroidal sets}

\author{Peter Bonventre, Lu\'is A. Pereira}%

\date{\today}


% ---- Document body --------

\begin{document}

\maketitle

\begin{abstract}
      Things and stuff
\end{abstract}

\tableofcontents

\vskip 10pt

All functors below are right adjoints.
\[
	\begin{tikzcd}
		\mathsf{PreOp}^G & 
		\mathsf{sOper}^G \ar[dashed]{l}[swap]{N_d}
		\ar[dashed]{d}{hcN_d}
\\
		\mathsf{sdSet}^G \ar{r}[swap]{(-)_0} \ar{u}{\gamma_{\**}} &
		\mathsf{dSet}^G
	\end{tikzcd}
\]

What we need:

\begin{itemize}
\item set up Grothendieck description of $\mathsf{sOper}^G$, and build fiberwise model structures
\item combine into overall model structure on $\mathsf{sOper}^G$
\item prove that $(W_!,hcN_d)$ is a Quillen adjunction (Proposition \ref{W!_COF_PROP})
\item establish tame model structure and prove that $(W,N_d)$ is a Quillen equivalence (Proposition \ref{PREQUIEQUIV PROP})
\item combine everything by showing that the square commutes up to homotopy (Proposition \ref{COMUOTOHOM PROP})
\end{itemize}






\section{Introduction}

\begin{itemize}
\item describe $G$-objects in colored operads
\end{itemize}


\subsection{Main Results}

\begin{theorem}
      \label{THM1_C}
      Let $(\V,\otimes)$ denote either $(\sSet, \times)$ or $(\sSet_{\**} \wedge)$,
      and fix a $G$-set $\mathfrak C$.
      Then there exist model structures on $\Op^{G, \mathfrak C}(\V)$ such that
      $\O \to \O'$ is a weak equivalence (resp. fibration) if the maps
      \begin{equation}
            \label{THM1_C_EQ}
            \O(\xi)^\Gamma \to \O'(\xi)^\Gamma
      \end{equation}
      are weak equivalences (resp. fibrations) in $\V$ for all
      $\mathfrak C$-signatures $\xi$ and
      graph subgroups $\Gamma \leq \Stab(\xi)$.

      More generally, for $\F = \set{\F_n}_{n \geq 0}$ an \textit{arbitrary} $(G, \Sigma)$-family,
      there exists a model category structure on $\Op^{G, \mathfrak C}(\V)$, which we denote $\Op^{G, \mathfrak C}_\F(\V)$,
      with weak equivalences (resp. fibrations) determined by \eqref{THM1_C_EQ} for $\Gamma \leq (\F_n)_\ksi = \F_n \cap \Aut(\ksi)$.

      Lastly, analogous semi-model category structures $\Op^{G, \mathfrak C}(\V)$, $\Op^{G, \mathfrak C}_\F(\V)$ exist provided that
      $(\V, \otimes)$:
      \begin{enumerate*}[label = (\roman*)]
      \item is a cofibrantly generated model category;
      \item is a closed monoidal model category with cofibrant unit;
      \item has cellular fixed points;
      \item has cofibranty symmetric pushout powers.
      \end{enumerate*}
\end{theorem}

\begin{remark}
      As $\Cat^{G, \mathfrak C}(\V) = \Op^{G, \mathfrak C}(\V) \downarrow \**$,
      this produces $G$-model structures on $\V$-enriched categories with a single set of objects.
\end{remark}



\todo[inline]{come back}

Dwyer-Kan weak equiavlences were introduced in the context of simplicial categories by Dwyer-Kan and Bergner \cite{DK80, Ber07b}.
A simplicial functor $F: \mathcal C \to \mathcal D$ between simplicial categories is called a Dwyer-Kan equivalence if
it is ``homotopically'' fully-faithful and essentially surjective:
each morphism of mapping spaces $\mathcal C(x, y) \to \mathcal D(F(x), F(y))$ is a Kan-equivalence of simplicial sets, and
the functor of 1-categories $\pi_0F: \pi_0\mathcal C \to \pi_0 \mathcal D$ is essentially surjective.

Equivariantly, we require that the morphism of mapping $G$-spaces to be a genuine $G$-Kan equivalence, and
each $\pi_0F^H$ essentially surjective.

Generalizing to (single-colored) operads,
each $\O(n)$ is a $G \times \Sigma_n$-object, and we can ask that the maps $\O(n) \to \P(n)$ are some sort of $G \times \Sigma_n$-equivalence,
with perhaps the most useful being the ``graph subgroup equivalences'' (see, e.g. \cite{BP_geo}, \cite{BH15}),
while the notion of essential surjectivity is vacuous.
Further generalizing to multi-colored operads (or multicategories),
we have a set $\mathfrak C$ of ``colors'' (or objects),
and a mapping space $\O(C)$ for each ``signature'' $C = (c_1, \dots, c_n; c_0) \in \mathfrak C^{\times n} \times \mathfrak C$,
such that each $\O(C)$ has an action by the subgroup
\[
      \Aut(C) = \Aut_{G \ltimes \Sigma_{\mathfrak C}}(C) = \Stab_{G \times \Sigma_n}(C) \leq G \times \Sigma_n.
\]
%\Aut(C) \leq G \times \Sigma_n$.
We can ask that the maps $\O(C) \to \P(F(C))$ be some sort of $\Aut(C)$-equivalence,
again with perhaps the most useful being the graph equivalences.
Here, homotopical essential surjectivity is defined on the underlying categories of path components.

\begin{theorem}
      \label{INTRO_MODEL_THM}
      Let $(\V, \otimes)$, denote either $(\sSet, \times)$ or $(\sSet_{\**}, \wedge)$.
      Then there exists a cofibrantly generated model structure on the category $\Op^G(\V)$,
      where weak equivalences are ``graph Dwyer-Kan'' equivalences (see Definition \ref{DK_MODEL_DEF}). 

      More generally, for any $(G, \Sigma)$-family $\F = \set{\F_n}$ with units,
      there exists an ``$\F$-Dwyer-Kan'' model structure on $\Op^G(\V)$.
      weak $\F$-equivalences, $\F$-fibrations, and $\F$-cofibrations defines as in Definition \ref{MODEL_DEFN}.
           
      Moreover, analogous semi-model category structures $\Op^G_\F(\V)$ exist
      provided that $(\V, \otimes)$:
      \begin{enumerate}[label = (\roman*)]\itemsep-4pt
      \item is a cofibrantly generated model category,
      \item is a closed monoidal model category with cofibrant unit
            \footnote{Cofibrant unit also needed for \ref{J-CELL_PROP}.},
      \item has cellular fixed-point functors,
      \item \label{I_CSPP_LBL} has cofibrant symmetric pushout powers  (Defn. \ref{CSPP_DEF}),
            \footnote{Also needed for Props \ref{CAV_4.14_PROP2}, \ref{J-CELL_PROP}}, % \ref{LOCAL_COF_LEM}            
            % --------------------
      \item \label{I_RP_LBL} is right proper
            \footnote{Needed for Lemma \ref{RIGHTPROPER_LEM} and Lemma \ref{2OUTOF3_PROP}.},
      \item \label{I_GENSET_LBL} has a set $\mathscr{G}$ of generating $\V$-intervals
            \footnote{Needed so we have a \textit{set} of generating trivial cofibrations},
      \item satisfies the coherence axiom.
      \end{enumerate}
\end{theorem}
\begin{proof}
      [Proof Outline]
      This result is proved in two steps \S \ref{MS_SEC}:
      First, Theorem \ref{MODEL_THM} establishes the existance of a certain $\F$-(semi)-model structure on $\Op^G(\V)$,
      where essential surjectivity is instead defined using a cofibrant replacement of the $\V$-category detecting isomorphisms
      (see Definition \ref{PL_ES_DEFN} and Remark \ref{ESS_SUR_REM}).
      Second,
      Propositions \ref{WE_ARE_DK_PROP} and \ref{COH_DK_ARE_WE_PROP},
      combined with Proposition \ref{SSET_COH_PROP} and Corollary \ref{PTSSETCOH_COR},
      imply that when $\V$ satisfies the coherence axiom,
      to two notions of essential surjectivity agree.
\end{proof}


% There is a second possible notion of ``homotopically essentially surjective'':
% we say that a map $F: \O \to \P$ is essentially surjective if any object in $\P$ is ``equivalent'' to an object in the image of $F$,
% where equivalence is defined used a cofibrant replacement of the $\V$-category detecting isomorphisms (see Definition \ref{PL_ES_DEFN} and Remark \ref{ESS_SUR_REM}).
% In the case where $\V$ does not satisfy the coherence axiom, then this notion is potentially \textit{stronger} than the notion used above.
% This case is covered in Theorem \ref{MODEL_THM}.


\todo[inline]{come back}

\begin{theorem}
      Tame model structure exists, and we have Quillen equivalences
      $\mathsf{PreOp}^G \leftrightarrows \mathsf{PreOp}^G_{tame} \rightleftarrows \sOp^G$.
\end{theorem}

\begin{theorem}
      \label{QE_THM}
      The adjunction $\dSet^G \rightleftarrows \sOp^G$ is a Quillen equivalence.
\end{theorem}

\section{Preliminaries}

\subsection{Wreath products and Grothendieck fibrations}

% -------------------- Types of Grothendieck fibrations --------------------

% \begin{remark}
%       \label{GRTH_RMK}
%       Given a functor
%       \begin{equation}
%             \mathcal B^{op} \longrightarrow \Cat,
%             \qquad
%             b \mapsto \mathcal C_b
%       \end{equation}
%       there are four possible ``Grothendieck constructions''.
%       Two are fibrations over $\mathcal B$, two over $\mathcal B^{op}$.
%       They all have as objects pairs $(b \in \mathcal B, X \in \mathcal C_b)$ and arrows pairs of maps,
%       but the directionality of these maps is different:
%       \begin{enumerate}[label = (\roman*)]
%       \item The \textit{standard} Grothendieck construction $\mathcal B \ltimes \mathcal C$
%             is the fibration over $\mathcal B$ with as arrows pairs of maps $(f, \phi)$ with
%             $f: b \to \bar b$ and
%             $\phi: X \to f^{\**} \bar X$.
%       \item $(\mathcal B \ltimes \mathcal C)^{\underline{op}}$
%             is the fibration over $\mathcal B$ with as arrows pairs of maps $(f, \phi)$ with
%             $f: b \to \bar b$ and
%             $\phi: f^{\**} \bar X \to X$.
%       \item $(\mathcal B \ltimes \mathcal C)^{op}$
%             is the fibration over $\mathcal B^{op}$ with as arrows pairs of maps $(f, \phi)$ with
%             $f: \bar b \to b$ and
%             $\phi: \bar X \to f^{\**} X$.
%       \item $\mathcal B \underline{\ltimes} \mathcal C = \left((\mathcal B \ltimes \mathcal C)^{\underline{op}}\right)^{op}$
%             is the fibration over $\mathcal B^{op}$ with as arrows pairs of maps $(f, \phi)$ with
%             $f: \bar b \to b$ and 
%             $\phi: f^{\**}X \to \bar X$.
%       \end{enumerate}
      
%       Unless otherwise specified, \textit{the} Grothendieck construction will refer to the first one.
% \end{remark}
\begin{notation}
      Given a functor $\mathcal C_{(-)}: \mathcal B \to \Cat$, $b \mapsto \mathcal C_b$, we let
      $\mathcal B \ltimes \mathcal C_{(-)}$ denote the (covariant) \textit{Grothendieck construction},
      with objects pairs $(b,X)$ with $b \in \mathcal B$ and $X \in \mathcal C_b$, and
      maps pairs $(f,g)$ with $f: b \to \bar b$ and $g: f_{\**}X \to \bar X$.
\end{notation}



% -------------------- Wreath Products --------------------

Now, recall the notation $\mathsf F \wr \mathcal C$ for a category $\mathcal C$ from \cite{BP_geo}.

\begin{notation}
      \label{F_WR_NOT}
      We let $\mathsf F$ denote a (fixed) full subcategory of \textit{ordered finite sets} and set maps,
      such that the only ordered isomorphisms are the identity.
      
      Given a category $\mathcal C$, let $\mathsf F \wr \mathcal C$ denote the contravariant Grothendieck construction
      $(\mathsf F^{op} \ltimes \mathcal C^{\times (-)})^{op}$ on the functor
      \begin{equation}
            \mathsf F^{op} \longto \Cat,
            \qquad \qquad
            A \mapsto \mathcal C^{\times A}.
      \end{equation}
      Explicitly, objects are tuples of elements of $\mathcal C$, and maps are composites of ``shuffles'' and tuples of maps in $\mathcal C$.
\end{notation}

\begin{remark}
      \label{WR_DIAG_REM}
      We observe that we have a natural diagonal map
      % \begin{equation}
      $
      \mathsf F \times \mathcal C \into \mathsf F \wr \mathcal C,
      $
      % \end{equation}
      for any category $\mathcal C$,
      and thus for any functor $F: \mathcal D \to \mathsf F$, we have an induced functor
      $F: \mathcal D \times \mathcal C \to \mathsf F \wr \mathcal C$.

      More generally, for any $G$-category $\mathcal C$ we have a natural ``diagonal'' map
      \begin{equation} 
            G \ltimes (\mathsf F \wr \mathcal C) \to \mathsf F \wr (G \ltimes \mathcal C).
      \end{equation}
\end{remark}

\begin{definition}[{cf. \cite[Defn 4.3]{BP_geo}}]
      Let $\mathsf{WSpan}^l(\mathcal C, \mathcal D)$ (resp. $\mathsf{WSpan}^r(\mathcal C, \mathcal D)$)
      denote the category of \textit{left (resp. right) weak spans}, with objects
      \begin{equation}
            \mathcal C \xleftarrow{k} \mathcal A \xrightarrow{X} \mathcal D
      \end{equation}
      and arrows those diagrams as on the left (resp. right) below
      \begin{equation}
            \begin{tikzcd}[row sep = tiny]
                  & \mathcal A_1 \arrow[dr, "X_1", ""'{name=U}] \arrow[dl, "k_1"'] \arrow[dd, "i"']
                  &
                  &&
                  &
                  \mathcal A_1 \arrow[dr, "X_1", ""'{name=A}] \arrow[dl, "k_1"'] \arrow[dd, "i"']
                  \\
                  \mathcal C
                  &&
                  \mathcal D
                  &&
                  \mathcal C
                  &&
                  \mathcal D
                  \\
                  & |[alias=V]| \mathcal A_2 \arrow[ur, "X_2"'] \arrow[ul, "k_2"]
                  &
                  &&
                  &
                  |[alias=B]| \mathcal A_2 \arrow[ur, "X_2"'] \arrow[ul, "k_2"]
                  \arrow[Rightarrow, from = U, to = V]
                  \arrow[Rightarrow, from = B, to = A]
            \end{tikzcd}
      \end{equation}
      denoted by $(i,\phi): (k_1,X_1) \to (k_2,X_2)$, with composition defined in the natural way.      
\end{definition}




\subsection{Fibered category theory}
\label{FIBCAT_SEC}

Motivations for discussing fibered category theory:
\begin{itemize}
\item the model structure on operads is built by first building fiberwise model structures
\item however, describing the fibers is awkward
\item the generating cofibrations of both $\mathsf{sOp}$
and (the tame model structure on) $\mathsf{PreOp}$
can be described using a fiberwise simplicial tensor,
so that the fact that the adjunction
$\tau \colon \mathsf{PreOp} \rightleftarrows \mathsf{sOp}\colon N$ is Quillen is readily understood by noting that this is a fibered (simplicial) adjunction.
\end{itemize}




\subsubsection{Fibered adjunctions}

\begin{notation}
Given a Grothendieck fibration $p\colon \mathcal{C} \to \mathcal{E}$, 
two objects $\bar{c},c \in \mathcal{C}$
and an arrow $f \colon p(\bar{c}) \to p(c)$, 
we write $\mathcal{C}_f(\bar{c},c)$ for the subset of maps over $f$, i.e. to the pullback in the following diagram.
\begin{equation}
\begin{tikzcd}
		\mathcal{C}_f\left(\bar{c},c \right) \arrow[d] \arrow[r]
	&
		\mathcal{C}\left(\bar{c},c \right) \arrow[d]
\\
		\{f\} \arrow[r]
	&
		\mathcal{E}\left(p(\bar{c}),p(c)\right)
\end{tikzcd}
\end{equation}
\end{notation}


\begin{remark}
One has a natural decomposition
\begin{equation}
	\mathcal{C}(\bar{c},c) \simeq \coprod_{f \in \mathcal{E}(p(\bar{c}),p(c))} \mathcal{C}_f\left(\bar{c},c \right)
\end{equation}
\end{remark}


\begin{remark}\label{CARTCHAR REM}
An arrow $F\colon \bar{c} \to c$ in a Grothendieck fibration
$p \colon \mathcal{C} \to \mathcal{E}$ is cartesian
precisely if it induces natural isomorphisms
\[
\mathcal{C}_{id_{p(\bar{c})}}
\left(- ,\bar{c} \right)
\xrightarrow{\simeq}
\mathcal{C}_{p(F)}\left(- ,c \right)
\]
(note that this condition is weaker than the condition defining cartesian arrows; the claim being made here is that if cartesian arrows are already known to exist, then they are detected by this condition).
\end{remark}



\begin{definition}
Let 
$p\colon \mathcal{C} \to \mathcal{E}$,
$p\colon \mathcal{D} \to \mathcal{E}$,
be Grothendieck fibrations.
A \emph{fibered adjunction} is an adjunction
\[
L \colon \mathcal{C} \rightleftarrows \mathcal{D} \colon R
\]
where the functors, unit and counit are all fibered, i.e.
$pL=p$, $pR=p$, $p \eta = id_{p}$, $p\eta = id_p$.
\end{definition}


\begin{remark}
A fibered adjunction induces natural isomorphisms
\[
\mathcal{D}_f\left(Lc,d\right)
\simeq
\mathcal{C}_f\left(c,Rd\right)
\]
for each $c\in \mathcal{C}$, $d \in \mathcal{D}$, $f\colon p(c)\to p(d)$. 
\end{remark}



\begin{proposition}\label{FIBADJCAR PROP}
Let $L \colon \mathcal{C} \rightleftarrows \mathcal{D} \colon R$
be an adjunction between Grothendieck fibrations.

If the adjunction is a fibered then 
%the right adjoint
$R$ is 
a fibered functor which preserves cartesian arrows.

Conversely, if the right adjoint $R$ is 
a fibered functor which preserves cartesian arrows, then it is possible to modify the adjunction so that it becomes a fibered adjunction.
\end{proposition}


\begin{proof}
For the first claim, 
letting $F \colon \bar{d} \to d$ be a cartesian arrow, 
the fact that $R(F)$ is again cartesian follows from
Remark \ref{CARTCHAR REM} applied to the composite
\[
\mathcal{C}_{id_{p(\bar{d})}}
	\left(-,R\bar{d}\right)
	\simeq 
\mathcal{D}_{id_{p(\bar{d})}}
	\left(L(-),\bar{d}\right)
	\xrightarrow{\simeq}
\mathcal{D}_{p(F)}\left(L(-),d\right)
	\simeq
\mathcal{C}_{p(F)}\left(-,Rd\right)
\]
For the ``conversely'' claim,
noting that by assumption $pRL = pL$,
one can choose a cartesian natural transformation $\bar{L} \to L$
(i.e. a cartesian arrow in $\mathcal{D}^{\mathcal{C}}$)
over the projection of the adjunction unit
$ p \xrightarrow{p \eta} pRL$
(which is an arrow in $\mathcal{E}^{\mathcal{C}}$).
Moreover, noting that by assumption
$R\bar{L} \to RL$ is again cartesian, we write
$id_{\mathcal{C}} \xrightarrow{\bar{\eta}} R \bar{L} \to RL$
for the natural factorization
as well as $\bar{\epsilon}$
for the composite
$\bar{L}R \to LR \xrightarrow{\epsilon} id_{\mathcal{D}}$.
We claim that $\bar{L},R,\bar{\eta},\bar{\epsilon}$
now form provide fibered adjunction, with the non obvious claim being that this is in fact still an adjunction.
That the composite
$R\xrightarrow{\bar{\eta}R} R\bar{L}R \xrightarrow{R\bar{\epsilon}} R$
is the identity follows since this is 
$R \xrightarrow{\bar{\eta} R} R\bar{L}R \to RLR \xrightarrow{R \epsilon} R$ and thus
$R \xrightarrow{\eta R} RLR \xrightarrow{R \epsilon} R$.
The remaining claim is that the top horizontal composite in the diagram below is the identity, 
\begin{equation}
\begin{tikzcd}
		\bar{L} \arrow[d] \arrow{r}{\bar{L}\bar{\eta}}
	&
		\bar{L} R \bar{L} \arrow[d] \ar{r} \ar{d}
		\arrow[bend left]{rr}{\bar{\epsilon}\bar{L}}
	&
		L R \bar{L} \ar{r}{\epsilon \bar{L}} \ar{d}
	&
		\bar{L} \ar {d}
\\
		L \arrow{r}{L\bar{\eta}}
		\arrow[bend right]{rr}[swap]{L \eta}
	&
		LR\bar{L} \ar{r}
	&
		LRL \ar{r}{\epsilon L}
	&
		L
\end{tikzcd}
\end{equation}
and since 
$\bar{L} \to L$ is cartesian, 
is in fact enough to show that the overall composite $\bar{L} \to L$
is the standard map, which is clear. 
\end{proof}


\begin{example}
      \label{COTENS_EX}
      Letting $\mathcal{O} \in \mathsf{sOp}$ be a simplicial colored operad
      and $K \in \mathsf{sSet}$,
      one has a pointwise simplicial cotensoring
      $\{K,\O\}_{\mathsf{F}} \in \mathsf{sOp}$
      given by
      $\{K,\O\}_{\mathsf{F}}(\vect C) = 
      \left(\O(\vect C)\right)^K$
      for each signature $\vect C$.
      %
      $\{K,-\}_{\mathsf{F}}\colon \mathsf{sOp} \to \mathsf{sOp}$
      is then a fibered right adjoint which preserves pullback arrows,
      and thus the corresponding adjunction is also fibered.
\end{example}

\begin{example}
      \label{TENS_EX}
      We warn that the tensoring left adjoint, denoted $(-) \otimes_{\Fin} K: \sOp \to \sOp$, is in general \textit{not} levelwise.
      However, for $C \in \Sigma$, we have
      \begin{enumerate}
      \item $\Omega(C) \otimes_\F K(\vect D) = \Omega(C)(\vect D) \times K$,
      \item $\Omega(C) \otimes_\F K = (\mathbb F_{\Set} \Sigma_\bullet[C]) \otimes_F K = \mathbb F_{\sSet} (\Sigma_\bullet[C] \cdot K)$.
      \item $\Hom_{\Op}(\Omega(C) \otimes_\Fin K, \O) = \coprod_{\mbox{$C$-profiles $\vect C$}} \Hom_{\sSet}(K, \O(\vect C)),$
      \item The set of squares on the left below are in bijection with the set of squares are the right, where $\vect C$ runs over all $C$-profiles.
            \begin{equation}
                  \begin{tikzcd}
                        \Omega(C) \otimes_\F K \arrow[r] \arrow[d, "u"']
                        &
                        \O \arrow[d, "F"]
                        & %----------
                        K \arrow[r] \arrow[d, "u"']
                        &
                        \O(\vect C) \arrow[d, "{F(\vect C)}"]
                        \\
                        \Omega(C) \otimes_\F L \arrow[r]
                        &
                        \P
                        & % ----------
                        L \arrow[r]
                        &
                        \P(F(\vect C))
                  \end{tikzcd}
            \end{equation}
      \end{enumerate}
\end{example}





\subsubsection{Fibered monads}


\begin{definition}\label{FIBMON DEF}
Given a Grothendieck fibration $p\colon \mathcal{C} \to \mathcal{D}$,
a \textit{fibered monad} is a monad $T\colon \mathcal{C} \to \mathcal{C}$ such that the diagram below commutes
\[
\begin{tikzcd}
\mathcal{C} \ar{rr}{T} \ar{rd}[swap]{p} && \mathcal{C} \ar{dl}{p}
\\
& \mathcal{D}
\end{tikzcd}
\]
and the multiplication 
$\mu \colon TT \Rightarrow T$
and unit $\eta \colon I \Rightarrow T$
satisfy
$p\mu=p\eta=id_{p}$.

Moreover, a \textit{fiber algebra} is a $T$-algebra $c \in \mathcal{C}$
such that the multiplication map
$Tc \xrightarrow{m} c$ satisfies 
$p(m)=id_{\pi(c)}$.

Lastly, we write $\mathsf{Alg}^{p}_T(\mathcal{C}) \subseteq \mathsf{Alg}_T(\mathcal{C})$ for the full subcategory of fiber algebras.
\end{definition}

\begin{remark}
For each $d\in \mathcal{D}$, a fibered monad $T$ restricts to a monad on each fiber $\mathcal{C}_d$, and we write $T_d$ to denote that restricted monad.
\end{remark}


\begin{remark}
If $T$ is a fibered monad then any free algebra $Tc$ is automatically a fiber algebra, so that the free $T$-algebra functor factors
as 
$\mathcal{C} \to \mathsf{Alg}^{p}_T(\mathcal{C}) \subseteq \mathsf{Alg}_T(\mathcal{C})$.
\end{remark}



\begin{proposition}\label{FIBALGGR PROP}
Given a fibered monad on $p\colon \mathcal{C} \to \mathcal{D}$ the projection $\mathsf{Alg}^{p}_T(\mathcal{C}) \to \mathcal{D}$
is again a Grothendieck fibration.

Moreover, the free-algebra and forgetful functors then form a fibered adjunction
$\mathcal{C} \rightleftarrows \mathsf{Alg}^{p}_T(\mathcal{C})$.
\end{proposition}

The key to this proof is that fiber algebra structures can be ``pulled back'' along cartesian arrows
(which, by Proposition \ref{FIBADJCAR PROP}, must be the case if $\mathcal{C} \rightleftarrows \mathsf{Alg}^{p}_T(\mathcal{C})$ is to be a fibered adjunction). 

\begin{proof}
Given a cartesian arrow $f\colon \bar{c} \to c$ on $\mathcal{C}$ and a fiber algebra structure on $c$, we claim there is a unique fiber algebra structure on $\bar{c}$ making $f$ into an algebra map. Indeed, the properties of cartesian arrows imply that the is a unique way to choose a dashed fiber arrow in the diagram
\[
\begin{tikzcd}
	T \bar{c} \ar{r}{Tf} \ar[dashed]{d} & T c \ar{d}
\\
	\bar{c} \ar{r}[swap]{f} & c.
\end{tikzcd}
\]
The claims that $T\bar{c} \to \bar{c}$ is then an algebra map and that 
$f$ is also cartesian as an algebra map again follow from 
$f$ being cartesian in $\mathcal{C}$.

For the ``moreover'' claim concerning the 
$\mathcal{C} \rightleftarrows \mathsf{Alg}^{p}_T(\mathcal{C})$
adjunction,
this follows by noting that the adjunction unit is the unit
$I \Rightarrow T$ of the monad $T$, 
which is fibered by assumption,
while the counit, evaluated on a fiber algebra $c$, is the multiplication $Tc \to c$, and hence fibered by definition of fiber algebra.
\end{proof}



\begin{remark}
Suppose a cleavage of $p\colon \mathcal{C} \to \mathcal{D}$ 
has been chosen, so that for each arrow $f \colon d \to d'$ in $\mathcal{D}$ there is a chosen functor 
$f^{\**} \colon \mathcal{C}_{d'} \to \mathcal{C}_{d}$
together with a natural transformation
$f^{\**} \Rightarrow id_{\mathcal{C}_{d'}}$
consisting of cartesian arrows.

A fibered monad $T$ is then equivalent to the data of the fiber monads 
$T_d$ on the fibers $\mathcal{C}_d$ together with,
for each arrow $f \colon d \to d'$ in $\mathcal{D}$,
natural transformations
$\varphi_f \colon T_{d} f^{\**} \Rightarrow f^{\**} T_{d'}$
such that
\begin{itemize}
\item[(a)]
for composites $d \xrightarrow{f} d' \xrightarrow{g} d''$
and identities $d \xrightarrow{id_d} d$
the induced diagrams below commute
\begin{equation}\label{GROTHASS EQ}
\begin{tikzcd}
	T_d f^{\**} g^{\**} \ar[Rightarrow]{r} \ar[Leftrightarrow]{d}[swap]{\simeq} &
	f^{\**} T_{d'} g^{\**} \ar[Rightarrow]{r} &
	f^{\**}g^{\**}  T_{d''}  \ar[Leftrightarrow]{d}{\simeq} &
	T_d \ar[equal]{r} \ar[Leftrightarrow]{d}[swap]{\simeq} &
	T_d \ar[Leftrightarrow]{d}{\simeq}
\\
	T_d (gf)^{\**} \ar[Rightarrow]{rr}{} &&
	(gf)^{\**} T_{d''} &
	T_d id_d^{\**} \ar[Rightarrow]{r}{} &
	id_d^{\**} T_{d}
\end{tikzcd}
\end{equation}
\item[(b)] the natural squares below commute 
\begin{equation}\label{GROTHCART EQ}
\begin{tikzcd}
	T_d T_d f^{\**} \ar[Rightarrow]{r} \ar[Rightarrow]{d}[swap]{} &
	T_d f^{\**} T_{d'} \ar[Rightarrow]{r} &
	f^{\**} T_{d'} T_{d'} \ar[Rightarrow]{d}{} &
	f^{\**} \ar[equal]{r} \ar[Rightarrow]{d}&
	f^{\**} \ar[Rightarrow]{d}
\\
	T_d f^{\**} \ar[Rightarrow]{rr}{} &&
	f^{\**} T_{d'} &
	T_d f^{\**} \ar[Rightarrow]{r}{} &
	f^{\**} T_{d'}
\end{tikzcd}
\end{equation}
\end{itemize}
% $\varphi_f$ is induced by applying $T$ to the chosen pullback arrows, 
% (a) is then functoriality of $T$ with respect to pullback arrows
% \eqref{GROTHCART EQ} is the naturality of $\mu \colon TT \Rightarrow T$ and $\eta I \Rightarrow T$ with regard to pullback arrows.
\end{remark}


\begin{remark}\label{ABSPUSH REM}
      If $p\colon \mathcal{C} \to \mathcal{D}$
      is a Grothendieck fibration then the dual map
      $p^{op}\colon \mathcal{C}^{op} \to \mathcal{D}^{op}$
      will also be a Grothendieck fibration iff
      all the cleavages $f^{\**}$ fit into adjunctions
      $f_! \colon \mathcal{C}_d \rightleftarrows \mathcal{C}_{d'}\colon f^{\**}$,
      in which case we a natural transformation $id_{\mathcal C_d} \Rightarrow f_!$ consisting of cocartesian arrows.
      \todo[inline]{reference?}
      % \end{remark}
      
      % \begin{remark}
      % Should the pullback functors $f^{\**}$ in the previous remark admit left adjoints $f_{!}$,
      In this case,
      the commutativity of the diagrams in $\eqref{GROTHCART EQ}$
      is equivalent to the claim that the induced natural transformations
      $T_{d} \Rightarrow f^{\**}T_{d'}f_{!}$
      are maps of monads.
\end{remark}


\begin{remark}\label{ALGPUSHLL REM}
Suppose that both $\mathcal{C}$ and $\mathsf{Alg}_T^p(\mathcal{C})$
admit adjunctions as in 
Remark \ref{ABSPUSH REM} for each map $f\colon d \to d'$.
We denote these two adjunctions by
\[
f_! \colon \mathcal{C}_d \rightleftarrows \mathcal{C}_{d'}\colon f^{\**}
\qquad
\check{f}_! \colon \mathsf{Alg}_{T_d}(\mathcal{C}_d) 
\rightleftarrows 
\mathsf{Alg}_{T_{d'}}(\mathcal{C}_{d'})\colon f^{\**}
\]
to emphasize the fact that the algebraic $f^{\**}$ functor lifts the underlying $f^{\**}$ (cf. the proof of Proposition \ref{FIBALGGR PROP}).

On the other hand, the algebraic $\check{f}_!$ functor is not a lift of
the underlying $f_!$.
Rather, by the dual of Proposition \ref{FIBADJCAR PROP}
one has that $T \colon \mathcal{C} \to \mathsf{Alg}_F^p(\mathcal{C})$ preserves cocartesian arrows,
so that the canonical map of cocartesian arrows
$T_d \Rightarrow \check{f}_! T_d$
is identified with the canonical map of the image under $T$ of cocartesian arrows
$T_d \Rightarrow T_{d'} f_!$.
Thus since any 
$c \in \mathsf{Alg}_{T_d}(\mathcal{C}_d)$
is given by a coequalizer of free algebras
$c \simeq coeq(T_dT_d c \rightrightarrows T_d c)$,
so that one has the formula
$\check{f}_! c \simeq 
coeq(T_{d'}f_!T_d c \rightrightarrows T_{d'}f_! c)$,
{\color{OliveGreen} with the interesting arrow given intrinsically by the composite
  \[
        T_{d'} f_! T_d \Rightarrow T_{d'} f_! f^{\**} T_{d'} f_! \Rightarrow T_{d'} T_{d'} f_! \Rightarrow T_{d'} f_!.
  \]
}
\end{remark}







\begin{proposition}
      \label{DIAGRAMFM_PROP}
Let $I$ be a fixed diagram category, and $T$ a fibered monad with respect to a Grothendieck fibration with respect to
$p\colon \mathcal{C} \to \mathcal{D}$. Then:
\begin{itemize}
\item[(i)] $p^I\colon \mathcal{C}^I \to \mathcal{D}^I$ is again a Grothendieck fibration;
\item[(ii)] $T^I$ is a fibered monad with respect to $p^I\colon \mathcal{C}^I \to \mathcal{D}^I$;
\item[(iii)] there is a natural identification 
$\mathsf{Alg}_{T^I}^{p^I}(\mathcal{C}^I)\simeq
\left(\mathsf{Alg}_T^p(\mathcal{C})\right)^I$.
\end{itemize}
\end{proposition}

\begin{proof}
(i) is well known (one can simply create cartesian arrows pointwise), and both (ii) and (iii) follow readily from the definitions.
\end{proof}








\section{Colored operads}


When working with the category $\mathsf{Op}(\mathcal{V})$ of colored operads
it is often useful to consider the subcategories $\mathsf{Op}^{\mathfrak{C}}(\mathcal{V})$
of those operads with a chosen fixed set of objects
$\mathfrak{C} \in \mathsf{F}$ and those maps which are the identity on the set $\mathfrak{C}$ of objects.
In particular, these $\mathsf{Op}^{\mathfrak{C}}(\mathcal{V})$ can be regarded as the category of algebras over a monad 
$\mathbb{F}_{\mathfrak{C}}$ on a simpler category
$\mathsf{Sym}^{\mathfrak{C}}(\mathcal{V})$ of $\mathfrak C$-colored symmetric sequences,
and it has been shown (e.g. \cite{BM07,BB17,WY18,Cav}) that
$\mathsf{Op}^{\mathfrak{C}}(\mathcal{V})$ may often be equipped with a model structured via transfer across the associated monadic adjunction,
which assemble (\cite{BM13,Cav}) into a Dwyer-Kan model structure structure on $\mathsf{Op}(\mathcal{V})$.

% As an example, to build the model structure on $\mathsf{Op}(\mathcal{V})$ one usually starts by building suitable model structures on the subcategories $\mathsf{Op}^{\mathfrak{C}}(\mathcal{V})$.
% Moreover, we note that $\mathsf{Op}^{\mathfrak{C}}(\mathcal{V})$ can be regarded as the category of algebras over a monad 
% $\mathbb{F}_{\mathfrak{C}}$ on a simpler category
% $\mathsf{Sym}^{\mathfrak{C}}(\mathcal{V})$ of symmetric sequences, 
% with the model structure on
% $\mathsf{Op}^{\mathfrak{C}}(\mathcal{V})$
% obtained by transferring a model structure on 
% $\mathsf{Sym}^{\mathfrak{C}}(\mathcal{V})$
% via the monadic adjunction.

As such, in order to build our desired model structure on the category $\mathsf{Op}(\mathcal{V})^G$ of $G$-equivariant colored operads, one should suitably generalize the discussion in the paragraph above.
However, a little care is needed, since the objects of a $G$-equivariant operad in $\mathsf{Op}(\mathcal{V})^G$ are now a $G$-set $\mathfrak{C} \in \mathsf{F}^G$, meaning that even when describing the objects of 
$\mathsf{Op}(\mathcal{V})^G$ one must consider maps of 
$\mathsf{Op}(\mathcal{V})$ that are not the identity on objects.
Additionally, the above cited formal approaches fail to see the automorphisms of objects in $\Sym^G_\bullet$,
which is necessary to detect the appropriate notion of weak equivalence in this context.

For these reasons, we find is useful to be able to describe 
$\mathsf{Op}(\mathcal{V})$ 
in a way that does not explicitly mention the color fixed structures
$\mathsf{Op}^{\mathfrak{C}}(\mathcal{V}),
\mathsf{Sym}^{\mathfrak{C}}(\mathcal{V}),
\mathbb{F}_{\mathfrak{C}}$.

Our solution is as follows: % set-up
there are Grothendieck fibrations 
$\mathsf{Op}(\mathcal{V}) \to \mathsf{F}$
(resp. $\mathsf{Sym}^\bullet(\mathcal{V}) \to \mathsf{F}$)
such that the fibers are precisely the categories 
$\mathsf{Op}^{\mathfrak{C}}(\mathcal{V})$
(resp. $\mathsf{Sym}^{\mathfrak{C}}(\mathcal{V})$),
as well as a monad $\mathbb{F}$ on 
$\mathsf{Sym}^\bullet(\mathcal{V})$ which is suitably fibered (\S \ref{FIBCAT_SEC}) over $\mathsf{F}$
such that the category of ``fibered algebras'' is exactly
$\mathsf{Op}(\mathcal{V})$.

\[
      \begin{tikzcd}[row sep = small, column sep = tiny]
            \Sym_\bullet(\V) \arrow[rr, "\mathbb F"] \arrow[dr]
            &&
            \Op(\V) \arrow[dl]
            &&& % ----------
            \Sym^G_\bullet(\V) \arrow[rr, "\mathbb F^G"] \arrow[dr]
            &&
            \Op^G(\V) \arrow[dl]
            \\
            &
            \Fin
            &
            &&& % ----------
            &
            \Fin^G
      \end{tikzcd}
\]

With this set-up, it is then entirely formal using Proposition \ref{DIAGRAMFM_PROP} to show that 
$\mathsf{Op}^G(\mathcal{V}) \to \mathsf{F}^G$
and
$\mathsf{Sym}^G(\mathcal{V}) \to \mathsf{F}^G$
are Grothendieck fibrations and that 
$\mathsf{Op}^G(\mathcal{V})$ is the category of fibered algebras for $\mathbb{F}^G$.
Additionally, by fibering over a fixed $G$-set $\mathfrak{C} \in \mathsf{F}^G$,
we obtain a description of the subcategory 
$\mathsf{Op}^{G,\mathfrak{C}}(\mathcal{V})$
of those $G$-operads with $G$-set of objects $\mathfrak{C}$
as the algebras for a monad $\mathbb{F}^G_{\mathfrak{C}}$
on a suitable category $\mathsf{Sym}^{G,\mathfrak{C}}(\mathcal{V})$
of symmetric sequences.


We elaborate on this narrative in the following section.
In \S \ref{SYMC_SEC}, we build the category $\Sym(\V)$ (Definition \ref{SYMV_DEF}),
describe the fibered monad $\mathbb F$ (Equation \eqref{FREEOP_EQ}, elaborated in Appendix \ref{MONAD_APDX}),
and determine the fibers of the equivariant monad $\mathbb F^G$ (Equation \eqref{FGC_EQ}).
In \S \ref{SYMC_MS_SEC} and \S \ref{OPC_MS_SEC} we build model structures on the color-fixed categories $\mathsf{Sym}^{G,\mathfrak{C}}(\mathcal{V})$ and $\mathsf{Op}^{G,\mathfrak{C}}(\mathcal{V})$,
using several pieces of equivariant homotopy theory developed in \S \ref{EHT_SEC}.





\subsection{Symmetric sequences and colored operads}
\label{SYMC_SEC}

\begin{definition}\label{CSYM DEF}
	Let $\mathfrak {C} \in \mathsf{F}$ be a set of \textit{colors}.
	A tuple
	$\ksi = (c_1, \ldots, c_n; c_0) \in \mathfrak C^{\times n} \times \mathfrak C$
	is called a \textit{$\mathfrak {C}$-signature}.
	The \textit{$\mathfrak C$-symmetric category} $\Sigma_{\mathfrak C}$ is then the category whose objects are the $\mathfrak{C}$-signatures and morphisms action maps
\[
(c_1, \ldots, c_n; c_0) \xrightarrow{\sigma} (c_{\sigma^{-1}(1)}, \ldots, c_{\sigma^{-1}(n)}; c_0)
\]
	for each permutation $\sigma \in \Sigma_n$, with the natural notion of composition.

Alternatively, we will find it useful to visualize signatures as \textit{corollas with edges decorated by colors in $\mathfrak{C}$}, as depicted below, so that the maps labeled $\sigma$ interchange the edges of the leftmost tree in such a way that one obtains the colored corolla on the left.
\[
\begin{tikzpicture}
      [grow=up,auto,level distance=2.3em,every node/.style = {font=\footnotesize},dummy/.style={circle,draw,inner sep=0pt,minimum size=1.75mm}]
      
      \node at (0,0) [font=\normalsize]{}
		child{node [dummy] {}
			child{
			edge from parent node [swap,near end] {$c_n$} node [name=Kn] {}}
			child{
			edge from parent node [near end] {$c_1$}
node [name=Kone,swap] {}}
		edge from parent node [swap] {$c_0$}
		};
		\draw [dotted,thick] (Kone) -- (Kn) ;
	\node at (5,0) [font=\normalsize]{}
		child{node [dummy] {}
			child{
			edge from parent node [swap,near end] {$c_{\sigma^{-1}(n)}$} node [name=Kn] {}}
			child{
			edge from parent node [near end] {$c_{\sigma^{-1}(1)}$}
node [name=Kone,swap] {}}
		edge from parent node [swap] {$c_0$}
		};
		\draw [dotted,thick] (Kone) -- (Kn) ;

\draw[->] (1.5,0.8) -- node{$\sigma$} (3,0.8);
\end{tikzpicture}
\]
Given any map $f \colon \mathfrak{C} \to \mathfrak{D}$ on the sets of colors, there is then a functor
$f_{\**} \colon \Sigma_{\mathfrak{C}} \to \Sigma_{\mathfrak{D}}$
given by $f_{\**} (c_1,\cdots,c_n;c_0) = (f(c_1),\cdots,f(c_n);f(c_0))$. 
\end{definition}


\begin{definition}
Let $\mathcal{V}$ be a category.
The category $\mathsf{Sym}(\mathcal{V})$ of
\textit{symmetric sequences on $\mathcal{V}$} is the category with:
\begin{itemize}
\item objects given by a set of colors $\mathfrak{C} \in \mathsf{F}$
and a functor $\Sigma_{\mathfrak{C}}^{op} \to \mathcal{V}$;
\item arrows given by a map 
$f \colon \mathfrak{C} \to \mathfrak{D}$ of colors and a natural transformation

		\begin{equation}
		\begin{tikzcd}[row sep = tiny, column sep = 35pt]
			\Sigma_{\mathfrak{C}}^{op} \arrow{dr}[name=U]{} \arrow{dd}[swap]{f_{\**}}
		\\
			& \mathcal{V}
		\\
			|[alias=V]| \Sigma_{\mathfrak{D}}^{op} \arrow{ur}[swap]{}
		\arrow[Leftarrow, from=V, to=U,shorten >=0.25cm,shorten <=0.25cm
		,swap
		]
		\end{tikzcd}
		\end{equation}
\end{itemize}
\end{definition}


\begin{remark} The natural projection map
$\mathsf{Sym}(\mathcal{V}) \to \mathsf{F}$
is a (split) Grothendieck fibration with fibers the categories
$\mathsf{Fun}(\Sigma_{\mathfrak{C}}^{op},\mathcal{V})$.
\end{remark}


\begin{remark}\label{SUBCATDOWNL REM}
$\mathsf{Sym}(\mathcal{V})$ is naturally a subcategory of the $2$-overcategory
$\mathsf{Cat}\downarrow^l \mathcal{V}$.
\end{remark}

\todo[inline]{come back - need exposition here}

We next describe the fibered monad on $\mathsf{Sym}(\mathcal{V})$ such that the fiber algebras are
$\mathsf{Op}(\mathcal{V})$.

First, much as in Definition \ref{CSYM DEF}, we need colored versions of the tree categories $\Omega$ and $\Omega^0$.
Informally, and given a set of colors $\mathcal{C}$, 
one simply lets $\Omega_{\mathfrak{C}}$
be the category of trees whose edges are decorated by colors in $\mathfrak{C}$, together with color respecting maps. More formally, $\Omega_{\mathfrak{C}}$ is given by the following pullback.
\begin{equation}
	\begin{tikzcd}
		\OC \arrow[d] \arrow[r, "E"] &
		\mathsf F \wr \mathfrak C \arrow[d]
\\
		\Omega \arrow[r, "E"] &
		\mathsf F
	\end{tikzcd}
\end{equation}

Keeping track of the colors on each edge then allows us to generalize the leaf-root and vertex functors of 
\cite{BP_geo} to get analogous functors
\[
\Omega_{\mathfrak{C}} \xrightarrow{\mathsf{lr}} \Sigma_{\mathfrak{C}}
\qquad
\Omega_{\mathfrak{C}} \xrightarrow{V} \Sigma \wr \Sigma_{\mathfrak{C}}
\]
which are readily seen to be natural with respect to maps 
$f \colon \mathfrak{C} \to \mathfrak{D}$.

If $\mathcal{V}$ is a closed symmetric monoidal category, the free operad monad $\mathbb{F}$ on $\mathsf{Sym}(\mathcal{V})$ 
assigns to a functor
$\Sigma_{\mathfrak{C}}^{op} \xrightarrow{X} \mathcal{V}$
the left Kan extension
\begin{equation}
      \label{FREEOP_EQ}
\begin{tikzcd}
	\Omega^{0,op}_{\mathfrak{C}}
	\arrow[d, "\mathsf{lr}"']
	\arrow[r, "V"]
&
	(\Sigma \wr \Sigma_{\mathfrak{C}})^{op} \arrow[r, "X"]
	\arrow[dl, Rightarrow]
&
	(\Sigma \wr \V^{op})^{op} \arrow[r, "\otimes"]
&
	\V
\\
	\Sigma^{op}_{\mathfrak{C}} \arrow[urrr, "\Lan = \mathbb F_{\mathfrak{C}} X"']
\end{tikzcd}
\end{equation}

A complete discussion and rigorous definition this monad can be found in Appendex \ref{MONAD_APDX},
culmintating in Definition \ref{COLORMON DEF}.

Our next goal is to provide a convenient explicit description of the fibers of $\mathsf{Sym}(\mathcal{V})^G \to \mathsf{F}^G$
for each $\mathfrak{C} \in \mathsf{F}^G$
and of the restriction of the monad $\mathbb{F}^G$ to those fibers.

We start with the following, which is a slight strengthening of
\cite[Lemma A.6]{BP_geo}.

\begin{lemma}\label{EQUIVFUNCONV LEM}
Let $G$ be a group and $\mathfrak{C} \colon G \to \mathsf{F}$ be a $G$-set (or, more generally, $\mathcal{D}$ a category and 
$\mathcal{C}_{\bullet}\colon \mathcal{D} \to \mathsf{Cat}$ a functor). Then category of sections as on the left below (or, more generally, as on the left)
\begin{equation}
	\begin{tikzcd}
		&
		\mathsf{Sym}(\mathcal{V}) \arrow{d}{\mathsf{fgt}}
&
		&
		\mathsf{Cat}\downarrow^l \mathcal{V} \arrow{d}{\mathsf{fgt}}
\\
		G \arrow{r}[swap]{\mathfrak{C}} \arrow[dashed]{ru} &
		\mathsf{F}
&
		\mathcal{D} \arrow{r}[swap]{\mathcal{C}_{\bullet}} \arrow[dashed]{ru} &
		\mathsf{Cat}
	\end{tikzcd}
\end{equation}
is isomorphic to the functor category
$\mathsf{Fun}(G\ltimes \Sigma_{\mathfrak{C}}^{op},\mathcal{V})$
(or, more generally, $\mathsf{Fun}(\mathcal{D} \ltimes \mathcal{C}_{\bullet},\mathcal{V})$).
\end{lemma}


\begin{proof}
Since there is a natural inclusion
$\mathsf{Sym}(\mathcal{V}) 
\subseteq 
\mathsf{Cat} \downarrow^l \mathcal{V}$
and the assignment $\mathsf{F} \to \mathsf{Cat}$ 
given by $\mathfrak{C} \mapsto \Sigma_{\mathfrak{C}}$ 
is faithful, it suffices to check the more general claim concerning
$\mathsf{Cat} \downarrow^l \mathcal{V}$.

The remainder of the proof is simply a matter of unpacking definitions. For example, in both cases objects consist of collections of functors $\mathcal{C}_d \to \mathcal{V}$ for $d \in \mathcal{D}$ together with natural transformations
	\begin{equation}
	\begin{tikzcd}[row sep = tiny, column sep = 35pt]
		\mathcal{C}_{d} \arrow{dr}[name=U]{} \arrow{dd}[swap]{\mathcal{C}_{f}}
	\\
		& \mathcal{V}
	\\
		|[alias=V]| \mathcal{C}_{\bar{d}} \arrow{ur}[swap]{}
	\arrow[Leftarrow, from=V, to=U,shorten >=0.25cm,shorten <=0.25cm
	,swap
	]
	\end{tikzcd}
	\end{equation}
for each arrow $f \colon d \to \bar{d}$ in $\mathcal{D}$, subject to natural unitality and associativity requirements.
\end{proof}



In accordance with the previous lemma, we will represent elements of $\mathsf{Sym}^G(\mathcal{V})$
by functors
$G \ltimes \Sigma_{\mathfrak{C}}^{op} \to \mathcal{V}$ for some $\mathfrak{C} \in \mathsf{F}^G$.
Our next step is describe the monad $\mathbb{F}^G$ on $\mathsf{Sym}^G(\mathcal{V})$.

We first note that for any 
$A \in \mathsf{Cat}^G$ there is a natural transformation
$G \ltimes \Sigma \wr A \to \Sigma \wr G \ltimes A$
characterized by sending a $G$-action arrow 
$(a_i) \xrightarrow{g} (g a_i)$
to the diagonal tuple of $G$-action arrows
$(a_i \xrightarrow{g} g a_i)$.
This natural transformation can then be used to describe the $G$-equivariant $\Sigma \wr (-)$ functor, via
\begin{equation}\label{RHOPURP EQ}
\begin{tikzcd}[column sep =40,row sep =0]
	\left( \mathsf{Cat} \downarrow^{l} \mathcal{V} \right)^G
	\ar{r}{\left(\Sigma \wr (-) \right)^G} &
	\left( \mathsf{Cat} \downarrow^{l} \Sigma \wr \mathcal{V} \right)^G
\\
        G \ltimes A \to \mathcal{V} \ar[mapsto]{r} &
	(G \ltimes \Sigma \wr A \to 
	\Sigma \wr G \ltimes  A \to \Sigma \wr \mathcal{V})
\end{tikzcd}
\end{equation}
and combining this with the observation that Kan extensions along
$G \ltimes \Omega_{\mathfrak{C}} \to G \ltimes \Sigma_{\mathfrak{C}}$
are simply the underlying Kan extension along 
$\Omega_{\mathfrak{C}} \to \Sigma_{\mathfrak{C}}$
together with equivariance data (this strengthens Lemma \ref{REDUCELAN LEM} a little),
one obtains that the monad $\mathbb{F}^G$ applied to 
$X\colon G \ltimes \Sigma_{\mathfrak{C}}^{op} \to \mathcal{V}$
is the left Kan extension
\begin{equation}\label{FGC_EQ}
	\begin{tikzcd}
		G \ltimes \Omega^{0,op}_{\mathfrak{C}}
		\arrow[d, "\mathsf{lr}"']
		\arrow[r, "V"]
	&
		(G\ltimes \Sigma \wr \Sigma_{\mathfrak{C}})^{op} \arrow{r}
		\arrow[dl, Rightarrow]
	&
		(\Sigma \wr G\ltimes \Sigma_{\mathfrak{C}})^{op} \arrow[r, "X"]
	&
		(\Sigma \wr \V^{op})^{op} \arrow[r, "\otimes"]
	&
		\V
\\
	G\ltimes\Sigma^{op}_{\mathfrak{C}}
	\arrow[urrrr, "\Lan = \mathbb F_{\mathfrak{C}}^G X"']
	\end{tikzcd}
\end{equation}


{\color{red} HERE}

\todo[inline]{these should go somewhere else...}
\begin{remark}
Unlike in the single-colored case, $\Op^G(\V)$ does \textit{not} coincide with the category of colored operads in $\V^G$.
Indeed, objects in $\Op(\V^G)$ have a fixed $G$-set of colors,
and each level $\O(\xi)$ has an action by the full group $G$
(though only a partial action by $\Sigma_{|\xi|}$).
\end{remark}



\begin{remark}
We have inclusion-forgetful fibered adjunctions
\begin{equation}\label{JSTAR_CAT_EQ}
\begin{tikzcd}
	\Cat^{G}(\V)
	\arrow[shift left]{r}{j_!}
	\arrow[d, "{(-)^H}"']
&
	\Op^{G}(\V)
	\arrow[shift left]{l}{j^{\**}}
	\arrow{d}{(-)^H}
\\
	\Cat(\V)
	\arrow[shift left]{r}{j_!}
&
	\Op(\V)
	\arrow[shift left]{l}{j^{\**}}
\end{tikzcd}
\end{equation}
where both left and right adjoints are compatible the $H$-fixed point functors.
\end{remark}


\subsubsection{Change of color stuff}

\begin{remark}\label{OP_MAP REM}
      Following Remark \ref{ALGPUSHLL REM},
      for any map of $G$-sets $f \colon \mathfrak C \to \mathfrak D$
      and $\mathfrak D$-symmetric sequence $X$
      one has a pullback $\mathfrak C$-symmetric sequence $f^{\**}X$
      given by
      $ f^{\**}X(\xi') = X(f(\xi'))$
      which is an operad if $X$ itself is an operad.
      Moreover, the two $f^{\**}$ functors fit into a pair of adjunctions 
      \begin{equation}\label{GC_CHANGE_EQ}
            \begin{tikzcd}
                  \Op^G_{\mathfrak C}(\V) 
                  \arrow[shift left]{r}{\check{f}_!}
                  \arrow[d, "\mathsf{fgt}"']
                  &
                  \Op^G_{\mathfrak D}(\V) 
                  \arrow[shift left]{l}{f^{\**}}
                  \arrow[d, "\mathsf{fgt}"]
                  \\
                  \Sym^G_{\mathfrak C}(\V) 
                  \arrow[shift left]{r}{f_!}
                  &
                  \Sym^G_{\mathfrak D}(\V) 
                  \arrow[shift left]{l}{f^{\**}}
            \end{tikzcd}
      \end{equation}
      where we highlight that
      the right adjoints are compatible with the forgetful functors, but the left adjoints are not:
      $f_!$ is given by a left Kan extension, while $\check{f}_!$ is given by the coequalizer
      \begin{equation}
            \label{CFS_EQ}
            \check{f_!} \O \simeq \mathop{coeq}(\mathbb F_{\mathfrak D} f_! \mathbb F_{\mathfrak C}\O \rightrightarrows \mathbb F_{\mathfrak D} f_! \O).
      \end{equation}
      In general, we cannot give an explicit description of either $f_!$ or $\check f_!$.
      However, when $f$ is injective, $U \check f_! = f_! U$ and $f_!X$ is simply $X \amalg \varnothing_{\mathfrak D \setminus \mathfrak C}$;
      see Appendix \S \ref{FREEOPCOL_SEC}.
\end{remark}



\begin{remark}\label{COLOR_SQ_REM}
      We record the following straightforward observations.
      If we are given a map $F: \O_1 \to \O_2$ that is color-fixed,
      and a square in $\Op^G(\V)$ as in the middle of \eqref{COLOR_SQ_EQ}, then:
      \begin{enumerate}[label = (\roman*)]
      \item the square in the middle commutes iff the square on the right commutes iff the square on the left commutes, and;
      \item the square in the middle is a pushout in $\Op^G(\V)$ iff
            the square on the left is a pushout in $\Op^{G}_{\mathfrak C_{\P_1}}(\V)$.
      \end{enumerate}
      \begin{equation}
            \label{COLOR_SQ_EQ}
            \begin{tikzcd}
                  a_! \O_1 \arrow[d, "a_! F"'] \arrow[r, "a"]
                  &
                  \P_1 \arrow[d, "p"]
                  &&
                  \O_1 \arrow[d, "F"'] \arrow[r, "a"]
                  &
                  \P_1 \arrow[d, "p"]
                  &&
                  \O_1 \arrow[d, "F"'] \arrow[r, "a"]
                  &
                  a^{\**} \P_1 \arrow[d]
                  \\
                  a_! \O_2 \arrow[r]
                  &
                  \P_2
                  &&
                  \O_2 \arrow[r]
                  &
                  \P
                  &&
                  \O_2 \arrow[r]
                  &
                  a^{\**} p^{\**} \P
            \end{tikzcd}
      \end{equation}
\end{remark}







\subsection{Equivariant homotopy theory}
\label{EHT_SEC}


\begin{definition}\label{FAMGROUPOID DEF}
Let $\G$ be a groupoid.
A family of subgroups of $\G$
is a collection 
$\mathcal{F} = \left\{\mathcal{F}_x | x\in \G\right\}$
where each $\F_x$ is itself a collection of subgroups
$H \leq Aut(x)$ and such that:
\begin{itemize}
\item if $H \in \F_x$ and $K \leq H$ then $K \in \mathcal{F}_x$;
\item if $H \in \mathcal{F}_x$
then for any arrow $x \xrightarrow{g} x'$
it is $H^g=g H g^{-1} \in \mathcal{F}_{x'}$.
\end{itemize}
\end{definition}



\begin{remark}
If $\G$ has a single object, i.e. if $\G$ is a group $G$ regarded as a category with a single object, Definition \ref{FAMGROUPOID DEF} recovers the usual definition of a family of subgroups of $G$ as a collection of subgroups $H\leq G$ closed under inclusion and conjugation.

Moreover, if $\F$ is a family of subgroups of $\G$, each $\F_x$ is a family of subgroups of $Aut(x)$ in the usual sense. 
In fact, $\mathcal{F}$ is completely determined by a choice of families
$\F_x$ for $x$ ranging over a set of representatives of the isomorphism classes/components of $\G$.
\end{remark}


\begin{remark}
One allows for some (or even all) of the $\F_x$ to be empty.
\end{remark}



The following definition is adapted from \cite{Ste16}.

\begin{definition}\label{CELLFP_DEF}
We say $\V$ has \textit{cellular fixed points} if
for all finite groups $G$ and subgroups $H, K \leq G$ one has that:
	\begin{enumerate}[label = (\roman*)]
	\item $(-)^H \colon \V^G \to \V$ preserves direct colimits of diagrams in $\V^G$ where each underlying arrow in $\V$ is a cofibration;
      \item $(-)^H \colon \V^G \to \V$ preserves pushouts where one leg is $(G/K) \cdot f$, for $f$ a cofibration in $\V$.
      \item for each object $X \in \V$, the natural map $(G/K)^H \cdot X \to ((G/K) \cdot X)^H$ is an isomorphism.
      \end{enumerate}
\end{definition}


\begin{remark}\label{LEVEL_COF_REM}
	A consequence of $\V$ having cellular fixed points (cf. the proof of \cite[Prop. 6.3(i)]{BP_geo})
	is that for any genuine (trivial) cofibration $f \in \V^G_{gen}$,
	the map $f^H$ is a (trivial) cofibration in $\V$ for all $H \leq G$.
	{\color{blue} (Luis: this might need cofibrant generation as well)}
\end{remark}



For any groupoid $\G$ there is an equivalence of categories
$\G \simeq \coprod_{[x] \in ob(\G)/\simeq} Aut(x)$,
where $ob(\G)/\simeq$
denotes isomorphism classes of objects,
and thus also equivalences
$\V^{\G} \simeq \prod_{[x] \in ob(\G)/\simeq} \V^{Aut(x)}$,
so that \cite[Thm. 2.10]{Ste16}
immediately implies the following.


\begin{theorem}\label{FMODEL THM}
	If $\V$ is a cofibrantly generated model category with cellular fixed points, $\G$ is a groupoid and $\F$ a family of subgroups on $\G$,
	then $\mathcal{V}^{\G}$ has a model structure where a map
	$A \to B$ in $\mathcal{V}^{\G}$ is a 
	weak equivalence (resp. fibration) if the maps
	$A(x)^H \to B(x)^H$
	are weak equivalences (fibrations) in $\V$
	for all $x \in \G$, $H \in \F_x$.
\end{theorem}      

We refer to the model structure on $\V^{\G}$
in the result above as the $\F$-model structure, 
and denote it $\V^{\G}_{\F}$.


\begin{remark}\label{VGFGEN REM}
	Letting $\mathcal{I}$ (resp. $\mathcal{J}$)
	denote the sets of generating (resp. trivial) cofibrations of 
	$\V$, the sets of generating (trivial) cofibrations of $\V^{\G}_{\F}$ are then given by the sets of maps
\begin{equation}\label{VGFGEN EQ}
\left\{
	\G(x,-)/H \cdot i\right\} \qquad \left\{\G(x,-)/H \cdot j
\right\}
\end{equation}
where $x$ ranges over the objects of $\G$, $H$ ranges over $\F_x$,
$i$ ranges over $\mathcal{I}$ and 
$j$ ranges over $\mathcal{J}$.
\end{remark}


%For completeness, we recall a main result of \cite{Ste16}, which transfers model structures along several adjuctions.
%\begin{theorem}[{\cite[Thm. 2.10]{Ste16}}]
%      Let $\V$ be cofibrantly generated model category with cellular fixed points, $\F$ any family of subgroups of $G$ containing the trivial subgroup, and $\mathsf O_\F$ the full subcategory of $\mathsf O_G$ spanned by the $G/H$ with $H \in \F$.
%      Then
%      \begin{itemize}
%      \item the projective model structure on $\V^{\mathsf O_\F^{op}}$ exists;
%      \item the $\F$-model structure on $\V^G$ exists, where an arrow $f \in \V$ is a weak equivalence or fibration iff each $f^H$ is in $\V$ for each $H \in \F$; and
%      \item the inclusion $\V^G \xrightarrow{i_{\**}} \V^{\mathsf O_F^{op}}$, $i_{\**}X(G/H) = X^H$ is a right Quillen equivalence.
%      \end{itemize}
%\end{theorem}      


\begin{definition}\label{GENMOD DEF}
If $\F = \F_{all}$ is the collection of all subgroups of a group $G$, we denote the model structure $\V^{G}_{\F_{all}}$ simply by $\V^G$,
and refer to it as the \textit{genuine} or \textit{fine} model structure on $\V^G$,
with genuine (trivial) (co)fibrations and genuine weak equivalences.
\end{definition}


\begin{remark}
The families of subgroups $\F$ of a groupoid $\G$ 
form a poset (in fact lattice) under inclusion,
which we denote $\mathsf{Fam}_{\G}$. 

Given any map of groupoids $\phi\colon \G \to \bar{\G}$
one obtains a pullback functor
$\phi^{\**} \colon \mathsf{Fam}_{\bar{\G}} \to \mathsf{Fam}_{\G}$
defined by
\[
\left(\phi^{\**} \bar{\F} \right)_{x}
=
\left\{H \leq Aut(x)| \phi(H) \in \bar{\F}_{\phi(x)}\right\}.
\]
(in fact, since $\phi^{\**}$ preserves arbitrary unions and intersection of families and 
$\mathsf{Fam}_{\bar{\G}}$
is a complete lattice, $\phi^{\**}$ admits both a left and a right adjoint).
%In fact, one further has adjunctions
%\[
%	\phi_! \colon
%	\mathsf{Fam}_{\G}
%	\rightleftarrows
%	\mathsf{Fam}_{\bar{\G}}
%	\colon \phi^{\**}
%\qquad
%	\phi^{\**} \colon
%	\mathsf{Fam}_{\bar{\G}}
%	\rightleftarrows
%	\mathsf{Fam}_{\G}
%	\colon \phi_{\**}
%\]
%where $\phi_!, \phi_{\**}$ are defined by
%\[
%	\left(\phi_! \F\right)_{\bar{x}}	=
%	\left\{\bar{g}\phi(H)\bar{g}^{-1}|
%	x\in \G,H\in\F_x,\bar{g}\colon \phi(x) \to \bar{x}
%	\right\}
%\]
%\[
%	\left(\phi_{\**} \F\right)_{\bar{x}}	=
%	\left\{\bar{H} \leq Aut(\bar{x})|
%	\forall_{x \in \G, \bar{g} \colon \bar{x} \to \phi(x) }
%	\exists_{H \in \F_x}
%	\bar{g} 	\bar{H} \bar{g}^{-1} = \phi(H) 
%	\right\}
%\]
\end{remark}

The relevance of the previous remark is given by the following result,
which rephrases \cite[Props. 6.6, 6.7]{BP_geo}.

\begin{proposition}\label{EQQUILADJ PROP}
Let $\phi \colon \G \to \bar{\G}$
be a map of groupoids and
$\F,\bar{\F}$ families of subgroups of $\G,\bar{\G}$.
Then the adjunctions
\[
	\phi_! \colon \V^{\G}_{\F}
	\rightleftarrows
	\V^{\bar{\G}}_{\bar{\F}} \colon \phi^{\**}
\qquad
	\phi^{\**} \colon \V^{\bar{\G}}_{\bar{\F}} 
	\rightleftarrows
	\V^{\G}_{\F} \colon \phi_{\**}
\]
are Quillen adjunctions provided
$\F \subseteq \phi^{\**} \bar{\F}$
in the first case and 
$\phi^{\**} \bar{\F} \subseteq \F$
in the second case.
\end{proposition}





We next discuss the interactions of equivariant model structures on 
$\mathcal{V}$ with the monoidal structure $\otimes$ on $\mathcal{V}$.
The following is immediate from
\cite[Rem. 6.14]{BP_geo}.


\begin{proposition}\label{RESGEN PROP}
Suppose $(\V, \otimes)$ is a closed monoidal model category which is cofibrantly generated and has cellular fixed points.
Further, let $\G, \bar{\G}$ be groupoids and $\F,\bar{\F}$
families of subgroups of $\G, \bar{\G}$.
Then $\otimes$ incudes a left Quillen bifunctor
\[
	\V^{\G}_{\F} \times \V^{\bar \G}_{\bar{\F}} \xrightarrow{\otimes} \V^{\G \times \bar \G}_{\F \sqcap \bar{\F}}
\]
where the family $\F \sqcap \bar{\F}$ of $\G \times \bar{\G}$ is defined by
\[
\left(\F \sqcap \bar{\F}\right)_{(x,\bar{x})}
=
\left\{K\leq Aut(x,\bar{x})\ |\ \pi_{\G} (K) \in \F_x,
\pi_{\bar{\G}} (K) \in \F_{\bar{x}}
%\leq H \times \bar{H} \text{ for some }H\in \F_x,\bar{H} \in \bar{\F}%_{\bar{x}}
\right\}.
\]
\end{proposition}

Unpacking the definition of left Quillen bifunctor,
Proposition \ref{RESGEN PROP}
says that if maps
$f, \bar{f}$
in 
$\V^{\G}$,
$\V^{\bar{\G}}$
are 
$\F,\bar{\F}$
cofibrations
then the map
$f\square \bar{f}$
in $\mathcal{V}^{\G \times \bar{\G}}$
is a 
$\F \sqcap \bar{\F}$-cofibration,
which is trivial if either $f$ or $\bar{f}$ are.

Note, however, that should $\otimes$ be a symmetric monoidal structure, then when $\G = \bar \G$ and $f = \bar f$
the map $f \square f$
admits an additional $\Sigma_2$-action
(and, more generally, $f^{\square n}$ admits a $\Sigma_n$-action)
which is ignored by Proposition \ref{RESGEN PROP}.
To discuss such ``actions on powers'' we need a few preliminaries, 
starting with the following additional hypothesis on $\V$.


\begin{definition}[{\cite[Defn 6.16]{BP_geo}}]\label{CSPP_DEF}
      We say a symmetric monoidal model category $\V$ has \textit{cofibrant symmetric pushout powers} if
      for all (trivial) cofibrations $f$, the pushout product power $f^{\square n}$
      is a $\Sigma_n$-genuine (trivial) cofibration in $\V^{\Sigma_n}$ for all $n \geq 1$. 
\end{definition}

\begin{remark}
We purposefully excluded the $n=0$ case in Definition \ref{CSPP_DEF}
for the case of consistency with Proposition \ref{SIGMAWRGF PROP}
where the trivial cofibrancy case can only be expected to hold for $n\geq 1$.

Nonetheless, unwinding definitions one sees that
$f^{\square 0}$
should be the map 
$\emptyset \to 1$
from the initial object to the monoidal unit
(indeed, $\square$ can be regarded as a monoidal structure on the category of arrows, with $\emptyset \to 1$ being the unit).
As such, the analogue of Definition \ref{CSPP_DEF} for $n=0$ 
recovers the condition that $\V$ has a cofibrant unit.
\end{remark}


Given a map $f$ in $\mathcal{V}^{\G}$
we write 
$f^{\square n}$ 
for the map in $\mathcal{V}^{\Sigma_n \wr \G}$
given by
\begin{equation}\label{FSQNDEF EQ}
f^{\square n}\left((x_i)_{1\leq i \leq n }\right)
=
\underset{1\leq i \leq n}{\mathlarger{\mathlarger{\square}}} f(x_i).
\end{equation}
Next, we write
\[
\pi_{\Sigma} \colon \Sigma_n \ltimes \G^{\times n} \to \Sigma_n
\qquad
\pi^i_{\G} \colon \Sigma_n \ltimes \G^{\times n} \to \G, 1\leq i \leq n
\]
for the projection onto each coordinate.
Note that while $\pi_{\Sigma}$ is a map of groupoids, 
the $\pi^i_{\G}$ are not.
\todo[inline]{$\pi^i$ is only a set map, yes?}
Nonetheless, writing $\Sigma^i_n \leq \Sigma_n$
for the subgroup of permutations that fix $i$, 
one has that $\pi_{G}^i$ is a homomorphism when restricted to 
$\pi^{-1}_{\Sigma}(\Sigma_n^i)$ 
(indeed, one has an isomorphism 
$\pi^{-1}_{\Sigma}(\Sigma_n^i) \simeq 
(\Sigma_{1} \wr \G) \times (\Sigma_{n-1} \wr \G)$
which identifies $\pi_{\G}^i$ with the projection to 
$(\Sigma_{1} \wr \G) \simeq \G$).
Given $(x_i) \in \Sigma_n \wr \G$
and a subgroup
$H \leq Aut((x_i))$,
we then write
$H_i = H \cap \pi^{-1}_{\Sigma}(\Sigma_n^i)$
for the subgroup whose projection to $\Sigma$ fixes $i$.

Given a family $\F$ of subgroups of $\G$
we can now finally define the family $\F^{\ltimes n}$
of subgroups of $\Sigma_n \wr \G$ for $n\geq 1$ by
\begin{equation}\label{FWRNXI EQ}
\left(\F^{\ltimes n}\right)_{(x_i)}
=
\left\{
H \leq Aut((x_i))|
\pi^i_{\G}(H_i) \in \F_{x_i} \text{ for all i}
\right\}
\end{equation}

We then have the following, which is a minor strengthening of 
\cite[Prop. 6.24]{BP_geo}.

\begin{proposition}\label{SIGMAWRGF PROP}
Suppose $(\V, \otimes)$ is as in Proposition \ref{RESGEN PROP} and, in addition, it also has cofibrant symmetric pushout powers.
Further, let $\G$ be a groupoid and
$\F$ a family of subgroups of $\G$.
Then:
\begin{enumerate}[label=(\roman*)]
\item if $f$ is a (trivial) $\F$-cofibration in $\V^{\G}$
then $f^{\square}$ is a (trivial)
$\F^{\ltimes}$-cofibration in $\V^{\Sigma \wr \G}$;
\end{enumerate}
\end{proposition}


\begin{proof}
Since any groupoid is equivalent to a disjoint union of groups, we reduce to that case. In other words, we may assume that two objects of $\G$ are isomorphic iff they are equal.

Given $(x_i) \in \Sigma_n \ltimes \G^{\times n}$,
we thus want to check that \eqref{FSQNDEF EQ}
is a (trivial) $\left(\F^{\ltimes n}\right)_{(x_i)}$-cofibration.
Now form the partition 
$\{1,\cdots,n\} = \lambda_1 \amalg \cdots \amalg \lambda_k$
such that $i,j$ are in the same piece iff $x_i=x_j$ and write
$n_l = |\lambda_l|$.
Writing $x_{\lambda_l}$ for the common value of the $x_i$ with $i\in \lambda_l$, we then have
\[
f^{\square n}\left((x_i)_{1\leq i \leq n }\right)
\simeq
\underset{1\leq l \leq k}{\mathlarger{\mathlarger{\square}}} f(x_{\lambda_l})^{\square n_l}.
\]
Writing $G_l$ for the automorphism group of $x_{\lambda_l}$,
the analogue result for groups \cite[Prop. 6.24]{BP_geo}
(which is the hard case) implies that
$f(x_{\lambda_l})^{\square n_l}$
is a (trivial) $\F_{x_{\lambda_l}}^{\ltimes n_l}$-cofibration
in $\V^{\Sigma_{n_l}\ltimes G_l}$


Next, note that for any subgroup
$H \leq Aut((x_i))$ the projection 
$\pi_{\Sigma}(H)$ must preserve the partition 
(or else there would be distinct $x_i$ which are isomorphic) 
so that, writing 
$\pi_{\G^{\times \lambda_l}} \colon
\Sigma_n \ltimes \G^{\times n} \to \G^{\times \lambda_l}$
for the projections one has that the 
$\pi_{\G^{\times \lambda_l}}$ are homomorphisms when restricted to $H$ so that the result now follows by Proposition \ref{RESGEN PROP}
and the observation that
$H \in \left(\F^{\ltimes n}\right)_{(x_i)}$
iff
$\pi_{\G^{\times \lambda_l}}(H) \in \F_{x_{\lambda_l}}^{\ltimes n_l}$
for all $l$.
\end{proof}


\begin{remark}
Proposition \ref{SIGMAWRGF PROP} almost subsumes Proposition \ref{RESGEN PROP}. By considering the disjoint union groupoid 
$\G \amalg \bar{\G}$
with the disjoint union family 
$\F \amalg \bar{\F}$
one can check that for
$x \in \G$, $\bar{x} \in \bar{\G}$,
then for the point $(x,\bar{x}) \in \Sigma_2 \wr (\G \amalg \bar{\G})$
it is
$\left(\left(\F \amalg \bar{\F}\right)^{\ltimes 2}\right)_{(x,\bar{x})} = \F \sqcap \bar{\F}$.
\end{remark}


\subsection{Homotopy theory of symmetric sequences with a fixed color $G$-set}



Our main example of interest will be given by the groupoids of the form
$G \ltimes \Sigma^{op}_{\mathfrak{C}}$ for 
$\mathfrak{C} \in \mathsf{F}^G$.
Moreover, our choice of families will be such that the
change of colors $f\colon \mathfrak{C} \to \mathfrak{D}$ 
induce Quillen adjunctions.


\begin{definition}\label{GSFAM_DEF}
A \emph{$(G,\Sigma)$-family} is a family of subgroups $\mathcal{F}$ in $G \times \Sigma^{op}$.

Moreover, given a $(G,\Sigma)$-family $\F$
one defines a family
$\mathcal{F}_{\mathfrak{C}}$ in
$G \ltimes \Sigma^{op}_{\mathfrak{C}}$
for each $\mathfrak{C} \in \mathsf{F}^G$
via $\mathcal{F}_{\mathfrak{C}} = \pi^{\**}_{\mathfrak{C}}(\mathcal{F})$
for $\pi_{\mathfrak{C}} \colon G \ltimes \Sigma_{\mathfrak{C}}^{op} \to G \times \Sigma^{op}$
the canonical forgetful functor.
\end{definition}



\begin{remark}
A $(G,\Sigma)$-family $\F$
is determined by a collection of families
$\F_n$ of the groups $G\times \Sigma_n^{op}$ for each $n \geq 0$.
\end{remark}



\begin{notation}\label{HASUNIT NOT}
If $\F_1$ consists of all the subgroups 
$H \leq G \simeq G \times \Sigma_1^{op}$, 
we say that \textit{$\F$ has units}.
\end{notation}



\begin{remark}\label{SIGACT REM}
The group $G \times \Sigma_n^{op}$
naturally on the set $\Sigma_{\mathfrak{C},n}$
of $n$-ary $\mathfrak{C}$-corollas
$\vect{C} = (\mathfrak{c}_1,\cdots,\mathfrak{c}_n;\mathfrak{c}_0)$
via the assigment (where $g \in G$, $\sigma \in \Sigma_n$)
\[
g \sigma \vect{C} =
g \sigma (\mathfrak{c}_1,\cdots,\mathfrak{c}_n;\mathfrak{c}_0)
=
(g\mathfrak{c}_{\sigma(1)},\cdots,g\mathfrak{c}_{\sigma(n)};g\mathfrak{c}_{\sigma(0)})
\]
or, more compactly, $g \sigma (\mathfrak{c}_i) = (g \mathfrak{c}_{\sigma(i)})$.

Moreover, the subcategory of $G \ltimes \Sigma^{op}_{\mathfrak{C}}$
spanned by the $n$-ary corollas is the action groupoid for this action.
\end{remark}



\begin{remark}
The functors
$\pi_{\mathfrak{C}} \colon
G \ltimes \Sigma_{\mathfrak{C}}^{op} \to
G \times \Sigma^{op}$
are faithful and can thus be regarded as inclusions.
Hence, letting $\vect{C} \in \Sigma_{\mathfrak{C}}$ be a
$\mathfrak{C}$-colored corolla with $n$ leaves,
one has that
$\F_{\mathfrak{C}} = \{\F_{\vect{C}}\}$ where
\begin{equation}\label{FAMC_DEF_EQ}
	\F_{\vect{C}} = \F_n \cap \Aut_{G \ltimes \Sigma_{\mathfrak C}^{op}}(\vect C).
\end{equation}
\end{remark}


We will consider three main examples of $(G,\Sigma)$-families.

First, there is the family $\F^{all}$ of all subgroups of $G \times \Sigma^{op}$
(in which case the $\F^{all}_{\mathfrak{C}}$ are also the families of all subgroups), which is useful for technical purposes.

Secondly, there is the family of $\F^{\Gamma}$
of $G$-graph subgroups (further discussed in Definition \ref{GRAPHSUB_DEF}), where $\F^{\Gamma}_n$
consists of the subgroups
$\Gamma \leq G \times \Sigma_n^{op}$
such that $\Gamma \cap \Sigma_n^{op} = \**$.
We note that the elements of $\Gamma$
have the form $(h,\phi(h)^{-1})$
for $h$ ranging over some subgroup $H \leq G$
and $\phi \colon H \to \Sigma_n$
a homomorphism,
motivating the ``graph subgroup'' terminology.

Lastly, there are the indexing systems of Blumberg and Hill,
which are certain special subfamilies of $\F^{\Gamma}$,
and are explored in \S \ref{INDSYS SEC}.



\begin{example}
Let $G = \mathbb{Z}_{/2} = \{\pm 1\}$ and 
$\mathfrak{C} = \{\mathfrak{a}, -\mathfrak{a}, \mathfrak{b}\}$ where we implicitly have
$-\mathfrak{b} = \mathfrak{b}$.
Consider the two $\mathfrak{C}$-corollas 
$\vect{C},\vect{D} \in \Sigma_{\mathfrak{C}}$ below.
\begin{equation}
	\begin{tikzpicture}[auto,grow=up, level distance = 2.2em,
	every node/.style={font=\scriptsize,inner sep = 2pt}]%
		\tikzstyle{level 2}=[sibling distance=3em]%
			\node at (0,0) [font = \normalsize] {$\vect{C}$}%	
				child{node [dummy] {}%
					child{node {}%
					edge from parent node [swap] {$-\mathfrak{a}$}}%
					child[level distance = 2.9em]{node {}%
					edge from parent node [swap,	near end] {$\mathfrak{b}$}}%
					child[level distance = 2.9em]{node {}%
					edge from parent node [near end] {$\mathfrak{b}$}}%
					child{node {}%
					edge from parent node  {$\mathfrak{a}$}}%
				edge from parent node [swap] {$\mathfrak{b}$}};%
			\node at (7,0) [font = \normalsize] {$\vect{D}$}%	
				child{node [dummy] {}%
					child{node {}%
					edge from parent node [swap] {$-\mathfrak{a}$}}%
					child[level distance = 2.9em]{node {}%
					edge from parent node [swap,	near end] {$-\mathfrak{a}$}}%
					child[level distance = 2.9em]{node {}%
					edge from parent node [near end] {$\mathfrak{a}$}}%
					child{node {}%
					edge from parent node  {$\mathfrak{a}$}}%
				edge from parent node [swap] {$\mathfrak{b}$}};%
	\end{tikzpicture}%
\end{equation}%
The non-trivial $G$-graph subgroups of
$\F^{\Gamma}_{\vect{C}}$,
$\F^{\Gamma}_{\vect{D}}$
then correspond to the possible $\mathbb{Z}_{/2}$-actions on the underlying trees $C,D$ which are compatible with the action on labels
(in the sense that the composites
$E(C) \xrightarrow{-1} E(C) \to \mathfrak{C}$
and 
$E(C) \to \mathfrak{C} \xrightarrow{-1} \mathfrak{C}$ coincide).
In this particular case, both 
$\F^{\Gamma}_{\vect{C}}$,
$\F^{\Gamma}_{\vect{D}}$
have exactly two non-trivial groups,
which correspond to the $\mathbb{Z}_{/2}$-actions on the underlying corollas depicted below.
\begin{equation}
	\begin{tikzpicture}[auto,grow=up, level distance = 2.2em,
	every node/.style={font=\scriptsize,inner sep = 2pt}]%
		\tikzstyle{level 2}=[sibling distance=3em]%
			\node at (-1.6,0) [font = \normalsize] {$C_1$}%	
				child{node [dummy] {}%
					child{node {}%
					edge from parent node [swap] {$-a$}}%
					child[level distance = 2.9em]{node {}%
					edge from parent node [swap,	near end] {$c\phantom{b}$}}%
					child[level distance = 2.9em]{node {}%
					edge from parent node [near end] {$b$}}%
					child{node {}%
					edge from parent node  {$a$}}%
				edge from parent node [swap] {$r$}};%
			\node at (1.6,0) [font = \normalsize] {$C_2$}%	
				child{node [dummy] {}%
					child{node {}%
					edge from parent node [swap] {$-a$}}%
					child[level distance = 2.9em]{node {}%
					edge from parent node [swap,	near end] {$-b$}}%
					child[level distance = 2.9em]{node {}%
					edge from parent node [near end] {$b$}}%
					child{node {}%
					edge from parent node  {$a$}}%
				edge from parent node [swap] {$r$}};%
			\node at (5.4,0) [font = \normalsize] {$D_1$}%	
				child{node [dummy] {}%
					child{node {}%
					edge from parent node [swap] {$-a$}}%
					child[level distance = 2.9em]{node {}%
					edge from parent node [swap,	near end] {$-b$}}%
					child[level distance = 2.9em]{node {}%
					edge from parent node [near end] {$b$}}%
					child{node {}%
					edge from parent node  {$a$}}%
				edge from parent node [swap] {$r$}};%
			\node at (8.6,0) [font = \normalsize] {$D_2$}%	
				child{node [dummy] {}%
					child{node {}%
					edge from parent node [swap] {$-b$}}%
					child[level distance = 2.9em]{node {}%
					edge from parent node [swap,	near end] {$-a$}}%
					child[level distance = 2.9em]{node {}%
					edge from parent node [near end] {$b$}}%
					child{node {}%
					edge from parent node  {$a$}}%
				edge from parent node [swap] {$r$}};%
	\end{tikzpicture}%
\end{equation}%
\end{example}




\begin{definition}\label{SYMGFV DEF}
Let $\mathfrak C$ be a $G$-set and $\F$ a $(G, \Sigma)$-family.
Then the \textit{$\F$-model structure} (if it exists) on the fiber $\Sym^{G}_{\mathfrak C}(\V)$ of $\Sym(\V)$ over $\mathfrak C$
is the $\F_{\mathfrak{C}}$-model structure
\begin{equation}
	\Sym^{G}_{\mathfrak{C},\F}(\V) = \V^{G \ltimes \Sigma_{\mathfrak C}^{op}}_{\F_{\mathfrak{C}}}
\end{equation}
where $\F_{\mathfrak{C}}$ is as in Definition \ref{GSFAM_DEF}.

Moreover, when $\mathcal{F}=\mathcal{F}_{all}$ is the family of all subgroups we refer to this model structure simply as the \emph{genuine model structure} on $\mathsf{Sym}^G_{\mathfrak{C}}$.
\end{definition}



Formally speaking, the generating (trivial) cofibrations in 
$\mathsf{Sym}^G_{\mathfrak{C},\mathcal{F}}$
can be described by simply setting 
$\G = G \ltimes \Sigma^{op}_{\mathfrak{C}}$ in \eqref{VGFGEN EQ}.
However, we will prefer to have a somewhat more conceptual description in terms of colored $G$-trees.

Firstly, adapting the notation 
$\Omega[T] \in \mathsf{Set}^{\Omega^{op}}$
for the representable presheaf $\Omega[T](S) = \Omega(S,T)$,
we similarly write
$\Sigma_{\mathfrak{C}}[\vect{C}] 
\in \mathsf{Sym}_{\mathfrak{C}} = \mathsf{Set}^{\Sigma_{\mathfrak{C}}^{op}}$ for the representable presheaf
$\Sigma_{\mathfrak{C}}[\vect{C}](\vect{D})
= \Sigma_{\mathfrak{C}}(\vect{D},\vect{C})$.

Moreover, similarly to {\color{blue} \cite{BP_edss} (add details)},
we would like to extend the
$\Sigma_{\mathfrak{C}}[-]$ representable presheaf notation
to allow for inputs other than just $\mathfrak{C}$-corollas.
Firstly, if $\vect{T} \in \Omega_{\mathfrak{C}}$ is a $\mathfrak{C}$-tree, we define 
$\Sigma_{\mathfrak{C}}[\vect{T}] \in \mathsf{Sym}_{\mathfrak{C}}$
by
\[
\Sigma_{\mathfrak{C}}[\vect{T}](\vect{D})
=\{ \text{outer face $\mathfrak{C}$-maps } \vect{D} \to \vect{T}\}
\]
while for a $\mathfrak{C}$-colored forest
$\vect{F} \in \Phi_{\mathfrak{C}}$
with tree component decomposition
$\vect{F} = \amalg_i \vect{F}_i$
we set
\[
\Sigma_{\mathfrak{C}}[\vect{F}] =
\amalg^{\mathfrak{C}}_i 
\Sigma_{\mathfrak{C}}[\vect{F}_i]
\]
where we note that the coproduct $\amalg^{\mathfrak{C}}$ is fibered, i.e. it takes place in $\mathsf{Sym}_{\mathfrak{C}}$.

Moreover, note that if $\mathfrak{C}$ is a $G$-set, and we consider a 
$\mathfrak{C}$-colored $G$-forest
$\vect{F} \in \Phi^G_{\mathfrak{C}}$
then one can naturally regard
$\Sigma_{\mathfrak{C}}[\vect{T}] \in \mathsf{Sym}^G_{\mathfrak{C}}
= \mathsf{Set}^{G \ltimes \Sigma^{op}_{\mathfrak{C}}}$.
To see why the semi-direct product is needed, note that the $G$-action on 
$\vect{F}$
consists of a $G$-action on the index set $\{i\}$
together with unital and associative isomorphisms
$\vect{F_i} \xrightarrow{g} \vect{F}_{gi}$ over the action arrow
$\{i\} \xrightarrow{g} \{i\}$.
Since the $\Sigma_{\mathfrak{C}}[\vect{F}_i]$
are presheaves of fixed color maps, this does not induce isomorphisms
$\Sigma_{\mathfrak{C}}[\vect{F}_i] \simeq \Sigma_{\mathfrak{C}}[\vect{F}_{gi}]$
but rather isomorphisms
$\Sigma_{\mathfrak{C}}[\vect{F}_i] \simeq g^{\**}\Sigma_{\mathfrak{C}}[\vect{F}_{gi}]$.
In other words, given an outer face $\mathfrak{C}$-map 
$\vect{D} \to \vect{F}_i$ in the square below (where the top map is the canonical action map)
there is a unique compatible
outer face $\mathfrak{C}$-map 
$g \vect{D} \to \vect{F}_{gi}$  
\begin{equation}
\begin{tikzcd}
	\vect{D}  
	\arrow{d}
	 \arrow{r}{g}
&
	g \vect{D} \arrow[d]
\\
	\vect{F}_i \arrow{r}{g}
&
	\vect{F}_{gi}
\end{tikzcd}
\end{equation}
{\color{blue} this diagram takes place in 
$G\ltimes \Sigma_{\mathfrak{C}}$ rather than in 
$G^{op} \ltimes \Sigma_{\mathfrak{C}}$ - I'm starting to think that this is an intrinsic feature, not poor organization}


\begin{remark}
Alternatively, the representables
$\Sigma_{\mathfrak{C}}[-]$
assemble into a fiber functor
\[
\Sigma_{\bullet}[-] \colon \Phi_{\bullet} \to \mathsf{Sym}_{\bullet}
\]
over $\mathsf{F}$ and thus induces a fiber functor
\[
\Sigma_{\bullet}[-]^G \colon \Phi_{\bullet}^G \to \mathsf{Sym}_{\bullet}^G
\]
over $\mathsf{F}^G$.
\end{remark}


\begin{definition}
Given a $G$-set of colors $\mathfrak{C}$
and a $\mathfrak{C}$-tree $\vect{T}\in \Omega_{\mathfrak{C}}$
we write $G \cdot_{\mathfrak{C}} \vect{T} \in \Phi^G_{\mathfrak{C}}$
for the $G$-forest
\[
G \cdot_{\mathfrak{C}} \vect{T}
= 
\coprod_{g \in G}
g \vect{T}
\]
together with the canonical action maps.
\end{definition}


\begin{remark}
One can more generally define a functor
$G \cdot_{\mathfrak{C}} (-) \colon \Phi_{\mathfrak{C}}
\to \Phi_{\mathfrak{C}}^G$
which is the left adjoint to the forgetful functor
$ \Phi_{\mathfrak{C}}^G
\to \Phi_{\mathfrak{C}}$.
\end{remark}





\begin{example}
Let $G = \{1,i,-1,-i\} \simeq \mathbb{Z}_{/4}$ 
be the group of quartic roots of unit and
$\mathfrak{C} = \{\mathfrak{a}, -\mathfrak{a}, i\mathfrak{a},-i,\mathfrak{a}, \mathfrak{b}, i \mathfrak{b} \}$ where we implicitly have
$-\mathfrak{b} = \mathfrak{b}$.
The following depicts the forest (of corollas) $G \cdot_{\mathfrak{C}} \vect{C}
$
for $\vect{C}$ the leftmost corolla.
\begin{equation}
	\begin{tikzpicture}[auto,grow=up, level distance = 2.2em,
	every node/.style={font=\scriptsize,inner sep = 2pt}]%
		\tikzstyle{level 2}=[sibling distance=3em]%
			\node at (0,0) [font = \normalsize] {$\vect{C}$}%	
				child{node [dummy] {}%
					child{node {}%
					edge from parent node [swap] {$-\mathfrak{a}$}}%
					child[level distance = 2.9em]{node {}%
					edge from parent node [swap,	near end] {$i\mathfrak{b}$}}%
					child[level distance = 2.9em]{node {}%
					edge from parent node [near end] {$i\mathfrak{b}$}}%
					child{node {}%
					edge from parent node  {$\mathfrak{a}$}}%
				edge from parent node [swap] {$\mathfrak{b}$}};%
			\node at (3.5,0) [font = \normalsize] {$i\vect{C}$}%	
				child{node [dummy] {}%
					child{node {}%
					edge from parent node [swap] {$-i\mathfrak{a}$}}%
					child[level distance = 2.9em]{node {}%
					edge from parent node [swap,	near end] {$\mathfrak{b}$}}%
					child[level distance = 2.9em]{node {}%
					edge from parent node [near end] {$\mathfrak{b}$}}%
					child{node {}%
					edge from parent node  {$i\mathfrak{a}$}}%
				edge from parent node [swap] {$i\mathfrak{b}$}};%
			\node at (7,0) [font = \normalsize] {$-\vect{C}$}%	
				child{node [dummy] {}%
					child{node {}%
					edge from parent node [swap] {$\mathfrak{a}$}}%
					child[level distance = 2.9em]{node {}%
					edge from parent node [swap,	near end] {$i\mathfrak{b}$}}%
					child[level distance = 2.9em]{node {}%
					edge from parent node [near end] {$i\mathfrak{b}$}}%
					child{node {}%
					edge from parent node  {$-\mathfrak{a}$}}%
				edge from parent node [swap] {$\mathfrak{b}$}};%
			\node at (10.5,0) [font = \normalsize] {$-i\vect{C}$}%	
				child{node [dummy] {}%
					child{node {}%
					edge from parent node [swap] {$i\mathfrak{a}$}}%
					child[level distance = 2.9em]{node {}%
					edge from parent node [swap,	near end] {$\mathfrak{b}$}}%
					child[level distance = 2.9em]{node {}%
					edge from parent node [near end] {$\mathfrak{b}$}}%
					child{node {}%
					edge from parent node  {$-i\mathfrak{a}$}}%
				edge from parent node [swap] {$i\mathfrak{b}$}};%
	\end{tikzpicture}%
\end{equation}%
Note that the pairs $\vect{C},-\vect{C}$
and  $i\vect{C},-i\vect{C}$ are isomorphic in $\Sigma_{\mathfrak{C}}$
while any other pair such as, say, $\vect{C},i\vect{C}$ is not.
In general, it is moreover possible for two or more tree components of
$G \cdot_{\mathfrak{C}} \vect{C}$ to be equal.
\end{example}





\begin{remark}\label{REPALTDESC REM}
Fix a $G$-color set $\mathfrak{C}$
and $X \in \mathsf{Sym}^G_{\mathfrak{C}} = \mathsf{Set}^{G \ltimes \Sigma_{\mathfrak{C}}^{op}}$.
Unpacking definitions,
a map $\Sigma_{\mathfrak{C}} [G \cdot_{\mathfrak{C}} \vect{C}] \to X$
consists of an underlying map
$\coprod_{g \in G} \Sigma_{\mathfrak{C}}[g \vect{C}] \to X$
in $\mathsf{Sym}_{\mathfrak{C}}$ which also respects the $G$-actions.
But by Yoneda this is equivalent to a choice of points
$x_g \in X(g \vect{C})$ such that, writing
$X(g \vect{C}) \xrightarrow{\bar{g}} X(\bar{g} g \vect{C})$ for the $G$-action structure maps of $X$,
it is $\bar{g}(x_g) = x_{\bar{g}g}$.

But we have now shown that a map 
$\Sigma_{\mathfrak{C}} [G \cdot_{\mathfrak{C}} \vect{C}] \to X$
is freely determined by a choice of a point $x_e \in X(\vect{C})$,
showing that
$\Sigma_{\mathfrak{C}} [G \cdot_{\mathfrak{C}} \vect{C}]$
is in fact the representable functor
$(G \ltimes \Sigma^{op}_{\mathfrak{C}})(\vect{C},-)$.
\end{remark}




\begin{remark}\label{VGSIGF REM}
Combining \eqref{VGFGEN EQ} with the 
$\Sigma_{\mathfrak{C}}[G \cdot_{\mathfrak{C}} \vect{C}]$ notation
for the representable functors for the groupoid
$\G = G \ltimes \Sigma^{op}_{\mathfrak{C}}$
one has that the generating (trivial) cofibratons of
$\V^{G \ltimes \Sigma_{\mathfrak{C}}^{op}}_{\F_{\mathfrak{C}}}$
are given by the sets of maps
\begin{equation}\label{VGSIGF EQ}
	\left\{
	\Sigma_{\mathfrak{C}}[G \cdot_{\mathfrak{C}} \vect{C}]/\Lambda \cdot i
	\right\}
\qquad \qquad
	\left\{
	\Sigma_{\mathfrak{C}}[G \cdot_{\mathfrak{C}} \vect{C}]/\Lambda \cdot j
	\right\}
\end{equation}
where $\vect{C}$ ranges over $\Sigma_{\mathfrak{C}}$,
$\Lambda$ ranges over $\F_{\vect{C}}$,
$i$ ranges over $\mathcal{I}$ and
$j$ ranges over $\mathcal{J}$.
\end{remark}





\begin{remark}
It follows immediately from the definition of $\mathsf{Sym}^G_{\mathfrak{C}}$ (cf. Definition \ref{GSFAM_DEF}) and 
Proposition \ref{EQQUILADJ PROP} that, for any 
$(G,\Sigma)$-family $\F$
and map of colors $f \colon \mathfrak{C} \to \mathfrak{D}$
the induced adjunction
\[
	f_! \colon \mathsf{Sym}^G_{\mathfrak{C},\F}
	\rightleftarrows
	\mathsf{Sym}^G_{\mathfrak{D},\F} \colon f^{\**}
\]
is a Quillen adjunction.
\end{remark}




 
\subsection{Homotopy theory of operads with a fixed color $G$-set}
\label{OPC_MS_SEC}

Our goal in this section is to show that the model structures 
$\mathsf{Sym}^G_{\mathfrak{C},\F}$
from Definition \ref{SYMGFV DEF} can be transferred 
along the free-forgetful adjunction
(where $\mathbb{F}^G_{\mathfrak{C}}$
denotes the fiber monad of $\mathbb{F}^G$ on the $\mathfrak{C}$ fiber, cf. Definition \ref{COLORMON DEF}, \eqref{FGC_EQ})
\begin{equation}\label{OPAUTADJ EQ}
\mathbb{F}^G_{\mathfrak{C}} \colon
\mathsf{Sym}^G_{\mathfrak{C}}
\rightleftarrows
\mathsf{Op}^G_{\mathfrak{C}}
\colon \mathsf{fgt}
\end{equation}
to obtain model structures
$\mathsf{Op}^G_{\mathfrak{C},\F}$
where a map is a weak equivalence/fibration iff the underlying map in 
$\mathsf{Sym}^G_{\mathfrak{C}}$ is.

The key to proving this result will be given by suitable filtrations of free operad extensions in $\mathsf{Op}_{\mathfrak{C}}^G(\V)$
which are built in Appendix \ref{MONAD_APDX} \S \ref{PUSHOUT_SEC}, most notably Proposition \ref{FILTPUSHG PROP}.
The following lemma summarizes the result we will need.
We write
$\mathsf{lr}_k \colon G^{op} \ltimes \Omega^a_{\mathfrak{C}}[k] 
\to
G^{op} \ltimes \Sigma_{\mathfrak{C}}$
for the natural leaf root functor.



\begin{lemma}
Fix a $G$-set of colors $\mathfrak{C}$ and let
$u\colon X \to Y$ be a map in $\mathsf{Sym}^G_{\mathfrak{C}}$
and
$\mathbb{F} X \to \O$ be a map in $\mathsf{Op}^G_{\mathfrak{C}}$.
Then, for the pushout 
\begin{equation}\label{OURE EQ}
\begin{tikzcd}
	\mathbb F X \arrow[d, "\mathbb{F}u"'] \arrow[r]
&
	\O \arrow[d]
\\
	\mathbb F Y \arrow[r]
&
\O[u]
\end{tikzcd}
\end{equation}
in the category of operads $\mathsf{Op}^G_{\mathfrak{C}}$
the map $\O \to \O[u]$ admits an underlying filtration
\begin{equation}\label{OUFILRE EQ}
\O = \O_0 \to \O_1 \to \O_2 \to \cdots \to \O_{\infty} = \O[u]
\end{equation}
of maps in $\mathsf{Sym}^G_{\mathfrak{C}}$ 
where each map $\O_{k-1} \to \O_k$ fits into a pushout
\begin{equation}\label{OUPUSHRE EQ}
\begin{tikzcd}
	\bullet 
	\arrow{d}[swap]{\left(\mathsf{lr}_k^{op}\right)_!
	n_k^{(\O,X,Y)}}
	 \arrow[r]
&
	\O_{k-1} \arrow[d]
\\
	\bullet \arrow[r]
&
	\O_k
\end{tikzcd}
\end{equation}
where 
$
\left(\mathsf{lr}_k^{op}\right)_! \colon
\mathcal{V}^{G \ltimes \Omega^a_{\mathfrak{C}}[k]^{op}}
\to
\mathcal{V}^{G \ltimes \Sigma_{\mathfrak{C}}^{op}}
=
\mathsf{Sym}^G_{\mathfrak{C}}
$
is as in Proposition \ref{EQQUILADJ PROP}
and $n_k^{(\O,X,Y)}$ is the natural transformation in 
$\mathcal{V}^{G \ltimes \Omega^a_{\mathfrak{C}}[k]^{op}}$
whose constituent arrows for $\vect{U} \in \Omega^a[k]$ are
\begin{equation}\label{NKOXY EQ}
n_k^{(\O,X,Y)}(\vect{U})=
	\left(
		\bigotimes_{v \in V^{ac}(\vect{U})}\O(\vect{U}_v)
	\otimes
		\mathop{\mathlarger{\mathlarger{\mathlarger{\square}}}}\limits_{v \in V^{in}(\vect{U})} u(\vect{U}_v)
		\right)
	=
	\left(
		\mathop{\mathlarger{\mathlarger{\mathlarger{\square}}}}\limits_{v \in V^{ac}(\vect{U})} \left( \emptyset \to \O(\vect{U}_v)\right) 
	\square
		\mathop{\mathlarger{\mathlarger{\mathlarger{\square}}}}\limits_{v \in V^{in}(\vect{U})} u(\vect{U}_v)
		\right)
\end{equation}
\end{lemma}

We can now prove our first main theorem.

\begin{proof}[Proof of Theorem \ref{THM1_C}]
Let us first address the main case where $\V$
is either $\mathsf{sSet}$ or $\mathsf{sSet}_{\**}$, 
so that all objects in $\V^{\G}$ are genuine cofibrant for any groupoid $\G$.

By \cite[Thm. 11.3.2]{Hir03} we need only show that
the pushouts maps $\O \to \O[u]$ in \eqref{OURE EQ}
are $\F$-weak equivalences in $\mathsf{Sym}^G_{\mathfrak{C}}$
whenever 
$u \colon X \to Y$
is one of the generating trivial $\F$-cofibrations in 
$\mathsf{Sym}^G_{\mathfrak{C}}$.
Moreover, since trivial $\F$-cofibrations are genuine cofibrations and 
genuine weak equivalences are $\F$-weak equivalences, one needs only consider the genuine case, i.e. the case of $\F_{all}$ the family of all subgroups.

By the filtration \eqref{OUFILRE EQ}, it thus suffices to check that each map $\O_{k-1} \to \O_k$
is a genuine trivial cofibration in $\mathsf{Sym}^G_{\mathfrak{C}}$
so that, from the pushouts \eqref{OUPUSHRE EQ}
and Proposition \ref{EQQUILADJ PROP}
it is enough to show that the map
$n_k^{(\O,X,Y)}$ in \eqref{NKOXY EQ}
is a genuine cofibration in $\V^{G \ltimes \Sigma_{\mathfrak{C}}^{op}}$.
But now notice that the maps
\[
\mathop{\mathlarger{\mathlarger{\mathlarger{\square}}}}\limits_{v \in V^{ac}(\vect{U})} \left( \emptyset \to \O(\vect{U}_v)\right) 
\qquad \qquad
\mathop{\mathlarger{\mathlarger{\mathlarger{\square}}}}\limits_{v \in V^{in}(\vect{U})} u(\vect{U}_v)
\]
in \eqref{NKOXY EQ} are precisely 
$(\emptyset \to \O)^{\square m_a}$ and 
$u^{\square m_i}$ 
as defined by \eqref{FSQNDEF EQ} for
$m_a = |V^{ac}(\vect U)|$,
$m_i = |V^{in}(\vect U)|$
\todo[inline]{these are \textit{evaluations} of those maps on particular things... aren't they?}
so that, by Proposition \ref{SIGMAWRGF PROP},
$(\emptyset \to \O)^{\square m_a}$ is a 
genuine cofibration in
$\V^{\Sigma_{m_a} \wr (G \ltimes \Sigma_{\mathfrak{C}}^{op})}$ and
$u^{\square m_i}$ a
genuine trivial cofibration in 
$\V^{\Sigma_{m_i} \wr (G \ltimes \Sigma_{\mathfrak{C}}^{op})}$.
Propositions \ref{EQQUILADJ PROP} and \ref{RESGEN PROP}
now yields that \eqref{NKOXY EQ} is indeed a genuine trivial cofibration in $\V^{G \ltimes \Sigma_{\mathfrak{C}}^{op}}$
(note that $k \geq 1$)
\todo[inline]{which $k$ is this?}
finishing the proof of the 
case where $\V$ is $\mathsf{sSet}$ or $\mathsf{sSet}_{\**}$.



For the semi-model structure claim, 
we replace the use of \cite[Thm. 11.3.2]{Hir03} with that of 
\cite[Thm. 2.2.2]{WY18}
which now requires us to show that 
$\O \to \O[u]$
is a (trivial) genuine cofibration in $\mathsf{Sym}^G_{\mathfrak{C}}$
whenever $u$ is a (trivial) genuine cofibration in $\mathsf{Sym}^G_{\mathfrak{C}}$
and $\O$ is cofibrant in $\mathsf{Op}^G_{\mathfrak{C}}$.
But now note that the argument given above proves precisely the analogue claim with 
``$\O$ is cofibrant in $\mathsf{Op}^G_{\mathfrak{C}}$''
replaced with 
``$\O$ is cofibrant in $\mathsf{Sym}^G_{\mathfrak{C}}$'',
so that we are reduced to showing that if 
$\O$ is cofibrant in $\mathsf{Op}^G_{\mathfrak{C}}$
then it is also cofibrant in $\mathsf{Sym}^G_{\mathfrak{C}}$.
This now follows by induction on the cell complex decomposition of $\O$,
with the base case stating that the initial operad $\mathbb{F}\emptyset$ is cofibrant in $\mathsf{Sym}^G_{\mathfrak{C}}$
following from the assumption that $\mathcal{V}$ has a cofibrant unit
and the induction step following from the already established cofibrancy of the maps $\O \to \O[u]$ in $\mathsf{Sym}^G_{\mathfrak{C}}$.
\end{proof}



\begin{remark}
Following Remark \ref{VGSIGF REM}
the generating (trivial) cofibrations in
$\mathsf{Sym}^G_{\mathfrak{C},\F}(\V)$
are given by applying $\mathbb{F}^G_{\mathfrak{C}}$
to the maps in \eqref{VGSIGF EQ}, i.e. they are given by the sets of maps
\begin{equation}\label{FVGSIGF EQ}
	\left\{
	\mathbb{F}^G_{\mathfrak{C}}
	\left(\Sigma_{\mathfrak{C}}[G \cdot_{\mathfrak{C}} \vect{C}]/\Lambda \cdot i \right)
	\right\}
\qquad \qquad
	\left\{
	\mathbb{F}^G_{\mathfrak{C}}
	\left(\Sigma_{\mathfrak{C}}[G \cdot_{\mathfrak{C}} \vect{C}]/\Lambda \cdot j \right)
	\right\}
\end{equation}
where $\vect{C}$ ranges over $\Sigma_{\mathfrak{C}}$,
$\Lambda$ ranges over $\F_{\vect{C}}$,
$i$ ranges over $\mathcal{I}$ and
$j$ ranges over $\mathcal{J}$.
\end{remark}





{\color{red} HERE}







We record a particular step in the proof of Theorem \ref{THM1_C}.
\begin{corollary}
      \label{LGC_COR}
      Suppose $\V$ has the hypotheses of Theorem \ref{THM1_C},
      let $\mathfrak C$ be a $G$-set, and $f: \O \to \P$ a map in $\Op^{G,\mathfrak C}(\V)$.
      If $f$ is a (trivial) cofibration in $\Op^{G, \mathfrak C}_\F(\V)$ for some $(G, \Sigma)$-family $\F$,
      and $\O$ is level genuine cofibrant, then
      $f(\vect C)$ is a genuine (trivial) cofibration in $\V^{\Aut(C)}_{gen}$ for all $\mathfrak C$-signatures $\vect C \in G \ltimes \Sigma_{\mathfrak C}$.
\end{corollary}



\begin{remark}
In both Theorem \ref{THM1_C} and Corollary \ref{LGC_COR}, as well as in \cite{BP_geo},
the cofibrancy of the unit is necessary so that the initial operad $\mathbb F(\varnothing)$ is level genuine cofibrant.
\end{remark}




\begin{remark}
      \label{CATV_MC_REM}
      As a corollary, we get (semi)-model structures on $\Cat^{G}_{ \mathfrak C}(\V) = \Op^{G}_{\mathfrak C}(\V) \downarrow \**$,
      so in particular we have cofibrant replacements in $\Cat_{\mathfrak C}(\V)$ for any set $\mathfrak C$.
\end{remark}


\begin{remark}
      \label{TOP_FULL_REM}
      The category $\Op^{G,\mathfrak C}(\Top)$ actually has full model structures lifted from $\Sym^{G, \mathfrak C}_\F(\Top)$
      for any $(G, \Sigma)$-family $\F$,
      using an argument analogous to \cite[Thm. 3.1]{GW18}.
\end{remark}




\begin{corollary}\label{COLOR_CHANGE_Q_COR}
      The change of color adjunctions from \eqref{GC_CHANGE_EQ},
      and the forgetful functor $j^{\**}$ from \eqref{JSTAR_CAT_EQ}
      form Quillen adjunctions.
\end{corollary}













\subsection{Remainders (equivariant operads explicitly)}

\begin{definition}
      A \textit{$\mathfrak C$-colored operad} in $\V$ 
      is a $\mathfrak C$-symmetric sequence $\O$ along with appropriate composition laws.
      Explicitly, this consists of the following data:
      \begin{itemize} %{enumerate}[label = (\arabic*)]
      \item An object $\O(\ksi) \in \V$ for each signature $\ksi$.
      \item For all signatures $\ksi \in \mathfrak C^{\times n+1}$ and $\sigma \in \Sigma_n$,
            unital and associative times maps $\O(\xi) \to \O(\sigma \cdot \xi)$,
            where $\Sigma_n$ acts on the first $n$ coordinates of $\mathfrak C^{\times n+1}$;
            i.e., maps
            \begin{equation}
                  \O(c_1, \ldots, c_n; c_0) \xrightarrow{\sigma} \O(c_{\sigma^{-1}(1)}, \ldots, c_{\sigma^{-1}(n)}; c_0).
            \end{equation}
      \item For each $c \in \mathfrak C$, a \textit{unit} $1_c \in \O(c;c)$.                        
      \item For all compatible signatures $\ksi$, $\ksi_1$, $\dots$, $\ksi_n$,
            \textit{composition} maps
            \begin{equation}
                  \O(\xi) \otimes \O(\xi_1) \otimes \ldots \otimes \O(\xi_n) \to \O(\xi \circ (\xi_i)),
            \end{equation}
            which are unital, associative, and appropriately $\Sigma$-equivariant.
      \end{itemize}
\end{definition}




\begin{example}
      Now, for $\Op(\V)$,
      we see that an object $\O \in \Op^G(\V)$ consists of the following data:
      \begin{enumerate}[label = (\arabic*), start = 0]
      \item A $G$-set $\mathfrak C = \mathfrak C_{\O}$ of colors.
      \item For each $\mathfrak C$-signature $\ksi$, an object $\O(\ksi) \in \V$.
      \item \label{SACTION_LBL}
            For all signatures $\ksi \in \mathfrak C^{\times n + 1}$ and $\sigma \in \Sigma_n$,
            maps $\O(\ksi) \to \O(\sigma \cdot \ksi)$
            which are unital and associative.
      \item \label{GUNIT_LBL} 
            For each $c \in \mathfrak{C}$, a \textit{unit} $1_c \in \O(c;c)$.
      \item \label{COMP_LBL}
            For all compatible signatures $\ksi, \ksi_1,\dots, \ksi_n$,
            \textit{composition maps} $\O(\ksi) \otimes \O(\ksi_1) \otimes \dots \otimes \O(\ksi_n) \to \O(\ksi \circ (\ksi_i))$
            which are unital, associative, and appropriately $\Sigma$-equivariant
      \item \label{GACTION_LBL}
            For all $g \in G$, maps $\O(\ksi) \to g^{\**}\O(\ksi) = \O(g \cdot \ksi)$
            (with $G$ acting on $\mathfrak C^{\times n+1}$ diagonally)
            which are unital and associative, and which commute with the composition maps
            (note that if $\ksi, (\ksi_i)$ are compatible, then so are $g \ksi, (g \ksi_i)$).
      \end{enumerate}
      Synthesizing, we may combine \ref{SACTION_LBL} and \ref{GACTION_LBL} into
      \begin{enumerate}
      \item[($2'$)] For all signatures $\xi \in \mathfrak C^{\times n+1}$ and $(g,\sigma) \in G\times \Sigma_n$, maps
            $\O(\xi) \to \O((g,\sigma)\cdot \xi)$
            which are unital and associative.
      \end{enumerate}
      
      and replace \ref{GUNIT_LBL} and \ref{COMP_LBL} with
      \begin{enumerate}
      \item[($3'$)] For each $c \in \mathfrak C$, a $G_c$-fixed unit $1_c \in \O(c;c)^{G_c}$.
      \item[($4'$)] For all compatible signatures $\ksi, (\ksi_i)$,
            composition maps $\O(\ksi) \otimes \bigotimes_i \O(\ksi_i) \to \O(\ksi \circ (\ksi_i))$
            which are unital, assocative, $G$-equivariant, and approriately $\Sigma$-equivariant.
      \end{enumerate}
\end{example}











\newpage


\section{Operadic model structures on $\mathsf{Op}^G(\mathcal{V})$}\label{MS_SEC}
\renewcommand{\C}{\mathfrak C}



The previous section built model structures on each category
$\mathsf{Op}_{\mathfrak{C}}^G(\V)$
of $G$-equivariant operads with a fixed $G$-set of colors $\mathfrak{C}$.
Adapting \cite{BM13,Cav,CM13b}, 
our goal in this section is to build a model structure on the full category $\mathsf{Op}^G(\mathcal{V})$
of $G$-equivariant operads with varying colors. 

The model structures on $\mathsf{Op}^G(\V)$ are characterized by their weak equivalences and fibrations
({\color{red} see wherever}), 
with both classes of maps described by combining a ``local condition'', i.e. a condition involving the fixed color categories $\mathsf{Op}_{\mathfrak{C}}^G(\V)$,
with a ``homotopical equivalence condition'',
i.e. a condition involving equivalences between colors within
some fixed $\O \in \mathsf{Op}^G(\V)$.



\subsection{Types of maps in $\mathsf{Op}^G(\V)$}

We now discuss the several types of maps in 
$\mathsf{Op}^G(\V)$ we will be interested in.

We start with the ``local notions'', 
i.e. those notions determined by the fixed color categories $\mathsf{Op}_{\mathfrak{C}}^G(\V)$.


\begin{definition}
Write $G \times \Sigma^{op} \xrightarrow{\pi} 
G \simeq G \times \Sigma_1^{op} \xrightarrow{\iota} G \times \Sigma^{op}$
for the natural projection and inclusion.

We say that a family $\F$ of subgroups of $G\times \Sigma^{op}$
\emph{respects units} if $\F \subseteq \pi^{\**} \iota^{\**} \F$.
\end{definition}


\begin{definition}
Let $\F$ be a $(G, \Sigma)$-family which respects units.

We say a map $F: \O \to \P$ in $\mathsf{Op}^G(\V)$
is a \emph{local $\F$-weak equivalence (resp. local $\F$-fibration, local $\F$-trivial fibration)}
if the induced fixed color map
\[\O \to F^{\**} \P\]
is a $\F_{\mathfrak{C}_{\O}}$-weak equivalence (resp. $\F_{\mathfrak{C}_{\O}}$-fibration, $\F_{\mathfrak{C}_{\O}}$-trivial fibration) in the fiber $\mathsf{Op}^G_{\mathfrak{C}_{\O}}(\V)$.
\end{definition}

Local $\F$-(trivial) fibrations admit the following alternative characterization.

\begin{proposition}\label{LOCALTCHAR PROP}
The local $\F$-fibrations
and trivial $\F$-fibrations)
in $\mathsf{Op}^G(\V)$
are characterized as the maps with the right lifting property against the generating sets of maps 
$\mathcal{J}_{\mathfrak{C}}$ and $\mathcal{I}_{\mathfrak{C}}$
(cf. \eqref{FVGSIGF EQ})
of the fibers 
$\mathsf{Op}^G_{\mathfrak{C}}(\V) \hookrightarrow \mathsf{Op}^G(\V)$
for all $\mathfrak{C} \in \mathsf{F}^G$.
\end{proposition}

We note that this result is formulated to be compatible with semi-model structures.
If the fibers $\mathsf{Op}^G_{\mathfrak{C}}(\V)$ have full model structures one can replace the role of
``$\mathcal{J}_{\mathfrak{C}}$'' and ``$\mathcal{I}_{\mathfrak{C}}$''
with simply ``trivial cofibrations'' and ``cofibrations''.


\begin{proof}
This follows by Remark \ref{COLOR_SQ_REM}(i)
and the fact that the pullback functors
$f^{\**} \colon \mathsf{Op}^G_{\mathfrak{D},\F}(\V)
\to \mathsf{Op}^G_{\mathfrak{C},F}(\V)$
preserve (trivial) fibrations.
\end{proof}




We next turn to the homotopical notions of essential surjectivity and isofibration, which concern equivalences between objects within some 
$\O \in \mathsf{Op}^G(\V)$.
We first recall some definitions from \cite{BM13}.
As usual, we let $1_\V$ and $\emptyset$ denote, respectively, the unit object and initial object of $\V$.

\begin{notation}
We write $\mathbbm{1}$ (resp. $\widetilde{\mathbbm{1}}$)
for the $\V$-category which represents arrows (resp. isomorphisms):
it has two objects $0,1$
and mapping objects
$\mathbbm{1}(i,j)= 1_{\V}$
if $i \leq j$
and 
$\mathbbm{1}(1,0)= \emptyset$
(resp. $\widetilde{\mathbbm{1}}(i,j)= 1_{\V}$ for all $i,j$)
and composition defined by the natural isomorphisms.
\end{notation}


\begin{definition}
	A {\em $\V$-interval} is a cofibrant object $\I$ in $\Cat_{\set{0,1}}(\V)$
	which is equivalent to $\widetilde{\mathbbm{1}}$.
\end{definition}



\begin{remark}
	We note that, since $\widetilde{\mathbbm{1}}$ is typically not fibrant,
	an arbitrary interval $\mathbb{I}$
	needs not admit a map to $\widetilde{\mathbbm{1}}$,
	but only a map $\mathbb{I} \to \widetilde{\mathbbm{1}}_f$,
	where $\widetilde{\mathbbm{1}}_f$ denotes some fixed chosen fibrant replacement.
\end{remark}




\begin{definition}\label{PL_ES_DEFN}
We say a functor $F: \mathcal C \to \mathcal D$ in $\Cat(\V)$ is
\begin{itemize}
\item \textit{path-lifting}
	if it has the right lifting property against all maps of the form
	$\{0\} \to \I$, $\{1\} \to \I$
	where $\I$ is a $\V$-interval; 
\item \textit{essentially surjective} 
	if for any object $d \in \mathcal{D}$
	there is an object $c \in \mathcal{C}$,
	$\V$-interval $\mathbb{I}$,
	and map $i \colon \mathbb{I} \to \D$
	such that $i(0) = F(c)$ and $i(1)=d$.
%	if for any object $d: \1 \to \mathcal D$,
%	there is an object $c: \1 \to \mathcal C$
%	and a map $\J \to \mathcal D$ out of a $\V$-interval fitting in to the commuting diagram below.
%            \begin{equation}
%                  \label{ESSURJ_EQ}
%                  \begin{tikzcd}
%                        \1 \arrow[rr, dashed, "c"] \arrow[dr, "i_0"]
%                        &&
%                        \mathcal C \arrow[dd, "F"]
%                        \\
%                        &
%                        \J \arrow[dr, dashed]
%                        \\
%                        \1 \arrow[ur, " i_1"] \arrow[rr,"d"]
%                        &&
%                        \mathcal D
%                  \end{tikzcd}
%            \end{equation}
      \end{itemize}
\end{definition}


We now adapt the previous definition for $G$-operads.
Recall that $j^{\**} \colon \mathsf{Op}^G(\V) \to \mathsf{Cat}^G(\V)$
denotes the functor that forgets all non-unary operations and that, moreover, 
$j^{\**}$ comutes with all fixed points $(-)^H$.


\begin{definition}\label{FESSENSURJ DEF}
Let $\F$ be a $(G, \Sigma)$-family which respects units.

We say a map $F: \O \to \P$ in $\mathsf{Op}^G(\V)$
is $\F$-essentially surjective (resp. $\F$-path-lifting)
if the maps
$j^{\**}\O^H \to j^{\**} \P^H$
in $\mathsf{Cat}(\V)$ are essentially surjective (path-lifting) for all $H \in \F_1$.
\end{definition}



We can finally define the classes of maps in the desired model structures in $\mathsf{Op}^G(\V)$.



\begin{definition}\label{MODEL_DEFN}
Fix a $(G, \Sigma)$-family $\F$ which respects units.

We say a map $F: \O \to \P$ in $\mathsf{Op}^G(\V)$ is:
\begin{itemize}
	\item a {\em $\F$-fibration} if it is both a local $\F$-fibration and $\F$-path lifting;
	\item a {\em $\F$-weak equivalence} if it is both a local $\F$-weak equivalence and $\F$-essentially surjective;
	\item a \textit{$\F$-cofibration} if it has the left lifting property against all trivial $\F$-fibrations (i.e. $\F$-fibrations which are also $\F$-weak equivalences).
\end{itemize}
\end{definition}


Our goal in the remainder of this section will be to prove that, 
under suitable assumptions, 
Definition \ref{MODEL_DEFN} describes a (semi-)model structure on 
$\mathsf{Op}^G(\V)$, which we denote by
$\mathsf{Op}^G_{\F}(\V)$.

We first provide a simpler description of $\F$-trivial fibrations,
adapting \cite[4.8]{Cav}, \cite[2.3]{BM13}, \cite[1.18]{CM13b}.

\begin{proposition}\label{FTRIVCHAR PROP}
A map in $F: \O \to \P$ in $\mathsf{Op}^G(\V)$ is 
a $\F$-trivial fibration (i.e. both a $\F$-fibrations and a $\F$-weak equivalence) 
iff it is a local $\F$-trivial fibration which is surjective on 
$H$-fixed colors for all $H \in \F_1$.
\end{proposition}


\begin{proof}
If is enough to show that,
if $F \colon \O \to \P$ is a local $\F$-trivial fibration,
then $F$ is both $\F$-path lifting and $\F$-essentially surjective
iff it is is surjective on $H$-fixed colors for $H \in \F_1$.

For the ``if'' direction, it is immediate that if $F$ is
$\F$-essentially surjective, so it remains to show that
the maps $\O^H\to \P^J$
have the right lifting property against the maps
$\eta \to \mathbb{I}$.
But this follows by factoring the latter maps as
$\eta \to \eta \amalg \eta \to \mathbb{I}$, 
since the lifting property against 
$\eta \to \eta \amalg \eta$ follows from surjectivity on $H$-fixed objects while the lifting property against
$\eta \amalg \eta \to \mathbb{I}$
follows from Proposition \ref{LOCALTCHAR PROP}.

For the ``only if direction'', let $y \in \P^H$ be a $H$-fixed object with $H \in \F_1$.
$\F$-essential surjectivity yields $x \in \O^H$ and map 
$i \colon \mathbb{I} \to \P^H$
with $i(0)=F(x)$, $i(1)=y$.
The $\F$-path lifting property then gives a lift $\tilde{i}$ as below, 
so that $\tilde{i}(1)$ gives the desired lift of $y$.
\[
\begin{tikzcd}
	\eta \ar{d}[swap]{0} \ar{r}{c}  
&
	\O^H \arrow{d}{F}
\\
	\mathbb{I} \ar{r}[swap]{i} \ar{ru}[swap]{\tilde{i}}
&
	\P^H
\end{tikzcd}
\]   
\end{proof}




\begin{remark}\label{FIBERGLMOD REM}
Definition \ref{MODEL_DEFN} implies that a fixed color map
$\O \to \P$ in 
$\mathsf{Op}^G_{\mathfrak{C},\F}(\V)$ is:
\begin{enumerate}[label=(\roman*)]
\item a weak equivalence in the fiber
$\mathsf{Op}_{\mathfrak{C},\F}^G(\V)$
iff it is a weak equivalence in
$\mathsf{Op}_{\F}^G(\V)$;
\item a cofibration in the fiber 
$\mathsf{Op}_{\mathfrak{C},\F}^G(\V)$
iff it is a cofibration in
$\mathsf{Op}_{\F}^G(\V)$;
\item a fibration in the fiber 
$\mathsf{Op}_{\mathfrak{C},\F}^G(\V)$
whenever it is a fibration in
$\mathsf{Op}_{\F}^G(\V)$.
\end{enumerate}
Indeed, (i) follows since fixed color maps are certainly essentially surjective while (iii) is tautological since 
fibrations in $\mathsf{Op}_{\F}^G(\V)$ must be local fibrations.
As for (ii), 
Proposition \ref{LOCALTCHAR PROP} yields the ``if'' direction
while the ``only if'' direction follows from 
Proposition \ref{FTRIVCHAR PROP}, 
which implies all trivial fibrations in $\mathsf{Op}_{\mathfrak{C},\F}^G(\V)$
are trivial fibrations in 
$\mathsf{Op}_{\F}^G(\V)$.
\end{remark}




\begin{remark}
In contrast to the other parts therein,
the implication in Remark \ref{FIBERGLMOD REM}(iii) 
only holds in one direction. 
As an simple example where its converse fails, 
note that the map $\eta \amalg \eta \to \tilde{\mathbbm{1}}$
in $\mathsf{Cat}(\mathsf{sSet})$ is a local fibration, 
and thus a fibration in $\mathsf{Cat}_{\{0,1\}}(\mathsf{sSet})$,
but not path-lifting, and thus not a fibration in $\mathsf{Cat}(\mathsf{sSet})$.

Lastly, note that Proposition \ref{FTRIVCHAR PROP}
guarantees that the analogue of Remark \ref{FIBERGLMOD REM}
for $\F$-trivial fibrations is indeed an iff.
\end{remark}




\subsection{Generating cofibrations and trivial cofibrations}

Our next step in proving that the classes of maps in Definition \ref{MODEL_DEFN}
define a model structure will be to identify sets of generating cofibrations and generating trivial cofibrations.

Proposition \ref{LOCALTCHAR PROP} suggests that one might want to include the generating sets of maps for the fibers 
$\mathsf{Op}^G_{\mathfrak{C},\F}(\V)$
in the generating sets of maps for 
$\mathsf{Op}^G_{\F}(\V)$.
However, it is inefficient to include all such maps.
To see why, note that comparing
\eqref{FAMC_DEF_EQ} and \eqref{FVGSIGF EQ},
a subgroup
$\Lambda \leq G \times \Sigma_n^{op}$ in $\F_n$
gives rise to a generating map in
$\mathsf{Op}^G_{\mathfrak{C},\F}(\V)$
for each $n$-ary $\mathfrak{C}$-corolla such that
$\Lambda \leq \mathsf{Aut}_{G\ltimes \Sigma_{\mathfrak{C}}}(\vect{C})
\leq G \times \Sigma_n^{op}$.
However, it turns out that, among all such $\vect{C}$ there is one which is initial, in the sense that it generates all other such 
$\vect{C}$ under change of colors.


We first discuss some preliminaries.
Firstly, there is a ``tautological coloring'' functor
$(-)^{\tau} \colon \Phi \to \Phi_{\bullet}$
such that for a forest 
$F \in \Phi$
the colored forest $F^{\tau}$
has set of colors $\boldsymbol{E}(F)$ of edges of $F$,
with coloring being the identity (i.e. sending each edge to itself).
Moreover, $(-)^{\tau}$ clearly restricts to functors
$(-)^{\tau} \colon \Omega \to \Omega_{\bullet}$,
$(-)^{\tau} \colon \Sigma \to \Sigma_{\bullet}$.



We will adapt the $\Sigma_{\mathfrak{C}}[-]$ notation
(discussed after Definition \ref{SYMGFV DEF})
by, for each forest $F \in \Phi$, abbreviating
$\Sigma_{\tau}[F] = \Sigma_{\boldsymbol{E}(T)}[F^{\tau}] \in \mathsf{Sym}_{\bullet}$.





\begin{remark}
	If $\vect{F} \in \Phi_{\mathfrak{C}}$ is a $\mathfrak{C}$-colored forest and we write $F \in \Phi$ for the underlying forest
	then one has a canonical map
	$F^{\tau} \to \vect{F}$ in $\Phi_{\bullet}$
	which is the underlying identity on forests. In fact, this map is simply the change of color map (i.e. the pushforward in $\Phi_{\bullet}$) along the color map
	$\boldsymbol{E}(T) \xrightarrow{\mathfrak{c}} \mathfrak{C}$
	given by the coloring of $\vect {F}$.
	
	Moreover, writing 
	$\mathfrak{c}_! \colon 
	\mathsf{Sym}_{\boldsymbol{E}(T)} \to
	\mathsf{Sym}_{\mathfrak{C}}$
	one then has the identification
\begin{equation}\label{CANPUSH EQ}
	\mathfrak{c}_! \left(\Sigma_{\tau}[F]\right) = \Sigma_{\mathfrak{C}}[\vect{F}]
\end{equation}	
	
\end{remark}




\begin{remark}
	Suppose now that $\vect{C} \in \Sigma_{\mathfrak{C}}$
	is a $\mathfrak{C}$-corolla with underlying corolla
	$C \in \Sigma$.
	Then the underlying forest of 
	$G \cdot_{\mathfrak{C}} \vect{C}$
	is the free $G$-forest $G \cdot C$.
	
	Moreover, $(G\cdot C)^{\tau} = G \cdot C^{\tau}$,
	where the second formula refers to the free functor
	$G \cdot (-) \colon \Phi_{\bullet} \to \Phi^G_{\bullet}$.
	
	Next, note that the automorphism group of 
	$\Sigma_{\tau}[G\cdot C]$ in $\mathsf{Sym}_{\bullet}^G$
	is then naturally identified with 
	$G^{op} \times \Sigma_n$ (where we assume $C$ is a $n$-corolla),
	where we stress that we are including those automorphisms which change colors.
	
	The claim that 
	$\Lambda \leq \mathsf{Aut}_{G \ltimes \Sigma_{\mathfrak{C}}^{op}}(\vect{C}) 
	\leq G \times \Sigma_{n}^{op}$
	is then the claim that, for each $\lambda \in \Lambda$ then the left vertical isomorphism in the diagram	
\[
\begin{tikzcd}
	G \cdot C^{\tau} \ar{d}[swap]{\lambda} \ar{r}  
&
	G \cdot_{\mathfrak{C}} \vect{C}
	\ar{d}{\lambda}
\\
	G \cdot C^{\tau} \ar{r}
&
	G \cdot_{\mathfrak{C}} \vect{C}
\end{tikzcd}
\]
induces the right vertical isomorphism which is moreover a map in 
$\Phi_{\mathfrak{C}}$.

We now claim that, adapting \eqref{CANPUSH EQ},
$\Sigma_{\mathfrak{C}}[G\cdot_{\mathfrak{C}} \vect{C}]/\Lambda$
is a pushout of
$\Sigma_{\tau}[G\cdot C]/\Lambda$.
Indeed, this follows from the way colimits in (co)fibered categories are calculated:
since $\Lambda$ does not act trivially on 
$\boldsymbol{E}(G\cdot C)$,
the set of colors of 
$\Sigma_{\tau}[G\cdot C]/\Lambda$
is instead given by the orbits
$\boldsymbol{E}(G\cdot C)/ \Lambda$
so that, writing 
$\pi \colon \boldsymbol{E}(G\cdot C) \to \boldsymbol{E}(G\cdot C)/ \Lambda$ one has a natural identification 
$\Sigma_{\tau}[G\cdot C]/\Lambda \simeq \left(\pi_!\Sigma_{\tau}[G\cdot C]\right)/\Lambda$
where the second coequalizer now takes place in a fiber category,
and writing 
$\phi \colon \boldsymbol{E}(G\cdot C)/ \Lambda \to \mathfrak{C}$
for the induced map one further has
\begin{equation}\label{CANPUSHQ EQ}
\Sigma_{\mathfrak{C}}[G\cdot_{\mathfrak{C}} \vect{C}]/\Lambda
\simeq
\phi_! \left( \Sigma_{\tau}[G\cdot C]/\Lambda \right).
\end{equation}
\end{remark}

\eqref{CANPUSHQ EQ} will allow us to minimize  
the representatives of the generating sets in \eqref{FVGSIGF EQ}
that are needed to obtain the generating cofibrations in 
$\mathsf{Op}^G_{\F}(\V)$.

Before describing the generating sets, however, we need to address the path-lifting condition.
Since fibrations in $\mathsf{Op}^G_{\F}(\V)$ are to have the lifting property against all maps $\eta \to \mathbb I$
with $\mathbb{I}$ a $\V$-interval.
But since the collection of all intervals form a class, one must be able to select a suitable representative set of intervals, leading to the following definition (cf. \cite{BM13}).




\begin{definition}
	A set $\mathscr{G}$ of $\V$-intervals is \textit{generating} if,
	in the model category on $\Cat_{\set{0,1}}(\V)$,
	any $\V$-interval $\mathbb{I}$ can be obtained
	as a retract of a trivial extension of some element
	$\mathbb{G} \in \mathscr{G}$.
	More explicitly, this means that there is a diagram in 
	$\Cat_{\set{0,1}}(\V)$ as below
	where the left arrow is a trivial cofibration and
	$ri = id_{\mathbb{I}}$.
\begin{equation}\label{GTILGI EQ}
	\begin{tikzcd}
		\mathbb{G} \arrow[r,rightarrowtail, "\sim"]
	&
		\widetilde{\mathbb{G}} \arrow[r,yshift=-.3em, "r"']
	&
		\mathbb{I} \arrow[l,yshift=.3em, "i"']
	\end{tikzcd}
\end{equation}
\end{definition}


The following essentially recalls \cite[1.20]{CM13b}, \cite[\S 4.3]{Cav}.


\begin{remark}
When $\V$ is either $\mathsf{sSet}$ or $\mathsf{sSet}_{\**}$
one can take $\mathscr{G}$ to be a set of representatives of isomorphism classes of intervals with countably many cells.
Indeed, since in both cases the mapping spaces of a $\V$-interval
$\mathbb{I}$ are a simplicial set with (either one or two) contractible components,
a standard argument ({\color{red} add reference})
shows that $\mathbb{I}$ has a countable subcomplex 
$\mathbb{G}$ for which the inclusion 
$\mathbb{G} \to \mathbb{I}$
is an equivalence in $\mathsf{Cat}_{\{0,1\}}(\V)$.
But then forming the cofibration followed by trivial fibration factorization
$\mathbb{G} \rightarrowtail \widetilde{\mathbb{G}}
\overset{\sim}{\twoheadrightarrow} \mathbb{I}$
in $\mathsf{Cat}_{\{0,1\}}(\V)$
one has that the first map is a trivial cofibration by $2$-out-of-$3$
and that the second has a section since $\mathbb{I}$ is cofibrant by assumption, yielding \eqref{GTILGI EQ}.

More generally, a more careful argument \cite[Lemma 1.12]{BM13}
shows that every combinatorial monoidal model category
has a generating set of intervals.
\end{remark}




\begin{proposition}\label{GENIN PROP}
If $\V$ has a generating set of intervals $\mathscr{G}$ then a local $\F$-fibration $F \colon \O \to \P$ is $\F$-path lifting iff it has the right lifting property against the maps 
$\{\eta \to \mathbb{G}\}_{\mathbb{G}\in \mathscr{G}}$.
\end{proposition}



\begin{proof}
Given some chosen interval $\mathbb{I}$,
let $\mathbb{G}, \widetilde{\mathbb{G}}$
be as in \eqref{GTILGI EQ}.
A standard argument concerning retractions shows that to solve a lifting problem against $\eta \to \mathbb{I}$
is suffices to solve the induced lifting problem against
$\eta \to \widetilde{\mathbb{G}}$.
But now given a lifting problem against 
$\eta \to \widetilde{\mathbb{G}}$
we consider the diagram where the solid lift exits by hypothesis on $F$.
\[
\begin{tikzcd}
	\eta \ar{d} \ar{rr}  
&&
	\O 	\ar{d}{F}
\\
	\mathbb{G} \ar[rightarrowtail]{r}{\sim} \ar{rru}
&
	\widetilde{\mathbb{G}} \ar{r} \ar[dashed]{ru}
&
	\P
\end{tikzcd}
\]
But then since $\mathbb{G} \overset{\sim}{\rightarrowtail} \widetilde{\mathbb{G}}$
is a trivial cofibration in $\mathsf{Cat}_{\{0,1\}}(\V)$
and $F$ is a local fibration
the desired dashed lift exists
by Proposition \ref{LOCALTCHAR PROP}.
\end{proof}


We can now finally identify the generating (trivial) cofibrations of
$\mathsf{Op}^G_{\F}$.


\begin{definition}
Suppose that $\V$ has a generating set of intervals $\mathscr{G}$.

Then the generating cofibrations in $\mathsf{Op}^G_{\F}$
are the maps
\begin{itemize}
\item[(C1)] $\emptyset \to G/H \cdot \eta$ for $H \in \F_1$,
\item[(C2)] $\mathbb{F} \left( \Sigma_{\tau}[G \cdot C_n]/\Lambda \cdot i\right)$
for $n \geq 0$, $\Lambda \in \F_n$ and $i \in \mathcal{I}$,
\end{itemize}
while the generating trivial cofibrations are the maps 
\begin{itemize}
\item[(TC1)] 
$G/H \cdot \left(\eta \to \mathbb{G}\right)$ for $H \in \F_1$ and $\mathbb{G} \in \mathscr{G}$,
\item[(TC2)] 
$\mathbb{F} \left( \Sigma_{\tau}[G \cdot C_n]/\Lambda \cdot j\right)$
for $n \geq 0$, $\Lambda \in \F_n$ and $j \in \mathcal{J}$.
\end{itemize}
\end{definition}


\begin{remark}
	The required claim that the maps with the right lifting property against (TC1) and (TC2) are the $\F$-fibrations as defined by
	Definition \ref{MODEL_DEFN} follows from 
	Propositions \ref{LOCALTCHAR PROP} and \ref{GENIN PROP},
	while the claim that the maps with the lifting property against (C1) and (C2) are the $\F$-trivial fibrations follows from
	Proposition \ref{FTRIVCHAR PROP}.
\end{remark}




\begin{lemma}[{cf. \cite[1.19]{CM13b}}]\label{POINT_4_LEMMA}
	The maps in (TC1),(TC2) are in the saturation of (C1),(C2),
	i.e. trivial cofibrations are cofibrations.
\end{lemma}

\begin{proof}
	Clearly (TC2) is in the saturation of (C2).
	As for (TC1), one has factorizations
\begin{equation}
	\begin{tikzcd}
	G/H \cdot \1 \arrow[r, rightarrowtail]
&
	G/H \cdot (\1 \amalg \1) \arrow[r, rightarrowtail]
&
	G/H \cdot \mathbb{G}
	\end{tikzcd}
\end{equation}
where the first map is a pushout of a map in (C1) and 
the second map is in the saturation of (C2).
\end{proof}







\newpage

\subsection{Required observations on intervals}


We start by recalling the following, 
which is a challenging technical result in \cite{BM13}.

\begin{theorem}
[Interval Cofibrancy Theorem; cf. {\cite[Thm. 1.15]{BM13}}]
\label{INTCOF THM}
Suppose $\V$ is {\color{blue} adequate}.
Then for any cofibrant $\mathbb{I} \in \mathsf{Cat}_{\{0,1\}}(\V)$
one has that 
$\mathbb{I}(0,0)$ %and $\mathbb{I}(1,1)$
is a cofibrant monoid, i.e. cofibrant
in $ \mathsf{Cat}_{\{0\}}(\V)$.
\end{theorem}



\begin{remark}
By symmetry, one must also have that $\mathbb{I}(1,1)$ is cofibrant.
Moreover, the formulation in \cite[Thm. 1.15]{BM13}
includes additional cofibrancy conditions for
$\mathbb{I}(0,1),\mathbb{I}(1,0)$
as modules over $\mathbb{I}(0,0),\mathbb{I}(1,1)$.
These additional conditions are essential for the proof therein, 
but not needed for our application.
\end{remark}


We note that the Interval Cofibrancy Theorem is a particular case of the following conjecture when $\mathfrak{C} \to \mathfrak{D}$
is the inclusion $\{0\} \to \{0,1\}$.


\begin{conjecture}\label{CATOP CONJ}
Let $f \colon \mathfrak{C} \to \mathfrak{D}$
be an injection of colors.
Then the functors
\[
	\mathsf{Cat}_{\mathfrak{D},\F}^G
	\xrightarrow{f^{\**}}
	\mathsf{Cat}_{\mathfrak{C},\F}^G
\qquad
	\mathsf{Op}_{\mathfrak{D},\F}^G
	\xrightarrow{f^{\**}}
	\mathsf{Op}_{\mathfrak{C},\F}^G
\]
preserve cofibrations between cofibrant objects.
\end{conjecture}




\begin{remark}
To see why Conjecture \ref{CATOP CONJ} is at least plausible,
we claim that $f^{\**}$ does at least send generating cofibrations to generating cofibrations, 
which is essentially tantamount to it sending free objects to free objects.
To see this, consider the simplest example 
where the inclusion of colors is 
$f \colon \{0\} \to \{0,1\}$
and $\mathbb{F}_{\{0,1\}}X \in \mathsf{Cat}_{\{0,1\}}$ be free.
Then one can check that $f^{\**} \mathbb{F}_{\{0,1\}}X \in \mathsf{Cat}_{\{0\}}$ is the free monoid
\begin{equation}\label{PULLFREEEX EQ}
	f^{\**} \mathbb{F}_{\{0,1\}}X
\simeq
	\mathbb{F}_{\{0\}}
	\left(X(0,0) \amalg 
	\coprod_{n \geq 0}
	X(0,1)\otimes X(1,1)^{\otimes n} \otimes X(1,0) 
	\right)
\end{equation}
where we note that the expression inside
$\mathbb{F}_{\{0\}}$
in \eqref{PULLFREEEX EQ}
can be intuitively described as the formal composites
$0 \to 1 \to 1 \to \cdots \to 1 \to 0$
of ``arrows'' in $X$ which start and end at $0$ and where all intermediate objects are $1$.
More generally, for an inclusion of colors 
$f \colon \mathfrak{C} \to \mathfrak{D}$
one has that $f^{\**} \mathbb{F}_{\mathfrak{D}} X$
is similarly free on formal composites
$c_0 \to d_1 \to d_2 \to \cdots \to d_n \to c_{n+1}$
of arrows in $X$
where $c_i \in \mathfrak{C}$
and $d_j \in \mathfrak{D} \setminus \mathfrak{C}$,
while for operads the analogue claim involves labeled trees whose root and leaves are labeled by $\mathfrak{C}$
and whose inner edges are labeled by 
$\mathfrak{D} \setminus \mathfrak{C}$.
 
It is then straightforward to check that, under mild assumptions on $\V$,
$f^{\**} \mathbb{F}_{\mathfrak{D}} X$
will be a (trivial) cofibration in 
$\mathsf{Cat}^G_{\mathfrak{C},\F}$
(resp. $\mathsf{Cat}^G_{\mathfrak{C},\F}$))
when $\mathbb{F}_{\mathfrak{D}} X$
is a generating (trivial) cofibration
in $\mathsf{Cat}^G_{\mathfrak{D},\F}$
(resp. $\mathsf{Cat}^G_{\mathfrak{D},\F}$)).
However, the argument just given \emph{does not} constitute a proof of Conjecture \ref{CATOP CONJ}
since $f^{\**}$ does not preserve pushouts, 
so that to truly prove Conjecture \ref{CATOP CONJ}
one would need a rather careful analysis of the interaction of $f^{\**}$ with pushouts of free categories/operads, 
as featured in the proof of \cite[Thm. 1.15]{BM13}.

Lastly, we make note of a very similar conjecture:
it is natural to ask if the restriction functor
$j^{\**} \colon \mathsf{Op}_{\mathfrak{C},\F}^G 
\to \mathsf{Cat}_{\mathfrak{C},\F}^G$
preseves cofibrations between cofibrant objects,
and again one has that $j^{\**}$ 
sends generating (trivial) cofibrations to (trivial) cofibrations but,
since we are working with operads with $0$-ary operations/units,
$j^{\**}$ does not preserve pushouts
(indeed, this would be roughly tantamount to the claim that trees with a single leaf are linear trees, which is not true if we work with trees that may have stumps).
\end{remark}


\newpage

\subsection{Equivalences of objects}

One of the main challenges in showing that the classes in Definition \ref{MODEL_DEFN} is to show that the prescribed 
$\F$-weak equivalences satisfy $2$-out-of-$3$,
with the main difficulty being the fact that essential surjectivity is 
defined using $\V$-intervals.
To address this, we will relate our prescribed weak equivalences
with the so called Dwyer-Kan equivalences,
for which $2$-out-of-$3$ is easier to establish
(although, at in our intended equivariant setting, 
this claim requires a little more care than it does non-equivariantly).



\begin{definition}\label{HTPY_DEFN}
	Suppose $\V$ has a cofibrant unit.

	Given $\mathcal C \in \Cat_{\mathfrak{C}}(\V)$,
	we define $\pi_0 \mathcal C \in \Cat_{\mathfrak{C}}(\mathsf{Set})$ 
	to be the ordinary category with the same objects and
\[
	\pi_0(\mathcal{C})(c,d)=
	\Ho(\V)(1_\V, \mathcal C(c,c'))=
	[1_\V, \mathcal{C}_f(c,c')]
\]
where $[-,-]$ denotes homotopy equivalence classes of maps
and $\mathcal{C}_f$ denotes some fibrant replacement of
$\mathcal C$ in $\Cat_{\mathfrak{C}}(\V)$.
\end{definition}


\begin{remark}
$\pi_0\colon \mathsf{Cat}_{\mathfrak{C}}(\V)
\to \mathsf{Cat}_{\mathfrak{C}}(\mathsf{Set})$ is functorial,
i.e. a $\V$-functor
$\mathcal{C} \to \mathcal{D}$
induces a functor 
$\pi_0\mathcal{C} \to \pi_0\mathcal{D}$.
Moreover, $\pi_0$ sends weak equivalences to isomorphisms.
\end{remark}

\begin{remark}
Since any two fibrant replacements are connected by a zigzag of weak equivalences,
(the isomorphism class of) $\pi_0 \mathcal{C}$ does not depend on the choice of fibrant replacement $\mathcal{C}_f$.
\end{remark}



\begin{remark}
The composition $[g]\circ [f]$
in $\pi_0(\mathcal{C})$
of classes $[f],[g]$
represented by
$1_{\mathcal{V}} \xrightarrow{f} \mathcal{C}_f({c,c'})$
and 
$1_{\mathcal{V}} \xrightarrow{g} \mathcal{C}_f({c',c''})$
is given by the class $[gf]$, where $gf$ denotes the composite
\[
	1_{\mathcal{V}} \simeq
	1_{\mathcal{V}} \otimes 1_{\mathcal{V}} \xrightarrow{g \otimes f}
	\mathcal{C}_f({c',c''}) \otimes  \mathcal{C}_f({c,c'}) \xrightarrow{\circ}
	\mathcal{C}_f({c,c''}).
\]
Note that the claim that this respects equivalence classes uses the fact that $1_{\mathcal{V}}$ is cofibrant.
\end{remark}



\begin{remark}
The map $\widetilde{\mathbbm{1}} \to \widetilde{\mathbbm{1}}_f$
readily shows that in $\pi_0 \widetilde{\mathbbm{1}}$
the two objects $0,1$ are isomorphic.
\end{remark}




Following \cite[Def. 2.6]{BM13} (also, \cite{Cav}),
we make the following definitions.

\begin{definition}\label{EQUIV_DEF}
	Given $\mathcal{C}$ in  $\Cat(\V)$ and $c,c'\in\mathrm{Ob}(\mathcal C)$, we say $c$ and $c'$ are
\begin{itemize}
	\item {\em equivalent} if there exists a $\V$-interval $\mathbb{I}$
	and map $\gamma: \mathbb{I} \to \mathcal C$ such that
	$\gamma(0)= c$, $\gamma(1)= c'$;
	\item {\em virtually equivalent} if $c,c'$ are equivalent in some fibrant replacement
	$\mathcal C_f$ of $\mathcal C$ in $\Cat_{\mathrm{Ob}(\mathcal C)}(\V)$;
	\item {\em homotopy equivalent} if $c,c'$ are isomorphic in the unenriched category $\pi_0 \mathcal C$.

	Explicitly, this means there are maps 
	$1_\V \xrightarrow{\alpha} \mathcal C_f(c,c')$, 
	$1_\V \xrightarrow{\beta} \mathcal C_f(c',c)$ such that
	$1_{\V} \xrightarrow{\beta \alpha} \mathcal C_f(c,c)$,
	$1_{\V} \xrightarrow{\alpha \beta} \mathcal C_f(c',c')$
	are homotopic to the identities
	$1_{\V} \xrightarrow{id_c} \mathcal C_f(c,c)$,
	$1_{\V} \xrightarrow{id_{c'}} \mathcal C_f(c',c')$.
\end{itemize}
\end{definition}


\begin{remark}
It is immediate that the notions in Definition \ref{EQUIV_DEF}
are nested: the map $\mathcal{C} \to \mathcal{C}_f$
yields that equivalence implies virtual equivalence;
a map $\mathbb{I} \to \mathcal{C}_f$
with $\mathbb{I}$ a $\mathcal{V}$-interval
induces a map
$\pi_0 \widetilde{\mathbbm{1}} \simeq \pi_0 \mathbb{I} \to \pi_0 \mathcal{C}$ so that (since $0,1 \in \pi_0 \widetilde{\mathbbm{1}}$ are isomorphic)
virtual equivalence implies homotopy equivalence.

Moreover, \cite{BM13,Cav} show that, under suitable assumptions on $\V$, the converse implications also hold, as summarized below.
We discuss these converse results {\color{red} later \S \ref{DK_SEC}}.
\[
	\begin{tikzcd}[column sep = large]
            \mbox{equivalent}
            \arrow[r, Rightarrow, shift left=2, "\mathrm{always}"]
            &
            \mbox{ virtually equivalent}
            \arrow[r, Rightarrow, shift left=2, "\mathrm{always}"]
            \arrow[l, Rightarrow, shift left = 2, "\substack{$\V\phantom{ }$\mathrm{right} \\ \mathrm{proper}}"]
            &
            \mbox{ homotopy equivalent}
            \arrow[l, Rightarrow, shift left = 2, "\substack{\mathrm{coherence} \\ \mathrm{condition}}"]
	\end{tikzcd}
\]
\end{remark}



Replacing the notion of equivalence of objects
with that of homotopy equivalence, 
leads to the following variation on the 
$\F$-weak equivalences in Definition \ref{MODEL_DEFN}. 


\begin{definition}\label{DKEQUIV_DEF}
Let $\F$ be a $(G,\Sigma)$-family which respects units.
We say a map $\O \to \P$ in $\Op^G(\V)$ is:
\begin{itemize}
\item \textit{$\F$-$\pi_0$-essentially surjective} if
	$j^{\**}\pi_0 \O^H \to j^{\**}\pi_0 \P^H$
	is essentially surjective for $H \in \F_1$;
\item a \textit{$\F$-Dwyer-Kan equivalence} if
	it is a local $\F$-weak equivalence and $\F$-$\pi_0$-essentially
	surjective.
\end{itemize}
\end{definition}


\begin{remark}
The requirement that 
a $\F$-Dwyer-Kan equivalence $\O \to \P$
be a local $\F$-weak equivalance
implies that the maps of categories 
$j^{\**}\pi_0 \O^H \to j^{\**}\pi_0 \P^H$
must also be local isomorphisms, 
and thus equivalences in the category $\mathsf{Cat}$
of (unenriched) categories in the usual sense.
And since equivalences in $\mathsf{Cat}$ satisfy $2$-out-of-$3$
while local $\F$-weak equivalences 
are easily seen to satisfy two out of the three cases in $2$-out-of-$3$,
the $2$-out-of-$3$ condition for 
$\F$-Dwyer-Kan equivalences reduces to showing that, 
if $F$ and $\bar{F}F$ are $\F$-Dwyer-Kan equivalences,
then $\bar{F}$ is a local $\F$-weak equivalence.

As it turns out, this case is interesting enough to warrant its own section, and we prove it separately as
Proposition \ref{23HARDCASE PROP}
\end{remark}




\subsection{Homotopy equivalences and fully faithfulness}

The following is essentially the equivariant analogue of
\cite[Lemma 4.14]{Cav}.

\begin{proposition}\label{23HARDCASE PROP}
Let $\F$ be a $(G,\Sigma)$-family which respects units. 
Consider the diagram below 
in $\mathsf{Op}^G(\V)$.
\begin{equation}\label{23HARDDIAG EQ}
	\begin{tikzcd}[row sep=5pt]
		\O \arrow{rr}{\bar{F}F}
		\arrow{dr}[swap]{F}
	&&
		\mathcal{Q} 
	\\
	&
		\mathcal{P} \ar{ru}[swap]{\bar{F}}
	\end{tikzcd}
\end{equation}
If $F$ and $\bar{F}F$ are $\F$-Dwyer-Kan equivalences
then $\bar{F}$ is a local $\F$-weak equivalence.
\end{proposition}

The proof of this result will mostly adapt the proof of the non-equivariant case,
but one must be more careful since equivariance introduced a address a number of subtleties.
Moreover, these subtleties are (for the most part)
present even when $\V = \mathsf{Set}$
with weak equivalences the isomorphisms.
Hence, for the sake of motivation, 
we start by discussing the $\V = \mathsf{Set}$ case
in a concrete example.

{\color{red} HERE}



\begin{example}
Let $G = \{1,i,-1,-i\} \simeq \mathbb{Z}_{/4}$ 
be the group of quartic roots of unit and
$\mathfrak{C} = \{\mathfrak{a}, \mathfrak{b}, i \mathfrak{b}, 
\mathfrak{c} \}$ where we implicitly have
$i\mathfrak{a} = \mathfrak{a}$,
$-\mathfrak{b} = \mathfrak{b}$,
$i\mathfrak{c} = \mathfrak{c}$.
Consider the $\mathfrak{C}$-corollas below.
\begin{equation}
	\begin{tikzpicture}[auto,grow=up, level distance = 2.2em,
	every node/.style={font=\scriptsize,inner sep = 2pt}]%
		\tikzstyle{level 2}=[sibling distance=3em]%
			\node at (0,0) [font = \normalsize] {$\vect{B}$}%	
				child{node [dummy] {}%
					child{node {}%
					edge from parent node [swap] {$\mathfrak{b}$}}%
					child[level distance = 2.9em]{node {}%
					edge from parent node [swap,	near end] {$i\mathfrak{b}$}}%
					child[level distance = 2.9em]{node {}%
					edge from parent node [near end] {$i\mathfrak{b}$}}%
					child{node {}%
					edge from parent node  {$\mathfrak{b}$}}%
				edge from parent node [swap] {$\mathfrak{a}$}};%
			\node at (4.5,0) [font = \normalsize] {$\vect{C}$}%	
				child{node [dummy] {}%
					child{node {}%
					edge from parent node [swap] {$\mathfrak{c}$}}%
					child[level distance = 2.9em]{node {}%
					edge from parent node [swap,	near end] {$\mathfrak{c}$}}%
					child[level distance = 2.9em]{node {}%
					edge from parent node [near end] {$\mathfrak{c}$}}%
					child{node {}%
					edge from parent node  {$\mathfrak{c}$}}%
				edge from parent node [swap] {$\mathfrak{a}$}};%
	\end{tikzpicture}%
\end{equation}%
Now consider maps
$\O \xrightarrow{F} \P \xrightarrow{\bar{F}} \mathcal{Q}$
as in \eqref{23HARDDIAG EQ},
and suppose 
$\mathfrak{C}_{\O} = \{\mathfrak{a},\mathfrak{c}\}$,
$\mathfrak{C}_{\P} = \mathfrak{C} = \{\mathfrak{a},\mathfrak{b},i\mathfrak{b},\mathfrak{c}\}$.
Then if $F$ and $\bar{F}F$ are local isomorphisms
it is immediate that
$\P(\vect{C}) \to \mathcal{Q}(\bar{F}(\vect{C}))$
is an isomorphism since 
$\vect{C}$ is in the image of $F$.
However, to guarantee that
$\P(\vect{B}) \to \mathcal{Q}(\bar{F}(\vect{B}))$
is also an isomorphism, 
one must further impose an essential surjectivity requirement on $F$.
For concreteness,
suppose there was an isomorphism
$\alpha \colon \mathfrak{b} \to \mathfrak{c}$
in $\P$
(note that $\alpha \in \P(\mathfrak{b};\mathfrak{c})$).
Then, by $G$-equivariance, one also has an isomorphism
$i\alpha \colon i\mathfrak{b} \to \mathfrak{c}$
in $\P$ and precomposing  
with $\alpha$, $i \alpha$ gives a string of isomorphisms
\begin{equation}\label{ITERWE EQ}
\P(\vect{C})
=
\P(\mathfrak{c},\mathfrak{c},\mathfrak{c},\mathfrak{c};\mathfrak{a})
\simeq 
\P(\mathfrak{b},\mathfrak{c},\mathfrak{c},\mathfrak{c};\mathfrak{a})
\simeq
\P(\mathfrak{b},i\mathfrak{b},\mathfrak{c},\mathfrak{c};\mathfrak{a})
\simeq
\P(\mathfrak{b},i\mathfrak{b},i\mathfrak{b},\mathfrak{c};\mathfrak{a})
\simeq
\P(\mathfrak{b},i\mathfrak{b},i\mathfrak{b},\mathfrak{b};\mathfrak{a})
=
\P(\vect{B})
\end{equation}
and similarly
$\mathcal{Q}(\bar{F}(\vect{C})) 
\simeq
\cdots 
\simeq
\mathcal{Q}(\bar{F}(\vect{B}))
$,
yielding that
$\P(\vect{B}) \to \mathcal{Q}(\bar{F}(\vect{B}))$
is indeed an isomorphism.

Our discussion thus far has ignored a key feature of the equivariant setting: the choice of the $(G,\Sigma)$-family $\F$
(or, more precisely, our discussion above covers the case of $\F$ the naive/coarse family consisting of only the trivial subgroups).
Given such $\F$, 
and assuming $F$ and $\bar{F} F$ are $\F$-local isomorphisms,
it is again clear that
$\P(\vect{C})^{\Lambda} \to \mathcal{Q}(\bar{F}(\vect{C}))^{\Lambda}$
is an isomorphism for all $\Lambda \in \F_{\vect{C}}$.
And, yet again, to conclude that
$\P(\vect{B})^{\Lambda} \to \mathcal{Q}(\bar{F}(\vect{B}))^{\Lambda}$
is also an isomorphism for $\Lambda \in \F_{\vect{B}}$ we will further need to impose an essential surjectivity requirement on $F$.
As it turns out, the exact requirement on $F$ depends on 
$\Lambda \leq G \times \Sigma_4^{op}$ itself so that, for concreteness, we set (we write $(g,\sigma) \in G \times \Sigma_4^{op}$ simply as $g\sigma$)
\[
\Lambda = \langle (14)(23), i (12)(34) \rangle.
\]
and assume that $\Lambda \in \F_4$.
Note that $\Lambda$ fixes both 
$\vect{B}$ and $\vect{C}$
(cf. Notation \ref{SIGACT REM})
so that 
$\Lambda \leq \mathsf{Aut}_{G \ltimes \Sigma^{op}_{\mathfrak{C}}}(\vect{B})$,
$\Lambda \leq \mathsf{Aut}_{G \ltimes \Sigma^{op}_{\mathfrak{C}}}(\vect{C})$
and $\P(\vect{B})^{\Lambda}, \P(\vect{C})^{\Lambda}$
are well defined.
At this point it may be tempting to think that, 
given an isomorphism 
$\alpha \colon \mathfrak{b} \to \mathfrak{c}$,
one may simply apply $\Lambda$-fixed points 
to the isomorphisms in \eqref{ITERWE EQ}
to obtain an isomorphism 
$\P(\vect{B})^{\Lambda} \simeq \P(\vect{C})^{\Lambda}$.



{\color{red} HERE}

\end{example}







Then $\Lambda$ acts on both $\vect{B},\vect{C}$, i.e. 
$\Lambda \leq \mathsf{Aut}_{G \ltimes \Sigma^{op}_{\mathfrak{C}}}(\vect{B})$,
$\Lambda \leq \mathsf{Aut}_{G \ltimes \Sigma^{op}_{\mathfrak{C}}}(\vect{C})$.
Hence, to account for the case of a $(G,\Sigma)$-family $\F$ such that 
$\Lambda \in \F_4$,
it is natural to ask if the chosen homotopy equivalence
$\alpha \colon 1_{\V} \to \P(\mathfrak{b},\mathfrak{c})$
between $\mathfrak{b},\mathfrak{c}$ also induces a fixed point equivalence
$\P(\vect{B})^{\Lambda} \sim \P(\vect{C})^{\Lambda}$.
This turns out to essentially work, provided one 
accounts for the following caveats:


\begin{itemize}
\item one can not change colors one by one as in \eqref{ITERWE EQ},
since the intermediate signatures therein are not $\Lambda$-equivariant. 
Instead, colors in $\vect{B}$ which are in the same orbit
under the $\Lambda$-action must be changed simultaneously;
\item moreover, noting that the action of 
$\Lambda $

colors of $\vect{B}$
\end{itemize}









\newpage

\begin{equation}
	\begin{tikzcd}
		\O_1 \arrow{rr}{gf}
		\arrow{dr}[swap]{f}
	&&
		\O_3 
	\\
	&
		\O_2 \ar{ru}[swap]{g}
	\end{tikzcd}
\end{equation}



For 2-out-of-3 must consider three cases

\begin{itemize}
\item[(f)] if $g$ and $gf$ are weak equivalences then $f$ is a weak equivalence
\item[(g)] if $f$ and $gf$ are weak equivalences then $g$ is a weak equivalence
\item[(gf)] if $f$ and $g$ are weak equivalences then $gf$ is a weak equivalence
\end{itemize}


Each of these cases requires checking that the third map 
is both a local $\F$-weak equivalence and essentially surjective, and in each case one of these conditions is immediate while the other requires a more careful analysis.
More explicitly, the non-trivial half of case (f) is the $\F$-equivalence claim while the non-trivial halves of (g),(gf) are essential surjectivity. 
Since essential surjectivity only depends on the underlying fixed point systems of categories of an equivariant operad, 
cases (f),(gf) in fact reduce to the work in \cite{Cav,BM13}, 
so that case (g) is the true novelty of our work in this section.
 







\newpage






\subsection{Trivial cofibrations}


The first of these two results, Proposition \ref{J-CELL_PROP}, uses the following straightforward lemmas.

\begin{lemma}
      \label{TRANSCOMP_ES_LEM}
      Transfinite composition of essentially surjective maps in $\Op^G(\V)$ is essentially surjective.
\end{lemma}
\begin{proof}
      Since taking fixed points commutes with filtered colimits, they commute with transfinite composition,
      and hence by \cite[4.17]{Cav}, we are done.
\end{proof}

\begin{lemma}
      \label{TRANSCOMP_LGC_LEM}
      Local genuine cofibrations in $\Op^G(\V)$ are closed under transfinite composition.
\end{lemma}

\begin{proof}
      Since for any $\mathfrak C$-signature $C \in G \ltimes \SC^{op}$ and any map of $\mathfrak C$-colored operads $f$ we have that
      the restriction map
      $\V^{\Aut(f(C))}_{gen} \to \V^{\Aut(C)}_{gen}$
      is left Quillen,
      and any transfinite composition of operads locally is of the form
      \begin{equation}
            \O_0(C) \to \O_1(F_1(C)) \to \O_2(F_2(C)) \to \dots   
      \end{equation}
      in $\V^{\Aut(C)}_{gen}$
      for $F_\alpha$ the (transfinite) composite $\O_0 \to \O_1 \to \dots \to \O_\alpha$,
      the result follows.
\end{proof}

\begin{proposition}[{c.f. \cite[4.20]{Cav}}]\label{J-CELL_PROP}
      Suppose $\V$ is as in Convention \ref{ALLCOLOR_CONV}. %has a cofibrant unit $1_\V$ and has cofibrant symmetric pushout powers.
      Then relative $J_{\F}$-cells with locally genuine cofibrant source are weak equivalences.
\end{proposition}
\begin{proof}
      By Lemmas \ref{TRANSCOMP_ES_LEM} and \ref{TRANSCOMP_LGC_LEM}, it suffices to show that
      the pushout of a map $j \in J_\F$ is both
      essentially surjective and a local genuine trivial cofibration.

      Firstly, if $j = \mathbb F_\Gamma[\Gamma/\Gamma \cdot i] \in J_{\F, loc}$,
      then by Remark \ref{COLOR_SQ_REM}(ii) the pushout can be computed in a fixed-color category $\mathsf{Op}^{G,\mathfrak C_{\P}}(\V)$.
      By Corollary \ref{COLOR_CHANGE_Q_COR}, the relevant span is a trivial cofibration in one leg, while the other leg has a locally cofibrant target.
      Thus by the existence of the $\F$-semi-model structures from Theorem \ref{THM1_C},
      the pushout is again a trivial cofibration, and hence by Corollary \ref{LGC_COR} a local genuine trivial cofibration.
      As it is the identity on colors, it is also essentially surjective.

      Secondly, supppose $j$ is of the form $G/H \cdot (\1 \to \J)$ for $\J$ a $\V$-interval.
      We split this pushout into a composition of two pushouts
      \begin{equation}
            \begin{tikzcd}
                  G/H \cdot \1 \arrow[r, "a"] \arrow[d, "G/H \cdot \phi"']
                  % \arrow[dr,phantom, yshift=.1em, xshift=.5em, "\lrcorner" near end]
                  &
                  \O \arrow[d,"\phi'"]
                  \\
                  G/H \cdot \J_{\set{0}} \arrow[r] \arrow[d, "G/H \cdot \psi"']
                  % \arrow[dr,phantom, yshift=.1em, xshift=.5em, "\lrcorner" near end]
                  &
                  \O' \arrow[d,"\psi'"]
                  \\
                  G/H \cdot \J \arrow[r]
                  &
                  \P
            \end{tikzcd}
      \end{equation}
      where $\J_{\set{0}}$ is the full subcategory of $\J$ spanned by the object $0$.
      It suffices to show both $\psi'$ and $\phi'$ are local genuine trivial cofibrations which are essentially surjective on fixed points. 

      We first consider the bottom pushout.
      We know that $\psi$ is injective on colors and a local isomorphism in $\Op(\V)$,
      and hence so is $G/H \cdot \psi$ in $\Op^G(\V)$.
      Since colimits are created non-equivariantly, and equivariant isomorphisms are detected by invertible equivariant maps,
      Corollary \ref{LOCALISO_COR} below implies that $\psi'$ is also a local isomorphism in $\Op^G(\V)$ \footnote{
        This also follows from the more general result \cite[Prop. B.22]{Cav},
        and from \cite[Prop. 1.28]{CM13b} in the case $(\V, \otimes) = (\sSet, \times)$.
      }.
      % so in particular a local trivial $\F$-cofibration.
      
      Moreover, we observe that $\C_{\P} = \C_{\O'} \amalg (G/H \times \set{1})$.
      Thus, if $x \in \C_{\P}^K$ is in $\C_{\O'}$ for some $K \leq G$, we have essential surjectivity trivially,
      as shown on the left below in \eqref{J-CELL_EQ},
      where $\I_c \to \I$ is a cofibrant replacement in $\Op^{\set{0,1}}(\V)$.
      \begin{equation}
            \label{J-CELL_EQ}
            \begin{tikzcd}
                  \1 \arrow[r, "0"] \arrow[dr, "i_0"']
                  &
                  \J_{\set{0}} \arrow[d] \arrow[r, "g"]
                  &
                  (\O')^K \arrow[d, "{\psi'}"]
                  \\
                  &
                  \J \arrow[r, "g"]
                  &
                  \P^K \arrow[d, equal]
                  \\
                  \1 \arrow[rr, "g \cdot 1"] \arrow[ur, "i_1"]
                  &&
                  \P^K
            \end{tikzcd}
      \end{equation}
      If instead $x  = g \cdot 1 \in (G/H \cdot 1)^K \subseteq \mathfrak C_\P^K$,
      the pushout square yields the diagram on the right above in \eqref{J-CELL_EQ},
      where the maps $\J_{\set{0}} \xrightarrow{g} (\O')^K$, $\J \xrightarrow{g} \P^K$ are adjoint to the composites
      \begin{equation}
            G/K \cdot \J_{\set{0}} \xrightarrow{g} G/H \cdot \J_{\set{0}} \longto \O',
            \qquad \qquad
            G/K \cdot \J \xrightarrow{g} G/H \cdot \J \longto \P
      \end{equation}
      (using that $(G/H)^K \simeq \Hom(G/K, G/H)$).
      % Lastly, if we consider (any element in the orbit of) the new object $1\in \C(\P)^H$,
      % there is an associated object $0 \in \C(\O')^H$ such that the essentially surjectivity diagram
      % factors through the pushout diagram for $\psi$:
      % \begin{equation}
      %       \begin{tikzcd}
      %             G/H \cdot \1 \arrow[r,"0"] \arrow[dr, "G/H \cdot i_0"']
      %             &
      %             G/H \cdot \J_{\set{0}} \arrow[r] \arrow[d]
      %             &
      %             \O' \arrow[d, "\psi'"]
      %             \\
      %             &
      %             G/H \cdot \J \arrow[r]
      %             &
      %             \P \arrow[d, equal]
      %             \\
      %             G/H \cdot \1 \arrow[ur, "G/H \cdot i_1"] \arrow[rr, "1"]
      %             &&
      %             \P.
      %       \end{tikzcd}
      % \end{equation}
      Hence $\psi'$ is also essentially surjective.

      Now, consider the top pushout. Remark \ref{COLOR_SQ_REM}(ii) again implies that this pushout is created in $\Op^{G, \mathfrak C_\O}(\V)$.
      In particular, this implies $\phi'$ is bijective on objects, and hence essentially surjective.
      Further, since $1_\V$ is cofibrant in $\V$, \cite[Thm. 1.15]{BM13} implies that $\J_{\set 0}$ is cofibrant in $\Op^{\**}(\V)$,
      and since $\1$ is the initial object here, $\phi$ is a trivial cofibration here.
      Thus $a_! (G/H \cdot \phi)$ is a trivial $\F$-cofibration in $\Op^{G, \mathfrak C_\O}(\V)$ by Corollary \ref{COLOR_CHANGE_Q_COR}.
      % {(as $G/H \cdot \phi$ is one in $\Op^{G, G/H}(\V)$,
      %   since $\O \to \P$ a trivial $\F$-fibration in $\Op^{G, \mathfrak C}(\V)$
      %   implies $j^{\**}\O^H \to j^{\**}\P^H$ is one in $\Cat^{\mathfrak C^H}(\V)$)}.
      Hence, again using the $\F$-semi-model structure on $\mathsf{Op}^{G, \mathfrak C_\O}(\V)$ and the fact that $\O$ is locally cofibrant,
      $\phi'$ is a trivial $\F$-cofibration in $\mathsf{Op}^{G,\C_\O}(\V)$,
      and thus a local genuine trivial cofibration by Corollary \ref{LGC_COR}.
      
      Since both $\phi'$ and $\psi'$ are essentially surjective and local genuine trivial cofibrations,
      the result is proved.
\end{proof}

\begin{remark}
      \label{OPGCV_F_JC_REM}
      If for an independent reason we know that each $\mathsf{Op}^{G, \mathfrak C}(\V)$ had a full Quillen model structure,
      then the above proof would show that \textit{all} relative $J_\F$-cells are weak equivalences.
\end{remark}





\subsection{Homotopy in a general model category}



\begin{definition}
      For any $A \in \V$, a \textit{cylinder object for $A$} is an object $\mathbb C(A)$ equipped with a factorization of the fold map
      \begin{equation}
            \begin{tikzcd}
                  A \amalg A \arrow[r, tail, "{(i_1,i_2)}"]
                  &
                  \mathbb C(A) \arrow[r, "\sim"]
                  &
                  A
            \end{tikzcd}
      \end{equation}
      into a cofibration followed by a weak equivalence.
      
      For the tensor unit $1_\V$, we write $\mathbb C = \mathbb C(1_\V)$, and call this simply a \textit{cylinder} in $\V$.
      % A cylinder for $A$ is called \textit{good} if the second map is a trivial fibration.
      
      A (left) \textit{homotopy} between maps $f,g: A \to B$ in $\V$ is a map $H_{fg}: \mathbb C(A) \to \V(A,B)$ such that
      the diagram below commutes.
      \begin{equation}
            \begin{tikzcd}[row sep = tiny]
                  A \amalg A \arrow[rr, "{(f,g)}"] \arrow[dr]
                  &&
                  B
                  \\
                  &
                  \mathbb C(A) \arrow[ur, "H_{fg}"']
            \end{tikzcd}
      \end{equation}
      We say $f$ and $g$ are \textit{homotopic} if there exists a homotopy $H_{f g}$ between them.
\end{definition}

% \begin{remark}
%       If $B$ is fibrant, we may lift any homotopy to a homotopy out of a good cylinder,
%       using the functorial factorization
%       $\mathbb C(A) \overset{\sim}{\rightarrowtail} \mathbb C'(A) \xrightarrow{\sim}{\twoheadrightarrow} A$.
% \end{remark}

\begin{remark}
      \label{CYL_REM}
      The maps $i_\epsilon: A \to \mathbb C(A)$ are always weak equivalences by 2-out-of-3 for $\epsilon \in \set{0,1}$.
      Moreover, if $A \in \V$ is cofibrant, then it is additionally a cofibration.
      Furthermore, if $\mathbb C$ a cylinder in $\V$,
      then $A \otimes \mathbb C$ is a cylinder object for $A$,
      as the fold map can be written
      \begin{equation}
            A \amalg A \simeq A \otimes (1_\V \amalg 1_\V) \rightarrowtail A \otimes \mathbb C \xrightarrow{\sim} A
      \end{equation}
      as $A \otimes (-)$ preserves cofibrations, and, by Ken Brown's Lemma, weak equivalences between cofibrant objects.
\end{remark}

These cylinder objects provide another description of the mapping sets in the homotopy category $\Ho \V$ of $\V$.

\begin{proposition}[{\cite[1.2.10]{Hov99}}]
      If $A$ is cofibrant and $B$ fibrant, then
      homotopy is an equivalence relation $\sim$ on $\V(A,B)$.
      Moreover, 
      $\Ho \V (A,B) = \V(A_c, B_f)/\sim$.
\end{proposition}

We can use these cylinder objects to extend the notion of homotopy to $\V$-categories or $\V$-operads.





% ------------------------------ ASSEMBLING HOMOTOPIES ------------------------------


We may assemble homotopies in the following manner.

\begin{lemma}
      Suppose $\V$ is as in Convention \ref{ALLCOLOR_CONV} \footnote{
        We don't need that the unit is cofibrant.}.
        % has cofibrant symmetric pushout powers and cellular fixed points.
      Let $h: Z_1 \to Z_2$ in $\V$ be a (trivial) cofibration between cofibrant objects.
      Then $h^{\otimes n}: Z_1^{\otimes n} \to Z_2^{\otimes n}$ is a genuine (trivial) cofibration in $\V^{\Sigma_n}_{gen}$.
\end{lemma}
\begin{proof}
      We apply arguments from \cite[Prop. 6.24]{BP_geo} and \cite[Lemma 4.8]{Pe16}).
      Given composable arrows $Z_0 \xrightarrow{g} Z_1 \xrightarrow{h} Z_2$ in $\V$,
      we denote by $Q^n(g), Q^n(h)$ the domains of $g^{\square n}, h^{\square n}$.
      There is a filtration of the box product of the composite $(hg)^{\square n}$ by a series of pushouts as on the left below
      \begin{equation}
            \label{HGBOX_EQ}
            \begin{tikzcd}
                  \bullet \arrow[d, "\Sigma_n \cdot_{\Sigma_{n-r} \times \Sigma_r} (g^{\square n-r} \square h^{\square r})"'] \arrow[r]
                  &
                  \bullet \arrow[d, "k_r"]
                  & &% ----------
                  Q^n(g) \arrow[r] \arrow[d, "g^{\square n}"']
                  &
                  Q^n(hg) \arrow[d, "k_0"]
                  \\
                  \bullet \arrow[r]
                  &
                  \bullet
                  & &% ----------
                  Z_1^{\otimes n} \arrow[r]
                  &
                  \bullet
            \end{tikzcd}
      \end{equation}
      for $0 \leq r \leq n$,
      built out of a filtration $P_0 \subseteq P_1 \subseteq \dots \subseteq P_n$ of the poset $P_n = (0 \to 1 \to 2)^{\times n}$,
      where $P_0$ is all tuples containing at least one 0, and
      $P_{r+1}$ is built from $P_r$ by adding all tuples with exactly $(n-r)$ 1-coordinates and $r$ 2-coordinates.
      In the zeroth stage \footnote{
        In the language of \cite[Lemma 4.8]{Pe16}, this is the map associated to the subsets
        $T = P_0$ and $\bar T = \sets{e}{e \leq (1,1,\dots,1)}$.}
      we have the pushout on the right in \eqref{HGBOX_EQ}.
      But when $Z_0 = \varnothing$, $Q^n(g) = Q^n(hg) = \varnothing$, and so
      $k_0: \varnothing \to Z_1^{\otimes n}$ and
      \[
            k_n k_{n-1} \dots k_0 = h^{\otimes n}: Z_1^{\otimes n} \to Z_2^{\otimes n}.
      \]

      Now, since the functor $G \cdot_H (-): \V^H \to \V^G$ sends genuine $H$-cofibrations to genuine $G$-cofibrations,
      the fact that $\V$ has cofibrant symmetric pushout powers implies that
      if $g$ is a cofibration and $h$ a (trivial) cofibration in $\V$,
      $k_r$ is a genuine cofibration in $\V^{\Sigma_n}$ for $r \geq 0$ (which is trivial if $r \geq 1$).

      Thus when $Z_0 = \varnothing$, we have that $h^{\otimes n}$ is a genuine (trivial) $\Sigma_n$-cofibration.
\end{proof}

\begin{lemma}
      \label{ASSEM_HOM_LEM}
      Suppose $\V$ is as in Convention \ref{ALLCOLOR_CONV}. %has cofibrant symmetric pushout powers and cellular fixed points, and cofibrant unit.
      If $\mathbb C$ is a cylinder, then so is each $\left(\mathbb C^{\otimes n}\right)^{\Lambda}$ for all $\Lambda \leq \Sigma_n$.
\end{lemma}
\begin{proof}
      By the above lemma,
      $1_\V = (1_\V)^{\otimes n} \xrightarrow{i_\epsilon} \mathbb C^{\otimes n}$
      is a genuine trivial $\Sigma_n$-cofibrations,
      and hence the result follows by Remark \ref{LEVEL_COF_REM} and the following composite,
      \[
            \begin{tikzcd}
                  1_\V \amalg 1_\V
                  =
                  1_\V^{\otimes n} \amalg 1_\V^{\otimes n}
                  =
                  (1_\V^{\otimes n})^{\Lambda} \amalg (1_\V^{\otimes n})^{\Lambda}
                  % \left((1_V \amalg 1_\V)^{\otimes n}\right)^{\Sigma_n}
                  % \arrow[r, tail]
                  % &
                  % \left((1_V \amalg 1_\V)^{\otimes n}\right)^\Lambda
                  \arrow[r, tail]
                  &
                  \left(\mathbb C^{\otimes n}\right)^\Lambda
                  \arrow[r, "\simeq"]
                  &
                  \left(1_V^{\otimes n}\right)^\Lambda = 1_\V.
            \end{tikzcd}
      \]
      where the first map is a cofibration since it factors through $((1_\V \amalg 1_\V)^{\otimes n})^\Lambda$,
      and the last map is a genuine $\Sigma_n$-weak equivalence by 2-out-of-3 $1_\V \xrightarrow{\simeq} \mathbb C^{\otimes n} \to 1_\V$.


      % ----------------------------------------------------------------------------------------------------
      % ---------- PROBLEM: In what I've written, $\mathbb C^{\otimes} \to \1_\V^{\otimes}$ is only a projective equivalence, not a genuine $\Sigma_n$-equivalence ----------
      %       % It suffices to show there exist
      %       % $1_\V \amalg 1_\V \rightarrowtail (\mathbb C^{\otimes n})^K$
      %       % and
      %       % $(\mathbb C^{\otimes n})^K \xrightarrow{\sim} (1_\V^{\otimes n})^K \simeq 1_\V$
      %       % (where the coherence axioms imply that $1_\V^{\otimes n}$ always has a trivial $\Sigma_n$-action).
      %       % 
      % We consider each structure map separately.
      
      % We note that, as $\otimes$ commutes with colimits in each variable,
      % $(1_\V \amalg 1_\V)^{\otimes n} \simeq \coprod_{\chi} 1_{\V,\chi}$
      % with $\chi$ running over all set maps $\underline{n} \to \set{0,1}$,
      % and $\Sigma_n$ acting by pre-composition on $\chi$.
      % Now, we have the composite
      % \begin{equation}
      % %       \begin{tikzcd}
      %             1_\V \amalg 1_\V = 1_{\V, 0} \amalg 1_{\V, 1}
      %             \simeq
      %             \left((1_\V \amalg 1_\V)^{\otimes n}\right)^{\Sigma_n}
      %             \longrightarrow
      %             (1_\V \amalg 1_\V)^{\otimes n}
      %             \longrightarrow
      %             \mathbb C^{\otimes n}
      % %       \end{tikzcd}
      % \end{equation}
      % where $i: \underline{n} \to \set{i} \into \set{0,1}$ is the constant map,
      % the first arrow is a genuine $\Sigma_n$-cofibration
      % as we may attach each $\Sigma_n$-orbit $\Sigma_n \chi$ individually via maps $\varnothing \to \amalg_{\V,\sigma\chi}$,
      %       % (and we've already attached the stable orbits).
      % and the second arrow is a genuine $\Sigma_n$-cofibration since $\V$ has cofibrant symmetric pushout powers.
      
      % Now,
      % we note that $(\mathbb C \to 1_\V)^{\otimes n}$ is a weak equivalence by induction using Ken Brown's lemma,
      % as $\mathbb C^{\otimes n}$ is cofibrant,
      % $\mathbb C^{\otimes n} \to 1_\V^{\otimes n}$ is a map between cofibrant objects,
      % and $\mathbb C \otimes (-)$ preserves all trivial cofibrations.
      %       % 
      % Let $Q(n) \to \mathbb C^{\otimes n}$ denote $(1_\V \to \mathbb C)^{\square n}$,
      % which is a genuine trivial cofibration in $\V^{\Sigma_n}$ by the assumption on $\V$.
      % % Consider the pushout $P$ and induced maps in $\V^{\Sigma_n}$ below
      % \begin{equation}
      %       \begin{tikzcd}
      %             Q(n) \arrow[r, tail, "\sim"] \arrow[d, tail, "\sim"']
      %             &
      %             \mathbb C^{\otimes n} \arrow[d, tail, "\sim"] \arrow[drr, bend left, "\sim"]
      %             \\
      %             \mathbb C^{\otimes n} \arrow[r, tail, "\sim"] \arrow[rrr, bend right, "\sim"]
      %             &
      %             P \arrow[r, tail, dashed, "\sim"]
      %             &
      %             P' \arrow[r, two heads, "\sim", dashed]
      %             &
      %             1_\V^{\otimes n}
      %       \end{tikzcd}
      % \end{equation}
      % where we have factored the unique map $P \to 1_\V^{\otimes n}$ into a cofibration and fibration.
      
      % Thus, as (the proof of) \cite[Prop 6.3]{BP_geo} shows that $(-)^H$ preserves pushouts over genuine cofibrations
      % as well as genuine \textit{trivial} cofibrations,
      % and since $(-)^H$ preserves trivial fibrations by construction,
      % we have the string of maps in $\V$ below for any $K \leq \Sigma_n$
      % \begin{equation}
      %       \begin{tikzcd}
      %             1_\V \amalg 1_\V \simeq \left((1_\V \amalg 1_\V)^{\otimes n}\right)^{\Sigma_n} \arrow[r, hookrightarrow]
      %             &
      %             \left((1_\V \amalg 1_\V)^{\otimes n}\right)^K \arrow[r, hookrightarrow]
      %             &
      %             (\mathbb C^{\otimes n})^K \arrow[r, tail, "\sim"]
      %             &
      %             P^K \arrow[r, tail, "\sim"]
      %             &
      %             P'^K \arrow[r, two heads, "\sim"]
      %             &
      %             (1_\V^{\otimes n})^K \simeq 1_\V.
      %       \end{tikzcd}
      % \end{equation}
\end{proof}


% ----------------------------------------------------------------------------------------------------
% ------------------------------ diagonals for cylinder objects [incorrect? certainly not needed] ----------

% \begin{definition}
%       The category $\V$ is said to \textit{have diagonals for cylinder objects} if
%       for any cylinder object $\mathbb C$ and $n \geq 0$ there exists a map
%       \begin{equation}
%             \Delta: \mathbb C \to \left(\mathbb C^{\otimes n}\right)^{\Sigma_n}
%       \end{equation}
%       such that the following diagram commutes.
%       \begin{equation}
%             \begin{tikzcd}
%                   1_\V \amalg 1_\V \arrow[r] \arrow[d]
%                   &
%                   \left((1_\V \amalg 1_\V)^{\otimes n}\right)^{\Sigma_n} \arrow[d]
%                   \\
%                   \mathbb C \arrow[r, "\Delta"]
%                   &
%                   \left(\mathbb C^{\otimes n}\right)^{\Sigma_n}
%             \end{tikzcd}
%       \end{equation}
% \end{definition}

% This implies the diagram below commutes for all $k \in \underline{n}$ (since $\mathbb C$ factors the fold map).
% \begin{equation}
%       \begin{tikzcd}
%             1_\V \arrow[rrr, "i_k"] \arrow[d, "i_k"]
%             &&&
%             \mathbb C
%             % &&&
%             % 1_\V \arrow[lll, "i_1"'] \arrow[d, "i_1"]
%             \\
%             \mathbb C \arrow[r, "\Delta"]
%             &
%             \mathbb C^{\otimes n} \arrow[r]
%             &
%             1_\V^{\otimes k-1} \otimes \mathbb C \otimes 1_\V^{\otimes n-k} \arrow[r, "\simeq"]
%             &
%             \mathbb C \arrow[u, equal]
%             % &
%             % \mathbb C \otimes 1_\V^{\otimes n-1} \arrow[l, "\simeq"']
%             % &
%             % \mathbb C^{\otimes n} \arrow[l]
%             % &
%             % \mathbb C \arrow[l, "\Delta"']
%       \end{tikzcd}
% \end{equation}

% It suffices to show the map $\mathbb C^{\otimes n} \to 1_\V^{\otimes n} \simeq 1_\V$ is a trivial $\Sigma_n$-fibration, as a lift of
% \begin{equation}
%       \begin{tikzcd}
%             1_\V \amalg 1_\V \arrow[r] \arrow[d, tail]
%             &
%             \left((1_\V \amalg 1_\V)^{\otimes n}\right)^{\Sigma_n} \arrow[r]
%             &
%             \left(\mathbb C^{\otimes n}\right)^{\Sigma_n} \arrow[d]
%             \\
%             \mathbb C \arrow[rr, two heads, "\simeq"]
%             &&
%             1_\V \simeq 1_\V^{\otimes n} \simeq \left(1_\V^{\otimes n}\right)^{\Sigma_n}
%       \end{tikzcd}
% \end{equation}
% would satisfy these properties.
% \todo[inline]{come back: this need not happen. It may only be a weak equivalence.}
% 
% } % END OF OLIVE GREEN





\subsection{2-out-of-3}

In this section, we prove Proposition \ref{2OUTOF3_PROP}, that weak equivalences satisfy 2-out-of-3.
%
The main ingredient comes from comparing and applying different notions of when two objects are ``homotopically equivalent'', as suggested in the introduction to this section.

{\color{red} HERE}





Analgous results hold equivariantly, as we will show below. For our group $G$, we define the following:
\begin{definition}
      \label{EQUIVG_DEF}
      Fix $H \leq G$, $\mathcal{C}\in \Cat^G(\V)$, and $a,b\in \mathrm{Ob}(\mathcal{C})^H$.
      We say $a$ and $b$ are 
      \textit{(virtually, homotopy) $H$-equivalent}
      if they are (resp. virtually, homotopy) equivalent in $\mathcal{C}^H$;
      % Two options for virtually equivalent:
      % (i) they are virtually equivalent in $\mathcal C^H$
      % (ii) they are $H$-equivalent in some fibrant replacement $\mathcal C_f$ of $\mathcal C$
      % in $\Cat^{G, \mathrm{Ob}(\mathcal C)}(\V)$.
      
      For a $G$-operad $\O\in \mathsf{Op}^G(\V)$ and $a,b\in \C(\O)^H$, we say $a$ and $b$ are
      {\em (virtually, homotopy) $H$-equivalent}
      if they are so in the underlying category $j^*\O$. 
\end{definition}


\begin{remark}
      \label{ESS_SUR_REM}
      Unraveling definitions, we see
      $F: \O \to \P$ in $\Op^G(\V)$ is essentially $\F$-surjective iff
      for all $H \in \F_1$ and any $b \in \P^H$ there exists $a \in \O^H$ such that $F(a)$ and $b$ are $H$-equivalent.
\end{remark}

We can now record the important steps of the proof of 2-out-of-3, broken into the three distinct cases.
%splits into the three cases. For the first, only preservation of local weak equivalences is non-trival, while for the second and third, the proofs relies on a deep understanding of essential surjectivity.
%We describe the important steps in each case.
\begin{description}
\item [$F$ and $LF$ implies $L$:] This relies on Proposition \ref{CAV_4.14_PROP2}, which shows that
      homotopy equivalence colors induce weak equivalences between certain associated mapping objects,
      and Lemma \ref{VIR_HTPY_LEM}, from which we conclude that equivalent colors are homotopy equivalent.
\item [$L$ and $LF$ implies $F$:] We would like that local equivalences reflect all equivalences, but Lemma \ref{REF_VIRT_LEM} says they only preserve the potentially weaker notion of virtual equivalence;
      however, Lemma \ref{RIGHTPROPER_LEM} says that these notions agree when $\V$ is right proper.
\item [$F$ and $L$ implies $LF$:] This will follow from the fact that equivalences are transitive (Lemma \ref{CAV_4.10_LEM}) and are preserved by functors.
\end{description}
%
Many of the results will follow immediately from (the proof of) their non-equivariant counterparts once the definitions have been established;
only Proposition \ref{CAV_4.14_PROP2} will require a more complex analysis.

We unpack these definitions with two quick remarks.

\begin{remark}
      \label{VE_CHOICE_REM}
      We note that by straightforward lifting and factoring arguments,
      virtual equivalence does not depend on the choice of fibrant replacement nor the choice of $\V$-interval.
      However, for naturality reasons we will almost exclusively use a functorial fibrant replacement.
      This does not cause any added difficult:
      If $a,b\in \mathcal C$ are virtually $H$-equivalent, %(which could more accurately be called ``$H$-virtually equivalent''),
      they are in fact virtually equivalent for $\mathcal C^H$,
      in that there exists a lift
      \begin{equation}
            \label{FIBFIX_LIFT_EQ}
            \begin{tikzcd}
                  \mathcal C^H \arrow[d, tail, "\sim"'] \arrow[r, "\sim"]
                  &
                  (\mathcal C_f)^H
                  \\
                  (\mathcal C^H)_f \arrow[ur, dashed, "\sim"']
            \end{tikzcd}
      \end{equation}
      and thus any equivalence in $(\mathcal C^H)_f$ induces an equivalence in $(\mathcal C_f)^H$.
      % This also follows from Remark \ref{VE_CHOICE_REM}, as both $(\mathcal C^H)_f$ and $(\mathcal C_f)^H$ are fibrant replacements for $\mathcal C^H$.
\end{remark}

\begin{remark}
      \label{HK_EQUIV_REM}
      We note that if $K \leq H \leq G$, then (virtually, homotopy) $H$-equivalent implies (virtually, homotopy) $K$-equivalent
      as we have functors
      \[
            j^{\**}(\O^H) \to j^{\**}(\O^K),
            \qquad
            j^{\**}(\O^H)_f \to j^{\**}(\O^K)_f,
            \qquad
            \pi_0(j^{\**}(\O^H)_f) \to \pi_0(j^{\**}(\O^K)_f).
      \]
      where the last two use a functorial fibrant replacement.
      % Further, if $\V$ has a fibrant replacement functor that commutes with taking fixed points for any subgroup of $G$,     
      % % (in which case the two definitions of virtually $H$-equivalent coincide)
      % then virtually (resp. homotopy) $H$-equivalent implies virtually (homotopy) $K$-equivalent,
      % as we would have an inclusion of categories
      % Moreover, this would imply that $a$ and $b$ are virtually $H$-equivalent iff
      % they are $H$-equivalent in some fibrant replacement $\mathcal C_f$ of $\mathcal C$ in $\Op^{G, \mathrm{Ob}(\mathcal C)}(\V)$.
\end{remark}


\begin{notation}
      As fixed points $(-)^\Gamma$ and fibrant replacement $(-)_f$ need not commute, we will write
      \begin{equation}
            \O_f(C)^\Gamma = (\O_f(C))^\Gamma,
            \qquad
            \O^\Gamma(C)_f = (\O(C)^{\Gamma})_f.
      \end{equation}
\end{notation}


Now, we begin by showing that these notions of equivalences are nested.
The following three lemmas follow exactly as in the non-equivariant setting,
by restricting to the categories $j^{\**}\O^H$ and using \eqref{FIBFIX_LIFT_EQ}.

\begin{lemma}
      [{cf. \cite[4.10]{Cav}}]
      \label{CAV_4.10_LEM}
      $H$-equivalence and virtual $H$-equivalence define equivalence relations on $\mathfrak C(\O)^H$.
\end{lemma}

% \begin{proof}
%       {\color{OliveGreen}
%         Follows exactly as in \textit{loc cite}; either version of virtual $H$-equivalence works.

%         $ $
        
%         Indeed,
%         symmetry follows from the transposition isomorphism $\tau^{\**}\J \to \J$.
        
%         Reflexivity follows from the composition $\I_c \to \I \to \mathcal C^H$,
%         \todo{either version (i) or (ii) works here}
%         $\mathcal C_f^H$
%         of cofibrant replacement followed by the map realizing the identity map on $a$.
        
%         Transitivity follows from the amalgamation of interval objects \cite[Cor. 1.16]{BM13}
%         by the following two claims.
%         First, for any maps of $G$-sets $f: A \to \mathfrak C(\O)$,
%         we have a canonical ``identity'' map $f^{\**}\O \to \O$.
%         Second, a chase through the adjunctions yields that
%         any pair of maps $h: \J \to \mathcal C$ and $h': \J' \to \mathcal C$
%         induces a map $h \** h' : \J \** \J' \to \mathcal C$ such that      
%         $(h \** h') i_0 = h i_0$ and $(h \** h') i_1 = h' i_1$.
%       }
%   \end{proof}

\begin{lemma}
      [{cf. \cite[4.12]{Cav}, \cite[2.10]{BM13}}]
      \label{RIGHTPROPER_LEM}
      If $\V$ is right proper, then two colors are virtually $H$-equivalent iff they are $H$-equivalent. 
\end{lemma}
% \begin{proof}
%       {\color{OliveGreen}
%         Need: virtual $H$-equivalent (i). Then it is an immediate consequence of \textit{loc cite}.
%       }
% \end{proof}

% For completeness (and clarity in \S \ref{DK_SEC}), we highlight and expand on the following proof from \cite{BM13}
% \todo[inline]{the only different really is the inclusion of the paragraph containing the diagram \eqref{J11_CYL_EQ} and the diagram itself,
% which make it clear why natural homotopy equivalences are useful. The rest of the proof is identical, if further unpacked here than in \textit{loc cite}.}



The key ideas in the proofs are:
the stacking of $\V$-intervals forms another $\V$-interval \todo{pushout trick and Cav4.14 - still need that $\mathbb J_0$ is a cofibrant category with one object},
``nice'' pullbacks in $\Cat(\V)$ preserve local equivalences which are bijective on objects, and
the two objects of any $\V$-interval $\mathbb J$ are isomorphic in $\pi_0(\mathbb J)$.

To finish off Case II, we will need that (local) weak equivalences preserve some notions of equivalence.

\begin{lemma}
      [{cf. \cite[4.11]{Cav}, \cite[2.9]{BM13}}]
      \label{REF_VIRT_LEM}
      Let $\F$ be a $(G, \Sigma)$-family. %Suppose the $G$-graph system $\F$ has units.
      Then local weak $\F$-equivalences reflect virtual $H$-equivalences for all $H \in \F_1$.
      Explicitly, for
      $a_0,a_1 \in \C(\O)^H$, and $F: \O \to \P$ in $\mathsf{Op}^G(\V)$ a local weak $\F$-equivalence,
      if $F(a_0)$ and $F(a_1)$ are virtually $H$-equivalent then so are $a_0$ and $a_1$.
\end{lemma}

\begin{proof}
      % {\color{OliveGreen}
      %   Need: virtual $H$-equivalent (i).
      % }
      %Since $\F$ has units, $\F_1$ contains all $H \leq G \times \Sigma_1$ (cf. \cite[Remark 4.50]{BP_geo}), and thus
      For all $H \in \F_1$,
      $j^{\**}F^H: j^{\**}\O^H \to j^{\**}\P^H$ is a local weak equivalence in $\Cat(\V)$.
      %
      The proof then follows as in \textit{loc cite}.
      % {\color{OliveGreen} % ------------------------------ OLIVE GREEN ------------------------------
      %   As in \cite{BM13}, we may build a fibrant replacement of $j^{\**}F^H$ which is a local trivial fibration in $\Cat(\V)$.
      %   \begin{equation}
      %         \begin{tikzcd}
      %               j^{\**} \O^H \arrow[r, "\sim_l"] \arrow[d, dashed, "\sim"']
      %               &
      %               F^{\**} (j^{\**}\P^H) \arrow[r, "\simeq_l", two heads] \arrow[d, "\sim"']
      %               &
      %               j^{\**} \P^H \arrow[d, "\sim"]
      %               \\
      %               j^{\**} (\O^H)_f \arrow[r, dashed, two heads]
      %               &
      %               F^{\**} (j^{\**}(\P^H)_f) \arrow[r, "\simeq_l", two heads]
      %               &
      %               j^{\**}(\P^H)_f
      %         \end{tikzcd}
      %   \end{equation}
      %   (where all designations $(-)_l$ on arrows denote ``local'').
      %   Thus we have a lift on the left in \eqref{REF_VIRT_EQ},
      %   where the bottom arrow realizes the virtual $H$-equivalence between $F(a_0)$ and $F(a_1)$,
      %   as such a lift in $\Cat(\V)$ is equivalent to a lift in $\Cat^{\set{0,1}}(\V)$
      %   after pulling the right vertical arrow back along the inclusion
      %   $a: \set{0,1} \to \set{a_0,a_1} \into \mathfrak C_\O^H$ as on the right (cf. \eqref{COLOR_SQ_EQ}),
      %   and this lift exists by locality.
      %   \begin{equation}
      %         \label{REF_VIRT_EQ}
      %         \begin{tikzcd}
      %               \1 \amalg \1 \arrow[r, "{(a_0,a_1)}"] \arrow[d, tail, "{(i_0, i_1)}"']
      %               &
      %               j^{\**}(\O^H)_f \arrow[d, two heads, "\sim_l"]
      %               &&
      %               \1 \amalg \1 \arrow[r, "{(a_0,a_1)}"] \arrow[d, tail, "{(i_0, i_1)}"']
      %               &
      %               a^{\**}j^{\**}(\O^H)_f \arrow[d, two heads, "\sim"]
      %               \\
      %               \J \arrow[r] \arrow[ur, dashed]
      %               &
      %               j^{\**}(\P^H)_f
      %               &&
      %               \J \arrow[r] \arrow[ur, dashed]
      %               &
      %               a^{\**} F^{\**} j^{\**}(\P^H)_f.
      %         \end{tikzcd}
      %   \end{equation}
      % } % ------------------------------ OLIVE GREEN ------------------------------
\end{proof}

Moving on to Case I, we now prove that a homotopy equivalence of colors induces a homotopy equivalence between certain hom-objects.
% 
We begin with some notation and a construction.
Let $\mathfrak C$ be a $G$-set of colors, and fix a $\mathfrak C$-signature $C = (c_1, \dots, c_n; c_0)$.
For each $0 \leq i \leq n$, let $\Aut_i(C) \leq \Aut(C) = \Aut_{G \ltimes \Sigma_{\mathfrak C}}(C)$ denote the subgroup of elements $(g, \sigma)$
such that $\sigma(i) = i$ (including $\Aut_0(C) = \Aut(C)$),
and let $H_i \leq G$ denote the $G$-projection of $\Aut_i(C)$.

Now, suppose $d_i$ and $c_i$ are $H_i$-homotopy equivalent.
We define a new $\mathfrak C$-signature by replacing $c_i$ and all of its ``$\Aut_i(C)$-orbits'' in $C$ with
the $\Aut_i(C)$-orbit of $d_i$.
Specifically, let $\lambda \subseteq \langle n \rangle = \set{0,1,2,\dots,n}$ denote the subset of all $j$ such that
there exists some $(k_j, \sigma_j) \in \Aut(C)$ such that $\sigma_j(i) = j$ (and hence $c_j = k_j c_i$).
% for each $j \in \lambda$, fix choices of such $(k_j, \sigma_j)$ (with $(k_i,\sigma_i) = (1,1)$).
Then for all $0 \leq j \leq n$, we define $d_j$ and $D$ by
\begin{equation}
      \label{DDEF_EQ}
      d_j =
      \begin{cases}
            k_j \cdot d_i \qquad & j \in \lambda
            \\
            c_j & \mbox{otherwise,}
      \end{cases}
      \qquad \qquad
      D = (d_1,\dots, d_n; d_0).
\end{equation}

\begin{remark}
      We note that these $d_j$, and hence $D$, do not depend on the choice of $(k_j,\sigma_j) \in \Aut(C)$:
      If $(k'_j, \sigma'_j) \in \Aut(C)$ also has $\sigma'_j(i) = j$, then
      $(k'_j, \sigma'_j) \cdot (k_j^{-1}, \sigma_j^{-1}) \in \Aut_i(C)$,
      so $k'_j k_j^{-1} \in \Aut_G(c_i) \mathop{\cap} \Aut_G(d_i)$,
      and thus $k'_j \cdot d_i = k_j \cdot d_i$.
\end{remark}


\begin{lemma}
      \label{AUTC_LEM}
      For $C$, $D$, $\lambda$, $H_i$, and $(k_j, \sigma_j)$ defined as above, we have the following:
      \begin{enumerate}[label = (\roman*)]
      \item $j \in \lambda$ iff $\pi(j) \in \lambda$ for all $(h,\pi) \in \Aut(C)$.
      \item $\Aut(C) \leq \Aut(D)$.
      \item $H_{\pi(i)} = h H_i h^{-1}$ for all $(h,\pi) \in \Aut(C)$.
      \item For any arrow
            $\alpha \colon 1_\V \to \O(c_i;d_i)^{H_i}$,
            and homotopy
            $H \colon \mathbb C \to \O(c_i;c_i)^{H_i}$
            in a $\mathfrak C$-colored $G$-operad $\O$,
            the maps
            \begin{equation}
                  \alpha_j \colon 1_\V \xrightarrow{\alpha} \O(c_i; d_i)^{H_i} \xrightarrow{ k_j \cdot (-) } \O(c_j; d_j)^{H_j},
                  \qquad \qquad
                  H_j \colon \mathbb C \xrightarrow{H} \O(c_i;c_i)^{H_i} \xrightarrow{k_j \cdot (-)} \O(c_j;c_j)^{H_j}
            \end{equation}
            are well-defined and indepedent of the choice of $(k_j, \sigma_j)$.
      \item For any $\alpha$ and $H$, the maps
            \begin{align*}
              \otimes_\lambda \alpha_j \colon & 1_\V \simeq \otimes_\lambda 1_\V \longto \otimes_\lambda \O(c_j; d_j)^{H_j},
              \\
              \otimes_\lambda H_j \colon & \otimes_\lambda \mathbb C \longto \otimes_\lambda \O(c_j;c_j)^{H_j},
              \\
              (\alpha_j)_{\**} \colon & \O(C) \xrightarrow{\otimes_\lambda \alpha_j} \O(C) \otimes \bigotimes_\lambda \O(c_j; d_j)^{H_j} \longto \O(D)
            \end{align*}
            are $\Aut(C)$-equivariant.
      \end{enumerate}
\end{lemma}
\begin{proof}
      These all follow more-or-less immediately from the same straightforward calculation:
      Fixing $(h,\pi) \in \Aut(C)$ and $j \in \lambda$, we have
      \begin{equation}
            (k_{\pi(j)}^{-1} h k_j, \sigma_{\pi(j)}^{-1} \pi \sigma_j) \in \Aut_i(C),
            \qquad
            \textrm{and}
            \qquad
            (h,\pi) \Aut_i(C) (h,\pi)^{-1} = \Aut_{\pi(i)}(C).
      \end{equation}
\end{proof}

\begin{proposition}[{c.f. \cite[Prop. 4.14]{Cav}, \cite[Prop. 2.12]{BM13}}]
      \label{CAV_4.14_PROP2}
      Suppose $\V$ is as in Convention \ref{ALLCOLOR_CONV}. %has cofibrant symmetric pushout powers and cofibrant unit.
      Let $\O \in \Op^G(\V)$ with color $G$-set $\mathfrak C$,
      fix a $\mathfrak C$-signature $C = (c_1,\dots, c_n;c_0)$,
      and let $H_i$ be defined as in \eqref{DDEF_EQ}.
      %
      Suppose $d_i$ and $c_i$ are $H_i$-homotopy equivalent.
      Then there is a zig-zag, natural in functors out of $\O$,
      of genuine $\Aut(C)$-weak equivalences between $\O(C)$ and $\O(D)$,
      with $D$ the $\mathfrak C$-signature defined as in \eqref{DDEF_EQ}.
\end{proposition}
\begin{proof}
      Without loss of generality, $i = 1$ or $i = 0$.
      We only consider $i=1$; the case $i = 0$, $D = (c_1, \dots, c_n; d_0)$ will follow by an analogous (and simpler) argument.

      First, we note that using appropriate lifts (and \eqref{FIBFIX_LIFT_EQ}), one can show that if $c$ and $d$ are $H$-homotopy equivalent in $\O$, they are
      homotopy $H$-equivalent in the functorial bifibrant replacement $\O_{fc}$ of $\O$ in $\Op^{G, \mathfrak C(\O)}_{gen}(\V)$,
      the $\F$-(semi)-model structure from Theorem \ref{THM1_C} for $\F = \set{\F_n}$ with $\F_n$ all subgroups of $G \times \Sigma_n$.
      %
      Moreover, for any functor $F$, we have the following commuting diagram of zig-zags
      \begin{equation}
            \begin{tikzcd}
                  \O(C) \arrow[r, "\simeq"] \arrow[d, "F"']
                  &
                  \O_f(C) \arrow[d]
                  &
                  \O_{fc}(C) \arrow[l, "\simeq"'] \arrow[r, "{(\alpha_j)_{\**}}"] \arrow[d]
                  &
                  \O_{fc}(D) \arrow[r, "\simeq"] \arrow[d]
                  &
                  \O_f(D) \arrow[d]
                  &
                  \O(D) \arrow[l, "\simeq"'] \arrow[d, "F"]
                  \\
                  \P(C) \arrow[r, "\simeq"]
                  &
                  \P_f(C)
                  &
                  \P_{fc}(C) \arrow[l, "\simeq"'] \arrow[r, "{F(\alpha_j)_{\**})}"]
                  &
                  \P_{fc}(D) \arrow[r, "\simeq"] 
                  &
                  \P_f(D) 
                  &
                  \P(D) \arrow[l, "\simeq"']
            \end{tikzcd}
      \end{equation}
      where the $\simeq$ denote weak equivalences in $\V^{\Aut(C)}_{gen}$.
      Thus it suffices to consider the case where $\O$ is bifibrant; the general case and naturality follow.

      Now, by assumption we have
      arrows $\alpha: 1_\V \to \O(c_1; d_1)^{H_1}$ and $\beta: 1_\V \to \O(d_1, c_1)^{H_1}$,
      and homotopies $H = H_{\beta\alpha,id}$ and $H_{\alpha\beta,id}$.
      Building off Lemma \ref{AUTC_LEM}$(v)$, we use Remark \ref{CYL_REM} and Lemma \ref{ASSEM_HOM_LEM} to assemble the homotopies $H_j = H_{\beta_j \alpha_j,id}$, yielding
      \begin{equation}
            (\beta_j \alpha_j)_{\**} = (\beta_j)_{\**} (\alpha_j)_{\**} \sim id_{\O(C)^\Gamma}
            \qquad
            \textrm{and}
            \qquad
            (\alpha_j \beta_j)_{\**} = (\alpha_j)_{\**} (\beta_j)_{\**} \sim id_{\O(D)^\Gamma}
      \end{equation}
      in $\V$ for all subgroups $\Gamma \leq \Aut(C)$ via the diagram
      \begin{equation}
            \begin{tikzcd}
                  \O(C)^\Gamma \otimes (1_\V \amalg 1_\V) \arrow[d, tail] \arrow[rr, "{((\beta_i)_{\**}(\alpha_i)_{\**}, id)}"]
                  &&
                  \O(C)^\Gamma
                  \\                  
                  \O(C)^\Gamma \otimes \left(\bigotimes\limits_\lambda \mathbb C\right)^{\Gamma}
                  \arrow[r, "\otimes_\lambda H_j"]
                  &
                  \O(C)^\Gamma \otimes \left(\bigotimes\limits_\lambda \O(c_j;c_j)^{H_j}\right)^{\Gamma} \arrow[r]
                  &
                  \left( \O(C) \otimes \bigotimes\limits_\lambda \O(c_j;c_j) \right)^{\Gamma} \arrow[u, "\circ"']
            \end{tikzcd}
      \end{equation}
      and it's partner switching the order of $\beta$ and $\alpha$.      
      Hence, as both $\O(C)^\Gamma$ and $\O(D)^\Gamma$ are bifibrant in $\V$ by Remark \ref{LEVEL_COF_REM},
      \[
            (\alpha_j)_{\**}: \O(C)^\Gamma \leftrightarrows \O(D)^\Gamma : (\beta_j){\**}
      \]
      are inverse isomorphisms in $\Ho(\V)$ for all $\Gamma \leq \Aut(C)$,
      and thus $(\alpha_j)_{\**}$ is a weak equivalence in $\V^{\Aut(C)}_{gen}$.

      
      % % %%%%%%%%%%%%%%%%%%%%%%%%%%%%%%%%%%%%%%%%%%%%%%%%%%%%%%%%%%%%%%%%%%%%%%%%%%%%%%%%%%%%%%%%%%%
      % % ---------------------------------------- OLD PROOF ----------------------------------------
      % First, we have some straightforward calculations. Fix $(h,\pi) \in \Aut(C)$, so $h c_i = c_{\pi(i)}$ for all $0 \leq i \leq n$.
      % %
      % Then $j \in \lambda$ iff $\pi(j) \in \lambda$.
      % Moreover, additionally fixing $j \in \lambda$, we have
      % \begin{equation}
      %       \label{STAB_CONJ_EQ}
      %       (k_{\pi(j)}^{-1} h k_j, \sigma_{\pi(j)}^{-1} \pi \sigma_j) \in \Aut_1(C),
      %       \qquad
      %       \textrm{and}
      %       \qquad
      %       (h,\pi) \Aut_1(C) (h,\pi)^{-1} = \Aut_{\pi(1)}(C).
      % \end{equation}
      % % \begin{enumerate}[label = (\roman*)]
      % % \item 
      % %       {\color{OliveGreen} % ----------------------------------------
      % %       since
      % %       \[
      % %             k_{\pi(j)}^{-1} h k_j c_i = k_{\pi(j)}^{-1} h c_{\sigma_j(i)} = k_{\pi(j)}^{-1} c_{\pi \sigma_j(i)} = c_{\sigma_{\pi(j)}^{-1} \pi \sigma_j(i)},
      % %             \qquad
      % %             \sigma_{\pi(j)}^{-1} \pi \sigma_j(1) = \sigma_{\pi(j)}^{-1}(\pi(j)) = 1.
      % %       \] %
      % %     } % ------------------------------------------------------------
      % % \item $\Stab_j(C) = (k_j, \sigma_j) \Stab_1(C) (k_j, \sigma_j)^{-1}$,
      % %       {\color{OliveGreen} % ------------------------------------------
      % %       as when $(h,\pi) \in \Stab_1(C)$,

      % %       \[
      % %             \sigma_j \pi \sigma_j^{-1}(j) = \sigma_j\pi(1) = \sigma_j(1) = j.
      % %       \]
      % %     } % ------------------------------------------------------------
      % % \end{enumerate}
      % %
      % From these, it is easy to see that $\Aut(C) \leq \Aut(D)$,
      % % {\color{OliveGreen} % ----------------------------------------
      % %   since
      % %   \[
      % %         h d_j = h k_j d_1 = k_{\pi(j)} k_{\pi(j)}^{-1} h k_j d_1 = k_{\pi(j)} d_1 = d_{\pi(j)},
      % %   \]
      % %   for $j \in \lambda$ and otherwise $h d_j = h c_j = c_{\pi(j)} = d_{\pi(j)}$. 
      % % } % --------------------------------------------------
      % and that $H_j = k_j H_1 k_j^{-1}$ for all $j \in \lambda$,
      % and more generally $H_{\pi(1)} = h H_1 h^{-1}$ for all $(h,\pi) \in \Aut(C)$.

      % % ---------- bifibrant case ----------------------------------------
      % Second, let us assume that $\O$ is bifibrant in $\Op^{G, \mathfrak C(\O)}_{gen}(\V)$,
      % the $\F$-(semi)-model structure from Theorem \ref{THM1_C} for $\F = \set{\F_n}$ with $\F_n$ all subgroups of $G \times \Sigma_n$.
      % By assumption, we have maps 
      % $\alpha_1: 1_\V \to \O(c_1;d_1)^{H_1}$ and $1_\V \to \O(d_1;c_1)^{H_1}$,
      % and homotopies $H_{\beta\alpha,1}$ and $H_{\alpha\beta,1}$.
      % Action by $k_j$ and \eqref{STAB_CONJ_EQ} yield a well-defined map
      % \[
      %       \alpha_j: 1_\V \xrightarrow{\alpha} \O(c_1;d_1)^{H_1} \xrightarrow{k_j \cdot (-)} \O{fc}(c_j; d_j)^{H_j},
      % \]
      % and similarly well-defined $\beta_j$,
      % and naturality implies that the pair $(\alpha_j,\beta_j)$ realize a homotopy $H_j$-equivalence
      % via
      % \[
      %       H_{\beta_j\alpha_j,1} = k_j \cdot H_{\beta\alpha,1}
      %       \qquad \textrm{and} \qquad
      %       H_{\alpha_j \beta_J, 1} = k_j \cdot H_{\alpha\beta,1}.
      % \]
      % All together, \eqref{STAB_CONJ_EQ} also implies that $\otimes_\lambda \O(c_j; c_j;)^{H_j}$ has an action by $\Aut(C)$
      % (and similarly replacing either/both $c_j$ with $d_j$),
      % and moreover that the products
      % \[
      %       \otimes_\lambda \alpha_j: 1_\V \simeq \otimes_\lambda 1_\V \longto \otimes_\lambda \O(c_j; d_j)^{H_j},
      %       \qquad
      %       H_\lambda := \otimes_\lambda H_{\beta_j\alpha_j, 1}: \otimes_\lambda \mathbb C \longto \otimes_\lambda \O(c_j;c_j)^{H_j},
      % \]
      % their induced composition operations
      % \[
      %       (\alpha_j)_{\**}: \O(C) \simeq \O(C) \otimes \bigotimes_\lambda 1_\V \xrightarrow{\alpha_j}
      %       \O(C) \otimes \bigotimes_\lambda \O(c_j; d_j)^{H_j} \longto \O(D),
      % \]
      % and their companions $\otimes \beta_j$, $\otimes H_{\alpha_j\beta_j,1}$, $(\beta_j)_{\**}$
      % are $\Aut(C)$-equivariant.

      % Using Remark \ref{CYL_REM} and Lemma \ref{ASSEM_HOM_LEM}, we assemble the homotopies $H_{\beta_j \alpha_j,1}$ to yield
      % \begin{equation}
      %       (\beta_j \alpha_j)_{\**} = (\beta_j)_{\**} (\alpha_j)_{\**} \sim id_{\O(C)^\Gamma}
      %       \qquad
      %       \textrm{and}
      %       \qquad
      %       (\alpha_j \beta_j)_{\**} = (\alpha_j)_{\**} (\beta_j)_{\**} \sim id_{\O(D)^\Gamma}
      % \end{equation}
      % in $\V$ for all subgroups $\Gamma \leq \Aut(C)$, via the diagram
      % \begin{equation}
      %       \begin{tikzcd}[column sep = small]
      %             \O(C)^\Gamma \otimes (1_\V \amalg 1_\V) \arrow[d, tail] \arrow[rr, "{((\beta_i)_{\**}(\alpha_i)_{\**}, id)}"]
      %             &&
      %             \O(C)^\Gamma
      %             \\                  
      %             \O(C)^\Gamma \otimes \left(\bigotimes\limits_\lambda \mathbb C\right)^{\Gamma}
      %             \arrow[r, "H_\lambda"]
      %             &
      %             \O(C)^\Gamma \otimes \left(\bigotimes\limits_\lambda \O(c_j;c_j)^{H_j}\right)^{\Gamma} \arrow[r]
      %             &
      %             \left( \O(C) \otimes \bigotimes\limits_\lambda \O(c_j;c_j) \right)^{\Gamma} \arrow[u, "\circ"']
      %       \end{tikzcd}
      % \end{equation}
      % and it's partner switching the order of $\beta$ and $\alpha$.      
      % Hence, as both $\O(C)^\Gamma$ and $\O(D)^\Gamma$ are bifibrant in $\V$ by Remark \ref{LEVEL_COF_REM},
      % \[
      %       (\alpha_j)_{\**}: \O(C)^\Gamma \leftrightarrows \O(D)^\Gamma : (\beta_j){\**}
      % \]
      % are inverse isomorphisms in $\Ho(\V)$ for all $\Gamma \leq \Aut(C)$,
      % and thus $(\alpha_j)_{\**}$ is a weak equivalence in $\V^{\Aut(C)}_{gen}$.

      % % ---------- general case --------------------------------------------------
      % Third, for the general case, 
      % we note that using appropriate lifts (and \eqref{FIBFIX_LIFT_EQ}), one can show that if $c$ and $d$ are $H$-homotopy equivalent in $\O$, they are
      % homotopy $H$-equivalent in the functorial bifibrant replacement $\O_{fc}$ of $\O$ in $\Op^{G, \mathfrak C(\O)}_{gen}(\V)$.
      % Moreover, for any functor $F$, we have the following commuting diagram of zig-zags
      % \begin{equation}
      %       \begin{tikzcd}
      %             \O(C) \arrow[r, "\simeq"] \arrow[d, "F"']
      %             &
      %             \O_f(C) \arrow[d]
      %             &
      %             \O_{fc}(C) \arrow[l, "\simeq"'] \arrow[r, "{(\alpha_j)_{\**}}"] \arrow[d]
      %             &
      %             \O_{fc}(D) \arrow[r, "\simeq"] \arrow[d]
      %             &
      %             \O_f(D) \arrow[d]
      %             &
      %             \O(D) \arrow[l, "\simeq"'] \arrow[d, "F"]
      %             \\
      %             \P(C) \arrow[r, "\simeq"]
      %             &
      %             \P_f(C)
      %             &
      %             \P_{fc}(C) \arrow[l, "\simeq"'] \arrow[r, "{F(\alpha_j)_{\**})}"]
      %             &
      %             \P_{fc}(D) \arrow[r, "\simeq"] 
      %             &
      %             \P_f(D) 
      %             &
      %             \P(D) \arrow[l, "\simeq"']
      %       \end{tikzcd}
      % \end{equation}
      % where the $\simeq$ denote weak equivalences in $\V^{\Aut(C)}_{gen}$.
      % Thus the result for general $\O$ follows from the bifibrant case,
      % and naturality is clear.

      
      % % %%%%%%%%%%%%%%%%%%%%%%%%%%%%%%%%%%%%%%%%%%%%%%%%%%%%%%%%%%%%%%%%%%%%%%%%%%%%%%%%%%%%%%%%%%
      % % -------------------- $i = 0$ case --------------------------------------------------
      % The case $i = 0$, $D = (c_1, \dots, c_n; d_0)$ follows by an analogous (simpler) argument.
      % % Using appropriate lifts (and \eqref{FIBFIX_LIFT_EQ}), one can show that if $c$ and $d$ are $H$-homotopy equivalent in $\O$, they are
      % % homotopy $H$-equivalent in the functorial bifibrant replacement $\O_{fc}$ of $\O$ in $\Op^{G, \mathfrak C(\O)}_{gen}(\V)$,
      % % the $\F$-(semi)-model structure from Theorem \ref{THM1_C} for $\F = \set{\F_n}$ with $\F_n$ all subgroups of $G \times \Sigma_n$.
      % % Thus we may assume we have representing maps
      % % $\alpha: 1_\V \to \O_{fc}(c;d)^H$ and $1_\V \to \O_{fc}(d;c)^H$.
      % % Now, for any functor $F$, we have the following commuting diagram of zig-zags
      % % \begin{equation}
      % %       \begin{tikzcd}
      % %             \O(C) \arrow[r, "\simeq"] \arrow[d, "F"']
      % %             &
      % %             \O_f(C) \arrow[d]
      % %             &
      % %             \O_{fc}(C) \arrow[l, "\simeq"'] \arrow[r, "\alpha_{\**}"] \arrow[d]
      % %             &
      % %             \O_{fc}(D) \arrow[r, "\simeq"] \arrow[d]
      % %             &
      % %             \O_f(D) \arrow[d]
      % %             &
      % %             \O(D) \arrow[l, "\simeq"'] \arrow[d, "F"]
      % %             \\
      % %             \P(C) \arrow[r, "\simeq"]
      % %             &
      % %             \P_f(C)
      % %             &
      % %             \P_{fc}(C) \arrow[l, "\simeq"'] \arrow[r, "{F(\alpha_{\**})}"]
      % %             &
      % %             \P_{fc}(D) \arrow[r, "\simeq"] 
      % %             &
      % %             \P_f(D) 
      % %             &
      % %             \P(D) \arrow[l, "\simeq"']
      % %       \end{tikzcd}
      % % \end{equation}
      % % where the $\simeq$ denote weak equivalences in $\V^{\Aut(C)}_{gen}$, and $\alpha_{\**}$ is the composite
      % % \[
      % %       \O_{fc}(C) \xrightarrow{\alpha} \O_{fc}(C) \otimes \O_{fc}(c,d) \longto \O_{fc}(D).
      % % \]
      % % Thus, it suffices to check that $\alpha_{\**}$ is a genuine $\Aut(C)$-weak equivalence when $\O$ is bifibrant in $\Op^{G,\mathfrak C(\O)}_{gen}(\V)$.

      % % Let $\beta: 1_V \to \O(d,c)^H$ denote the $H$-homotopy inverse of $\alpha$,
      % % and $H_{\beta\alpha,1}$, $H_{\alpha\beta,1}$ the associated homotopies.
      % % Naturality implies that we have composition maps of the form
      % % \[
      % %       \O(C)^\Gamma \otimes \O(c,d)^{H_\Gamma} \longto \left( \O(C) \otimes \O(c,d) \right)^\Gamma \xrightarrow{\circ} \O(D)^\Gamma
      % % \]
      % % for all $\Gamma \leq \Stab(C)$, with $H_\Gamma \leq G$ the $G$-projection,
      % % and so in particular the maps $\alpha_{\**}$ and $\beta_{\**}$ are $\Aut(C)$-equivariant.
      % % Thus $H_{\beta\alpha,1}$ and $H_{\alpha\beta,1}$ restrict and induce homotopies
      % % \[
      % %       (\beta\alpha)_{\**} \sim id_{\O(C)^{\Gamma}}
      % %       \qquad \textrm{and} \qquad
      % %       (\alpha\beta)_{\**} \sim id_{\O(D)^\Gamma}
      % % \]
      % % in $\V$ via diagrams of the form, e.g.,
      % % \begin{equation}
      % %       \begin{tikzcd}
      % %             \O_{f c}(C)^{\Gamma} \otimes (1_\V \amalg 1_\V) \arrow[d, tail] \arrow[r, "{((\beta\alpha)_{\**}, id)}"]
      % %             &
      % %             \O_{f c}(C)^\Gamma
      % %             % &
      % %             % 1_\V \amalg 1_\V \arrow[d, tail] \arrow[rr, "{((\alpha\beta)_{\**}, id)}"]
      % %             % &&
      % %             % \V((\O(\ksi_c^d)_f)^\Gamma, (\O(\ksi_c^d)_f)^\Gamma)
      % %             \\                  
      % %             \O_{f c}(C)^\Gamma \otimes \mathbb C \arrow[r, "H_{\beta\alpha,1}"]
      % %             &
      % %             \O_{f c}(C)^\Gamma \otimes \O_{f c}(c;c)^{H_\Gamma} \arrow[u, "\circ"']
      % %             % &
      % %             % \mathbb C \arrow[r, "H_{\alpha\beta,1}"']
      % %             % &
      % %             % (\O(d;d)^{H_\Gamma})_f \arrow[r]
      % %             % &
      % %             % (\O(d;d)_f)^{H_\Gamma} \arrow[u, "{(-)_{\**}}"],
      % %       \end{tikzcd}
      % % \end{equation}
      % % Hence, as both $\O(C)^\Gamma$ and $\O(D)^\Gamma$ are bifibrant in $\V$ by Remark \ref{LEVEL_COF_REM},
      % % \[
      % %       \alpha_{\**}: \O(C)^\Gamma \leftrightarrows \O(D)^\Gamma : \beta_{\**}
      % % \]
      % % are inverse isomorphisms in $\Ho(\V)$ for all $\Gamma \leq \Aut(C)$,
      % % and thus $\alpha_{\**}$ is a weak equivalence in $\V^{\Aut(C)}_{gen}$, as desired.
\end{proof}

We can now prove the main result of this subsection.
 
\begin{proposition}
      [{c.f. \cite[4.15]{Cav}, \cite[2.13]{BM13}}]
      \label{CAV_4.15_PROP}
      \label{2OUTOF3_PROP}
      Suppose $\V$ is as in Convention \ref{ALLCOLOR_CONV} and additionally is right proper.
      Then for any $(G, \Sigma)$-family $\F$ such that $H \in \F_1$ for all $H \leq G$, %with units,
      the class of weak $\F$-equivalences in $\mathsf{Op}^G(\V)$ satisfies the 2-out-of-3 condition.
\end{proposition}
\begin{proof}
      Let $\O \xrightarrow{F} \P \xrightarrow{L} \Q$ be a composition of maps in $\mathsf{Op}^G(\V)$.
      If $F$ and $L$ are weak $\F$-equivalences,
      the composite is obviously a local weak $\F$-equivalence:
      $\O(C)^\Gamma \xrightarrow{\sim} \P(F(C))^\Gamma \xrightarrow{\sim} \Q(LF(C))^\Gamma$.
      Moreover, as functors preserve equivalences of colors, $L F$ is essentially $\F$-surjective by transitivity from Lemma \ref{CAV_4.10_LEM}. 
      % Moreover, concatenation of $\V$-intervals collapses the consecutive essential surjectivity diagrams on the left
      % onto the one on the right.
      % \begin{equation}
      %       \begin{tikzcd}[row sep = tiny]
      %             \1 \arrow[rr, dashed, "a"] \arrow[dr]
      %             &&
      %             j^{\**}\O^H \arrow[dd]
      %             \\
      %             & \J \arrow[dr, dashed]
      %             &&
      %             \1 \arrow[rr, "a", dashed] \arrow[dr]
      %             &&
      %             j^{\**}\O^H \arrow[dd]
      %             \\
      %             \1 \arrow[rr, dashed, "b"] \arrow[ur] \arrow[dr]
      %             &&
      %             j^{\**} \P^H \arrow[dd]
      %             &&
      %             \J \** \J' \arrow[dr, dashed]
      %             \\
      %             & \J' \arrow[dr, dashed]
      %             &&
      %             \1 \arrow[rr, "c"] \arrow[ur]
      %             &&
      %             j^{\**}\Q^H
      %             \\
      %             \1 \arrow[ur] \arrow[rr, "c"]
      %             &&
      %             j^{\**} \Q^H
      %       \end{tikzcd}
      % \end{equation}
      
      If $L$ and $FL$ are weak $\F$-equivalences,
      then $F$ is a local weak $\F$-equivalence by 2-out-of-3 in each $\V^{\Aut(C)}_{\F_C}$.
      Moreover, if $b \in \mathfrak C(\P)^H$ for $H \in \F_1$, then by Remark \ref{ESS_SUR_REM}, there exists $a \in \mathfrak C(\O)^H$ such that
      $LF(a)$ and $L(b)$ are (virtually) $H$-equivalent.
      Lemma \ref{REF_VIRT_LEM} then implies $F(a)$ and $b$ are virtually $H$-equivalent, 
      and since $\V$ is right proper, Lemma \ref{RIGHTPROPER_LEM} implies they are $H$-equivalent.

      Lastly, suppose $F$ and $LF$ are weak $\F$-equivalences.
      It is immediate that $L$ is essentially $\F$-surjective.
      Now, given a signature $C = (c_1,\ldots,c_n;c_0$) in $\C(\P)$,
      we define a partition on $\underline{n}_+ = \set{0,1,2,\dots,n}$ where
      $i < j$ are in the same class iff there exists $(k_{i,j}, \sigma_{i,j}) \in \Aut(C)$ such that
      $\sigma_{i,j}(i) = j$ (so in particular $c_j = k_{i,j} c_i$).
      It is easy to check that this gives a well-defined partition/equivalence relation.
      Let $R \subseteq \underline{n}_+$ denote the subset of minimal representatives in each class
      (so in particular $0 \in R$),
      and $H_r \leq H_0 \leq G$ the projection of $\Aut_r(C)$ onto $G$ for each $r \in R$.
      % Finally, fix choices of $k_{r,j}$ for all $r \in R$ and $j$ in the same class as $r$ (again with $k_{r,r} = 1$).
      
      Now, by the {\color{red} closure condition on} $\F_1$ and the essential surjectivity of $F$,
      for all $r \in R$ there exist $d_r \in \C(\O)^{H_r}$ such that
      $F(d_r)$ is (homotopy) $H_r$-equivalent to $d_r$.
      %
      Extend the set $\set{d_r}_{r\in R}$ to a signature $D = (d_1,\ldots, d_n;d_0)$
      by defining $d_j = k_{r,j} \cdot d_r$;
      as in Lemma \ref{AUTC_LEM}, these are independent of the choice of $(k_{r,j}, \sigma_{r,j})$.
      % Consequently, $F(c_i)$ is homotopy equivalent to $d_i$ via $k_{r,i}\gamma_r$,
      % where $\gamma_r$ realizes the homotopy equivalence between $F(c_r)$ and $d_r$ for $i \in \lambda_r$.
      This yields a diagram of the form
      \begin{equation}
            \label{TWOOFTHREE_EQ}
            \begin{tikzcd}
                  \O(d_1,\ldots, d_n;d_0) \arrow[r, "(1)"]
                  &
                  \P(F(d_1),\ldots, F(d_n); F(d_0)) \arrow[d,dash, "(3)"] \arrow[r, "(2)"]
                  &
                  \Q(LF(d_1),\ldots, LF(d_n);LF(d_0)) \arrow[d, dash, "(4)"]
                  \\
                  &
                  \P(c_1,\ldots, c_n;c_0) \arrow[r, "(5)"]
                  &
                  \Q(L(c_1),\ldots, L(c_n); L(c_0)).
            \end{tikzcd}
      \end{equation}
      $(1)$ is a weak equivalence in $\V^{\Aut(D)}_{\F_D}$ by assumption, and
      $(2)$ is a weak-equivalence in $\V^{\Aut(D)}_{\F_D}$ by 2-out-of-3 here.
      $(3)$ and $(4)$ are zig-zags of weak equivalences in $\V^{\Aut(C)}_{gen}$ by iterating applications of
      Proposition \ref{CAV_4.14_PROP2},
      as each application only changes the colors in a single partition class.
      As these zig-zags are functorial, the above diagram commutes.
      %
      Finally, $\Aut(C) \leq \Aut(D)$ by the same argument as in Lemma \ref{AUTC_LEM}.
      Thus $(5)$ is a weak equivalence in $\V^{\Aut(C)}_{\F_C}$ by 2-out-of-3, and hence
      $L$ is a local weak $\F$-equivalence, as desired.
\end{proof}







% ------------------------------- DWYER-KAN DESCRIPTION -----------------------------

\subsection{Dwyer-Kan equivalences}
\label{DK_SEC}



\begin{remark}
      If $\V$ has diagonals, then one can show that $F \in \Op^G(\V)$ is a $DK$-$\F$-equivalence iff
      the following seemingly strong condition holds:
      $F$ is a local weak $\F$-equivalence such that 
      the associated map of \textit{$\F$-genuine equivariant operads} under the composite
      \begin{equation}
            \Op^G(\V) \to \Op_\F(\V) \xrightarrow{\pi_0} \Op_\F(\Set) 
      \end{equation}
      is an equivalence.
\end{remark}



% % ------------------------------ other interesting results, need some of the discussion below ------------------------------

% {\color{OliveGreen} % -------------------- OLIVE GREEN --------------------
%   We have the following (non-equivariant) consequence of Lemma \ref{VIR_HTPY_LEM}.
%   \begin{lemma}
%         Suppose $\V$ is a cofibrantly generated monoidal model category such that $\Op(\V)$ has the transferred model structure.
%         \todo[inline]{what are the actual hypotheses?}
%         Suppose $F: \O \to \P_f$ is a weak equivalence in $\Op(\V)$, with $\P_f$ fibrant.
%         Then $\pi_0(F)$ is an equivalence of operads.
%   \end{lemma}
%   \begin{proof}
%         Since $\mathcal C(c,d) \to \mathcal D_f(F(c), F(d))$ is a weak equivalence in $\V$ for all objects $c,d \in \mathcal C$,
%         it certainly becomes an isomorphism in $\Ho(\V)$.
%         Moreover, the fact that objects being virtually equivalent implies they are homotopy equivalent says that
%         when the target is fibrant, any equivalence $\mathbb J \to \ D_f$ yields a homotopy equivalence $\mathbb I \to \pi_0(\mathcal D_f)$,
%         and hence essential surjectivity in $\Cat(\V)$ implies essential surjectivity at $\pi_0$.
%   \end{proof}
  
%   \begin{corollary}
%         For any $\mathcal \O \in \Op(\V)$, we have a natural equivalence of operads
%         $\pi_0(\mathcal \O) \to \pi_0(\mathcal \O_f)$. 
%   \end{corollary}
% }% ------------------------------ OLIVE GREEN ------------------------------



For the reverse direction (cf. \cite[\S 2]{BM13}), we need to show that
homotopy equivalences are all virtual equivalences.
This requires another condition on $\V$, namely that the homotopy equivalences all satisfy a ``coherence'' condition.
% originally due to Boardman-Vogt \cite{BV73}, and extended by Berger-Moerdijk \cite{BM13}.

\begin{definition}
      Recall the category $\mathbb A \in \Cat^{\set{0,1}}(\V)$ which detects arrows.
      A cofibration $\mathbb A \to \J$ in $\Cat^{\set{0,1}}(\V)$ into a $\V$-interval is called \textit{natural} if
      it fits into a commuting diagram of the following form in $\Cat^{\set{0,1}}(\V)$.
      \begin{equation}
            \begin{tikzcd}
                  \mathbb A \arrow[d, tail] \arrow[r]
                  &
                  \I \arrow[d, "\sim"]
                  \\
                  \J \arrow[r, "\sim"']
                  &
                  \I_f.
            \end{tikzcd}
      \end{equation}

      A homotopy equivalence between two objects in a $\V$-category $\mathcal C$ is called \textit{coherent} if
      the detecting map $\alpha: \mathbb A \to \mathcal C_f$ factors along a natural cofibration
      \begin{equation}
            \begin{tikzcd}
                  \mathbb A \arrow[r, "\alpha"] \arrow[d, tail, dashed]
                  &
                  \mathcal C_f
                  \\
                  \J \arrow[ur, dashed]
            \end{tikzcd}
      \end{equation}
      A monoidal model category $\V$ is said to satisfy the \textit{coherence axiom} if
      all homotopy equivalences in every $\V$-category are coherent.
\end{definition}

\begin{proposition}[{cf. \cite[Prop. 2.20]{BM13}}]
      \label{COH_DK_ARE_WE_PROP}
      If $\V$ is right proper with cofibrant unit satisfying the coherence axiom, then
      DK-$\F$-equivalences are weak $\F$-equivalences in $\Op^G(\V)$.
\end{proposition}
\begin{proof}
      It suffices to show that any homotopy $H$-equivalence between objects in some $\O \in \Op^G(\V)$
      is in fact an $H$-equivalence.
      % Now, the proof of Lemma \ref{VIR_HTPY_LEM} in fact shows that
      % any arrow in a $\V$-category $\mathcal C$ which factors through a natural cofibration $\mathbb A \to \mathbb J$ encodes a homotopy equivalence,
      % \todo[inline]{confirm this} 
      The coherence axiom implies that all homotopy $H$-equivalences are virtual $H$-equivalences,
      while right properness and Lemma \ref{RIGHTPROPER_LEM} imply these are actual $H$-equivalences.
\end{proof}

\begin{remark}
      \label{FIB_ISOFIB_REM}
      When $\V$ satisfies the coherence condition, we also have an additional nice description of fibrations:
      A map $F: \O \to \P$ in $\Op^G_\F(\V)$ is a fibration iff
      $F$ is a local $\F$-fibration such that
      $j^{\**}\pi_0(F^H)$ is an isofibration of 1-categories.
      Indeed, the arguments in \cite[Propositions 2.3 and 2.5]{Ber07b} extend as written to the general case,
      notationally replacing $\mathcal H$ with an arbitrary $\V$-interval $\mathbb J$ and
      $\mathscr F$ with $\mathbb A$.
\end{remark}


\begin{remark}
      \label{COH_EX_REM}
      We note that the coherence condition is a \textit{non-equivariant} assumption on $\V$.
      It has been proven in the literature for many categories:
      \begin{enumerate}[label = (\roman*)]
      \item The notion originated with Boardman-Vogt, who showed it holds for compactly-generated weak Hausdorff spaces $(\Top, \times)$ \cite[Lem. 4.16]{BV73}.
      \item Using a generalization of this argument, Berger-Moerdijk showed it holds for any category $\V$ which satisfies transfer for operads, is right proper, and has a cofibrant unit \cite[Prop. 2.24]{BM13}.
      \item As recreated in Proposition \ref{SSET_COH_PROP} below, Joyal showed it holds for $(\sSet, \times)$ with the Quillen model structure.
      \item This axiom is also a consequence of Lurie's \textit{invertibility hypothesis} \cite[A.3.2.12]{Lur09} by an argument of Berger-Moerdijk \cite[Rem. 2.19]{BM13}.
            This adds the example of $\sSet$ with the Joyal model structure, among others \cite[A.3.2.23]{Lur09}.
      \end{enumerate}
\end{remark}




\subsection{Equivariant operads of simplicial sets and spaces}
\label{SSETMS_SEC}

We interpret the previous sections in the case where our enriching category is $\sSet$, $\sSet_{\**}$, or $\Top$.

First, we identify some definitions.
\begin{enumerate}[label = (\roman*)]
\item $\eta = [0] = \**$, $\mathbb A = [1]$, while $\mathbb I = \tilde [1]$ is the walking isomorphism category; all are discrete.
\item  The prototypical example of a $\sSet$-interval is $W_!J$ where $J = N(0 \leftrightarrows 1) = N \I \in \sSet$ is the nerve of the walking isomorphism.
      In particular, in $\Cat^{\set{0,1}}(\sSet)$,
      $W_!(\set{0,1} \to J)$ is a cofibration and
      the derived counit $W_!J \to \I = \I_f$ is a weak equivalence.
\item The set $\mathscr G$ of $\sSet$-intervals with countably-many simplicies are generating by \cite[Lemmas 4.2,4.3]{Ber07b}.
\item The descriptions of equivariant Dwyer-Kan equivalences and fibrations from Definition \ref{DKEQUIV_DEF} and Remark \ref{FIB_ISOFIB_REM} generalize the similarly-named non-equivariant notions.
\item Following Example \ref{FREEOP_EX}, we see that
      when $\F = \F^\Gamma$ is the is the $(G, \Sigma)$-family of graph subgroups,
      generating cofibrations and trivial cofibrations of $\Op^G_{\F^\Gamma}(\sSet)$ are, respectively, 
      \begin{gather*}
            \set{\Omega(C) \otimes_\Fin(\partial \Delta[n] \to \Delta[n])}_{C \in \Sigma_G, n \geq 0} \bigcup 
            \set{G/H \cdot (\varnothing \to \Omega(\eta))}_{H \leq G}
            \stepcounter{equation}\tag{\theequation}\label{SSETGENCOF_EQ}
            \\
            \set{\Omega(C) \otimes_\Fin(\Lambda^i[n] \to \Delta[n])}_{C \in \Sigma_G, 0 \leq i \leq n} \bigcup
            \set{G/H \cdot (\Omega(\eta) \to \mathbb J)}_{H \leq G, \mathbb J \in \mathscr G}.
      \end{gather*}     
\end{enumerate}    

Finally, we prove the coherence condition for simplicial sets and pointed simplicial sets.
\begin{proposition}
      [{cf. \cite[\S 1]{Joy02}}]
      \label{SSET_COH_PROP}
      $(\sSet, \times)$
      satisfies the coherence axiom.
\end{proposition}
\begin{proof}
      Let $\mathcal C_f \in \Cat(\sSet)$ be (locally) fibrant, and
      suppose $\alpha: \mathbb A \to \mathcal C_f$ realizes a homotopy equivalence.
      Recall the Quillen equivalence $W_!\colon \sSet_{Kan} \rightleftarrows \Cat(\sSet) \, \colon \! h c N$.
      Then, as $W_!\Delta[1] = [1] = \mathbb A$, we have an adjoint map $\tilde \alpha: \Delta[1] \to h c N \mathcal C$,
      which realizes a quasi-isomorphism in the $\infty$-category $h c N \mathcal C$ since $\Ho(h c N (-)) \simeq \pi_0(-)$ on fibrant objects.
      Since $\Delta[1] \to J = N \I$ is anodyne by \cite[Corollary 1.6]{Joy02} or \cite[Lemma 0.15]{Rie}, 
      $\tilde \alpha$ factors through $J$,
      so the adjoint $\alpha$ factors through $W_!J$.
      The result then follows since $W_!J$ is a $\sSet$-interval.
      % $W_!J$ is cofibrant in $\sCat^{\set{0,1}}$ as we have a factorization $\set{0,1} \to \Delta[1] \to J$ and $W_!$ is left Quillen,
      % and the counit of the Quillen equivalence % $W_!\colon \sSet \rightleftarrows \sCat \ \colon \! h c N$
      % yields a weak equivalence $W_!J \xrightarrow{\simeq} [\tilde 1] = \mathbb I = \mathbb I_f$ in $\Cat^{\set{0,1}}(\sSet)$.
      % %This weak equivalence also holds since $J$ is contractible (as it models $S^\infty$), we're done.
\end{proof}

\todo[inline]{do we need Lemmas \ref{INTER_LEM} and \ref{WJ EX}? Or, for that matter, \ref{SSET_COH_PROP}? This is basically a recreation of the cited results using minimal new terminology.}


% In Appendix \ref{PT_SEC}, we prove Proposition \ref{PT_MODEL_COR}, of which the following is a special case.
\begin{corollary}
      \label{PTSSETCOH_COR}
      $(\sSet_{\**}, \wedge)$ satisfies the coherence axiom.
\end{corollary}
\begin{proof}
      The disjoint basepoint functor $(-)_+: \sSet \to \sSet_{\**}$ is alway left Quillen.
      Since additionally the unit in $\sSet$ is the terminal object and is cofibrant, and $\sSet$ is left proper,
      this functor also preserves all weak equivalences.
      {\color{OliveGreen} % ----------------------------------------
        \begin{equation}
              \begin{tikzcd}
                    A \arrow[r, rightarrowtail] \arrow[d, "\simeq"']
                    &
                    A \amalg \** \arrow[d, "\simeq"]
                    \\
                    B \arrow[r, rightarrowtail]
                    &
                    B \amalg \**
              \end{tikzcd}
        \end{equation}
      } % --------------------------------------------------
      Using this and the fact that the unit is additionally fibrant, we have that $\mathbb J_+$ is a $\sSet_{\**}$-interval for any $\sSet$-interval $\mathbb J$,
      and then an easy adjunction argument finishes the proof.
\end{proof}


%We now have collected all the remaining pieces to prove Theorem \ref{INTRO_MODEL_THM} from Theorem \ref{MODEL_THM}.
% \begin{proof}
%       [Proof of Theorem \ref{INTRO_MODEL_THM}]
%       Propositions \ref{WE_ARE_DK_PROP} and \ref{COH_DK_ARE_WE_PROP},
%       combined with Proposition \ref{SSET_COH_PROP} and Corollary \ref{PTSSETCOH_COR},
%       imply that when $\V$ satisfies the coherence axiom, 
%       the $\F$-(semi)-model structures from Theorem \ref{MODEL_THM} are in fact the $\F$-Dwyer-Kan model structures.
% \end{proof}






We have similarly nice results for $V = \Top$.
\begin{example}
      The category $\Top$ of compactly-generated weak-Hausdorff spaces satisfies all the hypotheses of Theorem \ref{INTRO_MODEL_THM}.
      Indeed, $(\Top, \times)$:
      \begin{enumerate}[label = (\roman*)]\itemsep-4pt
      \item is cofibrantly generated (e.g. \cite{Pia91});
      \item is a closed monoidal model category with cofibrant unit (e.g. \cite[Prop. 4.2.11]{Hov99});
      \item has cellular fixed points by \cite{Pia91} (see also e.g. \cite[Lemma 3.18]{Ste16});
      \item has cofibrant symmetric pushout powers since geometric realization is left Quillen and strong monoidal;
      \item is right proper;
      \item has a generating set of intervals, as all objects are fibrant (see \cite[Lemma 2.1]{BM13}); and
      \item is coherent by \cite[Lem. 4.16]{BV73}.
      \end{enumerate}
      Thus, combining Theorem \ref{MODEL_THM} with Remarks \ref{TOP_FULL_REM} and \ref{OPGCV_FULL_REM},
      $\Op^G(\Top)$ has the $\F$-Dwyer-Kan model structure for any $(G, \Sigma)$-family $\F$. % with units.

      However, as is often the case, pointed spaces can be poorly behaved unless one restricts to the category of well-pointed or closedly pointed spaces. % \todo{though these may still be weakly cellular, which could be enough}
\end{example}


% % ----------------------------------------------------------------------------------------------------
% {\color{OliveGreen} % ---------------------------------------- OLIVE GREEN --------------------
%   \begin{example}
%         Possible examples:
%         \begin{itemize}
%         \item $\Gamma$-spaces?
%         \item dg-modules with the projective model structure?
%         \item simplicial modules over a ring? \cite[\S 3.1.15]{Rez96}, \cite[Example 4.23]{Cav}
%         \end{itemize}
%   \end{example}

%   % \begin{example}
%   %       $R$ a commutative ring containing the rational numbers,
%   %       $\mathcal A$ the abelian category of projective $R$-modules,
%   %       and consider $Ch(\mathcal A)$ with projective model strucutre:

%   %       \begin{itemize}
%   %       \item cellularity in \cite{Ste16},
%   %       \item cofibrant symmetric pushout powers implied by ``freely powered'', proved in \cite[Prop 7.1.4/7]{Lur},
%   %       \item right proper?
%   %       \item generating set of intervals?
%   %       \item cofibrant unit?            
%   %       \end{itemize}

%   %       More generally, $\mathcal C$ locally presentable quasi-abelian category,
%   %       $R$ a commutative monoid object in $\mathcal C$ containing the rational numbers,
%   %       and consider $dg_R(\mathcal C)$ with the projective model structure \cite[Prop 2.12]{Wal15};
%   %       this also satisfies cofibrant symmetric pushout powers by \cite[Prop 3.4]{Wal15}.
%   % \end{example}

%   \begin{example}
%         Non-examples:
%         \begin{itemize}
%         \item $\Set$, $\Cat$, (none are cellular)
%         \item $\mathsf{Ch}$ (only weakly cellular - can we do something with this?)
%         \item any model for $(\infty,1)$-cats (none are right proper)
%         \end{itemize}
%   \end{example}

%   \begin{remark}
%         We say one note about the choice of conditions. There are several notions which have a similar form to
%         ``cofibrant symmetric pushout powers'' and play a similar role in transferring model structures:
%         freely powered of Lurie,
%         the commutative monoid axiom of White-Yau,
%         symmetric h-monodial of Pavlov-Scholbach,
%         and the cofibration hypothesis of Mandell-May-Schwede-Shipley,
%         to name just a few.

%         It is straightforward to check that cofibrant symmetric pushout powers is a (much) weaker condition than freely powered,
%         but slightly stronger than the commutative monoid axiom.
%         % The first is obvious.
%         % The second follows since the adjunction
%         % \[
%         %       G/G \cdot (-): \V \leftrightarrows \V^G: (-)/G
%         % \]
%         % is Quillen, as for all $A \in \V$ and $H \leq G$, $(G/H \cdot A)/G \cong A$
%         % (see also \cite[Lemma 4.5.4.11]{Lur})

%         \todo[inline]{connection to the others?}
%   \end{remark}

%   \todo[inline]{how much of \cite{WY18} extends via this arrangement?}

% } % ---------------------------------------- OLIVE GREEN ----------------------------------------









\subsection{Untitled subsection}


Our main examples of interest will be $(\V, \otimes) = (\sSet, \times)$ or $(\sSet_{\**}, \wedge)$.
We specify to these cases in Section \ref{SSETMS_SEC}.





\begin{remark}
      We could have instead chosen to follow \cite{Mur15}, and in fact
      fully expect that the analogous statements would hold true.
\end{remark}

\begin{convention}
      \label{ALLCOLOR_CONV}
      For this section, we assume $\V$ is always a cofibrantly generated closed monoidal model category with
      cofibrant unit, cellular fixed points, and cofibrant symmetric pushout powers.
      In particular, Theorem \ref{THM1_C} and Remark \ref{CATV_MC_REM} imply that $\Cat^{\mathfrak C}(\V)$ is a (semi)-model category for each set $\mathfrak C$,
      and thus in particular has cofibrant replacements.

      We additionally assume that $\V$ has a generating set of intervals.
\end{convention}



The model structure on $\Op^G(\V)$ is characterized by the fibrations and weak equivalences:
weak equivalences will be level equivalences with some homotopical notion of essential surjectivity,
while fibrations will be level fibrations with a homotopical isofibration property.


We highlight three particular $\V$-categories.

\begin{itemize} 
\item Let $\1 = \Omega(\eta)$ be the $\V$-category that detects objects: it has a single object $\set{\**}$, with $\1(\**, \**) = 1_\V$.
\end{itemize}







% \begin{remark}
%       The prototypical example of a $\V$-interval is $W_!J$ where $J = N(0 \leftrightarrows 1) = N \I \in \sSet$ is the nerve of the walking isomorphism.
%       In particular, in $\Cat^{\set{0,1}}(\sSet)$,
%       $W_!(\set{0,1} \to J)$ is a cofibration and
%       the derived counit $W_!J \to \I = \I_f$ is a weak equivalence.

%       Moreover, the set of $\sSet$-intervals with countably-many simplicies are generating \cite[Lemmas 4.2,4.3]{Ber07b}.
% \end{remark}

Morally, $\V$-intervals detect ``homotopical isomorphisms'' in any $\V$-cat $\mathcal C$; cf. Definition \ref{EQUIV_DEF} below.
Additionally, we use these $\V$-intervals to define homotopical analogues of isofibrations and surjective maps of $\V$-categories.




Now, recall the notion of a $(G, \Sigma)$-family (with units) from Definition \ref{GSFAM_DEF}
and the extension to the groupoid $G \ltimes \Sigma_{\mathfrak C}^{op}$ in \eqref{FAMC_DEF_EQ}.



\begin{remark}
      \label{GRAPHF_REM}
      If $\F = \F^\Gamma$ is the $(G, \Sigma)$-family composed of all \textit{graph subgroups} $\Gamma \leq G \times \Sigma_n$,
      we refer to $\F$-equivalences (resp. $\F$-fibrations, etc) as \textit{graph} equivalences (fibrations, etc).
\end{remark}

% \begin{remark}
%       If $\V$ has diagonals, then $F \in \Op^G(\V)$ is a $DK$-$\F$-equivalence iff
%       $F$ is a local weak $\F$-equivalence such that 
%       the associated map of \textit{$\F$-genuine equivariant operads} under the composite
%       \begin{equation}
%             \Op^G(\V) \to \Op_\F(\V) \xrightarrow{\pi_0} \Op_\F(\Set) 
%       \end{equation}
%       is an equivalence.
% \end{remark}

% \begin{remark}
%       Trivial $\F$-fibrations are precisely local $\F$-fibrations which are surjective on objects.
%       Thus $\F$-cofibrations are $f: \O \to \P$ such that
%       each $\mathfrak C(\O)^H \to \mathfrak C(\P)^H$ is injective and
%       $f_!\O \to \P$ is an $\F$-cofibration in $\Op^{G, \mathfrak C(\P)}(\V)$.
% \end{remark}

% ----------------------------------------------------------------------------------------------------
% ---------------------------------------- HAS UNITS? ----------------------------------------
% The main result of this section applies to a large class of $G$-graph systems, with only the following minor condition.

% \begin{definition}
%       We say a $G$-graph system $\F$ \textit{has units} if
%       $\F_1$ contains all graph subgroups of the form $H \leq G \times \Sigma_1$.
%       % In particular, \cite[Remark 4.50]{BP_geo} implies that any \textit{weak indexing system} has units.
% \end{definition}

We can now state and outline the proof of the main result of this section,
which constitutes most of the work towards Theorem \ref{INTRO_MODEL_THM}. % (see the disucssion after Theorem \ref{INTRO_MODEL_THM}).

\begin{theorem}
      \label{MODEL_THM}
      Fix a $(G, \Sigma)$-family $\F = \set{\F_n}$ with units,
      and let $(\V, \otimes)$ denote either $(\sSet, \times)$ or $(\sSet_{\**}, \wedge)$.
      Then there exists a cofibrantly generated model structure on the category $\Op^G(\V)$,
      denoted $\Op^G_\F(\V)$, with
      weak $\F$-equivalences, $\F$-fibrations, and $\F$-cofibrations defined as in Definition \ref{MODEL_DEFN}.
           
      Moreover, analogous semi-model category structures $\Op^G_\F(\V)$ exist
      provided that $(\V, \otimes)$:
      \begin{enumerate}[label = (\roman*)]\itemsep-4pt
      \item is a cofibrantly generated model category,
      \item is a closed monoidal model category with cofibrant unit
            \footnote{Cofibrant unit also needed for \ref{J-CELL_PROP}.},
      \item has cellular fixed-point functors,
      \item \label{CSPP_LBL} has cofibrant symmetric pushout powers % (Defn. \ref{CSPP_DEF}),
            \footnote{Also needed for Props \ref{CAV_4.14_PROP2}, \ref{J-CELL_PROP}}, % \ref{LOCAL_COF_LEM}            
            % --------------------
      \item \label{RP_LBL} is right proper
            \footnote{Needed for Lemma \ref{RIGHTPROPER_LEM} and Lemma \ref{2OUTOF3_PROP}.},
      \item \label{GENSET_LBL} has a set $\mathbb{G}$ of generating $\V$-intervals
            \footnote{Needed so we have a \textit{set} of generating trivial cofibrations},
      \end{enumerate}
\end{theorem}
\begin{proof}
      In both cases, the (semi)-model structures $\Op_\F^{G, \mathfrak C}(\V)$ exist by Theorem \ref{THM1_C}
      (using conditions (i) -- (iv) in the second case).
      % In the first case, we have that the model category $\Op^{G,\mathfrak C}_\F(\sSet)$ exists
      % for any $G$-set $\mathfrak C$ and $G$-graph family $\F$ by Theorem \ref{THM1_C},
      % while in the second case, conditions $(i)$ -- \ref{CSPP_LBL} are sufficient to construct the
      % semi-model category $\Op^{G, \mathfrak C}_\F(\V)$ from said theorem.
      
      Moreover,
      % After this difference, the proofs of the two cases are identical, as
      every object in $\sSet^G$ or $\sSet^G_{\**}$ is genuine cofibrant by e.g. \cite[Remark 5.71]{BP_geo},
      $(iii)$ $\sSet$ and $\sSet_{\**}$ have cellular fixed-point functors by \cite[Example 2.14]{Ste16},
      $(iv)$ $\sSet$ and $\sSet_{\**}$ have cofibrant symmetric pushout powers by \cite[Remark 6.18]{BP_geo},
      \ref{RP_LBL} $\sSet$ and $\sSet_{\**}$ are right proper by e.g. \cite[Thm. 2.1.1 and Prop 4.1.1]{JT_simp},
      % by Lemma \ref{INTER_LEM} and e.g. \cite[Prop 2.1.5]{Cis06} or \cite[Lemma 1.12]{BM13},
      and
      \ref{GENSET_LBL} $\sSet$ and $\sSet_{\**}$ have a generating set of intervals
      by e.g. \cite[Lemma 1.12]{BM13};
      % \ref{TCWE_LBL} the class of genuine weak equivalences in $\mathsf{Op}^G(\sSet)$ is closed under transfinite compositions
      % by an argument analogous to \cite[Lemma 1.24]{CM13b}.
      % Now, we note that condition \ref{TCWE_LBL} proves the analogous statement for any $\F$,
      % since the transfinite composite of local $\F$-equivalences is a local $\F$-equivalence.
      thus we reduce to the second case.
      
      Since $\mathsf{Op}^G(\V)$ is complete and cocomplete, it thus suffices to prove,
      following \cite[Theorem 2.1.19]{Hov}, or Theorem \ref{SEMIMS_THM} in the semi-model structure case, that:
      \begin{enumerate}[label = (\arabic*)]
      \item the class of weak $\F$-equivalences has the 2-out-of-3 property and is closed under retracts;
      \item the domains of $I_{\F}$ (resp. $J_{\F}$) are small relative to $I_{\F}$-cell (resp. $J_{\F}$-cell (with cofibrant source));
      \item $I_{\F}$-inj $= W\cap J_{\F}$-inj; and
      \item $J_{\F}$-cell (with cofibrant source) $\subseteq W\cap I_{\F}$-cof,
      \end{enumerate}
      where $I_\F$ and $J_\F$ are the sets \eqref{IFJF_EQ} of generating (trivial) cofibrations.
      (1) follows from Proposition \ref{2OUTOF3_PROP} and the fact that if $L$ is a retract of $F$, $L^H$ is a retract of $F^H$.
      (2) follows since colimits in $\mathsf{Op}^G(\V)$ are created in $\Op(\V)$, and it holds non-equivariantly.
      (3) follows from Lemma \ref{CAV_4.8}.
      (4) follows from Proposition \ref{J-CELL_PROP}, along with Lemma \ref{POINT_4_LEMMA} and Corollary \ref{LGC_COR}.
\end{proof}



\begin{remark}
      \label{OPGCV_FULL_REM}
      Following Remark \ref{OPGCV_F_JC_REM} below, if $\V$ satisfies conditions $(i) - (vi)$ above,
      and additionally we know independently that
      the semi-model structures on each $\Op^{G, \mathfrak C}(\V)$ can in fact be extended to full Quillen model structures,
      then the resulting semi-model structure on $\Op^G(\V)$ can also be extended to a full Quillen model structure,
      as Proposition \ref{J-CELL_PROP} is the only place in the proof of Theorem \ref{MODEL_THM} where this distinction arises.
\end{remark}

\begin{remark}
      This recovers the main results of \cite{BM13, Cav} for $G = \set{e}$. 
\end{remark}



The rest of this section is devoted to proving the results utilized in the proof of Theorem \ref{MODEL_THM}.
We begin with a description of the sets of generating (trivial) cofibrations.
As expected, these will be freely generated by the generating (trivial) cofibrations of $\V$.

Fix a subgroup $\Gamma \in \F_n$ of $G \times \Sigma_n$,
and define the $G$-set of colors $\mathfrak C_\Gamma$ and $\mathfrak C_\Gamma$-signature $C_0$ by
\[
      \mathfrak C_\Gamma := \Gamma \backslash (G \times \Sigma_n) \cdot_{\Sigma_n} \underline{n}_+,
      \qquad
      C_0 = ([[e,e],1],[[e,e],2],\dots,[[e,e],n];[[e,e],0]).
\]
Noting that $\Aut_{G \ltimes \Sigma_{\mathfrak C_\Gamma}}(C_0)$ is precisely $\Gamma$,
following Notation \ref{FD_NOT} let $\mathbb F_\Gamma = \mathbb F_{C_0}$.
It is then straightforward that for each $K \in \V^\Gamma$,
the operad $\mathbb F_\Gamma[K]$ has the universal property that for all $\O \in \Op^G(\V)$,
\begin{equation}
      \Hom_{\Op^G(\V)}(\mathbb F_\Gamma[K], \O) = \mathop\coprod\limits_{C \in (\mathfrak C_{\O}^{\times n+1})^\Gamma}\Hom_{\V^\Gamma}(K, \O(C)).
\end{equation}

Define $I_{\F,loc}$ and $J_{\F, loc}$ to be the sets
\begin{equation}
      \label{FGAMMA_EQ}
      \set{\mathbb F_\Gamma[\Gamma/\Gamma \cdot (K \xrightarrow{i} L)]}_{\Gamma, i}
      \qquad \mbox{ and } \qquad
      \set{\mathbb F_\Gamma[\Gamma/\Gamma \cdot (A \xrightarrow{j} B)]}_{\Gamma,j}
\end{equation}
where $\Gamma$ runs over all subgroups of $G \times \Sigma_n$ in $\F_n$, $n \geq 0$,
and $i$ (resp. $j$) runs over all generating (trivial) cofibrations in $\V$.


Now, define
\begin{equation}
      \label{IFJF_EQ}
      I_{\F}:= I_{\F, loc} \mathbin{\cup} \set{\varnothing \to G/H \cdot \1}_{H \in \F_1},
      \qquad \qquad
      J_{\F} := J_{\F, loc} \mathbin{\cup} \set{G/H \cdot (\1 \to \J)}_{H \in \F_1,\ \J\in\mathscr{G}}
\end{equation}
where $\1$ defined as in Definition \ref{PL_ES_DEFN}, and $\mathscr{G}$ is a generating set of $\V$-intervals. 


\begin{example}
      \label{FREEOP_EX}
      We unpack these definitions with several examples.
      \begin{enumerate}[label = (\roman*)]
      \item With $n = 0$, we see $\mathbb F_H[\varnothing] = G/H \cdot \Omega(\eta)$ is cofibrant for all $H \in \mathcal F_0$.
      \item With $n = 1$ and $\Gamma = H \leq G \simeq G \times \Sigma_1$, $\mathfrak C_\Gamma = G/H \cdot \set{0,1}$, and
            \[
                  \mathbb F_{\Gamma}[\varnothing] = G/H \cdot (\set{0,1} \cdot \Omega(\eta)),
                  \qquad
                  \mathbb F_{\Gamma}[1_\V] = G/H \cdot \mathbb A^{op} = \Omega(G/H \cdot C_1).
            \]
            with two orbits of objects and a single orbit of non-trivial hom-objects in the latter case.
      \item For higher arity operations,
            if $\Gamma \leq G \times \Sigma_n$ is a \textit{graph subgroup}, i.e. the graph of a homomorphism $G \geq H \xrightarrow{\alpha} \Sigma_n$, then
            we have an identification $\mathfrak C_{\Gamma} \simeq G \cdot_H (A_+)$ where $A$ denotes $n$ equipped with the $H$-action provided by $\alpha$.
            Morevover, if $C = C_A \in \Sigma_G$ is the $G$-corolla (unique up to isomorphism) corresponding to the graph subgroup $\Gamma$, then
            \[
                  \mathbb F_\Gamma[\varnothing] = \partial\Omega(C) = \coprod_{Ge \in \boldsymbol{E}_G(C)} Ge \cdot \Omega(\eta),
                  \qquad
                  \mathbb F_{\Gamma}[1_\V] = \Omega(C),
                  \qquad
                  \mathbb F_\Gamma[\Gamma/\Gamma \cdot K] = \Omega(C) \otimes_\Fin K,
            \]
            with $K \in \V$ and $\otimes_\Fin$ denotes the $\V$-tensoring from Example \ref{TENS_EX};
            in this case, we note that the tensoring is levelwise, with
            $(\Omega(C) \otimes_\Fin K)(\vect C) = \Omega(C)(\vect C) \otimes K$.            
            
            In particular, we conclude that $\Omega(C) \in \Op^G(\V)$ is cofibrant for all $C \in \Sigma_G$ via the following factorization.
            \[
                  \varnothing \to \partial\Omega(C) \xrightarrow{\mathbb F_{\Gamma}(\varnothing \to 1_\V)} \Omega(C).
            \]
            \todo[inline]{connect to Definition \ref{OT_DEF}}
      \item Similarly, $\Omega(T) \in \Op^G(\V)$ is cofibrant for all $T \in \Omega_G$, as
            the grafting decompositions $T = C \coprod_{\boldsymbol{E}_G(C)} (T_{Ge})$ induces the following pushout in $\Op^G(\V)$.
            \[
                  \begin{tikzcd}
                        \displaystyle{
                          \coprod_{Ge \in \boldsymbol{E}_G(C)} \Omega(Ge)}
                        \arrow[r] \arrow[d]
                        &
                        \Omega(C) \arrow[d]
                        \\
                        \displaystyle{
                          \coprod_{Ge \in \boldsymbol{E}_G(C)} \Omega(T_e)}
                        \arrow[r]
                        &
                        \Omega(T)
                  \end{tikzcd}
            \]            
            \todo[inline]{$\Omega(T)$ is $\mathcal F$-cofibrant iff $T \in \Omega_{\mathcal F}$}
      \end{enumerate}
\end{example}
















\newpage

\section{Review of relevant categories and model structures}

In this mostly expository section, we recall the main features of the model structures necessary for the later sections.
Full details and discussion can be found in \cite{BP_edss} and \cite{Per18}.


\subsection{Equivariant dendroidal sets}
\label{EDS_SEC}

We begin with a brief overview of $G$-trees and equivariant dendroidal sets, whose discovery/definition is central and motivating for this entire project.

\begin{definition}
      The category $\Omega_G$ of \textit{$G$-trees} is the category of indecomposable forests with $G$-action.
      Equivalently, an object $T \in \Omega_G$ can be described as any of the following:
      \begin{enumerate}[label = (\roman*)]
      \item A collection $T = (T_i)_{i \in R}$ of trees in $\Omega$, together with a compatible $G$-action on the whole system.
      \item A functor $T: G \ltimes R \to \Omega$ for $R = \mathbf R(T)$ the transitive $G$-set of \textit{roots} of $T$.
      \item An induction $T \simeq G \cdot_H T_e$ for some $T_e \in \Omega^H$ and $H \leq G$.
      \item A quotient $T \simeq G \cdot U / \Gamma$ for some $U \in \Omega$ and graph subgroup $\Gamma \leq G \times \Aut(U)$.
            % \item The quotient $T \simeq (G \cdot T_e) / N$ for $N$ the graph of the homomorphism $H \to \Aut(T_e)$ encoding the $H$-action.
      \end{enumerate}
\end{definition}

We consider the na\"ive presheaf category 
$
\dSet^G = \Fun(G \times \Omega^{op}, \Set),
$
of \textit{equivariant dendroidal sets}.
There is a natural inclusion
\[
      \Omega[-] \colon \Omega_G \to \dSet^G,
      \qquad
      \Omega[T] = \coprod \Omega[T_i] \simeq G \cdot_H \Omega[T_e].
      % \simeq \left(G \cdot \Omega[T_e] \right) / N
\]
extending the Yoneda embedding $\Omega \times G \into \dSet^G$.
Even though the presheaf category is na\"ive, this functor allows us to see genuine equivariant information recorded by $G$-trees
and build a more ``genuine'' model structure.
To begin, we make the following definitions (see \cite[\S 6]{Per18}).

\begin{definition}
      For $T \in \Omega_G$, let $\partial \Omega[T]$ denote the \textit{boundary}
      \[
            \partial \Omega[T] = \coprod \partial \Omega[T_i] \simeq G \cdot_H \partial \Omega[T_e].
            % \simeq \left(G \cdot \partial \Omega[T_e] \right) / N.
      \]
      The \textit{boundary inclusions} are maps in $\dSet^G$ of the form
      \[
            \partial \Omega[T] \to \Omega[T] =
            \coprod \big( \partial \Omega[T_i] \to \Omega[T_i]\big) \simeq
            G \cdot_H \big( \partial \Omega[T_e] \to \Omega[T_e] \big).
            % \simeq
            % \left( G \cdot \left( \partial \Omega[T_e] \into \Omega[T_e] \right) \right) / N.
      \]
\end{definition}

\begin{definition}
      Given $T \in \Omega_G$ and $e \in E(T)$, the associated \textit{$G$-inner horn} is the sub presheaf
      \[
            \Lambda^{Ge}[T] \into \partial \Omega[T] \into \Omega[T],
            \qquad
            \Lambda^{Ge}[T] = \amalg \Lambda^{H_i e}{T_i} \simeq G \cdot_H \Lambda^{He}[T_e].
            % \simeq \left(G \cdot \Lambda^{He}[T_e] \right) / N.
      \]
      The \textit{generating $G$-inner horn inclusion} are maps in $\dSet^G$ of the form
      \[
            \Lambda^{Ge}[T] \to \Omega[T]
      \]
      for $T \in \Omega_G$ and $e \in E(T)$.

      A presheaf $X$ is called a \textit{$G$-$\infty$-operad} if $X \to \**$ has the right lifting property with respect to all generating $G$-inner horn inclusions.
\end{definition}

\begin{definition}
      The class of \textit{$G$-normal monomorphisms}
      is the smallest saturated\footnote{A class of maps is \textit{saturated} if it is closed under pushouts, retracts, and transfinite compositions.}
      class of maps containing the boundary inclusions $\partial \Omega[T] \into \Omega[T]$.
      A map is called a \textit{trivial fibration} if it has the right lifting property with respect to the $G$-normal monomorphisms.
      
      The class of \textit{$G$-inner anodyne extensions} is the smallest saturated class of maps containing the generating $G$-inner horn inclusions.
\end{definition}

These two classes of maps form the basis of a model structure on $\dSet^G$.

\begin{theorem}[{\cite[Thm 2.1, Thm 8.22]{Per18}}]
      There exists a model structure on $\dSet^G$ such that
      \begin{itemize}
      \item cofibrations are $G$-normal monomorphisms,
      \item fibrant objects are $G$-$\infty$-operads,
      \item fibrations between fibrant objects are precisely the maps $X \to Y$ such that the functors associated to each of the fixed-point homotopy categories $\tau j^{\**}(X^H \to Y^H)$ are categorical fibrations for all $H \leq G$.
      \item weak equivalences are the smallest class of maps closed under 2-out-of-3 which
            contains the $G$-inner anodyne extensions and trivial fibrations.
      \end{itemize}
\end{theorem}


These results will be applied in Proposition \ref{W!_COF_PROP}, the main result necessary to prove that $W_!$ is left Quillen.








\subsection{Equivariant simplicial operads}

We will denote by $\mathsf{sOp}$ the category of \textit{colored simplicial operads}, also know as \textit{simplicially enriched multicategories},
and $\mathsf{sOp}^G$ the category of $G$-objects in $\sOp$.

\begin{definition}
      Describe.
\end{definition}



\subsubsection{$\Sigma_G$-cofibrations}

\begin{definition}
      \label{GRAPHSUB_DEF}
      Given a $\mathfrak{C}$-signature $\vect C$, 
      a subgroup 
      $\Gamma \leq \mathsf{Aut}_{G \ltimes \Sigma_{\mathfrak{C}}^{op}}(\vect C)$
      is called a \textit{$G$-graph subgroup} if
      $\Gamma \cap \mathsf{Aut}_{\Sigma_{\mathfrak{C}}^{op}}(\vect C) = \**$;
      i.e. if $\Gamma$ is a $G$-graph subgroup when consider as a subgroup of $G \times \Sigma_{|\vect C|}$.
      
      We write $\mathcal{F}^{\Gamma} = \{\mathcal{F}^{\Gamma}_{\vect C}\}$
      for the collection of families of $G$-graph subgroups.

      A \textit{$G$-graph family} is any collection $\mathcal F = \set{\F_{\vect C}}$ of subfamilies of $\F^\Gamma$.
\end{definition}

\begin{definition}
      We say a map $F \colon X \to Y$ in $\Sym(\sSet)$ is a \textit{$\Sigma_G$-cofibration}
      $F(\vect C)\colon \O(\vect C) \to \P(F_0(\vect C))$ is an $\mathcal F^\Gamma_{\vect C}$-cofibration in $\sSet^{\Aut(\vect C)}_{\F^\Gamma_{\vect  C}}$
      for all signatures $\vect C$.
      
      We say a map $F \colon \O \to \P$ in $\sOp^G$ is a \textit{$\Sigma_G$-cofibration} if
      the map on underlying symmetric sequences is.
\end{definition}

\begin{example}
      All generating cofibrations of $\sOp^G$ are $\Sigma_G$-cofibrations,
      as all subgroups of $G$-graph subgroups are again $G$-graph subgroups, and the genuine cofibrations in $\sSet^G_{gen}$ are the monomorphisms.
\end{example}

STUFF

\begin{definition}
      We generalize the constructions in \cite[6.13, 6.19, 6.37]{BP_geo}.
      
      Let $\mathcal F$, $\bar{\mathcal F}$ be $G$-graph families of $\Lambda \leq G \times \Sigma$ and $\bar\Lambda \leq G \times \bar \Sigma$, respectively.
      We define the following four families:
      \begin{enumerate}[label = (\roman*)]
      \item $\mathcal \F \sqcap \bar{\mathcal F} := \pi^{\**}(\F) \cap \bar\pi^{\**}(\bar \F)$ of $\Lambda \times \bar \Lambda$,
      \item $\mathcal F \sqcap_G \bar \F := \Delta^{\**}(\F \sqcap \bar\F)$ of $\Lambda \times_G^\lrcorner \bar\Lambda$,
      \item $\F^{\ltimes n} := (i_e)_{\**}\big(\pi_{e,\Lambda}^{\**}(\F)\big)$ of $\Sigma_n \wr \Lambda$,
      \item $\F^{\ltimes_G n} := (\Sigma_n \wr \Delta)^{\**}(\F^{\ltimes n})$  of $\Sigma_G \wr_G \Lambda$,
      \end{enumerate}
      where $\pi$, $\bar\pi$, $\Delta$, $\pi_{e, \Lambda}$, $i_e$, $\Sigma_n \wr \Delta$ are the natural maps below,
      with $(-) \times_G^\lrcorner (-)$ denoting the fiber product over the projections $\Delta,\bar\Delta \to G$.
      \[
            \begin{tikzcd}
                  \Lambda
                  &
                  \Lambda \times \bar \Lambda \arrow[r, "\bar\pi"] \arrow[l, "\pi"']
                  &
                  \bar\Lambda
                  \\
                  &
                  \Lambda \times_G^\lrcorner \bar\Lambda \arrow[u, "\Delta"]
            \end{tikzcd}
      \]
      \[
            \Lambda \xleftarrow{\pi_{e,\Lambda}}
            \Sigma_{\lambda_e} \wr \Lambda = (\Sigma_{\set{e}} \times \Lambda) \times \Sigma_{\underline{n} \setminus e} \wr \Lambda \xrightarrow{i_e}
            \Sigma_n \wr \Lambda \xleftarrow{\Sigma_n \wr \Delta}
            \Sigma_n \wr_G \Lambda = \Sigma_n \ltimes \Lambda^{\times_G^\lrcorner n}
      \]
\end{definition}

\begin{proposition}[{cf. \cite[Prop. 6.40]{BP_geo}}]
      \label{640_LEM}
      We have left Quillen functors as below by \cite[Prop. 6.6, 6.14, 6.6]{BP_geo}.
      \[
            \V^{\Lambda}_\F \times \V^{\bar \Lambda}_{\bar \F} \longto
            \V^{\Lambda \times \bar\Lambda}_{\pi^{\**}(\F)} \times \V^{\Lambda \times \bar\Lambda}_{\bar\pi^{\**}(\bar \F)} \xrightarrow{\otimes}
            \V^{\Lambda \times \bar\Lambda}_{\F \sqcap \bar \F} \longto
            \V^{\Lambda \times_G^\lrcorner \bar\Lambda}_{\F \sqcap_G \bar\F}.
      \]
\end{proposition}

\begin{proposition}[{cf. \cite[Prop. 6.41]{BP_geo}}]
      \label{641_LEM}
      If $f$ is an $\F$ (trivial) cofibration in $\V^{\Lambda}_\F$, then $f^{\square n}$ is an $\F^{\ltimes_G n}$ (trivial) cofibration in $\V^{\Sigma_n \wr_G \Lambda}_{\F^{\ltimes_G n}}$.
\end{proposition}
\begin{proof}
      We have a composite of left Quillen functors by \cite[Prop. 6.24, 6.6]{BP_geo}.
      \[
            \V^{\Lambda}_{\F} \longto \V^{\Sigma_n \wr \Lambda}_{\F^{\ltimes n}} \longto \V^{\Sigma_n \wr_G \Lambda}_{\F^{\ltimes_G n}}.
      \]
\end{proof}

\todo[inline]{introduce $Aut(T)$ and colored-version  earlier}

\begin{remark}[{cf. \cite[Remark 6.48]{BP_geo}}]
      Let $T \in \Omega$. Then we have a grafting decomposition $T \simeq C \amalg (T_i)$ for $C \in \Sigma$ and $T_i \in \Omega$.
      If we let $\lambda$ denote the partition $\boldsymbol{L}(C) = \lambda_1 \amalg \dots \amalg \lambda_k$ such that
      $i$,$j$ are in the same class iff $T_i \simeq T_j$,
      and pick representatives $i_j \in \lambda_j$, we thus have a decomposition
      \[
            \Aut(T) \simeq \Sigma_{|\lambda_1|} \wr \Aut(T_{i_1}) \times \dots \times \Sigma_{|\lambda_r|} \wr \Aut(T_{i_r}).
      \]
      
      For $\vect T \in \Omega_{\mathfrak C}$, we have an analogous decomposition of $\vect T$ using a partition $\vect \lambda$,
      and controlling for the repeated $G$-action, we have
      \[
            \Aut(\vect T) \simeq \big(\Sigma_{|\lambda_1|} \wr_G \Aut(\vect T_{i_1}\big) \times_G^\lrcorner \dots \times_G^\lrcorner \big(\Sigma_{|\lambda_s|} \wr_G \Aut(\vect T_{i_s})\big)
            %\prod_{j=1}^s{}_G \Sigma_{|\lambda_j} \wr_G \Aut(\vect T_{i_j})
      \]
      where $(-) \times_G^\lrcorner (-)$ again denotes the fiber product over $G$. 
\end{remark}

\begin{remark}
      For any $\vect T \in G \ltimes \Omega_{\mathfrak C}$ with underlying tree $T \in \Omega$,
      we have a natural map $\phi: \Aut(\vect T) \to G \times \Aut(T)$ induced by the forgetful functor $G \ltimes \Omega_{\mathfrak C} \to G \times \Omega$.

      Now, fix a $G$-graph family $\mathcal F = \set{\F_n}$.
      Following \cite[Prop. 6.44, Lemma 6.49]{BP_geo}, for any $T \in \Omega$ and $\vect T \in \Omega_{\mathfrak C}$,
      we define the following families of subgroups of $G \times \Aut(T)$ and $\Aut(\vect T)$, respectively,
      using the grafting decomposition as above:
      \begin{align*}
        \F_T &= \pi_C^{\**}(\F_C) \cap \left( \F_{T_{i_1}}^{\ltimes_G |\lambda_1|} \sqcap_G \dots \sqcap_G \F_{T_{i_r}}^{\ltimes_G |\lambda_r|} \right),
        \\
        \F_{\vect T} &= \pi_{\vect C}^{\**}(\F_{\vect C}) \cap \left( \F_{\vect T_{i_1}}^{\ltimes_G |\lambda_1|} \sqcap_G \dots \sqcap_G \F_{\vect T_{i_s}}^{\ltimes_G |\lambda_s|} \right),
      \end{align*}
      with $\pi_C \colon G \times \Aut(T) \to G \times \Aut(C)$ and $\pi_{\vect C} \colon \Aut(\vect T) \to \Aut(\vect C)$.
\end{remark}

\begin{lemma}
      \label{FVECTT_LEM}
      $\F_{\vect T} = \phi^{\**}(\F_T)$.
\end{lemma}
\begin{proof}
      First, $\Gamma \in \F_T \cap \Aut(\vect T)$ iff $\pi_C(\Gamma) \in \Aut(\vect C)$, as the following diagram commutes.
      \[
            \begin{tikzcd}
                  \Aut(\vect T) \arrow[d] \arrow[r]
                  &
                  G \times \Aut(T) \arrow[d]
                  \\
                  \Aut(\vect C) \arrow[r]
                  &
                  G \times \Aut(C).
            \end{tikzcd}
      \]
      
      Second, \cite[Equation (6.51)]{BP_geo} says that $\Gamma \leq G \times \Aut(T)$ is in
      $\left(\F_{T_{i_1}}^{\ltimes_G |\lambda_1|} \sqcap_G \dots \sqcap_G \F_{T_{i_r}}^{\ltimes_G |\lambda_r|} \right)$
      iff for all $l \in \boldsymbol{L}(C)$,
      \[
            \pi_{G \times \Aut(T_l)}\Big(
            \big(\Sigma_{{e} \subseteq \boldsymbol{L}(C)} \times (G \times \Aut(T_l)) \big) \times_G^\lrcorner
            \big( \Sigma_{\lambda \setminus \set{l}} \wr_G (G \times \Aut(T_j))_{j \neq l} \big)
            \Big) \in \F_{T_l},
      \]
      where $\lambda \setminus \set{l}$ is the induction partition on $\boldsymbol{L}(C) \setminus \set{l}$ and $\pi_{G \times \Aut(T_l)}$ is the observable projection;
      similarly $\Gamma \leq \Aut(\vect T)$ is in 
      $\left(\F_{\vect T_{i_1}}^{\ltimes_G |\vect \lambda_1|} \sqcap_G \dots \sqcap_G \F_{\vect T_{i_s}}^{\ltimes_G |\vect \lambda_s|} \right)$
      iff for all $l \in \boldsymbol{L}(C)$,
      \[
            \pi_{G \times \Aut(T_l)}\Big(
            \big(\Sigma_{{e} \subseteq \boldsymbol{L}(C)} \times \Aut(\vect T_l)) \big) \times_G^\lrcorner
            \big( \Sigma_{\vect \lambda \setminus \set{l}} \wr_G (\Aut(\vect T_j))_{j \neq l} \big)
            \Big) \in \F_{\vect T_l}.
      \]
      While the wreath products in the above parenthesized groups may differ due to $\vect \lambda \subseteq \lambda$,
      the projections clearly commute with the maps $\Aut(\vect T_l) \to G \times \Aut(T_l)$. 

      Finally, $\Gamma \in \F_{\vect T}$ must always be in $\Aut(\vect T)$.

      The result follows.
\end{proof}

\begin{proposition}[{cf. \cite[Prop. 6.52]{BP_geo}}]
      \label{BOXVT_PROP}
      Suppose $f_s: A_s \to B_s$, $1 \leq s \leq l$, are level $\F$ (trivial) cofibrations in $\Sym(\V)$, i.e. $f_s(\vect C)$ is a (trivial) cofibration in $\V^{\Aut(\vect C)}_{\F_{\vect C}}$ for all signatures $\vect C$.
      Then for any $\underline{l}$-labeled signature $\vect T \in \Omega^{\underline{l}}_{\mathfrak C}$, the map
      \[
            f^{\square V(T)} = \mathop{\square}\limits_{1 \leq r \leq l} \mathop{\square}\limits_{v \in \boldsymbol{V}_r(T)} f_r(\vect T_v)
      \]
      is a (trivial) cofibration in $\V^{\Aut_{G \ltimes \Omega_{\mathfrak C}}(\vect T)}_{\F_{\vect T}}$.
\end{proposition}
\begin{proof}
      Using the same decomposition as before, we have
      \[
            f^{\square V(T)} = f_{r_C}(\vect C) \square \mathop{\mathlarger{\square}}\limits_{1 \leq i \leq s}\left(f^{\square V(\vect T_{i_j})}\right)^{\square |\vect \lambda_j|}.
      \]
      Lemmas \ref{640_LEM}, \ref{FVECTT_LEM} and Proposition \ref{RESGEN PROP} combine to show that the composite
      \[
            \V^{\Aut(\vect C)}_{\F_{\vect C}} \times
            \prod_{i=1}^s
            \V^{\Sigma_{|\vect \lambda_1|} \wr_G \Aut(\vect T_1)}_{\F_{\vect T_1}^{\ltimes_G |\vect \lambda_1|}}
            \times_G^\lrcorner
            \dots
            \times_G^\lrcorner
            \V^{\Sigma_{|\vect \lambda_s|} \wr_G \Aut(\vect T_s)}_{\F_{\vect T_s}^{\ltimes_G |\vect \lambda_s|}}
            \xrightarrow{\quad \otimes \quad}
            \V^{\Aut(\vect T)}_{\F_{\vect T}}
      \]
      is left Quillen,
      with automorphism groups taking place in $G \ltimes \Omega_{\mathfrak C}$. 
      The result now follows from Lemma \ref{641_LEM} and the induction hypothesis.
\end{proof}


\begin{corollary}
      \label{SIGMAG_COF_COR}
      If $f \colon \O \to \P$ is a cofibration such that $\O$ is $\Sigma_G$-cofibrant, then $f$ is a $\Sigma_G$-cofibration.
\end{corollary}
\begin{proof}
      Using the filtration from Proposition \ref{FILTPUSHG PROP},
      this follows from Proposition \ref{BOXVT_PROP}, Lemma \ref{RESGEN PROP}, and the fact that all generating cofibrations in $\sOp^G$ are $\Sigma_G$-cofibrations between $\Sigma_G$-cofibrant objects.
\end{proof}

We now consider the case $\V = \sSet$.
The following is immediate.
\begin{corollary}
      If $\O \in \sOp^G$ is cofibrant, then $\O$ is $\Sigma_G$-cofibrant.
\end{corollary}


Now, recall that for any group $G$ and family of subgroups $\F$,
$K \to L$ in $\sSet^{G}_\F$ is an $\F$-cofibration iff $L_n \setminus K_n$ has isotropy in $\F$; i.e. for all $x \in L_n \setminus K-n$, $\Stab_{G}(x) \in \F$.
% 
But we also note the easy observation that for any set $A$ with an action of $G \times \Sigma$,
$A$ has isotropy in the family $\F^\Gamma$ of $G$-graph subgroups
iff
$A$ is $\Sigma$-free.
% 
Thus $K \to L$ in $\sSet^{G \times \Sigma}_{\F^\Gamma}$ is an $\F^\Gamma$-cofibration
iff
it forgets to a cofibration in $\sSet^\Sigma_{proj}$ the coarse model structure.

More generally, if $\Upsilon \leq G \times \Sigma$ and $\F^\Gamma_{\Upsilon} = \F^\Gamma \cap \Upsilon$,
then $K \to L$ in $\sSet^{\Upsilon}_{\F^\Gamma_\Upsilon}$ is an $\F^\Gamma_\Upsilon$-cofibration
iff
it forgets to a cofibration in $\sSet^{\pi_{\Sigma}\Upsilon}_{proj}$ the coarse model structure.

All together, we conclude the following (cf. \cite[Remark 6.7]{Per18}, \cite[discussion before Thm. 2.31]{BP_edss}).
\begin{proposition}
      \label{SGS_COF_PROP}
      $\O \to \P$ in $\sOp^G$ is a $\Sigma_G$-cofibration iff it forgets to a $\Sigma$-cofibration in $\sOp$.

      In particular, $\O$ is $\Sigma_G$-cofibrant iff $\O(n)$ has a free $\Sigma_n$-action.
\end{proposition}



















\subsubsection{other stuff}

\todo[inline]{where does this go?}

\begin{definition}
      \label{OT_DEF}
      Fix $U \in \Omega$.
      We denote by $\Omega(U) \in \Op$ the free colored operad (of sets) generated by its edges and vertices:
      it has colors $\mathfrak C = \mathbf E(U)$ the set of edges, and
      for any $\mathbf E(U)$-signature $\vect C$, 
      \begin{equation}
           \label{OMEGADEF_EQ}
            \Omega(U)(\vect C) =
            \begin{cases}
                  \** \qquad \qquad \qquad & \mbox{there exists $C \xrightarrow{\phi} U$ such that $\partial \phi = \vect C$}
                  \\
                  \varnothing & \text{else,}
            \end{cases}
      \end{equation}
      where $\partial \phi$ is the $\mathbf E(U)$-signature $(\phi(1), \phi(2), \dots, \phi(n); \phi(0))$
      given by the (ordered) image of the edges of $C$.
      \todo[inline]{profile of a map? is this used in other places?}
      This extends to a functor $\Omega \into \Op$, and has an associated nerve-realization adjunction
      \begin{equation}
            \label{DSETADJ_EQ}
            \begin{tikzcd}
                  \dSet \arrow[r, shift left]
                  &
                  \Op \arrow[l, shift left, "N"]
            \end{tikzcd}
      \end{equation}
      and \cite[Prop. 2.5]{CM11} show that this adjunction is Quillen.
\end{definition}

\todo[inline]{move $W(T)$ discussion here!}








\subsection{Equivariant dendroidal Segal spaces and the joint Bousfield model structure}
\label{JT_SEC}

We recall the category $\mathsf{sdSet}^G = \Set^{\Delta^{op} \times \Omega^{op} \times G}$ of \textit{equivariant simplicial dendroidal sets} and the three model structures used in \cite{BP_edss}.
First, for $X \in \mathsf{sdSet}^G$, we write $X_n(U) \in \Set$ for the evaluation at $n \in \Delta$ and $U \in \Omega$,
and refer to $n$ and $U$ as the \textit{simplicial/vertical} and \textit{dendroidal/horizontal} directions.
Moreover, for any $X \in \mathsf{sdSet}^G$, have unique colimit-preserving functors
\begin{equation}
      \label{SDSET_EQ}
      X_{(-)} \colon \sSet \to \dSet^G,
      \qquad
      X(-) \colon \dSet^G \to \sSet
\end{equation}
such that $X_{\Delta[n]} = X_n$ and $X(\Omega[U]) = X(U)$ for $n \geq 0$, $U \in \Omega$.

We note that $\dSet^G$ and $\sSet$ are both subcategories, constant in one direction;
we will often not denote the inclusion funcotr.
Moreover, for $A \in \dSet^G$ and $K \in \sSet$ we will write $A \times K$ for the natural presheaf $(A \times K)_n(U) = A(U) \times K_n$.

The various model structures on $\mathsf{sdSet}^G$ arise from two associated (generalized) Reedy categories.
First, we may consider the simplicial Reedy model structure over $\dSet^G$ equipped with the model structure from \cite{Per18},
yielding the \textit{simplicial Reedy} model structure on $\mathsf{sdSet}^G = (\dSet^G)^{\Delta^{op}}$,
with $f \colon X \to Y$ an equivalence iff $f$ is a \textit{dendroidal equivalence}: for all $n \geq 0$, $f_n: X_n \to Y_n$ is a weak equivalence in $\dSet^G$.

Second, $\Omega^{op} \times G$ is generalized Reedy \cite[Example A.7]{BP_edss},
and we may consider the equivariant dendroidal Reedy model structure over $\sSet$ equipped with the Kan-Quillen model structure,
yielding the \textit{dendroidal Reedy} model structure on $\mathsf{sdSet}^G = (\sSet)^{\Omega^{op} \times G}$.
Here, $f\colon X \to Y$ is an equivalence iff $f$ is an \textit{(equivariant) simplicial equivalence}:
for all $U \in \Omega$, $f(U) \colon X(U) \to Y(U)$ is a $G$-graph Kan equivalence in $\sSet^{G \times \Aut(U)}$.
Equivalently, since any $T \in \Omega_G$ has a decomposition $T \simeq G \cdot U/\Gamma$ for some graph subgroup $G \leq G \times \Aut(U)$,
in which case $X(\Omega[T]) \simeq X(U)^\Gamma$,
we see that $f$ is a simplicial equivalence iff $f(\Omega[T]) \colon X(\Omega[T]) \to Y(\Omega[T])$ is a Kan equivalence in $\sSet$ for all $T \in \Omega_G$.

Third, we may consider the \textit{joint Bousfield localization} of these two model structures.
The following result is a summary of work from \cite[\S 4.1]{BP_edss}.
\begin{theorem}
      \label{JB_THM}
      The \textit{joint Reedy model structure} on $\mathsf{sdSet}^G$ is the smallest joint Bousfield localization of the dendroidal Reedy and simplicial Reedy model structures.
      Equivalences and fibrant objects are called \textit{joint equivalences} and \textit{joint fibrant objects}.
      Morever, it has the following properties:
      \begin{enumerate}[label = (\roman*)]
      % \item \label{GENREEDY_LBL} The Reedy generating (trivial) cofibrations of $\mathcal M^{\Delta^{op}}$ and $\mathcal M^{G \times \Omega^{op}}$ are
      %       \[
      %             (\partial \Delta[n] \to \Delta[n]) \square i,
      %             \qquad
      %             (\partial \Omega[T] \to \Omega[T]) \square i,
      %       \]
      %       for $n \geq 0$ and $T \in \Omega_G$,
      %       where $i$ is a generating (trivial) cofibration in $\mathcal M$
            % More generally, the $R$-Reedy generating (trivial) cofibrations in $\mathcal M^{\Delta^{op} \times R^{op}}$ are
            % \[
            %       (\partial \Delta[n] \to \Delta[n]) \square i'
            %       \qquad
            %       (\Lambda^i[n] \to \Delta[n]) \square i'
            % \]
            % where $i'$ is a generating cofibration for the Reedy model structure on $\mathcal M^{R^{op}}$.
      \item \label{PROPER_LBL} it is both left and right proper.
      \item \label{SDEQUIV_LBL} both vertical/simplicial and horizontal/dendroidal equivalences are also joint equivalences.
      \item \label{JTFIB_LBL} $X$ is joint fibrant iff $X$ is both simplicial and dendroidal Reedy fibrant.
            In particular, $X_n \in \dSet^G$ and $X(\Omega[T]) \in \sSet$ are fibrant.
      \item \label{JTCFIB_MAP_LBL} Joint Reedy (co)fibrations are levelwise (co)fibrations.
            Moreover, $f$ a joint (co)fibration implies that $f(\Omega[T])$ is a (co)fibration in $\sSet$ for all $T \in \Omega_G$.
      \item \label{SFIB_JEQ_LBL} $X,Y$ simplicial (resp. dendroidal) Reedy fibrant, then a map $X \to Y$ is a joint equivalence iff it is a simplicial (resp. dendroidal) equivalence.
      \end{enumerate}
\end{theorem}
\begin{proof}
      % \label{GENREEDY_LBL} combines \cite[Prop. A.33, Example A.34]{BP_edss}.
      \ref{PROPER_LBL} follows since $\sSet$ is left and right proper.
      \ref{SDEQUIV_LBL} is by construction.
      \ref{JTFIB_LBL} is \cite[Prop. 4.1(ii)]{BP_edss}.
      \ref{JTCFIB_MAP_LBL} follows from \cite[Lemmas A.27, A.29]{BP_edss}.
      \ref{SFIB_JEQ_LBL} follows from \cite[Prop. 4.5(iii), Cor. 4.29(iii)]{BP_edss}.      
\end{proof}


Fourth and finally, there is the \textit{(equivariant) dendroidal Segal space} model structure,
the localization of the simplicial Reedy model structure with respect to the Segal core inclusions
\[
      Sc[T] \longto \Omega[T],
      \qquad
      T \in \Omega_G.
\]

\begin{lemma}
      \label{FCOLIM_WE_LEM}
      The weak equivalences in the dendroidal Reedy, dendroidal Segal space, and joint Reedy model structures are closed under filtered colimits.
\end{lemma}
\begin{proof}
      Weak equivalences in the dendroidal Reedy are simplicial equivalences, which are closed under filtered colimits
      since this holds for $\sSet$ and colimits and weak equivalences are computed levelwise.
      The result for the latter two structures follows from the result for the first, as they are localizations.
\end{proof}



\subsection{Segal preoperads}
\label{SPREOP_SEC}

We recall some definitions from \cite[\S4,5]{BP_edss}.

\begin{definition}
      The category of \textit{preoperads} $\mathsf{PreOp}^G$ is the full subcategory of $\mathsf{sdSet}^G$ of those $X$ such that
      $X(\eta)$ is a discrete simplicial set.

      There is a natural pair of adjunctions
      \[
            \begin{tikzcd}[column sep =5em]
                  \mathsf{PreOp}^G \ar{r}[swap]{\gamma^{\**}} 
                  &
                  \mathsf{sdSet}^G
                  \ar[bend right]{l}[swap,midway]{\gamma_{!}}
                  \ar[bend left]{l}{\gamma_{\**}}
            \end{tikzcd}
      \]
      described by
      $\gamma_{!}X (U) = X(U)$ if $U \not \in \Delta$
      while $\gamma_{!}X ([n])$ for $[n] \in \Delta$ is given by the pushout on the left below; 
      $\gamma_{\**}X(U)$ is given by the pullback on the right below.
      \begin{equation}\label{GAMMASTAR_EQ}
            \begin{tikzcd}
                  X(\eta) \ar{r} \ar{d} \arrow[dr, phantom, "\ulcorner", very near start]  &
                  \pi_0 X(\eta) \ar{d}
                  && 
                  \gamma_{\**}X(U) \ar{r} \ar{d} & X(U) \ar{d}
                  \\
                  X([n]) \ar{r} & \gamma_! X([n]) 
                  &&
                  \prod_{\boldsymbol{E}(U)} X_0(\eta) \ar{r} &
                  \prod_{\boldsymbol{E}(U)} X(\eta)
                  \arrow[lu, phantom, "\lrcorner", very near start]
            \end{tikzcd}
      \end{equation}

      Moreover, following \cite[Remark 4.33]{BP_edss}, % GAMMASH REM
      we have that any monomorphism $A \to B$ in $\mathsf{sdSet}^G$
      such that $A(\eta) \to B(\eta)$ is an isomorphism
      induces a pushout square
      \begin{equation}\label{GAMMASH EQ}
            \begin{tikzcd}
                  A \ar{r} \ar{d} \arrow[dr, phantom, "\ulcorner", very near start]&
                  \gamma^{\**}\gamma_! A \ar{d}
                  \\
                  B \ar{r} & \gamma^{\**}\gamma_! B 
            \end{tikzcd}
      \end{equation}
\end{definition}


\begin{definition}
	A $G$-preoperad $X \in \mathsf{PreOp}^G$ is called a \textit{$G$-Segal operad} if, 
	for each $G$-tree $T$,
	the natural map 
	$X\left( \Omega[T] \right) \to 
	X \left( Sc[T] \right)$
	is a Kan equivalence.
\end{definition}

\begin{definition}%[{cf. \cite[Defn. 5.6,5.7]{BP_edss}}]
      Given a $G$-Segal operad $X$ and a $G$-corolla $C$,
      $X(\partial \Omega[C])$ is a discrete simplicial set whose elements
      are the \textit{$C$-signatures} $(c_1,\cdots,c_n;c_0)$ of $X$;
      explicitly, for $C \simeq C_A$, $A = H_0/H_1 \amalg \dots \amalg H_0/H_n$, this is the data of
      a list of objects $x_j \in X(\eta)^{H_j}$ for $0 \leq j \leq n$.

      Given a $C$-signature $(x_1, \dots, x_n; x_0)$ of $X$, we define the \textit{mapping space} $X(x_1,\dots,x_n; x_0)$ as the pullback
      \[
            \begin{tikzcd}
                  X(x_1, \dots, x_n; x_0) \arrow[r] \arrow[d, twoheadrightarrow]
                  &
                  X(\Omega[C]) \arrow[d, twoheadrightarrow]
                  \\
                  \Delta[0] \arrow[r, "{(x_1, \dots, x_n;x_0)}"]
                  &
                  \prod_{j = 0}^n X(\eta)^{H_j}.
            \end{tikzcd}
      \]

      The map $X(\Omega[C]) \to X(\partial \Omega[C])$ hence yields
      a coproduct decomposition 
      \[
            X(\Omega[C]) \simeq \coprod_{C\text{-signatures }(c_1,\cdots,c_n;c_0)}
            X(c_1,\cdots,c_n;c_0)
      \]    
\end{definition}

\begin{definition}
      Given a $G$-Segal operad $X$, define $ho(X) \in \dSet_G$ by
      \[
            ho(X) = \pi_0\left(\upsilon_{\**}\gamma^{\**}\gamma_{\**}X\right)
      \]
      with $\upsilon: \Omega \times G \to \Omega_G$, $\upsilon_{\**}: \dSet^G \to \dSet_G$.

      Consider the composite inclusion $i_G: \Delta \times \mathsf O_G \to \Omega \times \mathsf O_G \to \Omega_G$.
      Then $i_G^{\**}ho(X)$ is a coefficient system of (nerves of) catgories by \cite[Prop. 5.9, Remark 5.11]{BP_edss}
\end{definition}

\begin{definition}
      A map $f \colon X \to Y$ of $G$-Segal operads is called
      \begin{enumerate}[label = (\roman*)]
      \item \textit{fully-faithful} if for all $C \in \Sigma_G$ and all $C$-signatures $(x_1,\dots, x_n;x_0)$ of $X$, the induced map
            \[
                  X(x_1, \dots, x_n; x_0) \to Y(f(x_1), \dots, f(x_n); f(x_0))
            \]
            is a Kan equivalence of simplicial sets.
      \item \textit{essentially surjective} if the map $i_G^{\**}X \to i_G^{\**}Y$ of $G$-coefficient systems of categories
            is levelwse essentially surjective.
      \item a \textit{Dwyer-Kan (DK) equivalence} if it is both fully-faithful and essentially surjective.
      \end{enumerate}
\end{definition}


The following summarizes \cite[Theorems 4.39, 4.42, and 5.48, and Corollary 5.51]{BP_edss},
with left properness following from Theorem \ref{JB_THM} \ref{PROPER_LBL}.
\begin{theorem}[\cite{BP_edss}]
      The category $\mathsf{PreOp}^G$ has a left proper model structure such that
      cofibrations and weak equivalences are determined by $\gamma^{\**}$.
      The fibrant objects in $\mathsf{PreOp}^G$ are the Reedy-fibrant $G$-Segal operads,
      and a map betwen fibrant objects is a weak equivalence iff it is a DK-equivalence.
      
      Moreover, the adjunction $\gamma^{\**} \colon \mathsf{PreOp}^G \rightleftarrows \mathsf{sdSet}^G \colon \gamma_{\**}$
      is a Quillen equivalence.
\end{theorem}

\begin{lemma}
      \label{FCOLIM_WE2_LEM}
      Weak equivalences in $\mathsf{PreOp}^G$ are closed under filtered colimits.
\end{lemma}
\begin{proof}
      This is an immediate corollary of Lemma \ref{FCOLIM_WE_LEM}.
\end{proof}


\begin{remark}\label{SEOPDK REM}
Given a $G$-Segal operad $X$, consider a dendroidal Reedy fibrant replacement $X \to \tilde{X}$ such that $X(\eta) \simeq \tilde{X}(\eta)$. 
This means that all maps 
$X(\Omega[T]) \to \tilde{X}(\Omega[T])$ are Kan equivalences,
and moreover, by the following pullback diagram
\[
\begin{tikzcd}
	Z (Sc[T]) \ar{r} \ar{d} &
	\prod_{v \in \boldsymbol{V}_G(T)} Z
	(\Omega[T_v]) \ar{d}
\\
	\prod_{(G/H_i \cdot e_i) \in \boldsymbol{E}_G(T)} 
	\mathfrak{C}^{H_i} \ar{r}  &
	\prod_{v \in \boldsymbol{V}_G(T)}
	\prod_{(G/H_i \cdot e_i) \in \boldsymbol{E}_G(T_v)} 
	\mathfrak{C}^{H_i} 
	\arrow[ul, phantom, "\lrcorner", very near start]
\end{tikzcd}
\]
so are the maps $X(Sc[T]) \to \tilde{X}(Sc[T])$.
This shows that $\tilde{X}$ is also a Segal operad, 
and thus a fibrant object in $\mathsf{PreOp}^G$.

Furthermore, the Kan equivalences 
$X(\Omega[C]) \to \tilde{X}(\Omega[C])$
induce Kan equivalences 
$X(c_1,\cdots,c_n;c_0) \to \tilde{X}(c_1,\cdots,c_n;c_0)$.
It then follows from \cite[Thm. 5.48, Cor. 5.51]{BP_edss} that the complete equivalences between Segal operads are precisely the Dwyer-Kan equivalences.
\end{remark}



\section{The tame model structure}
\label{TAME_SEC}

In this section, we build an auxiliary model structure on preoperads, Quillen equivalent to the one recalled in \S \ref{SPREOP_SEC},
as well as prove Proposition \ref{KEYPR PROP},
the main tool which allows us to show that the nerve $\sOp^G \to \mathsf{PreOp}^G$ is an equivalence of homotopy theories.



\begin{definition}
	The \textit{colored tensor product} 
\[
\begin{tikzcd}[row sep = 0, column sep = 40pt]
	\mathsf{PreOp}^G \times \mathsf{sSet} \ar{r}{(-)\otimes_{\mathsf{F}}(-)} &
	\mathsf{PreOp}^G
\end{tikzcd}
\]
is defined by $(X \otimes_{\mathsf{F}} K)(T) = X(T) \times K$
whenever $T$ is a non-linear tree (equivalently, 
$\mathsf{Hom}_{\Omega}(T,\eta)=\emptyset$) and
is defined by the following pushout when $T=[n]$ is linear.
\[
\begin{tikzcd}
	X(\eta) \times K \ar{r} \ar{d} \arrow[dr, phantom, "\ulcorner", very near start]  &
	X(\eta) \ar{d}
\\
	X([n]) \times K \ar{r} & 
	(X \otimes_{\mathsf{F}} K)([n]) 
\end{tikzcd}
\]
\end{definition}

\begin{remark}
More concisely, $X \otimes_{\mathsf{F}} K$ is defined by the pushout in $\mathsf{sdSet}^G$
\[
\begin{tikzcd}
	\left(\mathsf{sk}_{\eta}X \right) \times K \ar{r} \ar{d} \arrow[dr, phantom, "\ulcorner", very near start]  &
	\mathsf{sk}_{\eta}X \ar{d}
\\
	X \times K \ar{r} & 
	X \otimes_{\mathsf{F}} K,
\end{tikzcd}
\]
where $\mathsf{sk}_{\eta}X$ is the 0-skeleton of $X$ in the equivariant dendroidal Reedy direction;
explicitly, $\mathsf{sk}_{\eta}X(U) = X(\eta)$ if $U = [n]$ is a linear tree, and $\varnothing$ otherwise.
\end{remark}




\begin{lemma}
      \label{TAUOTIMES_LEM}
      For all $X \in \mathsf{PreOp}^G$ and $K \in \sSet$, $\tau(X \otimes_\Fin K) = \tau(X) \otimes_\Fin K$,
      with the second $\otimes_\Fin$ the (fiberwise) simplicial tensoring of $\sOp^G$ defined as in Example \ref{TENS_EX}.
\end{lemma}
\begin{proof}
      We want to show the square of left adjoints on the left below commutes;
      instead, we show the square of right adjoints on the right commutes,
      \[
            \begin{tikzcd}
                  \mathsf{PreOp}^G \arrow[d, "\tau"'] \arrow[r, "{(-) \otimes_\Fin K}"]
                  &
                  \mathsf{PreOp}^G \arrow[d, "\tau"]
                  & % ----------
                  \mathsf{PreOp}^G
                  &
                  \mathsf{PreOp}^G \arrow[l, "\set{K,-}_\Fin"']
                  \\
                  \sOp^G \arrow[r, "{(-) \otimes_\Fin K}"]
                  &
                  \sOp^G
                  & %----------
                  \sOp^G \arrow[u, "N"]
                  &
                  \sOp^G \arrow[l, "\set{K,-}_\Fin"'] \arrow[u, "N"']
            \end{tikzcd}
      \]
      where $\set{K, X}_\Fin$ on $\mathsf{PreOp}^G$ is the right adjoint of $\otimes_\Fin$ defined as the pullback
      \[
            \begin{tikzcd}
                  \set{K,Y}_\Fin \arrow[d] \arrow[r]
                  &
                  \mathsf{csk}_\eta Y \arrow[d]
                  \\
                  \Map(K, Y) \arrow[r]
                  &
                  \Map(K, \mathsf{csk}_\eta Y)
            \end{tikzcd}
      \]
      and $\Map(K, Y) \in \mathsf{PreOp}^G$ is defined by $\Map(K,Y)(T) = \Map(K, Y(T))$.

      We observe that $N\set{K,\O}(U)$ consists of a choice of $\mathfrak C_\O$-signatures $(\vect C_v)$ for each $v \in V(U)$ plus choices of maps $\phi_v \in \Map(K, \O(\vect C_v))$.
      This differs from $\Map(K, N\O)$ precisely in that these choices of signatures are made universally for each $v \in V(U)$,
      which is exactly the condition imposed by the pullback.
\end{proof}

\begin{remark}
      In \cite[\S 71.]{CM13b}, $\Omega(T) \otimes_\Fin K \in \sOp$ and $\Omega[T] \otimes_\Fin K \in \mathsf{PreOp}$ were denoted $T[K]$, $\Omega[K,T]$ respectively, and were constructed by hand.
\end{remark}

\begin{remark}
For fixed $K \in \mathsf{sSet}$, the functor
$(-) \otimes_{\mathsf{F}} K
\colon \mathsf{PreOp}^G \to \mathsf{PreOp}^G$
preserves all colimits. % (indeed, it is not hard to build the right adjoint explicitly).

However, for a fixed $X \in \mathsf{PreOp}^G$,
the functor 
$X \otimes_{\mathsf{F}} (-)
\colon \mathsf{sSet} \to \mathsf{PreOp}^G$
does not preserve all colimits.
In particular, this functor can not preserve coproducts since, writing 
$\mathfrak{C} = X(\eta)$ for the $G$-set of objects of $X$,
the image of $X \otimes_{\mathsf{F}} (-)$ is entirely contained in the subcategory
$\mathsf{PreOp}^{G,\mathfrak{C}} \subset
\mathsf{PreOp}^G$
of preoperads with $G$-set of objects $\mathfrak{C}$ and maps which are the identity on objects. 
Instead, one has that the functor 
\[
X \otimes_{\mathsf{F}} (-) \colon
\mathsf{sSet} \to \mathsf{PreOp}^{G,\mathfrak{C}}
\]
does preserve colimits. 
In practice, this means that some standard arguments concerning tensor products can only be applied after adjusting the objects of the relevant preoperads
({\color{red} see later}).
\end{remark}


\begin{remark}\label{COLORTENSGAM REM}
Let $X \to Y$ be any map in $\mathsf{PreOp}^G$
which is the identity on colors and 
$K \in \mathsf{sSet}$. Then the squares below are pushout squares in $\mathsf{sdSet}^G$.
Moreover, whenever $K$ is connected the rightmost horizontal maps are isomorphisms.
\[
\begin{tikzcd}
	X \times K \ar{r} \ar{d} 
	\arrow[dr, phantom, "\ulcorner", very near start] &
	\gamma_! \left( X \times K \right) \ar{r} \ar{d} 
	\arrow[dr, phantom, "\ulcorner", very near start] &
	X \otimes_{\mathsf{F}} K \ar{d}
\\
	Y \times K \ar{r} &
	\gamma_! \left( Y \times K \right) \ar{r} &
	Y \otimes_{\mathsf{F}} K
\end{tikzcd}
\]
\end{remark}


\begin{definition}
	Let $f \colon \mathfrak{C} \to \mathfrak{D}$
	be a map of $G$-sets (of colors).
	We define adjoint functors
\[
	f_{!} \colon
	\mathsf{PreOp}^{G,\mathfrak{C}}
\rightleftarrows
	\mathsf{PreOp}^{G,\mathfrak{D}}
	\colon f^{\**}
\]
via the pushout and pullback squares
(note that $\mathsf{sk}_{\eta} f_! A$ depends only on 
$\mathfrak{C}$ while 
$\mathsf{csk}_{\eta} f^{\**} X$ depends only on
$\mathfrak{D}$)
\[
\begin{tikzcd}
	\mathsf{sk}_{\eta} A \ar{r} \ar{d} \arrow[dr, phantom, "\ulcorner", very near start]  &
	\mathsf{sk}_{\eta} f_! A \ar{d}
&&
	f^{\**} X \ar{r} \ar{d} &
	X \ar{d}
\\
	A \ar{r} & 
	f_! A
&&
	\mathsf{csk}_{\eta} f^{\**} X \ar{r} & 
	\mathsf{csk}_{\eta} X
	\arrow[ul, phantom, "\lrcorner", very near start]
\end{tikzcd}
\]
\end{definition}



\begin{remark}\label{SLIMOD REM}
Noting that for every fibrant 
$\tilde{X} \in \mathsf{PreOp}^G$
any equivalence in $\tilde{X}$ is in the image of a map
$J \to \tilde{X}$, 
a slight modification of the proof of Lemma \ref{INTER_LEM}
shows that for any Segal operad $X$
any equivalence in $X$ is in the image of a countable, contractible
$I \in \mathsf{PreOp}^G$
such that $\eta \amalg \eta \to I$
is a tame cofibration.
\end{remark}




\begin{theorem}
      \label{TAMEMS_THM}
	There is a model structure on 
	$\mathsf{PreOp}^G$,
	called the \textbf{tame model structure},
	such that:
\begin{itemize}
	\item the weak equivalences are the complete equivalences (i.e. detected by inclusion into 
	$\mathsf{sdSet}^G$);
	\item generating cofibrations are given by the maps
	\begin{itemize}
		\item[(TC1)] $G/H \cdot \left(\emptyset \to\Omega[\eta]\right)$ for $H\leq G$;
		\item[(TC2)] $\Omega[C] \otimes_{\mathsf{F}} \left(\partial \Delta[n] \to \Delta[n]\right)$ for $C \in \Sigma_G$, $n \geq 0$;
		\item[(TC3)] 
$\left( Sc[T] \to \Omega[T] \right) 
\square_{\mathsf{F}} 
\left(\partial \Delta[n] \to \Delta[n]\right)$ for $T \in \Omega_G$, $n \geq 0$.
	\end{itemize}
\end{itemize}
Furthermore, one has generating anodyne cofibrations the maps
\begin{itemize}
	\item[(TA1)] $G/H \cdot 
	\left(\Omega[\eta] \to I \right)$ for $H \leq G$,
	and $\Omega[\eta] \to I$ a weak equivalence in $\mathsf{PreOp}$ such that $I(\eta) = \{0,1\}$, $\Omega[\eta] \amalg \Omega[\eta] \to I$ is a tame cofibration, and $I$ is countable;
	\item[(TA2)] $\Omega[C] \otimes_{\mathsf{F}}\left(\Lambda^i[n] \to \Delta[n]\right)$ for $C \in \Sigma_G$, $0 \leq i \leq n$;
	\item[(TA3)] 
$\left( Sc[T] \to \Omega[T] \right) 
\square_{\mathsf{F}} 
\left(\partial \Delta[n] \to \Delta[n]\right)$ for $T \in \Omega_G$, $n \geq 0$.
	\end{itemize}
\end{theorem}

We first have a short lemma.
\begin{lemma}
      \label{TAMECOFCOF_LEM}
      Tame cofibrations and anodyne maps are normal cofibrations and trivial cofibrations in $\mathsf{PreOp}^G$, respectively.
\end{lemma}
\begin{proof}
      $(TC2)$ and $(TC3)$ with $n = 0$, and $(TC1)$ and $(TA1)$ for all $n$, are immediate.
      $(TC2)$ and $(TC3)$ with $n \geq 2$, and $(TA2)$ for all $n$, follow by Equation \eqref{GAMMASH EQ} and Remark \ref{COLORTENSGAM REM}.
      $(TC2)$ with $n = 1$ follows since $\Omega[C] \otimes_\Fin \Delta[1]$ is $G$-normal by \eqref{GAMMASH EQ}, and by hand we see that the map $\Omega[C] \otimes_{\Fin}(\partial \Delta[1] \to \Delta[1])$ is monic.
      For $(TC3)$ with $n=1$, consider the pair of diagrams below, with $K \to L$ in $\sSet$.
      \[
            \begin{tikzcd}
                  Sc[T] \times K \arrow[d] \arrow[r]
                  &
                  \Omega[T] \times K \arrow[d]
                  & % ----------
                  Sc[T] \times K \arrow[r] \arrow[d]
                  &
                  \Omega[T] \times K \arrow[d]
                  \\
                  Sc[T] \otimes_\Fin K \arrow[r] \arrow[d]
                  &
                  \Omega[T] \otimes_\Fin K \arrow[d]
                  & % ----------
                  Sc[T] \times L \arrow[r] \arrow[d]
                  &
                  Q^\times \arrow[r] \arrow[d]
                  &
                  \Omega[T] \times L \arrow[d]
                  \\
                  Sc[T] \otimes_\Fin L \arrow[r]
                  &
                  Q^\otimes
                  & % ----------
                  Sc[T] \otimes_\Fin L \arrow[r]
                  &
                  Q^\otimes \arrow[r, "{(TC3)}"]
                  &
                  \Omega[T] \otimes_\Fin L.
            \end{tikzcd}
      \]
      Using Remark \ref{COLORTENSGAM REM}, we see that all three squares on the left are pushouts (with the bottom square a pushout by definition).
      This rectangle may be factored as on the right with the bottom-left square a pushout (and the top square a pushout by definition),
      and when $L$ is connected, the bottom rectange is a pushout using Remark \ref{COLORTENSGAM REM} and Equation \eqref{GAMMASH EQ}.
      We conclude that the rightmost square is also a pushout.
      But when $K \to L$ is $\partial \Delta[n] \to \Delta[n]$ for $n \geq 1$,
      the top horizontal map is simply $(\partial \Delta[n] \to \Delta[n]) \square (Sc[T] \to \Omega[T])$,
      and hence the bottom horizontal map $(TC3)$ is a G-normal monomorphism that is also a simplicial equivalence, hence a trivial cofibration.
\end{proof}

\begin{proof}[Proof of Theorem \ref{TAMEMS_THM}]
	The existence of the model structure will follow by applying J. Smith's theorem \cite[Thm. 1.7]{Bek00}. Conditions c0 and c2 therein are inherited from $\mathsf{sdSet}^G$
	and the technical ``solution set condition'' c3 follows from
	\cite[Prop. 1.15]{Bek00} since weak equivalences are accessible, being the preimage by $\gamma^{\**}$ if the weak equivalences in 
	$\mathsf{sdSet}^G$ 
	(see \cite[Cor. A.2.6.5]{Lur09} and \cite[Cor. A.2.6.6]{Lur09}).
	
	For c1, we must show that any map $X \to Y$ with the right lifting property against (TC1), (TC2), (TC3) is a weak equivalence.
	Writing $f \colon \mathfrak{C} \to \mathfrak{D}$ for the underlying map of colors,
	consider the factorization $X \to f^{\**}Y \to Y$.
	Note that since maps out of (TC1) depend only on objects and both of (TC2) and (TC3) consist of maps which are identities on objects,
	$X \to Y$ will have the right lifting property against (TC1) iff 
	$f^{\**} X \to Y$ does
	and the right lifting property against 
	(TC2) and (TC3) iff $X \to f^{\**}Y$ does.
	
Note now that $f^{\**} Y \to Y$ has the right lifting proper against all maps 
	$\left(\partial \Omega[T] \to \Omega[T] \right) \times \Delta[n]$.
	Indeed, if $T \simeq G/H \cdot \eta$ is a stick, this is precisely the lifting condition agains (TC1), and otherwise it follows automatically since $\left(\partial \Omega[T] \to \Omega[T] \right) \times \Delta[n]$ is the identity on objects.
	Therefore, the levels 
	$\left(f^{\**} Y \right)_n \to Y_n$ are trivial fibrations in 
	$\mathsf{dSet}^G$, showing that 
	$f^{\**} Y \to Y$ is a dendroidal equivalence, 
	and thus a complete equivalence. 
	
	Since the maps in both of (TC2) and (TC3) are the identity on objects, $X \to Y$ has the right lifting property against these maps iff $X \to f^{\**}Y$ does.
The lifting property against (TC2) then says that the maps
$X(\Omega[C]) \to f^{\**} Y (\Omega[C])$
are trivial Kan fibrations for all $G$-corollas $C \in \Sigma_G$,
and thus so are the maps
$X(Sc[T]) \to f^{\**} Y (Sc[T])$ for all $G$-trees $T \in \Omega_G$.
But it then follows from the lifting property against
(TC3) that the maps 
$X(\Omega[T]) \to f^{\**} Y (\Omega[T])$
are trivial Kan fibrations for all $G$-trees,
showing that $X \to f^{\**} Y$ is a simplicial equivalence, and thus a complete equivalence. 
\[
\begin{tikzcd}
	X(\Omega[T]) \ar{r} \ar[->>]{d}{\sim} &
	X(Sc[T]) \ar{r} \ar[->>]{d}{\sim} &
	\prod_{v \in \boldsymbol{V}_G(T)} X(\Omega[T_v])
	\ar[->>]{d}{\sim}
\\
	f^{\**} Y(\Omega[T]) \ar{r} &
	f^{\**} Y(Sc[T]) \ar{r} \ar{d} &
	\prod_{v \in \boldsymbol{V}_G(T)} f^{\**} Y
	(\Omega[T_v]) \ar{d}
	\arrow[ul, phantom, "\lrcorner", very near start]
\\
	&
	\prod_{(G/H_i \cdot e_i) \in \boldsymbol{E}_G(T)} 
	\mathfrak{C}^{H_i} \ar{r}  &
	\prod_{v \in \boldsymbol{V}_G(T)}
	\prod_{(G/H_i \cdot e_i) \in \boldsymbol{E}_G(T_v)} 
	\mathfrak{C}^{H_i} 
	\arrow[ul, phantom, "\lrcorner", very near start]
\end{tikzcd}
\]
This completes the proof of c1, establishing the existence of the tame model structure.

We now turn to the ``further'' claim considering the claimed generating anodyne cofibrations, i.e., 
we wish to show that the maps in 
(TA1), (TA2), (TA3) satisfy the conditions in
Lemma \ref{SEMICOF LEM}.

We first check condition (i).
The case of maps in (TC1) is tautological.
Since $\Lambda^{i}[n]$ is connected, 
the maps in (TA2) have the form
$\gamma_{!} 
\left( \Omega[C] \times
\left( \Lambda^i[n] \to \Delta[n] \right) \right)$,
and are thus weak equivalences thanks to the pushouts
in Remark \ref{COLORTENSGAM REM}.
As for (TA3), it follows from Remark \ref{COLORTENSGAM REM}
that the maps
$\left( Sc[T] \to \Omega[T] \right) \otimes \partial \Delta[n]$
and 
$\left( Sc[T] \to \Omega[T] \right) \otimes \Delta[n]$
are trivial cofibrations, so that the claim follows from a standard pushout and 2-out-of-3 argument.

We now turn to condition (ii).
The lifting condition against (TA3) says that $J$-fibrant objects are such that the maps $X(\Omega[T]) \to X(Sc[T])$
are trivial fibrations, and thus that such $X$ are Segal operads.
Therefore, by Remark \ref{SEOPDK REM} it suffices to check that $J$-fibrations between Segal operads which are also DK equivalences are in fact trivial fibrations, i.e. that they have the right lifting property against the maps in (TC1),(TC2),(TC3).
Given $X \to Y$ a $J$-fibration with $J$-fibrant $Y$,
the lifting property against (TC3) is tautological since 
(TC3) equals (TA3).
Next, the lifting property against (TA2) says that the maps
$X(\Omega[T]) \to f^{\**} Y(\Omega[T])$
are Kan fibrations, and the DK condition says that these are Kan equivalences,
so that we conclude that such maps have the right lifting property against (TC2).
Lastly, given any lifting problem against a map in (TC1),
essential surjectivity and Remark \ref{SLIMOD REM}
produce a lifting problem against a map in (TA1) which has a solution, providing a solution to the original problem.
\end{proof}


{\color{red} To show that maps in (TA3) are normal cofibrations one can use a pushout of projective cofibrant cubes argument.}


\begin{theorem}
      The identity map is a left Quillen equivalence from the tame model structure to the normal model structure on $\mathsf{PreOp}^G$.
\end{theorem}
\begin{proof}
      We argue by Corollary \ref{SIMPLQUILL COR}.
      Lemma \ref{TAMECOFCOF_LEM} says that the left adjoint preserves cofibrations and anodyne maps.
      The standard adjunction argument then implies that the right adjoint preserves fibrations between fibrant objects.
\end{proof}


\subsection{Comparison with $\sOp^G$}

\begin{lemma}
      \label{OMEGATTAME_LEM}
      For all $T \in \Omega_G$, $\Omega[T] \in \mathsf{PreOp}^G$ is tame cofibrant.
\end{lemma}
\begin{proof}
      From (TC3) with $n=0$, it suffices to show that $Sc[T]$ is tame cofibrant.
      By (TC1), $\amalg_{Ge \in \boldsymbol{E}_G(T)} \Omega[Ge \cdot \eta]$ is tame cofibrant.
      Now, (TC2) with $n=0$ says that $\partial \Omega[C] \to \Omega[C]$ is a tame cofibration,
      and thus the result follows via the following pushout.
      \[
            \begin{tikzcd}
                  \displaystyle{
                    \coprod_{Gv \in \boldsymbol{V}_G(T)} \partial\Omega[T_{Gv}]
                  }
                  \arrow[d] \arrow[r]
                  &
                  \displaystyle{
                    \coprod_{Ge \in \boldsymbol{E}_G(T)} \Omega[Ge \cdot \eta]
                  }
                  \arrow[d]
                  \\
                  \displaystyle{
                    \coprod_{Gv \in \boldsymbol{V}_G(T)} \Omega[T_{Gv}]
                  }
                  \arrow[r]
                  &
                  Sc[T]
            \end{tikzcd}
            \]
\end{proof}


\begin{remark}
      \label{TAUCOFIB_REM}        
      We note that by Lemma \ref{TAUOTIMES_LEM}, $\tau \colon \mathsf{PreOp}^G \to \sOp^G$
      sends the collection of arrows in (TC1) and (TC2) surjectively onto the collection of generating cofibrations \eqref{SSETGENCOF_EQ}:
      \begin{gather*}
            \tau\Big(\varnothing \to G/H \cdot \Omega[\eta]\Big) = \Big(\varnothing \to \Omega(G/H \cdot \eta)\Big),
            \\
            \tau\Big(\Omega[C] \otimes_\Fin (\partial \Delta[n] \to \Delta[n])\Big) =
            \Omega(C) \otimes_\Fin \big(\partial \Delta[n] \to \Delta[n]\Big).
            % = \mathbb F_{\Gamma_C} \Big( \Gamma_C / \Gamma_C \cdot (\partial \Delta[n] \to \Delta[n]) \Big)
      \end{gather*}
      % where $\Gamma_C$ is the graph subgroup of $G \times \Sigma_n$ such that $C \simeq G \cdot C_n / \Gamma$, and
      % $\mathbb F_\Gamma: \sSet^\Gamma \to \sOp^G$ is as in the discussion around \eqref{FGAMMA_EQ}.
      Moreover, as $\tau$ sends (TC3) to isomorphisms, $\tau$ also preserves all cofibrations.
\end{remark}

The following results are adapted from \cite{JT07} (see Proposition 7.15 therein). 


\begin{proposition}
	A cofibration $A \to B$ is a weak equivalence iff it has the left lifting property against all fibrations between fibrant objects.
\end{proposition}

\begin{proof}
	Let $B \xrightarrow{\sim} \tilde{B}$ be a fibrant replacement and
	let $A \xrightarrow{\sim} \tilde{A} \twoheadrightarrow \tilde{B}$
	be a factorization of the composite $A \to \tilde{B}$ 
	as a trivial cofibration followed by a fibration.
	One then has a lift in the diagram
\[
\begin{tikzcd}
	A \ar{r}{\sim} \ar[>->]{d} & \tilde{A} \ar[->>]{d}
\\
	B \ar{r}{\sim} \ar[dashed]{ru} & \tilde{B}
\end{tikzcd}
\]
where the top and bottom horizontal maps are weak equivalences. 
But then the 2-out-of-6 property for weak equivalences says that all maps are weak equivalences.
\end{proof}


\begin{corollary}\label{SIMPLQUILL COR}
An adjunction 
\[
F \colon \mathcal{C}
	\rightleftarrows
\mathcal{D} \colon G
\]
between model categories is a Quillen adjunction
provided that $F$ preserves cofibrations
and $G$ preserves fibrations between fibrant objects.
\end{corollary}


\begin{lemma}
	Let $A \to B$ be a tame cofibration between tame cofibrant objects in $\mathsf{PreOp}^G$, 
	$\mathcal{O} \in \mathsf{sOp}^G$ a $\Sigma$-cofibrant 
	$G$-operad,
	and consider a pushout diagram in $\mathsf{sOp}^G$ of the form
\[
\begin{tikzcd}
	\tau A \ar{r} \ar{d} & \mathcal{O} \ar{d}
\\
	\tau B \ar{r} & \mathcal{P}
\end{tikzcd}
\]
	Then $\mathcal{O} \to \mathcal{P}$ is a $\Sigma$-cofibration and 
\begin{equation}\label{UNITEQUIV EQ}
B \amalg_{A} N \mathcal{O}
	\to 
N \mathcal{P}
\end{equation}
is a weak equivalence.
\end{lemma}

\begin{proof}
	We first consider the case where $A\to B$ is in one of (TC1),(TC2),(TC3). 
	
	The (TC1) case is immediate, 
	since $\mathcal{O} \to \mathcal{O} \amalg G/H \cdot \Omega(\eta)$ is a $\Sigma$-cofibration and
	\eqref{UNITEQUIV EQ}
	is the isomorphism
	$N\mathcal{O} \amalg G/H\cdot \Omega[\eta] \simeq 
	N\left( \mathcal{O} \amalg G/H \cdot \Omega(\eta) \right)$.

	The (TC3) case is also straightforward:
	since $\tau A \to \tau B$ is an isomorphism, one can take 
	$\mathcal{O}=\mathcal{P}$, so that 
	\eqref{UNITEQUIV EQ}
        becomes a section of the map
	$N \mathcal{O} \to B \amalg_{A} N \mathcal{O}$, which is a trivial cofibration (it is a pushout of $A \to B$),
	and 2-out-of-3 hence implies that \eqref{UNITEQUIV EQ} is a weak equivalence.

	The most interesting case is then (TC2), in which case
        Corllary \ref{SIGMAG_COF_COR} and Remark \ref{TAUCOFIB_REM}
        imply that $\mathcal{O} \to \mathcal{P}$ is a $\Sigma$-cofibration,
	and further each of the levels
        $(B \amalg_{A} N \mathcal{O})_n
        \to 
        (N \mathcal{P})_n$
        for $n \geq 0$
        is an equivalence in $\mathsf{dSet}^G$ by (an iteration of)
        Proposition \ref{KEYPR PROP} \todo{come back},        
        showing that \eqref{UNITEQUIV EQ} is in fact a dendroidal equivalence, and thus also a complete equivalence.
        
        We now turn to the case of $A \to B$ a general tame cofibration between tame cofibrant objects.
	As usual, $A \to B$ is a retract of a transfinite composition of pushouts of generating cofibrations.
	Since the conclusions of the result are invariant under retracts,
	we are free to assume that $A \to B$ is a transfinite composite
\[
A = A_0 \to A_1 \to A_2 \to \cdots \to A_{\beta} \to 
colim_{\beta < \kappa} A_{\beta} = B.
\]
where each map $A_{\beta} \to A_{\beta +1}$ is a pushout of a map in one of (TC1),(TC2),(TC3).

Defining $\mathcal{O}_{\beta}$ by replacing $A \to B$ with $A \to A_{\beta}$ in the pushout,
$\mathcal{O} \to \mathcal{P}$ becomes the transfinite composite of the maps $\mathcal{O}_{\beta} \to \mathcal{O}_{\beta + 1}$
and \eqref{UNITEQUIV EQ} becomes
$
colim_{\beta < \kappa} \left( 
N \mathcal{O} \amalg_{N \tau A} N \tau A_{\beta}
	\to 
N \mathcal{O}_{\beta}
\right)
$.
It thus suffices to show, by induction on $\beta < \kappa$, 
that the maps $\mathcal{O}_{\beta} \to \mathcal{O}_{\beta + 1}$ are $\Sigma$-cofibrations and that the maps 
$N \mathcal{O} \amalg_{N \tau A} N \tau A_{\beta}
	\to 
N \mathcal{O}_{\beta}$
are weak equivalences
(that this last condition suffices follows since
filtered colimits of weak equivalences in $\mathsf{PreOp}^G$ are weak equivalences by Lemma \ref{FCOLIM_WE2_LEM} ({\color{red} add this})).
Consider now the following diagrams.
\[
\begin{tikzcd}
	\tau A \ar{r} \ar{d} & \mathcal{O} \ar{d}
&&
	A_{\beta} \amalg_{A} N \mathcal{O}
	\ar[>->]{r} \ar{d}[swap]{\sim} &
	A_{\beta+1} \amalg_{A} N \mathcal{O}
	\ar{d}[swap]{\sim}
\\
	\tau A_{\beta} \ar{r} \ar{d} & \mathcal{O}_{\beta} \ar{d}
&&
	N \mathcal{O}_{\beta} \ar[>->]{r} &
	A_{\beta+1} \amalg_{A_{\beta}} N \mathcal{O}_{\beta} \ar{d}
\\
	\tau A_{\beta + 1} \ar{r} & \mathcal{O}_{\beta + 1}
&&
	&
	N \mathcal{O}_{\beta+1}
\end{tikzcd}
\]
The induction hypothesis states that
$\mathcal{O} \to \mathcal{O}_{\beta}$ is a $\Sigma$-cofibration and that the map
$A_{\beta} \amalg_A N \mathcal{O} \to \mathcal{O}_{\beta}$ is a weak equivalence.
Therefore, $\mathcal{O}_{\beta}$ is $\Sigma$-cofibrant 
and the both vertical maps marked $\sim$ in the rightmost diagram above are weak equivalences 
(this uses the fact that $\mathsf{PreOp}^G$ is left proper),
and thus the induction step will follow provided that the result holds for
the map $A_{\beta} \to A_{\beta + 1}$ and $\mathcal{O}_{\beta}$.
But $A_{\beta} \to A_{\beta + 1}$ is assumed to be a pushout of a map in (TC1),(TC2),(TC3), in which case the result is already known, and thus noting that the result is clearly invariant under pushouts finishes the proof.
\end{proof}

Setting $A = \emptyset $, $\mathcal{O}= \emptyset$ in the previous result yields the following.

\begin{corollary}\label{KEYEQUIV COR}
      If $B \in \mathsf{PreOp}^G$ is tame cofibrant, then 
      $B \to N \tau B$ is a weak equivalence.
\end{corollary}


\begin{proposition}\label{PREQUIEQUIV PROP}
The adjunction
\[
	\tau \colon \mathsf{PreOp}^G_{\text{tame}}
		\rightleftarrows 
	\mathsf{sOp}^G \colon N
\]
is a Quillen equivalence.
\end{proposition}


\begin{proof}
Firstly, note that $N$ preserves and detects weak equivalences.
Indeed, this follows since all objects in the image of $N$ are Segal operads, so that by Remark \ref{SEOPDK REM} a map in the image of $N$ is a weak equivalence iff it is a Dwyer-Kan equivalence;
It is clear that $N$ preserves and reflects fully-faithful maps,
and as we compute that $i_G^{\**}ho(N\O)(G/H)$ is the simplicial set $N j^{\**}\pi_0(\O^H)$
with $j^{\**}\colon \Op \to \Cat$ the forgetful functor,
$N$ preserves and reflects essentially surjective maps.

Next, we show that this is a Quillen adjunction using Corollary \ref{SIMPLQUILL COR}.
We see $\tau$ preserves cofibrations by Remark \ref{TAUCOFIB_REM}.
% The claim that $\tau$ preserves cofibrations follows since
% $\tau$ sends the maps in (TC1) and (TC2) to generating cofibrations of $\mathsf{sOp}^G$ and the maps in (TC3) to isomorphisms.
For the claim that $N$ preserves fibrations between fibrant,
we use a somewhat indirect argument
(though we note that a direct argument is also possible,
by showing that fibrations between fibrant objects in $\mathsf{PreOp}^G$
also satisfy a ``local fibration plus isofibration'' description).
By Corollary \ref{KEYEQUIV COR} and 2-out-of-3, 
one has that for any trivial tame cofibration between tame cofibrant objects
$A \to B$, the map $N \tau A \to N \tau B$ is a weak equivalence, and thus so is $\tau A \to \tau B$.
This shows that $\tau$ sends all maps in (TA1),(TA2),(TA3)
to trivial cofibrations, and since these maps detect fibrations between fibrant objects in $\mathsf{PreOp}^G$, 
the standard adjunction argument shows that 
$N$ indeed preserves fibrations between fibrant objects.

For the Quillen equivalence claim, 
let $B \in \mathsf{PreOp}^G$ be tame cofibrant and
$\mathcal{O} \in \mathsf{sOp}^G$ be fibrant.
We must show that the leftmost map below is a weak equivalence iff its adjoint, which is the rightmost composite, is.
\[
	\tau B \to \mathcal{O},
\qquad
	B \xrightarrow{\sim} N \tau B \to N \mathcal{O}
\]
This is immediate from Corollary \ref{KEYEQUIV COR}
and the fact that $N$ preserves and detects weak equivalences.
\end{proof}







\newpage







\section{Nerves of free extensions are homotopy pushouts}

\begin{definition}
      A map $U \to V$ is
      \begin{enumerate}[label = (\roman*)]
      \item \textit{tall}
      \item a \textit{face}
      \item an \textit{outer face}
      \end{enumerate}
\end{definition}

\begin{proposition}[{cf. \cite[Prop. 3.31]{BP_geo}}]
      \label{TALLOUTER_PROP}
      Any map $U \xrightarrow{\phi} V$ in $\Omega$ has a factorization, unique up to unique isomorphism,
      \[
            U \xrightarrow{\phi^t} W \xrightarrow{\phi^w} V
      \]
      as a tall map followed by an outer face.
\end{proposition}




{\color{red} bla bla something about how this is the same as choosing some nice equivalence classes of dendrices}

\subsection{The characteristic edge lemma}


\begin{notation}
Let $Y \in \mathsf{dSet}^G$ be a $G$-equivariant dendroidal set and 
$y \colon \Omega[U^y] \to Y$
a dendrex, $U^y \in \Omega$.

We write $\langle y \rangle = y\left(  \Omega[U^y] \right)$
and refer to
$\langle y \rangle \subseteq Y$
as the \emph{principal subpresheaf generated by $y$}.

Moreover, if some (and thus any)
non-degenerate representative $y$ is free 
with respect to the $\mathsf{Aut}(U^y)$-action (via precomposition),
we say $y$ and $\langle y \rangle$ are \emph{$\Sigma$-free}.
%If all dendrices $y$ are $\Sigma$-free, we say $Y$ itself is \textit{$\Sigma$-free}.

Given a map of trees $V \to U^y$ we write $\partial_V y$ for the composite to $\Omega[V] \to \Omega[U^y] \xrightarrow{y} Y$.
\end{notation}



\begin{remark}
Note that
$\langle y \rangle = \langle \bar{y} \rangle$
iff $y,\bar{y}$ are both degeneracies of a common non-degenerate dendrex.
In particular, if the chosen representatives $y,\bar{y}$ are both nondegenerate,
there must exist an isomorphism
$\varphi \colon U^y \xrightarrow{\simeq} U^{\bar{y}}$
(which is unique if 
$\langle y \rangle$ is $\Sigma$-free)
such that $y= \bar{y} \circ \varphi$.

\end{remark}


\begin{notation}
Given a $\Sigma$-free $\langle y \rangle$,
a \emph{coherent inner edge set $E^{\langle y \rangle}$ for $\langle y \rangle$}
is a collection of subsets 
$E^y \subseteq \boldsymbol{E}^{\mathsf{i}}(U^y)$
for each non-degenerate representative $y$ of $\langle y \rangle$, and such that 
$E^{\bar{y}}  = \varphi \left(E^y \right)$
for the unique $\varphi$ with $y= \bar{y} \circ \varphi$.
Note that $E^{\langle y \rangle} = \left\{E^y \right\}$
is entirely determined by any of the $E^y$.
%
%Note that for any representatives
%$x,\bar{x}$ one has
%$\langle \partial_{U^x - E^x} x\rangle
%=
%\langle \partial_{U^{\bar{x}} - E^{\bar{x}}} \bar{x}\rangle$,
%so that we abbreviate this presheaf as
%$\partial_{E^{\langle x \rangle}} \langle x\rangle$.
%Moreover, given coherent inner edge sets
%$E^{\langle x \rangle},F^{\langle x \rangle}$
%we write
%$E^{\langle x \rangle} \subseteq F^{\langle x \rangle}$
%if
%$E^{x} \subseteq F^{x}$
%for some (and thus all) non-degenerate representatives $x$.
\end{notation}


\begin{remark}
Recalling that $G$ acts on dendrices by postcomposition, i.e.
$gy$ is the composite
$\Omega[U^y] \xrightarrow{y} Y \xrightarrow{g} Y$
we see that $U^{gy} = U^{y}$.
Moreover, the action extends to principal subpresheaves and
$g \langle y \rangle = \langle g y \rangle$.

As such, if $\langle y\rangle$ is $\Sigma$-free, a coherent inner edge set 
$E^{\langle y \rangle} = \{E^y \subseteq \boldsymbol{E}^{\mathsf{i}}(U^y)\}$
for $\langle y \rangle$
gives rise to a coherent inner edge set 
$g E^{\langle y \rangle} = \{E^y \subseteq \boldsymbol{E}^{\mathsf{i}}(U^{gy})\}$
for $g\langle y \rangle$
with the same edge sets $E^y$.
\end{remark}



The following essentially replicates \cite[Def. 3.1]{BP_edss} as generalized in \cite[Rem. 3.7]{BP_edss},
except with dendrices
$y \colon \Omega[U^y] \to Y$
mostly replaced with the principal presheaves
$\langle y \rangle \subseteq Y$. 
The reformulation of (Ch0.2) and the descending chain condition
are discussed in Remarks \ref{CH02 REM}, \ref{DCC REM}.

\begin{definition}\label{CHAREDGE DEF}
Let $f\colon X \to Y$ be a monomorphism in 
$\mathsf{dSet}^G$ and 
$\left\{ \langle y \rangle\right\}$
a set of $\Sigma$-free principal subpresheaves of $Y$. 
Suppose further that 
$\left\{ \langle y \rangle \right\}$ is equipped 
with a poset structure compatible with the natural $G$-action
and which satisfies the descending chain condition.
For each $\langle y \rangle$ denote
\[
X_{< \langle y \rangle} = X \cup 
\bigcup_{\langle\bar{y}\rangle < \langle y \rangle} \langle \bar{y} \rangle
\]
Given a coherent inner edge set 
$
\Xi^{\langle y \rangle} =
\left\{ \Xi^y \subseteq \boldsymbol{E}^{\mathsf{i}}(U^y)\right\}$,
non-degenerate representative
$y \colon \Omega[U^y] \to Y$, and a subface $V \hookrightarrow U^y$,
we write
$\Xi^y_V = \Xi^y \cap \boldsymbol{E}^{\mathsf{i}}(V)$.

We say
$
\left\{ \Xi^{\langle y \rangle} \right \} 
%=\left\{ \Xi^y \subseteq \boldsymbol{E}^{\mathsf{i}}(U^y)\right\}
$
is a \emph{characteristic inner edge collection} 
of $\left\{ \langle y \rangle \right\}$ with respect to $X$ if
for some (and thus any) choice of non-degenerate representatives
$y\colon \Omega[U^y] \to Y$ one has that:
\begin{enumerate}
\item[(Ch0.1)] $y \colon \Omega[U^y] \to Y$ is injective away from
$y^{-1}\left( X_{< \langle y \rangle} \right)$; 
\item[(Ch0.2)]
$\{\langle y\rangle\}$ and
$\{\Xi^{\langle y \rangle}\}$ are $G$-equivariant, in the sense that
$g\langle y\rangle \in \{\langle y\rangle\}$ and 
$g \Xi^{\langle y \rangle} =
\Xi^{g \langle y \rangle}$,
i.e. $\Xi^y = \Xi^{gy}$;
\item[(Ch1)] if $V \hookrightarrow U^y$ is an outer face and $\Xi^y_V = \emptyset$,
then $\langle \partial_V y \rangle \subseteq X_{< \langle y \rangle}$;
\item[(Ch2)] if $V \hookrightarrow U^y$ is any face and
$\langle \partial_{V-\Xi^y_V} y\rangle \subseteq X$,
then
$\langle \partial_V y\rangle \subseteq X_{< \langle y \rangle}$;
\item[(Ch3)] if $\langle \bar{y} \rangle \not \geq \langle y \rangle$,
$V \hookrightarrow U^y$,
and
$\langle \partial_{V-\Xi^y_V} y\rangle \subseteq \langle \bar{y} \rangle$,
then
$\langle \partial_V y\rangle \subseteq X_{< \langle y \rangle}$.
\end{enumerate}
\end{definition}


\begin{remark}\label{CH02 REM}
In \cite[Rem. 3.7]{BP_edss} the role of each presheaf 
$\langle y \rangle$ is played by a special chosen representative,
which we here denote by
$y^{\mathsf{pl}} \in \langle y \rangle$. 
The motivation for this is that in some key examples, such as in \cite[Ex. 3.9]{BP_edss}, one can choose preferred ``planar representatives'',
allowing for a pictorial depiction of the dendrices and poset as in
\cite[Fig. 3.1]{BP_edss}. 

There is then a bijection $\{\langle y \rangle\} = \{ y^{\mathsf{pl}}\}$ between principal presheaves and the set of representatives, but while the former has a $G$-action the latter a priori does not (as $gy^{\mathsf{pl}}$ may not be planar). Translating the $G$-action along this bijection one has that the action of $g$ on $y^{\mathsf{pl}}$ is
$(g y^{\mathsf{pl}})^{\mathsf{pl}}$ and (ii),(iii) in 
(Ch0.2) of \cite[Rem. 3.7]{BP_edss} precisely 
encode this action on planar representatives.
\end{remark}

\begin{remark}\label{DCC REM}
Recall that a poset satisfies the descending chain condition if there are no infinite descending chains or, equivalently, if any subset has a minimal element. As such, while the proof of \cite[Lemma 3.4]{BP_edss}
assumed the poset $\{\langle y \rangle\}$ was finite,
since that proof follows by iteratively adding elements to $G$-equivariant convex subsets of the poset (cf. the last paragraph of the proof), the argument generalizes to any poset satisfying the descending chain condition.
\end{remark}

\begin{lemma}[{cf. \cite[Lemma 3.4]{BP_edss}}]
      \label{CHAREDGE LEM}
If
$
\left\{ \Xi^{\langle y \rangle} \right \} 
%=\left\{ \Xi^y \subseteq \boldsymbol{E}^{\mathsf{i}}(U^y)\right\}
$
is a \emph{characteristic inner edge collection} 
of $\left\{ \langle y \rangle \right\}$ with respect to $X$ then
\begin{equation}\label{CHAREDGE EQ}
X \to X \cup \bigcup_{\{\langle y \rangle\}} \langle y \rangle
\end{equation}
is $G$-inner anodyne.
\end{lemma}



\subsection{Nerves of free extensions are homotopy pushouts}



The following is an equivariant analogue of \cite[Prop. 3.2]{CM13b}.

\begin{proposition}\label{KEYPR PROP}
Fix a $G$-set $\mathfrak{C}$ of colors,
and suppose that 
$\mathcal{O} \in \mathsf{Op}^{G,\mathfrak{C}}$
and
$B \in \mathsf{Sym}^{G,\mathfrak{C}}$
are
$\Sigma$-cofibrant.
Then, writing
\begin{equation}\label{PCOPROD EQ}
\mathcal{P} = \mathcal{O} \amalg_{\mathsf{F}} \mathbb{F} B
\end{equation}
for the fixed color coproduct (i.e. the coproduct in $\Op^{G, \mathfrak C}$),
one has that the induced map
(here we write $\zeta$ for the composite
$\Sigma_{\mathfrak{C}}^{op} \to \Sigma^{op} \to \Omega^{op}$
and $\zeta_!$ for the left Kan extension)
\begin{equation}\label{ANODYNEMAP EQ}
N \mathcal{O} \amalg_{\zeta_!B (\eta)} \zeta_!B \to N \mathcal{P}
\end{equation}
is $G$-inner anodyne.
%
%Suppose that $\mathcal{O} \in \mathsf{Op}^{G,\mathfrak{C}}$
%is $\Sigma$-cofibrant.
%Further, let $C \in \Sigma_G$ be any $G$-corolla and consider 
%a pushout in $\mathsf{Op}^{G}$ of the form
%\begin{equation}\label{PUSHOUTPROP EQ}
%\begin{tikzcd}
%	\partial \Omega(C) \ar{r} \ar{d} & \mathcal{O} \ar{d}
%\\
%	\Omega(C) \ar{r} & \mathcal{P}.
%\end{tikzcd}
%\end{equation}
\end{proposition}


The proof of this result will follow from an instance of 
Lemma \ref{CHAREDGE LEM}. 
Before proving the result, we need to understand $\mathcal{P}$ itself.

First, it as $\partial \Omega[T] \to \Omega[T]$ is a generating cofibration in $\sOp^G$,
$\O \to \P$ is a cofibration with $\Sigma$-cofibrant source, and hence by Corollary \ref{SIGMAG_COF_COR},
$\P$ is $\Sigma$-cofibrant.

Second, noting that the coproduct \eqref{PCOPROD EQ}
is a particular case of \eqref{OU EQ} with $X=\emptyset$,
Proposition \ref{FILTPUSH PROP}
(where we note that the condition $X=\emptyset$ implies
$Q^{in}_T[u] = \emptyset$ in that result)
then implies that  
\begin{equation}\label{PUSHOPPR EQ}
	\mathcal{P}(C) = 
	\coprod_{
	[T] \in \mathsf{Iso}
	\left( \Omega_{\mathfrak{C}}^a \downarrow C \right)
	}
	\left(
		\prod_{v \in V^{ac}(T)} \mathcal{O}(T_v)
	\times
		\prod_{v \in V^{in}(T)} B(T_v)
	\right)
	\cdot_{\mathsf{Aut}_{\Omega^a_{\mathfrak{C}}}(T)} \mathsf{Aut}_{\Sigma_{\mathfrak{C}}}(C)
\end{equation}



The decomposition \eqref{PUSHOPPR EQ}
will be the key to verifying the 
characteristic edge conditions in Definition \ref{CHAREDGE DEF}.
To do so, we will first find it useful to discuss a number of special types of dendrices and principal subpresheaves of $N \mathcal{P}$, suggested by \eqref{PUSHOPPR EQ}.
%
Recall that, by the strict Segal condition characterization of nerves \cite[Cor. 2.7]{CM13a},
a dendrex $p \colon \Omega[U] \to N \mathcal{P}$
is uniquely specified by the tree $U \in \Omega$ together with a choice of operations
$\{p_v \in \mathcal{P}(U_v)\}_{v \in \boldsymbol{V}(T)}$.


\begin{definition}
A dendrex $p\colon \Omega[U] \to N \mathcal{P}$ 
is called:
\begin{itemize}
\item \emph{elementary} if for each vertex $U_v \hookrightarrow U$
it is $\langle \partial_{U_v} p\rangle \subseteq \mathcal{O} \amalg B$; % \O \amalg_{\mathfrak{C}}B$;
\item \emph{alternating} if $U \in \Omega^a$ is an alternating tree
and for each active (resp. inert) vertex 
$U_v \hookrightarrow U$ it is
$\langle \partial_{U_v} p \rangle \subseteq \O$
(resp. $\langle \partial_{U_v} p \rangle \subseteq B$);
\item \emph{canonical} if it is non-degenerate and has a degeneracy which is alternating.
\end{itemize}
\end{definition}


\begin{definition}
Let $\langle p \rangle \subseteq N \mathcal{P}$ be 
a principal subpresheaf. 
We say $\langle p \rangle$ is:
\begin{itemize}
\item \emph{unital} if there is a representative
$p\colon \Omega[U] \to N \mathcal{P}$ with $U=\eta$ the stick tree;
\item \emph{reduced} if there is a representative
$p\colon \Omega[U] \to N \mathcal{P}$ with $U \in \Sigma$, i.e. with $U$ a corolla;
\item \emph{elementary} 
if there is an elementary representative
$p\colon \Omega[U] \to N \mathcal{P}$;
\item \emph{canonical} 
if there is a canonical (or, equivalently, alternating) representative
$p\colon \Omega[U] \to N \mathcal{P}$.
\end{itemize}
\end{definition}



\begin{remark}
A dendrex is elementary iff its degeneracies are elementary,
so the definition of elementary subpresheaf does not depend on the choice of representative.
\end{remark}


\begin{notation}
Recalling that any tree $U$
has an associated corolla $\mathsf{lr}(U)$, 
we abbreviate
$\partial_r p = \partial_{\mathsf{lr}(U)}p$
and call
$\langle \partial_r p \rangle$
the \emph{reduction} of  
$\langle p \rangle$.
\end{notation}




\begin{remark}\label{PUSHOPPRRST REM}
Equation \eqref{PUSHOPPR EQ} 
%can be restated as saying 
implies
that for each reduced principal subpresheaf
$\langle r \rangle \subseteq N \mathcal{P}$
there exists an alternating dendrex $a$ of $N \mathcal P$, 
\emph{unique up to isomorphism}, 
such that
$\langle \partial_r a \rangle = \langle r \rangle$.
%
Moreover, $\langle a \rangle$ is thus the only canonical subpresheaf whose reduction is 
$\langle r \rangle$,
and we write
$\langle r \rangle_{\chi} = \langle a \rangle$
to denote this.

Lastly, note that one thus has that 
$\langle p \rangle$ is canonical iff
$\langle p \rangle = \langle \partial_r p \rangle_{\chi}$.
\end{remark}


\begin{remark}\label{UNITALCASE REM}
A reduced subpresheaf $\langle r \rangle$
is unital iff 
$\langle r \rangle_{\chi}$ is unital, in which case
$\langle r \rangle = \langle r \rangle_{\chi}$.
\end{remark}



\begin{remark}\label{ELEMLABEL REM}
If $e \colon \Omega[U^e] \to N \mathcal{P}$
is an elementary dendrex, 
the tree $U^e$ can be naturally regarded as an
$\{\O,B\}$-labeled tree by labeling each vertex $U^e_v \into U^e$ according to whether
it is $\langle \partial_{U^e_v} e\rangle \subseteq \O $ or
$\langle \partial_{U^e_v} e\rangle \subseteq B$.

By ({\color{blue} the alternating tree analogue of})
\cite[Prop. 5.48]{BP_geo}, there is hence an unique alternating tree $U^a$ together with a tall planar $B$-inert label map $U^a \to U^e$,
and it then follows that
$\partial_{U^a} e$ is an alternating dendrex so that
$\langle \partial_{U^a} e \rangle = \langle \partial_r e \rangle_{\chi}$.
%
In particular, this shows that
$\langle \partial_r e \rangle_{\chi} \subseteq 
\langle e \rangle$.
\end{remark}



\begin{definition}\label{XIEDGES DEF}
Let $e \colon \Omega[U^e] \to N \mathcal{P}$ be a non-degenerate elementary dendrex. 
We write
$\Xi^e \subseteq \boldsymbol{E}^{\mathsf{i}}(U^e)$
for the subset of inner edges of $U^e$ which are adjacent to at least one $B$-labeled vertex.
\end{definition}




\begin{remark}\label{XIEREDEF REM}
Let $e \colon \Omega[U^e] \to N \mathcal{P}$ be a non-degenerate elementary dendrex, 
$U^a \to U^e$ be as in Remark \ref{ELEMLABEL REM}, 
and write $a = \partial_{U^a} e$.
Since $U^a$ is alternating, all of its inner edges are adjacent to a 
$B$-labeled vertex. 
Therefore, the fact that $U^a \to U^e$ is a tall $B$-inert label map
implies that $\Xi^e$ consists of those inner edges which are in the image of $U^a$.
\end{remark}



\begin{proposition}\label{CANIFFXIE PROP}
Let $e \colon \Omega[U^e] \to N \mathcal{P}$ be a non-degenerate elementary dendrex.
%
Then $\langle e \rangle$
is canonical iff $\Xi^e = \boldsymbol{E}^{\mathsf{i}}(U)$.
\end{proposition}

\begin{proof}
We use the notation in Remark \ref{XIEREDEF REM}.
Since $\langle a\rangle = \langle \partial_r e \rangle_{\chi}$, 
$\langle e \rangle$
is canonical 
iff
 $\langle a\rangle = \langle e \rangle$, i.e.
iff
$U^a \to U^e$ is a degeneracy. 
But since a map of trees is a degeneracy iff it is tall and surjective,
it follows that $U^a \to U^e$ is a degeneracy
iff its image includes all inner edges, i.e. iff $\Xi^e = \boldsymbol{E}^{\mathsf{i}}(U)$.
\end{proof}



\begin{remark}\label{WHENLRINN REM}
If $U\neq \eta$ is not the stick tree,
then $\mathsf{lr}(U) \simeq U - \boldsymbol{E}^{\mathsf{i}}(U)$
is the inner face removing all inner edges.
\end{remark}



\begin{lemma}\label{ELEMEXIST LEM}
For any principal subpresheaf $\langle p \rangle \subseteq N \mathcal{P}$
there exists an elementary subpresheaf
$\langle e \rangle \subseteq N \mathcal P$, 
non-degenerate representative 
$e \colon \Omega[U^e] \to N \mathcal{P}$,
and a subset $E \subseteq \Xi^{e}$
such that
 $\partial_{U^e-E} e$ is non-degenerate and 
$\langle p \rangle = \langle \partial_{U^e-E} e \rangle$.
In particular,  
$\langle p \rangle \subseteq \langle e \rangle$.
\end{lemma}




\begin{proof}
Let $p\colon \Omega[U] \to N \mathcal{P}$
be a non-degenerate representative. We first build $e$.

For each vertex $U_v \hookrightarrow U$, write
$p_v = \partial_{U_v} p$
and,
noting that $\langle p_v \rangle$ is reduced,
we further write 
$e_v \colon \Omega[U^e_v] \to N \mathcal{P}$
for some canonical representative of 
$\langle p_v \rangle_{\chi}$
(cf. Remark \ref{PUSHOPPRRST REM}).


The identity
$\langle p_v \rangle = \langle \partial_r e_v \rangle$
implies that
$U_v \simeq \mathsf{lr}(U^e_v)$,
so that by choosing tall maps $U_v \to U^e_v$ 
one obtains a $U$-substitution datum
\cite[Def. 3.38]{BP_geo}
which by 
\cite[Prop. 3.41]{BP_geo}
can be assembled into a tree $U^e$
together with a tall map
$U \to U^e$ such that for every vertex $U_v$
the ``tall map followed by outer face'' factorization from Proposition \ref{TALLOUTER_PROP} of
the composite
$U_v \to U \to U^e$
is given by
$U_v \to U^e_v \to U^e$.
Since each vertex of $U^e$ is in exactly one of the outer trees $U^e_v$,
we define $e \colon \Omega[U^e] \to N \mathcal{P}$
as the unique dendrex such that
$\partial_{U^e_v} e = e_v$.
Note that since the $e_v$ are non-degenerate 
%and canonical, $e$ is non-degenerate and elementary. %
then so is $e$.


Since $\partial_{U}e = p$ was chosen to be non-degenerate,
%is non-degenerate by assumption,
it remains to show that
$U \to U^e$ identifies $U \simeq U^e - E$
for some $E \subseteq \Xi^{e}$.
%But since $p$ is non-degenerate,
Since $\langle p_v \rangle, \langle p_v \rangle_{\chi}$ are non-unital
(by assumption on $p$ and Remark \ref{UNITALCASE REM}),
the $U^e_v$ are not stick trees,
and Remark \ref{WHENLRINN REM} implies
$U \simeq U^e-E$ for 
$E = \amalg_{v \in \boldsymbol{V}(U)} \boldsymbol{E}^{\mathsf{i}}(U^e_v)$.
%
That
$E \subseteq \Xi^{e}$, i.e. that any edge in $E$ is adjacent to a $B$-labeled vertex,
follows from Proposition \ref{CANIFFXIE PROP}.
\end{proof}


\begin{lemma}\label{FORSAKEN LEM}
Suppose $e \colon \Omega[U^e] \to N \mathcal{P}$ is elementary
and $\langle \partial_r e\rangle$ is not unital.
Then there exists an inner face map
$U^c \to U^e$ such that $\partial_{U^c} e$ is canonical.

% In particular,
% $\langle \partial_{U^c} e \rangle = \langle \partial_r e \rangle_{\chi}$,
% and hence 
% $\langle \partial_r e \rangle_{\chi} \subseteq \langle e \rangle$
% \todo{don't we already know this from Remark \ref{ELEMLABEL REM}?}.
\end{lemma}


\begin{proof}
Let $U^a \to U^e$ be as in Remark \ref{ELEMLABEL REM}.

Writing
$a = \partial_{U^a} e$,
$r = \partial_r e$, 
and letting
$c \colon \Omega[U^c] \to N \mathcal{P}$
be a canonical representative of
$\langle a \rangle = \langle r \rangle_{\chi}$,
%and hence by Remark \ref{PUSHOPPRRST REM}
one has that $a$ is a degeneracy of $c$,
i.e. there is a degeneracy map $U^a \to U^c$
such that $\partial_{U^a} c = a$.
Since $\langle r \rangle = \langle \partial_r e\rangle$ is not unital,
%by assumption,
neither is $\langle r \rangle_{\chi}$ (cf. Remark \ref{UNITALCASE REM}), 
so that $U^c$ can not be the stick tree $\eta$,
and thus $U^a \to U^c$ has a section which is an inner face 
(this follows from \cite[Cor. 5.38]{Per18} since no edge of $U^c$ is both a root and a leaf).
But then the composite 
$U^c \to U^a \to U$
must be a face (or else $c = \partial_{U^c} e$ would be degenerate) and is tall, and is hence an inner face.%, finishing the proof.
\end{proof}



\begin{lemma}\label{UNIQINAN LEM}
Let 
$c \colon \Omega[U^c] \to N \mathcal{P}$ 
be a non-unital canonical dendrex,
$e \colon \Omega[U^{e}] \to N \mathcal{P}$
an elementary dendrex,
and $\mathsf{lr}(U^c) \to U^{e}$ a tall map.


Then, 
if the solid diagram below commutes, there exists a tall dashed map making the diagram commute.
%Moreover, if $C \to U^{e}$ is tall the dashed map is also tall.
%if $e$ is canonical the dashed map is an isomorphism.
\begin{equation}\label{UNIQINAN EQ}
\begin{tikzcd}
	\Omega\left[\mathsf{lr}(U^c)\right] \ar{r} \ar{rd}&
	\Omega[U^c] \ar{r}{c} \ar[dashed]{d} &
	N \mathcal{P}
\\
	 &
	\Omega[U^{e}] \ar{ru}[swap]{e} 
\end{tikzcd}
\end{equation}
\end{lemma}

\begin{remark}
The requirement that $\langle c \rangle$ is non-unital is essential, as  there may exist non-unital $\langle e \rangle \subseteq N \O$
such that $\langle \partial_r e\rangle = \langle \partial_r c\rangle$ is unital, in which case no dashed arrow as in \eqref{UNIQINAN EQ} can exist.
\end{remark}


\begin{proof}
Since commutativity of \eqref{UNIQINAN EQ} implies
$\langle \partial_r e \rangle =
\langle \partial_r c \rangle$,
which is not unital by assumption, 
by Lemma \ref{FORSAKEN LEM} there is an inner face $U^{\bar{c}} \to U^{e}$
such that $\bar{c} = \partial_{U^{\bar{c}}} e$ is canonical.

By definition of canonical dendrex, there are
degeneracies
$U^a \to U^c$,
$U^{\bar{a}} \to U^{\bar{c}}$
with $U^a,U^{\bar{a}}$
alternating trees
and such that the composites 
$\Omega[U^a] \to \Omega[U^c] \xrightarrow{c} N \mathcal{P}$,
$\Omega[U^{\bar{a}}] \to \Omega[U^{\bar{c}}] \xrightarrow{\bar{c}} N \mathcal{P}$  
are alternating dendrices. And since $\mathsf{lr}$ sends tall maps to isomorphisms, we can form the diagram
\[
\begin{tikzcd}
	\Omega[\mathsf{lr}(U^c)] \ar{r}{\simeq} \ar{rd}[swap]{\simeq} &
	\Omega\left[\mathsf{lr}(U^a)\right] \ar[dashed]{d}[swap]{\simeq} \ar{r} &
	\Omega[U^a] \ar[dashed]{d}[swap]{\simeq} \ar{r} &
	\Omega[U^c] \ar{r}{c} \ar[dashed]{d}[swap]{\simeq} &
	N \mathcal{P}
\\
	 &
	\Omega\left[\mathsf{lr}(U^{\bar{a}})\right] \ar{r} &
	\Omega[U^{\bar{a}}] \ar{r} &
	\Omega[U^{\bar{c}}] \ar{ru}[swap]{\bar c} \ar{r} &
	\Omega[U^{e}] \ar{u}[swap]{e}. &
\end{tikzcd}
\]
We will argue that all dashed vertical isomorphisms exist.
That the first vertical isomorphism exists is trivial.
The existence of the second vertical isomorphism follows from
\eqref{PUSHOPPR EQ} which implies that, for $a,\bar{a}$ alternating dendrices, all isomorphisms 
$\partial_r a \simeq \partial_r \bar{a}$
are induced from an isomorphism $a \simeq \bar{a}$ \todo{how?}.
Lastly, the existence of the third isomorphism follows 
from the fact that the factorization of degenerate dendrices through non-degenerate dendrices is unique up to (unique) isomorphism \cite[Prop. 5.62]{Per18}.
%
%The moreover claim is clear. 
\end{proof}


\begin{lemma}\label{UNIQINAN2 LEM}
Let 
$e \colon \Omega[U^e] \to N \mathcal{P}$ 
be a non-degenerate elementary dendrex,
$\bar{e} \colon \Omega[U^{\bar{e}}] \to N \mathcal{P}$
an elementary dendrex,
and 
$U^e-E \to U^{\bar{e}}$ a map where $E \subseteq \Xi^e$.

Then, 
if the solid diagram below commutes, there exists a dashed map making the diagram commute.
\begin{equation}\label{UNIQINAN2 EQ}
\begin{tikzcd}
	\Omega[U^e-E] \ar{r} \ar{rd}&
	\Omega[U^e] \ar{r}{e} \ar[dashed]{d} &
	N \mathcal{P}
\\
	 &
	\Omega[U^{\bar{e}}] \ar{ru}[swap]{\bar{e}} 
\end{tikzcd}
\end{equation}
\end{lemma}


\begin{proof}
We abbreviate $U' = U^e -E$. 
Note first that, 
by applying the
``tall map followed by outer face'' factorization to
$U' \to U^{\bar{e}}$
to obtain
$U' \to \tilde{U} \to U^{\bar{e}}$,
the dendrex $\partial_{\tilde{U}} \bar{e}$
is still elementary (being an outer face of an elementary dendrex),
so we reduce to the case where $U' \to U^{\bar{e}}$ is a tall map.

For each vertex $U'_v \hookrightarrow U'$ we apply the 
``inner face followed by outer face factorization''
to the composites
$U'_{v} \hookrightarrow U' \to U^{e}$,
$U'_{v} \hookrightarrow U' \to U^{\bar{e}}$
to get
$U'_{v} \to U_{v}^e \hookrightarrow U^e$,
$U'_{v} \to U_{v}^{\bar{e}} \hookrightarrow U^{\bar{e}}$
and, further writing
$e_v = \partial_{U^e_v} e$,
$\bar{e}_v = \partial_{U^{\bar{e}}_v} \bar{e}$, 
we obtain solid diagrams
\begin{equation}\label{UNIQINAN2A EQ}
\begin{tikzcd}
	\Omega[U'_v] \ar{r} \ar{rd}&
	\Omega[U^e_v] \ar{r}{e_v} \ar[dashed]{d} &
	N \mathcal{P}
\\
	 &
	\Omega[U^{\bar{e}}_v] \ar{ru}[swap]{\bar{e}_v} 
\end{tikzcd}
\end{equation}
We now claim that $e_v$ is canonical. Indeed, $e_v$ is non-degenerate elementary since it is an outer face of $e$, which is also non-degenerate elementary. And since all inner edges of $U^e_v$ are in 
$E \subseteq \Xi^e$, they are all adjacent to $B$-labeled vertices, 
so $e_v$
is indeed canonical by
Proposition \ref{CANIFFXIE PROP}.


Since $\langle e_v \rangle$ is non-unital
(or $U'_v \to U_v^e$ would be a degeneracy),
by Lemma \ref{UNIQINAN LEM} there is a tall dashed arrow in \eqref{UNIQINAN2A EQ} for each $v \in \boldsymbol{V}(U')$,
i.e. a $U^e$-substitution datum. Thus
by
\cite[Prop. 3.41]{BP_geo}
we obtain the desired dashed arrow in \eqref{UNIQINAN2 EQ}.
\end{proof}



\begin{corollary}\label{MINELEMSH COR}
If $e \colon \Omega[U^e] \to N \mathcal{P}$ is a non-degenerate elementary dendrex
and $E \subseteq \Xi^e$,
then 
$\langle e\rangle$ is the smallest elementary subpresheaf
containing $\langle \partial_{U^e - E} e\rangle$,
i.e. if 
$\langle \partial_{U^e - E} e\rangle
\subseteq \langle \bar{e} \rangle$
with $\langle \bar{e} \rangle$
elementary then 
$\langle e\rangle
\subseteq \langle \bar{e} \rangle$.
\end{corollary}

\begin{proof}
$\langle \partial_{U^e - E} e\rangle
\subseteq \langle \bar{e} \rangle$
yields the diagram \eqref{UNIQINAN2 EQ}
and the dashed arrow therein shows
$\langle e\rangle
\subseteq \langle \bar{e} \rangle$.
\end{proof}



\begin{corollary}\label{CANCHAR COR}
An elementary subpresheaf $\langle e \rangle$
is canonical iff
it is the smallest elementary subpresheaf containing
$\langle \partial_r e \rangle$.
\end{corollary}



\begin{proof}
As noted in Remark \ref{PUSHOPPRRST REM}, 
$\langle e \rangle$ is canonical iff
$\langle e \rangle = \langle \partial_r e \rangle_{\chi}$.
If $\langle \partial_r e \rangle$ is unital, then 
$\langle \partial_r e \rangle = \langle \partial_r e \rangle_{\chi}$,
which is elementary (cf. Remark \ref{UNITALCASE REM}), so the claim is clear.
%
Otherwise, letting 
$c \colon \Omega[U^c] \to N \mathcal{P}$
be a canonical representative of
$\langle \partial_r e \rangle_{\chi} $,
one has
$\Xi^c = \boldsymbol{E}^{\mathsf{i}}(U^c)$
and
$\langle \partial_r e \rangle = 
\langle \partial_{U^c-\Xi^c} c \rangle$
so, by Corollary \ref{MINELEMSH COR},
$\langle \partial_r e \rangle_{\chi} $
is the smallest elementary containing $\langle \partial_r e \rangle$.
\end{proof}



\begin{corollary}\label{ISODIFCL COR}
Suppose $N \mathcal{P}$ is $\Sigma$-free
and let $e \colon \Omega[U^e] \to N \mathcal{P}$
be an elementary non-degenerate dendrex and
$E,E' \subseteq \Xi^e$.
Then if 
$\langle \partial_{U^e-E} e \rangle
=
\langle \partial_{U^e-E'} e \rangle$
it must be $E = E'$.
\end{corollary}



\begin{proof}
      If 
$\langle \partial_{U^e-E} e \rangle
=
\langle \partial_{U^e-E'} e \rangle$
then one can find a solid diagram as below
\[
\begin{tikzcd}
	\Omega[U^e-E] \ar{d}[swap]{\simeq} \ar{r} &
	\Omega[U^e] \ar{r}{e} \ar[dashed]{d} &
	N \mathcal{P}
\\
	\Omega[U^e-E'] \ar{r} &
	\Omega[U^e] \ar{ru}[swap]{e} &
\end{tikzcd}
\]
and thus by Lemma \ref{UNIQINAN2 LEM} one can also find the vertical dashed arrow.
%
But since $N \mathcal{P}$ is $\Sigma$-free,
the only possibility is for the dashed arrow to be the identity,
so that $U^e-E=U^e-E'$ and $E=E'$.
\end{proof}



\begin{proof}[Proof of Proposition \ref{KEYPR PROP}]

We will verify the characteristic edge conditions in Definition \ref{CHAREDGE DEF}.

We set $X = N \mathcal{O} \amalg_{\zeta_!B (\eta)} \zeta_!B$.  % B \cup N\mathcal{O}$
and the $G$-poset of principal subpresheaves is formed by the 
elementary subpresheaves 
$\langle e \rangle$
under inclusion, and the characteristic inner edge sets
$\Xi^{\langle e \rangle} = \left\{\Xi^{e}\right\}$ are as given in Definition \ref{XIEDGES DEF}.
Note that by Lemma \ref{ELEMEXIST LEM}
every dendrex of $N \mathcal{P}$ is in some 
$\langle e \rangle$, so that
\eqref{CHAREDGE EQ} is indeed
$N \mathcal{O} \amalg_{\zeta_!B (\eta)} \zeta_!B \to N \mathcal{P}$.
%and moreover $\P$ being $\Sigma$-cofibrant implies that $N \mathcal P$ is $\Sigma$-free by Proposition \ref{SGS_COF_PROP}.



Let $e\colon \Omega[U^e] \to N \mathcal{P}$
be a non-degenerate elementary dendrex. Note the following: 
\begin{itemize}
\item[(a)] if $\Xi^e = \emptyset$ then 
either all vertices of $U^e$ are $\O$-labeled, i.e. $e \in N \mathcal{O}$, or $U^e$ is a $B$-labeled corolla, i.e. $e \in B$.
In other words, $\Xi^e = \emptyset$ iff 
$\langle e \rangle \subseteq X = N \mathcal{O} \amalg_{\zeta_!B (\eta)} \zeta_!B$.
\item[(b)] any outer face of $e$ is again elementary,
as is any inner face $\partial_{U^e-f} e$ such that $f \not \in \Xi^e$
(since then both vertices adjacent to $f$ are $\O$-labeled).
Therefore, by Corollary \ref{MINELEMSH COR},
we see that a face of $e$ is \emph{not} in
some elementary $\langle \bar{e} \rangle \subsetneq \langle e \rangle$
iff it is of the form
$\partial_{U^e - E} e$
for some $E \subseteq \Xi^e$.
\end{itemize}


We now check the characteristic edge conditions. (Ch0.2) is clear.

For (Ch1), by (b) any proper outer face of $e$ is in $X_{<\langle e\rangle}$, so we need only consider the case of
$V=U^e$ with $\Xi^e=\emptyset$, in which case
$\langle e \rangle \subseteq X \subseteq X_{<\langle e\rangle}$ by (a).

For (Ch2),(Ch3), by (a) and the first half of (b) one needs only consider the case of
$V \simeq U^e - E$ where $E \subseteq \Xi^e$ and $\Xi^e \neq \emptyset$.
But then
$V - \Xi^e_V \simeq U^e- \Xi^e$
and by the second half of (b)
one has
$\langle \partial_{U^e-\Xi^e}e \rangle \subseteq  X_{<\langle e\rangle}$
iff
$\langle e \rangle \subseteq  X_{<\langle e\rangle}$
iff
$\langle \partial_{U^e-E}e \rangle \subseteq  X_{<\langle e\rangle}$,
so (Ch2),(Ch3) follow.


Lastly, we address (Ch0.1). By (a) we need only consider the case of $\Xi^e \neq \emptyset$,
so that by (b)
the complement of
the preimage
$e^{-1}(X_{<\langle e\rangle})$
consists of the faces isomorphic to
$U^e-E$ for $E \subset \Xi^e$.
Injectivity of $e$ within each isomorphism class of the faces away from % in 
$e^{-1}(X_{<\langle e\rangle})$
follows from $N \mathcal{P}$ being $\Sigma$-free,
while injectivity across distinct isomorphism classes of faces is
Corollary \ref{ISODIFCL COR}.
\end{proof}




\begin{remark}
	Condition (Ch0.1) is, by some margin, the subtlest condition in the previous proof, and the main reason for the chosen formulations of 
	Lemmas \ref{UNIQINAN LEM}, \ref{UNIQINAN2 LEM}.
	In particular, we note that injectivity of 
	$e \colon \Omega^e \to N \mathcal{P}$ will in general fail away from 
	$e^{-1}(X_{< \langle e \rangle})$.
	For example, two edges/vertices of $U^e$
	may be assigned the same color/operation, and similary for larger outer faces. In fact, injectivity may even fail on inner faces
	$T^e-E$ where $E \not \subseteq \Xi^e$.
\end{remark}






\subsection{Previous discussion (to be refactored)}


\begin{example}\label{GCORMPA EX}
Given a $G$-corolla $C \in \Sigma_G$, we write $\partial C$ for the set of edges of $C$, which is naturally identified with the set of objects of the associated $G$-operad
$\Omega(C) \in \mathsf{Op}^G$.

One can then regard $\Omega(C) \in \mathsf{Op}^{G,\partial C}$ and, moreover, $\Omega(C)$ is in fact the free operad over the symmetric sequence obtained by removing the units of $\Omega(C)$,
which we denote by
$\Omega'(C) \in \mathsf{Sym}^{G,\partial C}$.

Given the non-equivariant decomposition
$C = C_1 \amalg \cdots \amalg C_k$
with $C_i \in \Sigma$, 
one can naturally regard the $C_i$ as objects of $\Sigma_{\partial C}$.
In fact, one then has an identification
\begin{equation}\label{SOMEIDEN EQ}
	\Omega'(C) \simeq 
	\Sigma_{\partial C}[C_1] \amalg \cdots \amalg \Sigma_{\partial C}[C_k]
\end{equation}
where $\Sigma_{\partial C}[-]$ denotes the representable presheaf in 
$\mathsf{Set}^{\Sigma_{\partial C}^{op}}$.
This claim requires some justification, since a priori the right hand side of \eqref{SOMEIDEN EQ} is an object in $\mathsf{Set}^{\Sigma_{\partial C}^{op}}$,
rather than in $\mathsf{Set}^{G \ltimes \Sigma_{\partial C}^{op}}$,
i.e. we need to describe the action of the additional action arrows
$D \xrightarrow{g} gD$ on this presheaf.
This action is given by the following diagram, where the vertical $g$ arrows simply act on labels, and all horizontal arrows are shuffle arrows (i.e. arrows in $\Sigma_{\partial C}$).
The diagonal $C_g$ arrow corresponds to the structural $G$-action on $C$. It is then straightforward to check that there is a unique dashed shuffle $\tau_g$ as indicated
\[
\begin{tikzcd}
	D \ar{d}[swap]{g} \ar{r}{\sigma} & C_i \ar{rd}{C_g} \ar{d}[swap]{g}
\\
	g D \ar{r}[swap]{g \sigma} & g C_i \ar[dashed]{r}{\simeq}[swap]{\tau_g} & C_{g i}
\end{tikzcd}
\]
and one defines $g_{\**}\colon \Sigma_{\partial C}[C_i] \to \Sigma_{\partial C}[C_{gi}]$
via $\sigma \mapsto \tau_g \circ (g \sigma)$.

Moreover, letting $f \colon \partial C \to \mathfrak{C}$
be a map of colors, 
one obtains $\mathfrak{C}$-corollas $C_i^{f} \in \Sigma_{\mathfrak{C}}$
by coloring each edge $e\in \partial C$ by $f(e) \in \mathfrak{C}$, resulting in a generalized identification 
\begin{equation}\label{SOMEIDENGEN EQ}
	f_{!} \Omega'(C) \simeq 
	\Sigma_{\mathfrak{C}}[C_1^f] \amalg \cdots \amalg \Sigma_{\mathfrak{C}}[C^f_k]
\end{equation}
Indeed, \eqref{SOMEIDENGEN EQ} follows from the observation
that the Kan extension 
$\mathsf{Lan}_{G \ltimes \Sigma_{\partial C} \to G \ltimes \Sigma_{\mathfrak{C}}}$ 
coincides, after forgetting with the $G$-action arrows,
with the Kan extension
$\mathsf{Lan}_{\Sigma_{\partial C} \to \Sigma_{\mathfrak{C}}}$
(cf. Lemma \ref{REDUCELAN LEM}).
\end{example}


\begin{definition}
	Given a $\mathfrak{C}$-corolla $C$, 
	a subgroup 
	$\Gamma \leq \mathsf{Aut}_{G \ltimes \Sigma_{\mathfrak{C}}^{op}}(C)$
	is called a \textit{$G$-graph subgroup} if
	$\Gamma \cap \mathsf{Aut}_{\Sigma_{\mathfrak{C}}^{op}}(C) = \**$.
	
	We write $\mathcal{F}^{\Gamma} = \{\mathcal{F}^{\Gamma}_C\}$
	for the collection of families of $G$-graph subgroups.
	
	A $G$-$\mathfrak{C}$-symmetric sequence
	$X \in \mathsf{Sym}^{G,\mathfrak{C}}$
	is called $\Sigma$-cofibrant if each level
	$X(C)$ is $\mathcal{F}^{\Gamma}_C$-cofibrant.
\end{definition}


\begin{remark}
	Write a $\mathfrak{C}$-corolla as $C^f \in \Sigma_{\mathfrak{C}}$,
	where $C \in \Sigma$ is the underlying corolla and
	$f\colon \partial C \to \mathfrak{C}$
	is the coloring.
	A $G$-graph subgroup 
	$\Gamma \leq \mathsf{Aut}_{G \ltimes \Sigma_{\mathfrak{C}}^{op}}(C^f)$ is, under the map 
	$G \ltimes \Sigma_{\mathfrak{C}}^{op} \to
	G \times \Sigma^{op}$,
	identified with a 
	$G$-graph subgroup of 
	$G \times \mathsf{Aut}_{\Sigma^{op}}(C)$,
	i.e., with the graph of a partial antihomomorphism
\[
	G^{op} \geq H^{op} \xrightarrow{(-)^{-1}} H 
	\xrightarrow{\tau_{(-)}} \mathsf{Aut}_{\Sigma^{op}}(C)
\]
	which is subject to the requirement
\[
	f(\tau_h(e)) = h f(e).
\]
Using the $\tau$ automorphisms one can then:
\begin{inparaenum}
\item[(i)] equip $C$ with a $H$-action,
so that one can regard $C \in \Sigma^H \subseteq \Sigma_H$;
\item[(ii)] extend $\Sigma_{\mathfrak{C}}[C^f]$ to 
an object in $\mathsf{Set}^{H \ltimes \Sigma_{\mathfrak{C}}}$
by defining the action of the $H$-action arrows $h$ via 
$\sigma \mapsto \tau_{h} \circ (h \sigma)$;
\item[(iii)]
following Example \ref{GCORMPA EX}, one thus has an identification
\[
	f_{!} \Omega'(C) \simeq \Sigma_{\mathfrak{C}}[C^f]
\]
of objects in $\mathsf{Set}^{H \ltimes \Sigma_{\mathfrak{C}}}$ and therefore an identification 
\[
	f_{!} \left( G \cdot_H \Omega'(C) \right) \simeq 
	G\cdot_H \Sigma_{\mathfrak{C}}[C^f]
\]
of objects in $\mathsf{Set}^{G \ltimes \Sigma_{\mathfrak{C}}}$.
\end{inparaenum}
\end{remark}




\begin{proposition}\label{KEYPR PROP OLD}
Suppose that $\mathcal{O} \in \mathsf{Op}^{G,\mathfrak{C}}$
is $\Sigma$-cofibrant.
Further, let $C \in \Sigma_G$ be any $G$-corolla and consider 
a pushout in $\mathsf{Op}^{G}$ of the form
\begin{equation}\label{PUSHOUTPROP EQ}
\begin{tikzcd}
	\partial \Omega(C) \ar{r} \ar{d} & \mathcal{O} \ar{d}
\\
	\Omega(C) \ar{r} & \mathcal{P}.
\end{tikzcd}
\end{equation}
Then the induced map
\begin{equation}\label{ANODYNE MAP}
	\Omega[C] \amalg_{\partial \Omega[C]} N\mathcal{O} \to N\mathcal{P}
\end{equation}
is $G$-inner anodyne.
\end{proposition}


Let us write $f \colon \partial C \to \mathfrak{C}$
for the induced map of colors.
The first step is to rewrite \eqref{PUSHOUTPROP EQ} as a pushout diagram in $\mathsf{Op}^{G,\mathfrak{C}}$, which can be done by applying $\check{f}_{\**}$
to the leftmost objects in \eqref{PUSHOUTPROP EQ}.
Since
\[
	\check{f}_{!} \Omega(C) \simeq 
	\check{f}_{!} \left( \mathbb{F} \Omega'(C) \right) \simeq 
	\mathbb{F} \left(f_{!}  \Omega'(C) \right)
\]
one has that, writing $C \simeq G \cdot_H C_{\star}$ one has the alternative pushout in $\mathsf{Op}^{G,\mathfrak{C}}$
\begin{equation}
\begin{tikzcd}
	\mathbb{F} ( \emptyset ) \ar{r} \ar{d} & \mathcal{O} \ar{d}
\\
	\mathbb{F} \left( 
	G \cdot_H \Sigma_{\mathfrak{C}}[C^f_{\star}] \right) \ar{r} & \mathcal{P}.
\end{tikzcd}
\end{equation}











\newpage

\section{The Quillen equivalence}
\label{QE_SEC}

In this section, we synthesize the above results as well as the results of the related papers \cite{BP_geo,BP_edss,Per18},
to prove the main theorem of this project, Theorem \ref{QE_THM}.

First, we discuss the adjunction in question.

\subsection{The $W$ construction}

To extend $\tau \colon \dSet \rightleftarrows \Op \colon N$ from \eqref{DSETADJ_EQ} to a Quillen adjunction out of $\sOp$,
we need a topological enrichment of the inclusion $\Omega \into \Op$ from \eqref{OMEGADEF_EQ}.
We do this by constructing the free simplicial resolution\footnote{Also called the \textit{Godement} resolution, c.f. \cite[\S 8.3]{BM06}.}
of $\Omega(T)$ as an algebra in the category of \textit{pointed} $\mathbf E(T)$-colored symmetric sequences $\Sym^{\mathbf E(T)}_{\eta}(\Set)$.

\begin{definition}
      Given a set of colors $\mathfrak C$, let $\eta_{\mathfrak C} = \eta$ denote the initial $\mathfrak C$-colored operad in $\V$,
      defined for all $\mathfrak C$-sequences $\vect C$ by
      \[
            \eta(\vect C) =
            \begin{cases}
                  1_\V \qquad \qquad & \vect C = (x;x) \mbox{ for some $x \in \mathfrak C$,}
                  \\
                  \varnothing & \mbox{else.}
            \end{cases}
      \]
      The category of \textit{pointed} $\mathfrak C$-colored symmetric sequences in $\V$ is the category
      \[
            \Sym^{\mathfrak C}_\eta(\V) := \eta_{\mathfrak C} \downarrow \Sym^{\mathfrak C}(\V).
      \]
\end{definition}

The free operad monad $\mathbb F^{\mathfrak C}$ factors
\begin{equation}
      \label{FC_FAC_EQ}
      \begin{tikzcd}
            \Sym^{\mathfrak C}(\V) \arrow[r, shift left, "{(-) \amalg \eta}"]
            &
            \Sym^{\mathfrak C}_\eta(\V) \arrow[l, shift left] \arrow[r, shift left, "\mathbb F^{\mathfrak C}_\eta"]
            &
            \Op^{\mathfrak C}(\V) \arrow[l, shift left, "U_\eta"]
      \end{tikzcd}
\end{equation}
where the map $\eta_{\mathfrak C} \to U_\eta(\O)$ selects the identity operations for each color,
$(-) \amalg \eta$ is the (levelwise) coproduct in $\Sym^{\mathfrak C}(\V)$,
and $\mathbb F^{\mathfrak C}_\eta$ is the ``free operad with prescribed units'' monad,
defined on $X \in \Sym^{\mathfrak C}_\eta(\V)$ to be the pushout below computed in $\Op^{\mathfrak C}(\V)$.
\begin{equation}
      \label{FETA_EQ}
      \begin{tikzcd}
            \mathbb F^{\mathfrak C} \eta_{\mathfrak C} \arrow[d] \arrow[r]
            &
            \eta_{\mathfrak C} \arrow[d]
            \\
            \mathbb F^{\mathfrak C} X \arrow[r]
            &
            \mathbb F^{\mathfrak C}_\eta X
      \end{tikzcd}
\end{equation}

We can now make the following definition of the $W$-construction.

\begin{definition}
      Given $T \in \Omega$, define $W(T)_\bullet \in \Op(\sSet) \subseteq (\Op^{\mathbf E(T)})^{\Delta^{op}}$ to be the bar construction
      \[
            W(T)_n = \left( \mathbb F^{\mathbf E(T)}_\eta \right)^{n+1} \left(\Omega(T)\right),
      \]
      with the simplicial maps the monadic unit and monadic multiplication.
\end{definition}

To work with this definition, we must ensure that
\begin{enumerate}
\item $\mathbb F_\eta$ is the left adjoint to the forgetful functor,
\item this adjunction is monadic, and
\item this recovers the usual description of the $W$-construction.
\end{enumerate}
While we could show this directly, we instead apply results from Appendix \ref{AMALGMON_SEC} to realize this simplicial resolution from a different perspective,
and the above will follow from Lemma \ref{FP_FETA_LEM}, Remark \ref{ADJSRMON REM}, and Corollary \ref{WT_COR} respectively.

\subsubsection{Non-unital operads and combined monads}

We consider an orthogonal factorization of the free operad monad
\[
      \Sym_\bullet \xrightarrow{\bar F} \Op_{\bullet,nu} \xrightarrow{P} \Op_\bullet
\]
where $\Op_{\bullet,nu}$ is the category of non-unital operads, and $P = (-) \amalg \eta_\bullet$ adds units.

\begin{definition}
      A \textit{non-unital operad} is a symmetric sequence equipped with an associative and $\Sigma_\bullet$-equivariant composition law, but no specified units.
      Specifically, we are including ``partial'' composition laws
      \[
            \O(n) \times \prod_{i \in \lambda} \O(k_i) \longto \O\left(n-|\lambda|+\sum_{i \in \lambda}k_i\right)
      \]
      for any subset $\lambda \subseteq \set{1,2,\dots,n}$.
\end{definition}

The monad $\bar F$ encoding non-unital operads is built analogously to $\mathbb F_\bullet$ as in Appendix \ref{MONAD_APDX},
with the key difference that,
for any set of colors $\mathfrak C$,
the fundamental category $\Omega_{\mathfrak C}^0$ of $\mathfrak C$-colored trees and isomorphisms
is replaced with the subcategory $\Omega_{\mathfrak C}^{\textrm{nu},0}$ of non-stick trees and isomorphisms.
This has the effect of replacing the categories $\Omega_{\mathfrak C}^n$ of $n$-strings of planar tall maps --- i.e. composites of inner faces and degeneracies --- 
with the subcategory $\Omega_{\mathfrak C}^{\textrm{nu},n}$ of
non-stick trees and $n$-strings of inner face maps:
a planar tall map $T \to S$ is equivalent to substitution data $(T_v \to S_v)_{v \in V(T)}$ by \cite[Prop. 3.41]{BP_geo},
and if we forbid $S_v$ from being a stick,
we remove the ability of $T \to S$ to be a degeneracy.

As observed in \cite{BP_geo}, degeneracies and sticks precisely record the unit and the unitality conditions.
Thus, the results from Appendix \ref{MONAD_APDX} may be adapted mutatis mutandis;
in particular, \eqref{FREEOP_EQ}, Definition \ref{COLORMON DEF}, and Proposition \ref{ASSOCIDS PROP} yield the following.

\begin{proposition}
      The free non-unital operad monad $\bar F$ on a $\mathfrak C$-colored symmetric sequence $X \in \Sym_{\mathfrak C}$ is given by the left Kan extension
      \[
            \begin{tikzcd}
                  \left(\Omega_{\mathfrak C}^{\text{nu},0}\right)^{op} \arrow[r, hookrightarrow] \arrow[d, "\mathsf{lr}"]
                  &
                  \Omega_{\mathfrak C}^{0,op} \arrow[r,"V"] \arrow[dl, Rightarrow,shorten >=0.4cm,shorten <=0.4cm]
                  &
                  \left(\Sigma \wr \Sigma_{\mathfrak C}\right)^{op} \arrow[r, "X"] 
                  &
                  \left(\Sigma \wr \V^{op}\right)^{op} \arrow[r, "\otimes"]
                  &
                  \V
                  \\
                  \Sigma_{\mathfrak C}^{op} \arrow[urrrr, "\Lan = \bar F X"']
            \end{tikzcd}
      \]
      Moreover, $\bar F^{\circ n}X$ is given by the left Kan extension      
      \begin{equation}
            \label{FREENUOP_EQ}
            \begin{tikzcd}
                  \left(\Omega_{\mathfrak C}^{\text{nu},n}\right)^{op} \arrow[r, hookrightarrow] \arrow[d, "\mathsf{lr}"]
                  &
                  \Omega_{\mathfrak C}^{n,op} \arrow[r,"{(V^0)^{\circ n+1}}"]
                  \arrow[dl, Rightarrow, shorten >=0.4cm,shorten <=0.4cm]
                  &[10pt]
                  \left(\Sigma ^{\wr n+1} \wr \Sigma_{\mathfrak C}\right)^{op} \arrow[r, "{(\sigma^0)^{\circ n+1}}"]
                  &[10pt]
                  \left(\Sigma \wr \Sigma_{\mathfrak C}\right)^{op} \arrow[r, "X"]
                  &
                  \left(\Sigma \wr \V^{op}\right)^{op} \arrow[r, "\otimes"]
                  &
                  \V
                  \\
                  \Sigma_{\mathfrak C}^{op} \arrow[urrrrr, start anchor = east, end anchor = south west, "\Lan = \bar F^{\circ n} X"']
            \end{tikzcd}
      \end{equation}
\end{proposition}


Now, we apply Appendix \ref{AMALGMON_SEC}.
First, we observe by hand % P-F-combined.pdf
that $P$ and $\bar F$ can be combined as in Definition \ref{AMALGMON DEF}:
The maps $P \Rightarrow \mathbb F$ and $\bar F \Rightarrow \mathbb F$ are simply
\[
      X \amalg \eta \xrightarrow{\eta \amalg e} \mathbb F X,
      \qquad
      \bar F X \hookrightarrow \eta \amalg \bar F X.
\]
Moreover, as $P$ and $F = \mathbb F$ both preserve reflexive coequalizers\footnote{
  For $\mathbb F$, this follows as $\otimes$ commuting with colimits in both variables implies it commutes with reflexive coequalizers; see e.g. \cite[Lemma 2.3.2]{Rez96}.},
the ``free operad with prescribed units'' functor, i.e. the left adjoint $F_p$ to the forgetful functor $\mathsf{fgt} \colon \Op_{\mathfrak C} \to \Sym^{\mathfrak C}_e$, exists,
and following \eqref{FRMONDES EQ} is given on $Y \in \Sym^{\mathfrak C}_e$ as the coequalizer of the following in $\Sym_{\mathfrak C}$.
\begin{equation}
      \label{FPYCOEQ_EQ}
      \mathbb F P Y = \mathbb F(Y \amalg \eta_{\mathfrak C}) \overset{f_1}{\underset{f_2}\rightrightarrows} \mathbb F Y,
      % \begin{tikzcd}
      %       \mathbb F P Y = \mathbb F(Y \amalg \eta_{\mathfrak C}) 
      %       \arrow[r, shift left, "{\mu \circ F P \eta_{\bar F}}"] \arrow[r, shift right, "{Fm}"']
      %       &[30pt]
      %       \mathbb F Y,
      % \end{tikzcd}
      \qquad
      f_1 = \mathbb F(Y \amalg \eta \xrightarrow{id,e} Y),
      \quad
      f_2 \colon \mathbb F(Y \amalg \eta) \xrightarrow{\eta_{\bar F} \amalg id} \mathbb F(\bar F Y \amalg \eta) = \mathbb F^2 Y \xrightarrow{\mu} \mathbb FY,
\end{equation}


This matches the above description of $\mathbb F_\eta$ from \eqref{FETA_EQ}:
\begin{lemma}
      The underlying functor for the ``free operad with prescribed units'' monad $F_P$ on $\Sym_e$ is naturally isomorphic to $\mathbb F_\eta$.
\end{lemma}
\begin{proof}
      We note that $f_2$ in \eqref{FPYCOEQ_EQ} has another decomposition, namely
      \[
            \mathbb F (Y \amalg \eta) = P\bar F Y \mathbin{\hat{\amalg}} \mathbb F_\eta \xrightarrow{\eta_{\bar F} \amalg id}
            P\bar F \bar F Y \mathbin{\hat{\amalg}} \mathbb F \eta \xrightarrow{\mu_{\bar F} \amalg (id \circ \mu)}
            F Y \amalg \eta = \mathbb F Y,
      \]
      where $\hat{\amalg}$ denotes the coproduct in $\mathbb F$-algebras
      (as $P \bar F \bar F \to \mathbb F Y$, by Proposition \ref{ALTMULT PROP}, and both $\mathbb F \eta \to \eta$ and the unique map $\eta \to \O$ are all maps of $\mathbb F$-algebras).

      But then this equalizier is of form
      $
      \O \mathbin{\hat{\amalg}} \P \rightrightarrows \O
      $
      with both maps the identity on the $\O$ component.
      As in any category, this is isomorphic to the coequalizer of the interesting pieces
      $\P \rightrightarrows \O$,
      and thus $F_P Y$ is the coequalizer below of $\mathbb F$-algebras.
      \[
            \mathbb F_\eta \overset{e \circ \mu}{\underset{\mathbb F(e)} \rightrightarrows} \mathbb F Y \xrightarrow{\mathrm{coeq}} F_P Y
      \]
      Converting this coequalizer into a pushout yields the diagram below.
      \[
            \begin{tikzcd}
                  \mathbb F \eta \mathbin{\hat{\amalg}} \mathbb F \eta \arrow[r] \arrow[d, "\nabla"']
                  &
                  \mathbb F Y \arrow[d]
                  \\
                  \mathbb F \eta \arrow[r]
                  &
                  F_P Y
            \end{tikzcd}
      \]
      Finally, one can check that this pushout has the same universal property as the pushout \eqref{FETA_EQ};
      the result follows.
\end{proof}

We can now describe $W(T)$.
\begin{corollary}
      \label{WT_COR}
      For any $T \in \Omega_G$, we have
      \begin{equation}
            \label{WT_EQ}
            W(T)_n = \mathbb F_\eta^{\circ n+1} \Omega(T) = F_P^{\circ n+1} P \bar F \Sigma_\bullet[T] = P \bar F^{\circ n+2}\Sigma_\bullet[T].
      \end{equation}
      Moreover, for any signature $\vect C = (e_1, \dots, e_k; e)$, we have
      $W(T)(\vect C) = \Delta[1]^{\times \boldsymbol{E}^i(T_{\underline{e} \leq e})}$ is the (nerve of the) poset of inner faces of $T_{e_1 e_2 \dots e_k \leq e_0}$.
\end{corollary}
\begin{proof}
      Equation \eqref{WT_EQ} follows from Corollary \ref{ALGLEFTEX_COR}.
      The evaluation $W(T)(\vect C)$ follows from \eqref{FREENUOP_EQ} that
      \[
            \bar F^{\circ n+2} \Sigma_\bullet[T] = \Lan_{(\Omega_\bullet^{r,n+1}\to \Sigma_\bullet)^{op}} \bar N^{T},
      \]
      with
      \[
            \left(\bar N^{T}\right)^{op} \colon
            \Omega_{\boldsymbol E(T)}^{r, n+1} \xrightarrow{\boldsymbol{V}}
            \Sigma^{\wr n+2} \wr \Sigma_{\boldsymbol E(T)} \longto
            \Sigma \wr \Sigma_{\boldsymbol E(T)} \xrightarrow{\Sigma_\bullet[T]}
            \Sigma \wr \V^{op} \xrightarrow{\otimes}
            \V^{op}.
      \]
      But $\bar N^T$ evaluated on an $(n+1)$-string of inner faces
      $T_0 \hookrightarrow T_1 
      \hookrightarrow T_2 
      \hookrightarrow \cdots
      \hookrightarrow T_{n+1}$
      is given by a point $\**$
      if $T_{n+1} \simeq T$
      and by $\emptyset$ otherwise.
      The result follows.
\end{proof}


\begin{definition}
      The \textit{homotopy coherent dendroidal nerve} and the \textit{$W_!$-construction}
      are the right- and left-adjoints of the nerve-realization adjunction associated to the functor $W: \Omega \into \Op(\sSet)$.
      \begin{equation}
            \label{SOPDSET_EQ}
            hc N \colon \sOp \leftrightarrows
            \dSet \colon W_!
      \end{equation}
\end{definition}

The culmination of extensive work by Cisinski-Moerdijk-Weiss \cite{CM13a,CM13b,CM11,MW09,MW07} is the following.

\begin{theorem}
      \label{CMW_THM}
      $hcN$ and $W_!$ form a Quillen equivalence with respect to the model structures
      which, in particular, are recovered by the $G = \**$ cases of Theorem \ref{MODEL_THM} and \cite[Thm. 2.1]{Per18}.
\end{theorem}



\todo[inline]{come back!}











\subsubsection{Applying this to the main case}

Working in the category 
$\mathsf{Sym}_{\bullet}(\mathcal{V})$
with $\bar{F}$ the non-unital operad monad and $P$ the ``adding units/points'' monad, one has
\[
\Omega(T) = F \Sigma_{\bullet}[T] 
	= P \bar{F} \Sigma_{\bullet}[T]
\]
and one then defines
\[
\left(W(T)\right)_n 
	= F_r^{\circ n+1} \Omega(T)
	= F_r^{\circ n+1} P \bar{F} \Sigma_{\bullet}[T]
	= P \bar{F}^{\circ n+2} \Sigma_{\bullet}[T]
\]
Moreover, noting that
\[\bar{F}^{\circ n+2} \Sigma_{\bullet}[T]=
\mathsf{Lan}_{
\left(\Omega^{r,n+1}_{\bullet} \to \Sigma_{\bullet}\right)^{op}}
\bigotimes \circ (\Sigma\wr\Sigma_{\bullet}[T])\circ V\]
and that
$\bigotimes \circ (\Sigma\wr\Sigma_{\bullet}[T])\circ V$
evaluated on a $n+1$-string of inner faces
$T_0 \hookrightarrow T_1 
\hookrightarrow T_2 
\hookrightarrow \cdots
\hookrightarrow T_{n+1}$
is given by a point $\**$
if $T_{n+1} \simeq T$
and by $\emptyset$ otherwise,
we recover the usual description of 
$W(T)(e_1e_2\cdots e_k\leq e)$
as the poset of inner faces of 
$T_{e_1e_2\cdots e_k\leq e}$.



{\color{red} HERE}

\begin{proposition}\label{WLEFTQPUSH PEOP}
For $\eta \neq T \in \Omega$
and $\emptyset \neq E \subseteq \boldsymbol{E}^{\mathsf{i}}(T)$
one has pushout digrams
\[
\begin{tikzcd}
	\Omega(C) \otimes_{\mathsf{F}}
	\partial \left(\Delta[1]^{\times \boldsymbol{E}^{\mathsf{i}}(T)}\right)
	\ar{r} \ar{d}
&
	W_! \left(\partial \Omega[T]\right) 
	\arrow[d, "i"]
&%
	\Omega(C) \otimes_{\mathsf{F}}
	\lambda^E \Delta[1]^{\times \boldsymbol{E}^{\mathsf{i}}(T)}
	\ar{r} \ar{d}
&
	W_! \left(\Lambda^E[T]\right) 
	\arrow[d, "j"]
\\
	\Omega(C) \otimes_{\mathsf{F}}
	\Delta[1]^{\times \boldsymbol{E}^{\mathsf{i}}(T)}
	\ar{r}
&
	W(T)
&%
	\Omega(C) \otimes_{\mathsf{F}}
	\Delta[1]^{\times \boldsymbol{E}^{\mathsf{i}}(T)}
	\ar{r}
&
	W(T)
\end{tikzcd}
\]
where
$C = \mathsf{lr}(T)$, and
$\partial \left(\Delta[1]^{\times \boldsymbol{E}^{\mathsf{i}}(T)}\right)
\to
\Delta[1]^{\times \boldsymbol{E}^{\mathsf{i}}(T)}$
and
$\lambda^E \Delta[1]^{\times \boldsymbol{E}^{\mathsf{i}}(T)}
\to \Delta[1]^{\times \boldsymbol{E}^{\mathsf{i}}(T)}$
are the iterated pushout products
\[
      \left(
            \partial\Delta[1] \to \Delta[1]
      \right)^{\square \boldsymbol E^{i(T)}},
      \qquad
      \left(
            \partial \Delta[1] \to \Delta[1]
      \right)^{\square (\boldsymbol{E}^{\mathsf{i}}(T) \setminus E)}
      \square
      \left(
            \{1\} \to \Delta[1]
      \right)^{\square E}
\]
\end{proposition}

\begin{proof}
      We note that $i$ is the identity on objects,
      and for any $\boldsymbol{E}(T)$-signature $\vect D$, $i(\vect D)$ is the identity unless $\vect D = \sigma \cdot \partial C$,
      where $\sigma \in \Sigma_n$ and $\partial C = (e_1,\dots,e_n;e)$ with $\set{e_1,\dots,e_n}$ the set of leaves of $T$ and $e$ the root.
      In this case, $i(\partial C) = (\partial \Delta[1] \to \Delta[1])^{\square \boldsymbol{E}^i(T)}$,
      and the top horizontal map is adjoint to the identity map via Examples \ref{TENS_EX}.
      \todo[inline]{come back!}
\end{proof}


\newpage


\subsection{The equivariant adjunction}

The main goal of this project \cite{BP_geo,BP_edss,Per18} is to understand the correct equivariant lifting of Theorem \ref{CMW_THM}.
On the one hand, the construction of the new adjunction is immediate:

\begin{definition}
      Define
      \[
            W_! \colon \dSet^G \rightleftarrows \sOp^G \colon hcN_d
      \]
      to be the
      extension of the adjunction \eqref{SOPDSET_EQ}
      given by post-composition.
\end{definition}

% Equivalently, given decompositions $T \simeq G \cdot_H U$ and $T \simeq G \cdot U / N$, we have
% $W(T) \simeq G \cdot_H W(U)$ and $W(T) \simeq G \cdot W(U)/N$.


On the other hand, the homotopy theories involved are much more involved to capture the genuine equivariance,
as noted in \S \ref{OPC_MS_SEC}, \S \ref{MS_SEC}, \S \ref{EDS_SEC}, and \cite{Per18}.

%We have the following generalization of \cite[Prop 4.5]{CM11} (see also \cite[Prop. 6.15]{Per18}).

\begin{proposition}
      \label{W!_COF_PROP}
      Suppose $\F = \set{\F_n}$ is a weak indexing system.
      Then $W_!: \dSet^G \to \sOp^G$ sends $\F$-normal monomorphisms to cofibrations and inner $\F$-anodyne extensions to trivial $\F$-cofibrations.
\end{proposition}
\begin{proof}
      It suffices to check this on the generating maps.
      To that end, let $T \in \Omega_\F$, choose a decomposition $T \simeq G \cdot_H T_e$ and consider the map
%      \begin{equation}
%            G \cdot_H \left( W_! \partial \Omega[T_e] \xrightarrow{\ i \ } W(T_e) \right)      
%      \end{equation}
      in $\sOp^G$.
      If $T_e = \eta$, then $i$ is the canonical map $\varnothing \to \eta$,
      % W_! \partial \Omega[\eta] = \varnothing \to W(\eta) = \eta$,
      and so $G \cdot_H i$ is a generating cofibration from \eqref{IFJF_EQ} in $\sOp^G$.

      Thus we may assume that $|V(T)| \geq 1$.
      In this case, $i$ is an isomorphism on the $H$-set of objects $\mathbf E(T_e)$,
      and for any $\mathbf E(T_e)$-signature $\vect C$,
      $i(\vect C)$ is the identity unless $\vect C = h \cdot \partial \mathsf{lr}(T_e)$ for some $h \in H$,
      where $\partial \mathsf{lr}(T_e)$ is the signature $(e_1, \dots, e_n; e)$ 
      with $\set{e_1,\dots, e_n}$ the set of leaves of $T_e$ and $e$ the root.
      When $\vect C = \partial \mathsf{lr}(T_e)$, $i(\vect C)$ is the iterated pushout product
%      \[
%            i(\vect C) = (\partial \Delta[1] \to \Delta[1])^{\square \mathbf E^i(T_e)}.
%      \]
      We need that $i(\vect C)$ is a genuine cofibration in $\sSet^{\Aut(\vect C)}_\F$.
      However, $\Aut(\vect C) = \Gamma$ for $\Gamma \leq H \times \Sigma_n$ the graph subgroup encoding the $H$-action on the set of leaves $\set{e_1, \dots, e_n}$,
      and since $\F$ is a weak indexing system and $T \in \Omega_\F$,
      $\Aut(\vect C) \in \F_n$,
      and thus, as $\F$ is closed under subgroups, $\sSet^{\Aut(\vect C)}_\F = \sSet^{\Aut(\vect C)}_{gen}$.
      Then the claim follows from the fact that monomorphisms are genuine cofibrations,
      {\color{OliveGreen} % ---------- OLIVE GREEN ----------------------------------------
        or using the fact that $\sSet$ has cofibrant symmetric pushout powers,
        so $(\partial \Delta[1] \to \Delta[1])^{\square m}$ is a genuine $\Sigma_m$-cofibration where $m = |\mathbf E^i(T_e)|$,
        and then as restriction is left Quillen on genuine model structures,
        the composite on the right below is a genuine $\Aut(\vect C)$-cofibration.
        \[
              \mathrm{Res}^{\Sigma_m}_{\Aut(C)}(\partial \Delta[1] \to \Delta[1])^{\square m} = i(\vect C),
              \qquad
              \Aut(\vect C) = \Gamma \xrightarrow{pr} H \xrightarrow{\mathbf E^i(T_e)} \Sigma_m
        \]

      } % ---------------------------------------- OLIVE GREEN --------------------
      %
      It is then straightforward to check that $G \cdot_H i$ has the left lifting property against all local trivial $\F$-fibrations in $\sOp^G$.
      {\color{OliveGreen} % ----------------------------------------
        Indeed, we clearly have a lifts on the level of sequences.
        To check that the resulting map is operadic, it suffices to check on composites of the form
        $C_0 \circ (C_1, \dots, C_n) = \mathsf{lr}(T_e)$.
        If $C_j = \mathsf{lr}(T_e)$ for some $j$, then the composite map is just the identity, and there is no content to check.
        If $C_j \neq \mathsf{lr}(T_e)$ for all $j$,
        then this follows by a simple diagram chase on the cube relating the compositions in $W(T)$ and $W_! \partial \Omega(T)$ with those in the source and target of the testing trivial fibration.
      } % COLOR OLIVEGREEN
      Hence $G \cdot_H i$ is an $\F$-cofibration in $\sOp^G$ by Lemma \ref{CAV_4.8}.
      
     
      Similarly, consider a generating $\F$-inner horn inclusion
%      \[
%            G \cdot_H \left(W_! \Lambda^{H e}[T_e] \xrightarrow{\ j \ } W(T_e) \right).
%      \]
      with $T \simeq G \cdot_H T_e \in \Omega_\F$ and $e \in \mathbf E(T_e)$ some inner edge.
      Again, $j$ is bijective on objects and $j(\vect C)$ is the identity unless $\vect C = h \cdot \partial \mathsf{lr}(T_e)$.
      When $\vect C = \partial \mathsf{lr}(T_e)$, $j(\vect C)$ is the iterated pushout product
      \begin{equation}
            j(\vect C) =
            \left(
                  \begin{tikzcd}
                        \partial \left(
                              \Delta[1]^{\times \mathbf E^i(T_e) \setminus H e}
                        \right) \arrow[d]
                        \\
                        \Delta[1]^{\times \mathbf E^i(T_e) \setminus H e}
                  \end{tikzcd}
            \right) \square
            \left(
                  \begin{tikzcd}
                        \set{1} \arrow[d]
                        \\
                        \Delta[1]
                  \end{tikzcd}
            \right)^{\square H e}
            =
            \left(
                  \begin{tikzcd}
                        \partial \Delta[1] \arrow[d]
                        \\
                        \Delta[1]
                  \end{tikzcd}
            \right)^{\square \mathbf E^i(T_e) \setminus H e}
            \square
            \left(
                  \begin{tikzcd}
                        \set{1} \arrow[d]
                        \\
                        \Delta[1]
                  \end{tikzcd}
            \right)^{\square H e}.                              
      \end{equation}
      A similar argument shows that $G \cdot_H j$ has the left lifting property against all local $\F$-fibrations,
      hence is a trivial $\F$-cofibration in $\sOp^G$ by Lemma \ref{CAV_4.3}, as desired. 
\end{proof}

We have the following immediate corollary.
\begin{corollary}
      [{cf. \cite[Prop. 6.15]{Per18}, \cite[Cor. 4.6]{CM11}}]
      For any $\F$-fibrant $\O \in \sOp^G$, $h c N \O$ is an $\F$-$\infty$-operad.
\end{corollary}

Another corollary is the following key step.

\begin{proposition}[{cf. \cite[Prop. 4.9]{CM11}}]
      $W_!: \dSet^G \to \sOp^G$ is left Quillen.
\end{proposition}
\begin{proof}
      Since Proposition \ref{W!_COF_PROP} states that $W_!$ preserves cofibrations,
      it suffices by Corollary \ref{SIMPLQUILL COR} to show that $h c N$ preserves fibrations between fibrant objects.
      Suppose $f: \O \to \P$ is an $\F$-fibration between $\F$-fibrant operads in $\sOp^G$.
      Then Proposition \ref{W!_COF_PROP} also implies that $h c N (f)$ is an $\F$-inner-fibration between $\F$-$\infty$-operads.
      By \cite[Thm. 8.22]{Per18}, this is an $\F$-fibration in $\dSet^G_\F$ iff $\tau (h c N_d(f)^H) = \tau (h c N_d(f^H))$ is a categorical fibration for all $H \leq G$.
      But by Remark \ref{FIB_ISOFIB_REM}, being a fibration in $\sOp^G_\F$ implies that $\pi_0(f^H)$ is a categorical fibration for all $H \leq G$, 
      and since $\pi_0 = \tau \circ h c N$ by \cite[Prop. 4.8]{CM11}, the result follows.
\end{proof}

\begin{remark}
      One can also show a more powerful equivariant version of \cite[Prop. 4.8]{CM11},
      namely that the following diagram commutes.
      \begin{equation}
            \begin{tikzcd}
                  \sOp^G \arrow[r, "i_{\**}"] \arrow[d, "h c N"']
                  &
                  \sOp_G \arrow[d, "h c N"] \arrow[r, "\pi_0"]
                  &
                  \Op_G \arrow[d, equal]
                  \\
                  \dSet^G \arrow[r, "i_{\**}"]
                  &
                  \dSet_G \arrow[r, "\tau_G"]
                  &
                  \Op_G
            \end{tikzcd}
      \end{equation}
      In particular, this recovers that $\pi_0 \circ (-)^H = \tau \circ h c N \circ (-)^H$ for all $H$,
      but also more subtle information about the interaction of $\pi_0$ and $h c N$ with the graph subgroup fixed points.
      This story, along with colored genuine equivariant operads as well as cofibrancy considerations of the above narrative,
      will be further explored in a sequel. 
\end{remark}

\begin{remark}
      For any $\mathcal F$-tree $T \in \Omega_{\mathcal F}$, $\Omega[T] \in \dSet^G$ is $\mathcal F$-cofibrant/normal,
      and hence $W(T)$ and $\Omega(T)$ are $\mathcal F$-cofibrant in $\sOp^G$ for all $T \in \Omega_{\mathcal F}$.
\end{remark}


\todo{come back!}




Finally, the following is a reformalized statement and proof of \cite[Thm. 8.14]{CM13b}, extended to the equivariant setting;
the main result is an immediate corollary.

\begin{proposition}\label{COMUOTOHOM PROP}
      % For all weak indexing systems $\F$,
      The (right) derived composite functors in the following diagram commute up to a zigzag of weak equivalences. 
      \[
            \begin{tikzcd}
                  \mathsf{PreOp}^G \ar{d}[swap]{\gamma^{\**}}&
                  \mathsf{sOp}^G \ar{l}[swap]{N} \ar{d}{hcN}
                  \\
                  \mathsf{sdSet}^G &
                  \mathsf{dSet}^G \ar{l}{c_{!}}
            \end{tikzcd}
      \]
\end{proposition}

Before the proof, we list some technical results yielded from \cite{BP_edss} that we will need.
\begin{lemma}
      \label{COMUOTOHOM_FACTS}
      Facts we need for Proposition \ref{COMUOTOHOM PROP}:
      \begin{enumerate}[label = (\roman*)]
      \item weak equivalences in $\mathsf{PreOp}^G$ are detected by $\gamma^{\**}$.
            % ----------
      \item \label{RJOINTT_LBL} If $X \in (\mathsf{sdSet}^G)^{\Delta^{op}}$ is fibrant in the Reedy-over-joint model structure,
            then for all $T \in \Omega_G$, $X(\Omega[T])$ is horizontal Reedy fibrant in
            $\left((\mathsf{sSet})^{\Delta^{op}}\right)^{\Aut(T)}$.
            %
      \item \label{JF_VERT_LBL}
            $X \in \mathsf{sSet}^{\Delta^{op}}$ is joint fibrant iff
            $X$ is horizontal Reedy fibrant and all vertex maps $X(m) \to X(0)$ are Kan equivalences in $\mathsf{sSet}$
            % 
      \item \label{DIAG_LBL}
             If $X \in \mathsf{sSet}^{\Delta^{op}}$ is horizontal Reedy fibrant, then
             $X_n \to X_0$ and hence $X_0 \to \delta^{\**}X$ is a Kan equivalence in $\mathsf{Set}^{\Delta^{op}}$.
      \end{enumerate}
\end{lemma}
\begin{proof}
      For \ref{RJOINTT_LBL},
      note that the identity gives a right Quillen functor
      \[
            \left( \mathsf{sdSet}_{jt}^G \right)^{\Delta^{op}} \longto
            \left(\mathsf{sSet}^{\Omega^{op} \times G}\right)^{\Delta^{op}} =
            \left( (\mathsf{sSet})^{\Delta^{op}} \right)^{\Omega^{op} \times G}.
      \]
      where $\left(\sSet^{\Omega^{op} \times G}\right)^{\Delta^{op}}$ and $\left(\sSet^{\Delta^{op}}\right)^{\Omega^{op} \times G}$
      are given the respective (generalized) Reedy model structures over the (generalized) Reedy model structure
      over the Kan model structure on $\sSet$,
      and the equality follows from \cite[Remark A.31]{BP_edss}.
      The result then follows from the description of (equivariant) simplicial equivalences before Theorem \ref{JB_THM}.
      
      \ref{JF_VERT_LBL} is \cite[Prop. 4.24(ii)]{BP_edss}.

      \ref{DIAG_LBL} is \cite[Prop. 4.5(iv)]{BP_edss}.
\end{proof}



\begin{proof}[Proof of Lemma \ref{COMUOTOHOM PROP}]
      First, we note that though $\gamma^{\**}$ and $c_{!}$ are left Quillen, they both preserve all equivalences, 
      so that one needs only perform fibrant replacements in 
      $\mathsf{sOp}$.

      Now, recall that, given an object $X$ in a model category $\mathcal{M}$, a simplicial frame of $X$ is a fibrant replacement
      $c_!(X) \to \widetilde{X}(\bullet)$ of the constant 
      simplcial object $c_!(X)$ in the Reedy model structure on $\mathcal{M}^{\Delta^{op}}$.
      Moreover, if $X$ was already fibrant one is free to assume that $\widetilde{X}(0) = X$.
      
      Let $\mathcal{O} \in \mathsf{sOp}^G$ be fibrant, 
      choose a (functorial) fibrant simplicial frame
      $\widetilde{\mathcal{O}}(\bullet) \in (\mathsf{sOp}^G)^{\Delta^{op}}$, where we assume $\widetilde{\mathcal{O}} (0) = \mathcal{O}$.
      Next, let 
      $\gamma^{\**} N \widetilde{\mathcal{O}}(\bullet) 
      \to \widetilde{Q}(\bullet)$
      be a Reedy fibrant replacement in  
      $(\mathsf{sdSet}^G)^{\Delta^{op}}$.
      
      We claim that the following is a zigzag of weak equivalences in $\mathsf{sdSet}^G$.
      \begin{equation}\label{BIGZIG EQ}
            \gamma^{\**} N \mathcal{O} \xrightarrow{\sim}
            \widetilde{Q}(0) \xrightarrow{\sim}
            \delta^{\**} \widetilde{Q} \xleftarrow{\sim}
            \widetilde{Q}_0 \xleftarrow{\sim}
            \left(\gamma^{\**} N \widetilde{\mathcal{O}}\right)_0
            \xrightarrow{\sim}
            hcN \widetilde{\mathcal{O}} \xleftarrow{\sim}
            c_{!} hcN \mathcal{O}
      \end{equation}
      
      That the first map is an equivalence is obvious from definition of $\widetilde{Q}$ and the assumption $\widetilde{\mathcal{O}}(0) = \mathcal{O}$.
      
      The second, third, fourth, and fifth maps are all in fact simplicial equivalences.
      For the second and third maps, note first that $\widetilde{\O}$ and hence
      $\widetilde{Q}$ are homotopically constant, in the sense that
      all vertex maps $\widetilde{Q}(m) \to \widetilde{Q}(0)$ are joint equivalences in $\mathsf{sdSet}^G$.
      % 
      Moreover, since the levels $\widetilde{Q}$ are joint fibrant in $\mathsf{sdSet}^G$,
      Theorem \ref{JB_THM} \ref{SFIB_JEQ_LBL} % \cite[Prop. 4.5(iii)]{BP_edss}
      implies that these are in fact simplicial equivalences, % $G$-graph simplicial equivalences,
      i.e. that for each $G$-tree $T \in \Omega_G$, the evaluations % tree $U \in \Omega$, the evaluations
      $\widetilde{Q}(\Omega[T])(m) \to \widetilde{Q}(\Omega[T])(0)$
      are Kan equivalences in $\mathsf{sSet}$.  %$G$-graph Kan equivalences in $\mathsf{sSet}^{G \times \Aut(U)}$.
      % 
      But since $\widetilde{Q}(\Omega[T]) \in \sSet^{\Delta^{op}}$ %$\widetilde{Q}(U) \in \mathsf{sSet}^{\Delta^{op} \times (G \times \Aut(U))}$
      is itself Reedy fibrant by
      Lemma \ref{COMUOTOHOM_FACTS} \ref{RJOINTT_LBL},
      Lemma \ref{COMUOTOHOM_FACTS} \ref{JF_VERT_LBL} % (the proof of) \cite[Prop. 4.24(ii)]{BP_edss} 
      implies that it is in fact joint Reedy fibrant in $\mathsf{sSet}^{\Delta^{op}}$ % $G$-graph joint Reedy fibrant in $\left(\mathsf{sSet}^{\Delta^{op}}\right)^{G \times \Aut(U)}$,
      and hence by Lemma \ref{COMUOTOHOM_FACTS} \ref{DIAG_LBL}, %\cite[Prop. 4.5(iv)]{BP_edss}
      one has Kan equivalences % $G$-graph Kan equivalences
      $\widetilde{Q}(\Omega[T])(0) \xrightarrow{\sim}
      \delta^{\**} \widetilde{Q}(\Omega[T]) \xleftarrow{\sim}
      \widetilde{Q}_0(\Omega[T])$, as desired.
      
      For the fourth equivalence, note that one can write
      \[
            \widetilde{Q}_0(\Omega[T]) = 
            \mathsf{Hom}_{\mathsf{sdSet}^G}(c_!\Omega[T],\widetilde{Q})=
            \mathsf{Hom}_{\mathsf{PreOp}^G}(c_!\Omega[T],\gamma_{\**}\widetilde{Q}),
      \] 
      \[
            \left(\gamma^{\**} N \widetilde{\mathcal{O}}\right)_0(\Omega[T]]) = 
            \mathsf{Hom}_{\mathsf{sdSet}}(c_!\Omega[T],\gamma^{\**} N \widetilde{\mathcal{O}})=
            \mathsf{Hom}_{\mathsf{PreOp}}(c_!\Omega[T], N \widetilde{\mathcal{O}}),
      \]
      in $\sSet$ for each $T \in \Omega_G$. % $\sSet^{G \times \Aut(U)}$ for each $U \in \Omega$.
      The claim now follows by noting that
      $N \widetilde{\mathcal{O}} \to \gamma_{\**} \widetilde{Q}$
      is an equivalence between Reedy fibrant replacements for $N \mathcal O$ in $(\mathsf{PreOp}^G_{tame})^{\Delta^{op}}$
      (as $\gamma^{\**}\gamma_{\**}\tilde Q \to \tilde Q$ is a joint equivalence and $N$, $\gamma_{\**}$ are right Quillen),
      and that $\Omega[T]$ is tame cofibrant by Lemma \ref{OMEGATTAME_LEM}.
      
      For the fifth equivalence, note that
      \[
            \left(\gamma^{\**} N \widetilde{\mathcal{O}}\right)_0(\Omega[T]) = 
            \mathsf{Hom}_{\mathsf{PreOp}^G}(\Omega[T], N \widetilde{\mathcal{O}}) =
            \mathsf{Hom}_{\mathsf{sOp}^G}(\Omega(T),  \widetilde{\mathcal{O}}),
      \]
      \[
            \left(hcN \widetilde{\mathcal{O}} \right)(\Omega[T]) = 
            \mathsf{Hom}_{\mathsf{sOp}^G}(W(T),  \widetilde{\mathcal{O}}),
      \]
      in $\sSet$ for each $T \in \Omega_G$, %$\sSet^{G \times \Aut(U)}$ for each $U \in \Omega$,
      so that the required claim follows since 
      $\widetilde{\mathcal{O}}$ is Reedy fibrant and
      $W(T) \to \Omega(T)$ is an equivalence of cofibrant operads in $\sOp^G$. % a $G$-graph weak equivalence of cofibrant operads in $\sOp^{G \times \Aut(U)}$.

      Finally, for the last map, one needs simply to note that
      $c_! hcN \mathcal{O} = hcN c_! \mathcal{O}$, so that the required claim follows since 
      $c_! \mathcal{O} \to \widetilde{O}$
      is a levelwise equivalence of levelwise fibrant operads
      and $hcN: \sOp^G \to \dSet^G$ is right Quillen.
\end{proof}

\begin{proof}
      [Proof of Theorem \ref{QE_THM}]
      Proposition \ref{COMUOTOHOM PROP} implies that $W_!$ is an equivalence of homotopy theories.
      Since by Proposition \ref{W!_COF_PROP} it is left Quillen,
      the result follows.
\end{proof}






\section{Indexing system analogue results}\label{INDSYS SEC}

As in \cite[\S 6]{BP_edss} and \cite[\S 9]{Per18}, we dedicate our final section to 
outlining the variations of the main results from Sections \ref{TAME_SEC} and \ref{QE_SEC} to
the \textit{indexing systems} of Blumberg-Hill \cite{BH15}, or more accurately
the mild generalization of \textit{weak indexing systems} of the authors \cite[\S 9]{Per18}, \cite[\S4.4]{BP_geo} (and independently identified by Gutierrez-White \cite{GW18}).


\subsection{Changes to model structures}
More generally, similar model structures exist for any (weak) indexing system.
\begin{definition}
      A \textit{weak indexing system} is a sieve\footnote{
        A subcategory $\mathcal B \subseteq \mathcal C$ is a \textit{sieve} if for all maps $A \to B$ in $\mathcal C$
        with $B \in \mathcal B$, both $A$ and $f$ are in $\mathcal B$.}
      $\Omega_\F \subseteq \Omega_G$
      satisfying the following Segal condition:
      come back!
\end{definition}

Unpacking, let $\Sigma_\F \subseteq \Sigma_G$ denote the image of $\Omega_\F$ under the functor $\mathsf{lr}$.
As $\Sigma_G$ is equivalent to the category $\coprod_n \O_{\mathrm{Gr}_n}$
where $\mathrm{Gr}_n$ is the collection of graph subgroups of $G \times \Sigma_n$ \todo{citation},
$\Sigma_\F$ corresponds to some sub-$(G, \Sigma)$-family $\F = \set{\F_n}$ of $\mathrm{Gr} = \set{\mathrm{Gr}_n}$
so each $\F_n$ is a family of graph subgroups of $G \times \Sigma_n$.      

Extending the above, a map is \textit{$\F$-normal} (resp. \textit{$\F$-anodyne})
if it is in the smallest saturated class of maps containing the
boundary inclusions (resp. generating $G$-inner horn inclusions) for $T \in \Omega_\F$.
Additionally, a map is a \textit{trivial $\F$-fibration} (resp. \textit{$\F$-$\infty$-operad}) if it has the right lifting property with respect to $\F$-normal maps (resp. $\F$-anodyne maps).

\begin{theorem}[{\cite[\S 9]{Per18}}]
      There exists an $\F$-model structure on $\dSet^G$, denoted $\dSet^G_\F$, such that
      cofibrations are $\F$-normal monomorphisms,
      fibrant objects are $\F$-$\infty$-operads,
      fibrations between fibrant objects are precisely the maps $X \to Y$ such that the functors associated to each of the fixed-point homotopy categories $\tau j^{\**}(X^H \to Y^H)$ are categorical fibrations for all $H \leq G$,
      and weak equivalences are the smallest saturated class of maps containing $\F$-inner anodyne extensions and trivial $\F$-fibrations, and is closed via 2-out-of-3.
\end{theorem}


For $\sOp^G_\F$:
\begin{enumerate}
\item $\Omega(T)$ is $\F$-cofibrant iff $\Omega[T]$ is $\F$-tame cofibrant iff $\Omega(T)$ is $\Sigma_\F$-cofibrant iff $T \in \Sigma_\F$.
\item all generating (trivial) $\F$-cofibrations are $\Sigma_\F$-cofibrations.
\end{enumerate}


\subsection{Changes to proofs}



For COMUOTOHOM PROP: replace all instances of $\Omega_G$ with $\Omega_{\mathcal F}$ - in particular, $X \to Y$ is a simplicial equivalence in $\mathsf{sdSet}^G_{\mathcal F}$ iff $X(\Omega[T]) \to Y(\Omega[T])$ is a Kan equivalence for all $T \in \Omega_{\mathcal F}$. 






\newpage

\section{Scratchwork}




\subsection{Semi-cofibrantly generated}


The following codifies a formal argument implicit in the proof of \cite[Thm. 7.19]{CM13b}.

\begin{definition}
Given a set $J$ of maps that admit the small object argument, we say that $X \in \mathcal{M}$ is \textit{$J$-fibrant} if $X \to \**$ has the right lifting property against maps in $J$.

Further, given $D$ a class of maps in $\mathcal{M}$,
we write $D_{J\text{-fib}} \subseteq D$ to denote 
the subclass of maps whose target is $J$-fibrant.
\end{definition}

\begin{lemma}\label{SEMICOF LEM}
	Let $\mathcal{M}$ be a model category with $(C,W,F)$
	the corresponding classes of cofibrations, weak equivalences and fibrations. 
	Further, $J$ be a set of maps admitting the small object argument and such that:
\begin{itemize}
	\item[(i)] $J \subseteq C \cap W$;
	\item[(ii)] 
	$\left(J^{\boxslash} \cap W \right)_{J\text{-fib}}
	\subseteq \left( F \cap W \right)_{J\text{-fib}}$.
\end{itemize}
Then one further has that:
\begin{itemize}
	\item[(a)]
	$\left(\prescript{\boxslash}{}{\left(J^{\boxslash}\right)}\right)_{J\text{-fib}}
	= 
	\left( C \cap W \right)_{J\text{-fib}}$;
	\item[(b)]
	$\left(J^{\boxslash} \right)_{J\text{-fib}}
	= F_{J\text{-fib}}$.
\end{itemize}
\end{lemma}

\begin{remark}
Rephrasing (b), one has that the fibrant objects of $\mathcal{M}$ are precisely the $J$-fibrant objects
and thus that the fibrations between fibrant objects are precisely the $J$-fibrations.
\end{remark}

\begin{proof}
	To check (a), recalling first that 
	$\prescript{\boxslash}{}{\left(J^{\boxslash}\right)}$
	is the saturation of $J$, one has that (i) in fact implies 
	$\prescript{\boxslash}{}{\left(J^{\boxslash}\right)}
		\subseteq C \cap W $.
	For the converse direction, given a trivial cofibration
	$A \to Y$ with $J$-fibrant target,
	form the factorization 
	$A \to X \to Y$ as a 
	$J$-cofibration followed by a $J$-fibration. 
	By the first direction the map $A\to X$ is a weak equivalence, and thus by 2-out-of-3 so is $X \to Y$.
	But then by (ii) the map $X \to Y$ is a trivial fibration, so that the lifting below exists,
	showing that $A \to Y$ is a retract of $A \to X$, and thus also in the saturation $\prescript{\boxslash}{}{\left(J^{\boxslash}\right)}$, 
	as desired.
\[
\begin{tikzcd}
	A \ar[>->]{r}{J} \ar[>->]{d}[swap]{\sim}&
	X \ar[->>]{d}{J}
%& &
%	A \ar[>->]{r}{\sim} \ar[>->]{d}[swap]{\sim}&
%	Y \ar[->>]{d}{\sim}
\\
	Y \ar[equal]{r} \ar[dashed]{ru} & Y
%& &
%	X \ar[equal]{r} \ar[dashed]{ru} & X
\end{tikzcd}
\]

To check (b), one direction is again immediate from (i),
since $J^{\boxslash} \supseteq (C \cap W)^{\boxslash} = F$.
For the converse direction, it suffices to show that 
a $J$-fibration $X\to Y$ with $J$-fibrant target has the right lifting property against trivial cofibrations, as on the left diagram below.
After factoring the bottom horizontal map as a $J$-cofibration followed by a $J$-fibration as on the right diagram, it suffices to shows that a lift $B' \to X$ exists.
But since $B'$ is $J$-fibrant, this follows from (a), which shows that the composite $A \to B \to B'$ is a $J$-cofibration.
\[
\begin{tikzcd}
	A \ar{r} \ar[>->]{d}[swap]{\sim}&
	X \ar[->>]{d}{J}
&&
	A \ar{rr} \ar[>->]{d}[swap]{\sim}&&
	X \ar[->>]{d}{J}
\\
	B \ar{r} \ar[dashed]{ru} & Y
&&
	B \ar[>->]{r}[swap]{J} &
	B' \ar[->>]{r}[swap]{J} \ar[dashed]{ru}
	& Y
\end{tikzcd}
\]
\end{proof}

\begin{remark}
	Analyzing the proof above, one is free to replace the class of fibrant objects with any other class that is compatible with $J$-fibrations, in the sense that if 
	$X \to Y$ is a $J$-fibration and $Y$ is in the class, then so is $X$.
\end{remark}





\subsection{Formalizing some stuff}



\begin{lemma} \label{INTER_LEM}
Let $\mathcal{O} \in \mathsf{sOp}$ and let
$g \colon x \to y$ be an equivalence in $\mathcal{O}$.

Then there exists a countable, cofibrant and contractile $H \in \mathsf{sOp}_{\{0,1\}}$ 
and a map 
$\varphi \colon H \to \mathcal{O}$
such that 
$g$ is in the image of $\varphi$. 
\end{lemma}


\begin{proof}
	We start by considering the case where $\mathcal{O}$ is locally fibrant.
	
	$g$ can be regarded as a map
	$[1] \xrightarrow{g} \mathcal{O}$,
	and one thus likewise gets a map
	$\Delta[1] \xrightarrow{g}  hcN \mathcal{O}$
	which is an equivalence in the 
	$\infty$-category $hcN \mathcal{O}$,
	so that one can find a (non-unique) factorization
	$\Delta[1] \to J \to hcN \mathcal{O}$
	which by adjunction yields a factorization
	$[1]=W_!\Delta[1] \to W_! J \to \mathcal{O}$,
	which establishes the desired claim 
	since $W_! J$ is contractible due to 
	Example \ref{WJ EX}.
	
	For a general $\mathcal{O}$, 
	consider first a local fibrant replacement
	$F \colon \mathcal{O} \to \mathcal{O}'$.
	One can hence find a map 
	$W_! J \to \mathcal{O}'$ such that
	$F(g)$ is in its image. 
	We now factor this map as
	$W_! J \xrightarrow{\sim} H \to \mathcal{O}'$
	where the second map is a local fibration.
	
	One can now form a pullback
\[
\begin{tikzcd}
	\tilde{H} \ar{r} \ar{d} & H' \ar{d}
\\
	\mathcal{O} \ar{r} & \mathcal{O}'
\end{tikzcd}
\]
where $\tilde{H}$ is seen to be contractible since
$\mathsf{sSet}$ is right proper.
	A priori, $\tilde{H}$ will need not be countable nor cofibrant, but this is easily rectified:
	indeed one can show that any countable subcomplex of $\tilde{H}$ is contained in a contractible countable subomplex, 
	yielding a countable contractible subcomplex whose image in $\mathcal{O}$ contains $g$. Lastly, performing a cofibrant replacement of that complex finishes the proof.
\end{proof}



\begin{example}\label{WJ EX}
	Let $J = N \widetilde{[1]}$ be the nerve of the contractible groupoid on two objects.
	
	Then there is an identification
\[
	W_{!} J \simeq \mathbb{F}^{\bullet} \widetilde{[1]}
\]
	where $\mathbb{F}$ denotes the (unital) free operad monad.

	To see this, we start by describing
	$\mathbb{F}^{\bullet} \widetilde{[1]}$.
	Writing $f \colon 0 \to 1$ and 
	$g \colon  1\to 0$ for the non-identity arrows in 
	$\widetilde{[1]}$ (so that $g=f^{-1}$), the $0$-simplices of $\mathbb{F}^{\bullet} \widetilde{[1]}$
	are the alternating words
	$f,g,fg,gf,fgf,gfg,fgfg,gfgf,\cdots$
	in the letters $f$, $g$.
	More generally
	$n$-simplices are given by equipping such alternating words with ``$n$ nested layers of brackets''
	(so that, for example, 
	$\left((f)(gf)\right) 
	\left( (gf) \right)$
	encodes a $2$-simplex).
	Alternatively, given an alternating word of length $l$, such bracketings are encoded by a flag of subsets
	$F_1 \subseteq F_2 \subseteq \cdots 
	\subseteq F_n \subseteq \{1,\cdots,l-1\}$.
	
	To describe $W_{!} J$, we apply the explicit description of the $W_{!} (-)$ construction given in 
	\cite{DS11}.
	Following \cite[Cor. 4.8]{DS11}, the $n$-simplices of $W_{!} J$ are uniquely encoded by a map
\begin{equation}\label{NECMAP EQ}
	N = \Delta^{k_1} \vee \Delta^{k_2} 
	\vee \cdots \vee \Delta^{k_r} 
	\to J 
\end{equation}
which is totally nondegenerate (this means that all simplices $\Delta^{k_i} \to J$ are nondegenerate)
together with a flag of subsets
	$\boldsymbol{J}(N) = 
	G_0 \subseteq 
	G_1 \subseteq \cdots \subseteq
	G_{n-1} \subseteq \boldsymbol{E}^{\mathsf{i}}(N)$.
	Noting that the nondegenerate simplices of $J$ (other than the points $0,1$)
	are themselves identified with alternating words
	$f,g,fg,gf,fgf,gfg,fgfg,gfgf,\cdots$,
	one sees that so is the map \eqref{NECMAP EQ}.
	Therefore, we see that a $n$-simplex of 
	$W_{!} J$ is uniquely, determined by some alternating word of some size $l$ together with a flag  
	$G_0 \subseteq 
	G_1 \subseteq \cdots \subseteq
	G_{n-1} \subseteq \boldsymbol{E}^{\mathsf{i}}(N)
	=\{1,\cdots,l-1\}$, since 
	$G_0$ suffices to recover the domain of 
	\eqref{NECMAP EQ}.
	
	This shows that $W_{!} J$ $\mathbb{F}^{\bullet} \widetilde{[1]}$ indeed have the same simplices.
	The fact that the simplicial operators coincide can be readily checked explicitly with the most interesting case is that of the top differential $d_n$, which in either case is induced by multiplication in 
	$\widetilde{[1]}$ 
	(that this is the case for $W_{!} J$ follows from the description of the simplicial operators in 
	\cite[Cor. 4.4]{DS11} together with the description of the ``flanking'' procedure in \cite[Lemma 4.5]{DS11}).
\end{example}




\subsection{Extra lifts for $\infty$-categories}


\begin{lemma}
The inclusion 
\[[0,1,2] \cup [0,2,3,\cdots,n] \cup 
\Lambda^0[0,1,3,\cdots,n]
 \to \Lambda^{0,2}[n]\]
is built cellularly from inclusions
$\Lambda^0[k] \to \Delta[k]$ with $k<n$.
Moreover, all such cells send $[0,1]$ to $[0,1]$.
\end{lemma}

\begin{proof}
Since $[0,2,3,\cdots,n]$ is in the domain, all missing faces must contain $1$.
Moreover, since the smallest face not containing $2$
that is not in $\Lambda^0[0,1,3,\cdots,n]$ is $[1,3,\cdots,n]$,
which is also the smallest face not in $\Lambda^{0,2}[n]$,
we see that all missing faces must contain $2$ as well.

It now suffices to check that $0$ is characteristic with respect to the missing faces, i.e.
that $12\underline{a}$ is missing iff $012\underline{a}$ is missing, and this is now obvious. 
\end{proof}

\begin{remark}
	The map $\Lambda^{0,2}[n] \to \Delta[n]$ is inner anodyne ($2$ is characteristic).
	
	This observation, together with the previous lemma, are the technical core of the observation that lifts
\[
	\begin{tikzcd}
	\Lambda^{0}[n] \ar{d}  \ar{r} & X
\\
	\Delta[n] \ar[dashed]{ru}
	\end{tikzcd}
\]
exist when $X$ is an $\infty$-category and $[0,1]$ is mapped to an equivalence in $X$.
\end{remark}




\subsection{TBD}


\begin{lemma}
Suppose that a category $\Xi$ has subcategories 
$\Xi^-,\Xi^+$ which contain the isomorphisms and satisfy the unique factorization up to unique isomorphism axiom.

Write $\mathsf{Arr}(\Xi)$ for the arrow category of $\Xi$ and 
$\mathsf{Arr}^{-}(\Xi), \mathsf{Arr}^{+}(\Xi)$
for the full subcategories whose objects are arrows in $\Xi^-,\Xi^+$. 
Then $\mathsf{Arr}^{-}(\Xi)$ (resp. $\mathsf{Arr}^{+}(\Xi)$)
is initial (resp. terminal) in $\mathsf{Arr}(\Xi)$.
\end{lemma}



\begin{proof}
Given $f \in \mathsf{Arr}(\Xi)$, we need to show that there exist diagrams as below, and moreover that all such diagrams are connected. 
Existence follows from the factorization assumption. Moreover, its is straightforward from the ``uniqueness up to isomorphism'' that all connections are connected.
But by factoring the left vertical map in the diagram below, we now see that all such diagrams are connected to a factorization.
\[
\begin{tikzcd}
	\bullet \ar{r}{f} \ar{d}& 
	\bullet \ar{d}
\\
	\bullet \ar{r}[swap]{+} &
	\bullet
\end{tikzcd}
\]
%
%\[
%\begin{tikzcd}
%	\bullet \ar{r}{-} \ar{dd} \ar{rd}[swap]{-} &
%	\bullet \ar{rrr}{+} \ar{rrd}{-} &&&
%	\bullet \ar{dd}
%\\
%	&
%	\bullet \ar{rrd}[swap]{+}  && 
%	\bullet \ar{rd}{+} \ar[dashed]{ll}[swap]{\simeq}
%\\
%	\bullet 	\ar{rrr}[swap]{-} &&&
%	\bullet \ar{r}[swap]{+} & 
%	\bullet
%\end{tikzcd}
%\]
\end{proof}



\begin{lemma}\label{REDUCELAN LEM}
	Suppose that $F \colon \mathcal{C} \to \mathcal{D}$ is a functor in 
	$\mathsf{Cat}^G$.
Then the following square commutes up to natural isomorphism
\[
\begin{tikzcd}[column sep=50pt]
	\mathcal{V}^{G \ltimes \mathcal{C}} 
	\ar{r}{\mathsf{Lan}_{G \ltimes \mathcal{C} \to G \ltimes \mathcal{D}}} \ar{d}[swap]{\mathsf{fgt}}&
	\mathcal{V}^{G \ltimes \mathcal{D}} \ar{d}{\mathsf{fgt}}
\\
	\mathcal{V}^{\mathcal{C}} 
	\ar{r}[swap]{\mathsf{Lan}_{\mathcal{C} \to\mathcal{D}}} &
	\mathcal{V}^{\mathcal{D}}
\end{tikzcd}
\]
\end{lemma}


\begin{proof}
For each $d \in \mathcal{D}$ (recall that $\mathcal{D}$ and $G \ltimes \mathcal{D}$ have the same objects) one has an obvious inclusion 
$\mathcal{C} \downarrow d \to G\ltimes \mathcal{C} \downarrow d$.
Moreover, for each object of $G\ltimes \mathcal{C} \downarrow d$
there is a unique $g \in G$ such that the object is described as a composite
$F(c) \xrightarrow{g} g F(c) \to d$,
were $g F(c) \to d$ can be regarded as an object of $\mathcal{C} \downarrow d$.
One thus has a retraction 
$G \ltimes \mathcal{C} \downarrow d \to \mathcal{C} \downarrow d$
showing that $\mathcal{C} \downarrow d$ is terminal in
$G \ltimes \mathcal{C} \downarrow d$
and finishing the proof. 
\end{proof}





\newpage


% -------------------- APPENDIX --------------------

\appendix


\section{Monad for colored operads}
\label{MONAD_APDX}


Our main goal in this appendix is to establish some necessary technical results concerning the category $\mathsf{Op}(\mathcal{V})$ of colored operads,
building off the work from \cite{BP_geo}.
We begin by establishing technical definitions and relations which are the key constituents of the monad in \S \ref{CSTRINGS_SEC} and \S \ref{WRACONST SEC}.
In \S \ref{NONEQMON SEC}, we construct the monad describing colored operads $\Op(\V)$, as well as provide a useful filtration on free extensions of operads.
In \S \ref{EQMON_SEC}, we upgrade these discussions to the equivariant case.

\todo[inline]{come back}



\subsection{Colored strings}
\label{CSTRINGS_SEC}


Throughout we will let $\Omega$ denote the dendroidal category of trees, introduced in \cite{MW07}.
Further, given a tree $T\in \Omega$ we will write 
$E(T)$ for the underlying set of edges.

\begin{definition}
Let $\mathfrak{C}$ be a set of colors.
The category $\Omega_{\mathfrak{C}}$ of $\mathfrak{C}$-colored trees has as objects pairs
$(T,\mathfrak{c}_T\colon E(T) \to \mathfrak{C})$ and a map
$f\colon (T,\mathfrak{c}_T) \to (T',\mathfrak{c}_{T'})$
is a map of underlying trees $f\colon T \to T'$
such that $\mathfrak{c}_T = \mathfrak{c}_{T'} f$.

A map in $\Omega_{\mathfrak{C}}$ is the called \textit{planar/tall} if the underlying map of trees is.

The category $\Omega_{\mathfrak{C}}^n$ of \textit{planar tall strings} has as objects strings $T_0 \to T_1 \to \cdots \to T_n$ of maps that are planar and tall, and as arrows a tuple of compatible isomorphisms.
As we will not work with strings which are not planar and tall, in practice we will drop these adjectives.
\end{definition}


The key functors discussed in \cite[\S 3.4]{BP_geo} then immediately extend to the strings
$\Omega_{\mathfrak{C}}^n$. Namely, one has simplicial operators
\[
d_i \colon \Omega_{\mathfrak{C}}^n \to \Omega_{\mathfrak{C}}^{n-1},
\quad 0 \leq i \leq n;
\qquad \qquad
s_j \colon \Omega_{\mathfrak{C}}^{n} \to \Omega_{\mathfrak{C}}^{n+1},
\quad -1 \leq j \leq n
\]
which remove (resp. repeat) the $i$-th (resp. $j$-th) tree in the string,
as well as vertex operators
\[
\Omega^{n}_{\mathfrak{C}}
\xrightarrow{\boldsymbol{V}^k}
\Sigma \wr \Omega^{n-k-1}_{\mathfrak{C}}
\]
Lastly, given a map of colors 
$f \colon \mathfrak{C} \to \mathfrak{D}$
one has natural change of color functors
$f_{\**} \colon \Omega^n_{\mathfrak{C}} \to \Omega^n_{\mathfrak{D}}$.



These operators satisfy a number of compatibilities (cf. \cite[Prop. 3.90]{BP_geo}). Firstly, the $d^i$, $s^j$ operators satisfy the usual simplicial identities, 
and the $\boldsymbol{V}^k$ operators are ``additive'' in the sense that
the composite
\begin{equation}\label{VKADD EQ}
	\Omega^{n}_{\mathfrak{C}} \xrightarrow{\boldsymbol{V}^l} 
	\Sigma \wr \Omega^{n-l-1}_{\mathfrak{C}} \xrightarrow{\Sigma \wr \boldsymbol{V}^k}
	\Sigma^{\wr 2} \wr \Omega^{n-k-l-2}_{\mathfrak{C}} \xrightarrow{\sigma^0}
	\Sigma \wr \Omega^{n-k-l-2}_{\mathfrak{C}}.
\end{equation}
equals $\boldsymbol{V}^{k+l+1}$.
The next results list the compatibilities between $d^i$, $s^j$ and $\boldsymbol{V}^k$ operators.

\begin{proposition}
      \label{CATDIAG PROP}
      One has the following diagrams in the $2$-category
$\mathsf{Cat}$.
\begin{itemize}
\item[(i)]
For $0\leq i < k \leq n$ there are $2$-isomorphisms $\pi_{i,k}$ and for $-1 \leq j \leq k \leq n$ there are commutative diagrams
\begin{equation}
\begin{tikzcd}[row sep = tiny, column sep = 35pt]
	\Omega_{\mathfrak{C}}^n
	\arrow{dr}[swap,name=U]{}{\boldsymbol{V}^k} \arrow{dd}[swap]{d^i} &
&
	\Omega_{\mathfrak{C}}^n
	\arrow{dr}{\boldsymbol{V}^k} \arrow{dd}[swap]{s^j} &
\\
	& \Sigma \wr \Omega_{\mathfrak{C}}^{n-k-1}
&
	& \Sigma \wr \Omega_{\mathfrak{C}}^{n-k-1}
\\
	|[alias=V]|
	\Omega_{\mathfrak{C}}^{n-1} \arrow{ur}[swap]{\boldsymbol{V}^{k-1}} &
&
	\Omega_{\mathfrak{C}}^{n+1} \arrow{ur}[swap]{\boldsymbol{V}^{k+1}} &
\arrow[Leftrightarrow, from=V, to=U,shorten >=0.15cm,shorten <=0.15cm
,swap,"\pi_{i,k}"
]
\end{tikzcd}
\end{equation}
\item[(ii)] 
For $-1 \leq k < i \leq n$ and for $-1 \leq k \leq j \leq n$
there are commutative diagrams
\begin{equation}
\begin{tikzcd}[row sep = 10pt, column sep = 35pt]
	\Omega^n_{\mathfrak{C}}
	\arrow{r}[swap,name=U]{}{\boldsymbol{V}^k} \arrow{dd}[swap]{d^i} &
	\Sigma \wr \Omega^{n-k-1}_{\mathfrak{C}} \ar{dd}{d^{i-k-1}}
&
	\Omega^n_{\mathfrak{C}}
	\arrow{r}{\boldsymbol{V}^k} \arrow{dd}[swap]{s^j} &
	\Sigma \wr \Omega^{n-k-1}_{\mathfrak{C}} \ar{dd}{s^{j-k-1}}
\\
\\
	|[alias=V]|
	\Omega^{n-1}_{\mathfrak{C}} \arrow{r}[swap]{\boldsymbol{V}^{k}} &
	\Sigma \wr \Omega^{n-k-2}_{\mathfrak{C}}
&
	\Omega^{n+1}_{\mathfrak{C}} \arrow{r}[swap]{\boldsymbol{V}^{k}} &
	\Sigma \wr \Omega^{n-k}_{\mathfrak{C}}
\end{tikzcd}
\end{equation}
\item[(iii)] 
all $d_i$, $s_j$, $\boldsymbol{V}^k$ and $\pi_{i,k}$
are natural in $\mathfrak{C}$, i.e. for each map of colors
$f \colon \mathfrak{C} \to \mathfrak{D}$ one has commutative diagrams
\[
\begin{tikzcd}[column sep = small, row sep = small]
	\Omega^n_{\mathfrak{C}} \ar{r}{d^i} \ar{dd}[swap]{f_{\**}} &
	\Omega^{n-1}_{\mathfrak{C}} \ar{dd}{f_{\**}}
&
	\Omega^n_{\mathfrak{C}} \ar{r}{s^j} \ar{dd}[swap]{f_{\**}} &
	\Omega^{n+1}_{\mathfrak{C}} \ar{dd}{f_{\**}}
&
	\Omega^n_{\mathfrak{C}} \ar{r}{\boldsymbol{V}^k} \ar{dd}[swap]{f_{\**}} &
	\Sigma \wr \Omega^{n-k-1}_{\mathfrak{C}} \ar{dd}{f_{\**}}
&
	\Omega^n_{\mathfrak{C}}
	\ar{rrrrr}[name=toE]{\boldsymbol{V}^k} \ar{rd}[swap]{d^i} \ar{dd}[swap]{f_{\**}}
	&&&
	&&
	\Sigma \wr \Omega^{n-k-1}_{\mathfrak{C}}  \ar{dd}{f_{\**}}
\\
	&
&
	&
&
	&
&
	&
	|[alias=DBE]|
	\Omega^{n-1}_{\mathfrak{C}} \ar{rrrru}[swap]{\boldsymbol{V}^{k-1}}
\\
	\Omega^n_{\mathfrak{D}} \ar{r}{d^i} &
	\Omega^{n-1}_{\mathfrak{D}}
&
	\Omega^n_{\mathfrak{D}} \ar{r}{s^j} &
	\Omega^n_{\mathfrak{D}}
&
	\Omega^n_{\mathfrak{D}} \ar{r}{\boldsymbol{V}^k} &
	\Sigma \wr \Omega^{n-k-1}_{\mathfrak{D}}
&
	\Omega^n_{\mathfrak{D}} \ar{rrrrr}[name=toB]{\boldsymbol{V}^k} \ar{rd}[swap]{d^i}
	&&&
	&&
	\Sigma \wr \Omega^{n-k-1}_{\mathfrak{D}}
\\
	&
&
	&
&
	&
&
	&
	|[alias=D]| \Omega^{n-1}_{\mathfrak{D}} \ar{rrrru}[swap]{\boldsymbol{V}^{k-1}}
\arrow[Leftrightarrow, from=DBE, to=toE, shorten <=0.15cm,shorten >=0.15cm
,swap,"\pi"
]
	\arrow[Leftrightarrow, from=D, to=toB, shorten <=0.15cm,shorten >=0.15cm,swap,"\pi"]
	\arrow[from=DBE, to=D, crossing over, near start, swap, "f_{\**}"]
\end{tikzcd}
\]
\end{itemize}
Furthermore, the diagrams in (ii) are pullback squares in $\mathsf{Cat}$.
\end{proposition}

The following lists the compatibilities of the $\pi_{i,k}$ isomorphisms, 
which are extensions of the additivity of $\boldsymbol{V}^k$ in \eqref{VKADD EQ} and of the simpicial identities between the $d^i$, $s^j$ operators.



\begin{proposition}\label{CATDIAG2 PROP}
In each of the following items, the two composite natural transformations coincide.
\begin{itemize}
\item[(IT1)]
For $0 \leq i < k $ and $-1 \leq l \leq n-k-1$
\begin{equation}
\begin{tikzcd}[row sep = 20pt, column sep = 25pt]
	|[alias=V]|
	\Omega_{\mathfrak{C}}^{n} \ar{r}{\boldsymbol{V}^{k}}[swap,name=UU]{} \arrow{d}[swap]{d^i}&
	\Sigma \wr \Omega_{\mathfrak{C}}^{n-k-1} \ar{r}{V^l} &
	\Sigma^{\wr 2} \wr \Omega^{n-k-l-2}_{\mathfrak{C}} \ar{r}{\sigma^0} &
	\Sigma \wr \Omega^{n-k-l-2}_{\mathfrak{C}}
&
	\Omega^{n}_{\mathfrak{C}} \ar{r}{\boldsymbol{V}^{k+l+1}}[swap,name=UUU]{} \arrow{d}[swap]{d^i}&
	\Sigma \wr \Omega^{n-k-l-2}_{\mathfrak{C}} &
\\
	|[alias=VV]|
	\Omega^{n-1}_{\mathfrak{C}} \arrow{ur}[swap]{\boldsymbol{V}^{k-1}} & & &
&
	|[alias=VVV]|
	\Omega^{n-1}_{\mathfrak{C}} \arrow{ur}[swap]{\boldsymbol{V}^{k+l}} &
\arrow[Leftrightarrow, from=VV, to=UU,shorten >=0.05cm,shorten <=0.05cm
,swap,"\pi"
]
\arrow[Leftrightarrow, from=VVV, to=UUU,shorten >=0.05cm,shorten <=0.05cm
,swap,"\pi"
]
\end{tikzcd}
\end{equation}

\item[(IT2)]
For $-1 \leq k < i < k + l + 1 \leq n$
\begin{equation}
\begin{tikzcd}[row sep = 20pt, column sep = 25pt]
	\Omega^n_{\mathfrak{C}} \ar{r}{\boldsymbol{V}^k} \ar{d}[swap]{d^i} &
	|[alias=V]|
	\Sigma \wr \Omega^{n-k-1}_{\mathfrak{C}} \ar{r}{\boldsymbol{V}^{l}}[swap,name=UU]{} \arrow{d}[swap]{d^{i-k-1}} &
	\Sigma^{\wr 2} \wr \Omega^{n-k-l-2}_{\mathfrak{C}} \ar{r}{\sigma^0} &
	\Sigma \wr \Omega^{n-k-l-2}_{\mathfrak{C}}
&
	\Omega^{n}_{\mathfrak{C}} \ar{r}{\boldsymbol{V}^{k+l+1}}[swap,name=UUU]{} \arrow{d}[swap]{d^i}&
	\Sigma \wr \Omega^{n-k-l-2}_{\mathfrak{C}} &
\\
	\Omega^{n-1}_{\mathfrak{C}} \ar{r}{\boldsymbol{V}^k} &
	|[alias=VV]|
	\Sigma \wr \Omega^{n-1}_{\mathfrak{C}} \arrow{ur}[swap]{\boldsymbol{V}^{l-1}} & &
&
	|[alias=VVV]|
	\Omega^{n-1}_{\mathfrak{C}} \arrow{ur}[swap]{\boldsymbol{V}^{k+l}} &
\arrow[Leftrightarrow, from=VV, to=UU,shorten >=0.05cm,shorten <=0.05cm
,swap,"\pi"
]
\arrow[Leftrightarrow, from=VVV, to=UUU,shorten >=0.05cm,shorten <=0.05cm
,swap,"\pi"
]
\end{tikzcd}
\end{equation}
\item[(FF1)]
For $0 \leq i < i' < k \leq n$
\begin{equation}
\begin{tikzcd}[row sep = 20pt, column sep = 35pt]
	\Omega^n_{\mathfrak{C}}
	\arrow{dr}[swap,name=U]{}{\boldsymbol{V}^k} \arrow{d}[swap]{d^{i'}} &
&
	\Omega^n_{\mathfrak{C}}
	\arrow{dr}[swap,name=UUU]{}{\boldsymbol{V}^k} \arrow{d}[swap]{d^i} &
\\
	|[alias=V]|
	\Omega^{n-1}_{\mathfrak{C}} \ar{r}[near start,swap]{\boldsymbol{V}^{k-1}}[swap,name=UU]{} \arrow{d}[swap]{d^i}&
	\Sigma \wr \Omega^{n-k-1}_{\mathfrak{C}}
&
	|[alias=VVV]|
	\Omega^{n-1}_{\mathfrak{C}} \ar{r}[near start, swap]{\boldsymbol{V}^{k-1}}[swap,name=UUUU]{} \ar{d}[swap]{d^{i'-1}} &
	\Sigma \wr \Omega^{n-k-1}_{\mathfrak{C}}
\\
	|[alias=VV]|
	\Omega^{n-2}_{\mathfrak{C}} \arrow{ur}[swap]{\boldsymbol{V}^{k-2}} &
&
	|[alias=VVVV]|
	\Omega^{n-2}_{\mathfrak{C}} \arrow{ur}[swap]{\boldsymbol{V}^{k-2}} &
\arrow[Leftrightarrow, from=V, to=U,shorten >=0.05cm,shorten <=0.05cm
,swap,"\pi"
]
\arrow[Leftrightarrow, from=VV, to=UU,shorten >=0.25cm,shorten <=0.05cm
,swap,"\pi"
]
\arrow[Leftrightarrow, from=VVV, to=UUU,shorten >=0.05cm,shorten <=0.05cm
,swap,"\pi"
]
\arrow[Leftrightarrow, from=VVVV, to=UUUU,shorten >=0.25cm,shorten <=0.05cm
,swap,"\pi"
]
\end{tikzcd}
\end{equation}
\item[(FF2)]
For $0 \leq i < k < i' \leq n$
\begin{equation}
\begin{tikzcd}[row sep = 20pt, column sep = 35pt]
	\Omega^n_{\mathfrak{C}}
	\arrow{r}[swap,name=U]{}{\boldsymbol{V}^k} \arrow{d}[swap]{d^{i'}} &
	\Sigma \wr \Omega^{n-k-1}_{\mathfrak{C}} \ar{d}{d^{i'-k-1}}
&
	\Omega^n_{\mathfrak{C}}
	\arrow{dr}[swap,name=UUU]{}{\boldsymbol{V}^k} \arrow{d}[swap]{d^i} &
\\
	|[alias=V]|
	\Omega^{n-1}_{\mathfrak{C}} \ar{r}{\boldsymbol{V}^{k}}[swap,name=UU]{} \arrow{d}[swap]{d^i}&
	\Sigma \wr \Omega^{n-k-2}_{\mathfrak{C}}
&
	|[alias=VVV]|
	\Omega^{n-1}_{\mathfrak{C}} \ar{r}[near start, swap]{\boldsymbol{V}^{k-1}}[swap,name=UUUU]{} \ar{d}[swap]{d^{i'-1}} &
	\Sigma \wr \Omega^{n-k-1}_{\mathfrak{C}} \ar{d}{d^{i'-k-1}}
\\
	|[alias=VV]|
	\Omega^{n-2}_{\mathfrak{C}} \arrow{ur}[swap]{\boldsymbol{V}^{k-1}} &
&
	|[alias=VVVV]|
	\Omega^{n-2}_{\mathfrak{C}} \ar{r}[swap]{\boldsymbol{V}^{k-1}} &
	\Sigma \wr \Omega^{n-k-2}_{\mathfrak{C}}
\arrow[Leftrightarrow, from=VV, to=UU,shorten >=0.05cm,shorten <=0.05cm
,swap,"\pi"
]
\arrow[Leftrightarrow, from=VVV, to=UUU,shorten >=0.05cm,shorten <=0.05cm
,swap,"\pi"
]
\end{tikzcd}
\end{equation}
\item[(DF1)]
For 
%$0 \leq j+1 < i < k +1 \leq n +1$ or 
$-1 \leq j < i \leq k \leq n$
\begin{equation}
\begin{tikzcd}[row sep = 20pt, column sep = 35pt]
	\Omega^{n}_{\mathfrak{C}}
	\arrow{dr}[swap,name=U]{}{\boldsymbol{V}^{k}} \arrow{d}[swap]{s^j} &
&
	\Omega^{n}_{\mathfrak{C}}
	\arrow{dr}[swap,name=UUU]{}{\boldsymbol{V}^{k}} \arrow{d}[swap]{d^{i-1}} &
\\
	|[alias=V]|
	\Omega^{n+1}_{\mathfrak{C}} \ar{r}{\boldsymbol{V}^{k+1}}[swap,name=UU]{} \arrow{d}[swap]{d^i}&
	\Sigma \wr \Omega^{n-k-1}
&
	|[alias=VVV]|
	\Omega^{n-1}_{\mathfrak{C}} \ar{r}[near start, swap]{\boldsymbol{V}^{k-1}}[swap,name=UUUU]{} \ar{d}[swap]{s^j} &
	\Sigma \wr \Omega^{n-k-1}
\\
	|[alias=VV]|
	\Omega^{n}_{\mathfrak{C}} \arrow{ur}[swap]{\boldsymbol{V}^{k}} &
&
	|[alias=VVVV]|
	\Omega^{n}_{\mathfrak{C}} \arrow{ur}[swap]{\boldsymbol{V}^{k}} &
\arrow[Leftrightarrow, from=VV, to=UU,shorten >=0.25cm,shorten <=0.05cm
,swap,"\pi"
]
\arrow[Leftrightarrow, from=VVV, to=UUU,shorten >=0.05cm,shorten <=0.05cm
,swap,"\pi"
]
\end{tikzcd}
\end{equation}
\item[(DF2)]
For $0 \leq j+1 = i \leq k \leq n$ or 
$0 \leq j = i \leq k \leq n$
\begin{equation}
\begin{tikzcd}[row sep = 20pt, column sep = 35pt]
	\Omega^n_{\mathfrak{C}}
	\arrow{dr}[swap,name=U]{}{\boldsymbol{V}^k} \arrow{d}[swap]{s^j} &
&
	\Omega^n_{\mathfrak{C}}
	\arrow{dr}[swap,name=UUU]{}{\boldsymbol{V}^k} \arrow[equal]{dd} &
\\
	|[alias=V]|
	\Omega^{n+1}_{\mathfrak{C}} \ar{r}{\boldsymbol{V}^{k+1}}[swap,name=UU]{} \arrow{d}[swap]{d^i}&
	\Sigma \wr \Omega^{n-k-1}_{\mathfrak{C}}
&
	&
	\Sigma \wr \Omega^{n-k-1}_{\mathfrak{C}}
\\
	|[alias=VV]|
	\Omega^{n}_{\mathfrak{C}} \arrow{ur}[swap]{\boldsymbol{V}^k} &
&
	|[alias=VVVV]|
	\Omega^{n}_{\mathfrak{C}} \arrow{ur}[swap]{\boldsymbol{V}^k} &
\arrow[Leftrightarrow, from=VV, to=UU,shorten >=0.25cm,shorten <=0.05cm
,swap,"\pi"
]
\end{tikzcd}
\end{equation}
\item[(DF3)]
For $0\leq i < j \leq k \leq n$
\begin{equation}
\begin{tikzcd}[row sep = 20pt, column sep = 35pt]
	\Omega^n_{\mathfrak{C}}
	\arrow{dr}[swap,name=U]{}{\boldsymbol{V}^k} \arrow{d}[swap]{s^j} &
&
	\Omega^n_{\mathfrak{C}}
	\arrow{dr}[swap,name=UUU]{}{\boldsymbol{V}^k} \arrow{d}[swap]{d^{i}} &
\\
	|[alias=V]|
	\Omega^{n+1}_{\mathfrak{C}} \ar{r}{\boldsymbol{V}^{k+1}}[swap,name=UU]{} \arrow{d}[swap]{d^i}&
	\Sigma \wr \Omega^{n-k-1}_{\mathfrak{C}}
&
	|[alias=VVV]|
	\Omega^{n-1}_{\mathfrak{C}} \ar{r}[near start, swap]{\boldsymbol{V}^{k-1}}[swap,name=UUUU]{} \ar{d}[swap]{s^{j-1}} &
	\Sigma \wr \Omega^{n-k-1}_{\mathfrak{C}}
\\
	|[alias=VV]|
	\Omega^{n}_{\mathfrak{C}} \arrow{ur}[swap]{\boldsymbol{V}^k} &
&
	|[alias=VVVV]|
	\Omega^{n}_{\mathfrak{C}} \arrow{ur}[swap]{\boldsymbol{V}^k} &
\arrow[Leftrightarrow, from=VV, to=UU,shorten >=0.25cm,shorten <=0.05cm
,swap,"\pi"
]
\arrow[Leftrightarrow, from=VVV, to=UUU,shorten >=0.05cm,shorten <=0.05cm
,swap,"\pi"
]
\end{tikzcd}
\end{equation}
\item[(DF4)]
For $0 \leq i < k \leq j \leq n$
\begin{equation}
\begin{tikzcd}[row sep = 20pt, column sep = 35pt]
	\Omega^n_{\mathfrak{C}}
	\arrow{r}[swap,name=U]{}{\boldsymbol{V}^k} \arrow{d}[swap]{s^j} &
	\Sigma \wr \Omega^{n-k-1}_{\mathfrak{C}} \ar{d}{s^{j-k-1}}
&
	\Omega^n_{\mathfrak{C}}
	\arrow{dr}[swap,name=UUU]{}{\boldsymbol{V}^k} \arrow{d}[swap]{d^i} &
\\
	|[alias=V]|
	\Omega^{n+1}_{\mathfrak{C}} \ar{r}{\boldsymbol{V}^{k}}[swap,name=UU]{} \arrow{d}[swap]{d^i}&
	\Sigma \wr \Omega^{n-k}_{\mathfrak{C}}
&
	|[alias=VVV]|
	\Omega^{n-1}_{\mathfrak{C}} \ar{r}[near start, swap]{\boldsymbol{V}^{k-1}}[swap,name=UUUU]{} \ar{d}[swap]{s^{j-1}} &
	\Sigma \wr \Omega^{n-k-1}_{\mathfrak{C}} \ar{d}{s^{j-k-1}}
\\
	|[alias=VV]|
	\Omega^{n}_{\mathfrak{C}} \arrow{ur}[swap]{\boldsymbol{V}^{k-1}} &
&
	|[alias=VVVV]|
	\Omega^{n}_{\mathfrak{C}} \ar{r}[swap]{\boldsymbol{V}^{k-1}} &
	\Sigma \wr \Omega^{n-k}_{\mathfrak{C}}
\arrow[Leftrightarrow, from=VV, to=UU,shorten >=0.05cm,shorten <=0.05cm
,swap,"\pi"
]
\arrow[Leftrightarrow, from=VVV, to=UUU,shorten >=0.05cm,shorten <=0.05cm
,swap,"\pi"
]
\end{tikzcd}
\end{equation}
\end{itemize}
\end{proposition}




\subsection{The $(-)\wr A$ construction}\label{WRACONST SEC}


One of the key ideas used in \cite{BP_geo} when describing the monad on spans was the use of categories 
$\Omega^n \wr A$ defined by pullbacks diagrams of the form
\begin{equation}\label{WRASAMPLE EQ}
\begin{tikzcd}
	\Omega^n \wr A \ar{r}{\boldsymbol{V}^n} \ar{d} &
	\Sigma \wr A  \ar{d}
\\
	\Omega^n \ar{r}{\boldsymbol{V}^n} &
	\Sigma \wr \Sigma
\end{tikzcd}
\end{equation}
Moreover, these categories are related by analogues of the operators $d^i$, $s^j$, $\boldsymbol{V}^k$, $\pi_{i,k}$
which satisfy all the analogues of the compatibilities 
listed in Propositions \ref{CATDIAG PROP} and \ref{CATDIAG2 PROP}.

In \cite{BP_geo} these analogue operators were built via a somewhat adhoc method, but here we will prefer a more systematic approach which regards the $(-) \wr A$ construction as an extension and refinement of the pullback operation in $\Cat$.
% sort of $2$-categorical pullback functor.
We first introduce relevant $2$-categories.


\begin{definition}
Let $\mathcal{E} \to \mathcal{B}$ be a split Grothendieck fibration.
We write $\mathsf{Cat}\downarrow^r_{\mathcal{B}} \mathcal{E}$ for the $2$-category such that:
\begin{itemize}
	\item objects are functors $F \colon \mathcal{C} \to \mathcal{E}$; 
	
	\item an $1$-arrow from 
	$F \colon \mathcal{C} \to \mathcal{E}$
	to
	$F' \colon \mathcal{C}' \to \mathcal{E}$
	is a pair $(f,\phi)$
	formed by a functor $f\colon \mathcal{C} \to \mathcal{C}'$ and a natural transformation $\phi \colon F' f \Rightarrow F$ consisting of pullback arrows\footnote{Recall that the data of a split Grothendieck fibration includes a chosen and fixed system of cartesian morphisms we call \textit{pullback} arrows.} over $\mathcal{B}$
		\begin{equation}
		\begin{tikzcd}[row sep = tiny, column sep = 35pt]
			\mathcal{C} \arrow{dr}[name=U]{F} \arrow{dd}[swap]{f}
		\\
			& \mathcal{E}
		\\
			|[alias=V]| \mathcal{C}' \arrow{ur}[swap]{F'}
		\arrow[Rightarrow, from=V, to=U,shorten >=0.25cm,shorten <=0.25cm
		,swap,"\phi"
		]
		\end{tikzcd}
		\end{equation}
	\item a $2$-arrow from $(f,\phi)$ to $(f',\phi')$ is a $2$-arrow $\varphi \colon f \to f'$ such that
	$\phi' \circ F' \varphi = \phi$.
		\begin{equation}
		\begin{tikzcd}[column sep = 50pt]
			\mathcal{C} \arrow{dr}[name=U]{F} 
			\arrow[bend right]{dd}[swap]{f}[name=F]{}
			\arrow[bend left]{dd}{f'}[swap,name=FF]{}
			&
		&
			\mathcal{C} \arrow{dr}[name=U2]{F} 
			\arrow[bend right]{dd}[swap]{f}
			&
		\\
			& \mathcal{E}
		&
			& \mathcal{E}
		\\
			\mathcal{C}' \arrow{ur}[swap]{F'}[near start, name=V]{}
			&
		&
			|[alias=V2]| \mathcal{C}' \arrow{ur}[swap]{F'}
			&
		\arrow[Rightarrow, from=V, to=U,shorten >=0.25cm,shorten <=0.25cm
		,swap,"\phi'"
		]
		\arrow[Rightarrow, from=F, to=FF,shorten >=0.0cm,shorten <=0.0cm
		,swap,"\varphi"
		]
		\arrow[Rightarrow, from=V2, to=U2,shorten >=0.25cm,shorten <=0.25cm
		,swap,"\phi"
		]
		\end{tikzcd}
		\end{equation}
\end{itemize}
\end{definition}

Given a map $\rho \colon \mathcal{E} \to \mathcal{F}$
of split Grothendieck fibrations over $\mathcal{B}$,
we now define a pullback $2$-functor on weak right spans
\begin{equation}\label{WSPANPULL EQ}
\rho^{\**} \colon
\mathsf{Cat} \downarrow^r_\mathcal{B} \mathcal{F} 
	\to
\mathsf{Cat} \downarrow^r_\mathcal{B} \mathcal{E}.
\end{equation}

On objects, i.e. functors $F \colon \mathcal{C} \to \mathcal{F}$, one sets 
$\rho^{\**}(\mathcal{C} \to \mathcal{F})=
(\mathcal{C} \times_{\mathcal{F}} \mathcal{E}
\to \mathcal{E})
$.

On $1$-arrows, i.e. pairs 
$(f,\phi \colon F_2 \circ f \Rightarrow F_1)$
as in the bottom of the diagram below
\[
\begin{tikzcd}[column sep = 20pt, row sep = small]
	\mathcal{C}_1 \times_{\mathcal{F}} \mathcal{E} 
	\ar{rrrrr}[name=toE]{}[near end]{E_1} \ar[dashed]{rd}[swap]{\bar{f}} \ar{dd}[swap]{\pi}
	&&&
	&&
	\mathcal{E}  \ar{dd}{\rho}
\\
	&
	|[alias=DBE]|
	\mathcal{C}_2 \times_{\mathcal{F}} \mathcal{E} \ar{rrrru}[swap]{E_2}
\\
	\mathcal{C}_1 \ar{rrrrr}[name=toB]{}[near end]{F_1} \ar{rd}[swap]{f}
	&&&
	&&
	\mathcal{F} 
\\
	&
	|[alias=D]| \mathcal{C}_2 \ar{rrrru}[swap]{F_2}
\arrow[Rightarrow, from=DBE, to=toE, shorten <=0.15cm,shorten >=0.15cm,dashed
,swap,"\bar{\phi}"
]
	\arrow[Rightarrow, from=D, to=toB, shorten <=0.15cm,shorten >=0.15cm,swap,"\phi"]
	\arrow[from=DBE, to=D, crossing over, near start, "\pi"]
\end{tikzcd}
\]
we define $\rho^{\**}(f,\phi)$ as the only possible choice of dashed data
$(\bar{f},\bar{\phi})$
such that $\bar{\phi}$ consists of pullback arrows over $\mathcal{B}$
and the diagram commutes in the sense that
$\pi \bar{f} = f \pi$ and 
$\rho \bar{\phi} = \phi \pi$.
%\begin{equation}
%\begin{tikzcd}[row sep = tiny, column sep = 35pt]
%	\mathcal{C}_1 \arrow{dr}[name=U]{F_1} \arrow{dd}[swap]{f}
%\\
%	& \mathcal{F}
%\\
%	|[alias=V]| \mathcal{C}_2 \arrow{ur}[swap]{F_2}
%\arrow[Rightarrow, from=V, to=U,shorten >=0.25cm,shorten <=0.25cm
%,swap,"\phi"
%]
%\end{tikzcd}
%\end{equation}
Alternatively, one has the explicit formula
\[
\rho^{\**} (f,\phi)=
\left(
	\left( f \pi,
	\left( \phi \pi \right)^{\**} E_1 \right),
	\left( \phi \pi \right)^{\**} E_1 \Rightarrow E_1
  \right)
  ,
  \qquad
  \bar f(c,e) = (f(c), \phi_c^{\**}(e)),
  \quad \bar \phi_{c,e} \colon \phi_c^{\**}(e) \to e.
\]

%writing 
%$\pi \colon \mathcal{C}_i \times_{\mathcal{B}} \mathcal{E}
%\to \mathcal{C}_i$
%and
%$E_i \colon \mathcal{C}_i \times_{\mathcal{B}} \mathcal{E}
%\to \mathcal{E}$
%for the projections



Lastly, on a $2$-arrow $\varphi \colon (f,\phi) \Rightarrow (f',\phi')$
as on the bottom of the leftmost diagram below
\begin{equation}\label{PULL2ARR EQ}
\begin{tikzcd}[column sep = 16pt, row sep = 17pt]
	\mathcal{C}_1 \times_{\mathcal{F}} \mathcal{E} 
	\ar{rrrrr}[name=toE]{}[near end]{E_1} 
	\ar[bend left]{rd}[near start,swap,name=FE]{}
	\ar[bend right]{rd}[name=FFE]{} \ar{dd}[swap]{\pi}
	&&&
	&&
	\mathcal{E}  \ar{dd}
&&
	\mathcal{C}_1 \times_{\mathcal{F}} \mathcal{E} 
	\ar{rrrrr}[name=toE2]{}[near end]{E_1} 
	\ar[bend right]{rd}{} \ar{dd}
	&&&
	&&
	\mathcal{E}  \ar{dd}
\\
	&
	|[alias=DBE]|
	\mathcal{C}_2 \times_{\mathcal{F}} \mathcal{E} \ar{rrrru}[swap]{E_2} &&&&
&&
	&
	|[alias=DBE2]|
	\mathcal{C}_2 \times_{\mathcal{F}} \mathcal{E} \ar{rrrru}[swap]{E_2} &&&&
\\
	\mathcal{C}_1 \ar{rrrrr}[name=toB]{}[near end]{F_1} 
	\ar[bend left]{rd}[swap,name=FF]{}
	\ar[bend right]{rd} [name=F]{}
	&&&
	&&
	\mathcal{F} 
&&
	\mathcal{C}_1 \ar{rrrrr}[name=toB2]{}[near end]{F_1} 
	\ar[bend right]{rd}{}
	&&&
	&&
	\mathcal{F} 
\\
	&
	|[alias=D]| \mathcal{C}_2 \ar{rrrru}[swap]{F_2} &&&&
&&
	&
	|[alias=D2]|
	\mathcal{C}_2 \ar{rrrru}[swap]{F_2} &&&&
\arrow[Rightarrow, from=DBE, to=toE, shorten <=0.15cm,shorten >=0.15cm
,swap,"\bar{\phi}'"
]
\arrow[Rightarrow, from=DBE2, to=toE2, shorten <=0.15cm,shorten >=0.15cm
,swap,"\bar{\phi}"
]
\arrow[Rightarrow, from=D, to=toB, shorten <=0.15cm,shorten >=0.15cm,swap,"\phi'"]
\arrow[Rightarrow, from=D2, to=toB2, shorten <=0.15cm,shorten >=0.15cm,swap,"\phi"]
\arrow[Rightarrow, from=F, to=FF, shorten <=0cm,shorten >=0cm,swap,"\varphi"]
\arrow[Rightarrow, from=FFE, to=FE, shorten <=0cm,shorten >=0cm,swap,dashed,"\bar{\varphi}"]
\arrow[from=DBE, to=D, crossing over,near start,"\pi"]
\arrow[from=DBE2, to=D2, crossing over]
\end{tikzcd}
\end{equation}
we define $\rho^{\**}(\varphi)$
as the only choice of dashed $\bar{\varphi}$
such that $\bar{\phi}' \circ E_2\bar{\varphi} = \bar{\phi}$
and $\pi \bar{\varphi} = \varphi \pi$.

%one sets $\pi^{\**} \varphi (c,b,e)$ to be the unique dashed arrow in the left diagram below that lifts $\varphi(c)$.
%\[
%\begin{tikzcd}
%	\left(\phi(c)\right)^{\**} e \ar{rr} \ar[dashed]{rd} &&
%	e
%&
%	B_2 f(c) \ar{rr}{\phi(c)} \ar{rd}[swap]{\varphi(c)} &&
%	b
%\\
%	& \left(\phi'(c)\right)^{\**} e \ar{ru} &
%&
%	& B_2 f'(c) \ar{ru}[swap]{\phi'(c)} &
%\end{tikzcd}
%\]


%The associativity and unitality conditions of $\rho^{\**}$ are straightforward.



%Alternatively, $\pi^{\**}\varphi$ this is the unique dashed natural transformation in the left diagram below such that the left section commutes 
%(meaning that the two natural transformations between the two functors
%$\mathcal{C}_1 \times_{\mathcal{B}} \mathcal{E} 
%\rightrightarrows \mathcal{C}_2$ coincide) and 
%the top composite natural transformation is 
%$\pi^{\**} \phi$.



We are now ready to generalize the $(-) \wr A$
construction from \eqref{WRASAMPLE EQ}.

First, note that using the functor
$\boldsymbol{V}^n \colon \Omega^n_{\mathfrak{C}} \to \Sigma \wr \Sigma_{\mathfrak{C}}$
the categories 
$\Omega^n_{\mathfrak{C}}$ may be considered as objects in
$\mathsf{Cat} \downarrow^r_{\Sigma} \Sigma \wr \Sigma_{\mathfrak{C}}$.
Hence, given a functor $A \to \Sigma_{\mathfrak{C}}$
we define 
\begin{equation}\label{WRADEF EQ}
(-) \wr A \colon 
\mathsf{Cat} \downarrow^r_{\Sigma} \Sigma \wr \Sigma_{\mathfrak{C}}
\to
\mathsf{Cat} \downarrow^r_{\Sigma} \Sigma \wr A
\end{equation}
as the pullback \eqref{WSPANPULL EQ} for the map
$\Sigma \wr A \to \Sigma \wr \Sigma_{\mathfrak{C}}$.

As a result, one obtains categories $\Omega_{\mathfrak{C}}^n \wr A$
together with maps 
$\Omega_{\mathfrak{C}}^n \wr A 
\xrightarrow{\boldsymbol{V}^n} \Sigma \wr A$
and simplicial operators $d^i$, $s^j$ between them.
To further obtain vertex functors
$\boldsymbol{V}^k \colon \Omega^n_{\mathfrak{C}} \wr A
\to 
\Sigma \wr \left(\Omega^{n-k-1}_{\mathfrak{C}} \wr A \right)$
we first note that the $\Sigma \wr (-)$ operation can be extended to a $2$-endofunctor
\[
\begin{tikzcd}[row sep = 0pt]
	\mathsf{Cat} \downarrow^r_{\Sigma} \Sigma \wr A \ar{r}{\Sigma \wr (-)} &
	\mathsf{Cat} \downarrow^r_{\Sigma} \Sigma \wr A
\\
	\mathcal{C} \to \Sigma \wr A \ar[mapsto]{r} &
	\Sigma \wr \mathcal{C} \to \Sigma^{\wr 2} \wr A \xrightarrow{\sigma^{0}} \Sigma \wr A 
\end{tikzcd}
\]
from which it follows that one has natural identifications
$\Sigma \wr \left(\Omega^{n}_{\mathfrak{C}} \wr A \right)
\simeq 
\left(\Sigma \wr \Omega^{n}_{\mathfrak{C}}\right) \wr A $
which are readily seen to be compatible with the cosimplicial operators on $\Sigma^{\wr k} \wr (-)$.
As such, we will henceforth suppress parenthesis and write 
simply 
$\Sigma \wr \Omega^{n}_{\mathfrak{C}} \wr A$
to denote 
$\Sigma \wr \left(\Omega^{n}_{\mathfrak{C}} \wr A \right)$,
so that the $2$-functor $(-)\wr A$ in \eqref{WRADEF EQ}
yields further vertex functors 
$\boldsymbol{V}^k \colon \Omega^n_{\mathfrak{C}} \wr A
\to 
\Sigma \wr \Omega^{n-k-1}_{\mathfrak{C}} \wr A$
and natural transformations $\pi_{i,k}$
satisfying all the analogues of the compatibility conditions
in Proposition \ref{CATDIAG PROP}(i)(ii) and Proposition \ref{CATDIAG2 PROP}.

The analogue of Proposition \ref{CATDIAG PROP}(iii) requires an extra argument, and is stated in the following result.



\begin{proposition}\label{SPANPIECE PROP}
A commutative square
\begin{equation}\label{SPANPIECE EQ}
\begin{tikzcd}
	A \ar{d} \ar{r}{f} &  \ar{d} B
\\
	\Sigma_{\mathfrak{C}} \ar{r}[swap]{f} & \Sigma_{\mathfrak{D}}
\end{tikzcd}
\end{equation}
induces natural maps 
$f_{\**} \colon
\Omega_{\mathfrak{C}}^n \wr A \to 
\Omega_{\mathfrak{D}}^n \wr B $
such that the diagrams below commute.
\[
\begin{tikzcd}[column sep = 6pt, row sep = small]
	\Omega^n_{\mathfrak{C}} \wr A \ar{r}{d^i} \ar{dd}[swap]{f_{\**}} &
	\Omega^{n-1}_{\mathfrak{C}} \wr A \ar{dd}{f_{\**}}
&
	\Omega^n_{\mathfrak{C}} \wr A \ar{r}{s^j} \ar{dd}[swap]{f_{\**}} &
	\Omega^{n+1}_{\mathfrak{C}}\wr A \ar{dd}{f_{\**}}
&
	\Omega^n_{\mathfrak{C}} \wr A \ar{r}{\boldsymbol{V}^k} \ar{dd}[swap]{f_{\**}} &
	\Sigma \wr \Omega^{n-k-1}_{\mathfrak{C}} \wr A \ar{dd}{f_{\**}}
&
	\Omega^n_{\mathfrak{C}} \wr A
	\ar{rrrrr}[name=toE]{\boldsymbol{V}^k} \ar{rd}[swap]{d^i} \ar{dd}[swap]{f_{\**}}
	&&&
	&&
	\Sigma \wr \Omega^{n-k-1}_{\mathfrak{C}} \wr A  \ar{dd}{f_{\**}}
\\
	&
&
	&
&
	&
&
	&
	|[alias=DBE]|
	\Omega^{n-1}_{\mathfrak{C}} \wr A \ar{rrrru}[swap]{\boldsymbol{V}^{k-1}}
\\
	\Omega^n_{\mathfrak{D}} \wr B \ar{r}{d^i} &
	\Omega^{n-1}_{\mathfrak{D}} \wr B
&
	\Omega^n_{\mathfrak{D}} \wr B \ar{r}{s^j} &
	\Omega^n_{\mathfrak{D}} \wr B
&
	\Omega^n_{\mathfrak{D}} \wr B \ar{r}{\boldsymbol{V}^k} &
	\Sigma \wr \Omega^{n-k-1}_{\mathfrak{D}} \wr B
&
	\Omega^n_{\mathfrak{D}} \wr B \ar{rrrrr}[name=toB]{\boldsymbol{V}^k} \ar{rd}[swap]{d^i}
	&&&
	&&
	\Sigma \wr \Omega^{n-k-1}_{\mathfrak{D}} \wr B
\\
	&
&
	&
&
	&
&
	&
	|[alias=D]| \Omega^{n-1}_{\mathfrak{D}} \wr B \ar{rrrru}[swap]{\boldsymbol{V}^{k-1}}
\arrow[Leftrightarrow, from=DBE, to=toE, shorten <=0.15cm,shorten >=0.15cm
,swap,"\pi"
]
	\arrow[Leftrightarrow, from=D, to=toB, shorten <=0.15cm,shorten >=0.15cm,swap,"\pi"]
	\arrow[from=DBE, to=D, crossing over, near start, swap, "f_{\**}"]
\end{tikzcd}
\]
\end{proposition}


\begin{proof}
The desired map 
$f_{\**} \colon
\Omega_{\mathfrak{C}}^n \wr A \to 
\Omega_{\mathfrak{D}}^n \wr B $
is obtained immediately by drawing the pullback diagrams defining each term, so we focus on the slightly more interesting claim that the given diagrams commute.
To see this, we first factor \eqref{SPANPIECE EQ} as
\[
\begin{tikzcd}
	A \ar{d} \ar{r} & B \times_{\Sigma_{\mathfrak{D}}} \Sigma_{\mathfrak{C}} \ar{r} \ar{d} &  \ar{d} B
\\
	\Sigma_{\mathfrak{C}} \ar[equal]{r} & \Sigma_{\mathfrak{C}} \ar{r} & \Sigma_{\mathfrak{D}}
\end{tikzcd}
\]
and note that it suffices to prove the result separately for each half.
For the left half, the desired commutativity claims are simply the functoriality of $(-) \wr A$. On the other hand, for the right square the commutativity claims follow by instead noting that all diagrams in 
Proposition \ref{CATDIAG PROP}(iii)
can be regarded as diagrams in the $2$-category
$\mathsf{Cat} \downarrow^r_{\Sigma} \Sigma \wr \Sigma_{\mathfrak{D}}$
(by using the composites 
$\Omega_{\mathfrak{C}}^n \xrightarrow{f_{\**}} \Omega_{\mathfrak{D}}^n \xrightarrow{\boldsymbol{V}^n} \Sigma \wr \Sigma_{\mathfrak{D}}$) 
and then applying the pullback functor $(-) \wr B$. 
\end{proof}



Next, note that using the composite functors 
$\Omega^n_{\mathfrak{C}} \wr A \to \Omega^n_{\mathfrak{C}} 
\xrightarrow{d^{0,\cdots,n}} \Sigma_{\mathfrak{C}}$,
one can regard the $\Omega_{\mathfrak{C}}^n \wr (-)$ constructions
as endofunctors on the regular $1$-overcategory
$\mathsf{Cat} \downarrow \Sigma_{\mathfrak{C}}$.


\begin{proposition}\label{ASSOCIDS PROP}
Let $k,l\geq -1$. One has canonical natural identifications 
$\Omega^k_{\mathfrak{C}} \wr \Omega^l_{\mathfrak{C}} \wr A
\simeq 
\Omega^{k+l+1}_{\mathfrak{C}} \wr A $.

Moreover, these identifications are associative in the sense that for any $k,l,m \leq -1$ the iterated composite identifications below coincide.
\[
\Omega^k_{\mathfrak{C}} \wr \Omega^l_{\mathfrak{C}} \wr \Omega^m_{\mathfrak{C}} \wr A
	\simeq 
\Omega^{k+l+1}_{\mathfrak{C}} \wr \Omega^m_{\mathfrak{C}} \wr A
	\simeq 
\Omega^{k+l+m+2}_{\mathfrak{C}} \wr A
\qquad
\Omega^k_{\mathfrak{C}} \wr \Omega^l_{\mathfrak{C}} \wr \Omega^m_{\mathfrak{C}} \wr A
	\simeq 
\Omega^{k}_{\mathfrak{C}} \wr \Omega^{l+m+1}_{\mathfrak{C}} \wr A
	\simeq 
\Omega^{k+l+m+2}_{\mathfrak{C}} \wr A
\]
Moreover, the identifications above further induce the following identifications
\[
d^i \wr \Omega^l \wr A \simeq d^i \wr A
	\quad
\pi_{i,k} \wr \Omega^l \wr A \simeq \pi_{i,k} \wr A
	\quad
s^j \wr \Omega^l \wr A \simeq d^j \wr A
	\quad
\Omega^k \wr d^i \wr A \simeq d^{k+i+1} \wr A
	\quad
\Omega^k \wr s^j \wr A \simeq s^{k+j+1} \wr A
\]
\end{proposition}


\begin{proof}
	The first claim follows by noting that all squares in the diagram below are pullback squares
\[
\begin{tikzcd}
	\Omega^{k+l+1}_{\mathfrak{C}} \wr A \ar{r}{\boldsymbol{V}^k} \ar{d} &
	\Sigma \wr \Omega^{l}_{\mathfrak{C}} \wr A  \ar{d} \ar{r}{\boldsymbol{V}^l} &
	\Sigma^{\wr 2} \wr A \ar{d}
\\
	\Omega^{k+l+1}_{\mathfrak{C}} \ar{r}{\boldsymbol{V}^k} 
	\ar{d}[swap]{d^{k+1,\cdots,k+l+1}} &
	\Sigma \wr \Omega^{l}_{\mathfrak{C}} \ar{r}{\boldsymbol{V}^l}
	\ar{d}{d^{0,\cdots,l}} &
	\Sigma^{\wr 2} \wr \Sigma_{\mathfrak{C}}
\\
	\Omega^{k}_{\mathfrak{C}} \ar{r}{\boldsymbol{V}^k} &
	\Sigma \wr \Sigma_{\mathfrak{C}}
\end{tikzcd}
\]
while associativity follows from the obvious extension of the above diagram.

For the additional identifications, 
those identifications concerning $d^i$ and $\pi_{i,k}$
follow from the left diagram below 
(the bottom section of which commutes by 
Proposition \ref{CATFDIAG2 PROP} (FF2)),
the identification concerning $d^{k+i+1}$ follows from the rightmost diagram, and the identifications concerning 
$s^j$ and $s^{k+j+1}$
follow from obvious analogues of these diagrams.
\[
\begin{tikzcd}[row sep = 10pt]
	\Omega^{k+l+1}_{\mathfrak{C}} \wr A \ar{rrr}[name=TT,swap]{} \ar{rd}[swap]{d^i} \ar{dd} &&&
	\Sigma \wr \Omega^{l}_{\mathfrak{C}} \wr A \ar{dd}
&
	\Omega^{k+l+1}_{\mathfrak{C}} \wr A \ar{rr} \ar{rd}[swap]{d^{k+j+1}} \ar{dd} &&
	\Sigma \wr \Omega^{l}_{\mathfrak{C}} \wr A \ar{rd} \ar{dd}
\\
	&
	|[alias=TD]|
	\Omega^{k+l}_{\mathfrak{C}} \wr A \ar{rru} &&
&
	&
	\Omega^{k+l}_{\mathfrak{C}} \wr A \arrow[rr, crossing over] &&
	\Sigma \wr \Omega^{l-1}_{\mathfrak{C}} \wr A 	 \ar{dd}
\\
	\Omega^{k+l+1}_{\mathfrak{C}} \ar{rrr}[name=MT,swap]{} \ar{rd}[swap]{d^i} \ar{dd} &&&
	\Sigma \wr \Omega^{l}_{\mathfrak{C}} \ar{dd}
&
	\Omega^{k+l+1}_{\mathfrak{C}} \ar{rr} \ar{rd}[swap]{d^{k+j+1}} \ar{dd}&&
	\Sigma \wr \Omega^{l}_{\mathfrak{C}} \ar{rd} \ar{dd}
\\
	&
	|[alias=MD]|
	\Omega^{k+l}_{\mathfrak{C}} \ar{rru} \arrow[uu, leftarrow, crossing over] &&
&
	&
	\Omega^{k+l}_{\mathfrak{C}} \arrow[rr, crossing over] \arrow[uu, leftarrow, crossing over] &&
	\Sigma \wr \Omega^{l-1}_{\mathfrak{C}} \ar{dd}
\\
	\Omega^{k}_{\mathfrak{C}} \ar{rrr}[name=DT,swap]{} \ar{rd}[swap]{d^i} &&&
	\Sigma \wr \Sigma_{\mathfrak{C}}
&
	\Omega^{k}_{\mathfrak{C}} \ar{rr} \ar[equal]{rd} &&
	\Sigma \wr \Sigma_{\mathfrak{C}} \ar[equal]{rd}
\\
	&
	|[alias=DD]|
	\Omega^{k-1}_{\mathfrak{C}} \ar{rru} \arrow[uu, leftarrow, crossing over] &&
&
	&
	\Omega^{k}_{\mathfrak{C}} \ar{rr} \arrow[uu, leftarrow, crossing over] &&
	\Sigma \wr \Sigma_{\mathfrak{C}} 
\arrow[Leftrightarrow, from=TT, to=TD,shorten >=0.05cm,shorten <=0.05cm,
"\pi_{i}"
]
\arrow[Leftrightarrow, from=MT, to=MD,shorten >=0.05cm,shorten <=0.05cm,
"\pi_{i}"
]
\arrow[Leftrightarrow, from=DT, to=DD,shorten >=0.05cm,shorten <=0.05cm,
"\pi_{i}"
]
\end{tikzcd}
\]
\end{proof}




\subsection{The non-equivariant monad} \label{NONEQMON SEC}

\subsubsection{The non-equivariant monad on colored spans}

Our task in this section is to provide a suitable description of
the catgory of colored operads $\mathsf{Op}(\mathcal{V})$.
We will mimic the approach used in \cite{BP_geo},
by first building a monad on a suitable category of spans, which is then transferred to $\mathsf{Sym}(\mathcal{V})$ along the Kan extension adjunction.



\begin{definition}
The category $\mathsf{WSpan}^l(\Sigma_{\bullet}^{op},\mathcal{V})$ has
\begin{itemize}
\item objects given by a choice of a set of colors $\mathfrak{C}$
and a span $\Sigma^{op}_{\mathfrak{C}} \leftarrow A^{op} \rightarrow \mathcal{V}$
\item morphisms given by a choice of a map of colors
$f \colon \mathfrak{C} \to \mathfrak{D}$
such that the square on the left below commuites, plus a natural transformation as written.
% together with the data in the following diagram
\begin{equation}\label{COLORSPANMAP EQ}
\begin{tikzcd}[column sep = 20pt]
	\Sigma_{\mathfrak{C}}^{op}
		\ar{d}[swap]{f_{\**}} &
	A^{op}
		\ar{r}[name=U,swap]{} \ar{d} \ar{l} &
	\mathcal{V}	
\\
	\Sigma_{\mathfrak{D}}^{op}
		&
	|[alias=V]|
	B^{op} \ar{l}
		\ar{ru}
\arrow[Leftarrow, from=V, to=U,shorten >=0.05cm,shorten <=0.05cm]
\end{tikzcd}
\end{equation}
\end{itemize}
\end{definition}



\begin{remark}
By definition there is a forgetful functor
$\mathsf{WSpan}^l(\Sigma_{\bullet}^{op},\mathcal{V}) \to \mathsf{F}$
which remembers the set of colors.
Moreover, one easily checks that this is a Grothendieck fibration, with the cartesian arrows being the diagrams \eqref{COLORSPANMAP EQ} where the square is a pullback square and the natural transformation is an identity.
\end{remark}



\begin{remark}\label{LANADJ REM}
Given a $\Sigma^{op}_{\mathfrak{C}} \leftarrow A^{op} \rightarrow \mathcal{V}$
one can the form the left Kan extension
\[
\begin{tikzcd}[column sep = 30pt]
	A^{op}
		\ar{r}[name=U,swap]{}{F} \ar{d} &
	\mathcal{V}	
\\
	|[alias=V]|
	\Sigma_{\mathfrak{C}}^{op} 
		\ar[dashed]{ru}[swap]{\mathsf{Lan} F}
\arrow[Leftarrow, from=V, to=U,shorten >=0.05cm,shorten <=0.05cm]
\end{tikzcd}
\]
and it is straightforward to check that this defines an adjunction
\[
	\mathsf{Lan} \colon
	\mathsf{WSpan}^l(\Sigma_{\bullet}^{op},\mathcal{V}) 
\rightleftarrows
	\mathsf{Sym}(\mathcal{V})
	\colon \iota
\]
where the inclusion $\iota$ sends $\Sigma^{op}_{\mathfrak{C}} \to \mathcal{V}$ 
to the span
$\Sigma^{op}_{\mathfrak{C}} \xleftarrow{=} \Sigma^{op}_{\mathfrak{C}} \to \mathcal{V}$.
Moreover, it is straightforward to check that this is a fibered adjunction over $\mathsf{F}$, meaning that both functors as well as the adjunction unit and counit are compatible with the projection to $\mathsf{F}$ in the obvious way (cf. Definition \ref{FIBMON DEF}).
\end{remark}



\begin{remark}
One can also define a larger category 
$\mathsf{WSpan}^l( - ,\mathcal{V})$
where the categories $\Sigma_{\mathfrak{C}}^{op}$ in the spans (and functors between them) are allowed to be any category (any functor),
in which case left Kan extension defines an adjunction (cf. Remark \ref{SUBCATDOWNL REM})
\[
	\mathsf{Lan} \colon
	\mathsf{WSpan}^l( - ,\mathcal{V}) 
\rightleftarrows
	\mathsf{Cat} \downarrow ^l \mathcal{V}
	\colon \iota
\]
\end{remark}



\begin{definition}\label{NCOLOR DEF}
The monad $N$ on 
$\mathsf{WSpan}^l(\Sigma_{\bullet}^{op},\mathcal{V})$
sends the span 
$\Sigma^{op}_{\mathfrak{C}} \leftarrow A^{op} \to \mathcal{V}$
to the (opposite of the) composite span in
\begin{equation}\label{NCOLOR EQ}
\begin{tikzcd}
	\Omega^0_{\mathfrak{C}} \wr A \ar{r}{\boldsymbol{V}^0} \ar{d} &
	\Sigma \wr A  \ar{d} \ar{r} &
	\Sigma \wr \mathcal{V}^{op} \ar{r}{\otimes} &
	\mathcal{V}^{op}
\\
	\Omega^0_{\mathfrak{C}} \ar{r}{\boldsymbol{V}^0} \arrow[d, "\mathsf{lr}"'] &
	\Sigma \wr \Sigma_{\mathfrak{C}} 
\\
	\Sigma_{\mathfrak{C}}
\end{tikzcd}
\end{equation}
has monad multiplication
$\mu \colon N N
\Rightarrow 
N$ given by the diagram
\begin{equation}\label{NMONMULTTR EQ}
\begin{tikzcd}
	\Sigma_{\mathfrak{C}} \ar[equal]{d}&
	\Omega^1_{\mathfrak{C}} \wr A \ar{l} \arrow[r, "V^0"] \ar{d}[swap]{d^0}&
	\Sigma \wr \Omega^0_{\mathfrak{C}} \wr A \arrow[r, "V^0"] &
	|[alias=U]|
	\Sigma^{\wr 2} \wr A \ar{r} \ar{d}[swap]{\sigma^0} &
	\Sigma^{\wr 2} \wr \mathcal{V}^{op} \ar{r}{\otimes} \ar{d}[swap]{\sigma^0} &
	\Sigma \wr \mathcal{V}^{op} \ar{r}{\otimes} &
	|[alias=UU]|
	\mathcal{V}^{op} \ar[equal]{d}
\\
	\Sigma_{\mathfrak{C}} &
	|[alias=V]|
	\Omega^0_{\mathfrak{C}} \wr A \ar{l} \ar{rr} & &
	\Sigma \wr A \ar{r} &
	|[alias=VV]|
	\Sigma \wr \mathcal{V}^{op} \ar{rr}{\otimes} & &
	\mathcal{V}^{op}
\arrow[Leftrightarrow, from=V, to=U,shorten >=0.15cm,shorten <=0.15cm
,swap,"\pi"
]
\arrow[Leftrightarrow, from=VV, to=UU,shorten >=0.15cm,shorten <=0.15cm
]
\end{tikzcd}
\end{equation}
(where we are using Proposition \ref{ASSOCIDS PROP} to identify the top span as $NN(A)$),
and unit
$\eta \colon id \Rightarrow N$ given by
\begin{equation}\label{NMONIDTR EQ}
\begin{tikzcd}
	\Sigma_{\mathfrak{C}} \ar[equal]{d} & 
	A \ar{d}[swap]{s^{-1}} \ar{l} \ar[equal]{r} &
	A \ar{d}[swap]{\delta^0} \ar{r} &
	\mathcal{V}^{op} \ar{d}[swap]{\delta^0} \ar[equal]{r} &
	\mathcal{V}^{op} \ar[equal]{d}
\\
	\Sigma_{\mathfrak{C}} &
	\Omega^0_{\mathfrak{C}} \wr A \ar{l} \ar{r} &
	\Sigma \wr A \ar{r} &
	\Sigma \wr \mathcal{V}^{op} \ar{r}{\otimes} &
	\mathcal{V}^{op}
\end{tikzcd}
\end{equation}

The fact that $N$ is functorial follows from Proposition \ref{SPANPIECE PROP}.
\end{definition}





\begin{proposition}\label{MONISMON PROP}
$N$ is a monad on $\mathsf{WSpan}^l(\Sigma_{\bullet}^{op},\mathcal{V})$.
\end{proposition}


\begin{proof}
To check associativity, the functor $\mu N \colon 
N N N
\Rightarrow N N$
is encoded by the diagram
\[
\begin{tikzcd}
	\Omega^2_{\mathfrak{C}} \wr A \ar{r} \ar{d}[swap]{d^0} &
	\Sigma \wr \Omega^1_{\mathfrak{C}} \wr A \ar{r} &
	|[alias=UUU]|
	\Sigma^{\wr 2} \wr \Omega^0_{\mathfrak{C}} \wr A
	\ar{d}[swap]{\sigma^0} \ar{r} &
	\Sigma^{\wr 3} \wr A \ar{d}[swap]{\sigma^0} \ar{r} &
	\Sigma^{\wr 3} \wr \mathcal{V}^{op} \ar{d}[swap]{\sigma^0} \ar{r}{\otimes} &
	\Sigma^{\wr 2} \wr \mathcal{V}^{op} \ar{d}[swap]{\sigma^0} \ar{r}{\otimes} &
	\Sigma \wr \mathcal{V}^{op} \ar{r}{\otimes} & 
	|[alias=UUUU]|
	\mathcal{V}^{op} \ar[equal]{d}
\\
	|[alias=VVV]|
	\Omega^1_{\mathfrak{C}} \wr A \ar{rr} \ar{d}[swap]{d^0} & &
	\Sigma \wr \Omega^0_{\mathfrak{C}} \wr A \ar{r} &
	|[alias=U]|
	\Sigma^{\wr 2} \wr A \ar{r} \ar{d}[swap]{\sigma^0} &
	\Sigma^{\wr 2} \wr \mathcal{V}^{op} \ar{r}{\otimes} \ar{d}[swap]{\sigma^0} &
	|[alias=VVVV]|
	\Sigma \wr \mathcal{V}^{op} \ar{rr}{\otimes} & &
	|[alias=UU]|
	\mathcal{V}^{op} \ar[equal]{d}
\\
	|[alias=V]|
	\Omega^0_{\mathfrak{C}} \wr A \ar{rrr} & & &
	\Sigma \wr A \ar{r} &
	|[alias=VV]|
	\Sigma \wr \mathcal{V}^{op} \ar{rrr}{\otimes} & & &
	\mathcal{V}^{op}
\arrow[Leftrightarrow, from=V, to=U,shorten >=0.15cm,shorten <=0.15cm
,swap,"\pi"
]
\arrow[Leftrightarrow, from=VV, to=UU,shorten >=0.15cm,shorten <=0.15cm
]
\arrow[Leftrightarrow, from=VVV, to=UUU,shorten >=0.15cm,shorten <=0.15cm
,swap,"\pi"
]
\arrow[Leftrightarrow, from=VVVV, to=UUUU,shorten >=0.15cm,shorten <=0.15cm
]
\end{tikzcd}
\]
while the functor
$ N \mu \colon 
N N N
\Rightarrow N N$
is encoded by
\[
\begin{tikzcd}
	\Omega^2_{\mathfrak{C}} \wr A \ar{d}[swap]{d^1} \ar{r} &
	\Sigma \wr \Omega^1_{\mathfrak{C}} \wr A \ar{d}[swap]{d^0} \ar{r} &
	\Sigma^{\wr 2} \wr \Omega^0_{\mathfrak{C}} \wr A \ar{r} &
	|[alias=UUU]|
	\Sigma^{\wr 3} \wr A \ar{d}[swap]{\sigma^1} \ar{r} &
	\Sigma^{\wr 3} \wr \mathcal{V}^{op} \ar{d}[swap]{\sigma^1} \ar{r}{\otimes} &
	\Sigma^{\wr 2} \wr \mathcal{V}^{op} \ar{r}{\otimes} &
	|[alias=UUUU]|
	\Sigma \wr \mathcal{V}^{op} \ar{r}{\otimes} \ar[equal]{d} &
	\mathcal{V}^{op} \ar[equal]{d}
\\
	\Omega^1_{\mathfrak{C}} \wr A \ar{r} \ar{d}[swap]{d^0} &
	|[alias=VVV]|
	\Sigma \wr \Omega^0_{\mathfrak{C}} \wr A \ar{rr} & &
	|[alias=U]|
	\Sigma^{\wr 2} \wr A \ar{r} \ar{d}[swap]{\sigma^0} &
	|[alias=VVVV]|
	\Sigma^{\wr 2} \wr \mathcal{V}^{op} \ar{rr}{\otimes} \ar{d}[swap]{\sigma^0} & &
	\Sigma \wr \mathcal{V}^{op} \ar{r}{\otimes} &
	|[alias=UU]|
	\mathcal{V}^{op} \ar[equal]{d}
\\
	|[alias=V]|
	\Omega^0_{\mathfrak{C}} \wr A \ar{rrr} & & &
	\Sigma \wr A \ar{r} &
	|[alias=VV]|
	\Sigma \wr \mathcal{V}^{op} \ar{rrr}{\otimes} & & &
	\mathcal{V}^{op}
\arrow[Leftrightarrow, from=V, to=U,shorten >=0.15cm,shorten <=0.15cm
,swap,"\pi"
]
\arrow[Leftrightarrow, from=VV, to=UU,shorten >=0.15cm,shorten <=0.15cm
]
\arrow[Leftrightarrow, from=VVV, to=UUU,shorten >=0.15cm,shorten <=0.15cm
,swap,"\pi"
]
\arrow[Leftrightarrow, from=VVVV, to=UUUU,shorten >=0.15cm,shorten <=0.15cm
]
\end{tikzcd}
\]
That the leftmost sections of these diagrams match follows by 
parts (IT1) and (FF1) of Proposition \ref{CATDIAG2 PROP},
while the fact that the rightmost sections coincide follows since
$\mathcal{V}$ is a monoidal category.

The unitality of the monad $N$
follows by a simpler version of the argument above.
\end{proof}



\subsubsection{The non-equivariant monad on symmetric sequences}


We will now use the adjunction
\[
	\mathsf{Lan} \colon
	\mathsf{WSpan}^l(\Sigma_{\bullet}^{op},\mathcal{V}) 
\rightleftarrows
	\mathsf{Sym}(\mathcal{V})
	\colon \iota
\]
from Remark \ref{LANADJ REM} to induce a monad on 
$\mathsf{Sym}(\mathcal{V})$.
To do so, we will verify the conditions in \cite[Proposition 2.26]{BP_geo}, % MONADADJ PROP
stating that the natural transformations
\[
	\mathsf{Lan} \iota \xrightarrow{\epsilon} id
\qquad
	\mathsf{Lan} N \xrightarrow{\eta} \mathsf{Lan} N \iota \mathsf{Lan}
\]
are natural isomorphisms.

This is clear for $\epsilon$ while for $\eta$ it follows from the following two lemmas, the first of which is proven exactly as \cite[Lemma 2.21]{BP_geo}.


\begin{lemma}\label{FINWRPRODLIM LEM}
If in $\mathcal{V}$
the monoidal product %products
commutes with colimits in each variable, and the leftmost diagram
\begin{equation}\label{WRLAN EQ}
	\begin{tikzcd}[column sep = 4.5em]
	\mathcal{C}^{op} \ar{r}[swap,name=F]{}{G} \ar{d}[swap]{k^{op}} & 
	\mathcal{V} & 
	(\Sigma \wr \mathcal{C})^{op} \ar{d}[swap]{(\Fin_s \wr k)^{op}} 
	\ar{r}[swap,name=FF]{}{(\Sigma \wr G^{op})^{op}} & 
	(\Sigma \wr \mathcal{V}^{op})^{op} \ar{r}{\otimes} &
	\mathcal{V}
\\
	|[alias=D]|\mathcal{D}^{op} \ar{ru}[swap]{H} &
	& 
	|[alias=FD]|(\Sigma \wr \mathcal{D})^{op} 
	\ar{ru}[swap]{(\Sigma \wr H^{op})^{op}}
	\ar[bend right=13]{rru}[swap]{\otimes \circ (\Sigma \wr H^{op})^{op}}
	&
	\arrow[Leftarrow, from=D, to=F,shorten <=0.10cm,"\epsilon"]
	\arrow[Leftarrow, from=FD, to=FF,shorten <=0.10cm]
	\end{tikzcd}
\end{equation}
is a left Kan extension diagram then so is the composite of the rightmost diagram. 
\end{lemma}


The following is a variation of \cite[Lemma 4.28]{BP_geo} % LANPULLCOMA LEM
\begin{lemma}\label{LANPULLCOMA LEM}
	Suppose that $\mathcal{V}$ is complete. If the rightmost triangle in 
\[
\begin{tikzcd}
	\Omega_{\mathfrak{C}}^{0} \wr A \ar{r}{V} 
	\ar{d} & 
	\Sigma \wr A  
	\ar{d}  \ar{r}[swap,name=F]{}&
	\mathcal{V}^{op}
\\
	\Omega_{\mathfrak{C}}^{0} \ar{r}[swap]{V} & 
	|[alias=FEG]|\Sigma \wr \Sigma_{\mathfrak{C}} \ar{ru}
\arrow[Rightarrow, from=FEG, to=F,shorten <=0.15cm]
\end{tikzcd}
\]
is a right Kan extension diagram then so is the composite diagram.
\end{lemma}



\begin{proof}
Our proof will be a slightly more formalized version of the proof in \cite[Lemma 4.28]{BP_geo}.

Firstly, note that the composite
\begin{equation}\label{COMPISO EQ}
T \downarrow \Omega^0_{\mathfrak{C}} 
	\to 
\left( T_v \right)_{V(T)} \downarrow \Sigma \wr \Sigma_{\mathfrak{C}}
	\to
\left( T_v \right)_{V(T)} \downarrow_{\Sigma} \Sigma \wr \Sigma_{\mathfrak{C}}
\end{equation}
is an isomorphism. Indeed, the objects of 
$T \downarrow \Omega^0_{\mathfrak{C}}$
are isomorphisms $T \simeq T'$, which are entirely determined by 
a tuple of isomorphisms $\left(T_v \simeq T'_v\right)_{V(T)}$,
which are the objects of the target. 

We now claim that the maps
\begin{equation}\label{COMPISO2 EQ}
T \downarrow \Omega^0_{\mathfrak{C}} \wr A
	\to 
\left( T_v \right)_{V(T)} \downarrow \Sigma \wr A
	\to
\left( T_v \right)_{V(T)} \downarrow_{\Sigma} \Sigma \wr A
\end{equation}
are likewise isomorphisms.
Indeed, writing $G$ (resp. $\bar{G}$) for the composite functors in \eqref{COMPISO EQ}, (resp. \eqref{COMPISO2 EQ}), one has
\[
\bar{G}^{-1}
\left((a_v), (T_v) \xrightarrow{g} (f(a_v))\right)=
\left(\left(\left(\pi_{\Sigma} V G^{-1}(g)\right)^{-1}\right)^{\**} (a_v),
%G^{-1}\left((T_v \xrightarrow{\simeq} f(a_v))_{V(T)}\right),
T \xrightarrow{G^{-1}(g)} \bullet
\right).
\]
Now consider the diagram
\begin{equation}\label{COMPISO3 EQ}
T \downarrow \Omega^0_{\mathfrak{C}} \wr A
	\to 
\left( T_v \right)_{V(T)} \downarrow \Sigma \wr A
	\to
\left( T_v \right)_{V(T)} \downarrow_{\Sigma} \Sigma \wr A
	\to 
\left( T_v \right)_{V(T)} \downarrow \Sigma \wr A.
\end{equation}
The result will follow provided that the first map in \eqref{COMPISO3 EQ} is final. But this now follows since this map is isomorphic to the full composite of \eqref{COMPISO3 EQ}, which is final since \eqref{COMPISO2 EQ} is an isomorphism and the last map in \eqref{COMPISO3 EQ} is known to be final.
\end{proof}



\begin{definition}\label{COLORMON DEF}
The (colored) free operad monad is the monad
$\mathbb{F}$ on $\mathsf{Sym}(\mathcal{V})$
with underlying functor
$\mathbb{F} = \mathsf{Lan} N \iota$ with multiplication and unit given by
\[
	\mathsf{Lan} N \iota \mathsf{Lan} N \iota \xleftarrow{\simeq} 
	\mathsf{Lan} N N \iota \to 
	\mathsf{Lan} N \iota
\qquad
	id \xleftarrow{\simeq} 
	\mathsf{Lan} \iota \to
	\mathsf{Lan} N \iota.
\]
\end{definition}

\todo[inline]{draw the diagram again here? or reference it?}





\subsection{Pushouts of operads}
\label{PUSHOUT_SEC}


Our goal in this section will be to understand free operad extensions, i.e. pushouts of the form 
\begin{equation}\label{OU EQ}
            \begin{tikzcd}
                  \mathbb F X \arrow[d, "\mathbb{F}u"'] \arrow[r]
                  &
                  \O \arrow[d]
                  \\
                  \mathbb F Y \arrow[r]
                  &
                  \O[u].
            \end{tikzcd}
\end{equation}
where $u \colon X \to Y$ is a map of symmetric sequences.
Moreover, we will require that \eqref{OU EQ} is a fibered diagram over $\mathsf{F}$, i.e. that all maps therein are the identity on colors.
However, rather than fix the set of colors, 
we will find it convenient to consider all colors simultaneously, 
and note that our constructions are natural 
on \eqref{OU EQ} with respect to change of colors.
More explicitly, this will mean that our work in this section will be natural with regard to commutative diagrams
\begin{equation}\label{COLORCHNAT EQ}
	\begin{tikzcd}
		X \arrow[r, "u"',swap] \arrow[d]
	&
		Y \arrow[d]
&
		\mathbb F X \arrow[d] \arrow[r]
	&
		\O \arrow[d]
\\
		X' \arrow[r, "u'"']
	&
		Y'
&
		\mathbb F X' \arrow[r]
	&
		\O'
	\end{tikzcd}
\end{equation}
where all vertical maps induce the same map on objects.


To understand the pushouts \eqref{OU EQ},
we will produce a filtration
\begin{equation}\label{FILT EQ}
      \O = \O_0 \into \O_1 \into \O_2 \into \dots \into \colim_k \O_k = \O[u]
\end{equation}
of the underlying symmetric sequences, i.e. with 
$\mathcal{O}_i \in \mathsf{Sym}(\mathcal{V})$
(and all maps in \eqref{FILT EQ} will, again, be the identity on colors).


Writing $\amalg_{\mathsf{F}}$ and $\mathbin{\check\amalg}_{\mathsf{F}}$
for the fibered coproducts in 
$\mathsf{Sym}(\mathcal{V})$ and
$\mathsf{Op}(\mathcal{V})$
(i.e. these are the coproducts within each fiber over $\mathsf{F}$, rather than the coproducts in the overall categories)
the discussion in $(5.3)$ through $(5.7)$ of \cite{BP_geo}
yields that
\begin{align*}
  \O[u]
  &
    \simeq \mathrm{coeq}\left(
          \O \mathbin{\check\amalg_{\mathsf{F}}} \mathbb F X \mathbin{\check\amalg}_{\mathsf{F}} \mathbb F Y \rightrightarrows \O \mathbin{\check\amalg}_{\mathsf{F}} \mathbb F Y
          \right)
  \\
  &
    \simeq \colim_{[l] \in \Delta^{op}} 
    B_l \left( \O, \mathbb F X, \mathbb F X, \mathbb F X, \mathbb FY \right)
  \\
  &
    \simeq \colim_{[l] \in \Delta^{op},[n] \in \Delta^{op}} 
    B_l \left( \mathbb F^{\circ n+1} \O, \mathbb F X, \mathbb F X, \mathbb F X, \mathbb FY \right)
  \\
  &
    \simeq \colim_{[l] \in \Delta^{op},[n] \in \Delta^{op}} 
    \mathsf{Lan} N \circ \left( N^{\circ n} \iota \O \amalg_{\mathsf{F}} \iota X^{\amalg_{\mathsf{F}} 2l +1} \amalg_{\mathsf{F}} \iota Y \right),
    \stepcounter{equation}\tag{\theequation}\label{OU EQ1}
    % =: \colim_{n,l} \mathsf{Lan} \hat N_{n,l}^{(\O,X,Y)},
\end{align*}
where $B_{\bullet}$ denotes the \textit{double bar construction}
with respect to $\mathbin{\check\amalg}_{\mathsf{F}}$,
$\mathbb{F}^{\bullet +1} \mathcal{O}$ denotes the simplicial resolution of $\mathcal{O}$, 
and $N$ is the span monad in Definition \ref{NCOLOR DEF}.
Crucially, we note that colimits over $\Delta^{op}$
are computed by the reflexive coequalizer determined by levels $0$ and $1$, 
so that the colimits in \eqref{OU EQ1}
can be computed in $\mathsf{Sym}(\mathcal{V})$
rather than in $\mathsf{Op}(\mathcal{V})$.


By construction,
$N \left(N^{\circ n} \iota \O \amalg_{\mathsf{F}} \iota X^{\amalg_{\mathsf{F}} 2l +1}\amalg_{\mathsf{F}} \iota Y \right)$
denotes a certain span
$\Sigma_{\mathfrak{C}}^{op} \leftarrow 
\left(\Omega^{n,\lambda_l}_{\mathfrak{C}}\right)^{op} \to \mathcal{V}$,
and, after we properly identify $\Omega_{\mathfrak C}^{n, \lambda_l}$ in \S \ref{LCS_SEC},
we can then apply (the natural analogue of)
\cite[Prop. 5.37]{BP_geo}
along each simplicial direction
to convert the last line of \eqref{OU EQ1}
into a $\mathsf{Lan}$
over a single span
$\Sigma_{\mathfrak{C}}^{op} \leftarrow 
\left|\Omega^{n,\lambda_l}_{\mathfrak{C}}\right|^{op} \to \mathcal{V}$.

The task of describing 
$\Omega^{n,\lambda_l}_{\mathfrak{C}}$
is similar to the upshot of Proposition \ref{ASSOCIDS PROP}, which shows that
$N^{\circ n+1}$ is naturally calculated using the
$\Omega_{\mathfrak{C}}^{n} \wr (-)$ 
construction.

Indeed, we will do a little more. For $\lambda = \lambda_a \amalg \lambda_i$
a partition of $\set{1,2,\dots,l}$
we will write 
$N^{\times \lambda}$
for the monad (cf. \cite[\S 2.3]{BP_geo}) on 
$\left(\mathsf{WSpan}_l(\Sigma_{\bullet}^{op},\mathcal{V})\right)^{\times l}$
given by
\[
\left(N^{\times \lambda} (A_j)\right)_k = 
\begin{cases}
N(A_k) & \text{if } k\in \lambda_a
\\
A_k & \text{if } k\in \lambda_i
\end{cases}
\]
where we note that $N^{\times \lambda}$
preserves the fibered product
$\left(\mathsf{WSpan}_l(\Sigma_{\bullet}^{op},\mathcal{V})\right)^{\times_{\mathsf{F}} l}$,
i.e. the subcategory consisting of those tuples $(A_j)$ with the same objects in $\mathsf{F}$ (and similarly for maps), and we will slightly abuse notation by also writing 
$N^{\times \lambda}$
for the monad restricted to this subcategory.


\subsubsection{Labeled colored strings}
\label{LCS_SEC}

The categories $\Omega_{\mathfrak C}^{n,s,\lambda}$ in the following definition will then represent
the functors
$N^{\circ s+1} \circ \coprod_{\mathsf{F}} \circ \left(N^{\times \lambda}\right)^{\circ n-s}$.

 

\begin{definition}[{cf. \cite[Defn. 5.10]{BP_geo}}]\label{CLPS DEF}
      Given $-1 \leq s \leq n$, $l \geq 0$, and a partition $\lambda = \lambda_a \amalg \lambda_i$ of $\set{1,2,\dots,l}$,
      define $\Omega_{\mathfrak C}^{n,s,\lambda}$ to have as objects
$n$-planar strings
\begin{equation}
	T_{-1} \xrightarrow{\varphi_0} T_0 \xrightarrow{\varphi_1} T_1 \xrightarrow{\varphi_2} \dots
	T_{s} \xrightarrow{\varphi_{s+1}} T_{s+1} \xrightarrow{\varphi_{s+2}}  \dots
	\xrightarrow{\varphi_n} T_n
\end{equation}
together with $l$-labelings of $T_s, T_{s+1}, \cdots, T_n$
such that
$\varphi_{r}, r>s$ are $\lambda_i$-inert label maps.

Arrows in $\Omega_{\mathfrak C}^{n,s,\lambda}$
are tuples of isomorphisms $\left(\pi_r \colon T_r \to T'_r\right)$
such that $\pi_r,r \geq s$ are label maps.

Further, for any $s<0$ or $n<s'$, we write
\[
      \Omega_{\mathfrak C}^{n,\lambda} = \Omega_{\mathfrak C}^{n,0,\lambda},
      \qquad
\Omega_{\mathfrak{C}}^{n,s,\lambda} = \Omega_{\mathfrak{C}}^{n,-1,\lambda},
\qquad
\Omega_{\mathfrak{C}}^{n,s',\lambda} = \Omega_{\mathfrak{C}}^{n}.
\]
\end{definition}

We now discuss the functors relating the $\Omega_{\mathfrak{C}}^{n,s,\lambda}$ categories. Firstly, for 
$s \leq s'$ 
and map of labels $g \colon \{1,\cdots,l'\} \to \{1,\cdots,l\}$
such that $\lambda'_a \subseteq g^{-1}\left( \lambda_a\right)$
there are natural functors
\[
\Omega_{\mathfrak{C}}^{n,s,\lambda} \to \Omega_{\mathfrak{C}}^{n,s',\lambda},
\qquad
\Omega_{\mathfrak{C}}^{n,s,\lambda'} \xrightarrow{g_{\**}} \Omega_{\mathfrak{C}}^{n,s,\lambda}.
\]
Second, by keeping track of labels on vertices,
the usual functors from \S \ref{CSTRINGS_SEC} relating the categories 
$\Omega^n_{\mathfrak{C}}$ extend to the categories
$\Omega_{\mathfrak{C}}^{n,s,\lambda}$. Indeed, for 
$k \leq n$
and 
$f \colon \mathfrak{C} \to \mathfrak{D}$ a map of colors
one has functors
\begin{equation}\label{FGTLABEL EQ}
\Omega_{\mathfrak{C}}^{n,s,\lambda} \xrightarrow{\boldsymbol{V}^k} \Sigma \wr\Omega_{\mathfrak{C}}^{n-k-1,s-k-1,\lambda},
\qquad
\Omega_{\mathfrak{C}}^{n,s,\lambda} \xrightarrow{f_{\**}} \Omega_{\mathfrak{D}}^{n,s,\lambda}.
\end{equation}
Lastly, one also has simplicial operators $d_i$, $s_j$, 
but some care is needed with the way these interact with the index $s$. To do so, defining functions $d_i,s_j\colon \mathbb{Z} \to \mathbb{Z}$ by
\begin{equation}\label{SIMPLEXP EQ}
 d_i(s) = 
\begin{cases}
s-1, & i<s
\\
s, & s\leq i
\end{cases}
\qquad
s_j(s) = 
\begin{cases}
s+1, & j<s
\\
s, & s\leq j
\end{cases}
\end{equation}
one has simplicial operators
\[
\Omega_{\mathfrak{C}}^{n,s,\lambda} \xrightarrow{d_i} \Sigma \wr\Omega_{\mathfrak{C}}^{n,d_i(s),\lambda},
\qquad
\Omega_{\mathfrak{C}}^{n,s,\lambda} \xrightarrow{s_j} \Sigma \wr\Omega_{\mathfrak{C}}^{n,s_j(s),\lambda},
\]
for $0\leq i \leq n$ and $-1\leq j \leq n$.
In practice, we will prefer to suppress $s$ from the notation,
and write 
$\Omega_{\mathfrak{C}}^{n,\bullet,\lambda}$ to denote the string of categories 
$\Omega_{\mathfrak{C}}^{n,s,\lambda}$ as a whole.
Lastly, the $\pi_{i,k}$ natural isomorphisms for $i<k$ from Proposition \ref{CATDIAG PROP}
generalize to natural isomorphisms
\begin{equation}
\begin{tikzcd}[row sep = tiny, column sep = 35pt]
	\Omega_{\mathfrak{C}}^{n,s,\lambda}
	\arrow{r}{\boldsymbol{V}^k} \arrow{dd}[swap]{d^i} &
	|[alias=U]|
	 \Sigma \wr \Omega_{\mathfrak{C}}^{n-k-1,s-k-1,\lambda}
	 \ar[equal]{dd}{}
\\
\\
	|[alias=V]|
	\Omega_{\mathfrak{C}}^{n-1,d_i(s),\lambda} \arrow{r}[swap]{\boldsymbol{V}^{k-1}} &
	 \Sigma \wr \Omega_{\mathfrak{C}}^{n-k-1,d_i(s)-k,\lambda}
\arrow[Leftrightarrow, from=V, to=U,shorten >=0.15cm,shorten <=0.15cm
,swap,"\pi_{i,k}"
]
\end{tikzcd}
\end{equation}
(note that the right vertical map is an identity even if
$s-k-1 \neq d_i(s)-k$, since that can only occur if $s\leq i \leq k$, implying that the rightmost terms are both $\Sigma \wr \Omega_{\mathfrak{C}}^{n-k-1,-1,\lambda}$).

\begin{remark}
We now discuss the naturality of the given functors on the categories
$\Omega_{\mathfrak{C}}^{n,s,\lambda}$ just described.
\begin{enumerate}[label=(\roman*)]
\item by keeping track of vertex labels, all the analogues of the properties in Propositions \ref{CATDIAG PROP} and \ref{CATDIAG2 PROP} extend (note that this includes the pullback claims in 
Proposition \ref{CATDIAG PROP}(ii))
\item the change of color functors $f_{\**}$, change of label functors $g_{\**}$, and the forgetful functors in
\eqref{FGTLABEL EQ} are all natural with respect to each other.
\item $d_i$, $s_j$, $\boldsymbol{V}^k$
and $\pi_{i,k}$, are natural with respect to the change of label functors $g_{\**}$ and the forgetful functors in
\eqref{FGTLABEL EQ}, 
in the sense that they satisfy the analogues of the properties in 
Proposition \ref{CATDIAG PROP}(iii) with 
the role of $f_{\**}$ replaced with the latter functors.
\item
For $k \leq s \leq s'$ the following squares are pullback squares
\[
\begin{tikzcd}[column sep = small, row sep = small]
	\Omega^{n,s,\lambda}_{\mathfrak{C}} \ar{r}{\boldsymbol{V}^k} \ar{dd} &
	\Sigma \wr \Omega^{n-k-1,s-k-1,\lambda}_{\mathfrak{C}} \ar{dd}
\\
\\
	\Omega^{n,s',\lambda}_{\mathfrak{C}} \ar{r}[swap]{\boldsymbol{V}^k} &
	\Sigma \wr \Omega^{n-k-1,s'-k-1,\lambda}_{\mathfrak{C}}
\end{tikzcd}
\]
\end{enumerate}
\end{remark}

The following is the main purpose of the 
$\Omega_{\mathfrak{C}}^{n,s,\lambda}$ categories,
adapting the work in \S \ref{WRACONST SEC}.

\begin{definition}[{cf. \cite[Notation 5.24]{BP_geo}}]\label{NA_DEF}
      Given a $l$-tuple of functors
      $\left(A_j \to \Sigma_{\mathfrak C} \right)_{1\leq j \leq l}$,
we write
\begin{equation}\label{WRAJDEF EQ}
(-) \wr (A_j) \colon 
\mathsf{Cat} \downarrow^r_{\Sigma} \Sigma \wr \Sigma_{\mathfrak{C}}^{\amalg l}
\to
\mathsf{Cat} \downarrow^r_{\Sigma} \Sigma \wr \amalg_j A_j
\end{equation}
for the pullback \eqref{WSPANPULL EQ} for the map
$\Sigma \wr \amalg_j A_j \to \Sigma \wr \Sigma_{\mathfrak{C}}^{\amalg l}$.
 
In particular, for all $-1\leq s \leq n$, this defines categories
$\Omega^{n,s,\lambda}_{\mathfrak{C}} \wr (A_j)$ via pullbacks
(note that the $s \leq n$ restriction guarantees that the target of the lower $\boldsymbol{V}^n$ is indeed $\Sigma \wr \Sigma_{\mathfrak{C}}^{\amalg l}$).
\begin{equation}\label{WRAJSAMPLE EQ}
\begin{tikzcd}
	\Omega^{n,s,\lambda}_{\mathfrak{C}} \wr (A_j) \ar{r}{\boldsymbol{V}^n} \ar{d} &
	\Sigma \wr \amalg_j A_j  \ar{d}
\\
	\Omega^{n,s,\lambda}_{\mathfrak{C}} \ar{r}{\boldsymbol{V}^n} &
	\Sigma \wr \Sigma_{\mathfrak{C}}^{\amalg l}
\end{tikzcd}
\end{equation}
along with analogues of $d_i$ (for $i<n$), $s_j$, $\boldsymbol{V}^k$, $\pi_{i,k}$
and of the forgetful functors in \eqref{FGTLABEL EQ}
(cf. the discussion following \eqref{WRADEF EQ}).
\end{definition}
 

\begin{proposition}\label{SPANPIECEJ PROP}
A tuple of commutative squares
\begin{equation}\label{SPANPIECEJ EQ}
\begin{tikzcd}
	A_j \ar{d} \ar{r}{f} &  \ar{d} B_j
\\
	\Sigma_{\mathfrak{C}} \ar{r}[swap]{f} & \Sigma_{\mathfrak{D}}
\end{tikzcd}
\end{equation}
induces natural maps 
$f_{\**} \colon
\Omega_{\mathfrak{C}}^{n,\bullet,\lambda} \wr (A_j) \to 
\Omega_{\mathfrak{D}}^{n,\bullet,\lambda} \wr (B_j) $.

Similarly, a map of tuples $A_j \to B_{g(j)}$ for 
$g \colon \{1,\cdots,l\} \to \{1,\cdots,l'\}$
induces natural maps 
$g_{\**} \colon
\Omega_{\mathfrak{C}}^{n,\bullet,\lambda} \wr (A_j) \to 
\Omega_{\mathfrak{C}}^{n,\bullet,\lambda'} \wr (B_{j'}) $.

Moreover, both $f_{\**}$ and $g_{\**}$ satisfy the analogues of the commutativity properties in Proposition \ref{SPANPIECE PROP}.
In particular, the diagrams below commute.
\[
\begin{tikzcd}[column sep = 4pt, row sep = small]
	\Omega^{n,\bullet,\lambda}_{\mathfrak{C}} \wr (A_j)
	\ar{rrrrr}[name=toE]{\boldsymbol{V}^k} \ar{rd}[swap]{d^i} \ar{dd}[swap]{f_{\**}}
	&&&
	&&
	\Sigma \wr \Omega^{n-k-1,\bullet,\lambda}_{\mathfrak{C}} \wr (A_j) \ar{dd}{f_{\**}}
&
	\Omega^{n,\bullet,\lambda}_{\mathfrak{C}} \wr (A_j)
	\ar{rrrrr}[name=toE2]{\boldsymbol{V}^k} \ar{rd}[swap]{d^i} \ar{dd}[swap]{f_{\**}}
	&&&
	&&
	\Sigma \wr \Omega^{n-k-1,\bullet,\lambda}_{\mathfrak{C}} \wr (A_j) \ar{dd}{f_{\**}}
\\
	&
	|[alias=DBE]|
	\Omega^{n-1,\bullet,\lambda}_{\mathfrak{C}} \wr (A_j) \ar{rrrru}[swap]{\boldsymbol{V}^{k-1}}
	&&&&
&
	&
	|[alias=DBE2]|
	\Omega^{n-1,\bullet,\lambda}_{\mathfrak{C}} \wr (A_j) \ar{rrrru}[swap]{\boldsymbol{V}^{k-1}}
\\
	\Omega^{n,\bullet,\lambda}_{\mathfrak{D}} \wr (B_j) \ar{rrrrr}[name=toB]{\boldsymbol{V}^k} \ar{rd}[swap]{d^i}
	&&&
	&&
	\Sigma \wr \Omega^{n-k-1,\bullet,\lambda}_{\mathfrak{D}} \wr (B_j)
&
	\Omega^{n,\bullet,\lambda'}_{\mathfrak{C}} \wr (B_{j'}) \ar{rrrrr}[name=toB2]{\boldsymbol{V}^k} \ar{rd}[swap]{d^i}
	&&&
	&&
	\Sigma \wr \Omega^{n-k-1,\bullet,\lambda'}_{\mathfrak{C}} \wr (B_{j'})
\\
	&
	|[alias=D]| \Omega^{n-1,\bullet,\lambda}_{\mathfrak{D}} \wr (B_j) \ar{rrrru}[swap]{\boldsymbol{V}^{k-1}}
	&&&&
&
	&
	|[alias=D2]| \Omega^{n-1,\bullet,\lambda'}_{\mathfrak{C}} \wr (B_{j'}) \ar{rrrru}[swap]{\boldsymbol{V}^{k-1}}
\arrow[Leftrightarrow, from=DBE, to=toE, shorten <=0.15cm,shorten >=0.15cm
,swap,"\pi"
]
	\arrow[Leftrightarrow, from=D, to=toB, shorten <=0.15cm,shorten >=0.15cm,swap,"\pi"]
	\arrow[from=DBE, to=D, crossing over, near start, swap, "f_{\**}"]
\arrow[Leftrightarrow, from=DBE2, to=toE2, shorten <=0.15cm,shorten >=0.15cm
,swap,"\pi"
]
	\arrow[Leftrightarrow, from=D2, to=toB2, shorten <=0.15cm,shorten >=0.15cm,swap,"\pi"]
	\arrow[from=DBE2, to=D2, crossing over, near start, swap, "f_{\**}"]
\end{tikzcd}
\]
\end{proposition}

\begin{proof}
This follows by repeating the argument in the proof of Proposition \ref{SPANPIECE PROP}.
\end{proof}


 
Using the composite functors
$\Omega_{\mathfrak{C}}^{n,s,\lambda} \wr (A_j)
\to \Omega_{\mathfrak{C}}^{n,s,\lambda} 
\to \Omega^{-1,0}_{\mathfrak{C}} = \Sigma_{\mathfrak{C}}$,
we can regard the 
$\Omega_{\mathfrak{C}}^{n,s,\lambda} \wr (-)$
construction as a functor
$\left(\mathsf{Cat}\downarrow \Sigma_{\mathfrak{C}}\right)^{\times l}
\to \mathsf{Cat}\downarrow \Sigma_{\mathfrak{C}}$

\begin{corollary}[{cf. \cite[Cor. 5.32]{BP_geo}}]
      \label{LABIDEN_COR}
Let $-1 \leq k, -1 \leq s \leq n$.
There are natural identifications
\[
	\OC^k \wr \OC^{n,s,\lambda} \wr (A_j) \simeq
	\OC^{n+k+1,s+k+1,\lambda} \wr (A_j),
\qquad
	\OC^{n,s,\lambda} \wr (\OC^k)^{\times \lambda} \wr (A_j) \simeq
	\OC^{n+k+1,s,\lambda} \wr (A_j)	
\]
which are unital and associative in the natural ways.
Moreover, these induce identifications
\[
d^i \wr \Omega^{n,s,\lambda} \wr (A_j) \simeq d^i \wr (A_j)
	\quad
\pi_{i,k} \wr \Omega^{n,s,\lambda} \wr (A_j) \simeq \pi_{i,k} \wr (A_j)
	\quad
s^j \wr \Omega^{n,s,\lambda} \wr (A_j) \wr A \simeq d^j \wr  (A_j)
\]
\[
\Omega^k \wr (d^i) \wr (A_j) \simeq d^{k+i+1} \wr (A_j)
	\quad
\Omega^k \wr (s^j) \wr (A_j) \simeq s^{k+j+1} \wr (A_j)
\]
\end{corollary}


\begin{proof}
Much as in Proposition \ref{ASSOCIDS PROP}, this follows by noting that all squares in the following diagrams are pullback squares.
\begin{equation}\label{LSTRINGS EQ}
\begin{tikzcd}[column sep = 9pt]
		\OC^{n+k+1,s+k+1,\lambda} \arrow{r}{\boldsymbol{V}^k} \wr (A_j)
		\ar{r} \ar{d}
		%\bullet \arrow[dashed]{d} \arrow[dashed]{r}
	&
		\Sigma \wr \OC^{n,s,\lambda} \wr (A_j) \ar{r} \ar{d}
		%\bullet \arrow[dashed]{d} \arrow[dashed]{r}
	&
		\Sigma^{\wr 2} \wr \amalg_j A_j \arrow{r} \ar{d}
	&
		\Sigma \wr \amalg_j A_j \ar{d}
\\
		\OC^{n+k+1,s+k+1,\lambda} \arrow{r}{\boldsymbol{V}^k} \arrow[d]
	&
		\Sigma \wr \OC^{n,s,\lambda} \arrow{r}{\boldsymbol{V}^n} \arrow[d]
	&
		\Sigma^{\wr 2} \wr \Sigma_{\mathfrak C}^{\amalg l} \ar{r}
	&
		\Sigma \wr \Sigma_{\mathfrak C}^{\amalg l}
\\
		\OC^k \arrow{r}{\boldsymbol{V}^k}
	&
		\Sigma \wr \Sigma_{\mathfrak C}
\\ % NEW DIAGRAM ------------------------------
		\OC^{n+k+1,s,\lambda} \arrow{r}{\boldsymbol{V}^n} \wr (A_j)
		\ar{r} \ar{d}
		%\bullet \arrow[dashed]{d} \arrow[dashed]{r}
	&
		\Sigma \wr \amalg \left(\OC^{k}\right)^{\times \lambda} \wr (A_j)
		\ar{r} \ar{d}
		%\bullet \arrow[dashed]{d} \arrow[dashed]{r}
	&
		\Sigma \wr \amalg_j \Sigma \wr A_j \arrow{r} \ar{d}
	&
		\Sigma^{\wr 2} \wr \amalg_j A_j \arrow{r} \ar{d}
	&
		\Sigma \wr \amalg_j A_j \ar{d}
\\
		\OC^{n+k+1,s,\lambda} \arrow{r}{\boldsymbol{V}^n} \arrow[d]
	&
		\Sigma \wr \amalg \left(\OC^{k}\right)^{\times \lambda} \arrow{r}{\boldsymbol{V}^k} \arrow[d]
	&
		\Sigma \wr \amalg_l \Sigma \wr \Sigma_{\mathfrak{C}} \ar{r}
	&
		\Sigma^{\wr 2} \wr \amalg_l \Sigma_{\mathfrak{C}} \ar{r}
	&
		\Sigma \wr \Sigma_{\mathfrak C}^{\amalg l}
\\
		\OC^{n,s,\lambda} \arrow{r}{\boldsymbol{V}^n}
	&
		\Sigma \wr \Sigma_{\mathfrak C}^{\amalg l}
            \end{tikzcd}
      \end{equation}
\end{proof}

\subsubsection{Filtration of free extensions}
\label{EQMON_SEC}

We now return to discussing free extensions \eqref{OU EQ}.
%
Let $\lambda_l$ denote the partition on 
\[
\langle \langle l \rangle \rangle
=
\{-\infty,-l,\cdots,-1,0,1,\cdots,+\infty\}
\]
such that $\left(\lambda_l\right)_a = \{-\infty\}$,
and define $N_{n,l}^{(\O,X,Y)}$ to be the opposite of the composite
\[
      \Omega_{\mathfrak C}^{n,0,\lambda_l} \xrightarrow{(\boldsymbol{V}^0)^{\circ n}}
      \Sigma^{\wr n} \wr \coprod_{\langle \langle l \rangle \rangle} \Sigma_{\mathfrak C} \xrightarrow{(\O,X,\dots,X,Y)}
      \Sigma^{\wr n} \wr \V^{op} \xrightarrow{\otimes}
      \V^{op}.
\]
The upshot of \S \ref{LCS_SEC}, in particular Corollary \ref{LABIDEN_COR}, is that
$\mbox{$N \left(N^{\circ n} \iota \O \amalg_{\mathsf{F}} \iota X^{\amalg_{\mathsf{F}} 2l +1}\amalg_{\mathsf{F}} \iota Y \right)$}$
is the span
$\mbox{$\Sigma_{\mathfrak C}^{op} \xleftarrow{\mathsf{lr}} (\Omega_{\mathfrak C}^{n,0,\lambda_l})^{op} \xrightarrow{N_{n,l}^{(\O,X,Y)}} \V$}$,
and hence following \eqref{OU EQ1} we conclude that
\begin{equation}\label{1STRED EQ}
\O[u] \simeq
\mathop{\colim}\limits_{(\Delta \times \Delta)^{op}}
\left(
	\mathsf{Lan}_{\left(\Omega_{\mathfrak C}^{n,\lambda_l} \to \Sigma_{\mathfrak C}\right)^{op}} N_{n,l}^{(\O,X,Y)}
\right).
\end{equation}


% The preceeding discussion shows that
% ({\color{blue} $N_{n,l}^{(\O,X,Y)}$ notation undiscussed})
% \begin{equation}\label{1STRED EQ}
% \O[u] \simeq
% \mathop{\colim}\limits_{(\Delta \times \Delta)^{op}}
% \left(
% 	\mathsf{Lan}_{\left(\Omega_{\mathfrak C}^{n,\lambda_l} \to \Sigma_{\mathfrak C}\right)^{op}} N_{n,l}^{(\O,X,Y)}
% \right)
% \end{equation}
% where $\lambda_l$ is the partition on 
% \[
% \langle \langle l \rangle \rangle
% =
% \{-\infty,-l,\cdots,-1,0,1,\cdots,+\infty\}
% \]
% such that $\left(\lambda_l\right)_a = \{-\infty\}$.
Moreover, the simplicial operators in the $l$ direction are described by antisymmetric functions $\langle \langle l \rangle \rangle
 \to \langle \langle l' \rangle \rangle
$
which are given by \eqref{SIMPLEXP EQ} on non-negative values.


\begin{proposition}\label{EXTENTREE PROP}
The double simplicial realization
$|\Omega_{\mathfrak C}^{n,\lambda_l}|$,
which we call the \textit{extension tree category}
and denote
$\OC^e$, has as objects the 
$\{\O,X,Y\}$-labeled trees
and as arrows tall maps $\varphi \colon T \to S$ such that
\begin{enumerate}[label=(\roman*)]
\item if $T_v$ has a $X$-label, then 
$S_v \in \Sigma_{\mathfrak{C}}$ and
$S_v \in \Sigma_{\mathfrak{C}}$ has a $X$-label;
\item if $T_v$ has a $Y$-label, then 
$S_v \in \Sigma_{\mathfrak{C}}$ and
$S_v \in \Sigma_{\mathfrak{C}}$ has either a $X$-label or a $Y$-label;
\item if $T_v$ has a $\O$-label, then 
$S_v \in \Sigma_{\mathfrak{C}}$ has only $X$ and $\O$ label;
\end{enumerate}
\end{proposition}


\begin{proof}
This is a direct analogue of \cite[Prop. 5.41]{BP_geo}, and the proof therein carries through without significant changes, so we only sketch the key arguments.
Firstly, it is straightforward to check that for each fixed $l$,
the realization $|\Omega^{n,\lambda_l}|$
in the $n$ direction is the category 
$\Omega^{t,\lambda_l}_{\mathfrak{C}}$
with objects the $\langle \langle l \rangle \rangle$-labeled trees and and maps the tall label maps which are inert on colors other than $-\infty$/$\O$ (cf. \cite[Rem. 5.36]{BP_geo}).
Moreover, maps 
$T \to S$
in 
$\Omega^e_{\mathfrak{C}}$
canonically factor 
as
$T \to T' \to S$
where the first map is a a relabel map (i.e. an underlying isomorphism of trees that simply changes labels) and the second map is a label map, 
so that the result follows from the observation that relabel maps in 
$\Omega^e_{\mathfrak{C}}$
correspond to objects of  
$\Omega^{t,\lambda_1}$
while label maps correspond to maps of
$\Omega^{t,\lambda_0}$.
\end{proof}


We note that the proof of \cite[Prop 5.37]{BP_geo} % RANTRANS PROP
holds when replacing $\Omega_G$ with $\Omega_{\mathfrak C}$,
and we thus 
apply it
\eqref{1STRED EQ}
(note that the ``natural transformation component of differential operators are isomorphisms for the $n$ direction follows from \eqref{NMONMULTTR EQ} and \eqref{NMONIDTR EQ}''
while in the $l$ direction it follows since the associate maps of tuples (cf. Proposition \ref{SPANPIECEJ PROP}) are the identity in each coordinate)
to yield
\begin{equation}\label{2NDRED EQ}
\O[u] \simeq
	\mathsf{Lan}_{\left(\Omega_{\mathfrak C}^{e} \to
	\Sigma_{\mathfrak C}\right)^{op}} N^{(\O,X,Y)}
\end{equation}

The desired filtration \eqref{FILT EQ} will now be obtained by
first replacing $\Omega_{\mathfrak C}^e$ in \eqref{2NDRED EQ} with a suitable subcategory $\widehat{\Omega}_{\mathfrak C}^{e}$,
and then producing a filtration
$\widehat{\Omega}_{\mathfrak C}^{e}[\leq k]$
of 
$\widehat{\Omega}_{\mathfrak C}^{e}$ itself.

% We can explain how the desired filtration 
% \eqref{FILT EQ} is obtained:
% after replacing  
% $\Omega_{\mathfrak C}^{e}$
% in \eqref{2NDRED EQ}
% with a suitable subcategory 
% $\widehat{\Omega}_{\mathfrak C}^{e}$,
% the filtration will follow from a filtration 
% $\widehat{\Omega}_{\mathfrak C}^{e}[\leq k]$
% of 
% $\widehat{\Omega}_{\mathfrak C}^{e}$ itself.


\begin{definition}
      Let
$\widehat{\Omega}_{\mathfrak C}^{e} \hookrightarrow \Omega_{\mathfrak C}^{e}$
denote the full subcategory of those labeled trees whose underlying tree is alternating, active nodes are labeled by $\O$ 
and passive nodes are labeled by $X$ or $Y$.

Moreover, writing $|T| = |V^X(T)|+ |V^Y(T)|$, we let
\begin{enumerate}[label=(\roman*)]
\item $\widehat{\Omega}_{\mathfrak C}^{e}[\leq k]$ (resp. $\widehat{\Omega}_{\mathfrak C}^{e}[k]$)
denote the full subcategory of those $T$ with $|T| \leq k$ ($|T|=k$);
\item $\widehat{\Omega}_{\mathfrak C}^{e}[\leq k \setminus Y]$ (resp. $\widehat{\Omega}_{\mathfrak C}^{e}[k \setminus Y]$)
denote the further subcategory of those $T$ with $|T|_Y \neq k$.
\end{enumerate}
Subcategories $\OC^a[\leq k], \OC^a[k]$ of $\OC^a$ are defined similarly.
\end{definition}



The following results follow exactly as in the cited result from 
\cite{BP_geo} that they generalize.

\begin{lemma}[{cf. \cite[Cor. 5.53, Lemma 5.58]{BP_geo}}]
\label{LANINT LEM}

	$\widehat\Omega_{\mathfrak C}^e \into 
	\Omega_{\mathfrak C}^e$
	is $\Ran$-initial over $\SC$.
     
	Similarly, $\widehat\Omega_{\mathfrak C}^e[\leq k-1] \into 
\widehat\Omega_{\mathfrak C}^e[\leq k \setminus Y]$
	is $\Ran$-initial over $\SC$.
\end{lemma}

\begin{remark}[{cf. \cite[Remark 5.57]{BP_geo}}]
      \label{OEFIB REM}
      The following diagram
      \begin{equation}
            \begin{tikzcd}
                  \widehat\Omega_{\mathfrak C}^e[k \setminus Y] \arrow[rr, hookrightarrow] \arrow[dr]
                  &&
                  \widehat\Omega_{\mathfrak C}^e[k] \arrow[dl]
                  \\
                  &
                  \Omega_{\mathfrak C}^a[k]
            \end{tikzcd}
      \end{equation}
	is a map of Grothendieck fibrations
	such that fibers over $T \in \Omega_{\mathfrak C}^a[k]$ are the punctured cube and cube categories
      \begin{equation}
            (Y \to X)^{\times V^{in}(T)} - Y^{\times V^{in}(T)},
            \qquad
            (Y \to X)^{\times V^{in}(T)}.
      \end{equation}
	for $V^{in}(T)$ the set of inert vertices.
\end{remark}


We now finally describe the filtration \eqref{FILT EQ}.
\begin{definition}\label{FILTSTAGE DEF}
Let $\O_k$ denote the left Kan extension
\begin{equation}
\begin{tikzcd}
	|[alias = A]|
	\widehat\Omega_{\mathfrak C}^e[\leq k]^{op}
	\arrow[r, "\tilde N"] \arrow[d, "\mathsf{lr}"']
&
	\V
\\
	\SC^{op}
	\arrow[ur, "\O_k"', ""{name = B}]
	\arrow[Rightarrow, from = A, to = B]
\end{tikzcd}
\end{equation}
\end{definition}

Since $\widehat\Omega_{\mathfrak C}^e[\leq 0] \simeq \SC$
and the nerve of $\widehat \Omega_{\mathfrak C}^e$ is the union of the nerves of the $\widehat\Omega_{\mathfrak C}^e[\leq k]$
the desired filtration \eqref{FILT EQ} follows.


Moreover, to prove Theorem \ref{THM1_C}, we will need to
understand how each filtration stage is built from the previous one.
To do so, we consider the following diagram, where the left square is a pushout at the level of nerves (cf. \cite[(5.65)]{BP_geo}),
so that after taking $\mathsf{Lan}$
one obtains the right pushout
(that the top right corner is indeed $\O_{k-1}$ follows from Lemma \ref{LANINT LEM})
\begin{equation}\label{FILTLAN EQ}
\begin{tikzcd}
		\widehat\Omega_{\mathfrak C}^e[k \setminus Y] \arrow[r] \arrow[d]
	&
		\widehat\Omega_{\mathfrak C}^e[\leq k \setminus Y] \arrow[d]
&
		\mathsf{Lan}_{\widehat\Omega_{\mathfrak C}^e[k \setminus Y]^{op}} \tilde N \arrow[r] \arrow[d]
	&
		\O_{k-1} \arrow[d]
\\
		\widehat\Omega_{\mathfrak C}^e[k] \arrow[r]
	&
		\widehat\Omega_{\mathfrak C}^e[\leq k]
&
		\mathsf{Lan}_{\widehat \Omega_{\mathfrak C}^e[k]^{op}} \tilde N \arrow[r]
	&
		\O_k
\end{tikzcd}
\end{equation}


Lastly, Remark \ref{OEFIB REM} allows us to 
rewrite the left vertical map of can extensions in \eqref{FILTLAN EQ}


\begin{proposition}[{cf. \cite[Prop. 5.66]{BP_geo}}]
      \label{FILTPUSH PROP}
%      Suppose $\V$ is a closed monoidal category, and fix a $G$-set $\mathfrak C$.
%      For $n \geq 0$, let $\O_n$ denote the $n$-stage of the filtration \eqref{FILT_EQ} in $\Sym^{G, \mathfrak C}(\V)$
%      of the pushout map from \eqref{OU_EQ}.
For each $\mathfrak C$-signature $C \in \Sigma_{\mathfrak C}$, one has a pushout diagram
      \vspace{-10pt}
\begin{align}\label{FILTPUSH EQ}
\begin{tikzcd}[ampersand replacement=\&]
	\mathop{\coprod}\limits_{[T] \in \Iso(C \downarrow \Omega_{\mathfrak C}^a[k])}
	\left(
		\mathop{\bigotimes}\limits_{v \in V^{ac}(T)} \O(T_v) \otimes
		Q^{in}_T[u]
	\right) \cdot_{\Aut_{\OC} (T)} \Aut_{\SC}(C)
		\arrow[r] \arrow[d]
\&[15pt]
	\O_{k-1}(C) \arrow[d]
\\                  
	\mathop{\coprod}\limits_{[T] \in \Iso(C \downarrow \Omega_{\mathfrak C}^a[k])}
	\left(
		\mathop{\bigotimes}\limits_{v \in V^{ac}(T)} \O(T_v) \otimes
		\mathop{\bigotimes}\limits_{v \in V^{in}(T)} Y(T_v)
	\right) \cdot_{\Aut_{\OC}(T)} \Aut_{\SC}(C)
		\arrow[r]
\&
\O_k(C)
\end{tikzcd}
\end{align}
      where $Q^{in}_T[u]$ denotes the source of the pushout-product map
      \begin{equation}
            \mathop{\mathlarger{\mathlarger{\mathlarger{\square}}}}_{v \in V^{in}(T)} u(T_v): Q^{in}_T[u] \to \bigotimes_{v \in V^{in}(T)} Y(T_v).
      \end{equation}
\end{proposition}

\begin{proof}
Computing the left Kan extensions iteratively by first extending to
$G \ltimes \OC^a[k]^{op}$, 
Remark \ref{OEFIB REM} allows us to rewrite 
the leftmost map in \eqref{FILTLAN EQ} as
\begin{equation}\label{FILTLANFIN EQ}
	\mathsf{Lan}_{(\OC^a[k] \to \SC)^{op}}\left(
		\bigotimes_{v \in V^{ac}(T)}\O(T_v) \otimes
		\mathop{\mathlarger{\mathlarger{\mathlarger{\square}}}}\limits_{v \in V^{in}(T)} u(T_v)
		\right).
\end{equation}
	The description of the leftmost maps in \eqref{FILTPUSH EQ} follows since the undercategories
	$C \downarrow \OC^a[k]^{op}$ are groupoids.
\end{proof}



\subsubsection{Filtration of free extensions in the equivariant case}

As discussed in \S \ref{SYMC_SEC},
Proposition \ref{DIAGRAMFM_PROP} implies that $\Op^G(\V)$ is the category of fibered algebras for the monad $\mathbb F^G$ on $\Sym^G(\V)$,
with underlying functor the composite
\[
      \left(
            \Fun(\Sigma^{op}_{\bullet}, \mathcal V)
      \right)^G
      \xrightarrow{\iota^G}
      \left(
            \mathsf{WSpan}_l
            (\Sigma^{op}_{\bullet},\mathcal{V})
      \right)^G
      \xrightarrow{\Lan^G}
      \left(
            \mathsf{Fun}
            (\Sigma^{op}_{\bullet},\mathcal{V})
      \right)^G
\]
and given on each fiber by \eqref{FGC_EQ}.
      
For pushouts, when \eqref{OU EQ} is a $G$-equivariant diagram (note that the maps are still assumed the identity on colors),
naturality with regard to data as in \eqref{COLORCHNAT EQ}
shows that the left Kan extension in 
\eqref{2NDRED EQ} 
is also compatible with $G$-equivariance, i.e. it is in fact $\Lan^G$ as above.
% $
% \mathsf{Lan}^G \colon 
% \left(
% \mathsf{WSpan}_l
% (\Sigma^{op}_{\bullet},\mathcal{V})
% \right)^G
% \to
% \left(
% \mathsf{Fun}
% (\Sigma^{op}_{\bullet},\mathcal{V})
% \right)^G
% $.
Hence, by (the span version of) Lemma \ref{EQUIVFUNCONV LEM},
we further have the alternative formula
\begin{equation}\label{3RDRED EQ}
\O[u] \simeq
	\mathsf{Lan}_{\left(G^{op} \ltimes \Omega_{\mathfrak C}^{e} \to
	G^{op} \ltimes \Sigma_{\mathfrak C}\right)^{op}} N^{(\O,X,Y)}
\end{equation}
and, since the subcategories 
$\widehat{\Omega}_{\mathfrak{C}}^e$,
$\widehat{\Omega}_{\mathfrak{C}}^e[\leq k]$,
$\widehat{\Omega}_{\mathfrak{C}}^e[k]$
are compatible with the $G$-action, by replacing these categories with 
$G^{op} \ltimes \widehat{\Omega}_{\mathfrak{C}}^e$,
$G^{op} \ltimes \widehat{\Omega}_{\mathfrak{C}}^e[\leq k]$,
$G^{op} \ltimes \widehat{\Omega}_{\mathfrak{C}}^e[k]$
in Definition \ref{FILTSTAGE DEF}
we see that the right square in \eqref{FILTLAN EQ}
is naturally a square in 
$\mathsf{Sym}(\mathcal{V})^G$.

Hence, by accounting for equivariance
we further obtain the following analogue of 
Proposition \ref{FILTPUSH PROP}.




\begin{proposition}[{cf. \cite[Prop. 5.66]{BP_geo}}]
      \label{FILTPUSHG PROP}
%      Suppose $\V$ is a closed monoidal category, and fix a $G$-set $\mathfrak C$.
%      For $n \geq 0$, let $\O_n$ denote the $n$-stage of the filtration \eqref{FILT_EQ} in $\Sym^{G, \mathfrak C}(\V)$
%      of the pushout map from \eqref{OU_EQ}.
For each $\mathfrak C$-signature $C \in \Sigma_{\mathfrak C}$, one has a pushout diagram
      \vspace{-10pt}
\begin{align}\label{FILTPUSHG EQ}
\begin{tikzcd}[ampersand replacement=\&]
	\mathop{\coprod}\limits_{[T] \in \Iso(C \downarrow G^{op} \ltimes \Omega_{\mathfrak C}^a[k])}
	\left(
		\mathop{\bigotimes}\limits_{v \in V^{ac}(T)} \O(T_v) \otimes
		Q^{in}_T[u]
	\right) \cdot_{\Aut_{G^{op} \ltimes \OC}(T)} \Aut_{G^{op} \ltimes \SC}(C)
		\arrow[r] \arrow[d]
\&
	\O_{k-1}(C) \arrow[d]
\\                  
	\mathop{\coprod}\limits_{[T] \in \Iso(C \downarrow G^{op} \ltimes \Omega_{\mathfrak C}^a[k])}
	\left(
		\mathop{\bigotimes}\limits_{v \in V^{ac}(T)} \O(T_v) \otimes
		\mathop{\bigotimes}\limits_{v \in V^{in}(T)} Y(T_v)
	\right) \cdot_{\Aut_{G^{op} \ltimes \OC}(T)} \Aut_{G^{op} \ltimes \SC}(C)
		\arrow[r]
\&
	\O_k(C)
\end{tikzcd}
\end{align}
\end{proposition}



\begin{remark} 
In the equivariant setting one is free to use either \eqref{FILTPUSH EQ} or \eqref{FILTPUSHG EQ}.
In fact, the coproduct summands of the left maps in \eqref{FILTPUSH EQ} are interchanged by the $G$-action, 
and each coproduct summand in \eqref{FILTPUSHG EQ} is then the coproduct of a $G$-conjugacy class of summands of \eqref{FILTPUSH EQ}.
%
%
%The connection between the two formulas is given by Lemma \ref{REDUCELAN LEM},
%though some care is needed.
%Namely, \eqref{PUSHOPPRG EQ} generally features fewer coproduct summands but this is compensated by the inductions
%$(-) \cdot_{\mathsf{Aut}_{G \ltimes \Omega^a_{\mathfrak{C}}}(T)} \mathsf{Aut}_{G \ltimes \Sigma_{\mathfrak{C}}}(C)$,
%which produce more terms than the 
%$(-) \cdot_{\mathsf{Aut}_{\Omega^a_{\mathfrak{C}}}(T)} \mathsf{Aut}_{\Sigma_{\mathfrak{C}}}(C)$
%inductions.
\end{remark}




\begin{remark}
Using the identification
$
\left(
\mathsf{WSpan}_l
\left(\Sigma_{\bullet}^{op},\mathcal{V}\right)
\right)^G
\simeq
\mathsf{WSpan}_l
\left(G \ltimes \Sigma_{\bullet}^{op},\mathcal{V}\right)
$
from (the analogue of) Lemma \ref{EQUIVFUNCONV LEM},
one has that the equivariant monad $N^G$
is described by the following analogue of \eqref{NCOLOR EQ}
(see the discussion in \eqref{RHOPURP EQ} for the discussion of the natural transformation
$G^{op} \ltimes \Sigma \wr (-) \to \Sigma \wr G^{op} \ltimes (-)$)
\begin{equation}\label{NCOLORG EQ}
\begin{tikzcd}
	G^{op} \ltimes \Omega^0_{\mathfrak{C}} \wr A \ar{r}{\boldsymbol{V}^0} \ar{d} &
	G^{op} \ltimes \Sigma \wr A  \ar{d} \ar{r} &
	\Sigma \wr G^{op} \ltimes A  \ar{d} \ar{r} &
	\Sigma \wr \mathcal{V}^{op} \ar{r}{\otimes} &
	\mathcal{V}^{op}
\\
	G^{op} \ltimes \Omega^0_{\mathfrak{C}} \ar{r}{\boldsymbol{V}^0} \ar{d} &
	G^{op} \ltimes \Sigma \wr \Sigma_{\mathfrak{C}} \ar{r} &
	\Sigma \wr G^{op} \ltimes \Sigma_{\mathfrak{C}} 
\\
	G^{op} \ltimes \Sigma_{\mathfrak{C}}
\end{tikzcd}
\end{equation}
This diagram suggests an alternative way to handle the equivariant case which is a little closer in spirit to the genuine equivariant operad work in \cite{BP_geo}.
Rather than using the naturality of filtrations with respect to change of color data as in \eqref{COLORCHNAT EQ},
one could instead regard the composites
$G^{op} \ltimes \OC^n \to 
G^{op} \ltimes \Sigma \wr \OC^{n-k-1} \to
\Sigma \wr G^{op} \ltimes  \OC^{n-k-1}$
as the formal equivariant analogues of the vertex functors.
It is then not hard to verify that all claims in Propositions \ref{CATDIAG PROP} and \ref{CATDIAG2 PROP},
so that all work in \S \ref{NONEQMON SEC} 
and ({\color{blue} fill in section reference})
generalizes by replacing occurrences 
of symbols like $\OC^{n}$ and $A$ with 
$G^{op} \ltimes \OC^{n}$ and $G^{op} \ltimes A$
(we note that, in particular,
in mimicking \eqref{WRADEF EQ}
this leads us to define categories 
$\left(G^{op} \ltimes \OC^{n}\right)
\wr
\left(G^{op} \ltimes A\right)$,
which are canonically isomorphic to the categories
$G^{op} \ltimes \OC^{n}
\wr A$ as in \eqref{NCOLORG EQ}).
\end{remark}


{\color{blue} HERE}






\subsection{Injective change of color}

In this subsection, we provide a description of the left adjoint $\check f_! \colon \Op^{\mathfrak D} \to \Op^{\mathfrak C}$
from Remark \ref{OP_MAP REM}
for an \textit{inclusion} $f \colon \mathfrak C \to \mathfrak D$ of colors,
and explore the interaction of the change of color functors $f_!$,$f^{\**}$ with free operads and pushouts.



\subsubsection{Free operads and injective change of color}
\label{FREEOPCOL_SEC}

Let $f \colon \mathfrak{C} \to \mathfrak{D}$ be an inclusion of colors.
Our goal in this section is to show that the pullback of a free $\mathfrak{D}$-colored operad is a free $\mathfrak{C}$-colored operad, in a natural way.
As consequence, we will show that the left adjoint $\check f_!$ is underlying.

More precisely, our goal is to identify a functor 
$\widehat{f^{\**}}$ such that the diagram below commutes (up to natural isomorphism).
\begin{equation}\label{HATFST EQ}
      \begin{tikzcd}
            \mathsf{Sym}^{\mathfrak{D}} \ar{r}{\mathbb{F}_{\mathfrak{D}}} \ar[dashed]{d}[swap]{\widehat{f^{\**}}} &
            \mathsf{Op}^{\mathfrak{D}} \ar{d}{f^{\**}}
            \\
            \mathsf{Sym}^{\mathfrak{C}} \ar{r}[swap]{\mathbb{F}_{\mathfrak{C}}} &
            \mathsf{Op}^{\mathfrak{C}}
      \end{tikzcd}
\end{equation}

We now introduce some notation.
We will write 
$\Omega^0_{\mathfrak{C},\mathfrak{D}} \subseteq \Omega^0_{\mathfrak{D}}$
for the full subcategory of those trees whose root and leaves are labeled by $\mathfrak{C}$.
One then has that in the diagram below the square is a pullback square.
\begin{equation}\label{PULLFREEFREE EQ}
\begin{tikzcd}
	\Omega^0_{\mathfrak{C},\mathfrak{D}} \ar{r}{} \ar{d}{} &
	\Omega^0_{\mathfrak{D}} \ar{d}{} \ar{r} &
	\Sigma \wr \Sigma_{\mathfrak{D}} \ar{r} &
	\Sigma \wr \mathcal{V}^{op} \ar{r} &
	\mathcal{V}^{op}
\\
	\Sigma_{\mathfrak{C}} \ar{r}[swap]{} &
	\Sigma_{\mathfrak{D}}
\end{tikzcd}
\end{equation}
Moreover, since $\Sigma_{\mathfrak{C}} \to \Sigma_{\mathfrak{D}}$ is an inclusion of components, it follows that performing a Kan extension to $\Sigma_{\mathfrak{D}}$ and then restricting to $\Sigma_{\mathfrak{C}}$
is equivalent to performing the Kan extension of the top composite.




We next note that there is a natural map 
$\Omega^0_{\mathfrak{C},\mathfrak{D}} \to \Omega^0_{\mathfrak{C}}$,
defined by $T \mapsto \partial_{\mathfrak{D} \setminus \mathfrak{C}}(T)$,
collapsing those inner edges not in $\mathfrak{C}$, and we write
$\Omega^{0,\partial}_{\mathfrak{C},\mathfrak{D}}
\subseteq 
\Omega^{0}_{\mathfrak{C},\mathfrak{D}}
$ for the full subcategory mapping to $\Sigma_{\mathfrak{C}}$ under $\partial_{\mathfrak{D} \setminus \mathfrak{C}}$ (more explicitly, elements of $\Omega^{0,\partial}_{\mathfrak{C},\mathfrak{D}}$ are non-stick trees with root and leaves in $\mathfrak{C}$ and inner edges in $\mathfrak{D} \setminus \mathfrak{C}$).
Moreover, noting that for the canonical maps 
$\partial_{\mathfrak{D} \setminus \mathfrak{C}}(T) \to T$
the subtrees $T_v$ for $v \in V(\partial_{\mathfrak{D} \setminus \mathfrak{C}}(T))$
must be in $\Omega^{0,\partial}_{\mathfrak{D}, \mathfrak{C}}$
one obtains a vertex functor 
$V \colon \Omega_{\mathfrak{C},\mathfrak{D}}^{0}
\to \Sigma \wr \Omega_{\mathfrak{C},\mathfrak{D}}^{0,\partial}$.
One then has a diagram
\begin{equation}\label{PULLFREEFREE2 EQ}
\begin{tikzcd}
	\Omega^0_{\mathfrak{C},\mathfrak{D}} \ar{r}{} \ar{d}[swap]{\partial_{\mathfrak{D} \setminus \mathfrak{C}}} &
	\Sigma \wr \Omega^{0,\partial}_{\mathfrak{C},\mathfrak{D}} \ar{d}[swap]{\partial_{\mathfrak{D} \setminus \mathfrak{C}}} \ar{r} &
	\Sigma \wr \Sigma \wr \Sigma_{\mathfrak{D}} \ar{r} &
	\Sigma \wr \Sigma_{\mathfrak{D}} \ar{r} &
	\Sigma \wr \mathcal{V}^{op} \ar{r} &
	\mathcal{V}^{op}
\\
	\Omega_{\mathfrak{C}} \ar{r}[swap]{} \ar{d}[swap]{\mathsf{lr}} &
	\Sigma \wr \Sigma_{\mathfrak{C}}
\\
	\Sigma_{\mathfrak{C}}
\end{tikzcd}
\end{equation}
where the top composite is naturally isomorphic to the top composite in 
\eqref{PULLFREEFREE EQ}, and whose Kan extension can alternatively be computed by first computing the Kan extension
\begin{equation}\label{HATFSTDEF EQ}
\begin{tikzcd}
	\Omega_{\mathfrak{C},\mathfrak{D}}^{0,\partial} \ar{r} \ar{d} &
	\Sigma \wr \Sigma_{\mathfrak{D}} \ar{r}{X} & 
	\Sigma \wr \mathcal{V}^{op} \ar{r} & 
	\mathcal{V}^{op}
\\
	\Sigma_{\mathfrak{C}} \ar[dashed]{rrru}[swap]{\widehat{f^{\**}}(X)}
\end{tikzcd}
\end{equation}
and then performing the associated Kan extension along 
$\Omega_{\mathfrak{C}} \to \Sigma_{\mathfrak{C}}$
in \eqref{PULLFREEFREE2 EQ}.

\begin{definition}
      \label{HATFST_DEF}
      Given $X \in \Sym^{\mathfrak D}$, define $\widehat{f^{\**}}(X)$ to be the left Kan extension in \eqref{HATFSTDEF EQ}.
\end{definition}

Put together, these observations show the desired claim that
$
f^{\**} \mathbb{F}_{\mathfrak{D}}(X) \simeq
\mathbb{F}_{\mathfrak{C}} \widehat{f^{\**}}(X)
$
as symmetric sequences.

Checking that this isomorphism agrees with the operad structures requires some extra work, which we now briefly sketch.
Firstly, to simplify the discussion concerning Kan extensions,
it is preferable to first prove the analogous result for spans, i.e. that there is 
$\widehat{f^{\**}_s}$ as in the following analogue of 
\eqref{HATFST EQ}.
\begin{equation}\label{HATFSTSP EQ}
\begin{tikzcd}
	\mathsf{WSpan}(\Sigma_{\mathfrak{D}},\mathcal{V}^{op})
	\ar{r}{N_{\mathfrak{D}}} \ar[dashed]{d}[swap]{\widehat{f^{\**}_s}} &
	\mathsf{Alg}_{N_{\mathfrak{D}}}
	\ar{d}{f^{\**}_s}
\\
	\mathsf{WSpan}(\Sigma_{\mathfrak{C}},\mathcal{V}^{op})
	\ar{r}{N_{\mathfrak{C}}} &
	\mathsf{Alg}_{N_{\mathfrak{C}}}
\end{tikzcd}
\end{equation}
Firstly, as a natural adaptation of 
\eqref{HATFSTDEF EQ}
one defines $\widehat{f^{\**}_s}$
on the span $\Sigma_{\mathfrak{D}} \rightarrow A \leftarrow \mathcal{V}^{op}$
to be given by 
\[
\begin{tikzcd}
	\Omega_{\mathfrak{C},\mathfrak{D}}^{0,\partial} \wr A
	\ar{r} \ar{d} &
	\Sigma \wr A \ar{r} \ar{d} &
	\Sigma \wr \mathcal{V}^{op} \ar{r} &
	\mathcal{V}^{op}
\\
	\Omega_{\mathfrak{C},\mathfrak{D}}^{0,\partial}
	\ar{r} \ar{d} &
	\Sigma \wr \Sigma_{\mathfrak{D}}
\\
	\Sigma_{\mathfrak{C}}
\end{tikzcd}
\]
where the square defining 
$\Omega^{0,\partial}_{\mathfrak{C},\mathfrak{D}} \wr A$ is a pullback. 

Assuming for the moment that \eqref{HATFSTSP EQ} indeed commutes up to natural isomorphism (i.e. that compatibility with the monad structures as been verified), one then has the string of isomorphisms
\[
f^{\**} \circ \mathbb{F}_{\mathfrak{D}} = 
f^{\**} \circ \mathsf{Lan} \circ N_{\mathfrak{D}} \circ \iota \xleftarrow{\simeq}
\mathsf{Lan} \circ f^{\**}_s \circ N_{\mathfrak{D}} \circ \iota \simeq
\mathsf{Lan} \circ N_{\mathfrak{C}} \circ \widehat{f^{\**}_s} \circ \iota
\xleftarrow{\simeq}
\mathsf{Lan} \circ N_{\mathfrak{C}} \circ \iota \circ \mathsf{Lan}
\circ \widehat{f^{\**}} \circ \iota
=
\mathbb{F}_{\mathfrak{C}} \circ \mathsf{Lan}
\circ \widehat{f^{\**}_s} \circ \iota
=
\mathbb{F}_{\mathfrak{C}}
\circ \widehat{f^{\**}}
\]
where the second step follows since 
$\Sigma_{\mathfrak{C}} \to \Sigma_{\mathfrak{D}}$ is an inclusion of components and the fourth step follows
from Lemmas \ref{FINWRPRODLIM LEM} and \ref{LANPULLCOMA LEM}.

We now tackle the remaining claim of showing that
\eqref{HATFSTSP EQ} commutes up to isomorphism.

We first need to discuss higher string generalizations of the tree categories 
$\Omega^{0}_{\mathfrak{C},\mathfrak{D}}$
and
$\Omega^{0,\partial}_{\mathfrak{C},\mathfrak{D}}$.

Namely, we define full subcategories 
$\Omega^{n,\partial}_{\mathfrak{C},\mathfrak{D}}
\subseteq
\Omega^{n}_{\mathfrak{C},\mathfrak{D}}
\subseteq
\Omega^{n}_{\mathfrak{D}}
$
where the objects of 
$\Omega^{n}_{\mathfrak{C},\mathfrak{D}}$
are strings $T_0 \to T_1 \to \cdots \to T_n$
of maps
such that $T_k \in \Omega_{\mathfrak{C}}$ for $k<n$
and the objects of 
$\Omega^{n,\partial}_{\mathfrak{C},\mathfrak{D}}$
satisfy the further restriction that
$T_{n-1}=\partial_{\mathfrak{D} \setminus \mathfrak{C}}(T_n)$.


Note that the vertex and simplicial operators then restrict to maps
\[
	\Omega_{\mathfrak{C},\mathfrak{D}}^{n, \partial} 
	\xrightarrow{V}
	\Sigma \wr \Omega_{\mathfrak{C},\mathfrak{D}}^{n-1, \partial} 
\quad
	n \geq 1
\qquad \qquad
	\Omega_{\mathfrak{C},\mathfrak{D}}^{n, \partial} 
	\xrightarrow{d_i}
	\Omega_{\mathfrak{C},\mathfrak{D}}^{n-1, \partial}
\quad
	n-2 \geq i \geq 0
\qquad \qquad
	\Omega_{\mathfrak{C},\mathfrak{D}}^{n, \partial} 
	\xrightarrow{d_{n-1},\simeq}
	\Omega_{\mathfrak{C},\mathfrak{D}}^{n-1}
\]
where we note that the last map is an isomorphism. 
Indeed, the $\Omega^0_{\mathfrak{C},\mathfrak{D}}$ 
on the top left corner of \eqref{PULLFREEFREE2 EQ} is best viewed as
being $\Omega^{1,\partial}_{\mathfrak{C},\mathfrak{D}}$, so that the leftmost horizontal map is the regular $V$ and the vertical maps are $d_1,d_0$.

The required compatibility with the operadic structure now follows by noting that the composite
$N_{\mathfrak{C}} N_{\mathfrak{C}} \widehat{f^{\**}_s}
\simeq
N_{\mathfrak{C}} f^{\**}_s N_{\mathfrak{D}}
\to
f^{\**}_s N_{\mathfrak{D}}
$
is encoded by the diagram
\[
\begin{tikzcd}[column sep =8pt]
	\Omega^{2,\partial}_{\mathfrak{C},\mathfrak{D}} \wr A \ar{d}[swap]{d^1}{\simeq}  \ar{r} &
	\Sigma \wr \Omega^{1,\partial}_{\mathfrak{C},\mathfrak{D}} \wr A \ar{d}[swap]{d^0}{\simeq} \ar{r} &
	\Sigma^{\wr 2} \wr \Omega^{0,\partial}_{\mathfrak{C},\mathfrak{D}} \wr A \ar{r} &
	|[alias=UUU]|
	\Sigma^{\wr 3} \wr A \ar{d}[swap]{\sigma^1} \ar{r} &
	\Sigma^{\wr 3} \wr \mathcal{V}^{op} \ar{d}[swap]{\sigma^1} \ar{r}{\otimes} &
	\Sigma^{\wr 2} \wr \mathcal{V}^{op} \ar{r}{\otimes} &
	|[alias=UUUU]|
	\Sigma \wr \mathcal{V}^{op} \ar{r}{\otimes} \ar[equal]{d} &
	\mathcal{V}^{op} \ar[equal]{d}
\\
	\Omega^1_{\mathfrak{C},\mathfrak{D}} \wr A \ar{r} \ar{d}[swap]{d^0} &
	|[alias=VVV]|
	\Sigma \wr \Omega^0_{\mathfrak{C},\mathfrak{D}} \wr A \ar{rr} & &
	|[alias=U]|
	\Sigma^{\wr 2} \wr A \ar{r} \ar{d}[swap]{\sigma^0} &
	|[alias=VVVV]|
	\Sigma^{\wr 2} \wr \mathcal{V}^{op} \ar{rr}{\otimes} \ar{d}[swap]{\sigma^0} & &
	\Sigma \wr \mathcal{V}^{op} \ar{r}{\otimes} &
	|[alias=UU]|
	\mathcal{V}^{op} \ar[equal]{d}
\\
	|[alias=V]|
	\Omega^0_{\mathfrak{C},\mathfrak{D}} \wr A \ar{rrr} & & &
	\Sigma \wr A \ar{r} &
	|[alias=VV]|
	\Sigma \wr \mathcal{V}^{op} \ar{rrr}{\otimes} & & &
	\mathcal{V}^{op}
\arrow[Leftrightarrow, from=V, to=U,shorten >=0.15cm,shorten <=0.15cm
,swap,"\pi"
]
\arrow[Leftrightarrow, from=VV, to=UU,shorten >=0.15cm,shorten <=0.15cm
]
\arrow[Leftrightarrow, from=VVV, to=UUU,shorten >=0.15cm,shorten <=0.15cm
,swap,"\pi"
]
\arrow[Leftrightarrow, from=VVVV, to=UUUU,shorten >=0.15cm,shorten <=0.15cm
]
\end{tikzcd}
\]
while the composite
$N_{\mathfrak{C}} N_{\mathfrak{C}} \widehat{f^{\**}_s}
\to
N_{\mathfrak{C}} \widehat{f^{\**}_s}
\simeq
f^{\**}_s N_{\mathfrak{D}}
$
is encoded by the diagram
\[
\begin{tikzcd}[column sep =8pt]
	\Omega^{2,\partial}_{\mathfrak{C},\mathfrak{D}} \wr A \ar{r} \ar{d}[swap]{d^0} &
	\Sigma \wr \Omega^{1,\partial}_{\mathfrak{C},\mathfrak{D}} \wr A \ar{r} &
	|[alias=UUU]|
	\Sigma^{\wr 2} \wr \Omega^{0,\partial}_{\mathfrak{C},\mathfrak{D}} \wr A
	\ar{d}[swap]{\sigma^0} \ar{r} &
	\Sigma^{\wr 3} \wr A \ar{d}[swap]{\sigma^0} \ar{r} &
	\Sigma^{\wr 3} \wr \mathcal{V}^{op} \ar{d}[swap]{\sigma^0} \ar{r}{\otimes} &
	\Sigma^{\wr 2} \wr \mathcal{V}^{op} \ar{d}[swap]{\sigma^0} \ar{r}{\otimes} &
	\Sigma \wr \mathcal{V}^{op} \ar{r}{\otimes} & 
	|[alias=UUUU]|
	\mathcal{V}^{op} \ar[equal]{d}
\\
	|[alias=VVV]|
	\Omega^{1,\partial}_{\mathfrak{C},\mathfrak{D}} \wr A \ar{rr} \ar{d}[swap]{d^0}{\simeq} & &
	\Sigma \wr \Omega^{0,\delta}_{\mathfrak{C},\mathfrak{D}} \wr A \ar{r} &
	|[alias=U]|
	\Sigma^{\wr 2} \wr A \ar{r} \ar{d}[swap]{\sigma^0} &
	\Sigma^{\wr 2} \wr \mathcal{V}^{op} \ar{r}{\otimes} \ar{d}[swap]{\sigma^0} &
	|[alias=VVVV]|
	\Sigma \wr \mathcal{V}^{op} \ar{rr}{\otimes} & &
	|[alias=UU]|
	\mathcal{V}^{op} \ar[equal]{d}
\\
	|[alias=V]|
	\Omega^0_{\mathfrak{C}} \wr A \ar{rrr} & & &
	\Sigma \wr A \ar{r} &
	|[alias=VV]|
	\Sigma \wr \mathcal{V}^{op} \ar{rrr}{\otimes} & & &
	\mathcal{V}^{op}
\arrow[Leftrightarrow, from=V, to=U,shorten >=0.15cm,shorten <=0.15cm
,swap,"\pi"
]
\arrow[Leftrightarrow, from=VV, to=UU,shorten >=0.15cm,shorten <=0.15cm
]
\arrow[Leftrightarrow, from=VVV, to=UUU,shorten >=0.15cm,shorten <=0.15cm
,swap,"\pi"
]
\arrow[Leftrightarrow, from=VVVV, to=UUUU,shorten >=0.15cm,shorten <=0.15cm
]
\end{tikzcd}
\]


This finishes the argument establishing \eqref{HATFST EQ}.

We explore several consequences.
\begin{corollary}
      \label{FTF_COR}
      For any injective $f \colon \mathfrak C \to \mathfrak D$, the following hold.
      \begin{enumerate}[label = (\roman*)]
      \item $f^{\**} \mathbb F_{\mathfrak D} f_! \simeq \mathbb F_{\mathfrak C}$.
      \item $f^{\**} f_! = id_{\Sym_{\mathfrak C}}$ and $f^{\**} \check{f_!} = id_{\Op_{\mathfrak C}}$.
      \end{enumerate}
\end{corollary}
\begin{proof}
      For (i), it suffices to show that $\widehat{f^{\**}} f_!$ is the identity,
      which is straightforward from the definition given in \eqref{HATFSTDEF EQ}.
      % $\widehat{f^{\**}} f_! X(\ksi) = \varnothing$ unless $\ksi \in \Sigma_{\mathfrak C}$, in which case
      % $\widehat{f^{\**}} f_! X(\ksi) = X(\ksi)$.
      For (ii), the first half is immediate,
      and using \eqref{CFS_EQ} plus the fact that $f^{\**}$ commutes with reflexive coequalizers\footnote{
        As with $\mathbb F$, this follows as $\otimes$ commuting with colimits in both variables implies it commutes with reflexive coequalizers; see e.g. \cite[Lemma 2.3.2]{Rez96}.},
      the second half follows from part (i).
\end{proof}

\begin{corollary}
      \label{FSU_COR}
      For any injective $f \colon \mathfrak C \to \mathfrak D$, $U \check{f_!} = f_! U$.
\end{corollary}
\begin{proof}
      Using Corollary \ref{FTF_COR}, it is straightforward from the definitions that we have natural isomorphisms
      \[
            \mathbb F_{\mathfrak D} f_! U \mathbb F_{\mathfrak C} \xrightarrow{\simeq} % \mathbb F_{\mathfrak D} f_! U(\simeq)
            \mathbb F_{\mathfrak D} f_! f^{\**} U \mathbb F_{\mathfrak D} f_! \xrightarrow{\simeq} % \mathbb F_{\mathfrak D}(\simeq)
            \mathbb F_{\mathfrak D} U \mathbb F_{\mathfrak D} f_!,
      \]
      so in particular for any $\O \in \Op^{\mathfrak C}$
      % there exists a unique $\mathfrak D$-colored operad structure on $f_! X$
      % such that the canonical map $X \to f_! X$ is a map of operads.
      there is an induced $\mathfrak D$-colored operad structure on $f_! X$,
      which is indeed described by the coequalizer \eqref{CFS_EQ}.
\end{proof}


{\color{OliveGreen} % --------------------------------------------------------------------------------
  \begin{corollary}
        For any $(G, \Sigma)$-family $\F$ and injective map of colors $f \colon \mathfrak C \to \mathfrak D$,
        the map $f^{\**}: \Op^{\mathfrak D}_\F(\V) \to \Op^{\mathfrak C}_\F(\V)$ preserves (trivial) cofibrations between cofibrant objects.
  \end{corollary}
  \todo[inline]{the statement is true as written, however the machinery to prove it entirely has not been developed.}
  \begin{proof}
        We first show $f^{\**}$ preserves generating (trivial) cofibrations, as in Proposition \ref{GCOF_SYMMS_PROP}.
        To start, let $X$ be a free symmetric sequence (cf. \eqref{FREESIGMA_EQ})
        \[
              X = \Sigma_{\mathfrak D}(-, \ksi)[K] = \Hom_{\Sigma_{\mathfrak D}}(-,\ksi) \cdot_{\Aut(\ksi)} K
        \]
        on some $\mathfrak D$-signature $\ksi$ and $K \in \V^{\Aut(\ksi)}$.
        % From the definition of $\Omega^{0,\partial}_{\mathfrak C, \mathfrak D}$ and $\widehat{f^{\**}}(X)$ from \eqref{HATFSTDEF},
        It is straightforward to check that the only case in which $\widehat{f^{\**}}(X)$ is not the initial symmetric sequence (i.e. constant to the initial object in $\V$)
        is when $\ksi$ is in fact a $\mathfrak C$-signature, in which case
        \begin{equation}
              \label{SDFI_EQ}
              \Sigma_{\mathfrak D}(-,\ksi)[K] = f_!\Sigma_{\mathfrak C}(-,\ksi)[K].
        \end{equation}
        The claim for generating (trivial) cofibrations then follows from Corollary \ref{FTF_COR}(i).

        For the general case, we have that since $\otimes$ commutes with colimits in each variable, it commutes with filtered colimits, in particular transfinite compositions (\cite[Lemma 2.3.3]{Rez96}), and hence $f^{\**}$ preserves transfinite compositions.
        Moreover, $f^{\**}$ preserves pushouts of free symmetric sequences
        $\Sigma_{\mathfrak D}(-,\ksi)[i]$
        with $\ksi \in \Sigma_{\mathfrak C}$ by \eqref{SDFI_EQ}, Corollary \ref{FSU_COR}, and Lemma \ref{BASICPUSH LEMMA}.

        It remains to check that the image under $f^{\**}$ of a pushout over $\Sigma_{\mathfrak D}(-,\ksi)[i]$ with $\ksi \notin \Sigma_{\mathfrak C}$ is a (trivial) cofibration.
        
        {\color{red} can't go on from here yet}
  \end{proof}

  The following example demonstrates that $f^{\**}$ need not preserve all pushouts.
  \begin{example}
        Let $f \colon \set{0} \hookrightarrow \set{0,1}$, and consider the following pushouts in $\Cat^{\set{0,1}}(\sSet)$.
        \[
              \begin{tikzcd}
                    \set{0,1} \arrow[d, "{\mathbb F(-, (1;1))[\partial \Delta[0] \to \Delta[0]]}"'] \arrow[r]
                    &
                    {[1]} \arrow[d]
                    &&& % --------------------
                    \set{0,1} \arrow[d, "{\mathbb F(-, (1;1))[\partial \Delta[0] \to \Delta[0]]}"'] \arrow[r]
                    &
                    \widetilde{[1]} \arrow[d]
                    \\
                    \set{0} \amalg \mathbb N_0\set{1} \arrow[r]
                    &
                    \mathcal P
                    &&& % ----------
                    \set{0} \amalg \mathbb N_0\set{1} \arrow[r]
                    &
                    \tilde{\mathcal P}
              \end{tikzcd}
        \]
        In both cases, $f^{\**}$ applied to the left vertical arrows yields $id_{\varnothing}$.
        However, the pushout on the left is preserved by $f^{\**}$, as
        $f^{\**}([1]) = f^{\**}(\mathcal P) = \**$,
        but the pushout on the right is not preserved, as
        $f^{\**}(\widetilde{[1]}) = \**$ but $f^{\**}(\tilde{\mathcal P}) = \mathbb N_0$.

        \todo[inline]{In this case, the RHS is still a pushout of a free thing, but I don't know how to make this systematic
          - it must preserve cofibrant objects, but Lemma \ref{BASICPUSH LEMMA} only takes care of the first case from \cite[Thm. 1.15]{BM13}.}
  \end{example}
} % ---------------------------------------- OLIVEGREEN ----------------------------------------



















\subsubsection{Pushouts and injective change of color}

In this subsection, we show that, for any inclusions $f \colon \mathfrak C \to \mathfrak D$ of colors,
the the right adjoint $f^{\**}$ preserves pushouts of operads where one leg is in the image of $\check f_! = f_!$.

First note that, by a variation of the arguments in \eqref{OU EQ1}, \eqref{1STRED EQ}, \eqref{2NDRED EQ},
any pushout 
\[
\begin{tikzcd}
	A \ar{r} \ar{d} & \mathcal{O} \ar{d}
\\
	B \ar{r} & \mathcal{P}
\end{tikzcd}
\]
has a description
\begin{align*}
  \mathcal{P}
  &
    \simeq \colim_{[l] \in \Delta^{op}} 
    B_l \left( \O, A, A, A, B \right)
  \\
  &
    \simeq \colim_{[l] \in \Delta^{op},[n] \in \Delta^{op}} 
    B_l \left( \mathbb F^{\circ n+1} \O, \mathbb F^{\circ n+1} A, 
    \mathbb F^{\circ n+1} A, \mathbb F^{\circ n+1} A, \mathbb F^{\circ n+1} B \right)
\\
&
    \simeq \colim_{[l] \in \Delta^{op},[n] \in \Delta^{op}} 
    \mathsf{Lan} N \circ \left( N^{\circ n} \iota \O 
		\amalg_{\mathsf{F}}
	\left( N^{\circ n} \iota X\right)^{\amalg_{\mathsf{F}} 2l +1}
		\amalg_{\mathsf{F}}
	N^{\circ n} \iota Y \right)
\\
&	
	\simeq
	\mathop{\colim}\limits_{(\Delta \times \Delta)^{op}}
\left(
	\mathsf{Lan}_{\left(\Omega_{\mathfrak C}^{n,\lambda^a_l} \to \Sigma_{\mathfrak C}\right)^{op}} N_{n,l}^{(\O,A,B)}
\right)
\\
&	
	\simeq
	\mathsf{Lan}_{\left(\Omega_{\mathfrak C}^{p} \to
	\Sigma_{\mathfrak C}\right)^{op}} N^{(\O,A,B)}
    \stepcounter{equation}\tag{\theequation}\label{OU EQ2}
    % =: \colim_{n,l} \mathsf{Lan} \hat N_{n,l}^{(\O,X,Y)},
\end{align*}
where the partition $\lambda^a_l$ in the fourth line is the fully-active partition with $\left(\lambda^a_l\right)_a = \langle \langle l \rangle \rangle$
and, similarly to Proposition \ref{EXTENTREE PROP},
the double realization
$\OC^p \simeq |\Omega_{\mathfrak C}^{n,\lambda^a_l}|$
%
%in $\mathsf{Op}^{\mathfrak{C}}$ can be described as a left Kan extension
%\[
%\begin{tikzcd}
%	\Omega_{\mathfrak{C}}^{p,op} \ar{rr}{N^{(B,A,\mathcal{O})}} \ar{d}[swap]{\mathsf{lr}} &&
%	\mathcal{V}
%\\
%	\Sigma_{\mathfrak{C}}^{op} \ar[dashed]{rru}[swap]{\mathcal{P}}
%\end{tikzcd}
%\]
%Where $\Omega^p_{\mathfrak{C}}$ 
is the category whose objects are the
$\{\mathcal{O},A,B\}$-labeled trees
%$T \in \Omega_{\mathfrak{C}}$ together with a map
%$V(T) \to \{\mathcal{O},A,B\}$ (i.e. a labeling of the vertices of $T$ by $\{\mathcal{O},A,B\}$)
and whose arrows are tall maps $T \to S$ such that
\begin{enumerate}[label=(\roman*)]
\item if $v \in V(T)$ is $A$-labeled, then all vertices in $S_{v}$ are $A$-labeled
\item if $v \in V(T)$ is $B$-labeled, then all vertices in $S_{v}$ are either $A$-labeled or $B$-labeled
\item if $v \in V(T)$ is $\mathcal{O}$-labeled, then all vertices in $S_{v}$ are either $A$-labeled or $\mathcal{O}$-labeled
\end{enumerate}
Moreover, one has the formula
\begin{equation}\label{NBAO EQ}
N^{(B,A,\mathcal{O})}(T) = 
\left(\bigotimes_{v \in V_B(T)} B(T_v) \right) \otimes
\left(\bigotimes_{v \in V_A(T)} A(T_v) \right) \otimes
\left(\bigotimes_{v \in V_{\mathcal{O}}(T)} \mathcal{O}(T_v) \right)
\end{equation}





\begin{lemma}\label{BASICPUSH LEMMA}
Let $f \colon \mathfrak{C} \hookrightarrow \mathfrak{D}$ be an injective map of colors, $A \to B$ a map in $\mathsf{Op}^{\mathfrak{C}}$
and $\mathcal{O} \in \mathsf{Op}^{\mathfrak{D}}$.

Then if the leftmost diagram in $\mathsf{Op}^{\mathfrak{D}}$ is a pushout, so is the adjoint rightmost diagram in $\mathsf{Op}^{\mathfrak{C}}$.
\[
\begin{tikzcd}
	f_! A \ar{r} \ar{d} & \mathcal{O} \ar{d}
&
	A \ar{r} \ar{d} & f^{\**} \mathcal{O} \ar{d}
\\
	f_! B \ar{r} & \mathcal{P}
&
	B \ar{r} & f^{\**} \mathcal{P}
\end{tikzcd}
\]
\end{lemma}


\begin{proof}[Proof of Lemma \ref{BASICPUSH LEMMA}]

We start by noting that the top composite in the diagram
\[
\begin{tikzcd}
	\Omega_{\mathfrak{C}}^{p,op} \ar{r}{f_{\**}} \ar{d}[swap]{\mathsf{lr}} &
	\Omega_{\mathfrak{D}}^{p,op} \ar{rr}{N^{(f_!B,f_!A,\mathcal{O})}} \ar{d}[swap]{\mathsf{lr}} &&
	\mathcal{V}
\\
	\Sigma_{\mathfrak{C}}^{op} \ar{r}{f_{\**}} &
	\Sigma_{\mathfrak{D}}^{op} %\ar[dashed]{rru}[swap]{\mathcal{P}}
\end{tikzcd}
\]
is $N^{(B,A,f^{\**}\mathcal{O})}$ by Corollary \ref{FTF_COR}(ii), so that our intended result is that the induced map
\begin{equation}\label{LANISO EQ}
	\mathsf{Lan}_{\Omega_{\mathfrak{C}}^{p,op} \to \Sigma_{\mathfrak{C}}^{op}}
	N^{(B,A,f^{\**}\mathcal{O})}
\xrightarrow{\simeq}
\left(
	\mathsf{Lan}_{\Omega_{\mathfrak{D}}^{p,op} \to \Sigma_{\mathfrak{D}}^{op}}
	N^{(f_!B,f_!A,\mathcal{O})}
\right) \circ f_{\**}
\end{equation}
is an isomorphism. To see this, first let $\widehat{\Omega}^p_{\mathfrak{D}}$
be the full subcategory of $\Omega^p_{\mathfrak{D}}$
such that if 
$v \in V(T)$ is $A$ or $B$-labeled then $T_v \in \Sigma_{\mathfrak C}$.
It follows from \eqref{NBAO EQ} and Corollary \ref{FSU_COR} that $N^{(f_! B, f_! A, \mathcal{O})}(T) = \emptyset$ whenever $T \not \in \widehat{\Omega}^p_{\mathfrak{D}}$,
and it is straightforward to check that 
$\widehat{\Omega}^p_{\mathfrak{D}}$
is a sieve of $\Omega^p_{\mathfrak{D}}$, i.e. for any map $T \to S$ in $\Omega^p_{\mathfrak{D}}$ such that $S \in \widehat{\Omega}^p_{\mathfrak{D}}$ then $T \in \widehat{\Omega}^p_{\mathfrak{D}}$.
From this is follows that 
$N^{(f_!B,f_!A,\mathcal{O})}$
is the left Kan extension of its restriction to 
$\widehat{\Omega}^p_{\mathfrak{D}}$, 
so we are free to replace
$\Omega^p_{\mathfrak{D}}$
with
$\widehat{\Omega}^p_{\mathfrak{D}}$
in \eqref{LANISO EQ}.

The result now follows by noting that,
for each $C \in \Sigma_{\mathfrak{C}}$
the inclusion
$(C \downarrow \Omega^p_{\mathfrak{C}})
\to
(C \downarrow \widehat{\Omega}^p_{\mathfrak{D}})
$
has a right adjoint given by the assignment $T \mapsto T - E_{\mathfrak{D} \setminus \mathfrak{C}}(T)$,
where 
$E_{\mathfrak{D} \setminus \mathfrak{C}}(T)$ is the set of edges not in $\mathfrak{C}$ (the fact that this has a natural labeling follows since all the edges being collapsed must connect $\mathcal{O}$-vertices, so that there is no ambiguity as to how to label the vertices of $T - E_{\mathfrak{D} \setminus \mathfrak{C}}(T)$).
\end{proof}


\begin{corollary}\label{FGTPUSH COR}
Suppose that $f,g$ on the left pushout diagram below are both injective on colors.
\[
\begin{tikzcd}
	A \ar{r}{f} \ar{d}[swap]{g} & \mathcal{O} \ar{d}
&
	A \ar{r} \ar{d}[swap]{g} & f^{\**} \mathcal{O} \ar{d}
\\
	B \ar{r}{\tilde{f}} & \mathcal{P}
&
	B \ar{r} & \tilde{f}^{\**} \mathcal{P}
\end{tikzcd}
\]
Then the rightmost diagram is also a pushout diagram.
\end{corollary}

\begin{proof}
This follows by adding to both $A$ and $\mathcal{O}$ a disjoint trivial operad with object set $\mathfrak{C}(B) \setminus \mathfrak{C}(A)$. In doing so one does not alter the pushout, but the map $g$ now becomes a fixed color map, so that the previous result can be applied to the left pushout.
\end{proof}


\begin{corollary}
      \label{LOCALISO_COR}
      Suppose that we have a pushout in $\Op(\V)$ such that $f,g$ are both injective on colors.
      \[
            \begin{tikzcd}
                  A \arrow[d, "g"'] \arrow[r, "f"]
                  &
                  \O \arrow[d]
                  \\
                  B \arrow[r]
                  &
                  \P
            \end{tikzcd}
      \]
      If $A \to B$ is a local isomorphism, then so is $\O \to \P$.
\end{corollary}
\begin{proof}
      A map $g: A \to B$ is a local isomorphism iff $A \to g^{\**}B$ is an isomorphism in $\Op^{\mathfrak C(A)}(\V)$.
      Thus the result follows by applying Corollary \ref{FGTPUSH COR} with the notations of $B$ and $\O$ switched.      
\end{proof}



3


\section{Amalgamating monads (to be renamed and reskined)}
\label{AMALGMON_SEC}

\subsection{Combinable monads}
\label{COMBMON_SEC}

We establish our setting for this section.

\begin{definition}\label{AMALGMON DEF}
Let $P$ and $\bar{F}$ be two monads on the same category $\mathcal{C}$.
We say \emph{$P$ and $\bar{F}$ can be combined} if there exists 
a monad structure on the composite 
$F = P \bar{F}$ such that
\begin{enumerate}[label=(\roman*)]
\item $P \eta_{\bar{F}} \colon P \Rightarrow P \bar{F} = F$
is a map of monads
\item $\eta_P \bar{F} \colon \bar{F} \Rightarrow P \bar{F} = F$
is a map of monads
\item the composite below is the identity
\[
\begin{tikzcd}[column sep=3em]
	P \bar{F} \ar[Rightarrow]{r}{P \eta_{\bar{F}} \eta_P \bar{F}}
&
	P \bar{F} P \bar{F} \ar[equal]{r}
&
	FF  \ar[Rightarrow]{r}{\mu_F}
&
	F \ar[equal]{r}
&
	P \bar{F}
\end{tikzcd}
\]
\end{enumerate}
\end{definition}


\begin{remark}\label{SHORTCONV REM}
To ease notation, in our diagrams we will follow the following conventions:
\begin{itemize}
\item by default, unlabeled maps
% of the form
$P \Rightarrow F$, $\bar{F} \Rightarrow F$ refer to the maps in
Definition \ref{AMALGMON DEF}(i)(ii);
\item we write $\mu$ to denote any of the monad multiplications
$PP \Rightarrow P$,
$\bar{F}\bar{F} \Rightarrow \bar{F}$,
$FF \Rightarrow F$,
as well as for the left/right action multiplications
$PF \Rightarrow F$,
$FP \Rightarrow F$,
$\bar{F} F \Rightarrow F$,
$F \bar{F} \Rightarrow F$
induced by Definition \ref{AMALGMON DEF}(i)(ii)
(recall that, by definition,
$PF \Rightarrow F$ is the composite
$PF \Rightarrow FF \overset{\mu}{\Rightarrow} F$,
and similarly for the other action multiplications).

Therefore, to avoid ambiguity, we always write the 
source of a $\mu$ labeled map as a composite of two functors and the target as a single functor
(for example, this means we write 
$\mu\colon FF \Rightarrow F$ for the multiplication of $F$,
but \emph{never} write this multiplication as $\mu \colon P \bar{F} P \bar{F} \Rightarrow P \bar{F}$);
\item we write $\eta$ for either of the units 
$I \Rightarrow P$,
$I \Rightarrow \bar{F}$,
$I \Rightarrow F$.
\end{itemize}
Note that, following these conventions, 
Definition \ref{AMALGMON DEF}(iii)
says that 
$P\bar{F} \Rightarrow FF \overset{\mu}{\Rightarrow} F$
is the identity.
\end{remark}


\begin{definition}\label{TWISTMAP DEF}
Given $P,\bar{F},F$ as in Definition \ref{AMALGMON DEF}, 
the twist map $\tau \colon \bar{F} P \Rightarrow P \bar{F}$ is the composite
\[
\begin{tikzcd}
	\bar{F} P \ar[Rightarrow]{r}{}
&
	FF  \ar[Rightarrow]{r}{\mu}
&
	F 
\end{tikzcd}
\]
We note that in our diagrams $\tau$ is written as either
$\tau \colon \bar{F} P \Rightarrow P \bar{F}$ or as
$\tau \colon \bar{F} P \Rightarrow F$.
\end{definition}


\begin{lemma}\label{MODSTRCOMRE LEM}
\begin{enumerate}[label=(\roman*)]
\item the monad maps
$P \Rightarrow F$ and 
$\bar{F} \Rightarrow F$ coincide with the composites
\[
\begin{tikzcd}
	P \ar[Rightarrow]{r}{\eta P} &
	\bar{F} P \ar[Rightarrow]{r}{\tau} &
	F 
&
	\bar{F} \ar[Rightarrow]{r}{\bar{F} \eta} &
	\bar{F} P \ar[Rightarrow]{r}{\tau} &
	F 
\end{tikzcd}
\]
\item
the multiplications $\mu \colon PF \Rightarrow F$ and
$\mu \colon F\bar{F} \Rightarrow F$
are also given by
\[
\begin{tikzcd}
	PP\bar{F} \ar[Rightarrow]{r}{\mu \bar{F}} &
	P \bar{F} 
&
	P \bar{F} \bar{F} \ar[Rightarrow]{r}{P \mu} &
	P \bar{F} 
\end{tikzcd}
\]
\item
the multiplications
$\mu \colon F P \Rightarrow F$ and
$\mu \colon \bar{F} F \Rightarrow F$ are also given by the composites
\[
\begin{tikzcd}
	P\bar{F} P \ar[Rightarrow]{r}{P \tau} &
	P F \ar[Rightarrow]{r}{\mu} &
	F 
&
	\bar{F} P \bar{F} \ar[Rightarrow]{r}{\tau \bar{F}} &
	F \bar{F} \ar[Rightarrow]{r}{\mu} &
	F  
\end{tikzcd}
\]
\item the following composites both coincide with $\mu\mu \colon PP\bar{F}\bar{F} \Rightarrow P \bar{F}$
\[
\begin{tikzcd}
	P F \bar{F} \ar[Rightarrow]{r}{\mu \bar{F}} &
	F \bar{F} \ar[Rightarrow]{r}{\mu} &
	F 
%&
%	PP\bar{F}\bar{F} \ar[Rightarrow]{r}{\mu \mu} &
%	P \bar{F}  
&
	P F \bar{F} \ar[Rightarrow]{r}{P \mu} &
	\bar{F} F \ar[Rightarrow]{r}{\mu} &
	F 
\end{tikzcd}
\]
\end{enumerate}
\end{lemma}



\begin{proof}
Since all of (i),(ii),(iii) are symmetric, we show only the first half of each. (i),(ii),(iii) will follow from the following diagrams.
\begin{equation}\label{MODSTRCOMRE EQ}
\begin{tikzcd}[column sep=1.9em]
	P \ar[Rightarrow]{d}
	\ar[Rightarrow]{r}{\eta P}
&
	\bar{F}P \ar[Rightarrow]{d} \ar[Rightarrow]{rd}{\tau}
&
&%
	P P \bar{F} \ar[Rightarrow]{d}[swap]{\mu \bar{F}}
	\ar[Rightarrow]{r}
&
	FFF \ar[Rightarrow]{d}[swap]{\mu F}
	\ar[Rightarrow]{r}{F \mu}
&
	FF \ar[Rightarrow]{d}{\mu}
&%
	P \bar{F} P  \ar[Rightarrow]{d}[swap]{P \tau}
	\ar[Rightarrow]{r}
&
	FFF \ar[Rightarrow]{d}[swap]{F \mu}
	\ar[Rightarrow]{r}{\mu F}
&
	FF \ar[Rightarrow]{d}{\mu}
\\
	F \ar[Rightarrow]{r}[swap]{\eta F}
&
	FF \ar[Rightarrow]{r}[swap]{\mu}
&
	F
&%
	P \bar{F} \ar[Rightarrow]{r}
&
	FF \ar[Rightarrow]{r}[swap]{\mu}
&
	F
&%
	P F \ar[Rightarrow]{r}
&
	FF \ar[Rightarrow]{r}[swap]{\mu}
&
	F
\end{tikzcd}
\end{equation}
For (i), consider the left diagram in \eqref{MODSTRCOMRE EQ},
where the square commutes since
$\bar{F} \to P$ is a monad map (more precisely, it is thus compatible with monad units) and the triangle commutes by definition of $\tau$.
Since the bottom composite is the identity (due to $F$ being a monad), (i) follows.

For (ii), consider the center diagram in \eqref{MODSTRCOMRE EQ},
whose left (resp. right) half commutes since
$P \Rightarrow F$ is a monad map (resp. $F$ is a monad).
But by Definition \ref{AMALGMON DEF}(iii), the lower composite therein
is the identity and the top composite is the natural map
$PF \Rightarrow FF$, so (ii) follows.

For (iii), consider the right diagram in \eqref{MODSTRCOMRE EQ},
whose left (resp. right) half commutes by definition of $\tau$
(resp. since $F$ is a monad).
The bottom composite therein is the multiplication
$\mu \colon F \bar{F} \Rightarrow F$
while, by Definition \ref{AMALGMON DEF}(iii), the top composite is
the natural map $FP \Rightarrow FF$, so (iii) follows.


For (iv), part (i) allows us to rewrite 
the two composites as
\[
\begin{tikzcd}
	PP\bar{F}\bar{F} \ar[Rightarrow]{r}{\mu \bar{F}\bar{F}} &
	P \bar{F} \bar{F} \ar[Rightarrow]{r}{P\mu} &
	P \bar{F}
&
	PP\bar{F}\bar{F} \ar[Rightarrow]{r}{PP \mu } &
	P P \bar{F} \ar[Rightarrow]{r}{\mu \bar{F}} &
	P \bar{F}
\end{tikzcd}
\]
which are both $\mu\mu\colon PP\bar{F}\bar{F} \Rightarrow P \bar{F}$.
\end{proof}


\begin{proposition}\label{ALTMULT PROP}
\begin{enumerate}[label=(\roman*)]
\item
the multiplication
$\mu \colon FF \Rightarrow F$ 
can also be described as the composite
\begin{equation}\label{ALTMULT EQ}
\begin{tikzcd}
	P \bar{F} P \bar{F} \ar[Rightarrow]{r}{P \tau \bar{F}}
&
	P P \bar{F} \bar{F}   \ar[Rightarrow]{r}{\mu \mu}
&
	P \bar{F} 
\end{tikzcd}
\end{equation}
\item
the following diagrams commute
\begin{equation}\label{DOUBLETWIST EQ}
\begin{tikzcd}
	\bar{F} P P \ar[Rightarrow]{r}{\tau P} 
	\ar[Rightarrow]{d}[swap]{\bar{F} \mu}
&
	P \bar{F} P \ar[Rightarrow]{r}{P\tau}
&
	P P \bar{F} \ar[Rightarrow]{d}{\mu \bar{F}}
&%%
	\bar{F} \bar{F} P  \ar[Rightarrow]{r}{\bar{F} \tau}
	\ar[Rightarrow]{d}[swap]{\mu P}
&
	\bar{F} P \bar{F}  \ar[Rightarrow]{r}{\tau \bar{F}}
&
	P \bar{F} \bar{F} \ar[Rightarrow]{d}{P \mu}
\\
	\bar{F} P \ar[Rightarrow]{rr}{\tau} 
&&
	P \bar{F}
&%%
	\bar{F} P \ar[Rightarrow]{rr}{\tau} 
&&
	P \bar{F}
\end{tikzcd}
\end{equation}
\item
the following diagrams commute
\begin{equation}\label{UNITTWIST EQ}
\begin{tikzcd}
	\bar{F} \ar[Rightarrow]{d}[swap]{\bar{F} \eta}
	\ar[Rightarrow]{rd}{\eta \bar{F}}
&
&%%
	P \ar[Rightarrow]{d}[swap]{\eta P}
	\ar[Rightarrow]{rd}{P \eta}
&
\\
	\bar{F} P \ar[Rightarrow]{r}{\tau} 
&
	P \bar{F}
&%%
	\bar{F} P \ar[Rightarrow]{r}{\tau} 
&
	P \bar{F}
\end{tikzcd}
\end{equation}
\end{enumerate}
\end{proposition}


\begin{proof}
Note first that (iii) is simply 
a rephrasing of Lemma \ref{MODSTRCOMRE LEM}(i), which we include here
for the sake of Remark \ref{ALTMULT REM}.
(i) and (ii) will follow from the following diagrams.
\begin{equation}\label{ALTMPR EQ}
\begin{tikzcd}
	P \bar{F} P \bar{F} \ar[Rightarrow]{d} 
	\ar[Rightarrow]{r}{P \tau \bar{F}}
&
	P F \bar{F}  \ar[Rightarrow]{r}{\mu \bar{F}} \ar[Rightarrow]{d}
&
	F \bar{F} \ar[Rightarrow]{r}{\mu}
	\ar[Rightarrow]{d}
&
	F \ar[equal]{d}
&&%
	\bar{F} P P \ar[Rightarrow]{rr}{\tau P}
	\ar[Rightarrow]{rd}{}
	\ar[Rightarrow]{dd}[swap]{\bar{F} \mu}
&&
	F P \ar[Rightarrow]{d}
\\
	FFFF \ar[Rightarrow]{r}{F \mu F}
	\ar[Rightarrow]{d}[swap]{\mu \mu}
&
	FFF \ar[Rightarrow]{r}{\mu F}
&
	FF \ar[Rightarrow]{r}{\mu}
&
	F \ar[equal]{d}
&&%
&
	FFF \ar[Rightarrow]{d}[swap]{F \mu}
	\ar[Rightarrow]{r}{\mu F}
&
	FF \ar[Rightarrow]{d}{\mu}
\\
	FF \ar[Rightarrow]{rrr}{\mu}
&&&
	F
&&%
	\bar{F} P \ar[Rightarrow]{r}
&
	FF \ar[Rightarrow]{r}[swap]{\mu}
&
	F
\end{tikzcd}
\end{equation}
For (i), consider the left diagram in \eqref{ALTMPR EQ}, where the top left square commutes by definition of the twist $\tau$,
the remaining top squares commute by definition of the action multiplication,
and the bottom rectangle commutes due to $F$ being a monad.
Definition \ref{AMALGMON DEF}(iii)
then says that the left vertical composite therein is the identity,
while by Lemma \ref{MODSTRCOMRE LEM}(iv) the top composite is 
\eqref{ALTMULT EQ}, so (i) follows.

For (ii), by symmetry it suffices to show the first half.
Consider now the right diagram in \eqref{ALTMPR EQ},
where the top right section commutes by definition of $\tau$,
the bottom left section commutes since $P \Rightarrow F$ is a monad map,
and the bottom right square commutes since $F$ is monad.
But now the bottom composite therein is $\tau$ by definition
while, by Lemma \ref{MODSTRCOMRE LEM}(iii),
the top right outer composite from $\bar{F}PP$ to $F$ matches the top right outer composite in (the left side of) \eqref{DOUBLETWIST EQ},
so (ii) follows.
\end{proof}



\begin{remark}\label{ALTMULT REM}
\eqref{ALTMULT EQ} shows that the monad structure on $F=P\bar{F}$ is determined by monoidal structures on $P$ and $\bar{F}$ together with  the twist map $\tau$.
Moreover, \eqref{DOUBLETWIST EQ} and \eqref{UNITTWIST EQ} provide necessary conditions on $P,\bar{F},\tau$
for the formula \eqref{ALTMULT EQ}
to describe a monoidal structure on $F=P\bar{F}$
that satisfies the conditions in 
Definition \ref{AMALGMON DEF}.
In fact, the conditions \eqref{DOUBLETWIST EQ}, \eqref{UNITTWIST EQ}
turn out to also be sufficient.
Indeed, it is not hard to check 
that \eqref{DOUBLETWIST EQ} implies that either of the two iterated multiplications
$FFF \Rightarrow F$ induced by \eqref{ALTMULT EQ}
match the map 
$P\bar{F} P\bar{F} P \bar{F} \Rightarrow 
PPP \bar{F} \bar{F} \bar{F} \Rightarrow P \bar{F}$,
where the first map is a (unique) combination of twists $\tau$
and the second map is given by the iterated multiplications for $P,\bar{F}$.
Similarly, \eqref{UNITTWIST EQ} implies that the maps
$P \Rightarrow F$, $\bar{F} \Rightarrow F$
are maps of monads.
\end{remark}



\subsection{Kleisli categories}

Given a map of monads $T \Rightarrow F$ (for example, $P \Rightarrow F$ or $\bar F \Rightarrow F$), one can explore the existence and properties of various free and forgtful functors.
Somewhat surprisingly, the most natural and general setting for these comparisons is not the entire category of algebras,
but instead in the Kleisli category of free algebras and all algebra maps between them.

\begin{definition}[{\cite{Kl65}}]
      Let $T$ be a monad on a category $\mathcal C$.
      The \textit{Kleisli category} $\mathsf{Kl}_T$ is the category defined as follows:
      it has the same objects as $\mathcal{C}$,
      and arrows from $A$ to $B$,
      which we denote $A \rightsquigarrow B$,
      are given by arrows $A \to TB$ in $\mathcal{C}$.
      Composition
      $A \rightsquigarrow B \rightsquigarrow C$
      of underlying maps 
      $A \xrightarrow{f} TB$,
      $B \xrightarrow{g} TC$
      given by the composite
      $A \xrightarrow{f} TB \xrightarrow{Tg} TTB \xrightarrow{\mu} TB$.

      We say an arrow
      $A \rightsquigarrow B$ in $\mathsf{Kl}_F$ is a 
      \emph{free arrow}
      if the underlying map has a factorization
      $A \xrightarrow{f} B \xrightarrow{\eta} TB$.
      In particular, the unit arrow $A \rightsquigarrow A$ in $\Kl_T$
      is the free arrow $A \xrightarrow{=} A \xrightarrow{\eta} TA$.
\end{definition}

\begin{remark}\label{KLEISLIDEF REM}
% Letting $T$ be a monad on the category $\mathcal{C}$,
% recall that the Kleisli category
% $\mathsf{Kl}_T$ is the category with:
% the same objects as $\mathcal{C}$;
% arrows from $A$ to $B$,
% which we denote $A \rightsquigarrow B$,
% the underlying arrows $A \to TB$ in $\mathcal{C}$;
% composition
% $A \rightsquigarrow B \rightsquigarrow C$
% of underlying maps 
% $A \xrightarrow{f} TB$,
% $B \xrightarrow{g} TC$
% given by the composite
% $A \xrightarrow{f} TB \xrightarrow{Tg} TTB \xrightarrow{\mu} TB$.
% Moreover,
      Justifying our intuitive description above,
      there is a canonical fully faithful inclusion
      \[
            \begin{tikzcd}[row sep = 0pt]
                  \Kl_T \ar[hookrightarrow]{r}{\iota} &
                  \Alg_T
                  \\
                  A \ar[mapsto]{r} &
                  TA
                  \\
                  (A \xrightarrow{f} TB) \ar[mapsto]{r} &
                  (TA \xrightarrow{Tf} TTB \xrightarrow{\mu} TB) 
            \end{tikzcd} 
      \]
      which identifies $\mathsf{Kl}_T$ with the full subcategory of
      $\mathsf{Alg}_T$ spanned by the free algebras.
      %
      % Moreover, we say an arrow
      % $A \rightsquigarrow B$ in $\mathsf{Kl}_F$ is a 
      % \emph{free arrow}
      % if the underlying map has a factorization
      % $A \xrightarrow{f} B \xrightarrow{\eta} TB$.
      Moreover, an arrow is free iff
      % This is equivalent to
      its image under $\iota$
      is the free map
      $TA \xrightarrow{Tf} TB$.
      % Note that free arrows provide a canonical functor
      % $\mathcal C \to \Kl_T$ and that the unit arrow $A \rightsquigarrow A$ in $\Kl_T$
      % is the free arrow $A \xrightarrow{=} A \xrightarrow{\eta} TA$.
      In particular, the free functor $\mathcal C \to \Alg_T$ factors through $\Kl_T$ via free arrows.
\end{remark}


To start our analysis,
for any map of monads $\psi \colon T \to F$,
one has a well known forgetful functor 
$\mathsf{Alg}_F \xrightarrow{\psi^{\**}} \mathsf{Alg}_T$
sending a $F$-monad $Y$ with multiplication
$FY \xrightarrow{m} Y$
to a $T$-monad with the same underlying object $Y$
and multiplication given by
$TY \xrightarrow{\psi} FY \xrightarrow{m} Y$.

A dual situation holds for Kleisli categories:
\begin{lemma}
      \label{KLEISLIPUSH LEM}
      Given a map of monads $\psi \colon T \to F$,
      there is a pushforward functor
      $\mathsf{Kl}_T \xrightarrow{\psi_!} \mathsf{Kl}_F$,
      defined as the identity on objects,
      and sending the map $A \rightsquigarrow B$
      given by the underlying map $A \xrightarrow{f} TB$
      to the map $\psi_!A \rightsquigarrow \psi_!B$
      given by the underlying composite
      $A \xrightarrow{f} TB \xrightarrow{\psi} FB$.
\end{lemma}
\begin{proof}
      To check $\psi_!$ respects composition of arrows, let
      $A \rightsquigarrow B$,
      $B \rightsquigarrow C$
      be maps in $\Kl_T$ with
      underlying maps
      $A \xrightarrow{f} TB$,
      $B \xrightarrow{f} TC$.
      Then $\psi_!$ of the composite 
      $A \rightsquigarrow B \rightsquigarrow C$
      is given by the top right composite in \eqref{CHECKFUN EQ}
      while the composite of 
      $\psi_!A \rightsquigarrow \psi_! B$,
      $\psi_! B \rightsquigarrow \psi_! C$
      is given by the bottom  left composite in \eqref{CHECKFUN EQ}.
      \begin{equation}\label{CHECKFUN EQ}
            \begin{tikzcd}[column sep = 1.55em]
                  A \ar{r}{f} 
                  &
                  TB \ar{r}{Tg} \ar{d}[swap]{\psi}
                  &
                  TTC \ar{rr}{\mu}  \ar{d}[swap]{\psi T}
                  &&
                  TC \ar{d}{\psi}
                  \\
                  &
                  FB \ar{r}{Fg}
                  &
                  FTB \ar{r}{F\psi}
                  &
                  FFC \ar{r}{\mu}
                  &
                  FC
            \end{tikzcd}
      \end{equation}
      Lastly, note that $\psi_!$ maps the free arrow 
      $A \xrightarrow{f} B \xrightarrow{\eta} TB$
      to the free arrow
      $A \xrightarrow{f} B \xrightarrow{\eta} FB$,
      so that in particular it preserves unit arrows.
\end{proof}




\begin{proposition}\label{PARTADJ PROP}
The functors in Remark \ref{KLEISLIDEF REM} and Lemma \ref{KLEISLIPUSH LEM}
fit into a (non-commutative) square
\begin{equation}\label{PARTADJ EQ}
\begin{tikzcd}
	\mathsf{Kl}_{T} \ar{r}{\psi_!} \ar[hookrightarrow]{d}[swap]{\iota}&
	\mathsf{Kl}_{F} \ar[hookrightarrow]{d}{\iota} \arrow[dl, phantom, "\times"]
\\
	\mathsf{Alg}_{T} &
	\mathsf{Alg}_{F}  \ar{l}{\psi^{\**}}.
\end{tikzcd}
\end{equation}
and there are isomorphisms as below, natural in $A \in \mathsf{Kl}_T$, $Y \in \mathsf{Alg}_F$.
\begin{equation}\label{PARTADJISO EQ}
	\mathsf{Alg}_T(\iota A, \psi^{\**} Y) \simeq
	\mathsf{Alg}_F(\iota \psi_! A,Y)
\end{equation}
\end{proposition}
Informally, this result says that
$\psi_!$ is a ``partial left adjoint'' to $\psi^{\**}$
along the fully-faithful inclusion 
$\mathsf{Kl}_T \overset{\iota}{\hookrightarrow} \mathsf{Alg}_T$,
motivating our choice of notation.


\begin{proof}
The isomorphisms \eqref{PARTADJISO EQ} are the composites of the isomorphisms
\begin{equation}\label{PARTADJPROOF EQ}
	\mathsf{Alg}_T(T A, \psi^{\**} Y) \simeq
	\mathcal{C}(A,Y) \simeq
	\mathsf{Alg}_F(FA,Y),
\end{equation}
so it remains to show naturality.
Naturality with respect to $Y \in \Alg_F$ is clear.
However, naturality with respect to $A \in \Kl_T$
requires care, since a priori \eqref{PARTADJPROOF EQ}
is only functorial with respect to free maps.
But now note that 
$\varphi \colon FA \to Y$
and 
$\bar{\varphi} \colon TA \to \psi^{\**} Y$
are identified precisely if the two composites
$A \to Y$ in the left diagram below coincide (morever, both triangles therein commute, with commutativity of the right triangle
inherited from commutatity of the full diagram, due $TA$ being free $T$-algebra).
Given an arrow $\bar{A} \rightsquigarrow A$ in $\Kl_T$
given by an underlying map $\bar{A} \xrightarrow{f} TA$, 
we must then show that the two outer composites
$\bar{A} \to Y$ in the rightmost diagram coincide.
\begin{equation}\label{ADJOINTCORR EQ}
\begin{tikzcd}[column sep = 1.55em]
	A \ar{r}{\eta} \ar{rd}[swap]{\eta}
&
	T A \ar{r}{\bar{\varphi}} \ar{d}[swap]{\psi}
&
	Y 
&&%%
	\bar{A} \ar{r}{\eta} \ar{rd}[swap]{\eta}
&
	T \bar{A} \ar{r}{\iota f} \ar{d}[swap]{\psi}
&
	T A \ar{r}{\bar{\varphi}} \ar{d}[swap]{\psi}
&
	Y 
\\
&
	FA \ar{ru}[swap]{\varphi}
&
&&%%
&
	F \bar{A} \ar{r}[swap]{\iota (\psi_! f)}
&
	FA \ar{ru}[swap]{\varphi}
&
\end{tikzcd}
\end{equation}
It hence suffices to check that the two composites
$\bar{A} \to FA$ in that diagram coincide and, after unpacking definitions, this follows from the commutativity of the following diagram.
\begin{equation}
\begin{tikzcd}[column sep = 1.55em]
	\bar{A} \ar{r}{\eta} \ar{rd}[swap]{\eta} 
&
	T\bar{A} \ar{r}{Tf} \ar{d}[swap]{\psi}
&
	TTA \ar{rr}{\mu}  \ar{d}[swap]{\psi T}
&&
	TA \ar{d}{\psi}
\\
&
	F \bar{A} \ar{r}{Ff}
&
	FTA \ar{r}{F\psi}
&
	FFA \ar{r}{\mu}
&
	FA
\end{tikzcd}
\end{equation}
\end{proof}

Necessary and sufficient conditions to extend this partial adjoint to a full adjoint are given below.
\begin{corollary}
      \label{ALGLEFTEX_COR}
      Fix a map of monads $\psi \colon T \to F$.
      \begin{enumerate}[label = (\roman*)]
      \item The left adjoint $\psi_!$ of $\psi^{\**} \colon \Alg_F \to \Alg_T$ exists iff the coequalizers of $F$-algebras
            \[
                  \psi_!(X):= \mathop{coeq}\left( FTX \overset{\mu}{\underset{Fm}{\rightrightarrows}} FX \right)
            \]
            exist for all $X \in \mathsf{Alg}_T$.
      \item If $\psi_!$ exists, then the triangle of left adjoints below commutes.
            \[
                  \begin{tikzcd}
                        \mathcal C \arrow[d, "T"'] \arrow[r, "F"]
                        &
                        \mathsf{Alg}_F
                        \\
                        \mathsf{Alg}_T \arrow[ur, "{\psi_!}"']
                  \end{tikzcd}
            \]           
      \end{enumerate}
\end{corollary}
\begin{proof}
      The existence of $\psi_!$ is equivalent to the representability of the functors 
      $\Alg_T(X,\psi^{\**}(-))\colon \Alg_{F} \to \mathsf{Set}$
      for each $X \in \Alg_T$,
      so that Proposition \ref{PARTADJ PROP} implies that these functors are representable whenever $X \simeq FA$ is free.
      Hence, since any $X \in \Alg_T$
      with multiplication $m \colon TX \to X$
      is canonically described by the coequalizer % of free $T$-algebras
      $coeq \left(
            T T X 
            \overset{\mu}{\underset{Fm}{\rightrightarrows}}
            T X
      \right)$,
      (i) result follows.

      Part (ii) follows since the associated triangle of right adjoints commutes.
\end{proof}


% \begin{remark}\label{ALGLEFTEX REM}
% Recall that the existence of a left adjoint $\psi_!$ to 
% $\psi^{\**} \colon \Alg_F \to \Alg_T$
% is equivalent to the representability of the functors 
% $\Alg_T(X,\psi^{\**}(-))\colon \Alg_{F} \to \mathsf{Set}$
% for each $X \in \Alg_T$,
% so that Proposition \ref{PARTADJ PROP} implies that these functors are representable whenever $X \simeq FA$ is free.

% Hence, since any $X \in \Alg_T$
% with multiplication $m \colon TX \to X$
% is canonically described by the coequalizer % of free $T$-algebras
% $coeq \left( T T X 
% \overset{\mu}{\underset{Fm}{\rightrightarrows}}
% T X\right)$
% it follows that the left adjoint 
% $\psi_! \colon \Alg_F \to \Alg_T$
% exists iff the coequalizers 
% $coeq \left( F T X 
% \overset{\mu}{\underset{Fm}{\rightrightarrows}}
% F X\right)
% $
% of $F$-algebras exist for all $X \in \Alg_T$, in which case it is
% $\psi_!(X)=coeq \left( F T X 
% \overset{\mu}{\underset{Fm}{\rightrightarrows}}
% F X\right)$.
% \end{remark}


\begin{remark}\label{ADJSRMON REM}
Beck's monadicity theorem (cf. \cite[VI.7 Thm. 1]{McL},\cite[Thm. 5.5.1]{Ri17})
characterizes monadic adjunctions $L\colon \mathcal{C} \rightleftarrows \mathcal{D}\colon R$
as those where $R$ creates $R$-split coequalizers. 

Hence, should the adjunction 
$\psi_! \colon \Alg_T \rightleftarrows \Alg_F \colon \psi^{\**}$
in Corollary \ref{ALGLEFTEX_COR} exist,
it readily follows from Beck's theorem that this adjunction is monadic
(since in 
$\mathcal{C} \rightleftarrows
\mathsf{Alg}_T \rightleftarrows
\Alg_F$
both the left and total adjunctions are monadic).
However, since $\psi_!$ is defined using coequalizers of $F$-algebras,
it is in general not possible to give a simple description of the adjunction monad
$F_T = \psi^{\**} \psi_!$.
%
%Nonetheless, should $T,F$ preserve reflexive coequalizers in $\mathcal{C}$,
%one has that reflexive coequalizers in algebras
%are underlying coequalizers in $\mathcal{C}$
%\cite[Thm. 5.6.5]{Ri17}.
%Therefore, by manipulating the coequalizers one can show that the multiplication for $F_T$ the induced map on vertical coequalizers for the diagram
%\[
%\begin{tikzcd}[column sep =40pt]
%	FTFT X
%	\ar[shift left=.3em]{d}{\mu\mu}
%	\ar[shift right=.3em]{d}[swap]{F\mu m}
%	\ar{r}{\mu FT \circ \mu T}
%&
%	FTX
%	\ar[shift left=.3em]{d}{\mu}
%	\ar[shift right=.3em]{d}[swap]{F m}
%\\
%	FF X
%	\ar{r}{\mu}
%&
%	F X
%\end{tikzcd}
%\]
\end{remark}


\subsection{Adjoints for combinable monads}

Returning to our setup from \S \ref{COMBMON_SEC},
we have the existence of a right adjoint to $\psi_!$ on Kleisli categories for a particular map of monads $\psi$.

\begin{definition}
      Let $P,\bar{F}$ and $F = P \bar{F}$ be as in 
      Definition \ref{AMALGMON DEF},
      and write
      \[
            \psi = P\eta_{\bar F} \colon P \Rightarrow P \bar{F} = F.
      \]
      Define the functor
      $\psi^{\**} \colon \Kl_F \to \Kl_P$
      on objects by
      $\psi^{\**}A = \bar{F} A$,
      and sending the arrow
      $A \rightsquigarrow B$
      with underlying map
      $A \xrightarrow{f} FB$
      to the arrow 
      $\bar{F} A \rightsquigarrow \bar{F} B$
      with underlying map the composite
      $\bar{F} A \xrightarrow{\bar{F}f} \bar{F} FB
      =
      \bar{F}P \bar{F} B \xrightarrow{\tau \bar{F}}
      P \bar{F} \bar{F} B \xrightarrow{P \mu}
      P\bar{F}B$
      or, equivalently (cf. Lemma \ref{MODSTRCOMRE LEM}(iii)(ii)),
      $\bar{F} A \xrightarrow{\bar{F}f} \bar{F} FB
      \xrightarrow{\mu} FB = P\bar{F}B$.
\end{definition}

\begin{proposition}\label{PARTADJP PROP}
Let $P,\bar{F}$ and $F = P \bar{F}$ be as above.
% Definition \ref{AMALGMON DEF},
% and write $\psi = P\eta \colon P \Rightarrow P \bar{F} = F$.
%
% Then there is a functor
% $\psi^{\**} \colon \Kl_F \to \Kl_P$
% given on objects by
% $\psi^{\**}A = \bar{F} A$
% and sending the arrow
% $A \rightsquigarrow B$
% with underlying arrow 
% $A \xrightarrow{f} FB$
% to the arrow 
% $\bar{F} A \rightsquigarrow \bar{F} B$
% with underlying arrow the composite
% $\bar{F} A \xrightarrow{\bar{F}f} \bar{F} FB
% =
% \bar{F}P \bar{F} B \xrightarrow{\tau \bar{F}}
% P \bar{F} \bar{F} B \xrightarrow{P \mu}
% P\bar{F}B$
% or, equivalently (cf. Lemma \ref{MODSTRCOMRE LEM}(iii)(ii)),
% $\bar{F} A \xrightarrow{\bar{F}f} \bar{F} FB
% \xrightarrow{\mu} FB = P\bar{F}B$.
%
In the diagram
\begin{equation}\label{PARTADJP EQ}
\begin{tikzcd}
	\mathsf{Kl}_{P} 
	\ar[shift left=0.25em]{r}{\psi_!} 
	\ar[hookrightarrow]{d}[swap]{\iota}
&
	\mathsf{Kl}_{F} 
	\ar[shift left=0.25em]{l}{\psi^{\**}}
	\ar[hookrightarrow]{d}{\iota}
\\
	\mathsf{Alg}_{P} &
	\mathsf{Alg}_{F}  \ar{l}{\psi^{\**}}
\end{tikzcd}
\end{equation}
one has that:
\begin{enumerate}[label=(\roman*)]
\item the top horizontal maps are adjoint,
with adjuntion unit
$A \rightsquigarrow \psi^{\**} \psi_! A$
for $A \in \Kl_P$
and counit
$\psi_! \psi^{\**} B \rightsquigarrow B$
for $B \in \Kl_F$
given respectively by 
\begin{equation}\label{UNITCOUNIT EQ}
	A \xrightarrow{\eta} 
	\bar{F} A \xrightarrow{\eta \bar{F}}
	P \bar{F} A
        \qquad
        \mbox{and}
        \qquad
	\bar{F} B \xrightarrow{\eta \bar{F}}
	P \bar{F} B =
	F B.
\end{equation}
\item the composites
$
	\mathsf{Kl}_F \xrightarrow{\psi^{\**}} 
	\mathsf{Kl}_P \xrightarrow{\iota} 
	\mathsf{Alg}_P
$
and
$
	\mathsf{Kl}_F \xrightarrow{\iota}
	\mathsf{Alg}_F  \xrightarrow{\psi^{\**}} 
	\mathsf{Alg}_P
$
coincide.
\end{enumerate}
\end{proposition}



\begin{proof}
We first address (ii).
On objects,
after unpacking notation
$\psi^{\**} \iota B$
is the $P$-algebra
$F B$ with multiplication
$PFB \rightarrow FFB \xrightarrow{\mu} B$
while 
$ \iota \psi^{\**} B$
is the $P$-algebra
$P \bar{F} B $
with multiplication
$P P \bar{F} B \xrightarrow{\mu \bar{F}} P \bar{F} B$,
and these coincide by Lemma \ref{MODSTRCOMRE LEM}(ii).
On a map
$B \rightsquigarrow \bar{B}$
with underlying arrow $B \xrightarrow{f} F \bar{B}$,
applying $\psi^{\**} \iota$
gives the map
$F B \xrightarrow{F f} FF\bar{B} \xrightarrow{\mu} F \bar{B}$
while applying $\iota \psi^{\**}$
gives the map
$P \bar{F} B \xrightarrow{P \bar{F} f} 
P \bar{F} F \bar{B} = P \bar{F} P \bar{F} \bar{B}
\xrightarrow{P \tau \bar{F}}
P P \bar{F} \bar{F} \bar{B}
\xrightarrow{PP \mu}
P P \bar{F} \bar{B} 
\xrightarrow{\mu \bar{F}} P\bar{F} \bar{B}$,
and these coincide by 
Proposition \ref{ALTMULT PROP}(i).


We now address the remaining claims. 
Recalling that the $\iota$ functors are fully faithful,
$\psi^{\**} \colon \Kl_F \to \Kl_P$
can be thought of as a restriction of
$\psi^{\**} \colon \Alg_F \to \Alg_P$. 
Hence, the implicit claim that $\psi^{\**} \colon \Kl_F \to \Kl_P$
is a functor is inherited from 
$\psi^{\**} \colon \Alg_F \to \Alg_P$ being a functor.
And, similarly, the adjunction claim in (i)
is inherited from the ``partial adjunction'' claim in 
Proposition \ref{PARTADJP PROP}.


Lastly, to describe the unit 
(resp. counit)
we must find the adjoint
to 
$\psi_! A \xrightarrow{=} \psi_! A$
(resp. $\psi^{\**} B \xrightarrow{=} \psi^{\**} B$)
which corresponds under $\iota$ to the identity
$FA = FA$
(resp. $P\bar{F} B = FB$).
By the string of identifications 
\eqref{PARTADJPROOF EQ}
the adjoint is given 
by the top (resp. bottom) composite on the left
(resp. diagram) below
(cf. \eqref{ADJOINTCORR EQ}).
\begin{equation}
\begin{tikzcd}[column sep = 1.55em]
	A \ar{r}{\eta} \ar{rd}[swap]{\eta}
&
	P A \ar[dashed]{r}{\psi} \ar{d}[swap]{\psi}
&
	FA
&&%%
	\bar{F} B \ar{r}{\eta} \ar{rd}[swap]{\eta}
&
	P \bar{F} B \ar[equal]{r} \ar{d}[swap]{\psi}
&
	F B 
\\
&
	FA \ar[equal]{ru}
&
&&%%
&
	F\bar{F} B \ar[dashed]{ru}[swap]{\mu}
&
\end{tikzcd}
\end{equation}
That these composites match the maps in
\eqref{UNITCOUNIT EQ}
follows since $\psi\colon P \Rightarrow F$ is a map of monads. 
\end{proof}



\begin{remark}\label{MONALGP REM}
By Proposition \ref{PARTADJP PROP}
one has an adjunction monad $\psi^{\**} \psi_!$
on $\Kl_P$ which is given on objects by $\bar{F}$.
Moreover the first half of
\eqref{UNITCOUNIT EQ} shows that the unit
$I \Rightarrow \psi^{\**} \psi_!$
consists of the free arrows on the unit $I \Rightarrow \bar{F}$
while, 
by applying the definition of $\psi^{\**}$ to the second half of
\eqref{UNITCOUNIT EQ}
gives that the monad multiplication
$\psi^{\**} \psi_!\psi^{\**} \psi_! \Rightarrow \psi^{\**} \psi_!$
is given by the left right composite below.
But hence, by the top right composite, 
this monad multiplication consists of the free arrows on
the monad multiplication $\bar{F} \bar{F} \Rightarrow \bar{F}$. 
\begin{equation}
\begin{tikzcd}[column sep = 1.55em]
	\bar{F} \bar{F} A \ar{d}[swap]{\bar{F}\eta \bar{F}}
	\ar[equal]{r} 
&
	\bar{F} \bar{F} A 
	\ar{r}{\mu} \ar{d}[swap]{\eta \bar{F} \bar{F}}
&
	\bar{F} A   \ar{d}[swap]{\eta \bar{F}}
\\
	\bar{F} P \bar{F} A \ar{r}[swap]{\tau \bar{F}}
&
	P \bar{F} \bar{F} A \ar{r}[swap]{P\mu}
&
	P \bar{F} A
\end{tikzcd}
\end{equation}
As an aside, we note that one can further check that the top adjunction in \eqref{PARTADJ EQ} is a Kleisli adjunction, i.e.
that it identifies $\Kl_F$ as the Kleisli category over 
$\Kl_P$ for the monad $\psi^{\**} \psi_!$.
\end{remark}


\begin{remark}
Should $\psi^{\**}\colon \Alg_F \to \Alg_P$ admit a left adjoint, 
one has that in the diagram
\begin{equation}\label{PARTADJDB EQ}
\begin{tikzcd}
	\mathsf{Kl}_{P} 
	\ar[shift left=0.25em]{r}{\psi_!} 
	\ar[hookrightarrow]{d}[swap]{\iota}
&
	\mathsf{Kl}_{F} 
	\ar[shift left=0.25em]{l}{\psi^{\**}}
	\ar[hookrightarrow]{d}{\iota}
\\
	\mathsf{Alg}_{P} 
	\ar[shift left=0.25em]{r}{\psi_!} &
	\mathsf{Alg}_{F}
	\ar[shift left=0.25em]{l}{\psi^{\**}}
\end{tikzcd}
\end{equation}
both functors in the bottom adjunction
extend the functors in the top adjunction.
As such, the monad
$F_P = \psi^{\**} \psi_!$ on $\Alg_P$
extends the monad $\psi^{\**} \psi_!$ on $\Kl_P$ which, 
following \ref{MONALGP REM}, can be thought of as the monad $\bar{F}$.

As noted in Remark \ref{ADJSRMON REM}, the monad $F_T$ is in general hard to describe. 
Nonetheless, should $P,F$ preserve reflexive coequalizers in $\mathcal{C}$,
one has that reflexive coequalizers in algebras
are underlying coequalizers in $\mathcal{C}$
\cite[Thm. 5.6.5]{Ri17}
and thus preserved by both functors in the bottom adjunction, and hence also by $F_T$.
It thus follows that $F_TF_TX$
can be described as the left vertical coequalizer in the diagram below 
(where the two squares coincide, 
with the right square included to more easily describe the vertical maps) 
\begin{equation}\label{FRMONDES EQ}
\begin{tikzcd}[column sep =40pt]
	P \bar{F}\bar{F} P X
	\ar[shift left=.3em]{d}{}
	\ar[shift right=.3em]{d}[swap]{}
	\ar{r}{P\mu P}
&
	P \bar{F} P X
	\ar[shift left=.3em]{d}{}
	\ar[shift right=.3em]{d}[swap]{}
&%
	F\bar{F} P X
	\ar[shift left=.3em]{d}{F\bar{F} m}
	\ar[shift right=.3em]{d}[swap]{\mu\bar{F} \circ F\tau}
	\ar{r}{\mu P}
&
	F P X
	\ar[shift left=.3em]{d}{F m}
	\ar[shift right=.3em]{d}[swap]{\mu}
\\
	P \bar{F}\bar{F} X
	\ar{r}{P\mu}
&
	P \bar{F} X
&%
	F\bar{F} X
	\ar{r}{\mu}
&
	F X
\end{tikzcd}
\end{equation}
while the multiplication $F_TF_TX \to F_TX$
is the induced is the induced map on the vertical coequalizers.
\end{remark}



%\begin{remark}
%When applied to the example of operads, 
%the right coequalizer in \eqref{FRMONDES EQ}
%(together with realization of categories)
%shows that 
%\[F_r X = \mathsf{Lan}_{(T \in \Omega^{0,r})^{op}}
%\left(\bigotimes_{v \in V(T)} X(T_v)\right)
%\]
%where $\Omega^{0,r}$ is the category of trees and degeneracies.
%Similarly, the left coequalizer shows that
%\[
%F_rF_r X = \mathsf{Lan}_{((T_0 \hookrightarrow T_1) \in \Omega^{1,r})^{op}}
%\left(\bigotimes_{v \in V(T_{1})} X(T_{1,v})\right)
%\]
%$\Omega^{0,r}$ is the category whose objects are inner planar face maps $T_0 \hookrightarrow T_1$
%and whose morphisms are degeneracies between them.
%\end{remark}


\begin{remark}
Proposition \ref{PARTADJP PROP} 
admits a complementary result by interchanging the roles of $P$ and $\bar{F}$ and the roles of Kleisli categories and algebra categories.

Explicitly, one has a functor
$\psi_!\colon \Alg_{\bar{F}} \to \Alg_F$
which: sends an $\bar{F}$-algebra $X$
with multiplication $\bar{F} X \xrightarrow{m} X$
to $PX$ with $F$-algebra multiplication given by
$FPX \xrightarrow{\mu} FX=P\bar{F}X \xrightarrow{Pm} PX$;
sends an arrow
$X \xrightarrow{f} Y$
to $PX \xrightarrow{Pf} PY$.

Then in the square below one has that the bottom horizontal maps are adjoint and that
$\iota \psi_! = \psi_!\iota$.
\begin{equation}\label{PARTADJBARF EQ}
\begin{tikzcd}
	\mathsf{Kl}_{\bar{F}} 
	\ar[shift left=0.25em]{r}{\psi_!} 
	\ar[hookrightarrow]{d}[swap]{\iota}
&
	\mathsf{Kl}_{F} 
	\ar[hookrightarrow]{d}{\iota}
\\
	\mathsf{Alg}_{\bar{F}} 
	\ar[shift left=0.25em]{r}{\psi_!} 
&
	\mathsf{Alg}_{F}  
	\ar[shift left=0.25em]{l}{\psi^{\**}}
\end{tikzcd}
\end{equation}
Moreover, in analogy to Remark \ref{MONALGP REM}
one can check that the adjunction monad
$\psi^{\**} \psi_!$ on $\Alg_{\bar{F}}$ has the same underlying functor, unit and multiplication as the monad $P$.
\end{remark}







\section{The homotopy genuine equivariant operad}


Our goal in this section is to build,
for each $G$-$\infty$-operad $X \in \mathsf{dSet}^G$,
the associated homotopy genuine equivariant operad
$\mathsf{ho} (X)$,
which we will describe as an object in
$\mathsf{dSet}_G$
satisfying a strict Segal condition.


We start with some notation. 
Given a multiset $I$ of edges of a tree $T \in \Omega$
(formally, $I$ is a function 
$I \colon \boldsymbol{E}(T) \to \mathbb{N}_0$),
we write $\sigma^I T \in \Omega$
for the tree obtained by degenerating $T$ once for each edge in $I$.
More explicitly, $\sigma^I T$ is the unique tree such that there is a planar degeneracy
$\pi \colon \sigma^I T \to T$
such that $|\pi^{-1}(e)| = I(e) + 1$.
Moreover,
note that if $T\in \Omega_G$ is a $G$-tree, 
then $\sigma^{I} T \in \Omega_{G}$
can be defined if $I$ is $G$-equivariant
(formally, this means that the multiset $I$ is a composite
$\boldsymbol{E}(T) \to \boldsymbol{E}_G(T)
\to \mathbb{N}_0$).

Our main interest will be in degeneracies of $G$-corollas. Recall that, up to isomorphism, 
a $G$-corolla $C \in \Sigma_G$ is determined the number $0 \leq k$ of leaf orbits
and isotropy subgroups
$H_i \leq H_0 \leq G$ for $0 \leq i \leq k$,
where $H_0$ is the isotropy of a (chosen) root edge.
Pictorially, such a $G$-corolla has the orbital representation given on the left below,
but in this section we will find it more convenient to label edge orbits using coset notation as on the right below,
so that $[e_i] = G e_i$ denotes the $G$-orbit of $e_i$.
\[
\begin{tikzpicture}
[grow=up,auto,level distance=2.3em,every node/.style = {font=\footnotesize},dummy/.style={circle,draw,inner sep=0pt,minimum size=1.75mm}]
	\node at (0,0) [font=\normalsize]{$C$}
		child{node [dummy] {}
			child{
			edge from parent node [swap,near end] {$G/H_k$} node [name=Kn] {}}
			child{
			edge from parent node [near end] {$G/H_1$}
node [name=Kone,swap] {}}
		edge from parent node [swap] {$G/H_0$}
		};
		\draw [dotted,thick] (Kone) -- (Kn) ;
	\node at (5,0) [font=\normalsize]{$C$}
		child{node [dummy] {}
			child{
			edge from parent node [swap,near end] {$[e_k]$} node [name=Kn] {}}
			child{
			edge from parent node [near end] {$[e_1]$}
node [name=Kone,swap] {}}
		edge from parent node [swap] {$[e_0]$}
		};
		\draw [dotted,thick] (Kone) -- (Kn) ;
\end{tikzpicture}
\]
We will then abbreviate $\sigma^i C = \sigma^{[e_i]} C$, and write $e_i$, $e_i'$ for the two edges of $\sigma^i C $ that degenerate the edge $e_i$ of $C$,
with $e_i$ denoting the inner edge and $e'_i$ the outer
edge.
\[
\begin{tikzpicture}
[grow=up,auto,level distance=3em,
every node/.style = {font=\footnotesize},
dummy/.style={circle,draw,inner sep=0pt,minimum size=1.75mm}]
	\node at (0,0) [font=\normalsize]{$\sigma^0 C$}
		child{node [dummy] {}
			child{node [dummy] {}
				child{
				edge from parent node [swap,near end] {$[e_k]$} node [name=Kn] {}}
				child{
				edge from parent node [near end] {$[e_1]$}
node [name=Kone,swap] {}}
			edge from parent node [swap] {$[e_0]$}}
		edge from parent node [swap] {$[e'_0]$}
		};
		\draw [dotted,thick] (Kone) -- (Kn) ;
	\node at (5,0) [font=\normalsize]{$\sigma^i C$}
		child{node [dummy] {}
			child{
			edge from parent node [swap,near end] {$[e_k]$} node [near start,inner sep=1pt,name=Kn] {}}
			child[level distance=3.4em]{node [dummy] {}
				child[level distance=2.7em]{
				edge from parent node [swap] {$[e'_i]$}
}
			edge from parent node [near end,swap] {$[e_i]$}
node [near start,inner sep=1pt,name=Kone,swap] {}
node [near start,inner sep=1pt,name=Kone1] {}}
			child{
			edge from parent node [near end] {$[e_1]$}
node [swap] {}
node [near start,inner sep=1pt,name=Kn1,swap]{}}
		edge from parent node [swap] {$[e_0]$}
		};
		\draw [dotted,thick] (Kone) -- (Kn) ;
		\draw [dotted,thick] (Kone1) -- (Kn1) ;
\end{tikzpicture}
\]
$\sigma^i C$ then has an orbital inner face
$\sigma^i C - [e_i]$ obtained by removing $[e_i]$
as well as an orbital outer face obtained by removing $e'_i$,
which we denote $\sigma^i C - [e'_i]$.
Moreover, note that we have natural identifications
$C = \sigma^i C - [e_i]$,
$C = \sigma^i C - [e'_i]$.

In what follows, we will find it convenient to simplify notation by denoting maps $\Omega[T] \to X$,
where $T \in \Omega_G$ and $X \in \mathsf{dSet}^G$,
simply as $T \to X$.


\begin{definition}
	Let $X \in \mathsf{dSet}^G$ be a $G$-$\infty$-operad and $C$ a $G$-corolla with edge orbits
	$[e_0],\cdots,[e_k]$.
	
	Given two operations 
	$f,g\colon C \rightrightarrows X$,
	we write $f \sim_i g$ if there exists a map
	$H \colon \sigma^i C \to X$ such that
\begin{itemize}
\item $f$ equals the restriction $H|_{\sigma^i C-[e'_i]}$;
\item $g$ equals the restriction $H|_{\sigma^i C-[e_i]}$;
\item the restriction $H|_{\sigma^i [e_i]}$
is the degeneracy $\sigma^i [e_i] \to [e_i] \to C \to X$.
\end{itemize}
\end{definition}


\begin{remark}\label{HOMOTBOUND REM}
	Note that if $f \sim_i g$ then it must be
	$f|_{\partial C} = g|_{\partial C}$.
\end{remark}


\begin{example}\label{EQUIVSIM EX}
	Let $G = \mathbb{Z}_{/2} = \{\pm 1\}$
	and consider the $G$-corolla with orbital and expanded representations as given on the left below.
\[
\begin{tikzpicture}
[grow=up,auto,level distance=2.3em,every node/.style = {font=\footnotesize},dummy/.style={circle,draw,inner sep=0pt,minimum size=1.75mm}]
	\node at (0,0) [font=\normalsize]{$C$}
		child{node [dummy] {}
			child{
			edge from parent node [swap] {$G \cdot e$}
node [name=Kone,swap] {}}
		edge from parent node [swap] {$G/G \cdot r$}
		};
	\node at (3,0) [font=\normalsize]{$C$}
		child{node [dummy] {}
			child{
			edge from parent node [swap,near end] {$-e$} node [name=Kn] {}}
			child{
			edge from parent node [near end] {$e$}
node [name=Kone,swap] {}}
		edge from parent node [swap] {$r$}
		};
	\node at (7,0) [font=\normalsize]{$\sigma^{\{e,-e\}} C$}
		child{node [dummy] {}
			child{node [dummy] {}
				child{
				edge from parent node [swap] {$G \cdot e'$}
node [swap] {}}
			edge from parent node [swap] {$G \cdot e$}
node [swap] {}}
		edge from parent node [swap] {$G/G \cdot r$}
		};
	\node at (10,0) [font=\normalsize]{$\sigma^{\{e,-e\}} C$}
		child{node [dummy] {}
			child{node [dummy] {}
				child{
				edge from parent node [swap] {$-e'$} node {}}
			edge from parent node [swap,near end] {$-e$} node {}}
			child{node [dummy] {}
				child{
				edge from parent node {$e'$}
node [swap] {}}
			edge from parent node [near end] {$e$}
node [swap] {}}
		edge from parent node [swap] {$r$}
		};
\end{tikzpicture}
\]
$C$ then has a single leaf $G$-edge orbit $[e] = G \cdot e$, so that for
$f,g \colon C \to X$ it is
$f \sim_1 g$
if there exists a 
$H \colon \sigma^{\{e,-e\}}C \to X$
such that 
\begin{equation}\label{EQUIVHOMOT EQ}
	f = H|_{\sigma^{\{e,-e\}}C - \{e',-e'\}}
\qquad
	g = H|_{\sigma^{\{e,-e\}}C - \{e,-e\}}
\qquad
	H_{\sigma^e e}, H|_{\sigma^{-e}-e} \text{ are degenerate}.
\end{equation}
It is worthwhile to compare this equivariant relation with the relations obtained if one forgets the $G$-actions. Indeed, while \eqref{EQUIVHOMOT EQ} implicitly assumes that all of $f,g,H$ are $G$-equivariant,
by omitting that assumption one can reinterpret 
\eqref{EQUIVHOMOT EQ}
as defining a relation
$f \sim_{[e]} g$ between not necessarily $G$-equivariant maps $f,g \colon C \to X$.

A priori, $\sim_{[e]}$ relation differs from the 
non-equivariant 
$\sim_{e}$ and $\sim_{-e}$
relations obtained by regarding $C$ as a non-equivariant corolla.
However, for $f,g,H$ as in \eqref{EQUIVHOMOT EQ} one has
\begin{equation}\label{EQUIVSIM EQ}
f = H|_{\sigma^{\{e,-e\}}C - \{e',-e'\}}
\sim_e H|_{\sigma^{\{e,-e\}}C - \{e,-e'\}}
\sim_{-e} H|_{\sigma^{\{e,-e\}}C - \{e,-e\}} =g
\end{equation}
so that, by Lemma \ref{EQUIVI LEM}(b) below one has that
$f \sim_{[e]} g$ in fact implies
$f \sim_{e} g$. Moreover, the converse statement follows immediately by using degeneracies.

More generally, similar considerations show that the $\sim$ relations are compatible with restricting the $G$-actions.
\end{example}


\begin{lemma}\label{EQUIVI LEM}
	Let $X \in \mathsf{dSet}^G$ be a $G$-$\infty$-operad and $C$ a $G$-corolla with edge orbits
	$[e_0],\cdots,[e_k]$. Then:
\begin{itemize}
	\item[(a)] each of the relations $\sim_i$ is an equivalence relation;
	\item[(b)] all the equivalence relations $\sim_i$ coincide.
\end{itemize}
\end{lemma}

\begin{proof}
	We first address (a). 
	
	For the reflexive condition, one can take $H$ to be the degeneracy
	$\sigma^i C \xrightarrow{\sigma^i} C \xrightarrow{f} X$.
	
	For the symmetry and transitive conditions, consider the tree
	$\sigma^{ii} C$, which degenerates $[e_i]$ twice.
\[
\begin{tikzpicture}
[grow=up,auto,level distance=3em,
every node/.style = {font=\footnotesize},
dummy/.style={circle,draw,inner sep=0pt,minimum size=1.75mm}]
	\node at (0,0) [font=\normalsize]{$\sigma^{ii} C$}
		child{node [dummy] {}
			child{
			edge from parent node [swap,near end] {$[e_k]$} node [near start,inner sep=1pt,name=Kn] {}}
			child[level distance=3.4em]{node [dummy] {}
				child[level distance=2.7em]{node [dummy] {}
					child[level distance=2.7em]{
					edge from parent node [swap] {$[e''_i]$}
}
				edge from parent node [swap] {$[e'_i]$}
}
			edge from parent node [near end,swap] {$[e_i]$}
node [near start,inner sep=1pt,name=Kone,swap] {}
node [near start,inner sep=1pt,name=Kone1] {}}
			child{
			edge from parent node [near end] {$[e_1]$}
node [swap] {}
node [near start,inner sep=1pt,name=Kn1,swap]{}}
		edge from parent node [swap] {$[e_0]$}
		};
		\draw [dotted,thick] (Kone) -- (Kn) ;
		\draw [dotted,thick] (Kone1) -- (Kn1) ;
\end{tikzpicture}
\]
Suppose $f \sim_i g$, and let 
$H \colon \sigma^{i} C \to X$ be the associated homotopy.
Define a map 
$\bar{H} \colon \Lambda^{[e_i]}_o[\sigma^{ii} C] \to X$ by
\[
	\bar{H}|_{\sigma^{ii}C - [e''_i]} = H,
		\qquad
	\bar{H}|_{\sigma^{ii}C - [e'_i]} = f \circ \sigma^i,
		\qquad
	\bar{H}|_{\sigma^{ii} [e_i]} = 
	f|_{[e_i]} \circ \sigma^{ii} =
	g|_{[e_i]} \circ \sigma^{ii}.
\]
Since the orbital inner horn inclusion
$\bar{H} \colon \Lambda^{[e_i]}_o[\sigma^{ii} C] \to \Omega[C]$
is $G$-inner anodyne,
$\bar{H}$ admits an extension $\widetilde{H} \colon \sigma^{ii}C \to X$.
The restriction $\bar{H}|_{\sigma^{ii}C - [e_i]}$ then provides the homotopy exhibiting $g \sim_i f$, and symmetry of $\sim_i$ follows.

Next, suppose $f \sim_i g$ and $g \sim_i h$, and let 
$H \colon \sigma^{i} C \to X$ and
$K \colon \sigma^{i} C \to X$ be be the associated homotopies.
Define a map 
$\bar{H} \colon \Lambda^{[e'_i]}_o[\sigma^{ii} C] \to X$ by
\[
	\bar{H}|_{\sigma^{ii}C - [e''_i]} = H,
		\qquad
	\bar{H}|_{\sigma^{ii}C - [e_i]} = K,
		\qquad
	\bar{H}|_{\sigma^{ii} [e_i]} = 
	f|_{[e_i]} \circ \sigma^{ii} =
	g|_{[e_i]} \circ \sigma^{ii} =
	h|_{[e_i]} \circ \sigma^{ii}.
\]
$\bar{H}$ again admits an extension $\widetilde{H} \colon \sigma^{ii}C \to X$, and the restriction $\bar{H}|_{\sigma^{ii}C - [e'_i]}$
provides the homotopy exhibiting $f \sim_i g$, so that transitivity of $\sim_i$.

We next turn to (b). Consider the tree $\sigma^{ij} C$ which degenerates $C$ once along each of $[e_i]$ and $[e_j]$.
\[
\begin{tikzpicture}
[grow=up,auto,level distance=2.75em,
every node/.style = {font=\footnotesize},
dummy/.style={circle,draw,inner sep=0pt,minimum size=1.75mm}]
	\node at (0,0) [font=\normalsize]{$\sigma^{ij} C$}
		child{node [dummy] {}
			child{
			edge from parent node [swap,near end] {$[e_k]$} node [near start,inner sep=1pt,name=Kn] {}}
			child[level distance=3.4em,sibling distance=2em]{node [dummy] {}
				child[level distance=2.7em]{
				edge from parent node [swap] {$[e'_j]$}
}
			edge from parent node [very near end,swap] {$[e_j]$}
node [near start,inner sep=1pt,name=Kone,swap] {}
node [inner sep=1pt,name=Kn2] {}}
			child[level distance=3.4em,sibling distance=2em]{node [dummy] {}
				child[level distance=2.7em]{
				edge from parent node {$[e'_i]$}
}
			edge from parent node [very near end] {$[e_i]$}
node [inner sep=1pt,name=Kone2,swap] {}
node [near start,inner sep=1pt,name=Kone1] {}}
			child{
			edge from parent node [near end] {$[e_1]$}
node [swap] {}
node [near start,inner sep=1pt,name=Kn1,swap]{}}
		edge from parent node [swap] {$[e_0]$}
		};
		\draw [dotted,thick] (Kn) -- (Kone) ;
		\draw [dotted,thick] (Kone1) -- (Kn1) ;
		\draw [dotted,thick] (Kone2) -- (Kn2) ;
\end{tikzpicture}
\]
Suppose $f \sim_i g$ with $H \colon \sigma^{i} C \to X$ the associated homotopy.
Define a map 
$\bar{H} \colon \Lambda^{[e_i]}_o[\sigma^{ij} C] \to X$ by
\[
	\bar{H}|_{\sigma^{ij}C - [e'_j]} = H,
		\qquad
	\bar{H}|_{\sigma^{ij}C - [e_j]} = f \circ \sigma^i,
		\qquad
	\bar{H}|_{\sigma^{ij}C - [e'_i]} = f \circ \sigma^j.
\]
Yet again, $\bar{H}$ admits an extension $\widetilde{H} \colon \sigma^{ij}C \to X$, and the restriction $\bar{H}|_{\sigma^{ij}C - [e_i]}$
provides a homotopy exhibiting $g \sim_j f$. (b) now follows.
\end{proof}

In light of Lemma \ref{EQUIVI LEM},
given $f,g \rightrightarrows C \to X$ with 
$C$ a $G$-corolla and $X$ a $G$-$\infty$-operad,
we will henceforth write $f \sim g$ whenever $f \sim_i g$ for some (and thus all) $i$.
We now extend the $\sim$ relation.

\begin{definition}\label{XTENDSIM DEF}
	Let $T \in \Omega_G$ be a $G$-tree
	and $X \in \mathsf{dSet}^G$ be a 
	$G$-$\infty$-operad.
	
	Given dendrices $x,y\colon T \to X$ we write
	$x \sim y$ if there are equivalences of restrictions
	$x|_{T_v} \sim y|_{T_v}$ for all $G$-vertices
	$v \in \boldsymbol{V}_G(T)$.
	
	Further, we define $\mathsf{ho}(X)(T) = X(T)/\sim$.
\end{definition}

\begin{proposition}
Let $X \in \mathsf{dSet}^G$ be a $G$-$\infty$-operad. Then the assignment 
		$T \mapsto \mathsf{ho}(X)(T)$
		is a contravariant functor in $T \in \Omega_G$, i.e.
		$\mathsf{ho}(X)\in \mathsf{dSet}_G$.
\end{proposition}


\begin{proof}
	It suffices to show that the $\sim$ equivalence relations are compatible with the generating classes of maps in $\Omega_G$, namely
	degeneracies, inner faces, outer faces, and quotient maps.
	
	The cases of degeneracies and outer faces are obvious. In the case of quotients, 
	since any quotient $\bar{T} \to T$ of $G$-trees induces quotients on $G$-vertices, it suffices to consider the case of a quotient
	$\bar{C} \xrightarrow{\pi} C$ of $G$-corollas.
	But it is then straightforward to check that if a homotopy exhibiting $f \sim_0 g$ also induces a homotopy exhibiting 
	$f \circ \pi \sim_0 g \circ \pi$
	(notably, the same needs not be true for the relations $f \sim_i g$ when $0<i$, 
	in which case the exhibiting homotopy 
	may instead exhibit a string of relations 
	$f \circ \pi \sim \cdots \sim g \circ \pi$
	as in \eqref{EQUIVSIM EQ}).

It remains to address the most interesting case,
that inner faces. Since inner faces can be factored as composites of inner faces that collapse a singe inner edge orbit,
it suffices to consider the case of faces
$D \to T$ where $T$ has a single edge edge orbit.
I.e. we can assume that there are $G$-corollas
$C_1$, $C_2$ such that 
$T = C_1 \amalg_{[e_i]} C_2$ and
$D = T - [e_i]$, as illustrated below.
\[
\begin{tikzpicture}
[grow=up,auto,level distance=3em,
every node/.style = {font=\footnotesize},
dummy/.style={circle,draw,inner sep=0pt,minimum size=1.75mm}]
	\node at (0,0) [font=\normalsize]{$C_1$}
		child{node [dummy] {}
			child{
			edge from parent node [swap,near end] {} node [near start,inner sep=1pt,name=Kn] {}}
			child[level distance=3.4em]{node {}
			edge from parent node [near end,swap] {$[e_i]$}
node [near start,inner sep=1pt,name=Kone,swap] {}
node [near start,inner sep=1pt,name=Kone1] {}}
			child{
			edge from parent node [near end] {}
node [swap] {}
node [near start,inner sep=1pt,name=Kn1,swap]{}}
		edge from parent node [swap] {$[e_0]$}
		};
		\draw [dotted,thick] (Kone) -- (Kn) ;
		\draw [dotted,thick] (Kone1) -- (Kn1) ;
	\node at (4,0) [font=\normalsize]{$C_2$}
		child{node [dummy] {}
			child{
			edge from parent node [swap,near end] {} node [name=Kn] {}}
			child{
			edge from parent node [near end] {}
node [name=Kone,swap] {}}
		edge from parent node [swap] {$[e_i]$}
		};
		\draw [dotted,thick] (Kone) -- (Kn) ;
	\node at (9,0) [font=\normalsize]{$T$}
		child{node [dummy] {}
			child{
			edge from parent node [swap,near end] {} node [near start,inner sep=1pt,name=Kn] {}}
			child[level distance=3.4em]{node [dummy] {}
				child{
				edge from parent node [swap,near end] {} node [name=Kn2] {}}
				child{
				edge from parent node [near end] {}
node [name=Kone2,swap] {}}
			edge from parent node [near end,swap] {$[e_i]$}
node [near start,inner sep=1pt,name=Kone,swap] {}
node [near start,inner sep=1pt,name=Kone1] {}}
			child{
			edge from parent node [near end] {}
node [swap] {}
node [near start,inner sep=1pt,name=Kn1,swap]{}}
		edge from parent node [swap] {$[e_0]$}
		};
		\draw [dotted,thick] (Kone) -- (Kn) ;
		\draw [dotted,thick] (Kone1) -- (Kn1) ;
		\draw [dotted,thick] (Kone2) -- (Kn2) ;
\end{tikzpicture}
\]
The claim is now that if
$x,y \colon T \to X$ are such that
$x|_{C_1} \sim y|_{C_1}$ and
$x|_{C_2} \sim y|_{C_2}$
then it is also 
$x|_{D} \sim y|_{D}$.
This will follow from the next two claims:
\begin{itemize}
\item[(i)] if $x,y \colon T \to X$ are such that
$x|_{C_1} = y|_{C_1}$ and
$x|_{C_2} = y|_{C_2}$
then $x|_{D} \sim y|_{D}$;
\item[(ii)]
given $x \colon T \to X$, $f\colon C_1 \to X$ and
$g \colon C_2 \to X$ such that
$f \sim x|_{C_1}$, $g \sim x|_{C_2}$,
there exists
$y \colon T \to X$ such that
$y|_{C_1} = f$, $y|_{C_2} = g$ and
$y|_D = x|_D$.
\end{itemize}
To show (i) and (ii), consider the degeneracies
$\sigma^0 T$ and $\sigma^i T$ pictured below.
\[
\begin{tikzpicture}
[grow=up,auto,level distance=3em,
every node/.style = {font=\footnotesize},
dummy/.style={circle,draw,inner sep=0pt,minimum size=1.75mm}]
	\node at (0,0) [font=\normalsize]{$\sigma^0 T$}
		child{node [dummy] {}
		child{node [dummy] {}
			child{
			edge from parent node [swap,near end] {} node [near start,inner sep=1pt,name=Kn] {}}
			child[level distance=3.4em]{node [dummy] {}
				child{
				edge from parent node [swap,near end] {} node [name=Kn2] {}}
				child{
				edge from parent node [near end] {}
node [name=Kone2,swap] {}}
			edge from parent node [near end,swap] {$[e_i]$}
node [near start,inner sep=1pt,name=Kone,swap] {}
node [near start,inner sep=1pt,name=Kone1] {}}
			child{
			edge from parent node [near end] {}
node [swap] {}
node [near start,inner sep=1pt,name=Kn1,swap]{}}
		edge from parent node [swap] {$[e_0]$}}
		edge from parent node [swap] {$[e'_0]$}
		};
		\draw [dotted,thick] (Kone) -- (Kn) ;
		\draw [dotted,thick] (Kone1) -- (Kn1) ;
		\draw [dotted,thick] (Kone2) -- (Kn2) ;
	\node at (6,0) [font=\normalsize]{$\sigma^i T$}
		child{node [dummy] {}
			child{
			edge from parent node [swap,near end] {} node [near start,inner sep=1pt,name=Kn] {}}
			child[level distance=3.4em]{node [dummy] {}
			child{node [dummy] {}
				child{
				edge from parent node [swap,near end] {} node [name=Kn2] {}}
				child{
				edge from parent node [near end] {}
node [name=Kone2,swap] {}}
			edge from parent node [swap] {$[e'_i]$}}
			edge from parent node [near end,swap] {$[e_i]$}
node [near start,inner sep=1pt,name=Kone,swap] {}
node [near start,inner sep=1pt,name=Kone1] {}}
			child{
			edge from parent node [near end] {}
node [swap] {}
node [near start,inner sep=1pt,name=Kn1,swap]{}}
		edge from parent node [swap] {$[e_0]$}
		};
		\draw [dotted,thick] (Kone) -- (Kn) ;
		\draw [dotted,thick] (Kone1) -- (Kn1) ;
		\draw [dotted,thick] (Kone2) -- (Kn2) ;
\end{tikzpicture}
\]
Given $x,y$ as in (i), one can now build a map
$H \colon \Lambda_o^{[e_i]}[\sigma^0 T] \to X$ by
\[
	H|_{\sigma^0 T - [e_0]} = x,
\qquad
	H|_{\sigma^0 T - [e'_0]} = y,
\qquad
	H|_{\sigma^0 C_1} = 
	x|_{C_1} \circ \sigma^0 = 
	y|_{C_1} \circ \sigma^0.
\]
Letting $\widetilde{H}\colon \sigma^0 T \to X$
be an extension of $H$,
the restriction $H|_{\sigma^0 T - [e_i]}$
provides the desired homotopy 
$x|_{D} \sim y|_{D}$, showing (i).


Lastly, let $x,f,g$ be as in (ii), 
and let
$K \colon \sigma^i C_1 \to X$ exhibit the relation
$f \sim_i x|_{C_1}$
and 
$ \bar{K} \colon \sigma^i C_2 \to X$
exhibit the relation
$x|_{C_2} \sim_i g$ (note the reversed order).
Now build the map
$H \colon \Lambda_o^{[e'_i]}[\sigma^i T] \to X$ by
\[
	H|_{\sigma^i T - [e_i]} = x,
\qquad
	H|_{\sigma^i C_1} = K,
\qquad
	H|_{\sigma^i C_2} = \bar{K}.
\]
Again letting 
$\widetilde{H} \colon \sigma^i T \to X$,
the restriction 
$\widetilde{H}|_{\sigma^i T - [e'_i]}$
provides the required $y \colon T \to X$,
showing (ii) and finishing the proof.
\end{proof}


\begin{corollary}\label{HOOPUNIV COR}
Let $X \in \mathsf{dSet}^G$ be a $G$-$\infty$-operad. Then
	\begin{itemize}
	\item[(a)] $\mathsf{ho}(X)\in \mathsf{dSet}_G$ is a genuine equivariant operad, i.e. it satisfies the strict right lifting condition against the Segal core inclusions
	$Sc[T] \to \Omega[T]$ for $T \in \Omega_G$;
	\item[(b)] the quotient map
	$\upsilon_{\**}X \to \mathsf{ho}(X)$ is the universal map from $\upsilon_{\**}X$ to a genuine equivariant operad.
	\end{itemize}
\end{corollary}

\begin{proof}
	Note first that by Remark \ref{HOMOTBOUND REM}
	any map 	$Sc[T] \to \mathsf{ho}(X)$ admits a factorization 
	$Sc[T] \to \upsilon_{\**}X \xrightarrow{q} \mathsf{ho}(X)$.
	
	The right lifting property for $\mathsf{ho}(X)$
	against the maps $Sc[T] \to \Omega[T]$
	is then automatic from the lifting property for $X$.

	For strictness,	
	note that Definition \ref{XTENDSIM DEF}
	can be reinterpreted as saying that
	$x,y \colon \Omega[T] \rightrightarrows X$
	give rise to the same point of 
	$\mathsf{ho}(X)$, i.e. 
	the composites 
	$\Omega[T] \rightrightarrows X \xrightarrow{q}
	\mathsf{ho}(X)$ coincide, 
	iff the composites 
	$Sc[T] \to \Omega[T] \rightrightarrows X \xrightarrow{q}
	\mathsf{ho}(X)$ coincide, showing strictness, and thus (a).
		
	For (b), since $\mathsf{ho}(X)$ is a quotient of
	$\upsilon_{\**} X$, it suffices to show that any map
	from $F \colon \upsilon_{\**}X \to Y$ with $Y$ a genuine equivariant operad must also enforce the $\sim$ relation.
	For a $G$-corolla $C$ and
	$f,g\colon C \to X$ such that 
	$H \colon \sigma^i C \to X$ exhibits
	$f \sim_i g$, 
	the strict lifting condition for $Y$
	shows that the maps
	$F\circ H \colon \sigma^i C \to Y$,
	$f \circ \sigma^i \colon \sigma^i C \to Y$
	must coincide, and thus that
	$F(f)=F(g)$.
	The further claim that $F$ respects equivalences
	of general dendrices $x,y\colon T \rightrightarrows X$
	is immediate from Definition \ref{XTENDSIM DEF}.
\end{proof}



{\color{red} HERE}

The following is the analogue of \cite[Prop. 4.8]{CM13b}.

\begin{proposition}
Let $\mathcal{O} \in \mathsf{sOp}^G$
be a fibrant. 
Then there is a natural isomorphism of genuine equivariant operads
\begin{equation}\label{HOOPID EQ}
ho(hcN(\mathcal{O})) \xrightarrow{\simeq}
\pi_0 \left( \upsilon_{\**} N\mathcal{O} \right).
\end{equation}
\end{proposition}


\begin{proof}
The existence of the map in \eqref{HOOPID EQ}
will be an application of
Corollary \ref{HOOPUNIV COR}(b).

Firstly,
note that we have the following, naturally in $T \in \Omega_G$.
\[
\upsilon_{\**}hcN(\O)(T)
\simeq
\mathsf{sOp}^G(W_!\Omega[T],\O)
\simeq
\mathsf{sdSet}^G(N W_!\Omega[T],N \O)
\simeq 
\mathsf{sdSet}_G(\upsilon_{\**}N W_!\Omega[T],\upsilon_{\**}N \O)
\]
there the second identification uses the fact that 
$N\colon \mathsf{Op} \to \mathsf{dSet}$
is a fully faithful inclusion.
But now one has a map
\begin{align*}
	\mathsf{sdSet}_G(\upsilon_{\**}N W_!\Omega[T],\upsilon_{\**}N \O)
\to  &
	\mathsf{sdSet}_G(\upsilon_{\**}N W_!\Omega[T],\pi_0\upsilon_{\**}N \O)
\simeq
	\mathsf{dSet}_G(\pi_0 \upsilon_{\**}  N W_!\Omega[T],\pi_0\upsilon_{\**}N \O)
\\
\simeq &
	\mathsf{dSet}_G(\pi_0\upsilon_{\**} N W_!\Omega[T],\pi_0\upsilon_{\**}N \O)
	\simeq
	\mathsf{dSet}_G(\upsilon_{\**}\Omega[T]],\pi_0\upsilon_{\**}N \O)
	=
	(\pi_0\upsilon_{\**}N \O)(T)
\end{align*}
so altogether we obtain a map
$\upsilon_{\**}hcN(\O) \to \pi_0 \upsilon_{\**} N \O$
and hence, by Corollary \ref{HOOPUNIV COR},
the desired map 
\[ho(hcN(\O)) \to \pi_0 \upsilon_{\**} N \O\]
Moreover, both of these are quotients of $\upsilon_{\**}hcN(\O)$,
so to prove that the map is an isomorphism one needs only show that any two operations $f,g\colon C \rightrightarrows hcN \O$ of $\upsilon_{\**}hcN(\O)$
that are identified in 
$\pi_0 \upsilon_{\**} N \O$
were already identified in 
$ho(hcN(\O))$.
But this now follows immediately from the pushout from Proposition \ref{WLEFTQPUSH PEOP}
\[
\begin{tikzcd}
	\Omega(C) \otimes_{\mathsf{F}}
	\partial \Delta[1]
	\ar{r} \ar{d}
&
	W_! \left(\partial \Omega[C \star \eta]\right) 
	\ar{d}
\\
	\Omega(C) \otimes_{\mathsf{F}}
	\Delta[1]
	\ar{r}
&
	W(C \star \eta)
\end{tikzcd}
\]
\end{proof}







\section{Semi-model categories}

We recall the definition of a semi-model category, following \cite{Spi01,Wh16}.

\begin{definition}
      Given a class of maps $ I$ in a category $\mathcal D$, we let
      \begin{enumerate}[label = (\roman*)]
      \item \textit{$ I$-inj} denote the class of maps with the right lifting property with respect to $ I$.
      \item \textit{$ I$-cof} denote the class of maps with the left lifting property with respect to $ I$-inj.
      \item \textit{$ I$-cell} denote the class of maps of transfinite compositions of pushouts of maps in $ I$.
      \end{enumerate}

      Now suppose $\mathcal D$ has an initial object.
      Given a second class of maps $ J$, let \textit{$ J$-ccd} denote
      the subclass of $ J$-cell with $ I$-cofibrant domains
      (i.e. $x \to y$ in $ J$-cell is in $ J$-ccd iff the map $\varnothing \to x$ is in $ I$-cof).
\end{definition}

\begin{definition}
      Suppose $\mathcal D$ is a category closed under all small limits and colimits.
      % Given an adjunction $F \colon \mathcal M \rightleftarrows \colon U \mathcal D$ with $\mathcal M$ a model category,
      A \textit{semi-model structure} on $\mathcal D$ consists of
      subcateogries of weak equivalences, cofibrations, and fibrations, such that:
      \begin{enumerate}[label = (\roman*)]
            % \item $U$ preserves fibrations and trivial fibrations.
      \item Weak equivalences are closed under two-out-of-three, and all three chosen subcategories are closed under retracts.
      \item Every map in $\mathcal D$ can be functorially factored into a cofibration followed by a trivial fibration.
            Every map in $\mathcal D$ whose domain is \textbf{cofibrant} in $\mathcal D$ can be functorially factored into a trivial cofibration followed by a fibration.
      \item Cofibrations in $\mathcal D$ have the left lifting property with respect to trivial fibrations.
            Trivial cofibrations in $\mathcal D$ whose domain is \textbf{cofibrant} in $\mathcal D$ have the left lifting property with respect to fibrations.
            % \item The initial object in $\mathcal D$ is cofibrant in $\mathcal D$.
      \item Fibrations and trivial fibrations are closed under pullback.
      \end{enumerate}

      A \textit{semi-model category} is such a category $\mathcal D$ equipped with a semi-model structure.

      A semi-model category $\mathcal D$ is \textit{cofibrantly generated} if there exist sets of maps $ I$, $ J$ such that
      \begin{enumerate}[label = (\roman*)]
      \item $ I$-inj is the class of trivial cofibrations, and $ J$-inj is the class of fibrations; and
      \item the domains of $ I$ are small relative to $ I$-cell, and
      \item the domains of $ J$ are small relative to $ J$-ccd.
      \end{enumerate}
\end{definition}

\begin{remark}
      In \cite{Spi01,Wh16}, these semi-model structures come equipped with an adjunction $\mathcal M \rightleftarrows \mathcal D$ with $\mathcal M$ a model category,
      with the added condition that $U$ preserves (trivial) fibrations.
      Our definition is a special case, where $\mathcal M$ is the terminal category with its unique model strucutre,
      and the left adjoint is the inclusion of the initial object.

      In all of the examples from White and Spitzweck, these adjunctions actually induce the semi-model structure on $\mathcal D$;
      however, this extra data is not necessary to generalize much of \cite{Hov99} to this context.
\end{remark}

The following is immediate from the Retract Argument \cite[Lemma 1.1.9]{Hov99}.
\begin{lemma}
      A map is a trivial fibration (resp. cofibration) iff it has the right (resp. left) lifting property with respect to cofibrations (resp. trivial fibrations).
      
      A map with cofibrant source is a fibration (resp. trivial cofibration) iff it has the right (resp. left) lifting property with respect to trivial cofibrations (resp. fibrations).
\end{lemma}

The existence result \cite[Theorem 2.1.19]{Hov99} immediately generalizes to the semi-model categorical context,
as the (proof of the) small object argument \cite[Lemma 2.1.14]{Hov99} implies that
if the domains of $J$ are small relative to $J$-ccd,
then any map in $\mathcal D$ with cofibrant source may be factored into a map in $J$-ccd followed by a map in $J$-inj.
\begin{theorem}
      \label{SEMIMS_THM}
      Suppose $\mathcal D$ is a category with all small limits and colimits.
      Suppose $\mathcal W$ is a subcategory of $\mathcal D$, and $I$ and $J$ are sets of maps of $\mathcal D$.
      Then there is a cofibrantly generated semi-model structure on $\mathcal D$ with
      $I$ (resp. $J$) the set of generating (trivial) cofibrations, and weak equivalences $\mathcal W$ iff:
      \begin{enumerate}[label = (\roman*)]
      \item $\mathcal W$ is closed under two-out-of-three and retracts.
      \item The domains of $I$ (resp. $J$) are small relative to $I$-cell (resp. $J$-ccd).
      \item $J\text{-ccd} \subseteq \mathcal W \cap I\text{-cof}$.
      \item $I$-inj $\subseteq$ $\mathcal W \cap J$-inj.
      \item Either $\mathcal W \cap I\text{-cof} \subseteq J\textit{-cof}$ or $\mathcal W \cap J\text{-inj} \subseteq I\text{-inj}.$.
      \end{enumerate}
\end{theorem}







\newpage
\bibliography{biblio3}{} % biblio-new
\bibliographystyle{amsalpha2}



\end{document}


%%% Local Variables:
%%% mode: latex
%%% TeX-master: t
%%% End:
