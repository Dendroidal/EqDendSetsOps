\documentclass[a4paper,10pt
,draft
%, final
]{article}%

\pdfcompresslevel=0
\pdfobjcompresslevel=0

\usepackage[hidelinks]{hyperref}
\hypersetup{
  % colorlinks,
  final,
  pdftitle={Equivariant simplicial operads and dendroidal sets},
  pdfauthor={Bonventre, P. and Pereira, L. A.},
  % pdfsubject={Your subject here},
  % pdfkeywords={keyword1, keyword2},
  linktoc=page
}
%\usepackage[open=false]{bookmark}

\input{commands.tex}%


%-------- Tikz ---------------------------

\usepackage{tikz}%
\usetikzlibrary{matrix,arrows,decorations.pathmorphing,
cd,patterns,calc}
\tikzset{%
  treenode/.style = {shape=rectangle, rounded corners,%
                     draw, align=center,%
                     top color=white, bottom color=blue!20},%
  root/.style     = {treenode, font=\Large, bottom color=red!30},%
  env/.style      = {treenode, font=\ttfamily\normalsize},%
  dummy/.style    = {circle,draw,inner sep=0pt,minimum size=2mm}%
}%

\usetikzlibrary[decorations.pathreplacing]



% ---- Commands on draft --------

\usepackage{ifdraft}
\ifdraft{
%  \color[RGB]{63,63,63}
%   \pagecolor[rgb]{0.5,0.5,0.5}
%  \pagecolor[RGB]{220,220,204}
%  \color[rgb]{1,1,1}
  \usepackage{showkeys}
}
{
  \usepackage[notref]{showkeys}
}


\usepackage{todonotes}%[obeyDraft]


% ----- Labels Changed? --------

\makeatletter

\def\@testdef #1#2#3{%
  \def\reserved@a{#3}\expandafter \ifx \csname #1@#2\endcsname
  \reserved@a  \else
  \typeout{^^Jlabel #2 changed:^^J%
    \meaning\reserved@a^^J%
    \expandafter\meaning\csname #1@#2\endcsname^^J}%
  \@tempswatrue \fi}

\makeatother


% ---- Commands --------

% new symbols

\newcommand{\mycircled}[2][none]{%
  \mathbin{
    \tikz[baseline=(a.base)]\node[draw,circle,inner sep=-1.5pt, outer sep=0pt,fill=#1](a){\ensuremath #2\strut};
cf.  }
}
\newcommand{\owr}{\mycircled{\wr}}

% replace symbols

\renewcommand{\hat}{\widehat}

% random

\renewcommand{\F}{\mathcal F}
\newcommand{\Q}{\mathcal Q}

\newcommand{\lltimes}{\underline{\ltimes}}


% detecting $\V$-categories:

\newcommand{\I}{\mathbb I}
\newcommand{\J}{\mathbb J}
\renewcommand{\1}{\eta}%{\ensuremath{\mathbb{id}}}

% lazy shortcuts

\newcommand{\SC}{\Sigma_{\mathfrak C}}
\newcommand{\OC}{\Omega_{\mathfrak C}}

\newcommand{\UV}{\underline{\mathcal V}}
\newcommand{\UC}{\underline{\mathfrak C}}



\usepackage{harpoon}
\newcommand{\vect}[1]{\overrightharp{\ensuremath{#1}}}


% ---- Title --------

\title{Equivariant Segal operads, simplicial operads, and dendroidal sets}

\author{Peter Bonventre, Lu\'is A. Pereira}%

\date{\today}


% ---- Document body --------

\begin{document}

\maketitle

\begin{abstract}
      Things and stuff
\end{abstract}

\tableofcontents

\vskip 10pt

All functors below are right adjoints.
\[
	\begin{tikzcd}
		\mathsf{PreOp}^G & 
		\mathsf{sOper}^G \ar[dashed]{l}[swap]{N_d}
		\ar[dashed]{d}{hcN_d}
\\
		\mathsf{sdSet}^G \ar{r}[swap]{(-)_0} \ar{u}{\gamma_{\**}} &
		\mathsf{dSet}^G
	\end{tikzcd}
\]

What we need:

\begin{itemize}
\item set up Grothendieck description of $\mathsf{sOper}^G$, and build fiberwise model structures
\item combine into overall model structure on $\mathsf{sOper}^G$
\item prove that $(W_!,hcN_d)$ is a Quillen adjunction (Proposition \ref{W!_COF_PROP})
\item establish tame model structure and prove that $(W,N_d)$ is a Quillen equivalence (Proposition \ref{PREQUIEQUIV PROP})
\item combine everything by showing that the square commutes up to homotopy (Proposition \ref{COMUOTOHOM PROP})
\end{itemize}



\iffalse%



\section{Introduction}



\subsection{Main Results}

\begin{theorem}
      \label{THM1_C}
      Let $(\V,\otimes)$ denote either $(\sSet, \times)$ or $(\sSet_{\**} \wedge)$,
      and fix a $G$-set $\mathfrak C$.
      Then there exist model structures on $\Op^{G, \mathfrak C}(\V)$ such that
      $\O \to \O'$ is a weak equivalence (resp. fibration) if the maps
      \begin{equation}
            \label{THM1_C_EQ}
            \O(\xi)^\Gamma \to \O'(\xi)^\Gamma
      \end{equation}
      are weak equivalences (resp. fibrations) in $\V$ for all
      $\mathfrak C$-signatures $\xi$ and
      graph subgroups $\Gamma \leq \Stab(\xi)$.

      More generally, for $\F = \set{\F_n}_{n \geq 0}$ an \textit{arbitrary} $(G, \Sigma)$-family,
      there exists a model category structure on $\Op^{G, \mathfrak C}(\V)$, which we denote $\Op^{G, \mathfrak C}_\F(\V)$,
      with weak equivalences (resp. fibrations) determined by \eqref{THM1_C_EQ} for $\Gamma \leq (\F_n)_\ksi = \F_n \cap \Aut(\ksi)$.

      Lastly, analogous semi-model category structures $\Op^{G, \mathfrak C}(\V)$, $\Op^{G, \mathfrak C}_\F(\V)$ exist provided that
      $(\V, \otimes)$:
      \begin{enumerate*}[label = (\roman*)]
      \item is a cofibrantly generated model category;
      \item is a closed monoidal model category with cofibrant unit;
      \item has cellular fixed points;
      \item has cofibranty symmetric pushout powers.
      \end{enumerate*}
\end{theorem}

\begin{remark}
      As $\Cat^{G, \mathfrak C}(\V) = \Op^{G, \mathfrak C}(\V) \downarrow \**$,
      this produces $G$-model structures on $\V$-enriched categories with a single set of objects.
\end{remark}



\todo[inline]{come back}

Dwyer-Kan weak equiavlences were introduced in the context of simplicial categories by Dwyer-Kan and Bergner \cite{DK80, Ber07b}.
A simplicial functor $F: \mathcal C \to \mathcal D$ between simplicial categories is called a Dwyer-Kan equivalence if
it is ``homotopically'' fully-faithful and essentially surjective:
each morphism of mapping spaces $\mathcal C(x, y) \to \mathcal D(F(x), F(y))$ is a Kan-equivalence of simplicial sets, and
the functor of 1-categories $\pi_0F: \pi_0\mathcal C \to \pi_0 \mathcal D$ is essentially surjective.

Equivariantly, we require that the morphism of mapping $G$-spaces to be a genuine $G$-Kan equivalence, and
each $\pi_0F^H$ essentially surjective.

Generalizing to (single-colored) operads,
each $\O(n)$ is a $G \times \Sigma_n$-object, and we can ask that the maps $\O(n) \to \P(n)$ are some sort of $G \times \Sigma_n$-equivalence,
with perhaps the most useful being the ``graph subgroup equivalences'' (see, e.g. \cite{BP_geo}, \cite{BH15}),
while the notion of essential surjectivity is vacuous.
Further generalizing to multi-colored operads (or multicategories),
we have a set $\mathfrak C$ of ``colors'' (or objects),
and a mapping space $\O(C)$ for each ``signature'' $C = (c_1, \dots, c_n; c_0) \in \mathfrak C^{\times n} \times \mathfrak C$,
such that each $\O(C)$ has an action by the subgroup
\[
      \Aut(C) = \Aut_{G \ltimes \Sigma_{\mathfrak C}}(C) = \Stab_{G \times \Sigma_n}(C) \leq G \times \Sigma_n.
\]
%\Aut(C) \leq G \times \Sigma_n$.
We can ask that the maps $\O(C) \to \P(F(C))$ be some sort of $\Aut(C)$-equivalence,
again with perhaps the most useful being the graph equivalences.
Here, homotopical essential surjectivity is defined on the underlying categories of path components.

\begin{theorem}
      \label{INTRO_MODEL_THM}
      Let $(\V, \otimes)$, denote either $(\sSet, \times)$ or $(\sSet_{\**}, \wedge)$.
      Then there exists a cofibrantly generated model structure on the category $\Op^G(\V)$,
      where weak equivalences are ``graph Dwyer-Kan'' equivalences (see Definition \ref{DK_MODEL_DEF}). 

      More generally, for any $(G, \Sigma)$-family $\F = \set{\F_n}$ with units,
      there exists an ``$\F$-Dwyer-Kan'' model structure on $\Op^G(\V)$.
      weak $\F$-equivalences, $\F$-fibrations, and $\F$-cofibrations defines as in Definition \ref{MODEL_DEFN}.
           
      Moreover, analogous semi-model category structures $\Op^G_\F(\V)$ exist
      provided that $(\V, \otimes)$:
      \begin{enumerate}[label = (\roman*)]\itemsep-4pt
      \item is a cofibrantly generated model category,
      \item is a closed monoidal model category with cofibrant unit
            \footnote{Cofibrant unit also needed for \ref{J-CELL_PROP}.},
      \item has cellular fixed-point functors,
      \item \label{I_CSPP_LBL} has cofibrant symmetric pushout powers  (Defn. \ref{CSPP_DEF}),
            \footnote{Also needed for Props \ref{CAV_4.14_PROP2}, \ref{J-CELL_PROP}}, % \ref{LOCAL_COF_LEM}            
            % --------------------
      \item \label{I_RP_LBL} is right proper
            \footnote{Needed for Lemma \ref{RIGHTPROPER_LEM} and Lemma \ref{2OUTOF3_PROP}.},
      \item \label{I_GENSET_LBL} has a set $\mathbb{G}$ of generating $\V$-intervals
            \footnote{Needed so we have a \textit{set} of generating trivial cofibrations},
      \item satisfies the coherence axiom.
      \end{enumerate}
\end{theorem}

This result is proved (in stages) in \S \ref{MS_SEC}.

There is a second possible notion of ``homotopically essentially surjective'':
we say that a map $F: \O \to \P$ is essentially surjective if any object in $\P$ is ``equivalent'' to an object in the image of $F$,
where equivalence is defined used a cofibrant replacement of the $\V$-category detecting isomorphisms (see Definition \ref{PL_ES_DEFN} and Remark \ref{ESS_SUR_REM}).
In the case where $\V$ does not satisfy the coherence axiom, then this notion is potentially \textit{stronger} than the notion used above.
This case is covered in Theorem \ref{MODEL_THM}.


\todo[inline]{come back}

\begin{theorem}
      Tame model structure exists, and we have Quillen equivalences
      $\mathsf{PreOp}^G \leftrightarrows \mathsf{PreOp}^G_{tame} \rightleftarrows \sOp^G$.
\end{theorem}

\begin{theorem}
      \label{QE_THM}
      The adjunction $\dSet^G \rightleftarrows \sOp^G$ is a Quillen equivalence.
\end{theorem}

\section{Preliminaries}

\subsection{Wreath products and Grothendieck fibrations}

% -------------------- Types of Grothendieck fibrations --------------------

% \begin{remark}
%       \label{GRTH_RMK}
%       Given a functor
%       \begin{equation}
%             \mathcal B^{op} \longrightarrow \Cat,
%             \qquad
%             b \mapsto \mathcal C_b
%       \end{equation}
%       there are four possible ``Grothendieck constructions''.
%       Two are fibrations over $\mathcal B$, two over $\mathcal B^{op}$.
%       They all have as objects pairs $(b \in \mathcal B, X \in \mathcal C_b)$ and arrows pairs of maps,
%       but the directionality of these maps is different:
%       \begin{enumerate}[label = (\roman*)]
%       \item The \textit{standard} Grothendieck construction $\mathcal B \ltimes \mathcal C$
%             is the fibration over $\mathcal B$ with as arrows pairs of maps $(f, \phi)$ with
%             $f: b \to \bar b$ and
%             $\phi: X \to f^{\**} \bar X$.
%       \item $(\mathcal B \ltimes \mathcal C)^{\underline{op}}$
%             is the fibration over $\mathcal B$ with as arrows pairs of maps $(f, \phi)$ with
%             $f: b \to \bar b$ and
%             $\phi: f^{\**} \bar X \to X$.
%       \item $(\mathcal B \ltimes \mathcal C)^{op}$
%             is the fibration over $\mathcal B^{op}$ with as arrows pairs of maps $(f, \phi)$ with
%             $f: \bar b \to b$ and
%             $\phi: \bar X \to f^{\**} X$.
%       \item $\mathcal B \underline{\ltimes} \mathcal C = \left((\mathcal B \ltimes \mathcal C)^{\underline{op}}\right)^{op}$
%             is the fibration over $\mathcal B^{op}$ with as arrows pairs of maps $(f, \phi)$ with
%             $f: \bar b \to b$ and 
%             $\phi: f^{\**}X \to \bar X$.
%       \end{enumerate}
      
%       Unless otherwise specified, \textit{the} Grothendieck construction will refer to the first one.
% \end{remark}
\begin{notation}
      Given a functor $\mathcal C_{(-)}: \mathcal B \to \Cat$, $b \mapsto \mathcal C_b$, we let
      $\mathcal B \ltimes \mathcal C_{(-)}$ denote the (covariant) \textit{Grothendieck construction},
      with objects pairs $(b,X)$ with $b \in \mathcal B$ and $X \in \mathcal C_b$, and
      maps pairs $(f,g)$ with $f: b \to \bar b$ and $g: f_{\**}X \to \bar X$.
\end{notation}



% -------------------- Translation Categories --------------------

\begin{example}
      \label{G_GR_REM}
      Given a group $G$, let $BG$ denote the category which encodes \textit{left} $G$-actions (so $h \circ g = h g$).
      For any left $G$-set $A$, the \textit{translation category} or \textit{action groupoid}
      has object set $A$
      and morphism set all pairs $(g,a): a \to g.a$ for all $(g,a) \in G \times A$
      (equivalently, $\Hom(a,b) = \sets{g \in G}{b = g.a}$).
      %
      More generally, if $\mathcal C$ is a left $G$-category, the \textit{translation category}
      has object set $\mathrm{Ob}(\mathcal C)$
      and morphism set all tuples $(g,a,b,f)$ with $g \in G$, $a,b\in \mathcal C$, and $f: g.a \to b$.
      %
      These are both isomorphic to the Grothendieck construction $BG \ltimes \mathcal C$
      on the functor $BG \to \Cat$ defining the $G$-action on $\mathcal C$.
      We will mildly abuse notation and denote these categories by simply $G \ltimes \mathcal C$
      \footnote{
        The action groupoid for a $G$-set $A$ is often denoted $B_A G$, generalizing the notation for the category $B G = B_{\**} G$.}.
      
      Further, we note that if $\mathcal C$ has an action by other groups $\Sigma$ which commute with the action by $G$,
      then the iterated Grothendieck constructions above are associative and commutative.
\end{example}

\begin{remark}
      \label{GOP_REM}
      If $\mathcal C$ is a left $G$-category, then we write
      $G^{op} \ltimes \mathcal C$ to denote the Grothendieck construction on the functor
      $BG^{op} \xrightarrow{(-)^{-1}} BG \xrightarrow{\mathcal C} \Cat$.

      Moreover, as $\mathcal C^{op}$ is also a left $G$-category, it is easy to check that
      $(G \ltimes \mathcal C)^{op} \simeq G^{op} \ltimes \mathcal C^{op}$,
      using the above convention.
\end{remark}



% -------------------- OTHER STUFF --------------------


% First, we record a basic result.

% \begin{lemma}
%       \label{PB_GR_LEM}
%       If $\mathcal C \to \mathcal D$ is a Grothendieck fibration, then the pullback
%       \begin{equation}
%             \begin{tikzcd}
%                   \mathcal A \arrow[d] \arrow[r]
%                   &
%                   \mathcal C \arrow[d]
%                   \\
%                   \mathcal B \arrow[r, "F"]
%                   &
%                   \mathcal D
%             \end{tikzcd}
%       \end{equation}
%       is isomorphic to the Grothendieck construction on
%       \begin{equation}
%             \label{PB_GR_EQ}
%             % \begin{tikzcd}[row sep = tiny]
%             %       \mathcal B^{op} \arrow[r]
%             %       &
%             %       \mathsf{Cat}
%             %       \\
%             %       b \arrow[r, mapsto]
%             %       &
%             %       \mathcal C_{F(b)},
%             % \end{tikzcd}
%             \mathcal B^{op} \longto \Cat,
%             \qquad \qquad
%             b \longmapsto \mathcal C_{F(b)},
%       \end{equation}
%       where $\mathcal C_{d}$ is the fiber over $d \in \mathcal D$.
% \end{lemma}
% \begin{proof}
%       The fact that the map \eqref{PB_GR_EQ} is a functor follows from $\mathcal C \to \mathcal D$ being a fibration;
%       the rest follows by unpacking definitions.
% \end{proof}



% -------------------- Wreath Products --------------------

Now, recall the notation $\mathsf F \wr \mathcal C$ for a category $\mathcal C$ from \cite{BP_geo}.

\begin{notation}
      \label{F_WR_NOT}
      We let $\mathsf F$ denote a (fixed) full subcategory of \textit{ordered finite sets} and set maps,
      such that the only ordered isomorphisms are the identity.
      
      Given a category $\mathcal C$, let $\mathsf F \wr \mathcal C$ denote the contravariant Grothendieck construction
      $(\mathsf F^{op} \ltimes \mathcal C^{\times (-)})^{op}$ on the functor
      \begin{equation}
            \mathsf F^{op} \longto \Cat,
            \qquad \qquad
            A \mapsto \mathcal C^{\times A}.
      \end{equation}
      Explicitly, objects are tuples of elements of $\mathcal C$, and maps are composites of ``shuffles'' and tuples of maps in $\mathcal C$.
\end{notation}

\begin{remark}
      \label{WR_DIAG_REM}
      We observe that we have a natural diagonal map
      % \begin{equation}
      $
      \mathsf F \times \mathcal C \into \mathsf F \wr \mathcal C,
      $
      % \end{equation}
      for any category $\mathcal C$,
      and thus for any functor $F: \mathcal D \to \mathsf F$, we have an induced functor
      $F: \mathcal D \times \mathcal C \to \mathsf F \wr \mathcal C$.

      More generally, for any $G$-category $\mathcal C$ we have a natural ``diagonal'' map
      \begin{equation} 
            G \ltimes (\mathsf F \wr \mathcal C) \to \mathsf F \wr (G \ltimes \mathcal C).
      \end{equation}
\end{remark}

\begin{definition}[{cf. \cite[Defn 4.3]{BP_geo}}]
      Let $\mathsf{WSpan}^l(\mathcal C, \mathcal D)$ (resp. $\mathsf{WSpan}^r(\mathcal C, \mathcal D)$)
      denote the category of \textit{left (resp. right) weak spans}, with objects
      \begin{equation}
            \mathcal C \xleftarrow{k} \mathcal A \xrightarrow{X} \mathcal D
      \end{equation}
      and arrows those diagrams as on the left (resp. right) below
      \begin{equation}
            \begin{tikzcd}[row sep = tiny]
                  & \mathcal A_1 \arrow[dr, "X_1", ""'{name=U}] \arrow[dl, "k_1"'] \arrow[dd, "i"']
                  &
                  &&
                  &
                  \mathcal A_1 \arrow[dr, "X_1", ""'{name=A}] \arrow[dl, "k_1"'] \arrow[dd, "i"']
                  \\
                  \mathcal C
                  &&
                  \mathcal D
                  &&
                  \mathcal C
                  &&
                  \mathcal D
                  \\
                  & |[alias=V]| \mathcal A_2 \arrow[ur, "X_2"'] \arrow[ul, "k_2"]
                  &
                  &&
                  &
                  |[alias=B]| \mathcal A_2 \arrow[ur, "X_2"'] \arrow[ul, "k_2"]
                  \arrow[Rightarrow, from = U, to = V]
                  \arrow[Rightarrow, from = B, to = A]
            \end{tikzcd}
      \end{equation}
      denoted by $(i,\phi): (k_1,X_1) \to (k_2,X_2)$, with composition defined in the natural way.      
\end{definition}



The following lemmas have straightforward proofs.

\begin{lemma}
      \label{GD_PULL_LEM}
      For any functor $\mathcal C \to \mathcal D$ of $G$-categories, the squares below are Cartesian.
      \begin{equation}
            \begin{tikzcd}
                  \mathcal C \arrow[d] \arrow[r]
                  &
                  G \ltimes \mathcal C \arrow[d]
                  &&
                  G \ltimes (\Sigma \wr \mathcal C) \arrow[d] \arrow[r]
                  &
                  \Sigma \wr (G \ltimes \mathcal C) \arrow[d]
                  \\
                  \mathcal D \arrow[r]
                  &
                  G \ltimes \mathcal D
                  &&
                  G \ltimes (\Sigma \wr \mathcal D) \arrow[r]
                  &
                  \Sigma \wr (G \ltimes \mathcal D)
            \end{tikzcd}
      \end{equation}
      Moreover, the square on the left lifts Kan extensions.
\end{lemma}

\begin{lemma}
      \label{GL_PULL_LEM}
      The functor $G \ltimes (-): \Cat^{G} \to \mathsf{Fib}(G)$ preserves pullbacks and coproducts.
\end{lemma}

\begin{lemma}
      \label{GL_GR_LEM}
      $G \ltimes (-)$ preserves Grothendieck fibrations, and in fact the fibers remain constant.
\end{lemma}
\begin{proof}
      A straightforward diagram chase shows that if $f$ is a Cartesian arrow in $\mathcal E$ over $\mathcal B$,
      then for any $g\in G$, $f \circ g$ is a Cartesian arrow in $G \ltimes \mathcal E$ over $G \ltimes \mathcal B$.
      The result follows immediately.
\end{proof}

\begin{lemma}
      \label{GL_RANINIT_LEM}
      $G \ltimes (-)$ preserves Ran-initiality:
      if $\mathcal C \to \mathcal D$ is a functor of $G$-categories over another $G$-category $\mathcal E$
      which is Ran-initial, then so is $G \ltimes \mathcal C \to G \ltimes \mathcal D$ over $G \ltimes \mathcal E$.
\end{lemma}
\begin{proof} 
      This follows from the unique description of any arrow in $G \ltimes \mathcal E$ as an element of $G$ plus an arrow in $\mathcal E$. 
\end{proof}

\begin{remark}
      For $G$-categories $\mathcal C$ and $\mathcal D$, we have equivalences of categories
      \begin{equation}
            \Fun^G(\mathcal C, \mathcal D)
            \simeq \Fun_{\Fib(G)}(G \ltimes \mathcal C, G \ltimes \mathcal D)
            \simeq \Fun_{\Fib(G^{op})}(G^{op} \ltimes \mathcal C, G^{op} \ltimes \mathcal D).
      \end{equation}
\end{remark}



\subsection{2-overcategories and pullback functors}

Include \S 8.1: 2-overcategories and \S 8.3: Pullback functors.

Random lemmas:

\begin{lemma}
      Suppose $F: \mathcal C \to \mathcal D$ is a simple 1-arrow in $\Cat \downarrow^r \mathcal B$.
      Consider the following cube in $\Cat$, where the bottom, top, and right faces are all (strict) pullbacks.
      \begin{equation}
            \begin{tikzcd}[row sep = small, column sep = small]
                  \pi^{\**} \mathcal C \arrow[rr] \arrow[dr] \arrow[dd]
                  &&
                  \mathcal E \arrow[dr, "\pi"] \arrow[dd, equal]
                  \\
                  &
                  \mathcal C \arrow[rr, crossing over]
                  &&
                  \mathcal B \arrow[dd, equal]
                  \\
                  \pi^{\**} \mathcal D \arrow[rr] \arrow[dr]
                  &&
                  \mathcal E \arrow[dr, "\pi"]
                  \\
                  &
                  \mathcal D \arrow[uu, crossing over, leftarrow] \arrow[rr]
                  &&
                  \mathcal B
            \end{tikzcd}
      \end{equation}
      Then the left square is also a pullback.
\end{lemma}
\begin{proof}
      We check that $\pi^{\**}\mathcal C$ has the correct universal property: for any category $Z$, we have
      \begin{align*}
        \Hom(Z,\mathcal C) \times_{\Hom(Z,\mathcal D)} \Hom(Z, \pi^{\**} \mathcal C)
        & =
          \Hom(Z, \mathcal C) \times_{\Hom(Z, \mathcal D)}\left(
          \Hom(Z, \mathcal D) \times_{\Hom(Z, \mathcal B)} \Hom(Z, \mathcal E)
          \right)
          \\
        & =
          \Hom(Z, \mathcal C) \times_{\Hom(Z, \mathcal B)}\Hom(Z, \mathcal E)
          = \Hom(Z, \pi^{\**}\mathcal C).
      \end{align*}
\end{proof}

\begin{corollary}
      \label{PI_GFIB_COR}
      If a simple 1-arrow $F:\mathcal C \to \mathcal D$ in $\Cat \downarrow^r \mathcal B$
      is an underlying Grothendieck fibration,
      then so is $\pi^{\**}F$.
\end{corollary}
\begin{proof}
      As Grothendieck fibrations are preserved by pullbacks, this follows from the previous lemma.
\end{proof}

\begin{lemma}
      \label{RANINIT_PULL_LEM}
      Suppose we have a triangle
      \begin{equation}
            \begin{tikzcd}
                  \mathcal C \arrow[rr, hookrightarrow] \arrow[dr, "p"] \arrow[ddr, bend right, "\delta"', ""{near end, name = V}]
                  &&
                  \mathcal D \arrow[dl, "p"'] \arrow[ddl, bend left, "\delta", ""'{near end, name = B}]
                  \\
                  &
                  |[alias = A]| \mathcal E \arrow[d, "\epsilon"]
                  \\
                  &
                  \mathsf F
                  \arrow[Rightarrow, from = A, to = B, "\Phi"]
                  \arrow[Rightarrow, from = A, to = V, "\Phi"']
            \end{tikzcd}
      \end{equation}
      in $\Cat \downarrow^r \mathsf F$,
      such that $\mathcal C \into \mathcal D$ is an inclusion of a subcategory by a simple 1-arrow.
      If $\mathcal C \to \mathcal D$ is $\Ran$-initial over $\mathcal E$,
      then $\mathcal C_{\mathfrak C} \to \mathcal D_{\mathfrak C}$ is $\Ran$-initial over $\mathcal E_{\mathfrak C}$.
\end{lemma}
\begin{proof}
      This is just a matter of unpacking definitions.
      Fixing some $(e, \epsilon(e) \xrightarrow{t} \mathfrak C)$ in $\mathcal E_{\mathfrak C}$,
      and $\ki = \begin{cases}
            (d, \delta(d) \xrightarrow{r} \mathfrak C)
            \\
            (e \longto p(d))
      \end{cases}$
      in $(e,t) \downarrow \mathcal D_{\mathfrak C}$,
      so in particular the triangle below commutes.
      \begin{equation}
            \begin{tikzcd}[row sep = small,column sep = tiny]
                  \epsilon(e) \arrow[rr] \arrow[ddr]
                  &&
                  \epsilon p(d) \arrow[d, "\Phi"]
                  \\
                  &&
                  \delta(d) \arrow[dl]
                  \\
                  &
                  \mathfrak C
            \end{tikzcd}
      \end{equation}
      We must show that $\left((e,t) \downarrow \mathcal C_{\mathfrak C} \right) \downarrow \ki$ is non-empty and connected.

      First, by hypothesis we know there exists $c \in \mathcal C$, $c \xrightarrow{f} d$, and $e \to p(c)$ such that the obvious triangle commutes.
      Now, the object 
      $\begin{cases}
            (c, \delta(c) \xrightarrow{\delta(f)} \delta(d) \to \mathfrak C)
            \\
            (e \longto p(c)
      \end{cases}$
      in $(e,t) \downarrow \mathcal C_{\mathfrak C}$ clearly maps to $\ki$.

      Second, any such object over $\ki$ must factor this way, and hence the connectedness of $(e \downarrow C) \downarrow (d,r)$
      implies the desired connectedness.
      \begin{equation}
            \begin{tikzcd}
                  &
                  \epsilon(e) \arrow[dr] \arrow[dl]
                  &
                  && %                  
                  &
                  \epsilon(e) \arrow[dr] \arrow[dl] \arrow[dd]
                  \\
                  \epsilon p(c) \arrow[rr, "{\epsilon p (f)}"] \arrow[d, "\Phi"]
                  &&
                  \epsilon p(d) \arrow[d, "\Phi"]
                  && %
                  \epsilon p (c) \arrow[dr, "{\epsilon p (f)}"'] \arrow[dd, "\Phi"'] \arrow[rr, dashed]
                  &&
                  \epsilon p (c') \arrow[dl, "{\epsilon p (f')}"] \arrow[dd, "\Phi"]
                  \\
                  \delta(c) \arrow[rr, "{P\delta(f)}"] \arrow[dr, "{r \circ \delta(f)}"']
                  &&
                  \delta(d) \arrow[dl, "r"]
                  && %
                  &
                  \epsilon p (d) \arrow[dd]
                  \\
                  &
                  \mathfrak C
                  &
                  && %
                  \delta (c) \arrow[dr, "{\delta(f)}"] \arrow[ddr] \arrow[rr, dashed]
                  &&
                  \delta(c') \arrow[dl, "{\delta(f')}"'] \arrow[ddl]
                  \\
                  &&&& %
                  &
                  \delta(d) \arrow[d]
                  \\
                  &&&& %
                  &
                  \mathfrak C
            \end{tikzcd}
      \end{equation}
\end{proof}





\subsection{Fibered adjunctions}



\begin{notation}
Given a Grothendieck fibration $p\colon \mathcal{C} \to \mathcal{E}$, 
two objects $\bar{c},c \in \mathcal{C}$
and an arrow $f \colon p(\bar{c}) \to p(c)$, 
we write $\mathcal{C}_f(\bar{c},c)$ for the subset of maps over $f$, i.e. to the pullback in the following diagram.
\begin{equation}
\begin{tikzcd}
		\mathcal{C}_f\left(\bar{c},c \right) \arrow[d] \arrow[r]
	&
		\mathcal{C}\left(\bar{c},c \right) \arrow[d]
\\
		\{f\} \arrow[r]
	&
		\mathcal{E}\left(p(\bar{c}),p(c)\right)
\end{tikzcd}
\end{equation}
\end{notation}


\begin{remark}
One has a natural decomposition
\begin{equation}
	\mathcal{C}(\bar{c},c) \simeq \coprod_{f \in \mathcal{E}(p(\bar{c}),p(c))} \mathcal{C}_f\left(\bar{c},c \right)
\end{equation}
\end{remark}


\begin{remark}\label{CARTCHAR REM}
An arrow $F\colon \bar{c} \to c$ in a Grothendieck fibration
$p \colon \mathcal{C} \to \mathcal{E}$ is cartesian
precisely if it induces natural isomorphisms
\[
\mathcal{C}_{id_{p(\bar{c})}}
\left(- ,\bar{c} \right)
\xrightarrow{\simeq}
\mathcal{C}_{p(F)}\left(- ,c \right)
\]
(note that this condition is weaker than the condition defining cartesian arrows; the claim being made here is that if cartesian arrows are already known to exist, then they are detected by this condition).
\end{remark}



\begin{definition}
Let 
$p\colon \mathcal{C} \to \mathcal{E}$,
$p\colon \mathcal{D} \to \mathcal{E}$,
be Grothendieck fibrations.
A \emph{fibered adjunction} is an adjunction
\[
L \colon \mathcal{C} \rightleftarrows \mathcal{D} \colon R
\]
where the functors, unit and counit are all fibered, i.e.
$pL=p$, $pR=p$, $p \eta = id_{p}$, $p\eta = id_p$.
\end{definition}


\begin{remark}
A fibered adjunction induces natural isomorphisms
\[
\mathcal{D}_f\left(Lc,d\right)
\simeq
\mathcal{C}_f\left(c,Rd\right)
\]
for each $c\in \mathcal{C}$, $d \in \mathcal{D}$, $f\colon p(c)\to p(d)$. 
\end{remark}



\begin{proposition}
Let $L \colon \mathcal{C} \rightleftarrows \mathcal{D} \colon R$
be an adjunction between Grothendieck fibrations.

If the adjunction is a fibered then 
%the right adjoint
$R$ is 
a fibered functor which preserves cartesian arrows.

Conversely, if the right adjoint $R$ is 
a fibered functor which preserves cartesian arrows, then it is possible to modify the adjunction so that it becomes a fibered adjunction.

\end{proposition}


\begin{proof}
For the first claim, 
letting $F \colon \bar{d} \to d$ be a cartesian arrow, 
the fact that $R(F)$ is again cartesian follows from
Remark \ref{CARTCHAR REM} applied to the composite
\[
\mathcal{C}_{id_{p(\bar{d})}}
	\left(-,R\bar{d}\right)
	\simeq 
\mathcal{D}_{id_{p(\bar{d})}}
	\left(L(-),\bar{d}\right)
	\xrightarrow{\simeq}
\mathcal{D}_{p(F)}\left(L(-),d\right)
	\simeq
\mathcal{C}_{p(F)}\left(-,Rd\right)
\]
For the ``conversely'' claim,
noting that by assumption $pRL = pL$,
one can choose a cartesian natural transformation $\bar{L} \to L$
(i.e. a cartesian arrow in $\mathcal{D}^{\mathcal{C}}$)
over the projection of the adjunction unit
$ p \xrightarrow{p \eta} pRL$
(which is an arrow in $\mathcal{E}^{\mathcal{C}}$).
Moreover, noting that by assumption
$R\bar{L} \to RL$ is again cartesian, we write
$id_{\mathcal{C}} \xrightarrow{\bar{\eta}} R \bar{L} \to RL$
for the natural factorization
as well as $\bar{\epsilon}$
for the composite
$\bar{L}R \to LR \xrightarrow{\epsilon} id_{\mathcal{D}}$.
We claim that $\bar{L},R,\bar{\eta},\bar{\epsilon}$
now form provide fibered adjunction, with the non obvious claim being that this is in fact still an adjunction.
That the composite
$R\xrightarrow{\bar{\eta}R} R\bar{L}R \xrightarrow{R\bar{\epsilon}} R$
is the identity follows since this is 
$R \xrightarrow{\bar{\eta} R} R\bar{L}R \to RLR \xrightarrow{R \epsilon} R$ and thus
$R \xrightarrow{\eta R} RLR \xrightarrow{R \epsilon} R$.
The remaining claim is that the top horizontal composite in the diagram below is the identity, 
\begin{equation}
\begin{tikzcd}
		\bar{L} \arrow[d] \arrow{r}{\bar{L}\bar{\eta}}
	&
		\bar{L} R \bar{L} \arrow[d] \ar{r} \ar{d}
		\arrow[bend left]{rr}{\bar{\epsilon}\bar{L}}
	&
		L R \bar{L} \ar{r}{\epsilon \bar{L}} \ar{d}
	&
		\bar{L} \ar {d}
\\
		L \arrow{r}{L\bar{\eta}}
		\arrow[bend right]{rr}[swap]{L \eta}
	&
		LR\bar{L} \ar{r}
	&
		LRL \ar{r}{\epsilon L}
	&
		L
\end{tikzcd}
\end{equation}
and since 
$\bar{L} \to L$ is cartesian, 
is in fact enough to show that the overall composite $\bar{L} \to L$
is the standard map, which is clear. 
\end{proof}


\begin{example}
Letting $\mathcal{O} \in \mathsf{sOp}$ be a simplicial colored operad
and $K \in \mathsf{sSet}$,
one has a pointwise simplicial cotensoring
$\{K,\O\}_{\mathsf{F}} \in \mathsf{sOp}$
given by
$\{K,\O\}_{\mathsf{F}}(\ksi) = 
\left(\O(\ksi)\right)^K$
for each signature $\ksi$.

$\{K,-\}_{\mathsf{F}}\colon \mathsf{sOp} \to \mathsf{sOp}$
is then a fibered right adjoint which preserves pullback arrows,
and thus the corresponding adjunction is also fibered.
\end{example}





\section{Colored operads}


When working with the category $\mathsf{Op}(\mathcal{V})$
of colored operads it is usually useful to consider the subcategories $\mathsf{Op}^{\mathfrak{C}}(\mathcal{V})$
of those operads with a chosen fixed set of objects
$\mathfrak{C} \in \mathsf{F}$ and those maps which are the identity on the set $\mathfrak{C}$ of objects.
As an example, to build the model structure on $\mathsf{Op}(\mathcal{V})$ one usually starts by building suitable model structures on the subcategories $\mathsf{Op}^{\mathfrak{C}}(\mathcal{V})$.
Moreover, we note that $\mathsf{Op}^{\mathfrak{C}}(\mathcal{V})$ can be regarded as the category of algebras over a monad 
$\mathbb{F}_{\mathfrak{C}}$ on a simpler category
$\mathsf{Sym}^{\mathfrak{C}}(\mathcal{V})$ of symmetric sequences, 
with the model structure on
$\mathsf{Op}^{\mathfrak{C}}(\mathcal{V})$
obtained by transferring a model structure on 
$\mathsf{Sym}^{\mathfrak{C}}(\mathcal{V})$
via the monadic adjunction.

As such, in order to build our desired model structure on the category $\mathsf{Op}(\mathcal{V})^G$ of $G$-equivariant colored operads, one should suitably generalize the discussion in the paragraph above.
However, a little care is needed, since the objects of a $G$-equivariant operad in $\mathsf{Op}(\mathcal{V})^G$ are now a $G$-set $\mathfrak{C} \in \mathsf{F}^G$, meaning that even when describing the objects of 
$\mathsf{Op}(\mathcal{V})^G$ one must consider maps of 
$\mathsf{Op}(\mathcal{V})$ that are not the identity on objects.
For this reason, we find is useful to be able to describe 
$\mathsf{Op}(\mathcal{V})$ 
in a way that does not explicitly mention the color fixed structures
$\mathsf{Op}^{\mathfrak{C}}(\mathcal{V}),
\mathsf{Sym}^{\mathfrak{C}}(\mathcal{V}),
\mathbb{F}_{\mathfrak{C}}$.

Our set-up is as follows: 
there are Grothendieck fibrations 
$\mathsf{Op}(\mathcal{V}) \to \mathsf{F}$
(resp. $\mathsf{Sym}(\mathcal{V}) \to \mathsf{F}$)
such that the fibers are the categories 
$\mathsf{Op}^{\mathfrak{C}}(\mathcal{V})$
(resp. $\mathsf{Sym}^{\mathfrak{C}}(\mathcal{V})$)
as well as a monad $\mathbb{F}$ on 
$\mathsf{Sym}(\mathcal{V})$ which is suitably fibered over $\mathsf{F}$
and such that the category of ``fibered algebras'' is 
$\mathsf{Op}(\mathcal{V})$.

With this set-up, it is then entirely formal to show that 
$\mathsf{Op}^G(\mathcal{V}) \to \mathsf{F}^G$
and
$\mathsf{Sym}^G(\mathcal{V}) \to \mathsf{F}^G$
are Grothendieck fibrations and that 
$\mathsf{Op}^G(\mathcal{V})$ is the category of fibered algebras for $\mathbb{F}^G$.
And, by fibering over a fixed $G$-set $\mathfrak{C} \in \mathbb{F}^G$,
we thus obtain a description of the subcategory 
$\mathsf{Op}^{G,\mathfrak{C}}(\mathcal{V})$
of those $G$-operads with $G$-set of objects $\mathfrak{C}$
as the algebras for a monad $\mathbb{F}^G_{\mathfrak{C}}$
on a suitable category $\mathsf{Sym}^{G,\mathfrak{C}}(\mathcal{V})$
of symmetric sequences.





\subsection{Fibered monads}


\begin{definition}\label{FIBMON DEF}
Given a Grothendieck fibration $p\colon \mathcal{C} \to \mathcal{D}$,
a \textit{fibered monad} is a monad $T\colon \mathcal{C} \to \mathcal{C}$ such that the diagram below commutes
\[
\begin{tikzcd}
\mathcal{C} \ar{rr}{T} \ar{rd}[swap]{p} && \mathcal{C} \ar{dl}{p}
\\
& \mathcal{D}
\end{tikzcd}
\]
and the multiplication 
$\mu \colon TT \Rightarrow T$
and unit $\eta \colon I \Rightarrow T$
satisfy
$p\mu=p\eta=id_{p}$.

Moreover, a \textit{fiber algebra} is a $T$-algebra $c \in \mathcal{C}$
such that the multiplication map
$Tc \xrightarrow{m} c$ satisfies 
$p(m)=id_{\pi(c)}$.

Lastly, we write $\mathsf{Alg}^{p}_T(\mathcal{C}) \subseteq \mathsf{Alg}_T(\mathcal{C})$ for the full subcategory of fiber algebras.
\end{definition}

\begin{remark}
For each $d\in \mathcal{D}$, a fibered monad $T$ restricts to a monad on each fiber $\mathcal{C}_d$, and we write $T_d$ to denote that restricted monad.
\end{remark}


\begin{proposition}
Given a fibered monad on $p\colon \mathcal{C} \to \mathcal{D}$ the projection $\mathsf{Alg}^{p}_T(\mathcal{C}) \to \mathcal{D}$
is again a Grothendieck fibration.
\end{proposition}

\begin{proof}
Given a cartesian arrow $f\colon \bar{c} \to c$ on $\mathcal{C}$ and a fiber algebra structure on $c$, we claim there is a unique fiber algebra structure on $\bar{c}$ making $f$ into an algebra map. Indeed, the properties of cartesian arrows imply that the is a unique way to choose a dashed fiber arrow in the diagram
\[
\begin{tikzcd}
	T \bar{c} \ar{r}{Tf} \ar[dashed]{d} & T c \ar{d}
\\
	\bar{c} \ar{r}[swap]{f} & c.
\end{tikzcd}
\]
The claims that $T\bar{c} \to \bar{c}$ is then an algebra map and that 
$f$ is also cartesian as an algebra map again follow from 
$f$ being cartesian in $\mathcal{C}$.
\end{proof}



\begin{remark}
Suppose $p\colon \mathcal{C} \to \mathcal{D}$ happens to be a split Grothendieck fibration, so that for each arrow $f \colon \bar{d} \to d$ in $\mathcal{D}$ there are chosen functorial pullback functors
$f^{\**} \colon \mathcal{C}_{d} \to \mathcal{C}_{\bar{d}}$.

A fibered monad $T$ is is then equivalent to the data of the fiber monads
$T_d$ on the fibers $\mathcal{C}_d$
together with, for each arrow $f \colon \bar{d} \to d$ in $\mathcal{D}$, natural transformations
$\varphi_f \colon T_{\bar{d}} f^{\**} \Rightarrow f^{\**} T_{d}$
such that
\begin{itemize}
\item[(a)] the composite 
$T_d f^{\**} g^{\**} 
\overset{\varphi_f g^{\**}}{ \Rightarrow} 
f^{\**} T_{d'} g^{\**} 
\overset{f^{\**}\varphi_g }{ \Rightarrow} 
 f^{\**} g^{\**} T_{d''}$
coincides with 
$T_d (gf)^{\**} \overset{\varphi_{gf} }{ \Rightarrow}  (gf)^{\**} T_{d''}$
and $T_d id_d^{\**} \overset{\varphi_{id_d}}{\Rightarrow} id_d^{\**}  T_d$ is the identity;
\item[(b)] the squares
\begin{equation}\label{GROTHCART EQ}
\begin{tikzcd}
	T_d T_d f^{\**} \ar{r} \ar{d}[swap]{} &
	T_d f^{\**} T_{d'} \ar{r} &
	f^{\**} T_{d'} T_{d'} \ar{d}{} &
	f^{\**} \ar[equal]{r} \ar{d}&
	f^{\**} \ar{d}
\\
	T_d f^{\**} \ar{rr}{} &&
	f^{\**} T_{d'} &
	T_d f^{\**} \ar{r}{} &
	f^{\**} T_{d'}
\end{tikzcd}
\end{equation}
commute.
\end{itemize}
% $\varphi_f$ is induced by applying $T$ to the chosen pullback arrows, 
% (a) is then functoriality of $T$ with respect to pullback arrows
% \eqref{GROTHCART EQ} is the naturality of $\mu \colon TT \Rightarrow T$ and $\eta I \Rightarrow T$ with regard to pullback arrows.
\end{remark}

\begin{remark}
Should the pullback functors $f^{\**}$ in the previous remark admit left adjoints $f_{!}$,
the commutativity of the diagrams in $\eqref{GROTHCART EQ}$
is equivalent to the claim that the induced natural transformations
$T_{d} \Rightarrow f^{\**}T_{d'}f_{!}$
are maps of monads.
\end{remark}


\begin{proposition}
Let $I$ be a fixed diagram category, and $T$ a fibered monad with respect to a Grothendieck fibration with respect to
$p\colon \mathcal{C} \to \mathcal{D}$. Then:
\begin{itemize}
\item[(i)] $p^I\colon \mathcal{C}^I \to \mathcal{D}^I$ is again a Grothendieck fibration;
\item[(ii)] $T^I$ is a fibered monad with respect to $p^I\colon \mathcal{C}^I \to \mathcal{D}^I$;
\item[(iii)] there is a natural identification 
$\mathsf{Alg}_{T^I}^{p^I}(\mathcal{C}^I)\simeq
\left(\mathsf{Alg}_T^p(\mathcal{C})\right)^I$.
\end{itemize}
\end{proposition}

\begin{proof}
(i) is well known (one can simply create cartesian arrows pointwise), and both (ii) and (iii) follow readily from the definitions.
\end{proof}



\subsection{Symmetric sequences and colored operads (Luis' version)}

\begin{definition}\label{CSYM DEF}
	Let $\mathfrak {C} \in \mathsf{F}$ be a set of \textit{colors}.
	A tuple
	$\ksi = (c_1, \ldots, c_n; c_0) \in \mathfrak C^{\times n} \times \mathfrak C$
	is called a \textit{$\mathfrak {C}$-signature}.
	The \textit{$\mathfrak C$-symmetric category} $\Sigma_{\mathfrak C}$ is then the category whose objects are the $\mathfrak{C}$-signatures and with action maps
\[
(c_1, \ldots, c_n; c_0) \xrightarrow{\sigma} (c_{\sigma^{-1}(1)}, \ldots, c_{\sigma^{-1}(n)}; c_0)
\]
	for each permutation $\sigma \in \Sigma_n$, with the natural notion of composition.

Alternatively, we will find it useful to visualize signatures as \textit{corollas with edges decorated by colors in $\mathfrak{C}$}, as depicted below, so that the maps labeled $\sigma$ interchange the edges of the leftmost tree in such a way that one obtains the colored corolla on the left.
\[
\begin{tikzpicture}
      [grow=up,auto,level distance=2.3em,every node/.style = {font=\footnotesize},dummy/.style={circle,draw,inner sep=0pt,minimum size=1.75mm}]
      
      \node at (0,0) [font=\normalsize]{}
		child{node [dummy] {}
			child{
			edge from parent node [swap,near end] {$c_n$} node [name=Kn] {}}
			child{
			edge from parent node [near end] {$c_1$}
node [name=Kone,swap] {}}
		edge from parent node [swap] {$c_0$}
		};
		\draw [dotted,thick] (Kone) -- (Kn) ;
	\node at (5,0) [font=\normalsize]{}
		child{node [dummy] {}
			child{
			edge from parent node [swap,near end] {$c_{\sigma^{-1}(n)}$} node [name=Kn] {}}
			child{
			edge from parent node [near end] {$c_{\sigma^{-1}(1)}$}
node [name=Kone,swap] {}}
		edge from parent node [swap] {$c_0$}
		};
		\draw [dotted,thick] (Kone) -- (Kn) ;

\draw[->] (1.5,1) -- node{$\sigma$} (3,1);
\end{tikzpicture}
\]
Given any map $f \colon \mathfrak{C} \to \mathfrak{D}$ on the sets of colors, there is then a functor
$f_{\**} \colon \Sigma_{\mathfrak{C}} \to \Sigma_{\mathfrak{D}}$
given by $f_{\**} (c_1,\cdots,c_n;c_0) = (f(c_1),\cdots,f(c_n);f(c_0))$. 
\end{definition}


\begin{definition}
Let $\mathcal{V}$ be a category.
The category $\mathsf{Sym}(\mathcal{V})$ of
\textit{symmetric sequences on $\mathcal{V}$} is the category with:
\begin{itemize}
\item objects given by a set of colors $\mathfrak{C} \in \mathsf{F}$
and a functor $\Sigma_{\mathfrak{C}}^{op} \to \mathcal{V}$;
\item arrows given by a map 
$f \colon \mathfrak{C} \to \mathfrak{D}$ of colors and a natural transformation

		\begin{equation}
		\begin{tikzcd}[row sep = tiny, column sep = 35pt]
			\Sigma_{\mathfrak{C}}^{op} \arrow{dr}[name=U]{} \arrow{dd}[swap]{f_{\**}}
		\\
			& \mathcal{V}
		\\
			|[alias=V]| \Sigma_{\mathfrak{D}}^{op} \arrow{ur}[swap]{}
		\arrow[Leftarrow, from=V, to=U,shorten >=0.25cm,shorten <=0.25cm
		,swap
		]
		\end{tikzcd}
		\end{equation}
\end{itemize}
\end{definition}


\begin{remark} The natural projection map
$\mathsf{Sym}(\mathcal{V}) \to \mathsf{F}$
is a (split) Grothendieck fibration with fibers the categories
$\mathsf{Fun}(\Sigma_{\mathfrak{C}}^{op},\mathcal{V})$.
\end{remark}


\begin{remark}\label{SUBCATDOWNL REM}
$\mathsf{Sym}(\mathcal{V})$ is naturally a subcategory of the $2$-overcategory
$\mathsf{Cat}\downarrow^l \mathcal{V}$.
\end{remark}



Our next goal is to describe the fibered monad on $\mathsf{Sym}(\mathcal{V})$ such that the fiber algebras are
$\mathsf{Op}(\mathcal{V})$.

First, much as in Definition \ref{CSYM DEF}, we need colored versions of the tree categories $\Omega$ and $\Omega^0$.
Informally, and given a set of colors $\mathcal{C}$, 
one simply lets $\Omega_{\mathfrak{C}}$
be the category of trees whose edges are decorated by colors in $\mathfrak{C}$, together with color respecting maps. More formally, $\Omega_{\mathfrak{C}}$ is given by the following pullback.
\begin{equation}
	\begin{tikzcd}
		\OC \arrow[d] \arrow[r, "E"] &
		\mathsf F \wr \mathfrak C \arrow[d]
\\
		\Omega \arrow[r, "E"] &
		\mathsf F
	\end{tikzcd}
\end{equation}

Keeping track of the colors on each edge then allows us to generalize the leaf-root and vertex functors of 
\cite{BP_geo} to get analogous functors
\[
\Omega_{\mathfrak{C}} \xrightarrow{\mathsf{lr}} \Sigma_{\mathfrak{C}}
\qquad
\Omega_{\mathfrak{C}} \xrightarrow{V} \Sigma \wr \Sigma_{\mathfrak{C}}
\]
which are readily seen to be natural with respect to maps 
$f \colon \mathfrak{C} \to \mathfrak{D}$.

If $\mathcal{V}$ is a closed symmetric monoidal category, the free operad monad $\mathbb{F}$ on $\mathsf{Sym}(\mathcal{V})$ 
assigns to a functor
$\Sigma_{\mathfrak{C}}^{op} \xrightarrow{X} \mathcal{V}$
the left Kan extension
\begin{equation}
\begin{tikzcd}
	\Omega^{op}_{\mathfrak{C}}
	\arrow[d, "\mathsf{lr}"']
	\arrow[r, "V"]
&
	(\Sigma \wr \Sigma_{\mathfrak{C}})^{op} \arrow[r, "X"]
	\arrow[dl, Rightarrow]
&
	(\Sigma \wr \V^{op})^{op} \arrow[r, "\otimes"]
&
	\V
\\
	\Sigma^{op}_{\mathfrak{C}} \arrow[urrr, "\Lan = \mathbb F_{\mathfrak{C}} X"']
\end{tikzcd}
\end{equation}


Our next goal is to provide a convenient explicit description of the fibers of $\mathsf{Sym}(\mathcal{V})^G \to \mathsf{F}^G$
for each $\mathfrak{C} \in \mathsf{F}^G$
and of the restriction of the monad $\mathbb{F}^G$ to those fibers.

We start with the following, which is a slight strengthening of
\cite[Lemma A.6]{BP_geo}.

\begin{lemma}\label{EQUIVFUNCONV LEM}
Let $G$ be a group and $\mathfrak{C} \colon G \to \mathsf{F}$ be a $G$-set (or, more generally, $\mathcal{D}$ a category and 
$\mathcal{C}_{\bullet}\colon \mathcal{D} \to \mathsf{Cat}$ a functor). Then category of sections as on the left below (or, more generally, as on the left)
\begin{equation}
	\begin{tikzcd}
		&
		\mathsf{Sym}(\mathcal{V}) \arrow{d}{\mathsf{fgt}}
&
		&
		\mathsf{Cat}\downarrow^l \mathcal{V} \arrow{d}{\mathsf{fgt}}
\\
		G \arrow{r}[swap]{\mathfrak{C}} \arrow[dashed]{ru} &
		\mathsf{F}
&
		\mathcal{D} \arrow{r}[swap]{\mathcal{C}_{\bullet}} \arrow[dashed]{ru} &
		\mathsf{Cat}
	\end{tikzcd}
\end{equation}
is isomorphic to the functor category
$\mathsf{Fun}(G\ltimes \Sigma_{\mathfrak{C}}^{op},\mathcal{V})$
(or, more generally, $\mathsf{Fun}(\mathcal{D} \ltimes \mathcal{C}_{\bullet},\mathcal{V})$).
\end{lemma}


\begin{proof}
Since there is a natural inclusion
$\mathsf{Sym}(\mathcal{V}) 
\subseteq 
\mathsf{Cat} \downarrow^l \mathcal{V}$
and the assignment $\mathsf{F} \to \mathsf{Cat}$ 
given by $\mathfrak{C} \mapsto \Sigma_{\mathfrak{C}}$ 
is faithful, it suffices to check the more general claim concerning
$\mathsf{Cat} \downarrow^l \mathcal{V}$.

The remainder of the proof is simply a matter of unpacking definitions. For example, in both cases objects consist of collections of functors $\mathcal{C}_d \to \mathcal{V}$ for $d \in \mathcal{D}$ together with natural transformations
	\begin{equation}
	\begin{tikzcd}[row sep = tiny, column sep = 35pt]
		\mathcal{C}_{d} \arrow{dr}[name=U]{} \arrow{dd}[swap]{\mathcal{C}_{f}}
	\\
		& \mathcal{V}
	\\
		|[alias=V]| \mathcal{C}_{\bar{d}} \arrow{ur}[swap]{}
	\arrow[Leftarrow, from=V, to=U,shorten >=0.25cm,shorten <=0.25cm
	,swap
	]
	\end{tikzcd}
	\end{equation}
for each arrow $f \colon d \to \bar{d}$ in $\mathcal{D}$, subject to natural unitality and associativity requirements.
\end{proof}



In accordance with the previous lemma, we will represent elements of $\mathsf{Sym}^G(\mathcal{V})$
by functors
$G \ltimes \Sigma_{\mathfrak{C}}^{op} \to \mathcal{V}$ for some $\mathfrak{C} \in \mathsf{F}^G$.
Our next step is describe the monad $\mathbb{F}^G$ on $\mathsf{Sym}^G(\mathcal{V})$.

We first note that for any 
$A \in \mathsf{Cat}^G$ there is a natural transformation
$G \ltimes \Sigma \wr A \to \Sigma \wr G \ltimes A$
characterized by sending a $G$-action arrow 
$(a_i) \xrightarrow{g} (g a_i)$
to the diagonal tuple of $G$-action arrows
$(a_i \xrightarrow{g} g a_i)$.
This natural transformation can then be used to describe the $G$-equivariant $\Sigma \wr (-)$ functor, via
\begin{equation}\label{RHOPURP EQ}
\begin{tikzcd}[column sep =40,row sep =0]
	\left( \mathsf{Cat} \downarrow^{l} \mathcal{V} \right)^G
	\ar{r}{\left(\Sigma \wr (-) \right)^G} &
	\left( \mathsf{Cat} \downarrow^{l} \Sigma \wr \mathcal{V} \right)^G
\\
	G \ltimes A \to \mathcal{V} \ar[mapsto]{r} &
	(G \ltimes \Sigma \wr A \to 
	\Sigma \wr G \ltimes  A \to \Sigma \wr \mathcal{V})
\end{tikzcd}
\end{equation}
and combining this with the observation that Kan extensions along
$G \ltimes \Omega_{\mathfrak{C}} \to G \ltimes \Sigma_{\mathfrak{C}}$
are simply the underlying Kan extension along 
$\Omega_{\mathfrak{C}} \to \Sigma_{\mathfrak{C}}$
together with equivariance data (this strengthens Lemma \ref{REDUCELAN LEM} a little),
one obtains that the monad $\mathbb{F}^G$ applied to 
$X\colon G \ltimes \Sigma_{\mathfrak{C}}^{op} \to \mathcal{V}$
is the left Kan extension
\begin{equation}
      \label{FGC_EQ}
      \begin{tikzcd}
            G \ltimes \Omega^{op}_{\mathfrak{C}}
            \arrow[d, "\mathsf{lr}"']
            \arrow[r, "V"]
            &
            (G\ltimes \Sigma \wr \Sigma_{\mathfrak{C}})^{op} \arrow{r}
            \arrow[dl, Rightarrow]
            &
            (\Sigma \wr G\ltimes \Sigma_{\mathfrak{C}})^{op} \arrow[r, "X"]
            &
            (\Sigma \wr \V^{op})^{op} \arrow[r, "\otimes"]
            &
            \V
            \\
            G\ltimes\Sigma^{op}_{\mathfrak{C}} \arrow[urrrr, "\Lan = \mathbb F_{\mathfrak{C}}^G X"']
      \end{tikzcd}
\end{equation}

{\color{red} HERE}


\subsection{Symmetric sequences and colored operads}

Fix a closed symmetric monoidal category $(\V, \otimes)$.

\begin{definition}
      Fix a set $\mathfrak C$ of \textit{colors}.
      A list
      $\ksi = (c_1, \ldots, c_n; c_0) \in \mathfrak C^{\times n} \times \mathfrak C$
      is called a \textit{$\mathfrak C$-signature}.
      A collection $\ksi, \ksi_1,\dots, \ksi_n$ of $\mathfrak C$-signatures is called \textit{compatible} if
      $\ksi \in \mathfrak C^{\times n+1}$, and the target of $\ksi_i$ is the $i$-th source of $\ksi$;
      e.g.  $\xi = (c_1, \ldots, c_n; c_0)$, $\xi_i = (c_{1}^i, \ldots, c_{m_i}^i; c_i)$.
      In this case, define the \textit{composite} to be $\ksi \circ (\ksi_i) = (c_1^1,c_2^1,\dots,c_{m_n}^{n}; c_0)$.
\end{definition}

\begin{definition}
      For a set $\mathfrak C$, define the \textit{$\mathfrak C$-symmetric category} $\Sigma_{\mathfrak C}$
      \begin{equation}
            \Sigma_{\mathfrak C} = \coprod_{n \geq 0} \Sigma_n \ltimes \mathfrak C^{\times n+1} = \Sigma^{+1} \wr \mathfrak C,
      \end{equation}
      where $\mathfrak C^{\times n+1}$ has a natural \textit{right} $\Sigma_n$-action on the first $n$-coordinates.
      A \textit{$\mathfrak C$-symmetric sequence} is a functor $X: \Sigma_{\mathfrak C}^{op} \to \V$.
      Explicitly, this consists of
      \begin{itemize} %{enumerate}[label = (\arabic*)]
      \item An object $X(\ksi) \in \V$ for each signature $\ksi$.
      \item For all signatures $\ksi \in \mathfrak C^{\times n+1}$ and $\sigma \in \Sigma_n$,
            unital and associative maps $X(\xi) \to X(\sigma \cdot \xi)$,
            where $\Sigma_n$ acts on the first $n$ coordinates of $\mathfrak C^{\times n+1}$;
            e.g. maps
            \begin{equation}
                  X(c_1, \ldots, c_n; c_0) \xrightarrow{\sigma} X(c_{\sigma^{-1}(1)}, \ldots, c_{\sigma^{-1}(n)}; c_0).
            \end{equation}
      \end{itemize}
\end{definition}

\begin{definition}
      A \textit{$\mathfrak C$-colored operad} in $\V$ 
      is a $\mathfrak C$-symmetric sequence $\O$ along with appropriate composition laws \footnote{
        (Colored) operads are also known as \textit{symmetric multicategories}.}.
      Explicitly, this consists of the following data:
      \begin{itemize} %{enumerate}[label = (\arabic*)]
      \item An object $\O(\ksi) \in \V$ for each signature $\ksi$.
      \item For all signatures $\ksi \in \mathfrak C^{\times n+1}$ and $\sigma \in \Sigma_n$,
            unital and associative times maps $\O(\xi) \to \O(\sigma \cdot \xi)$,
            where $\Sigma_n$ acts on the first $n$ coordinates of $\mathfrak C^{\times n+1}$;
            i.e., maps
            \begin{equation}
                  \O(c_1, \ldots, c_n; c_0) \xrightarrow{\sigma} \O(c_{\sigma^{-1}(1)}, \ldots, c_{\sigma^{-1}(n)}; c_0).
            \end{equation}
      \item For each $c \in \mathfrak C$, a \textit{unit} $1_c \in \O(c;c)$.                        
      \item For all compatible signatures $\ksi$, $\ksi_1$, $\dots$, $\ksi_n$,
            \textit{composition} maps
            \begin{equation}
                  \O(\xi) \otimes \O(\xi_1) \otimes \ldots \otimes \O(\xi_n) \to \O(\xi \circ (\xi_i)),
            \end{equation}
            which are unital, associative, and appropriately $\Sigma$-equivariant.
      \end{itemize}

      A map of $\mathfrak C$-colored operads is a collection of maps
      $\set{\O(\xi) \to \O'(\xi)}_{\xi}$
      which commutes with the above structure maps.
      
      Let $\Op^{\mathfrak C}(\V)$ denote the category of $\mathfrak C$-colored operads in $\V$.
\end{definition}

\begin{remark}
      There is a natural monad $\mathbb F^{\mathfrak C}$ on $\Sym^{\mathfrak C}(\V)$ such that
      the category of $\mathbb F^{\mathfrak C}$-algebras is precisely $\Op^{\mathfrak C}(\V)$.
      This monad will be explored more (especially in the equivariant context) in Section \ref{COMEGA_SEC}.
\end{remark}

\begin{definition}
      \label{OP_MAP_DEFN}
      Given a map of $G$-sets $F: \mathfrak C' \to \mathfrak C$ and a $\mathfrak C$-symmetric sequence $X$
      there is a natural $\mathfrak C'$-symmetric sequence $F^{\**}X$
      \begin{equation}
            F^{\**}X(\xi') = X(f(\xi')),
      \end{equation}
      which is an operad if $X$ was an operad.
      In fact, we have a pair of adjunctions as below for any such $F$.
      \begin{equation}
            \label{C_CHANGE_EQ}
            \begin{tikzcd}
                  \Op^{\mathfrak C'}(\V) \arrow[r, shift right, "F^{\**}"'] \arrow[d, "\mathsf{fgt}"']
                  &
                  \Op^{\mathfrak C}(\V) \arrow[l, shift right, "F_!"'] \arrow[d, "\mathsf{fgt}"]
                  \\
                  \Sym^{\mathfrak C'}(\V) \arrow[r, shift right, "F^{\**}"']
                  &
                  \Sym^{\mathfrak C}(\V) \arrow[l, shift right, "F_{!!}"'].
            \end{tikzcd}
      \end{equation}
      where we highlight that $F^{\**}$ commutes with $\mathsf{fgt}$, but the left adjoint does not.

      We let $\Sym(\V) = \left(\mathsf F^{op} \ltimes \Sym^{(-)}(\V)\right)^{op}$ and $\Op(\V) = \left(\mathsf F^{op} \ltimes \Op^{(-)}(\V)\right)^{op}$
      denote the categories of
      \textit{(colored) symmetric sequences} and \textit{(colored) operads}
      defined by the Grothendieck constructions on the functors
      \begin{align*}
        \mathsf F^{op} \longrightarrow \Cat,
        \qquad \qquad
        &
          \mathfrak C \longmapsto \Sym^{\mathfrak C}(\V)
          \mbox{ or }
          \mathfrak C \longmapsto \Op^{\mathfrak C}(\V).
      \end{align*}
      
      Explicitly, a map of sequences (resp. operads) $X \to Y$ is given by a map of colors
      $F: \mathfrak C(X) \to \mathfrak C(Y)$
      and a map of $\mathfrak C(X)$-colored sequences (operads)
      $X \to F^{\**} Y$.
\end{definition}


\subsection{Equivariant colored operads}

Now and forever, fix a finite group $G$.

\begin{convention}
      \label{G_CONV}
      We make the following conventions throughout the rest of the paper.
      \begin{itemize} %{enumerate}[label = (\roman*)]
      \item We equip $G$ with a fixed total order.
      \item All $G$-objects will be \textit{left} $G$-objects,
            and we let $BG$ denote the category which encodes left actions.            
      \end{itemize}
\end{convention}

% \begin{remark}
%       \label{G_GR_REM}
%       For any (left) $G$-set $A$, the opposite of the Grothendieck construction $(G \ltimes A)^{op}$
%       on the functor characterizing the $G$-action 
%       \begin{equation}
%             G^{op} \longto \Set, \qquad \qquad \** \longmapsto A
%       \end{equation}
%       is equal to the \textit{translation category}, often denoted $B_A G$,
%       where objects are elements of $A$, and arrows $g: a \to b$ with  $b = g a$.
%       Dually, the standard Grothendieck construction $G \ltimes A$ has arrows
%       $g: a \to b$ with $a = g b$.
      
%       Moreover, if $A$ has a (left) action by $G \times \Sigma$ for another group $\Sigma$, then
%       iterating either type of Grothendieck construction above is associative and commutative.
% \end{remark}

\begin{notation}
      [{cf. \cite{BP_geo}}]
      Let $\mathsf F^G$ denote the category of finite ordered $G$-sets (see Notation \ref{F_WR_NOT} for our convention on $\mathsf F$).
      Moreover, let $\mathsf O_G \into \mathsf F^G$ denote the full subcategory of \textit{transitive} $G$-sets.
      In particular, we note that the orbits $G/H$ are well-defined
      (using the chosen total order on $G$
      and the ``minimal representative'' total order on $G/H$).
\end{notation}

\begin{definition}
      The category $\Op^G(\V)$ of  \textit{$G$-colored operads} in $\V$ is the category of
      (left) $G$-objects in $\Op(\V)$.
\end{definition}

% Unpacking this definition, we see $\O \in \Op^G(\V)$ consists of the following data:
% \begin{enumerate}[label = (\arabic*)]
%       \setcounter{enumi}{-1}
% \item A $G$-set $\mathfrak C$ of colors.
% \item For each signature $\xi$ of $\mathfrak C$, an object $\O(\xi) \in \V$.
% \item where $G$ acts on $\mathfrak C^{\times n+1}$ diagonally (across all $n+1$ coordinates), and $\Sigma_n$ acts on the first $n$.
% \item For each $c \in \mathfrak C$, a \textit{unit} $1_c \in \O(c;c)^{G_c}$.
% \item For compatible signatures $\xi$, $\xi_1$, $\ldots$, $\xi_n$, \textit{composition maps}
%       \begin{equation}
%             \O(\xi) \otimes \O(\xi_1) \otimes \ldots \otimes \O(\xi_n) \to \O(\xi \circ (\xi_i)),
%       \end{equation}
% \end{enumerate}
% such that composition is
% compatible with the $G$-action on each component as well as the appropriate actions of $\Sigma$,
% and is unital and associative. 

We would like to unpack this definition, and first do so in general.

% -------------------- Equivariant Grothendieck Constructions --------------------

Suppose $E = (\mathcal D^{op} \ltimes \mathcal C_{(-)})^{op}$ is the contravariant Grothendieck construction on the functor
\begin{equation}
      % \begin{tikzcd}[row sep = tiny]
      %       \mathcal D^{op} \arrow[r]
      %       &
      %       \mathsf{Cat}
      %       \\
      %       d \arrow[r, mapsto]
      %       &
      %       \mathcal C_{d}.
      % \end{tikzcd}
      \mathcal D^{op} \longto \Cat,
      \qquad \qquad
      d \longmapsto \mathcal C_d.
\end{equation}

A (left) $G$-action on an element $(d,c_d)$ is given by compatible maps
$d \xrightarrow{g} d$ and
$c_d \xrightarrow{g_c} g^{\**}c_d$
(with the $g_c$ compatible in the sense that
\begin{equation}
      \begin{tikzcd}
            c_d \arrow[r, "g_c"] \arrow[dr, "(h g)_c"']
            &
            g^{\**} c_d \arrow[d, "g^{\**} h_c"]
            \\
            &
            (h g)^{\**} c_d
      \end{tikzcd}
\end{equation}
commutes),
and $G$-maps between such objects
are pairs of maps $(d \xrightarrow{f} d, c_d \xrightarrow{f_c} f^{\**}c_{d'})$ which commute with the action.

Equivalently, and more concretely,
an object in $E^G$ is given by a pair $(d, \underline{c}_d)$ where
$d \in \mathcal D^G$ and
$\underline{c}_d$ is a $G/e$-indexed diagram
% \footnote{
%   We recall that $(G \ltimes (G/e))^{op} = B_{G/e} G$.}
$E G := G \ltimes (G/e) \to \mathcal C_d$,
such that $\underline{c}_d(g) = g_{d}^{\**}\underline{c}_d(e)$.
A $G$-map in this context is given by a pair $(f, \Phi_c)$ where
$f: d \to d'$ in $\mathcal D^G$ and
$\Phi: \underline{c}_d \Rightarrow f^{\**}\underline{c}_{d'}: E G \to \mathcal C_d$
(where we note that $f^{\**}\underline{c}_{d'}$ is a functor since $f$ is a $G$-map).

Put another way, we have the following.
\begin{lemma}
      \label{G_GR_LEM}
      The category of $G$-objects
      $\left(\mathcal D^{op} \ltimes \mathcal C_{(-)}\right)^{op,G}$
      is given by the Grothendieck construction on the functor
      \begin{equation}
            % \begin{tikzcd}[row sep = tiny]
            %       \mathcal D^{G,op} \arrow[r]
            %       &
            %       \Cat
            %       \\
            %       d \arrow[r, mapsto]
            %       &
            %       \Fun^{G^{op}}(E G^{op}, \mathcal C_d)
            % \end{tikzcd}
            \mathcal D^{G,op} \longrightarrow \Cat,
            \qquad \qquad
            d \longmapsto \Fun^{G^{op}}(E G, \mathcal C_d)
      \end{equation}
      where $\Fun^{G^{op}}$ denotes the category of $G^{op}$-functors and $G^{op}$-natural transformations.
\end{lemma}
\begin{proof}
      We observe that $EG$ has a natural $G^{op}$-action via the \textit{right} action of $G$ on $G/e$,
      and $\mathcal C_d$ has a $G^{op}$ action induced by the (left) $G$-action on $d$.
      The rest follows by unpacking definitions.
      {\color{OliveGreen}
        A map $\underline{c}: EG \to \mathcal C_d$ gives elements $c_g$ and maps $(g,h): c_g \to c_{hg}$ such that
        $(g, kh) = (hg, k) \circ (g, h)$.
        $x \in G^{op}$ acts on $EG$ by sending $g$ to $g x$ and the map $(g,h)$ to $(g x,h)$,
        and on $\mathcal C_d$ by sending $c$ to $x^{\**}c$ and $f$ to $x^{\**}f$.
        Equivariance implies that $c_x = x^{\**}c_e$ and $(x,g) = x^{\**}(e,g)$, finishing the proof.
      }
\end{proof}

% ----------------------------------------------------------------------

We apply this result to both $\Sym(\V) = (\mathsf F^{op} \ltimes \Sym^{(-)}(\V))^{op}$ and $\Op(\V) = (\mathsf F^{op} \ltimes \Op^{(-)}(\V))^{op}$.

\begin{definition}
      For the category $E = \Sym^G(\V) = (\Sym(\V))^G$ of \textit{$G$-symmetric sequences},
      we note that, for any $G$-set $\mathfrak C$, we have
      \begin{equation}
            (E G \times \Sigma_{\mathfrak C}^{op})/G \simeq G \ltimes \Sigma_{\mathfrak C}^{op},
      \end{equation}
{\color{red} Note(Luis): Since the chirality of the actions don't match, one should probably replace 
$(E G \times \Sigma_{\mathfrak C}^{op})/G$ with
$E G \times_G \Sigma_{\mathfrak C}^{op}$ 
}

      where $\Sigma_{\mathfrak C}$ inherits a $G$-action from the action on $\mathfrak C$;
      we refer to this later category
      \begin{equation}
            G \ltimes \Sigma_{\mathfrak C}^{op} = \coprod_{n \geq 0}(G \times \Sigma_n^{op}) \ltimes \mathfrak C^{\times n+1}
      \end{equation}
      as the \textit{$(G, \mathfrak C)$-symmetric category},
      and the category $\Sym^{G, \mathfrak C}(\V)$ of functors $X: G \ltimes \Sigma_{\mathfrak C} ^{op}\to \V$
      the category of \textit{$(G, \mathfrak C)$-symmetric sequences}.
      % 
      Then, by adjunction and Lemma \ref{G_GR_LEM}, we have that
      $\Sym^G(\V)$ is isomorphic to the Grothendieck construction on the functor
      \begin{equation}
            \mathsf F^{G,op} \longrightarrow \Cat,
            \qquad
            \qquad
            \mathfrak C \longmapsto \Sym^{G, \mathfrak C}(\V).
      \end{equation}
\end{definition}

\begin{remark}
      When $G = \set{e}$ and $\mathfrak C = \set{*}$, $G \ltimes \Sigma_{\mathfrak C}^{op} = \Sigma^{op}$.
\end{remark}


\begin{example}
      Now, for $E = \Op(\V)$,
      we see that an object $\O \in \Op^G(\V)$ consists of
      a $G$-set $\mathfrak C = \mathfrak C_\O$ of colors, and a compatible $G/e$-indexed diagram of $\mathfrak C$-colored operads;
      in particular, $\O$ has an underlying object in $\Sym^G(\V)$
      (i.e., satisfying items $(0)$ through $(2)$ below)
      by composition with the forgetful functor $\Op^{\mathfrak C}(\V) \to \Sym^{\mathfrak C}(\V)$.
      % 
      Explicitly, an object $\O \in \Fun^{G^{op}}(E G, \Op^{\mathfrak C}(\V)) =: \Op^{G, \mathfrak C}(\V)$
      is given by the following data:
      \begin{enumerate}[label = (\arabic*), start = 0]
      \item A $G$-set $\mathfrak C = \mathfrak C_{\O}$ of colors.
      \item For each $\mathfrak C$-signature $\ksi$, an object $\O(\ksi) \in \V$.
      \item \label{SACTION_LBL}
            For all signatures $\ksi \in \mathfrak C^{\times n + 1}$ and $\sigma \in \Sigma_n$,
            maps $\O(\ksi) \to \O(\sigma \cdot \ksi)$
            which are unital and associative.
      \item \label{GUNIT_LBL} 
            For each $c \in \mathfrak{C}$, a \textit{unit} $1_c \in \O(c;c)$.
      \item \label{COMP_LBL}
            For all compatible signatures $\ksi, \ksi_1,\dots, \ksi_n$,
            \textit{composition maps} $\O(\ksi) \otimes \O(\ksi_1) \otimes \dots \otimes \O(\ksi_n) \to \O(\ksi \circ (\ksi_i))$
            which are unital, associative, and appropriately $\Sigma$-equivariant
      \item \label{GACTION_LBL}
            For all $g \in G$, maps $\O(\ksi) \to g^{\**}\O(\ksi) = \O(g \cdot \ksi)$
            (with $G$ acting on $\mathfrak C^{\times n+1}$ diagonally)
            which are unital and associative, and which commute with the composition maps
            (note that if $\ksi, (\ksi_i)$ are compatible, then so are $g \ksi, (g \ksi_i)$).
      \end{enumerate}
      Synthesizing, we may combine \ref{SACTION_LBL} and \ref{GACTION_LBL} into
      \begin{enumerate}
      \item[($2'$)] For all signatures $\xi \in \mathfrak C^{\times n+1}$ and $(g,\sigma) \in G\times \Sigma_n$, maps
            $\O(\xi) \to \O((g,\sigma)\cdot \xi)$
            which are unital and associative.
      \end{enumerate}
      
      and replace \ref{GUNIT_LBL} and \ref{COMP_LBL} with
      \begin{enumerate}
      \item[($3'$)] For each $c \in \mathfrak C$, a $G_c$-fixed unit $1_c \in \O(c;c)^{G_c}$.
      \item[($4'$)] For all compatible signatures $\ksi, (\ksi_i)$,
            composition maps $\O(\ksi) \otimes \bigotimes_i \O(\ksi_i) \to \O(\ksi \circ (\ksi_i))$
            which are unital, assocative, $G$-equivariant, and approriately $\Sigma$-equivariant.
      \end{enumerate}
      
      From Lemma \ref{G_GR_LEM}, we can see that
      maps of $G$-operads $\O \to \P$ are given by a pair $(f, F)$ where
      $f: \mathfrak C(\O) \to \mathfrak C(\P)$ is a map of $G$-sets, and
      $F: \O \to f^{\**}\P$ is a map in $\Op^{\mathfrak C(\O)}(\V)$ that commutes with the $G$-action
      (or, equivalently, so that the $g^{\**}F$ assemble into a map of $G$-indexed diagrams).
\end{example}

As in \eqref{C_CHANGE_EQ}, we in fact have a pair of adjoints
\begin{equation}
      \label{GC_CHANGE_EQ}
      \begin{tikzcd}
            \Op^{G, \mathfrak C'}(\V) \arrow[r, shift right, "F^{\**}"'] \arrow[d, "\mathsf{fgt}"']
            &
            \Op^{G, \mathfrak C}(\V) \arrow[l, shift right, "F_!"'] \arrow[d, "\mathsf{fgt}"]
            \\
            \Sym^{G, \mathfrak C'}(\V) \arrow[r, shift right, "F^{\**}"']
            &
            \Sym^{G, \mathfrak C}(\V) \arrow[l, shift right, "F_{!!}"'].
      \end{tikzcd}
\end{equation}
where again only the right adjoint commutes with $\mathsf{fgt}$. 
%
Each $\Op^{G, \mathfrak C}(\V) \subseteq \Op^G(\V)$
is hence the subcategory of $\mathfrak C$-colored operads and maps which are the identity on colors.

\begin{lemma}
      $\Op^G(\V)$ is isomorphic to the Grothendieck construction on the functor
      \begin{equation}
            % \begin{tikzcd}[row sep = tiny]
            %       \mathsf (F^G)^{op} \arrow[r] & \mathsf{Cat}
            %       \\
            %       \mathfrak C \arrow[r, mapsto] & \Op^{G,\mathfrak C}(\V).
            % \end{tikzcd}
            \mathsf F^{G,op} \longto \Cat,
            \qquad \qquad
            \mathfrak C \longmapsto \Op^{G, \mathfrak C}(\V).
      \end{equation}
\end{lemma}


\begin{remark}
      \label{COLOR_SQ_REM}
      We record the following straightforward observations.
      If we are given a map $F: \O_1 \to \O_2$ that is color-fixed,
      and a square in $\Op^G(\V)$ as in the middle of \ref{COLOR_SQ_EQ}, then
      \begin{enumerate}[label = (\roman*)]
      \item the square in the middle commutes iff the square on the right commutes iff the square on the left commutes, and
      \item the square in the middle is a pushout in $\Op^G(\V)$ iff
            the square on the left is a pushout in $\Op^{G, \mathfrak C_{\P_1}}(\V)$.
      \end{enumerate}
      % a commuting square (resp. lifting diagram, pullback, pushout) as in the middle below is
      % equivalent to such squares on the left and right.
      \begin{equation}
            \label{COLOR_SQ_EQ}
            \begin{tikzcd}
                  a_! \O_1 \arrow[d, "a_! F"'] \arrow[r, "a"]
                  &
                  \P_1 \arrow[d, "p"]
                  &&
                  \O_1 \arrow[d, "F"'] \arrow[r, "a"]
                  &
                  \P_1 \arrow[d, "p"]
                  &&
                  \O_1 \arrow[d, "F"'] \arrow[r, "a"]
                  &
                  a^{\**} \P_1 \arrow[d]
                  \\
                  a_! \O_2 \arrow[r]
                  &
                  \P_2
                  &&
                  \O_2 \arrow[r]
                  &
                  \P
                  &&
                  \O_2 \arrow[r]
                  &
                  a^{\**} p^{\**} \P
            \end{tikzcd}
      \end{equation}
\end{remark}

\begin{remark}
      We also have a inclusion-forgetful adjunction
      \begin{equation}
            \label{JSTAR_CAT_EQ}
            \begin{tikzcd}
                  \Op^{G, \mathfrak C}(\V) \arrow[r, shift right, "j^{\**}"'] \arrow[d, "{(-)^H}"']
                  &
                  \Cat^{G, \mathfrak C}(\V) \arrow[l, shift right, "j_!"'] \arrow[d, "{(-)^H}"']
                  \\
                  \Op^{\mathfrak C^H}(\V)  \arrow[r, shift right, "j^{\**}"']
                  &
                  \Cat^{\mathfrak C^H}(\V) \arrow[l, shift right, "j_!"']
            \end{tikzcd}
      \end{equation}
      such that $j^{\**}$ commutes with taking $H$-fixed points.
\end{remark}


\begin{remark}
	Unlike in the single-colored case, $\Op^G(\V)$ does \textit{not} coincide with the category of colored operads in $\V^G$.
	Indeed, objects in $\Op(\V^G)$ have a fixed $G$-set of colors,
        and each level $\O(\xi)$ has an action by the full group $G$
	(though only a partial action by $\Sigma_{|\xi|}$).
\end{remark}



\subsection{Colored trees and a monadic description of equivariant operads}
\label{COMEGA_SEC}

\todo[inline]{merge with Luis / provide more introduction to these categories}

We would like a more algebraic description of equivariant colored operads.
To do so, we investigate $\SC$ and $G \ltimes \Sigma_{\mathfrak C}^{op}$ and related categories,
leading to a monad on $\Sym^{\G, \mathfrak C}(\V)$ which defines $\Op^{G, \mathfrak C}(\V)$.

% \begin{remark}
%       As a warning, recall Convention \ref{G_CONV} about the chirality of the action of groups and their associated categories.
% \end{remark}

\begin{remark}
      For much of this section, $\mathsf F$ will often denote not just $\mathsf F$ itself, but also any of the subcategories
      $\Sigma$, $\mathsf F_i$, or $\mathsf F_s$.
      We will specify precisely when one of the categories is needed.
\end{remark}

\begin{definition}
      Given a category $\mathcal C$ over $\mathsf F$, let $\mathcal C_{\mathfrak C}$ denote the pullback
      \begin{equation}
            \begin{tikzcd}
                  \mathcal C_{\mathfrak C} \arrow[r] \arrow[d]
                  &
                  \mathsf F \wr \mathfrak C \arrow[d]
                  \\
                  \mathcal C \arrow[r]
                  &
                  \mathsf F
            \end{tikzcd}
      \end{equation}
\end{definition}

\begin{remark}
      We observe that the above definition does not conflict with our notation $\Sigma_{\mathfrak C}$ from the previous section,
      as $\SC$ is isomorphic to the pullback on the left below,
      where $E: \Sigma \to \mathsf F$ sends $\underline{n}$ to $\underline{n+1}$.
      % and the second arrow is the diagonal functor from Remark \ref{WR_DIAG_REM}. 
      \begin{equation}
            \label{COMEGA_B_EQ}
            \begin{tikzcd}
                  \SC \arrow[r, "E"] \arrow[d]
                  &
                  \mathsf F \wr \mathfrak C \arrow[d]
                  & %
                  \OC \arrow[d] \arrow[r, "E"]
                  &
                  \mathsf F \wr \mathfrak C \arrow[d]
                  & %
                  G^{op} \ltimes \OC \arrow[d] \arrow[r, "E"]
                  &
                  G^{op} \ltimes (\mathsf F \wr \mathfrak C) \arrow[d] \arrow[r]
                  &
                  \mathsf F \wr (G^{op} \ltimes \mathfrak C) \arrow[d]
                  \\
                  \Sigma \arrow[r, "E"]
                  &
                  \mathsf F
                  & %
                  \Omega \arrow[r, "E"]
                  &
                  \mathsf F
                  & %
                  G^{op} \times \Omega \arrow[r, "E"]
                  &
                  G^{op} \times \mathsf F \arrow[r]
                  &
                  \mathsf F \wr G^{op}
            \end{tikzcd}
      \end{equation}
\end{remark}

\begin{definition}
      Let $\Omega_{\mathfrak C}$ denote the middle pullback in \eqref{COMEGA_B_EQ},
      where $E: \Omega \to \mathsf F$ sends a tree $U$ to its set $E(U)$ of edges.
      We note that $G^{op} \ltimes \OC$ is given by the right pullback,
      as $G^{op} \ltimes (-)$ preserves pullbacks and the rightmost square is a pullback
      by Lemmas \ref{GL_PULL_LEM} and \ref{GD_PULL_LEM},
      and a similar pair of squares yields $G^{op} \ltimes \SC$. 

      We have a natural inclusion of categories $\Sigma_{\mathfrak C} \into \Omega_{\mathfrak C}$,
      and as such we will called elements of these categories
      \textit{colored corollas} and \textit{colored trees},
      and denote them by $(U,\mathfrak c)$, where $\mathfrak c: E(U) \to \mathfrak C$ is a map of sets,
      or just by $U$ if there is no confusion.
\end{definition}

Unpacking definitions (cf. Remark \ref{G_GR_REM}), we see that a map $(U, \mathfrak c) \to (V, \mathfrak d)$ in $G^{op} \ltimes \Omega_{\mathfrak C}$
is given by
a map $f: U \to V$ in $\Omega$ and an element $g\in G$,
such that $\mathfrak c(f(e)) = g.\mathfrak d(e)$ for all $e \in E(V)$.
\begin{equation}
      \begin{tikzcd}
            E(U) \arrow[d, "\mathfrak c"'] \arrow[r, "f"]
            &
            E(V) \arrow[d, "\mathfrak d"]
            \\
            \mathfrak C \arrow[r, "g"] 
            &
            \mathfrak C
      \end{tikzcd}
\end{equation}

In particular, we have maps of the form
\begin{equation}
      g = (id, g): g^{\**} U \to U
      % (U, E(U) \xrightarrow{\mathfrak c} \mathfrak C \xrightarrow{g \cdot} \mathfrak C)
      % \to
      % (U, E(U) \xrightarrow{\mathfrak c} \mathfrak C).
\end{equation}



Now, many of the natural functors around $\Omega$ and $\Sigma$ have generalizations to the colored setting,
which can be built through a straightforward use of the universal property of pullbacks.

\begin{definition}
      We have natural \textit{vertex} functors
      \begin{equation}
            \OC \to \Sigma \wr \SC,
            \qquad
            % G \ltimes \OC^{op} =
            % (G^{op} \ltimes \OC)^{op} \xrightarrow{V}
            % (G^{op} \ltimes (\Sigma \wr \SC))^{op} \to
            % (\Sigma \wr (G^{op} \ltimes \SC))^{op} =
            % (\Sigma \wr (G \ltimes \SC^{op})^{op})^{op},
            G^{op} \ltimes \OC \to G^{op} \ltimes (\Sigma \wr \SC) \to \Sigma \wr (G^{op} \wr \SC),
      \end{equation}
      as colorings of a tree restrict to colorings of each vertex corolla.

      Similarly, there is a \textit{leaf-root} functor
      $\mathsf{lr}: G \ltimes \Omega_{\mathfrak C}^{op} \to G \ltimes \Sigma_{\mathfrak C}^{op}$,
      where the coloring of $\mathsf{lr}(T)$ is a restrict of the coloring of $T$.
\end{definition}

\begin{remark}
      Equivalently, the first map is given by $\pi_{\mathfrak C}^{\**}$ applied to the original vertex functor
      $V: \Omega \to \Sigma \wr \Sigma$;
      see {\color{red} LATER SECTIONS}).
\end{remark}

With these definitions in place, we make the following definition.

\begin{definition}
      Given $X \in \Sym^{G, \mathfrak C}$, let $\mathbb F^{\mathfrak C} X$ denote the left Kan extension below.
      \begin{equation} 
            \begin{tikzcd}
                  G \ltimes \Omega_{\mathfrak C}^{op}
                  \arrow[d, "\mathsf{lr}"']
                  \arrow[r, "V"]
                  &
                  (\Sigma \wr (G \ltimes \Sigma_{\mathfrak C}^{op})^{op})^{op} \arrow[r, "\Sigma \wr X"]
                  \arrow[dl, Rightarrow]
                  &
                  (\Sigma \wr \V^{op})^{op} \arrow[r, "\otimes"]
                  &
                  \V
                  \\
                  G \ltimes \Sigma_{\mathfrak C}^{op} \arrow[urrr, "\Lan = \mathbb F^{\mathfrak C} X"']
            \end{tikzcd}
      \end{equation}
\end{definition}

\begin{theorem}
      \label{FC_MONAD_PROP}
      For all $G$-sets $\mathfrak C$,
      $\mathbb F^{\mathfrak C}$ is a monad on $\Sym^{G, \mathfrak C}(\V)$,
      with category of algebras $\Op^{G, \mathfrak C}(\V)$.
\end{theorem}

A complete analysis of this functor, including the proof of Theorem \ref{FC_MONAD_PROP},
can be found in Appendix \ref{MONAD_APDX}.

To end this section, we show that this definition is consistent,
in that when $\mathfrak C = \**$, this is isomorphic to the free single-colored operad monad from \cite[{Eq. (4.1)}]{BP_geo}:

% Note that when $\mathfrak C = \set{\**}$, we have
% $G \ltimes \Omega_{\mathfrak C} = \Omega \times G$ and 
% $G \ltimes \Sigma_{\mathfrak C} = \Sigma \times G$.


\begin{notation}[\cite{BP_geo}]
      Let $\mathbb F'$ denote the \textit{free single-colored operad monad} on $\V$, given by the left Kan extension of the following diagram.
      \begin{equation}
            \begin{tikzcd}
                  \Omega^{op}
                  \arrow[d, "\mathsf{lr}"']
                  \arrow[r, "V"]
                  &
                  (\Sigma \wr \Sigma)^{op} \arrow[r, "X"]
                  \arrow[dl, Rightarrow]
                  &
                  (\Sigma \wr \V^{op})^{op} \arrow[r, "\otimes"]
                  &
                  \V
                  \\
                  \Sigma^{op} \arrow[urrr, "\Lan = \mathbb F' X"']
            \end{tikzcd}
      \end{equation}
\end{notation}

First, one lemma.

\begin{notation}
      Given a functor $X : \C \to \mathsf{Fun}(\mathcal D, \V)$,
      let $\tilde X$ denote the adjoint $\tilde X: \C \times \mathcal D \to \V$.
\end{notation}

\begin{lemma}
      \label{SPAN_LAN_LEM}
      Conisder the two spans below.
      \begin{equation}
            \begin{tikzcd}
                  \C \arrow[d, "p"] \arrow[r, "X"]
                  &
                  \mathsf{Fun}(\mathcal D, \V)
                  &&
                  \C \times \mathcal D \arrow[d, "p \times \mathsf{id}"] \arrow[r, "\tilde X"]
                  &
                  \V
                  \\
                  \mathcal E
                  &
                  &&
                  \mathcal E \times \mathcal D
            \end{tikzcd}
      \end{equation}
      
      Then $\Lan_p X$ is adjoint to $\Lan_{p \times \mathsf{id}} \tilde X$. 
\end{lemma}
\begin{proof}
      Using the pointwise description of the Kan extension, we have
      \begin{align}
        \widetilde{\Lan_p X}(e,d)
        &= (\Lan_p X(e))(d)
          = \left(
          \colim\limits_{\substack{ \C \downarrow e \\ p(c) \to e}} X(c)
        \right)(d)
        = \colim\limits_{\substack{ \C \downarrow e \\ p(c) \to e}}(X(c)(d))
        = \colim\limits_{\substack{ \C \downarrow e \\ p(c) \to e}}(\tilde X(c,d))\\
        &= \colim\limits_{\substack{ \C \times \set{d} \downarrow (e,d) \\ p(c) \to e}}(\tilde X(c,d))
        \cong \colim\limits_{\substack{ \C \times \mathcal D \downarrow (e,d) \\ (p(c),d') \to (e,d)}}(\tilde X(c,d'))
        = \Lan_{p \times \mathsf{id}}\tilde X(c,d),
      \end{align}
      where the isomorphism holds by a straightforward finality argument.
      On maps, a similar argument holds.
\end{proof}

\begin{proposition}
      \label{TEST_PROP}
      $\mathbb F^{\set{\**}}$ is a monad, and moreover
      the category of $\mathbb F^{\set{\**}}$-algebras in $\mathsf{Fun}(G \times \Sigma^{op}, \V)$ is equivalent to
      the category of $\mathbb F'$-algebras in $\mathsf{Fun}(\Sigma^{op}, \V^G)$.
\end{proposition}
\begin{proof}
      Let $\tau: \tilde X \mapsto X$ denote the isomorphism of categories
      $\mathsf{Fun}(G \times \Sigma^{op}, \V) \xrightarrow{\tau} \mathsf{Fun}(\Sigma^{op}, \V^G)$.
      Then $\mathbb F^{\set{\**}} = \tau^{-1} \mathbb F' \tau$ by \ref{SPAN_LAN_LEM}, and so
      $\mathbb F^{\set{\**}}$ is in fact a monad, and
      the isomorphism lifts to an isomorphism on the category of algebras.
\end{proof}





% \subsection{Monad - placeholder subsection}

% \todo[inline]{merge with Luis work}

% We define the monad on $\mathsf{WSpan}^r_{\Cat \downarrow^r \mathsf F}(\Sigma^{op}, \V)$.
% \begin{definition}
%       Let $\mathcal A$ be in this category. This consists of the data of a category $\mathcal A$ and the following diagram.
%       \begin{equation}
%             \begin{tikzcd}
%                   |[alias = U]| \Sigma \arrow[dr, "E"']
%                   &
%                   \mathcal A \arrow[d, "\epsilon"'{name = V}, ""{name = D}] \arrow[l, "\lambda"'] \arrow[r]
%                   &
%                   |[alias = C]| \V \arrow[dl, "\varnothing"]
%                   \\
%                   & \mathsf F
%                   \arrow[Rightarrow, from = U, to = V, "\alpha"]
%                   \arrow[Rightarrow, from = C, to = D, "\beta"']
%             \end{tikzcd}
%       \end{equation}
%       Then we define $N(\mathcal A)$ to be the span given by the pullback
%       \begin{equation}
%             \begin{tikzcd}
%                   \Omega \wr \mathcal A \arrow[r] \arrow[d]
%                   &
%                   \mathsf F \wr \mathcal A \arrow[d] \arrow[r]
%                   &
%                   \mathsf F \wr \V \arrow[r, "\otimes"]
%                   &
%                   \V
%                   \\
%                   \Omega \arrow[d] \arrow[r, "V"]
%                   &
%                   \mathsf F \wr \Sigma
%                   \\
%                   \Sigma
%             \end{tikzcd}
%       \end{equation}
%       with the map over $\mathsf F$ given by the pushout in $\Fun(\Omega \wr \mathcal A, \mathsf F)$ of the natural transformations
%       \begin{equation}
%             E(T) \Leftarrow \coprod_{v \in V(T)} (\lambda(a_v)+1) \Rightarrow \coprod_{v \in V(T)}\epsilon(a_v)
%       \end{equation}
%       on the objects $(T, (a_v)) \in \Omega \wr \mathcal A$,
%       where the maps between these functors are given by the following diagram
%       (and in particular, we have a map $\lambda(a_v) + 1 \to E(T_v)$).
%       \begin{equation}
%             \begin{tikzcd}
%                   \Omega \wr \mathcal A \arrow[r, equal] \arrow[d, equal]
%                   &
%                   \Omega \wr \mathcal A \arrow[d] \arrow[r, "V"]
%                   &
%                   \mathsf F \wr \mathcal A \arrow[d, "\lambda"] \arrow[r, "\epsilon"]
%                   &
%                   \mathsf F \wr \mathsf F \arrow[r, "\amalg"] \arrow[d, equal]
%                   &
%                   \mathsf F \arrow[d, equal]
%                   \\
%                   \Omega \wr \mathcal A \arrow[d, equal] \arrow[r, "\lambda"]
%                   &
%                   \Omega \wr \Sigma \arrow[d, equal] \arrow[r, "V"]
%                   &
%                   \mathsf F \wr \Sigma \arrow[r, "{(-)+1}"] \arrow[ur, Rightarrow, "\alpha"] \arrow[drr, Rightarrow] 
%                   &
%                   \mathsf F \wr \mathsf F \arrow[r, "\amalg"]
%                   &
%                   \mathsf F \arrow[d, equal]
%                   \\
%                   \Omega \wr \mathcal A \arrow[r, "\lambda"]
%                   &
%                   \Omega \wr \Sigma \arrow[r, "\simeq"]
%                   &
%                   \Omega \arrow[rr, "E"] \arrow[u, "V"]
%                   &&
%                   \mathsf F
%             \end{tikzcd}
%       \end{equation}
      
%       \todo[inline]{come back: check that this gives maps over $\mathsf F$}
% \end{definition}

% \begin{lemma}
%       $\Omega \wr \mathcal A$ is the strict 2-pullback in $\Cat \downarrow^r \mathsf F$.
% \end{lemma}
% \begin{proof}
%       This follows immediately from Remark \ref{SPANLIM REM}.
% \end{proof}

% \begin{proposition}
%       $N$ is a monad on $\mathsf{WSpan}^r_{\Cat \downarrow^r \mathsf F}(\Sigma^{op}, \V)$.
% \end{proposition}
% \begin{proof}
%       \todo[inline]{come back}
% \end{proof}



% % \begin{proof}
% %       This follows from the work in {\color{red} LATER SECTIONS}:
% %       in particular, $\mathbb F^{\mathfrak C} = \Lan \circ N^{\mathfrak C} \circ \iota$,
% %       where $N^{\mathfrak C}$ is the operation on spans...
% %       The fact that all the necessarily diagrams, as shown in \cite{BP_geo}, commute/agree follows by breaking those diagrams in half:
% %       the left sides never mention $\V$, and we can show that the generalized diagrams commute/agree by applying $\pi^{\**}$ to the diagrams in \cite{BP_geo},
% %       where $\pi: \mathsf F \wr (G \ltimes \mathfrak C) \to \mathsf F$,
% %       and using the fact that this is a functor which preserves standard limits;
% %       the right sides only deal with the permutative structure of our underlying category $\V$.
% %       These are connected by trivially commuting information about our $(G, \mathfrak C)$-symmetric sequence $X$.
% %       \todo[inline]{come back: this doesn't actually say anything yet.}
% % \end{proof}


 
\subsection{The homotopy theory of operads with a fixed color $G$-set}
\label{MSC_SEC}

We begin our analysis of the homotopy theory of equivariant colored operads by
considering the subcategories $\Op^{G, \mathfrak C}(\V) \subseteq \Op^G(\V)$
where the $G$-set of colors is fixed, and all maps are the identity on $\mathfrak C$.
The main result of this section is Theorem \ref{THM1_C}, which states that
the model structure on $\Op^{G, \mathfrak C}(\V)$ is transferred from $\Sym^{G, \mathfrak C}(\V)$.

\begin{definition}
      Given an adjunction $U: \mathcal C \leftrightarrows \mathcal M :F$ where $\mathcal M$ is a model category,
      the \textit{transferred model structure} on $\mathcal C$, if it exists, is such that
      an arrow $f \in \mathcal C$ is a weak equivalence or fibration iff $U(f)$ is so in $\mathcal M$.
\end{definition}


Following \cite{Ste16, BP_geo}, in order to produce appropriate equivariant theories,
we require that the fixed point functors interact properly with the underlying model structures.
\begin{definition}
      \label{CELLFP_DEF}
      We say $\V$ has \textit{cellular fixed points} if
      for all finite groups $G$ and subgroups $H, K \leq G$ one has that:
      \begin{enumerate}[label = (\roman*)]
      \item $(-)^H$ preserves direct colimits of diagrams in $\V^G$ where each underlying arrow in $\V$ is a cofibration;
      \item $(-)^H$ preserves pushouts where one leg is $(G/K) \cdot f$, for $f$ a cofibration in $\V$.
      \item for each object $X \in \V$, the natural map $(G/K)^H \cdot X \to ((G/K) \cdot X)^H$ is an isomorphism.
      \end{enumerate}
\end{definition}

\begin{remark}
      \label{LEVEL_COF_REM}
      An immediate consequence of cellularity (cf. the proof of \cite[Prop. 6.3(i)]{BP_geo})
      is that for any genuine (trivial) cofibration $f \in \V^G_{gen}$,
      the map $f^H$ is a (trivial) cofibration in $\V$ for all $H \leq G$.
\end{remark}


For completeness, we recall a main result of \cite{Ste16}, which transfers model structures along several adjuctions.
\begin{theorem}[{\cite[Thm. 2.10]{Ste16}}]
      Let $\V$ be cofibrantly generated model category with cellular fixed points, $\F$ any family of subgroups of $G$ containing the trivial subgroup, and $\mathsf O_\F$ the full subcategory of $\mathsf O_G$ spanned by the $G/H$ with $H \in \F$.
      Then
      \begin{itemize}
      \item the projective model structure on $\V^{\mathsf O_\F^{op}}$ exists;
      \item the $\F$-model structure on $\V^G$ exists, where an arrow $f \in \V$ is a weak equivalence or fibration iff each $f^H$ is in $\V$ for each $H \in \F$; and
      \item the inclusion $\V^G \xrightarrow{i_{\**}} \V^{\mathsf O_F^{op}}$, $i_{\**}X(G/H) = X^H$ is a right Quillen equivalence.
      \end{itemize}
\end{theorem}      

This result formally extends to diagrams $\V^{\mathcal D}$ where $\mathcal D$ is a groupoid.
\begin{definition}
      For any groupoid $\mathcal D$, a \textit{family} of subgroups of $\mathcal D$ is
      a collection $\F = \set{\F_d}$ of families of subgroups $\F_d$ of $\Aut(d)$ for all $d \in \mathcal D$,
      such that for any map $\alpha: d \to d'$, $\alpha^{\**}\F_{d'} \subseteq \F_d$ (or equivalently, equality).
\end{definition}

\begin{proposition}
      \label{TRANS_MODEL_PROP}
      Let $\V$ be a cofibrantly generated model category with cellular fixed points, $\mathcal D$ a groupoid, and $\F$ a family of subgroups of $\mathcal D$.
      Then the category of diagrams $\V^{\mathcal D}$ has an \textit{$\F$-model structure} $\V^\G_\F$, where
      a map $f: X \to Y$ is a
      weak equivalence (resp. fibration) iff $f(d): X(d) \to Y(d)$ is so in $\V^{\mathsf{Aut}(d)}_{\F_d}$ for each $d \in \mathcal D$.

      Moreover, if $X \in \V^{\mathcal D}$ is cofibrant, it is also levelwise cofibrant.
\end{proposition}
\begin{proof}
      This is the model structure transferred along the adjunction
      \begin{equation}
            \begin{tikzcd}
                  \V^{\mathcal D} \leftrightarrows
                  \mathop{\prod}\limits_{d \in \mathcal D}\V^{\mathsf{Aut}(d)}_{\F_d}
            \end{tikzcd}
      \end{equation}
      which exists by a straightforward exercise adapting and combining the proofs of
      \cite[Thm 11.6.1]{Hir03} and \cite[Prop 2.6]{Ste16};
      the generating (trivial) cofibrations are given by
      \begin{equation}
            I^{\mathcal D}_\F = \sets{\mathcal D(d,-) \cdot_{\mathcal D(d,d)} \mathcal D(d,d)/H \cdot i}{d \in \mathcal D, i \in I, H \in \F_d}
      \end{equation}
      where $I$ is the set of generating (trivial) cofibrations in $\V$.
\end{proof}

We note that for any $G$-operad with $\mathfrak C = \set{\**}$, each level $\O(n)$ has a left action by $G \times \Sigma_n$.
In the homotopy theory of $G$-operads, as indicated in \cite{BH15} and the main result of \cite{BP_geo}, it is
necessary to consider the homotopy type of various fixed point spaces.
% for particular subgroups of $G \times \Sigma_n$.
% not all of them, and not just the subgroups of $G$, but instead the graph subgroups.
For full generality, we introduce the following terminology.

\begin{definition}
      \label{GSFAM_DEF}
      A \textit{$(G, \Sigma)$-family} is a collection $\F = \set{\F_n}_{n \geq 0}$ of families $\F_n$ of subgroups of $G \times \Sigma_n$.
      We say $\F$ \textit{has units} if $H \in \F_1$ for all $H \leq G$. \footnote{
        As in \cite{BP_geo} and \cite{BH15}, one often restricts to \textit{$G$-graph families}; see Definition \ref{COR_GGRAPH_DEF}.}.
      
\end{definition}

% % ------------------------------ we don't need $G$-graph systems here... everything in this part works for any $(G,\Sigma)$-family ------------------------------
% {\color{OliveGreen} % ------------------------------ OLIVE GREEN ------------------------------
  
%   \begin{definition}
%         \label{GGRAPHFAM_DEF}
%         A subgroup $\Gamma \leq G \times \Sigma_n$ is called a \textit{graph subgroup} if $\Gamma \cap \Sigma_n = \**$;
%         equivalently, if $\Gamma$ is the graph $\Gamma(\phi)$ of a homomorphism $G \leq H \xrightarrow{\phi} \Sigma_n$.
        
%         A \textit{$G$-graph system} is a collection $\F = \set{\F_n}_{n \geq 0}$ of families $\F_n$ of graph subgroups of $G \times \Sigma_n$, one family for each $n$.
%         The maximum example will be denoted $\mathrm{Gr} = \set{\mathrm{Gr}_n}$, with $\mathrm{Gr}_n$ the entire collection of graph subgroups of $G \times \Sigma_n$.
%   \end{definition}
% } % ------------------------------------------------------------------------------------------------

Extending this discussion to the many-colored setting, we note that
any $(G,\Sigma)$-family automatically extends to a family for the groupoid $\mathcal D = G \ltimes \Sigma_{\mathfrak C}^{op}$,
by defining for any $\mathfrak C$-signature $C = (c_1,\dots,c_n;c_0)$ of length $n+1$,
\begin{equation}
      \label{FAMC_DEF_EQ}
      \F_C = \F_n \cap \Aut_{G \ltimes \Sigma_{\mathfrak C}}(C)
\end{equation}
the subgroups $\Gamma \in \F_n$ such that $\Gamma \leq \Aut_{G \times \Sigma_n}(C)$.
Applying Proposition \ref{TRANS_MODEL_PROP} induces an $\F$-model structure on the category of $\mathfrak C$-symmetric sequences in $\V$.

\begin{definition}
      Let $\V$ be a cofibrantly generated model category with cellular fixed points,
      $\mathfrak C$ a $G$-set, and $\F$ a $(G, \Sigma)$-family.
      Then the \textit{$\F$-model structure} on the fiber $\Sym^{G,\mathfrak C}(\V)$ of $\Sym(\V)$ over $\mathfrak C$
      is the $\F$-model structure
      \begin{equation}
            \Sym^{G,\mathfrak C}_\F(\V) = \V^{G \ltimes \Sigma_{\mathfrak C}^{op}}_{\F}
      \end{equation}
      where $\F$ on the right-hand-side is the associated family of subgroups of $G \ltimes \Sigma_{\mathfrak C}^{op}$.
      % $\V^{B_{\mathfrak C^{\times n+1}}(G \times \Sigma_n)}_{\F_n}$ has the model structure lifted from the adjunction to
      % $\prod_{(c_i)}\V^{\mathsf{Aut}(c_1,\dots,c_n; c_0)}_{\F_{(c_i)}}$ where
      % $\F_{(c_i)}$ is the set of all $\Gamma \in \F_n$ such that $\Gamma \leq \mathsf{Aut}(c_1,\dots, c_n;c_0)$.
\end{definition}

% On the other hand, let $\Sym_{\F, \mathfrak C}(\V)$ denote the category $\V^{\UC\Sigma_\F}$ with the projective model structure.

% \begin{lemma}
%       \label{EXMAIN_LEM}
%       Let $\V$ be {\color{red} GOOD}.
%       Let $\P \in \Sym_{G, \mathfrak C}(\V)$ be level genuine cofibrant, and
%       $u: X\ to Y$ in $\Sym_{G, \mathfrak C}(\V)$ be a level genuine cofibration.
%       Then, for each $(T, \mathfrak c) \in \UC \Omega_G^a [k]$, and writing $C = \mathsf{lr}(T)$, the map
%       \begin{equation}
%             \left(
%                   \bigotimes_{v \in V_G^{ac}(T)}\P(T_v, \mathfrak c) \otimes
%                   {\mathlarger{\mathlarger{\mathlarger{\square}}}}u(T_v, \mathfrak d_v)
%             \right)
%             \mathop{\otimes}\limits_{\Aut(T, \mathfrak c)}\Aut(C, \mathfrak c)
%       \end{equation}
%       is a genuine cofibration in $\V^{\Aut(C, \mathfrak c)}_{\mbox{gen}}$,
%       which is trivial if $k \geq 1$ and $u$ is trivial.
% \end{lemma}
% \begin{proof}
%       This follows from \cite[Prop 6.24]{BP_geo} analogously as \cite[Lemma 5.72]{BP_geo} did.
% \end{proof}

As in \cite{BP_geo}, we will transfer this model structure along the free-forgetful adjunction
\[
      \Op^{G, \mathfrak C}_\F(\V) \leftrightarrows \Sym^{G, \mathfrak C}_\F: \mathbb F^G_{\mathfrak C}
\]
where $\mathbb F^G_{\mathfrak C}$ is the fiber of the cartesian monad $\mathbb F^G$ over $\mathfrak C$,
described explicitly in \eqref{FGC_EQ}.

This transfer will follow much in the same was as in \cite{BP_geo};
we relegate the technical discussion to Appendix \ref{MONAD_APDX}.

We do recall the necessary additional hypotheses required.

\begin{definition}[{\cite[Defn 6.16]{BP_geo}}]
      \label{CSPP_DEF}
      We say a symmetric monoidal model category $\V$ has \textit{cofibrant symmetric pushout powers} if
      for all (trivial) cofibrations $f$, the pushout product power $f^{\square n}$
      is a $\Sigma_n$-genuine (trivial) cofibration in $\V^{\Sigma_n}$. 
\end{definition}


% We say $X \to Y \in \Sym^{G, \mathfrak C}(\V)$ is a \textit{level genuine cofibration} if for each $C \in \G \ltimes \SC^{op}$,
% $X(C) \to Y(C)$ is a genuine cofibration in $\V^{\Aut(C)}$;
% we say $X$ is \textit{level genuine cofibrant} if $\varnothing \to X$ is a level genuine cofibration.

% \begin{lemma}[{cf. \cite[5.73]{BP_geo}}] % EXMAINLEM LEM
%       \label{EXMAINLEM LEM}
%       Suppose $\mathcal{V}$ is a cofibrantly generated closed monoidal model category
%       \todo{do we need to introduce monoidal model categories again? I wouldn't think so}
%       with cellular fixed points and
%       with cofibrant symmetric pushout powers (Defn \ref{CSPP_DEF}).
      
%       Let $\O \in \mathsf{Sym}^{G,\mathfrak C}(\mathcal{V})$
%       be level genuine cofibrant
%       and  
%       $u: X \to Y$ in $\Sym^{G, \mathfrak C}(\V)$ a level genuine cofibration. 
%       Then for each $T \in G \ltimes \Omega^a[k]^{op}$ and writing
%       $C = \mathsf{lr}(T)$, the map	
%       \begin{equation}\label{EXMAINLEM EQ}
%             \left(
%                   \bigotimes\limits_{v \in V^{ac}(T)}\P(T_v) \otimes
%                   \underset{v \in V^{in}(T)}
%                   {\mathlarger{\mathlarger{\mathlarger{\square}}}}
%                   u(T_v)
%             \right) 
%             \mathop{\otimes}\limits_{\mathsf{Aut}(T)} \mathsf{Aut}(C).
%       \end{equation}
%       is a genuine cofibration in 
%       $\mathcal{V}^{\mathsf{Aut}(C)}_{\text{gen}}$,
%       which is trivial if $k \geq 1$ and $u$ is trivial.	
% \end{lemma}
% \begin{proof}
%       This follows as in \cite{BP_geo}, as we may take the automorphism groups to live in the category $F \wr (G^{op} \ltimes \Sigma_{\mathfrak C}).$ 
% \end{proof}

\begin{proof}[Proof of Theorem \ref{THM1_C}]
      Proposition \ref{FILTPUSH PROP} shows that we have an appropriate filtration of free extensions of colored operads,
      while Lemma \ref{EXMAINLEM LEM} says that this filtration is levelwise homotopically well-behaved.
      Thus Theorem \ref{THM1_C} follows completely analogously to the proof of Theorem I from \cite{BP_geo}.
\end{proof}



\begin{remark}
      \label{TOP_FULL_REM}
      The category $\Op^{G,\mathfrak C}(\Top)$ actually has full model structures lifted from $\Sym^{G, \mathfrak C}_\F(\Top)$
      for any $(G, \Sigma)$-family $\F$,
      using an argument analogous to \cite[Thm. 3.1]{GW}.
\end{remark}

We record a particular step in the proof of Theorem \ref{THM1_C}.
\begin{corollary}
      \label{LGC_COR}
      Suppose $\V$ has the hypotheses of Theorem \ref{THM1_C},
      let $\mathfrak C$ be a $G$-set, and $f: \O \to \P$ a map in $\Op^{G,\mathfrak C}(\V)$.
      If $f$ is a (trivial) cofibration in $\Op^{G, \mathfrak C}_\F(\V)$ for some indexing family $\F$,
      and $\O$ is level genuine cofibrant, then
      $f(C)$ is a genuine (trivial) cofibration in $\V^{\Aut(C)}_{gen}$ for all $\mathfrak C$-signatures $\ksi$.
\end{corollary}
% \begin{proof}
%       It suffices to show that the map $\O \to \O[u]$ given by the pushout
%       \begin{equation}
%             \begin{tikzcd}
%                   \mathbb F^{\mathfrak C}(X) \arrow[r] \arrow[d, "u"']
%                   &
%                   \O \arrow[d]
%                   \\
%                   \mathbb F^{\mathfrak C}(Y) \arrow[r]
%                   &
%                   \O[u],
%             \end{tikzcd}
%       \end{equation}
%       with $u: X \to Y$ a generating (trivial) $\F$-cofibration in $\Sym^{G, \mathfrak C}_\F(\V)$
%       (and so in particular a local genuine (trivial) cofibration),
%       is a local genuine (trivial) cofibration.
      
%       Recalling the filtration in \eqref{FILT_PUSHG_EQ}, % We have/will show {\color{red} COME BACK} that we have a filtration
%       % \begin{equation}
%       %       \O = \O_0 \to \O_1 \to \dots \to \colim_n \O_n = \O[u]
%       % \end{equation}
%       % where locally we have $\O_{n-1}(\ksi) \to \O_n(\ksi)$ is given by the pushout over
%       % \begin{gather*}
%       %       \Lan_{\left(G \ltimes \Omega_{\mathfrak C} \to G \ltimes \Sigma_{\mathfrak C}\right)^{op}}
%       %       \left(
%       %             \bigotimes_{v \in V^{ac}(T)}\O(T_v, \mathfrak d) \otimes \mathop{\square}\limits_{v \in V^{in}(T)}u(T_v, mathfrak d)
%       %       \right)
%       %       \\
%       %       \simeq
%       %       \coprod_{[T] \in \Iso\left( C \downarrow_r G \ltimes \Omega^a[k] \right)}
%       %       \left(
%       %             \bigotimes_{v \in V^{ac}(T)}\O(T_v, \mathfrak d) \otimes
%       %             \left(
%       %                   Q^{in}_T[u] \to \bigotimes_{v \in V^{in}(T)}Y(T_v, \mathfrak d)
%       %             \right)
%       %       \right)
%       %       \otimes_{\Aut(T, \mathfrak d)} \Aut(C, \mathfrak c),
%       % \end{gather*}
%       % where $(C, \mathfrak c) = (C_{|\ksi|}, i \mapsto \ksi_i)$.
%       by \cite[Remarks 5.72(i) and 5.72(ii)]{BP_geo}, it suffices to check that the two maps
%       \begin{gather*}
%             \left(\varnothing \to \bigotimes_{v \in V^{ac}(T)}\O(T_v)\right)
%             =
%             \mathop{\square}\limits_{v \in V^{ac}(T)}(\varnothing \to \O)(T_v),
%             \qquad
%             \mathop{\square}\limits_{v \in V^{in}(T)}u(T_v)
%       \end{gather*}
%       are $\Aut\left((T_v)_{v \in V^{ac}(T)}\right)$- and $\Aut\left(((T_v)_{v \in V^{in}(T)}\right))$-genuine cofibration,
%       with the latter trivial if $u$ is.
%       These automophism groups are taken in $\Sigma \wr \SC$, the codomain of the vertex functor,
%       and are of the form $\prod_i \Sigma_{\lambda_i} \wr \Aut_{\SC}(T_{v_i})$ for the partition
%       $\lambda$ of $V(T)$ where with two vertices in the same class iff they are isomorphic in $\SC$.
%       % CYAN = With $G \ltimes (-)$
%       % {\color{cyan}
%       %   These automorphism groups are taken in $G \ltimes (\Sigma \wr \Sigma_{\mathfrak C})$, the codomain of the vertex functor.
%       %   These groups inject into $\Sigma \wr (G \ltimes \Sigma_{\mathfrak C})$ along the diagonal,
%       % and here the automorphism group is of the form
%       % $\Pi_i \Sigma_{|\lambda_i}| \wr \Aut(T_{v_i}, \mathfrak d)$.}    
%       Thus the result follows by applying, in order, Remark 5.72(ii), Proposition 6.24, and Remark 5.72(i) of \cite{BP_geo}.     
%       % \todo[inline]{come back: reference the used equations when they appear in hypothetical earlier discussion}      
% \end{proof}


In both Theorem \ref{THM1_C} and Corollary \ref{LGC_COR}, as well as in \cite{BP_geo},
the cofibrancy of the unit is necessary so that the initial operad $\mathbb F(\varnothing)$ is level genuine cofibrant.

\begin{remark}
      \label{CATV_MC_REM}
      As a corollary, we get (semi)-model structures on $\Cat^{G, \mathfrak C}(\V) = \Op^{G, \mathfrak C}(\V) \downarrow \**$,
      so in particular we have cofibrant replacements in $\Cat^{\mathfrak C}(\V)$ for any set $\mathfrak C$.
\end{remark}










The following is immediate.
\todo[inline]{the above are not in context, as they have no mention of the change-of-color monad work of Luis}
\begin{corollary}
      \label{COLOR_CHANGE_Q_COR}
      The change of color adjunctions from \eqref{GC_CHANGE_EQ},
      and the forgetful functor $j^{\**}$ from \eqref{JSTAR_CAT_EQ}
      form Quillen adjunctions.
\end{corollary}
% \begin{proof}
%       For any $F: \mathfrak C \to \mathfrak C'$, $F_{\**}$ clearly preserves (trivial) fibrations.
% \end{proof}

\begin{remark}
      In particular, this implies that $j^{\**}$ commutes with fibrant replacement.
\end{remark}





















\newpage

\section{Model structure for all colors}
\label{MS_SEC}
\renewcommand{\C}{\mathfrak C}

% \todo[inline]{check use of color-sq stuff}

In the previous section, we built model structures on each category operads with a single fixed color $G$-set $\mathfrak C$,
i.e. on each fiber of $\Op^G(\V) \to \Fin^G$. 
We will now generalize and synthesize \cite{BM13}, \cite{Cav}, and \cite{CM13b} to assemble these
into a single model structure on the full category of $G$-operads $\Op^G(\V)$ with varying colors.


% \begin{definition}
%       We say $\V$ \textit{admits (semi)-transfer for categories/operads} if for any set $\mathfrak C$,
%       the categories $\Cat^{\mathfrak C}(\V)$, $\Op^{\mathfrak C}(\V)$
%       may be equipped with the canonical (semi)-model structure.
% \end{definition}

% \begin{remark}
%       If $\V$ satisfies the \textit{monoid axiom} of \cite{SS00}
%       or has a monoidal fibrant replacement functor and a comonoidal Hopf interval object \cite{BM03},
%       then $\V$ admits transfers for categories and operads by \cite{Mur11,Mur14} and \cite{BM03}, respectively.
% \end{remark}

\begin{remark}
      We could have instead chosen to follow \cite{Mur15}, and in fact
      fully expect that the analogous statements would hold true.
\end{remark}

\begin{convention}
      \label{ALLCOLOR_CONV}
      For this section, we assume $\V$ is always a cofibrantly generated closed monoidal model category with
      cofibrant unit, cellular fixed points, and cofibrant symmetric pushout powers.
      In particular, Theorem \ref{THM1_C} and Remark \ref{CATV_MC_REM} imply that $\Cat^{\mathfrak C}(\V)$ is a (semi)-model category for each set $\mathfrak C$,
      and thus in particular has cofibrant replacements.

      We will need to add a small set-theoretic condition below in Convention \ref{GENINT_CONV}.
\end{convention}


% \begin{definition}
%       Given $\V$ with cellular fixed points, we say
%       $\V$ \textit{admits (semi)-transfers for $G$-categories/operads} if
%       for any $G$-set $\mathfrak C$ and indexing collection $\F$,
%       the categories $\Cat^{G, \mathfrak C}(\V)$, $\Op^{G, \mathfrak C}(\V)$
%       may be equipped with the canonical $\F$-(semi)-model structure.
% \end{definition}

% \begin{example}
%       By Theorem \ref{THM1_C}, $(\sSet, \times)$ and $(\sSet_{\**}, \wedge)$ admit transfers for $G$-operads,
%       and any $(\V, \otimes)$ satisfying the hypotheses of said theorem admits semi-transfers for $G$-operads.
% \end{example}

% \begin{example}
%       \cite[Theorem 3.1]{GW} shows that $(\Top, \times)$ admits transfers for $G$-operads when $\mathfrak C = \set{*}$.
% \end{example}

% \textbf{\color{OliveGreen} In this section, further assume that $\V$
%   has cellular fixed points and admits (semi)-transfer for $G$-categories and $G$-operads}.

The model structure on $\Op^G(\V)$ is characterized by the fibrations and weak equivalences:
weak equivalences will be level equivalences with some homotopical notion of essential surjectivity,
while fibrations will be level fibrations with a homotopical isofibration property.

We begin by formalizing these notions, starting with some mild enriched category theory.
We highlight three particular $\V$-categories.
As usual, we let $1_\V$ and $\varnothing$ denote the unit object and initial object of $\V$, respectively.
\begin{itemize} %{enumerate}[label = (\roman*)]
\item Let $\I$ be the $\V$-category that detects isomorphisms: it has objects $\set{0,1}$,
      with $\I(0,0) = \I(0,1) = \I(1,0) = \I(1,1) = 1_\V$.
\item Let $\mathbb A$ be the $\V$-category that detects arrows: it has objects $\set{0,1}$,
      with $\mathbb A(0,0) = \mathbb A(0,1) = \mathbb A(1,1) = 1_\V$, and $\mathbb A(1,0) = \varnothing$.
\item Let $\1$ be the $\V$-category that detects objects: it has a single object $\set{\**}$, with $\1(\**, \**) = 1_\V$
      \footnote{
        This notation matches the fact that this category is equal to the image of the ``stick'' tree $\eta \in \Omega$.}.
\end{itemize}
%\end{definition}

We recall and extend the following definitions from \cite{BM13}. 

\begin{definition}
      A {\em $\V$-interval} is a cofibrant object $\J$ in $\Cat^{\set{0,1}}(\V)$ (with the transfered model structure)
      for which there exists a weak equivalence $\J \to \I_f$.
      A set $\mathcal{G}$ of $\V$-intervals is {\em generating} if all $\V$-intervals $\J$ can be obtained
      as a retract of a trivial extension of an element in $\mathcal{G}$ in $\Cat^{\set{0,1}}(\V)$:
      \begin{equation}
            \begin{tikzcd}
                  \mathbb{G} \arrow[r,rightarrowtail, "\sim"]
                  &
                  \mathbb{K} \arrow[r,yshift=-.3em, "r"']
                  &
                  \mathbb{J} \arrow[l,yshift=.3em, "i"']
            \end{tikzcd}
      \end{equation}
\end{definition}

\begin{remark}
      In \cite{BM13}, $\J$ just needs to be weakly equivalent to $\I$.
      However, any zig-zag of weak-equivalences between any $A$ and $X$ with $A$ cofibrant
      can be lifted to a single map $A \to X_f$.
      % We illustrate with a single example:
      % given $A \to Y \leftarrow X$, $A$ maps to $Y_c$, and then using the functoriality of $(-)_c$ on $Y \leftarrow X$, we get
      % $Y_c \twoheadleftarrow X_c' \leftarrowtail X_c$.
      % But then we have a lift $Y_c \to X_c'$, and a lift $X_c' \to X_f$, yielding our weak equivalence $A \to X_f$.      
\end{remark}

\begin{remark}
      The prototypical example of a $\V$-interval is $J = N(0 \leftrightarrows 1) \in \sSet$ the nerve of the walking isomorphism; see the proof of Proposition \ref{SSET_COH_PROP}.
\end{remark}

Morally, $\V$-intervals detect ``homotopical isomorphisms'' in any $\V$-cat $\mathcal C$; cf. Definition \ref{EQUIV_DEF} below.
Additionally, we use these $\V$-intervals to define homotopical analogues of isofibrations and surjective maps of $\V$-categories.

\begin{definition}
      \label{PL_ES_DEFN}
      We say a functor $F: \mathcal C \to \mathcal D$ in $\Cat(\V)$ is
      \begin{itemize} %{enumerate}[label = (\roman*)]
      \item \textit{path-lifting}
            if it has the right lifting property against all maps of the form
            $\1 \to \J$
            where $\J$ is a $\V$-interval.
      \item \textit{essentially surjective}
            if for any object $d: \1 \to \mathcal D$,
            there is an object $c: \1 \to \mathcal C$
            and a map $\J \to \mathcal D$ out of a $\V$-interval fitting in to the commuting diagram below.
            \begin{equation}
                  \label{ESSURJ_EQ}
                  \begin{tikzcd}
                        \1 \arrow[rr, dashed, "c"] \arrow[dr, "i_0"]
                        &&
                        \mathcal C \arrow[dd, "F"]
                        \\
                        &
                        \J \arrow[dr, dashed]
                        \\
                        \1 \arrow[ur, " i_1"] \arrow[rr,"d"]
                        &&
                        \mathcal D
                  \end{tikzcd}
            \end{equation}
      \end{itemize}
\end{definition}


We may now extend the categorical model theoretic notions to the equivariant context.
First, extending \eqref{JSTAR_CAT_EQ}, we have another inclusion-forgetful adjunction,      
\begin{equation}
      \label{JSTAR_EQ}
      \begin{tikzcd}
            \mathsf{Op}^G(\V) \arrow[d, "(-)^H"']
            \arrow[r, shift right, "j^*"']
            &
            \mathsf{Cat}^G(\V) \arrow[l, shift right, swap, "j_!"] \arrow[d, "(-)^H"]
            \\
            \Op(\V) \arrow[r, shift right, "j^*"']
            &
            \Cat(\V) \arrow[l, shift right, swap, "j_!"]
      \end{tikzcd}
\end{equation}
where again $j^{\**}$ commutes with $H$-fixed points and fibrant replacement.

Now, recall the notion of a $(G, \Sigma)$-family (with units) from Definition \ref{GSFAM_DEF}
and the extension to the groupoid $G \ltimes \Sigma_{\mathfrak C}^{op}$ in \eqref{FAMC_DEF_EQ}.


\begin{definition}
      \label{MODEL_DEFN}
      Fix a $(G, \Sigma)$-family $\F$.
      We call a map $F: \O \to \P$ in $\mathsf{Op}^G(\V)$
      \begin{itemize}
      \item a {\em local $\F$-fibration} (resp. {\em local weak $\F$-equivalence}) if
            $F(C): \O(C)\to \P(F(C))$
            is a fibration (resp. weak equivalence) in $\V^{\Aut(C)}_{\F_C}$ for all $C \in G \ltimes \Sigma_{\mathfrak C_\O}$
            % \in \C(\O)^{\times n+1}$ and all $n$.
      \item a {\em local trivial $\F$-fibration} if both a local $\F$-fibration and a local weak $\F$-equivalence.
      \item {\em essentially $\F$-surjective} (resp. {\em $\F$-path lifting}) if $j^*F^H$ is essentially surjective (resp. path lifting) in $\Cat(\V)$ for all $H = H \times \set{e} \in \F_1$.
      \item a {\em $\F$-fibration} if both $\F$-path lifting and a local $\F$-fibration.
      \item a {\em weak $\F$-equivalence} if both essentially $\F$-surjective and a local weak $\F$-equivalence.
      % \item a \textit{DK-$\F$-equivalence} if a local weak $\F$-equivalence such that
      %       $\pi_0 j^{\**}F^H$ (cf. Definition \ref{HTPY_DEFN}) is essentially surjective.
      \item a \textit{(trivial) $\F$-cofibration} if it has the left lifting property against all trivial $\F$-fibrations (resp. $\F$-fibrations).
      \end{itemize}
\end{definition}

\begin{remark}
      \label{GRAPHF_REM}
      If $\F = \mathrm{Gr}$ is the $(G, \Sigma)$-family composed of all \textit{graph subgroups} $\Gamma \leq G \times \Sigma_n$,
      we refer to $\F$-equivalences (resp. $\F$-fibrations, etc) as \textit{graph} equivalences (fibrations, etc).
\end{remark}

% \begin{remark}
%       If $\V$ has diagonals, then $F \in \Op^G(\V)$ is a $DK$-$\F$-equivalence iff
%       $F$ is a local weak $\F$-equivalence such that 
%       the associated map of \textit{$\F$-genuine equivariant operads} under the composite
%       \begin{equation}
%             \Op^G(\V) \to \Op_\F(\V) \xrightarrow{\pi_0} \Op_\F(\Set) 
%       \end{equation}
%       is an equivalence.
% \end{remark}

% \begin{remark}
%       Trivial $\F$-fibrations are precisely local $\F$-fibrations which are surjective on objects.
%       Thus $\F$-cofibrations are $f: \O \to \P$ such that
%       each $\mathfrak C(\O)^H \to \mathfrak C(\P)^H$ is injective and
%       $f_!\O \to \P$ is an $\F$-cofibration in $\Op^{G, \mathfrak C(\P)}(\V)$.
% \end{remark}

% ----------------------------------------------------------------------------------------------------
% ---------------------------------------- HAS UNITS? ----------------------------------------
% The main result of this section applies to a large class of $G$-graph systems, with only the following minor condition.

% \begin{definition}
%       We say a $G$-graph system $\F$ \textit{has units} if
%       $\F_1$ contains all graph subgroups of the form $H \leq G \times \Sigma_1$.
%       % In particular, \cite[Remark 4.50]{BP_geo} implies that any \textit{weak indexing system} has units.
% \end{definition}

We can now state and outline the proof of the main result of this section,
which constitutes most of the work towards Theorem \ref{INTRO_MODEL_THM}. % (see the disucssion after Theorem \ref{INTRO_MODEL_THM}).

\begin{theorem}
      \label{MODEL_THM}
      Fix a $(G, \Sigma)$-family $\F = \set{\F_n}$ with units,
      and let $(\V, \otimes)$ denote either $(\sSet, \times)$ or $(\sSet_{\**}, \wedge)$.
      Then there exists a cofibrantly generated model structure on the category $\Op^G(\V)$,
      denoted $\Op^G_\F(\V)$, with
      weak $\F$-equivalences, $\F$-fibrations, and $\F$-cofibrations defined as in Definition \ref{MODEL_DEFN}.
           
      Moreover, analogous semi-model category structures $\Op^G_\F(\V)$ exist
      provided that $(\V, \otimes)$:
      \begin{enumerate}[label = (\roman*)]\itemsep-4pt
      \item is a cofibrantly generated model category,
      \item is a closed monoidal model category with cofibrant unit
            \footnote{Cofibrant unit also needed for \ref{J-CELL_PROP}.},
      \item has cellular fixed-point functors,
      \item \label{CSPP_LBL} has cofibrant symmetric pushout powers % (Defn. \ref{CSPP_DEF}),
            \footnote{Also needed for Props \ref{CAV_4.14_PROP2}, \ref{J-CELL_PROP}}, % \ref{LOCAL_COF_LEM}            
            % --------------------
      \item \label{RP_LBL} is right proper
            \footnote{Needed for Lemma \ref{RIGHTPROPER_LEM} and Lemma \ref{2OUTOF3_PROP}.},
      \item \label{GENSET_LBL} has a set $\mathbb{G}$ of generating $\V$-intervals
            \footnote{Needed so we have a \textit{set} of generating trivial cofibrations},
      \end{enumerate}
\end{theorem}
\begin{proof}
      In both cases, the (semi)-model structures $\Op_\F^{G, \mathfrak C}(\V)$ exist by Theorem \ref{THM1_C}
      (using conditions (i) -- (iv) in the second case).
      % In the first case, we have that the model category $\Op^{G,\mathfrak C}_\F(\sSet)$ exists
      % for any $G$-set $\mathfrak C$ and $G$-graph family $\F$ by Theorem \ref{THM1_C},
      % while in the second case, conditions $(i)$ -- \ref{CSPP_LBL} are sufficient to construct the
      % semi-model category $\Op^{G, \mathfrak C}_\F(\V)$ from said theorem.
      
      Moreover,
      % After this difference, the proofs of the two cases are identical, as
      every object in $\sSet^G$ or $\sSet^G_{\**}$ is genuine cofibrant by e.g. \cite[Remark 5.71]{BP_geo},
      $(iii)$ $\sSet$ and $\sSet_{\**}$ have cellular fixed-point functors by \cite[Example 2.14]{Ste16},
      $(iv)$ $\sSet$ and $\sSet_{\**}$ have cofibrant symmetric pushout powers by \cite[Remark 6.18]{BP_geo},
      \ref{RP_LBL} $\sSet$ and $\sSet_{\**}$ are right proper by e.g. \cite[Thm. 2.1.1 and Prop 4.1.1]{JT_simp},
      % by Lemma \ref{INTER_LEM} and e.g. \cite[Prop 2.1.5]{Cis06} or \cite[Lemma 1.12]{BM13},
      and
      \ref{GENSET_LBL} $\sSet$ and $\sSet_{\**}$ have a generating set of intervals
      by e.g. \cite[Lemma 1.12]{BM13};
      % \ref{TCWE_LBL} the class of genuine weak equivalences in $\mathsf{Op}^G(\sSet)$ is closed under transfinite compositions
      % by an argument analogous to \cite[Lemma 1.24]{CM13b}.
      % Now, we note that condition \ref{TCWE_LBL} proves the analogous statement for any $\F$,
      % since the transfinite composite of local $\F$-equivalences is a local $\F$-equivalence.
      thus we reduce to the second case.
      
      Since $\mathsf{Op}^G(\V)$ is complete and cocomplete, it thus suffices to prove,
      following \cite[Theorem 2.1.19]{Hov}, or \cite[Theorem 2.2.2]{WY} in the semi-model structure case, that:
      \begin{enumerate}[label = (\arabic*)]
      \item the class of weak $\F$-equivalences has the 2-out-of-3 property and is closed under retracts;
      \item the domains of $I_{\F}$ (resp. $J_{\F}$) are small relative to $I_{\F}$-cell (resp. $J_{\F}$-cell);
      \item $I_{\F}$-inj $= W\cap J_{\F}$-inj; and
      \item $J_{\F}$-cell (with cofibrant source) $\subseteq W\cap I_{\F}$-cof,
      \end{enumerate}
      where $I_\F$ and $J_\F$ are the sets \eqref{IFJF_EQ} of generating (trivial) cofibrations.
      (1) follows from Proposition \ref{2OUTOF3_PROP} and the fact that if $L$ is a retract of $F$, $L^H$ is a retract of $F^H$.
      (2) follows since colimits in $\mathsf{Op}^G(\V)$ are created in $\Op(\V)$, and it holds non-equivariantly.
      (3) follows from Lemma \ref{CAV_4.8}.
      (4) follows from Proposition \ref{J-CELL_PROP}, along with Lemma \ref{POINT_4_LEMMA} and Corollary \ref{LGC_COR}.
\end{proof}


% \begin{remark}
%       If we could show via some other method that $(\V, \otimes)$ satisfied actual transfer for $G$-operads, then
%       conditions \ref{CSPP_LBL} -- \ref{TCWE_LBL} would imply that the $\F$-model structure existed on $\Op^G(\V)$. 
% \end{remark}

\begin{remark}
      \label{OPGCV_FULL_REM}
      Following Remark \ref{OPGCV_F_JC_REM} below, if $\V$ satisfies conditions $(i) - (vi)$ above,
      and additionally we know independently that
      the semi-model structures on each $\Op^{G, \mathfrak C}(\V)$ can in fact be extended to full Quillen model structures,
      then the resulting semi-model structure on $\Op^G(\V)$ can also be extended to a full Quillen model structure,
      as Proposition \ref{J-CELL_PROP} is the only place in the proof of Theorem \ref{MODEL_THM} where this distinction arises.
\end{remark}

\begin{remark}
      This recovers the main results of \cite{BM13, Cav} for $G = \set{e}$. 
\end{remark}



The rest of this section is devoted to proving the results utilized in the proof of Theorem \ref{MODEL_THM}.
We begin with a description of the sets of generating (trivial) cofibrations.
As expected, these will be freely generated by the generating (trivial) cofibrations of $\V$.

Fix a subgroup $\Gamma \in \F_n$ of $G \times \Sigma_n$,
and define the $G$-set of colors $\mathfrak C_\Gamma$ and $\mathfrak C_\Gamma$-signature $C_0$ by
\[
      \mathfrak C_\Gamma := \Gamma \backslash (G \times \Sigma_n) \cdot_{\Sigma_n} \underline{n}_+,
      \qquad
      C_0 = ([[e,e],1],[[e,e],2],\dots,[[e,e],n];[[e,e],0]).
\]
% Let $C_0$ denote the $\mathfrak C_\Gamma$-signature $([[e,e],1],[[e,e],2],\dots,[[e,e],n];[[e,e],0])$.

We have a pair of composable free-forgetful adjunctions
\[
      \begin{tikzcd}
            \V^\Gamma \arrow[r, shift left]
            &
            \Sym^{G, \mathfrak C_\Gamma}(\V) \arrow[l, shift left, "ev_{C_0}"] \arrow[r, shift left, "\mathbb F^G_{\mathfrak C_\Gamma}"]
            &
            \Op^{G, \mathfrak C_\Gamma}(\V) \arrow[l, shift left]
      \end{tikzcd}
\]
where we note that $\Aut_{G \ltimes \Sigma_{\mathfrak C_\Gamma}}(C_0)$ is precisely $\Gamma$.
Let $\mathbb F_\Gamma$ denote the composite of left adjoints.
% ----------------------------------------------------------------------------------------------------
% ------------------------------ explicit description of first left adjoint ------------------------------
% Let $C_\Gamma[X]$ denote the $(G,\mathfrak C_\Gamma)$-symmetric sequence, defined on any signature $C \in G \ltimes \Sigma_{\mathfrak C_\Gamma}^{op}$ to be
% \begin{equation}
%       C_\Gamma[X](C) =
%       \left(\coprod_{(g,\sigma).C_0 = C} (g,\sigma)^{\**}X \right)_{/ H},
%       % \begin{cases}
%       %       (g,\sigma)^{\**} X \qquad \qquad & C = (g,\sigma).C_0
%       %       \\
%       %       \varnothing & \mbox{otherwise,}
%       % \end{cases}
% \end{equation}
% where the right $H$-action on the disjoint union (when it is non-empty) is defined as follows:
% Fix some $(g_0, \sigma_0)$ such that $(g_0,\sigma_0).C_0 = C$.
% Then any other such $(g,\sigma)$ is of the form $(g_0 h, \sigma \phi_h)$ where $\Gamma = \Gamma(\phi)$ for $\phi: H \to \Sigma_n$.
% The disjoint union can thus be indexed by $H$,
% and given $\bar h \in H$ and $(h,a) \in (g_0h,\sigma\phi_h)^{\**}X \subseteq \amalg (g,\sigma)^{\**}X$, we define the right action
% \[
%       (h,a) \cdot \bar h = \left( \bar h^{-1} h, \left(h^{-1} \bar h h, \phi_h^{-1} \phi_{\bar h} \phi_h\right) \cdot a \right).
% \]
% It is straightforward to check that this commutes with the (left) diagonal action of $\Aut(C)$ on the coproduct,
% does not depend on the choice of $g_0$, 
% and furthermore is natural in $C$.
% ----------------------------------------------------------------------------------------------------
It is then straightforward that for each $X \in \V^\Gamma$,
the operad $\mathbb F_\Gamma[X]$ has the universal property that for all $\O \in \Op^G(\V)$,
\begin{equation}
      \Hom_{\Op^G(\V)}(\mathbb F_\Gamma[X], \O) = \mathop\coprod\limits_{C \in (\mathfrak C_{\O}^{\times n+1})^\Gamma}\Hom_{\V^\Gamma}(X, \O(C)).
\end{equation}

Define $I_{\F,loc}$ and $J_{\F, loc}$ to be the sets
\begin{align*}
  \set{\mathbb F_\Gamma[\Gamma/\Gamma \cdot (A \xrightarrow{i} B)]}_{\Gamma, i}
  \qquad \mbox{ and } \qquad
  \set{\mathbb F_\Gamma[\Gamma/\Gamma \cdot (A \xrightarrow{j} B)]}_{\Gamma,j}
\end{align*}
where $\Gamma$ runs over all subgroups of $G \times \Sigma_n$ in $\F_n$,
and $i$ (resp. $j$) runs over all generating (trivial) cofibrations in $\V$.

The universal property makes the following immediate.
\begin{corollary}[{cf. \cite[Remark 4.6]{Cav}, \cite[1.16]{CM13b}}]
      $\O \to \O'$ is a local (trivial) $\F$-fibration iff
      $\O \to \O'$ has the right lifting property against $J_{\F, loc}$ (resp. $I_{\F, loc}$).
\end{corollary}

Now, define
\begin{equation}
      \label{IFJF_EQ}
      I_{\F}:= I_{\F, loc} \mathbin{\cup} \set{\varnothing \to G/H \cdot \1}_{H \in \F_1},
      \qquad \qquad
      J_{\F} := J_{\F, loc} \mathbin{\cup} \set{G/H \cdot (\1 \to \J)}_{H \in \F_1,\ \J\in\mathbb{G}}
\end{equation}
where $\1$ defined as in Definition \ref{PL_ES_DEFN}, and $\mathbb{G}$ is a generating set of $\V$-intervals. 


\begin{example}
      If $\Gamma$ is a graph subgroup, say corresponding to a map $G \geq H \xrightarrow{\alpha} \Sigma_n$, then
      $\mathfrak C_{\Gamma} \simeq G \cdot_H (A_+)$ where $A$ is the $H$-action on $n$ given by $\alpha$.

      Moreover, if $\V = \sSet$ and $C = C_A \in \Sigma_G$ is a $G$-corolla corresponding to $\Gamma$, then
      $\Omega(C) = \mathbb F_{\Gamma}(\**)$.
      In particular, $\Omega(C) \in \sOp^G$ is cofibrant for all $C \in \Sigma_G$.
      \todo[inline]{connect to Definition \ref{OT_DEF}}
      Similarly, $\Omega(T) \in \sOp^G$ is cofibrant for all $T \in \Omega_G$, as we have a pushout in $\sOp^G$
      \[
            \begin{tikzcd}
                  \displaystyle{
                    \coprod_{Ge \in \boldsymbol{E}_G(C)} \Omega(Ge)}
                  \arrow[r] \arrow[d]
                  &
                  \Omega(C) \arrow[d]
                  \\
                  \displaystyle{
                    \coprod_{Ge \in \boldsymbol{E}_G(C)} \Omega(T_e)}
                  \arrow[r]
                  &
                  \Omega(T)
            \end{tikzcd}
      \]
      for all grafting decompositions $T = C \coprod_{\boldsymbol{E}_G(C)} (T_{Ge})$.
\end{example}
\todo[inline]{$\Omega(T)$ is $\mathcal F$-cofibrant iff $T \in \Omega_{\mathcal F}$}

\begin{example}
      When $G = \**$ and $n=1$, so $\Gamma = \**$ and $\mathfrak C_\Gamma = \set{0,1}$, we have
      \[
            \mathbb F_{\Gamma}[1_\V] = \mathbb A^{op}, \qquad \mathbb F_{\Gamma}[\varnothing] = \eta \amalg \eta.
      \]

      More generally, for $G$ finite, $\Gamma = G \leq G \times \Sigma_1$, $\mathfrak C_\Gamma = \set{0,1}$,
      $\mathbb A_G := \mathbb F_{\Gamma}[1_\V]$ again has only one non-trivial hom object, namely
      $\mathbb A_G = G \cdot 1_\V$.
      For $H \leq G$, $\Gamma = H \leq G \times \Sigma_1$, $\mathbb F_{\Gamma}[1_\V] = G \cdot_H \mathbb A_H$.
      %
      Extending to higher arity operations, we see that,
      morally, cofibrations are built by freely adding an orbit's worth of operations.
      \todo[inline]{this example sucks}
\end{example}

\begin{lemma}
      [{cf. \cite[4.8]{Cav}, \cite[2.3]{BM13}, \cite[1.18]{CM13b}}]
      \label{CAV_4.8}
      Suppose $\V$ has a generating set of intervals.
      Then the following are equivalent for a map $F$ in $\Op^G(\V)$.
      \begin{enumerate}[label = (\arabic*)]
      \item $F$ is a trivial $\F$-fibration.
      \item $F$ is a local trivial $\F$-fibration such that $F^H$ is surjective on $H$-fixed colors for all $H \in \F_1$.
      \item $F$ has the right lifting property against $I_{\F}$.
      \end{enumerate}
\end{lemma}
\begin{proof}
      $(2) \Leftrightarrow (3)$ is immediate by the construction of $I_{\F}$.
      For $(1) \Leftrightarrow (2)$, we have by definition that
      $F$ is a trivial $\F$-fibration
      iff
      it is a local trivial $\F$-fibration such that $j^*F^H$ is path-lifting and essentially surjective for all $H \in \F_1$.
      \cite[2.4]{BM13} completes the proof. 
      % Moreover, right lifting against $I_{\F, loc}$ is identical to being a local trivial $\F$-fibration, while
      % lifting against $\varnothing \to G/H\otimes \1$ precisely say that $F^H$ is surjective on colors;
      % combining these observations yields the result.
\end{proof}

\begin{lemma}
      \label{CAV_4.3}
      [{cf. \cite[1.20]{CM13b}, \cite[\S 4.3]{Cav}}]
      $F$ has right lifting against $J_{\F}$ iff $F$ is an $\F$-fibration.
\end{lemma}
\begin{proof}
      Again, lifting against $J_{\F, loc}$ is identical to being a local $\F$-fibration, while lifting against $G/H \cdot (\1 \to \J)$
      is equivalent to $F^H$ lifting against $\1 \to \J$.
      The diagram after \cite[(4.3.2)]{Cav} shows that only considering intervals $\J$ in some generating set of intervals is sufficient.
\end{proof}

\begin{lemma}
      [{cf. \cite[1.19]{CM13b}}]
      \label{POINT_4_LEMMA}
      $J_{\F}\mbox{-cof} \subseteq I_{\F}\mbox{-cof}$; that is, trivial cofibrations are cofibrations.
\end{lemma}
\begin{proof}
      % It suffices to show that if $F$ has (right) lifting against $I_\F$, it has lifting aginst $J_{\F}$.
      Clearly a local trivial $\F$-fibration is a local $\F$-fibration.
      On the other hand, by locality and Remark \ref{COLOR_SQ_REM}(i),
      any cofibration in $\mathsf{Op}^{G, \mathfrak C}(\V)$ for any $G$-set $\C$
      is a cofibration when considered in $\mathsf{Op}^G(\V)$.
      Thus, since $G/H \cdot (\1 \to \1 \amalg \1)$ is a pushout of $G/H \cdot(\varnothing \to \1)$
      and hence is in $I_{\F}\mbox{-cof}$, the composite
      \begin{equation}
            \begin{tikzcd}
                  G/H \cdot \1 \arrow[r, rightarrowtail]
                  &
                  G/H \cdot (\1 \amalg \1) \arrow[r, rightarrowtail]
                  &
                  G/H \cdot \J 
            \end{tikzcd}
      \end{equation}
      is in $I_{\F}\mbox{-cof}$.
      Thus $J_\F \subseteq I_\F\mbox{-cof}$, implying the result.
\end{proof}

\begin{convention}
      \label{GENINT_CONV}
      Adding to Convention \ref{ALLCOLOR_CONV}, we will now additionally assume that $\V$ has a generating set of intervals.
      This is a small condition; in particular, \cite[Lemma 1.12]{BM13} says that this holds for every combinatorial monoidal model category.
\end{convention}

\subsection{Trivial cofibrations}

Similar to many cases in the literature, the two most difficult steps in the proof of Theorem \ref{MODEL_THM} are showing that
$J_\F$-cells are weak equivalences, and that weak equivalences satisfy 2-out-of-3.
We show these results in that order.

The first of these two results, Proposition \ref{J-CELL_PROP}, uses the following straightforward lemmas.

\begin{lemma}
      \label{TRANSCOMP_ES_LEM}
      Transfinite composition of essentially surjective maps in $\Op^G(\V)$ is essentially surjective.
\end{lemma}
\begin{proof}
      Since taking fixed points commutes with filtered colimits, they commute with transfinite composition,
      and hence by \cite[4.17]{Cav}, we are done.
\end{proof}

\begin{lemma}
      \label{TRANSCOMP_LGC_LEM}
      Local genuine cofibrations in $\Op^G(\V)$ are closed under transfinite composition.
\end{lemma}

\begin{proof}
      Since for any $\mathfrak C$-signature $C \in G \ltimes \SC^{op}$ and any map of $\mathfrak C$-colored operads $f$ we have that
      the restriction map
      $\V^{\Aut(f(C))}_{gen} \to \V^{\Aut(C)}_{gen}$
      is left Quillen,
      and any transfinite composition of operads locally is of the form
      \begin{equation}
            \O_0(C) \to \O_1(F_1(C)) \to \O_2(F_2(C)) \to \dots   
      \end{equation}
      in $\V^{\Aut(C)}_{gen}$
      for $F_\alpha$ the (transfinite) composite $\O_0 \to \O_1 \to \dots \to \O_\alpha$,
      the result follows.
\end{proof}

\begin{proposition}
      [{c.f. \cite[4.20]{Cav}}]
      \label{J-CELL_PROP}
      Suppose $\V$ is as in Convention \ref{ALLCOLOR_CONV}. %has a cofibrant unit $1_\V$ and has cofibrant symmetric pushout powers.
      Then relative $J_{\F}$-cells with locally genuine cofibrant source are weak equivalences.
\end{proposition}
\begin{proof}
      By Lemmas \ref{TRANSCOMP_ES_LEM} and \ref{TRANSCOMP_LGC_LEM}, it suffices to show that
      the pushout of a map $j \in J_\F$ is both
      essentially surjective and a local genuine trivial cofibration.

      Firstly, if $j = \mathbb F_\Gamma[\Gamma/\Gamma \cdot i] \in J_{\F, loc}$,
      then by Remark \ref{COLOR_SQ_REM}(ii) the pushout can be computed in a fixed-color category $\mathsf{Op}^{G,\mathfrak C_{\P}}(\V)$.
      By Corollary \ref{COLOR_CHANGE_Q_COR}, the relevent span is a trivial cofibration in one leg, while the other leg has a locally cofibrant target.
      Thus by the existance of the $\F$-semi-model structures from Theorem \ref{THM1_C},
      the pushout is again a trivial cofibration, and hence by Corollary \ref{LGC_COR} a local genuine trivial cofibration.
      As it is the identity on colors, it is also essentially surjective.

      Secondly, supppose $j$ is of the form $G/H \cdot (\1 \to \J)$ for $\J$ a $\V$-interval.
      We split this pushout into a composition of two pushouts
      \begin{equation}
            \begin{tikzcd}
                  G/H \cdot \1 \arrow[r, "a"] \arrow[d, "G/H \cdot \phi"']
                  % \arrow[dr,phantom, yshift=.1em, xshift=.5em, "\lrcorner" near end]
                  &
                  \O \arrow[d,"\phi'"]
                  \\
                  G/H \cdot \J_{\set{0}} \arrow[r] \arrow[d, "G/H \cdot \psi"']
                  % \arrow[dr,phantom, yshift=.1em, xshift=.5em, "\lrcorner" near end]
                  &
                  \O' \arrow[d,"\psi'"]
                  \\
                  G/H \cdot \J \arrow[r]
                  &
                  \P
            \end{tikzcd}
      \end{equation}
      where $\J_{\set{0}}$ is the full subcategory of $\J$ spanned by the object $0$.
      It suffices to show both $\psi'$ and $\phi'$ are local genuine trivial cofibrations which are essentially surjective on fixed points. 

      We first consider the bottom pushout.
      We know that $\psi$ is injective on colors and a local isomorphism in $\Op(\V)$,
      and hence so is $G/H \cdot \psi$ in $\Op^G(\V)$.
      Since colimits are created non-equivariantly, and equivariant isomorphisms are detected by invertible equivariant maps,
      Corollary \ref{LOCALISO_COR} below implies that $\psi'$ is also a local isomorphism in $\Op^G(\V)$ \footnote{
        This also follows from the more general result \cite[Prop. B.22]{Cav},
        and from \cite[Prop. 1.28]{CM13b} in the case $(\V, \otimes) = (\sSet, \times)$.
      }.
      % so in particular a local trivial $\F$-cofibration.
      
      Moreover, we observe that $\C_{\P} = \C_{\O'} \amalg (G/H \times \set{1})$.
      Thus, if $x \in \C_{\P}^K$ is in $\C_{\O'}$ for some $K \leq G$, we have essential surjectivity trivially,
      as shown on the left below in \eqref{J-CELL_EQ},
      where $\I_c \to \I$ is a cofibrant replacement in $\Op^{\set{0,1}}(\V)$.
      %
      \begin{equation}
            \label{J-CELL_EQ}
            \begin{tikzcd}
                  \1 \arrow[rrr, "x"] \arrow[dr, " i_0"]
                  &&&
                  (\O')^K \arrow[dd, "\psi'"]
                  &[15pt] % ----------
                  \1 \arrow[r, "0"] \arrow[dr, "i_0"']
                  &
                  \J_{\set{0}} \arrow[d] \arrow[r, "g"]
                  &
                  (\O')^K \arrow[d, "{\psi'}"]
                  \\
                  &
                  \I_c \arrow[r]
                  &
                  \I \arrow[dr, "x"]
                  &
                  & % ----------
                  &
                  \J \arrow[r, "g"]
                  &
                  \P^K \arrow[d, equal]
                  \\
                  \1 \arrow[ur, " i_1"] \arrow[rrr,"x"]
                  &&&
                  (\P)^K
                  & % ----------
                  \1 \arrow[rr, "g \cdot 1"] \arrow[ur, "i_1"]
                  &&
                  \P^K
            \end{tikzcd}
      \end{equation}
      %
      If instead $x  = g \cdot 1 \in (G/H \cdot 1)^K \subseteq \mathfrak C_\P^K$,
      the pushout square yields the diagram on the right above in \eqref{J-CELL_EQ},
      where the maps $\J_{\set{0}} \xrightarrow{g} (\O')^K$, $\J \xrightarrow{g} \P^K$ are adjoint to the composites
      \begin{equation}
            G/K \cdot \J_{\set{0}} \xrightarrow{g} G/H \cdot \J_{\set{0}} \longto \O',
            \qquad \qquad
            G/K \cdot \J \xrightarrow{g} G/H \cdot \J \longto \P
      \end{equation}
      (using that $(G/H)^K \simeq \Hom(G/K, G/H)$).
      % Lastly, if we consider (any element in the orbit of) the new object $1\in \C(\P)^H$,
      % there is an associated object $0 \in \C(\O')^H$ such that the essentially surjectivity diagram
      % factors through the pushout diagram for $\psi$:
      % \begin{equation}
      %       \begin{tikzcd}
      %             G/H \cdot \1 \arrow[r,"0"] \arrow[dr, "G/H \cdot i_0"']
      %             &
      %             G/H \cdot \J_{\set{0}} \arrow[r] \arrow[d]
      %             &
      %             \O' \arrow[d, "\psi'"]
      %             \\
      %             &
      %             G/H \cdot \J \arrow[r]
      %             &
      %             \P \arrow[d, equal]
      %             \\
      %             G/H \cdot \1 \arrow[ur, "G/H \cdot i_1"] \arrow[rr, "1"]
      %             &&
      %             \P.
      %       \end{tikzcd}
      % \end{equation}
      Hence $\psi'$ is also essentially surjective.

      Now, consider the top pushout. Remark \ref{COLOR_SQ_REM}(ii) again implies that this pushout is created in $\Op^{G, \mathfrak C_\O}(\V)$.
      In particular, this implies $\phi'$ is bijective on objects, and hence essentially surjective.
      Further, since $1_\V$ is cofibrant in $\V$, \cite[Thm. 1.15]{BM13} implies that $\J_{\set 0}$ is cofibrant in $\Op^{\**}(\V)$,
      and since $\1$ is the initial object here, $\phi$ is a trivial cofibration here.
      Thus $a_! (G/H \cdot \phi)$ is a trivial $\F$-cofibration in $\Op^{G, \mathfrak C_\O}(\V)$ by Corollary \ref{COLOR_CHANGE_Q_COR}.
      % {(as $G/H \cdot \phi$ is one in $\Op^{G, G/H}(\V)$,
      %   since $\O \to \P$ a trivial $\F$-fibration in $\Op^{G, \mathfrak C}(\V)$
      %   implies $j^{\**}\O^H \to j^{\**}\P^H$ is one in $\Cat^{\mathfrak C^H}(\V)$)}.
      Hence, again using the $\F$-semi-model structure on $\mathsf{Op}^{G, \mathfrak C_\O}(\V)$ and the fact that $\O$ is locally cofibrant,
      $\phi'$ is a trivial $\F$-cofibration in $\mathsf{Op}^{G,\C_\O}(\V)$,
      and thus a local genuine trivial cofibration by Corollary \ref{LGC_COR}.
      
      Since both $\phi'$ and $\psi'$ are essentially surjective and local genuine trivial cofibrations,
      the result is proved.
\end{proof}

\begin{remark}
      \label{OPGCV_F_JC_REM}
      If for an independent reason we know that each $\mathsf{Op}^{G, \mathfrak C}(\V)$ had a full Quillen model structure,
      then the above proof would show that \textit{all} relative $J_\F$-cells are weak equivalences.
\end{remark}





\subsection{Homotopy in a general model category}

Before proving 2-out-of-3 for weak equivalences, we make a brief technical digression about homotopies in any $\V$-category $\mathcal C$.
First, we recall the following about homotopies in general model categories.
\begin{definition}
      For any $A \in \V$, a \textit{cylinder object for $A$} is an object $\mathbb C(A)$ equipped with a factorization of the fold map
      \begin{equation}
            \begin{tikzcd}
                  A \amalg A \arrow[r, tail, "{(i_1,i_2)}"]
                  &
                  \mathbb C(A) \arrow[r, "\simeq"]
                  &
                  A
            \end{tikzcd}
      \end{equation}
      into a cofibration followed by a weak equivalence.
      
      For the tensor unit $1_\V$, we write $\mathbb C = \mathbb C(1_\V)$, and call this simply a \textit{cylinder} in $\V$.
      % A cylinder for $A$ is called \textit{good} if the second map is a trivial fibration.
      
      A (left) \textit{homotopy} between maps $f,g: A \to B$ in $\V$ is a map $H_{fg}: \mathbb C(A) \to \V(A,B)$ such that
      the diagram below commutes.
      \begin{equation}
            \begin{tikzcd}[row sep = tiny]
                  A \amalg A \arrow[rr, "{(f,g)}"] \arrow[dr]
                  &&
                  B
                  \\
                  &
                  \mathbb C(A) \arrow[ur, "H_{fg}"']
            \end{tikzcd}
      \end{equation}
      We say $f$ and $g$ are \textit{homotopic} if there exists a homotopy $H_{f g}$ between them.
\end{definition}

% \begin{remark}
%       If $B$ is fibrant, we may lift any homotopy to a homotopy out of a good cylinder,
%       using the functorial factorization
%       $\mathbb C(A) \overset{\sim}{\rightarrowtail} \mathbb C'(A) \xrightarrow{\sim}{\twoheadrightarrow} A$.
% \end{remark}

\begin{remark}
      \label{CYL_REM}
      The maps $i_\epsilon: A \to \mathbb C(A)$ are always weak equivalences by 2-out-of-3 for $\epsilon \in \set{0,1}$.
      Moreover, if $A \in \V$ is cofibrant, then it is additionally a cofibration.
      Furthermore, if $\mathbb C$ a cylinder in $\V$,
      then $A \otimes \mathbb C$ is a cylinder object for $A$,
      as the fold map can be written
      \begin{equation}
            A \amalg A \simeq A \otimes (1_\V \amalg 1_\V) \rightarrowtail A \otimes \mathbb C \xrightarrow{\sim} A
      \end{equation}
      as $A \otimes (-)$ preserves cofibrations, and, by Ken Brown's Lemma, weak equivalences between cofibrant objects.
\end{remark}

These cylinder objects provide another description of the mapping sets in the homotopy category $\Ho \V$ of $\V$.

\begin{proposition}       [{\cite[1.2.10]{Hov99}}]
      If $A$ is fibrant and $B$ cofibrant, then
      homotopy is an equivalence relation $\sim$ on $\V(A,B)$.
      Moreover, 
      $\Ho \V (A,B) = \V(A_c, B_f)/\sim$.
\end{proposition}

We can use these cylinder objects to extend the notion of homotopy to $\V$-categories or $\V$-operads.

\begin{definition}
      \label{HTPY_DEFN}
      Given $\mathcal C \in \Cat(\V)$, define $\pi_0 \mathcal C$ to be the (unenriched) category with
      the same objects as $\mathcal C$, and $\pi_0 \mathcal C (c,d) = \Ho(\V)(1_\V, \mathcal C(c,d))$.

      We say maps $f,g \in \mathcal C(c,d)$ are \textit{homotopic}
      if the representing maps $f,g: 1_\V \to \mathcal C(c,d)$ are homotopic in $\V$.
      If $\mathcal C$ is fibrant, this is equivalent to $[f] = [g]$ in $\pi_0\mathcal C$.

      For an operad $\O \in \Op(\V)$, we say operations $f,g \in \O(c;d)$ are \textit{homotopic} if they are homotopic in the underlying category $j^{\**}\O$. 
\end{definition}







% ------------------------------ ASSEMBLING HOMOTOPIES ------------------------------


We may assemble homotopies in the following manner.

\begin{lemma}
      Suppose $\V$ is as in Convention \ref{ALLCOLOR_CONV} \footnote{
        We don't need that the unit is cofibrant.}.
        % has cofibrant symmetric pushout powers and cellular fixed points.
      Let $h: Z_1 \to Z_2$ in $\V$ be a (trivial) cofibration between cofibrant objects.
      Then $h^{\otimes n}: Z_1^{\otimes n} \to Z_2^{\otimes n}$ is a genuine (trivial) cofibration in $\V^{\Sigma_n}_{gen}$.
\end{lemma}
\begin{proof}
      We apply arguments from \cite[Prop. 6.24]{BP_geo} and \cite[Lemma 4.8]{Pe16}).
      Given composable arrows $Z_0 \xrightarrow{g} Z_1 \xrightarrow{h} Z_2$ in $\V$,
      we denote by $Q^n(g), Q^n(h)$ the domains of $g^{\square n}, h^{\square n}$.
      There is a filtration of the box product of the composite $(hg)^{\square n}$ by a series of pushouts as on the left below
      \begin{equation}
            \label{HGBOX_EQ}
            \begin{tikzcd}
                  \bullet \arrow[d, "\Sigma_n \cdot_{\Sigma_{n-r} \times \Sigma_r} (g^{\square n-r} \square h^{\square r})"'] \arrow[r]
                  &
                  \bullet \arrow[d, "k_r"]
                  & &% ----------
                  Q^n(g) \arrow[r] \arrow[d, "g^{\square n}"']
                  &
                  Q^n(hg) \arrow[d, "k_0"]
                  \\
                  \bullet \arrow[r]
                  &
                  \bullet
                  & &% ----------
                  Z_1^{\otimes n} \arrow[r]
                  &
                  \bullet
            \end{tikzcd}
      \end{equation}
      for $0 \leq r \leq n$,
      built out of a filtration $P_0 \subseteq P_1 \subseteq \dots \subseteq P_n$ of the poset $P_n = (0 \to 1 \to 2)^{\times n}$,
      where $P_0$ is all tuples containing at least one 0, and
      $P_{r+1}$ is built from $P_r$ by adding all tuples with exactly $(n-r)$ 1-coordinates and $r$ 2-coordinates.
      In the zeroth stage \footnote{
        In the language of \cite[Lemma 4.8]{Pe16}, this is the map associated to the subsets
        $T = P_0$ and $\bar T = \sets{e}{e \leq (1,1,\dots,1)}$.}
      we have the pushout on the right in \eqref{HGBOX_EQ}.
      But when $Z_0 = \varnothing$, $Q^n(g) = Q^n(hg) = \varnothing$, and so
      $k_0: \varnothing \to Z_1^{\otimes n}$ and
      \[
            k_n k_{n-1} \dots k_0 = h^{\otimes n}: Z_1^{\otimes n} \to Z_2^{\otimes n}.
      \]

      Now, since the functor $G \cdot_H (-): \V^H \to \V^G$ sends genuine $H$-cofibrations to genuine $G$-cofibrations,
      the fact that $\V$ has cofibrant symmetric pushout powers implies that
      if $g$ is a cofibration and $h$ a (trivial) cofibration in $\V$,
      $k_r$ is a genuine cofibration in $\V^{\Sigma_n}$ for $r \geq 0$ (which is trivial if $r \geq 1$).

      Thus when $Z_0 = \varnothing$, we have that $h^{\otimes n}$ is a genuine (trivial) $\Sigma_n$-cofibration.
\end{proof}

\begin{lemma}
      \label{ASSEM_HOM_LEM}
      Suppose $\V$ is as in Convention \ref{ALLCOLOR_CONV}. %has cofibrant symmetric pushout powers and cellular fixed points, and cofibrant unit.
      If $\mathbb C$ is a cylinder, then so is each $\left(\mathbb C^{\otimes n}\right)^{\Lambda}$ for all $\Lambda \leq \Sigma_n$.
\end{lemma}
\begin{proof}
      By the above lemma,
      $1_\V = (1_\V)^{\otimes n} \xrightarrow{i_\epsilon} \mathbb C^{\otimes n}$
      is a genuine trivial $\Sigma_n$-cofibrations,
      and hence the result follows by Remark \ref{LEVEL_COF_REM} and the following composite,
      \[
            \begin{tikzcd}
                  1_\V \amalg 1_\V
                  =
                  1_\V^{\otimes n} \amalg 1_\V^{\otimes n}
                  =
                  (1_\V^{\otimes n})^{\Lambda} \amalg (1_\V^{\otimes n})^{\Lambda}
                  % \left((1_V \amalg 1_\V)^{\otimes n}\right)^{\Sigma_n}
                  % \arrow[r, tail]
                  % &
                  % \left((1_V \amalg 1_\V)^{\otimes n}\right)^\Lambda
                  \arrow[r, tail]
                  &
                  \left(\mathbb C^{\otimes n}\right)^\Lambda
                  \arrow[r, "\simeq"]
                  &
                  \left(1_V^{\otimes n}\right)^\Lambda = 1_\V.
            \end{tikzcd}
      \]
      where the first map is a cofibration since it factors through $((1_\V \amalg 1_\V)^{\otimes n})^\Lambda$,
      and the last map is a genuine $\Sigma_n$-weak equivalence by 2-out-of-3 $1_\V \xrightarrow{\simeq} \mathbb C^{\otimes n} \to 1_\V$.


      % ----------------------------------------------------------------------------------------------------
      % ---------- PROBLEM: In what I've written, $\mathbb C^{\otimes} \to \1_\V^{\otimes}$ is only a projective equivalence, not a genuine $\Sigma_n$-equivalence ----------
      %       % It suffices to show there exist
      %       % $1_\V \amalg 1_\V \rightarrowtail (\mathbb C^{\otimes n})^K$
      %       % and
      %       % $(\mathbb C^{\otimes n})^K \xrightarrow{\sim} (1_\V^{\otimes n})^K \simeq 1_\V$
      %       % (where the coherence axioms imply that $1_\V^{\otimes n}$ always has a trivial $\Sigma_n$-action).
      %       % 
      % We consider each structure map separately.
      
      % We note that, as $\otimes$ commutes with colimits in each variable,
      % $(1_\V \amalg 1_\V)^{\otimes n} \simeq \coprod_{\chi} 1_{\V,\chi}$
      % with $\chi$ running over all set maps $\underline{n} \to \set{0,1}$,
      % and $\Sigma_n$ acting by pre-composition on $\chi$.
      % Now, we have the composite
      % \begin{equation}
      % %       \begin{tikzcd}
      %             1_\V \amalg 1_\V = 1_{\V, 0} \amalg 1_{\V, 1}
      %             \simeq
      %             \left((1_\V \amalg 1_\V)^{\otimes n}\right)^{\Sigma_n}
      %             \longrightarrow
      %             (1_\V \amalg 1_\V)^{\otimes n}
      %             \longrightarrow
      %             \mathbb C^{\otimes n}
      % %       \end{tikzcd}
      % \end{equation}
      % where $i: \underline{n} \to \set{i} \into \set{0,1}$ is the constant map,
      % the first arrow is a genuine $\Sigma_n$-cofibration
      % as we may attach each $\Sigma_n$-orbit $\Sigma_n \chi$ individually via maps $\varnothing \to \amalg_{\V,\sigma\chi}$,
      %       % (and we've already attached the stable orbits).
      % and the second arrow is a genuine $\Sigma_n$-cofibration since $\V$ has cofibrant symmetric pushout powers.
      
      % Now,
      % we note that $(\mathbb C \to 1_\V)^{\otimes n}$ is a weak equivalence by induction using Ken Brown's lemma,
      % as $\mathbb C^{\otimes n}$ is cofibrant,
      % $\mathbb C^{\otimes n} \to 1_\V^{\otimes n}$ is a map between cofibrant objects,
      % and $\mathbb C \otimes (-)$ preserves all trivial cofibrations.
      %       % 
      % Let $Q(n) \to \mathbb C^{\otimes n}$ denote $(1_\V \to \mathbb C)^{\square n}$,
      % which is a genuine trivial cofibration in $\V^{\Sigma_n}$ by the assumption on $\V$.
      % % Consider the pushout $P$ and induced maps in $\V^{\Sigma_n}$ below
      % \begin{equation}
      %       \begin{tikzcd}
      %             Q(n) \arrow[r, tail, "\sim"] \arrow[d, tail, "\sim"']
      %             &
      %             \mathbb C^{\otimes n} \arrow[d, tail, "\sim"] \arrow[drr, bend left, "\sim"]
      %             \\
      %             \mathbb C^{\otimes n} \arrow[r, tail, "\sim"] \arrow[rrr, bend right, "\sim"]
      %             &
      %             P \arrow[r, tail, dashed, "\sim"]
      %             &
      %             P' \arrow[r, two heads, "\sim", dashed]
      %             &
      %             1_\V^{\otimes n}
      %       \end{tikzcd}
      % \end{equation}
      % where we have factored the unique map $P \to 1_\V^{\otimes n}$ into a cofibration and fibration.
      
      % Thus, as (the proof of) \cite[Prop 6.3]{BP_geo} shows that $(-)^H$ preserves pushouts over genuine cofibrations
      % as well as genuine \textit{trivial} cofibrations,
      % and since $(-)^H$ preserves trivial fibrations by construction,
      % we have the string of maps in $\V$ below for any $K \leq \Sigma_n$
      % \begin{equation}
      %       \begin{tikzcd}
      %             1_\V \amalg 1_\V \simeq \left((1_\V \amalg 1_\V)^{\otimes n}\right)^{\Sigma_n} \arrow[r, hookrightarrow]
      %             &
      %             \left((1_\V \amalg 1_\V)^{\otimes n}\right)^K \arrow[r, hookrightarrow]
      %             &
      %             (\mathbb C^{\otimes n})^K \arrow[r, tail, "\sim"]
      %             &
      %             P^K \arrow[r, tail, "\sim"]
      %             &
      %             P'^K \arrow[r, two heads, "\sim"]
      %             &
      %             (1_\V^{\otimes n})^K \simeq 1_\V.
      %       \end{tikzcd}
      % \end{equation}
\end{proof}


% ----------------------------------------------------------------------------------------------------
% ------------------------------ diagonals for cylinder objects [incorrect? certainly not needed] ----------

% \begin{definition}
%       The category $\V$ is said to \textit{have diagonals for cylinder objects} if
%       for any cylinder object $\mathbb C$ and $n \geq 0$ there exists a map
%       \begin{equation}
%             \Delta: \mathbb C \to \left(\mathbb C^{\otimes n}\right)^{\Sigma_n}
%       \end{equation}
%       such that the following diagram commutes.
%       \begin{equation}
%             \begin{tikzcd}
%                   1_\V \amalg 1_\V \arrow[r] \arrow[d]
%                   &
%                   \left((1_\V \amalg 1_\V)^{\otimes n}\right)^{\Sigma_n} \arrow[d]
%                   \\
%                   \mathbb C \arrow[r, "\Delta"]
%                   &
%                   \left(\mathbb C^{\otimes n}\right)^{\Sigma_n}
%             \end{tikzcd}
%       \end{equation}
% \end{definition}

% This implies the diagram below commutes for all $k \in \underline{n}$ (since $\mathbb C$ factors the fold map).
% \begin{equation}
%       \begin{tikzcd}
%             1_\V \arrow[rrr, "i_k"] \arrow[d, "i_k"]
%             &&&
%             \mathbb C
%             % &&&
%             % 1_\V \arrow[lll, "i_1"'] \arrow[d, "i_1"]
%             \\
%             \mathbb C \arrow[r, "\Delta"]
%             &
%             \mathbb C^{\otimes n} \arrow[r]
%             &
%             1_\V^{\otimes k-1} \otimes \mathbb C \otimes 1_\V^{\otimes n-k} \arrow[r, "\simeq"]
%             &
%             \mathbb C \arrow[u, equal]
%             % &
%             % \mathbb C \otimes 1_\V^{\otimes n-1} \arrow[l, "\simeq"']
%             % &
%             % \mathbb C^{\otimes n} \arrow[l]
%             % &
%             % \mathbb C \arrow[l, "\Delta"']
%       \end{tikzcd}
% \end{equation}

% It suffices to show the map $\mathbb C^{\otimes n} \to 1_\V^{\otimes n} \simeq 1_\V$ is a trivial $\Sigma_n$-fibration, as a lift of
% \begin{equation}
%       \begin{tikzcd}
%             1_\V \amalg 1_\V \arrow[r] \arrow[d, tail]
%             &
%             \left((1_\V \amalg 1_\V)^{\otimes n}\right)^{\Sigma_n} \arrow[r]
%             &
%             \left(\mathbb C^{\otimes n}\right)^{\Sigma_n} \arrow[d]
%             \\
%             \mathbb C \arrow[rr, two heads, "\simeq"]
%             &&
%             1_\V \simeq 1_\V^{\otimes n} \simeq \left(1_\V^{\otimes n}\right)^{\Sigma_n}
%       \end{tikzcd}
% \end{equation}
% would satisfy these properties.
% \todo[inline]{come back: this need not happen. It may only be a weak equivalence.}
% 
% } % END OF OLIVE GREEN





\subsection{2-out-of-3}

In this section, we prove Proposition \ref{2OUTOF3_PROP}, that weak equivalences satisfy 2-out-of-3.
%
The main ingredient comes from comparing and applying different notions of when two objects are ``homotopically equivalent'', as suggested in the introduction to this section.

% These are of nested strength by construction and Lemma \ref{VIR_HTPY_LEM}.
Following the work of \cite{Cav, BM13}, we make the following definitions.

\begin{definition}
      \label{EQUIV_DEF}
      Given $\mathcal{C}$ in  $\Cat(\V)$ and $a,b\in\mathrm{Ob}(\mathcal C)$, we say $a$ and $b$ are
      \begin{itemize}
      \item {\em equivalent} if there exists a map $\gamma: \J \to \mathcal C$ such that
            $\gamma i_0 = a$, $\gamma i_1 = b$
            for some $\V$-interval $\J$;
      \item {\em virtually equivalent} if $a$ and $b$ are equivalent in some fibrant replacement
            $\mathcal C_f$ of $\mathcal C$ in $\Cat^{\mathrm{Ob}(\mathcal C)}(\V)$;
      \item {\em homotopy equivalent} if $a$ and $b$ are isomorphic in the unenriched category $\pi_0 \mathcal C_f$ (see Definition \ref{HTPY_DEFN})
            for some fibrant replacement $\mathcal C_f$ of $\mathcal C$;
            equivalently,
            there existing maps
            $\alpha: 1_\V \to \mathcal C_f(a,b)$ and $\beta: 1_\V\to \mathcal C_f(b,a)$ such that
            $\beta\alpha$ and $\alpha\beta$ are homotopic (cf. Definition \ref{HTPY_DEFN})
            to the identity arrows
            $1_\V \to \mathcal C_f(a,a)$ and $1_\V \to \mathcal C_f(b,b)$, respectively.
      \end{itemize}
\end{definition}

\cite{Cav, BM13} proved the following comparison results,
where the bottom arrows hold if $\V$ satisfies the written condition \footnote{The coherence condition will be discussed in \S \ref{DK_SEC}.}.
\[
      \begin{tikzcd}[column sep = large]
            \mbox{equivalent}
            \arrow[r, Rightarrow, shift left=2, "\mathrm{always}"]
            &
            \mbox{ virtually equivalent}
            \arrow[r, Rightarrow, shift left=2, "\mathrm{always}"]
            \arrow[l, Rightarrow, shift left = 2, "\substack{\mathrm{right} \\ \mathrm{proper}}"]
            &
            \mbox{ homotopy equivalent}
            \arrow[l, Rightarrow, shift left = 2, "\substack{\mathrm{coherence} \\ \mathrm{condition}}"]
      \end{tikzcd}
\]

Analgous results hold equivariantly, as we will show below. For our group $G$, we define the following:
\begin{definition}
      \label{EQUIVG_DEF}
      Fix $H \leq G$, $\mathcal{C}\in \Cat^G(\V)$, and $a,b\in \mathrm{Ob}(\mathcal{C})^H$.
      We say $a$ and $b$ are 
      \textit{(virtually, homotopy) $H$-equivalent}
      if they are (resp. virtually, homotopy) equivalent in $\mathcal{C}^H$;
      % Two options for virtually equivalent:
      % (i) they are virtually equivalent in $\mathcal C^H$
      % (ii) they are $H$-equivalent in some fibrant replacement $\mathcal C_f$ of $\mathcal C$
      % in $\Cat^{G, \mathrm{Ob}(\mathcal C)}(\V)$.
      
      For a $G$-operad $\O\in \mathsf{Op}^G(\V)$ and $a,b\in \C(\O)^H$, we say $a$ and $b$ are
      {\em (virtually, homotopy) $H$-equivalent}
      if they are so in the underlying category $j^*\O$. 
\end{definition}


\begin{remark}
      \label{ESS_SUR_REM}
      Unraveling definitions, we see
      $F: \O \to \P$ in $\Op^G(\V)$ is essentially $\F$-surjective iff
      for all $H \in \F_1$ and any $b \in \P^H$ there exists $a \in \O^H$ such that $F(a)$ and $b$ are $H$-equivalent.
\end{remark}

We can now record the important steps of the proof of 2-out-of-3, broken into the three distinct cases.
%splits into the three cases. For the first, only preservation of local weak equivalences is non-trival, while for the second and third, the proofs relies on a deep understanding of essential surjectivity.
%We describe the important steps in each case.
\begin{description}
\item [$F$ and $LF$ implies $L$:] This relies on Proposition \ref{CAV_4.14_PROP2}, which shows that
      homotopy equivalence colors induce weak equivalences between certain associated mapping objects,
      and Lemma \ref{VIR_HTPY_LEM}, from which we conclude that equivalent colors are homotopy equivalent.
\item [$L$ and $LF$ implies $F$:] We would like that local equivalences reflect all equivalences, but Lemma \ref{REF_VIRT_LEM} says they only preserve the potentially weaker notion of virtual equivalence;
      however, Lemma \ref{RIGHTPROPER_LEM} says that these notions agree when $\V$ is right proper.
\item [$F$ and $L$ implies $LF$:] This will follow from the fact that equivalences are transitive (Lemma \ref{CAV_4.10_LEM}) and are preserved by functors.
\end{description}
%
Many of the results will follow immediately from (the proof of) their non-equivariant counterparts once the definitions have been established;
only Proposition \ref{CAV_4.14_PROP2} will require a more complex analysis.

We unpack these definitions with two quick remarks.

\begin{remark}
      \label{VE_CHOICE_REM}
      We note that by straightforward lifting and factoring arguments,
      virtual equivalence does not depend on the choice of fibrant replacement nor the choice of $\V$-interval.
      However, for naturality reasons we will almost exclusively use a functorial fibrant replacement.
      This does not cause any added difficult:
      If $a,b\in \mathcal C$ are virtually $H$-equivalent, %(which could more accurately be called ``$H$-virtually equivalent''),
      they are in fact virtually equivalent for $\mathcal C^H$,
      in that there exists a lift
      \begin{equation}
            \label{FIBFIX_LIFT_EQ}
            \begin{tikzcd}
                  \mathcal C^H \arrow[d, tail, "\sim"'] \arrow[r, "\sim"]
                  &
                  (\mathcal C_f)^H
                  \\
                  (\mathcal C^H)_f \arrow[ur, dashed, "\sim"']
            \end{tikzcd}
      \end{equation}
      and thus any equivalence in $(\mathcal C^H)_f$ induces an equivalence in $(\mathcal C_f)^H$.
      % This also follows from Remark \ref{VE_CHOICE_REM}, as both $(\mathcal C^H)_f$ and $(\mathcal C_f)^H$ are fibrant replacements for $\mathcal C^H$.
\end{remark}

\begin{remark}
      \label{HK_EQUIV_REM}
      We note that if $K \leq H \leq G$, then (virtually, homotopy) $H$-equivalent implies (virtually, homotopy) $K$-equivalent
      as we have functors
      \[
            j^{\**}(\O^H) \to j^{\**}(\O^K),
            \qquad
            j^{\**}(\O^H)_f \to j^{\**}(\O^K)_f,
            \qquad
            \pi_0(j^{\**}(\O^H)_f) \to \pi_0(j^{\**}(\O^K)_f).
      \]
      where the last two use a functorial fibrant replacement.
      % Further, if $\V$ has a fibrant replacement functor that commutes with taking fixed points for any subgroup of $G$,     
      % % (in which case the two definitions of virtually $H$-equivalent coincide)
      % then virtually (resp. homotopy) $H$-equivalent implies virtually (homotopy) $K$-equivalent,
      % as we would have an inclusion of categories
      % Moreover, this would imply that $a$ and $b$ are virtually $H$-equivalent iff
      % they are $H$-equivalent in some fibrant replacement $\mathcal C_f$ of $\mathcal C$ in $\Op^{G, \mathrm{Ob}(\mathcal C)}(\V)$.
\end{remark}


\begin{notation}
      As fixed points $(-)^\Gamma$ and fibrant replacement $(-)_f$ need not commute, we will write
      \begin{equation}
            \O_f(C)^\Gamma = (\O_f(C))^\Gamma,
            \qquad
            \O^\Gamma(C)_f = (\O(C)^{\Gamma})_f.
      \end{equation}
\end{notation}


Now, we begin by showing that these notions of equivalences are nested.
The following three lemmas follow exactly as in the non-equivariant setting,
by restricting to the categories $j^{\**}\O^H$ and using \eqref{FIBFIX_LIFT_EQ}.

\begin{lemma}
      [{cf. \cite[4.10]{Cav}}]
      \label{CAV_4.10_LEM}
      $H$-equivalence and virtual $H$-equivalence define equivalence relations on $\mathfrak C(\O)^H$.
\end{lemma}

% \begin{proof}
%       {\color{OliveGreen}
%         Follows exactly as in \textit{loc cite}; either version of virtual $H$-equivalence works.

%         $ $
        
%         Indeed,
%         symmetry follows from the transposition isomorphism $\tau^{\**}\J \to \J$.
        
%         Reflexivity follows from the composition $\I_c \to \I \to \mathcal C^H$,
%         \todo{either version (i) or (ii) works here}
%         $\mathcal C_f^H$
%         of cofibrant replacement followed by the map realizing the identity map on $a$.
        
%         Transitivity follows from the amalgamation of interval objects \cite[Cor. 1.16]{BM13}
%         by the following two claims.
%         First, for any maps of $G$-sets $f: A \to \mathfrak C(\O)$,
%         we have a canonical ``identity'' map $f^{\**}\O \to \O$.
%         Second, a chase through the adjunctions yields that
%         any pair of maps $h: \J \to \mathcal C$ and $h': \J' \to \mathcal C$
%         induces a map $h \** h' : \J \** \J' \to \mathcal C$ such that      
%         $(h \** h') i_0 = h i_0$ and $(h \** h') i_1 = h' i_1$.
%       }
%   \end{proof}

\begin{lemma}
      [{cf. \cite[4.12]{Cav}, \cite[2.10]{BM13}}]
      \label{RIGHTPROPER_LEM}
      If $\V$ is right proper, then two colors are virtually $H$-equivalent iff they are $H$-equivalent. 
\end{lemma}
% \begin{proof}
%       {\color{OliveGreen}
%         Need: virtual $H$-equivalent (i). Then it is an immediate consequence of \textit{loc cite}.
%       }
% \end{proof}

% For completeness (and clarity in \S \ref{DK_SEC}), we highlight and expand on the following proof from \cite{BM13}
% \todo[inline]{the only different really is the inclusion of the paragraph containing the diagram \eqref{J11_CYL_EQ} and the diagram itself,
% which make it clear why natural homotopy equivalences are useful. The rest of the proof is identical, if further unpacked here than in \textit{loc cite}.}

\begin{lemma}
      [{cf. \cite[4.13]{Cav}, \cite[2.11]{BM13}}]
      \label{VIR_HTPY_LEM}
      Virtually $H$-equivalent colors are homotopy $H$-equivalent. 
\end{lemma}
% \begin{proof}
%       {\color{OliveGreen} % ------------------------------ OLIVE GREEN ------------------------------
%         Let $\mathcal C_f$ denote $j^{\**}(\O^H)_f$.
%         Suppose $H: \J \to \mathcal C_f(x,y)$ realizes a virtual equivalence between $x$ and $y$.
%         Let's factor the equipped map $\J \xrightarrow{\sim} \I_f$ (a weak equivalence in $\Cat^{\set{0,1}}(\V)$)
%         as a trivial cofibration and trivial fibration
%         $\J \xrightarrow{\sim} \J' \xrightarrow{\sim} \I_f$,
%         and then $H$ extends to $\J'$ since $\mathcal C_f$ is fibrant.
        
%         Thus we have lifts $f_{01}$ and $f_{10}$ in the diagrams below,
%         where all designations on arrows are as maps in $\Cat^{\set{0,1}}(\V)$.
%         \begin{equation}
%               \begin{tikzcd}
%                     &
%                     \J \arrow[r, "H"] \arrow[d, tail, "\sim"]
%                     &
%                     \mathcal C_f
%                     &&
%                     \J \arrow[r, "H"] \arrow[d, tail, "\sim"]
%                     &
%                     \mathcal C_f
%                     \\
%                     \1 \amalg \1 \arrow[r] \arrow[d, tail]
%                     &
%                     \J' \arrow[ur, dashed, "H'"'] \arrow[d, two heads, "\sim"]
%                     &&
%                     \1 \amalg \1 \arrow[r] \arrow[d, tail]
%                     &
%                     \J' \arrow[ur, dashed, "H'"'] \arrow[d, two heads, "\sim"]
%                     \\
%                     \mathbb A \arrow[r] \arrow[ur, dashed, "f_{01}"]
%                     &
%                     \I_f
%                     &&
%                     \mathbb A^{op} \arrow[r] \arrow[ur, dashed, "f_{10}"]
%                     &
%                     \I_f
%               \end{tikzcd}
%         \end{equation}
%         This provides maps
%         $\alpha = H'_{01} f_{01}: 1_\V \to \mathcal C_f(x,y)$
%         and
%         $\beta = H'_{10} f_{10}: 1_\V \to \mathcal C_f(y,x)$.
%         By construction, the composite $\alpha\beta$ (resp. $\beta\alpha$) factors through $\J'(1,1)$ (resp. $\J'(0,0)$)
%         by a map we denote $f_1$ (resp. $f_0$).
%         \begin{equation}
%               \begin{tikzcd}
%                     1_\V \arrow[r, "\simeq"] \arrow[drr, "f_1"']
%                     &
%                     1_\V \otimes 1_\V \arrow[r, "f_{01} \otimes f_{10}"]
%                     &
%                     \J'(0,1) \otimes \J'(1,0) \arrow[d, "\circ"] \arrow[r, "H'_{01} \otimes H'_{10}"]
%                     &
%                     \mathcal C_f(x,y) \otimes \mathcal C_f(y,x) \arrow[d, "\circ"]
%                     \\
%                     &&
%                     \J'(1,1) \arrow[r, "H'_{11}"]
%                     &
%                     \mathcal C_f(y,y)
%               \end{tikzcd}
%         \end{equation}
        
%         Now, if $\mathbb C$ is any cylinder in $\V$, we have a lift in the following diagram, for $i \in \set{0,1}$.
%         \begin{equation}
%               \label{J11_CYL_EQ}
%               \begin{tikzcd}
%                     1_\V \amalg 1_\V \arrow[rr, "{(id, f_i)}"] \arrow[d, tail]
%                     &&
%                     \J'(i,i) \arrow[d, "\sim", two heads]
%                     \\
%                     \mathbb C \arrow[r] \arrow[urr, dashed]
%                     &
%                     1_\V \arrow[r]
%                     &
%                     (1_\V)_f
%               \end{tikzcd}
%         \end{equation}
%         Thus $f_1$ (resp. $f_0$) factors through a cylinder,
%         implying $\alpha\beta$ (resp. $\beta\alpha)$ is homotopic to $id_y$ (resp. $id_x$),
%         and hence that the objects $x,y\in \mathcal C_f$ are homotopy equivalent.
%       }
% \end{proof}

The key ideas in the proofs are:
the stacking of $\V$-intervals forms another $\V$-interval \todo{pushout trick and Cav4.14 - still need that $\mathbb J_0$ is a cofibrant \textit{monoid}!},
``nice'' pullbacks in $\Cat(\V)$ preserve local equivalences which are bijective on objects, and
the two objects of any $\V$-interval $\mathbb J$ are isomorphic in $\pi_0(\mathbb J)$.

To finish off Case II, we will need that (local) weak equivalences preserve some notions of equivalence.

\begin{lemma}
      [{cf. \cite[4.11]{Cav}, \cite[2.9]{BM13}}]
      \label{REF_VIRT_LEM}
      Let $\F$ be a $(G, \Sigma)$-family. %Suppose the $G$-graph system $\F$ has units.
      Then local weak $\F$-equivalences reflect virtual $H$-equivalences for all $H \in \F_1$.
      Explicitly, for
      $a_0,a_1 \in \C(\O)^H$, and $F: \O \to \P$ in $\mathsf{Op}^G(\V)$ a local weak $\F$-equivalence,
      if $F(a_0)$ and $F(a_1)$ are virtually $H$-equivalent then so are $a_0$ and $a_1$.
\end{lemma}
\begin{proof}
      % {\color{OliveGreen}
      %   Need: virtual $H$-equivalent (i).
      % }
      %Since $\F$ has units, $\F_1$ contains all $H \leq G \times \Sigma_1$ (cf. \cite[Remark 4.50]{BP_geo}), and thus
      For all $H \in \F_1$,
      $j^{\**}F^H: j^{\**}\O^H \to j^{\**}\P^H$ is a local weak equivalence in $\Cat(\V)$.
      %
      The proof then follows as in \textit{loc cite}.
      % {\color{OliveGreen} % ------------------------------ OLIVE GREEN ------------------------------
      %   As in \cite{BM13}, we may build a fibrant replacement of $j^{\**}F^H$ which is a local trivial fibration in $\Cat(\V)$.
      %   \begin{equation}
      %         \begin{tikzcd}
      %               j^{\**} \O^H \arrow[r, "\sim_l"] \arrow[d, dashed, "\sim"']
      %               &
      %               F^{\**} (j^{\**}\P^H) \arrow[r, "\simeq_l", two heads] \arrow[d, "\sim"']
      %               &
      %               j^{\**} \P^H \arrow[d, "\sim"]
      %               \\
      %               j^{\**} (\O^H)_f \arrow[r, dashed, two heads]
      %               &
      %               F^{\**} (j^{\**}(\P^H)_f) \arrow[r, "\simeq_l", two heads]
      %               &
      %               j^{\**}(\P^H)_f
      %         \end{tikzcd}
      %   \end{equation}
      %   (where all designations $(-)_l$ on arrows denote ``local'').
      %   Thus we have a lift on the left in \eqref{REF_VIRT_EQ},
      %   where the bottom arrow realizes the virtual $H$-equivalence between $F(a_0)$ and $F(a_1)$,
      %   as such a lift in $\Cat(\V)$ is equivalent to a lift in $\Cat^{\set{0,1}}(\V)$
      %   after pulling the right vertical arrow back along the inclusion
      %   $a: \set{0,1} \to \set{a_0,a_1} \into \mathfrak C_\O^H$ as on the right (cf. \eqref{COLOR_SQ_EQ}),
      %   and this lift exists by locality.
      %   \begin{equation}
      %         \label{REF_VIRT_EQ}
      %         \begin{tikzcd}
      %               \1 \amalg \1 \arrow[r, "{(a_0,a_1)}"] \arrow[d, tail, "{(i_0, i_1)}"']
      %               &
      %               j^{\**}(\O^H)_f \arrow[d, two heads, "\sim_l"]
      %               &&
      %               \1 \amalg \1 \arrow[r, "{(a_0,a_1)}"] \arrow[d, tail, "{(i_0, i_1)}"']
      %               &
      %               a^{\**}j^{\**}(\O^H)_f \arrow[d, two heads, "\sim"]
      %               \\
      %               \J \arrow[r] \arrow[ur, dashed]
      %               &
      %               j^{\**}(\P^H)_f
      %               &&
      %               \J \arrow[r] \arrow[ur, dashed]
      %               &
      %               a^{\**} F^{\**} j^{\**}(\P^H)_f.
      %         \end{tikzcd}
      %   \end{equation}
      % } % ------------------------------ OLIVE GREEN ------------------------------
\end{proof}

Moving on to Case I, we now prove that a homotopy equivalence of colors induces a homotopy equivalence between certain hom-objects.
% 
We begin with some notation and a construction.
Let $\mathfrak C$ be a $G$-set of colors, and fix a $\mathfrak C$-signature $C = (c_1, \dots, c_n; c_0)$.
For each $0 \leq i \leq n$, let $\Aut_i(C) \leq \Aut(C) = \Aut_{G \ltimes \Sigma_{\mathfrak C}}(C)$ denote the subgroup of elements $(g, \sigma)$
such that $\sigma(i) = i$ (including $\Aut_0(C) = \Aut(C)$),
and let $H_i \leq G$ denote the $G$-projection of $\Aut_i(C)$.

Now, suppose $d_i$ and $c_i$ are $H_i$-homotopy equivalent.
We define a new $\mathfrak C$-signature by replacing $c_i$ and all of its ``$\Aut_i(C)$-orbits'' in $C$ with
the $\Aut_i(C)$-orbit of $d_i$.
Specifically, let $\lambda \subseteq \langle n \rangle = \set{0,1,2,\dots,n}$ denote the subset of all $j$ such that
there exists some $(k_j, \sigma_j) \in \Aut(C)$ such that $\sigma_j(i) = j$ (and hence $c_j = k_j c_i$).
% for each $j \in \lambda$, fix choices of such $(k_j, \sigma_j)$ (with $(k_i,\sigma_i) = (1,1)$).
Then for all $0 \leq j \leq n$, we define $d_j$ and $D$ by
\begin{equation}
      \label{DDEF_EQ}
      d_j =
      \begin{cases}
            k_j \cdot d_i \qquad & j \in \lambda
            \\
            c_j & \mbox{otherwise,}
      \end{cases}
      \qquad \qquad
      D = (d_1,\dots, d_n; d_0).
\end{equation}

\begin{remark}
      We note that these $d_j$ --- and hence $D$ --- do not depend on the choice of $(k_j,\sigma_j) \in \Aut(C)$:
      If $(k'_j, \sigma'_j) \in \Aut(C)$ also has $\sigma'_j(i) = j$, then
      $(k'_j, \sigma'_j) \cdot (k_j^{-1}, \sigma_j^{-1}) \in \Aut_i(C)$,
      so $k'_j k_j^{-1} \in \Aut_G(c_i) \mathop{\cap} \Aut_G(d_i)$,
      and thus $k'_j \cdot d_i = k_j \cdot d_i$.
\end{remark}


\begin{lemma}
      \label{AUTC_LEM}
      For $C$, $D$, $\lambda$, $H_i$, and $(k_j, \sigma_j)$ defined as above, we have the following:
      \begin{enumerate}[label = (\roman*)]
      \item $j \in \lambda$ iff $\pi(j) \in \lambda$ for all $(h,\pi) \in \Aut(C)$.
      \item $\Aut(C) \leq \Aut(D)$.
      \item $H_{\pi(i)} = h H_i h^{-1}$ for all $(h,\pi) \in \Aut(C)$.
      \item For any arrow
            $\alpha \colon 1_\V \to \O(c_i;d_i)^{H_i}$,
            and homotopy
            $H \colon \mathbb C \to \O(c_i;c_i)^{H_i}$
            in a $\mathfrak C$-colored $G$-operad $\O$,
            the maps
            \begin{equation}
                  \alpha_j \colon 1_\V \xrightarrow{\alpha} \O(c_i; d_i)^{H_i} \xrightarrow{ k_j \cdot (-) } \O(c_j; d_j)^{H_j},
                  \qquad \qquad
                  H_j \colon \mathbb C \xrightarrow{H} \O(c_i;c_i)^{H_i} \xrightarrow{k_j \cdot (-)} \O(c_j;c_j)^{H_j}
            \end{equation}
            are well-defined and indepedent of the choice of $(k_j, \sigma_j)$.
      \item For any $\alpha$ and $H$, the maps
            \begin{align*}
              \otimes_\lambda \alpha_j \colon & 1_\V \simeq \otimes_\lambda 1_\V \longto \otimes_\lambda \O(c_j; d_j)^{H_j},
              \\
              \otimes_\lambda H_j \colon & \otimes_\lambda \mathbb C \longto \otimes_\lambda \O(c_j;c_j)^{H_j},
              \\
              (\alpha_j)_{\**} \colon & \O(C) \xrightarrow{\otimes_\lambda \alpha_j} \O(C) \otimes \bigotimes_\lambda \O(c_j; d_j)^{H_j} \longto \O(D)
            \end{align*}
            are $\Aut(C)$-equivariant.
      \end{enumerate}
\end{lemma}
\begin{proof}
      These all follow more-or-less immediately from the same straightforward calculation.
      Fixing $(h,\pi) \in \Aut(C)$ and $j \in \lambda$, we have
      \begin{equation}
            (k_{\pi(j)}^{-1} h k_j, \sigma_{\pi(j)}^{-1} \pi \sigma_j) \in \Aut_i(C),
            \qquad
            \textrm{and}
            \qquad
            (h,\pi) \Aut_i(C) (h,\pi)^{-1} = \Aut_{\pi(i)}(C).
      \end{equation}
\end{proof}

\begin{proposition}[{c.f. \cite[Prop. 4.14]{Cav}, \cite[Prop. 2.12]{BM13}}]
      \label{CAV_4.14_PROP2}
      Suppose $\V$ is as in Convention \ref{ALLCOLOR_CONV}. %has cofibrant symmetric pushout powers and cofibrant unit.
      Let $\O \in \Op^G(\V)$ with color $G$-set $\mathfrak C$,
      fix a $\mathfrak C$-signature $C = (c_1,\dots, c_n;c_0)$,
      and let $H_i$ be defined as in \eqref{DDEF_EQ}.
      %
      Suppose $d_i$ and $c_i$ are $H_i$-homotopy equivalent.
      Then there is a zig-zag, natural in functors out of $\O$,
      of genuine $\Aut(C)$-weak equivalences between $\O(C)$ and $\O(D)$,
      with $D$ the $\mathfrak C$-signature defined as in \eqref{DDEF_EQ}.
\end{proposition}
\begin{proof}
      Without loss of generality, $i = 1$ or $i = 0$.
      We only consider $i=1$; the case $i = 0$, $D = (c_1, \dots, c_n; d_0)$ will follow by an analogous (and simpler) argument.

      First, we note that using appropriate lifts (and \eqref{FIBFIX_LIFT_EQ}), one can show that if $c$ and $d$ are $H$-homotopy equivalent in $\O$, they are
      homotopy $H$-equivalent in the functorial bifibrant replacement $\O_{fc}$ of $\O$ in $\Op^{G, \mathfrak C(\O)}_{gen}(\V)$,
      the $\F$-(semi)-model structure from Theorem \ref{THM1_C} for $\F = \set{\F_n}$ with $\F_n$ all subgroups of $G \times \Sigma_n$.
      %
      Moreover, for any functor $F$, we have the following commuting diagram of zig-zags
      \begin{equation}
            \begin{tikzcd}
                  \O(C) \arrow[r, "\simeq"] \arrow[d, "F"']
                  &
                  \O_f(C) \arrow[d]
                  &
                  \O_{fc}(C) \arrow[l, "\simeq"'] \arrow[r, "{(\alpha_j)_{\**}}"] \arrow[d]
                  &
                  \O_{fc}(D) \arrow[r, "\simeq"] \arrow[d]
                  &
                  \O_f(D) \arrow[d]
                  &
                  \O(D) \arrow[l, "\simeq"'] \arrow[d, "F"]
                  \\
                  \P(C) \arrow[r, "\simeq"]
                  &
                  \P_f(C)
                  &
                  \P_{fc}(C) \arrow[l, "\simeq"'] \arrow[r, "{F(\alpha_j)_{\**})}"]
                  &
                  \P_{fc}(D) \arrow[r, "\simeq"] 
                  &
                  \P_f(D) 
                  &
                  \P(D) \arrow[l, "\simeq"']
            \end{tikzcd}
      \end{equation}
      where the $\simeq$ denote weak equivalences in $\V^{\Aut(C)}_{gen}$.
      Thus it suffices to consider the case where $\O$ is bifibrant; the general case and naturality follow.

      Now, by assumption we have
      arrows $\alpha: 1_\V \to \O(c_1; d_1)^{H_1}$ and $\beta: 1_\V \to \O(d_1, c_1)^{H_1}$,
      and homotopies $H = H_{\beta\alpha,id}$ and $H_{\alpha\beta,id}$.
      Building off Lemma \ref{AUTC_LEM}$(v)$, we use Remark \ref{CYL_REM} and Lemma \ref{ASSEM_HOM_LEM} to assemble the homotopies $H_j = H_{\beta_j \alpha_j,id}$, yielding
      \begin{equation}
            (\beta_j \alpha_j)_{\**} = (\beta_j)_{\**} (\alpha_j)_{\**} \sim id_{\O(C)^\Gamma}
            \qquad
            \textrm{and}
            \qquad
            (\alpha_j \beta_j)_{\**} = (\alpha_j)_{\**} (\beta_j)_{\**} \sim id_{\O(D)^\Gamma}
      \end{equation}
      in $\V$ for all subgroups $\Gamma \leq \Aut(C)$ via the diagram
      \begin{equation}
            \begin{tikzcd}
                  \O(C)^\Gamma \otimes (1_\V \amalg 1_\V) \arrow[d, tail] \arrow[rr, "{((\beta_i)_{\**}(\alpha_i)_{\**}, id)}"]
                  &&
                  \O(C)^\Gamma
                  \\                  
                  \O(C)^\Gamma \otimes \left(\bigotimes\limits_\lambda \mathbb C\right)^{\Gamma}
                  \arrow[r, "\otimes_\lambda H_j"]
                  &
                  \O(C)^\Gamma \otimes \left(\bigotimes\limits_\lambda \O(c_j;c_j)^{H_j}\right)^{\Gamma} \arrow[r]
                  &
                  \left( \O(C) \otimes \bigotimes\limits_\lambda \O(c_j;c_j) \right)^{\Gamma} \arrow[u, "\circ"']
            \end{tikzcd}
      \end{equation}
      and it's partner switching the order of $\beta$ and $\alpha$.      
      Hence, as both $\O(C)^\Gamma$ and $\O(D)^\Gamma$ are bifibrant in $\V$ by Remark \ref{LEVEL_COF_REM},
      \[
            (\alpha_j)_{\**}: \O(C)^\Gamma \leftrightarrows \O(D)^\Gamma : (\beta_j){\**}
      \]
      are inverse isomorphisms in $\Ho(\V)$ for all $\Gamma \leq \Aut(C)$,
      and thus $(\alpha_j)_{\**}$ is a weak equivalence in $\V^{\Aut(C)}_{gen}$.

      
      % % %%%%%%%%%%%%%%%%%%%%%%%%%%%%%%%%%%%%%%%%%%%%%%%%%%%%%%%%%%%%%%%%%%%%%%%%%%%%%%%%%%%%%%%%%%%
      % % ---------------------------------------- OLD PROOF ----------------------------------------
      % First, we have some straightforward calculations. Fix $(h,\pi) \in \Aut(C)$, so $h c_i = c_{\pi(i)}$ for all $0 \leq i \leq n$.
      % %
      % Then $j \in \lambda$ iff $\pi(j) \in \lambda$.
      % Moreover, additionally fixing $j \in \lambda$, we have
      % \begin{equation}
      %       \label{STAB_CONJ_EQ}
      %       (k_{\pi(j)}^{-1} h k_j, \sigma_{\pi(j)}^{-1} \pi \sigma_j) \in \Aut_1(C),
      %       \qquad
      %       \textrm{and}
      %       \qquad
      %       (h,\pi) \Aut_1(C) (h,\pi)^{-1} = \Aut_{\pi(1)}(C).
      % \end{equation}
      % % \begin{enumerate}[label = (\roman*)]
      % % \item 
      % %       {\color{OliveGreen} % ----------------------------------------
      % %       since
      % %       \[
      % %             k_{\pi(j)}^{-1} h k_j c_i = k_{\pi(j)}^{-1} h c_{\sigma_j(i)} = k_{\pi(j)}^{-1} c_{\pi \sigma_j(i)} = c_{\sigma_{\pi(j)}^{-1} \pi \sigma_j(i)},
      % %             \qquad
      % %             \sigma_{\pi(j)}^{-1} \pi \sigma_j(1) = \sigma_{\pi(j)}^{-1}(\pi(j)) = 1.
      % %       \] %
      % %     } % ------------------------------------------------------------
      % % \item $\Stab_j(C) = (k_j, \sigma_j) \Stab_1(C) (k_j, \sigma_j)^{-1}$,
      % %       {\color{OliveGreen} % ------------------------------------------
      % %       as when $(h,\pi) \in \Stab_1(C)$,

      % %       \[
      % %             \sigma_j \pi \sigma_j^{-1}(j) = \sigma_j\pi(1) = \sigma_j(1) = j.
      % %       \]
      % %     } % ------------------------------------------------------------
      % % \end{enumerate}
      % %
      % From these, it is easy to see that $\Aut(C) \leq \Aut(D)$,
      % % {\color{OliveGreen} % ----------------------------------------
      % %   since
      % %   \[
      % %         h d_j = h k_j d_1 = k_{\pi(j)} k_{\pi(j)}^{-1} h k_j d_1 = k_{\pi(j)} d_1 = d_{\pi(j)},
      % %   \]
      % %   for $j \in \lambda$ and otherwise $h d_j = h c_j = c_{\pi(j)} = d_{\pi(j)}$. 
      % % } % --------------------------------------------------
      % and that $H_j = k_j H_1 k_j^{-1}$ for all $j \in \lambda$,
      % and more generally $H_{\pi(1)} = h H_1 h^{-1}$ for all $(h,\pi) \in \Aut(C)$.

      % % ---------- bifibrant case ----------------------------------------
      % Second, let us assume that $\O$ is bifibrant in $\Op^{G, \mathfrak C(\O)}_{gen}(\V)$,
      % the $\F$-(semi)-model structure from Theorem \ref{THM1_C} for $\F = \set{\F_n}$ with $\F_n$ all subgroups of $G \times \Sigma_n$.
      % By assumption, we have maps 
      % $\alpha_1: 1_\V \to \O(c_1;d_1)^{H_1}$ and $1_\V \to \O(d_1;c_1)^{H_1}$,
      % and homotopies $H_{\beta\alpha,1}$ and $H_{\alpha\beta,1}$.
      % Action by $k_j$ and \eqref{STAB_CONJ_EQ} yield a well-defined map
      % \[
      %       \alpha_j: 1_\V \xrightarrow{\alpha} \O(c_1;d_1)^{H_1} \xrightarrow{k_j \cdot (-)} \O{fc}(c_j; d_j)^{H_j},
      % \]
      % and similarly well-defined $\beta_j$,
      % and naturality implies that the pair $(\alpha_j,\beta_j)$ realize a homotopy $H_j$-equivalence
      % via
      % \[
      %       H_{\beta_j\alpha_j,1} = k_j \cdot H_{\beta\alpha,1}
      %       \qquad \textrm{and} \qquad
      %       H_{\alpha_j \beta_J, 1} = k_j \cdot H_{\alpha\beta,1}.
      % \]
      % All together, \eqref{STAB_CONJ_EQ} also implies that $\otimes_\lambda \O(c_j; c_j;)^{H_j}$ has an action by $\Aut(C)$
      % (and similarly replacing either/both $c_j$ with $d_j$),
      % and moreover that the products
      % \[
      %       \otimes_\lambda \alpha_j: 1_\V \simeq \otimes_\lambda 1_\V \longto \otimes_\lambda \O(c_j; d_j)^{H_j},
      %       \qquad
      %       H_\lambda := \otimes_\lambda H_{\beta_j\alpha_j, 1}: \otimes_\lambda \mathbb C \longto \otimes_\lambda \O(c_j;c_j)^{H_j},
      % \]
      % their induced composition operations
      % \[
      %       (\alpha_j)_{\**}: \O(C) \simeq \O(C) \otimes \bigotimes_\lambda 1_\V \xrightarrow{\alpha_j}
      %       \O(C) \otimes \bigotimes_\lambda \O(c_j; d_j)^{H_j} \longto \O(D),
      % \]
      % and their companions $\otimes \beta_j$, $\otimes H_{\alpha_j\beta_j,1}$, $(\beta_j)_{\**}$
      % are $\Aut(C)$-equivariant.

      % Using Remark \ref{CYL_REM} and Lemma \ref{ASSEM_HOM_LEM}, we assemble the homotopies $H_{\beta_j \alpha_j,1}$ to yield
      % \begin{equation}
      %       (\beta_j \alpha_j)_{\**} = (\beta_j)_{\**} (\alpha_j)_{\**} \sim id_{\O(C)^\Gamma}
      %       \qquad
      %       \textrm{and}
      %       \qquad
      %       (\alpha_j \beta_j)_{\**} = (\alpha_j)_{\**} (\beta_j)_{\**} \sim id_{\O(D)^\Gamma}
      % \end{equation}
      % in $\V$ for all subgroups $\Gamma \leq \Aut(C)$, via the diagram
      % \begin{equation}
      %       \begin{tikzcd}[column sep = small]
      %             \O(C)^\Gamma \otimes (1_\V \amalg 1_\V) \arrow[d, tail] \arrow[rr, "{((\beta_i)_{\**}(\alpha_i)_{\**}, id)}"]
      %             &&
      %             \O(C)^\Gamma
      %             \\                  
      %             \O(C)^\Gamma \otimes \left(\bigotimes\limits_\lambda \mathbb C\right)^{\Gamma}
      %             \arrow[r, "H_\lambda"]
      %             &
      %             \O(C)^\Gamma \otimes \left(\bigotimes\limits_\lambda \O(c_j;c_j)^{H_j}\right)^{\Gamma} \arrow[r]
      %             &
      %             \left( \O(C) \otimes \bigotimes\limits_\lambda \O(c_j;c_j) \right)^{\Gamma} \arrow[u, "\circ"']
      %       \end{tikzcd}
      % \end{equation}
      % and it's partner switching the order of $\beta$ and $\alpha$.      
      % Hence, as both $\O(C)^\Gamma$ and $\O(D)^\Gamma$ are bifibrant in $\V$ by Remark \ref{LEVEL_COF_REM},
      % \[
      %       (\alpha_j)_{\**}: \O(C)^\Gamma \leftrightarrows \O(D)^\Gamma : (\beta_j){\**}
      % \]
      % are inverse isomorphisms in $\Ho(\V)$ for all $\Gamma \leq \Aut(C)$,
      % and thus $(\alpha_j)_{\**}$ is a weak equivalence in $\V^{\Aut(C)}_{gen}$.

      % % ---------- general case --------------------------------------------------
      % Third, for the general case, 
      % we note that using appropriate lifts (and \eqref{FIBFIX_LIFT_EQ}), one can show that if $c$ and $d$ are $H$-homotopy equivalent in $\O$, they are
      % homotopy $H$-equivalent in the functorial bifibrant replacement $\O_{fc}$ of $\O$ in $\Op^{G, \mathfrak C(\O)}_{gen}(\V)$.
      % Moreover, for any functor $F$, we have the following commuting diagram of zig-zags
      % \begin{equation}
      %       \begin{tikzcd}
      %             \O(C) \arrow[r, "\simeq"] \arrow[d, "F"']
      %             &
      %             \O_f(C) \arrow[d]
      %             &
      %             \O_{fc}(C) \arrow[l, "\simeq"'] \arrow[r, "{(\alpha_j)_{\**}}"] \arrow[d]
      %             &
      %             \O_{fc}(D) \arrow[r, "\simeq"] \arrow[d]
      %             &
      %             \O_f(D) \arrow[d]
      %             &
      %             \O(D) \arrow[l, "\simeq"'] \arrow[d, "F"]
      %             \\
      %             \P(C) \arrow[r, "\simeq"]
      %             &
      %             \P_f(C)
      %             &
      %             \P_{fc}(C) \arrow[l, "\simeq"'] \arrow[r, "{F(\alpha_j)_{\**})}"]
      %             &
      %             \P_{fc}(D) \arrow[r, "\simeq"] 
      %             &
      %             \P_f(D) 
      %             &
      %             \P(D) \arrow[l, "\simeq"']
      %       \end{tikzcd}
      % \end{equation}
      % where the $\simeq$ denote weak equivalences in $\V^{\Aut(C)}_{gen}$.
      % Thus the result for general $\O$ follows from the bifibrant case,
      % and naturality is clear.

      
      % % %%%%%%%%%%%%%%%%%%%%%%%%%%%%%%%%%%%%%%%%%%%%%%%%%%%%%%%%%%%%%%%%%%%%%%%%%%%%%%%%%%%%%%%%%%
      % % -------------------- $i = 0$ case --------------------------------------------------
      % The case $i = 0$, $D = (c_1, \dots, c_n; d_0)$ follows by an analogous (simpler) argument.
      % % Using appropriate lifts (and \eqref{FIBFIX_LIFT_EQ}), one can show that if $c$ and $d$ are $H$-homotopy equivalent in $\O$, they are
      % % homotopy $H$-equivalent in the functorial bifibrant replacement $\O_{fc}$ of $\O$ in $\Op^{G, \mathfrak C(\O)}_{gen}(\V)$,
      % % the $\F$-(semi)-model structure from Theorem \ref{THM1_C} for $\F = \set{\F_n}$ with $\F_n$ all subgroups of $G \times \Sigma_n$.
      % % Thus we may assume we have representing maps
      % % $\alpha: 1_\V \to \O_{fc}(c;d)^H$ and $1_\V \to \O_{fc}(d;c)^H$.
      % % Now, for any functor $F$, we have the following commuting diagram of zig-zags
      % % \begin{equation}
      % %       \begin{tikzcd}
      % %             \O(C) \arrow[r, "\simeq"] \arrow[d, "F"']
      % %             &
      % %             \O_f(C) \arrow[d]
      % %             &
      % %             \O_{fc}(C) \arrow[l, "\simeq"'] \arrow[r, "\alpha_{\**}"] \arrow[d]
      % %             &
      % %             \O_{fc}(D) \arrow[r, "\simeq"] \arrow[d]
      % %             &
      % %             \O_f(D) \arrow[d]
      % %             &
      % %             \O(D) \arrow[l, "\simeq"'] \arrow[d, "F"]
      % %             \\
      % %             \P(C) \arrow[r, "\simeq"]
      % %             &
      % %             \P_f(C)
      % %             &
      % %             \P_{fc}(C) \arrow[l, "\simeq"'] \arrow[r, "{F(\alpha_{\**})}"]
      % %             &
      % %             \P_{fc}(D) \arrow[r, "\simeq"] 
      % %             &
      % %             \P_f(D) 
      % %             &
      % %             \P(D) \arrow[l, "\simeq"']
      % %       \end{tikzcd}
      % % \end{equation}
      % % where the $\simeq$ denote weak equivalences in $\V^{\Aut(C)}_{gen}$, and $\alpha_{\**}$ is the composite
      % % \[
      % %       \O_{fc}(C) \xrightarrow{\alpha} \O_{fc}(C) \otimes \O_{fc}(c,d) \longto \O_{fc}(D).
      % % \]
      % % Thus, it suffices to check that $\alpha_{\**}$ is a genuine $\Aut(C)$-weak equivalence when $\O$ is bifibrant in $\Op^{G,\mathfrak C(\O)}_{gen}(\V)$.

      % % Let $\beta: 1_V \to \O(d,c)^H$ denote the $H$-homotopy inverse of $\alpha$,
      % % and $H_{\beta\alpha,1}$, $H_{\alpha\beta,1}$ the associated homotopies.
      % % Naturality implies that we have composition maps of the form
      % % \[
      % %       \O(C)^\Gamma \otimes \O(c,d)^{H_\Gamma} \longto \left( \O(C) \otimes \O(c,d) \right)^\Gamma \xrightarrow{\circ} \O(D)^\Gamma
      % % \]
      % % for all $\Gamma \leq \Stab(C)$, with $H_\Gamma \leq G$ the $G$-projection,
      % % and so in particular the maps $\alpha_{\**}$ and $\beta_{\**}$ are $\Aut(C)$-equivariant.
      % % Thus $H_{\beta\alpha,1}$ and $H_{\alpha\beta,1}$ restrict and induce homotopies
      % % \[
      % %       (\beta\alpha)_{\**} \sim id_{\O(C)^{\Gamma}}
      % %       \qquad \textrm{and} \qquad
      % %       (\alpha\beta)_{\**} \sim id_{\O(D)^\Gamma}
      % % \]
      % % in $\V$ via diagrams of the form, e.g.,
      % % \begin{equation}
      % %       \begin{tikzcd}
      % %             \O_{f c}(C)^{\Gamma} \otimes (1_\V \amalg 1_\V) \arrow[d, tail] \arrow[r, "{((\beta\alpha)_{\**}, id)}"]
      % %             &
      % %             \O_{f c}(C)^\Gamma
      % %             % &
      % %             % 1_\V \amalg 1_\V \arrow[d, tail] \arrow[rr, "{((\alpha\beta)_{\**}, id)}"]
      % %             % &&
      % %             % \V((\O(\ksi_c^d)_f)^\Gamma, (\O(\ksi_c^d)_f)^\Gamma)
      % %             \\                  
      % %             \O_{f c}(C)^\Gamma \otimes \mathbb C \arrow[r, "H_{\beta\alpha,1}"]
      % %             &
      % %             \O_{f c}(C)^\Gamma \otimes \O_{f c}(c;c)^{H_\Gamma} \arrow[u, "\circ"']
      % %             % &
      % %             % \mathbb C \arrow[r, "H_{\alpha\beta,1}"']
      % %             % &
      % %             % (\O(d;d)^{H_\Gamma})_f \arrow[r]
      % %             % &
      % %             % (\O(d;d)_f)^{H_\Gamma} \arrow[u, "{(-)_{\**}}"],
      % %       \end{tikzcd}
      % % \end{equation}
      % % Hence, as both $\O(C)^\Gamma$ and $\O(D)^\Gamma$ are bifibrant in $\V$ by Remark \ref{LEVEL_COF_REM},
      % % \[
      % %       \alpha_{\**}: \O(C)^\Gamma \leftrightarrows \O(D)^\Gamma : \beta_{\**}
      % % \]
      % % are inverse isomorphisms in $\Ho(\V)$ for all $\Gamma \leq \Aut(C)$,
      % % and thus $\alpha_{\**}$ is a weak equivalence in $\V^{\Aut(C)}_{gen}$, as desired.
\end{proof}

We can now prove the main result of this subsection.
 
\begin{proposition}
      [{c.f. \cite[4.15]{Cav}, \cite[2.13]{BM13}}]
      \label{CAV_4.15_PROP}
      \label{2OUTOF3_PROP}
      Suppose $\V$ is as in Convention \ref{ALLCOLOR_CONV} and additionally is right proper.
      Then for any $(G, \Sigma)$-family $\F$ such that $H \in \F_1$ for all $H \leq G$, %with units,
      the class of weak $\F$-equivalences in $\mathsf{Op}^G(\V)$ satisfies the 2-out-of-3 condition.
\end{proposition}
\begin{proof}
      Let $\O \xrightarrow{F} \P \xrightarrow{L} \Q$ be a composition of maps in $\mathsf{Op}^G(\V)$.
      If $F$ and $L$ are weak $\F$-equivalences,
      the composite is obviously a local weak $\F$-equivalence:
      $\O(C)^\Gamma \xrightarrow{\sim} \P(F(C))^\Gamma \xrightarrow{\sim} \Q(LF(C))^\Gamma$.
      Moreover, as functors preserve equivalences of colors, $L F$ is essentially $\F$-surjective by transitivity from Lemma \ref{CAV_4.10_LEM}. 
      % Moreover, concatination of $\V$-intervals collapses the consecutive essential surjectivity diagrams on the left
      % onto the one on the right.
      % \begin{equation}
      %       \begin{tikzcd}[row sep = tiny]
      %             \1 \arrow[rr, dashed, "a"] \arrow[dr]
      %             &&
      %             j^{\**}\O^H \arrow[dd]
      %             \\
      %             & \J \arrow[dr, dashed]
      %             &&
      %             \1 \arrow[rr, "a", dashed] \arrow[dr]
      %             &&
      %             j^{\**}\O^H \arrow[dd]
      %             \\
      %             \1 \arrow[rr, dashed, "b"] \arrow[ur] \arrow[dr]
      %             &&
      %             j^{\**} \P^H \arrow[dd]
      %             &&
      %             \J \** \J' \arrow[dr, dashed]
      %             \\
      %             & \J' \arrow[dr, dashed]
      %             &&
      %             \1 \arrow[rr, "c"] \arrow[ur]
      %             &&
      %             j^{\**}\Q^H
      %             \\
      %             \1 \arrow[ur] \arrow[rr, "c"]
      %             &&
      %             j^{\**} \Q^H
      %       \end{tikzcd}
      % \end{equation}
      
      If $L$ and $FL$ are weak $\F$-equivalences,
      then $F$ is a local weak $\F$-equivalence by 2-out-of-3 in each $\V^{\Aut(C)}_{\F_C}$.
      Moreover, if $b \in \mathfrak C(\P)^H$ for $H \in \F_1$, then by Remark \ref{ESS_SUR_REM}, there exists $a \in \mathfrak C(\O)^H$ such that
      $LF(a)$ and $L(b)$ are (virtually) $H$-equivalent.
      Lemma \ref{REF_VIRT_LEM} then implies $F(a)$ and $b$ are virtually $H$-equivalent, 
      and since $\V$ is right proper, Lemma \ref{RIGHTPROPER_LEM} implies they are $H$-equivalent.

      Lastly, suppose $F$ and $LF$ are weak $\F$-equivalences.
      It is immediate that $L$ is essentially $\F$-surjective.
      Now, given a signature $C = (c_1,\ldots,c_n;c_0$) in $\C(\P)$,
      we define a partition on $\underline{n}_+ = \set{0,1,2,\dots,n}$ where
      $i < j$ are in the same class iff there exists $(k_{i,j}, \sigma_{i,j}) \in \Aut(C)$ such that
      $\sigma_{i,j}(i) = j$ (so in particular $c_j = k_{i,j} c_i$).
      It is easy to check that this gives a well-defined partition/equivalence relation.
      Let $R \subseteq \underline{n}_+$ denote the subset of minimal representatives in each class
      (so in particular $0 \in R$),
      and $H_r \leq G$ the projection of $\Aut_r(C)$ onto $G$ for each $r \in R$.
      % Finally, fix choices of $k_{r,j}$ for all $r \in R$ and $j$ in the same class as $r$ (again with $k_{r,r} = 1$).
      
      Now, by the completeness of $\F_1$ and the essential surjectivity of $F$,
      for all $r \in R$ there exist $d_r \in \C(\O)^{H_r}$ such that
      $F(d_r)$ is (homotopy) $H_r$-equivalent to $d_r$.
      %
      Extend the set $\set{d_r}_{r\in R}$ to a signature $D = (d_1,\ldots, d_n;d_0)$
      by defining $d_j = k_{r,j} \cdot d_r$;
      as in Lemma \ref{AUTC_LEM}, these are independent of the choice of $(k_{r,j}, \sigma_{r,j})$.
      % Consequently, $F(c_i)$ is homotopy equivalent to $d_i$ via $k_{r,i}\gamma_r$,
      % where $\gamma_r$ realizes the homotopy equivalence between $F(c_r)$ and $d_r$ for $i \in \lambda_r$.
      This yields a diagram of the form
      \begin{equation}
            \label{TWOOFTHREE_EQ}
            \begin{tikzcd}
                  \O(d_1,\ldots, d_n;d_0) \arrow[r, "(1)"]
                  &
                  \P(F(d_1),\ldots, F(d_n); F(d_0)) \arrow[d,dash, "(3)"] \arrow[r, "(2)"]
                  &
                  \Q(LF(d_1),\ldots, LF(d_n);LF(d_0)) \arrow[d, dash, "(4)"]
                  \\
                  &
                  \P(c_1,\ldots, c_n;c_0) \arrow[r, "(5)"]
                  &
                  \Q(L(c_1),\ldots, L(c_n); L(c_0)).
            \end{tikzcd}
      \end{equation}
      $(1)$ is a weak equivalence in $\V^{\Aut(D)}_{\F_D}$ by assumption, and
      $(2)$ is a weak-equivalence in $\V^{\Aut(D)}_{\F_D}$ by 2-out-of-3 here.
      $(3)$ and $(4)$ are zig-zags of weak equivalences in $\V^{\Aut(C)}_{gen}$ by iterating applications of
      Proposition \ref{CAV_4.14_PROP2},
      as each application only changes the colors in a single partition class.
      As these zig-zags are functorial, the above diagram commutes.
      %
      Finally, $\Aut(C) \leq \Aut(D)$ by the same argument as in Lemma \ref{AUTC_LEM}.
      Thus $(5)$ is a weak equivalence in $\V^{\Aut(C)}_{\F_C}$ by 2-out-of-3, and hence
      $L$ is a local weak $\F$-equivalence, as desired.
\end{proof}

Thus we have established the veracity of Theorem \ref{MODEL_THM}.












% ------------------------------- DWYER-KAN DESCRIPTION -----------------------------

\subsection{Dwyer-Kan equivalences}
\label{DK_SEC}

To complete the proof of Theorem \ref{INTRO_MODEL_THM}, 
we need to recognize our weak $\F$-equivalences slightly differently.

\begin{definition}
      We say $F: \O \to \P$ in $\Op^G(\V)$ is called \textit{$\pi_0$-essentially surjective} if
      $j^{\**}\pi_0(F^H)$ is essentially surjective for all $H \leq G$.
      
      A map $F: \O \to \P$ in $\Op^G(\V)$ is a \textit{Dwyer-Kan} (or \textit{DK}) \textit{$\F$-equivalence} if
      $F$ is a local weak $\F$-equivalence and $\pi_0$-essentially surjecitve.
\end{definition}

\begin{remark}
      If $\V$ has diagonals, then one can show that $F \in \Op^G(\V)$ is a $DK$-$\F$-equivalence iff
      the following seemingly strong condition holds:
      $F$ is a local weak $\F$-equivalence such that 
      the associated map of \textit{$\F$-genuine equivariant operads} under the composite
      \begin{equation}
            \Op^G(\V) \to \Op_\F(\V) \xrightarrow{\pi_0} \Op_\F(\Set) 
      \end{equation}
      is an equivalence.
      % This will be explored further, along with colored genuine equivariant operads, in a sequal.
\end{remark}

\begin{definition}
      \label{DK_MODEL_DEF}
      We say $\Op^G(\V)$ has the \textit{Dwyer-Kan (semi)-model structure} if the (semi)-model structure from
      Theorem \ref{MODEL_THM} exists, and the weak $\F$-equivalences agree with the DK-$\F$-equivalences.
\end{definition}

We always have containment in one direction:
\begin{proposition}
      \label{WE_ARE_DK_PROP}
      Weak $\F$-equivalences in the sense of Theorem \ref{MODEL_THM} are DK-$\F$-equivalences.
\end{proposition}
\begin{proof}
      By Lemma \ref{VIR_HTPY_LEM}, essential surjectivity implies $\pi_0$-essential surjectivity. 
\end{proof}

This leads to the following expected result.
\begin{corollary}
      Suppose the $(G, \Sigma)$-family $\F$. % has units,
      and let $F: \O \to \P$ be a weak $\F$-equivalence.
      Then we have an equivalence $\pi_0(F^H)$ on underlying fixed-point categories
      for all $H \in \F_1$;
      if additionally $H \in \F_n$ for all $n \geq 0$, then      
      $\pi_0(F^H)$ is an equivalence of operads in $\Set$.
\end{corollary}
\begin{proof}
      Both statements have identical proof.
      As $\pi_0$ sends weak equivalences to isomorphisms, any local weak equivalence becomes a local isomorphism.
      % Further, the proof of Lemma \ref{VIR_HTPY_LEM} shows that any equivalence $\mathbb J \to \mathcal D_f$ in a fibrant category
      % yields a homotopy equivalence $[\tilde 1] \to \pi_0(\mathcal D_f)$.
      Then Lemma \ref{VIR_HTPY_LEM} implies that the composite
      \[
            \pi_0(\O^H) \to \pi_0(\P^H) \to \pi_0(\P^H)_f
      \]
      is essentially surjective, and hence an equivalence of operads.
      Thus the first arrow is a equivalence of operads by two-out-of-three, as desired.
\end{proof}

% % ------------------------------ other interesting results, need some of the discussion below ------------------------------

% {\color{OliveGreen} % -------------------- OLIVE GREEN --------------------
%   We have the following (non-equivariant) consequence of Lemma \ref{VIR_HTPY_LEM}.
%   \begin{lemma}
%         Suppose $\V$ is a cofibrantly generated monoidal model category such that $\Op(\V)$ has the transferred model structure.
%         \todo[inline]{what are the actual hypotheses?}
%         Suppose $F: \O \to \P_f$ is a weak equivalence in $\Op(\V)$, with $\P_f$ fibrant.
%         Then $\pi_0(F)$ is an equivalence of operads.
%   \end{lemma}
%   \begin{proof}
%         Since $\mathcal C(c,d) \to \mathcal D_f(F(c), F(d))$ is a weak equivalence in $\V$ for all objects $c,d \in \mathcal C$,
%         it certainly becomes an isomorphism in $\Ho(\V)$.
%         Moreover, the fact that objects being virtually equivalent implies they are homotopy equivalent says that
%         when the target is fibrant, any equivalence $\mathbb J \to \ D_f$ yields a homotopy equivalence $\mathbb I \to \pi_0(\mathcal D_f)$,
%         and hence essential surjectivity in $\Cat(\V)$ implies essential surjectivity at $\pi_0$.
%   \end{proof}
  
%   \begin{corollary}
%         For any $\mathcal \O \in \Op(\V)$, we have a natural equivalence of operads
%         $\pi_0(\mathcal \O) \to \pi_0(\mathcal \O_f)$. 
%   \end{corollary}
% }% ------------------------------ OLIVE GREEN ------------------------------



For the reverse direction (cf. \cite[\S 2]{BM13}), we need to show that
homotopy equivalences are all virtual equivalences.
This requires another condition on $\V$, namely that the homotopy equivalences all satisfy a ``coherence'' condition.
% originally due to Boardman-Vogt \cite{BV73}, and extended by Berger-Moerdijk \cite{BM13}.

\begin{definition}
      Recall the category $\mathbb A \in \Cat^{\set{0,1}}(\V)$ which detects arrows.
      A cofibration $\mathbb A \to \J$ in $\Cat^{\set{0,1}}(\V)$ into a $\V$-interval is called \textit{natural} if
      it fits into a commuting diagram of the following form in $\Cat^{\set{0,1}}(\V)$.
      \begin{equation}
            \begin{tikzcd}
                  \mathbb A \arrow[d, tail] \arrow[r]
                  &
                  \I \arrow[d, "\sim"]
                  \\
                  \J \arrow[r, "\sim"']
                  &
                  \I_f.
            \end{tikzcd}
      \end{equation}

      A homotopy equivalence between two objects in a $\V$-category $\mathcal C$ is called \textit{coherent} if
      the detecting map $\alpha: \mathbb A \to \mathcal C_f$ factors along a natural cofibration
      \begin{equation}
            \begin{tikzcd}
                  \mathbb A \arrow[r, "\alpha"] \arrow[d, tail, dashed]
                  &
                  \mathcal C_f
                  \\
                  \J \arrow[ur, dashed]
            \end{tikzcd}
      \end{equation}
      A monoidal model category $\V$ is said to satisfy the \textit{coherence axiom} if
      all homotopy equivalences in every $\V$-category are coherent.
\end{definition}

\begin{proposition}[{cf. \cite[Prop. 2.20]{BM13}}]
      \label{COH_DK_ARE_WE_PROP}
      If $\V$ is right proper with cofibrant unit satisfying the coherence axiom, then
      DK-$\F$-equivalences are weak $\F$-equivalences in $\Op^G(\V)$.
\end{proposition}
\begin{proof}
      It suffices to show that any homotopy $H$-equivalence between objects in some $\O \in \Op^G(\V)$
      is in fact an $H$-equivalence.
      % Now, the proof of Lemma \ref{VIR_HTPY_LEM} in fact shows that
      % any arrow in a $\V$-category $\mathcal C$ which factors through a natural cofibration $\mathbb A \to \mathbb J$ encodes a homotopy equivalence,
      % \todo[inline]{confirm this} 
      The coherence axiom implies that all homotopy $H$-equivalences are virtual $H$-equivalences,
      while right properness and Lemma \ref{RIGHTPROPER_LEM} imply these are actual $H$-equivalences.
\end{proof}

We note that this is a \textit{non-equivariant} condition on $\V$.
It has been proven in the literature for many categories:
\begin{itemize}      
\item The notion originated with Boardman-Vogt, who showed it holds for compactly-generated weak Hausdorff spaces $(\Top, \times)$ \cite[Lem. 4.16]{BV73};
\item Using a generalization of this argument, Berger-Moerdijk showed it holds for any category $\V$ which satisfies transfer for operads, is right proper, and has a cofibrant unit \cite[Prop. 2.24]{BM13}.
\item As recreated in Proposition \ref{SSET_COH_PROP} below, Joyal showed it holds for $(\sSet, \times)$ with the Quillen model structure;
\item This axiom is also a consequence of Lurie's \textit{invertibility hypothesis} \cite[A.3.2.12]{Lur09} by an argument of Berger-Moerdijk \cite[Rem. 2.19]{BM13}.
      This adds the example of $\sSet$ with the Joyal model structure, among others \cite[A.3.2.23]{Lur09}.
\end{itemize}

We prove it here for simplicial sets and pointed simplicial sets.
\begin{proposition}
      [{cf. \cite[\S 1]{Joy02}}]
      \label{SSET_COH_PROP}
      $(\sSet, \times)$
      satisfies the coherence axiom.
\end{proposition}
\begin{proof}
      Let $\mathcal C_f \in \Cat(\sSet)$ be (locally) fibrant, and
      suppose $\alpha: \mathbb A \to \mathcal C_f$ realizes a homotopy equivalence.
      Recall the Quillen equivalence $W_!\colon \sSet_{Kan} \rightleftarrows \Cat(\sSet) \, \colon \! h c N$.
      Then, as $W_!\Delta[1] = [1] = \mathbb A$, we have an adjoint map $\tilde \alpha: \Delta[1] \to h c N \mathcal C$,
      which realizes a quasi-isomorphism in the $\infty$-category $h c N \mathcal C$ since $\Ho(h c N (-)) \simeq \pi_0(-)$ on fibrant objects.
      Since $\Delta[1] \to J = N \I$ is anodyne by \cite[Corollary 1.6]{Joy02} or \cite[Lemma 0.15]{Rie}, 
      $\tilde \alpha$ factors through $J$,
      so the adjoint $\alpha$ factors through $W_!J$.
      The result then follows since $W_!J$ is a $\sSet$-interval:
      $W_!J$ is cofibrant in $\sCat^{\set{0,1}}$ as we have a factorization $\set{0,1} \to \Delta[1] \to J$ and $W_!$ is left Quillen,
      and the counit of the Quillen equivalence % $W_!\colon \sSet \rightleftarrows \sCat \ \colon \! h c N$
      yields a weak equivalence $W_!J \xrightarrow{\simeq} [\tilde 1] = \mathbb I = \mathbb I_f$ in $\Cat^{\set{0,1}}(\sSet)$.
      % This weak equivalence also holds sinceSince $J$ is contractible (as it models $S^\infty$), we're done.
\end{proof}

\todo[inline]{do we need Lemmas \ref{INTER_LEM} and \ref{WJ EX}? Or, for that matter, \ref{SSET_COH_PROP}? This is basically a recreation of the cited results using minimal new terminology.}

% In Appendix \ref{PT_SEC}, we prove Proposition \ref{PT_MODEL_COR}, of which the following is a special case.
\begin{corollary}
      \label{PTSSETCOH_COR}
      $(\sSet_{\**}, \wedge)$ satisfies the coherence axiom.
\end{corollary}
\begin{proof}
      The disjoint basepoint functor $(-)_+: \sSet \to \sSet_{\**}$ is alway left Quillen.
      Since additionally the unit in $\sSet$ is the terminal object and is cofibrant, and $\sSet$ is left proper,
      this functor also preserves all weak equivalences.
      {\color{OliveGreen} % ----------------------------------------
        \begin{equation}
              \begin{tikzcd}
                    A \arrow[r, rightarrowtail] \arrow[d, "\simeq"']
                    &
                    A \amalg \** \arrow[d, "\simeq"]
                    \\
                    B \arrow[r, rightarrowtail]
                    &
                    B \amalg \**
              \end{tikzcd}
        \end{equation}
      } % --------------------------------------------------
      Using this and the fact that the unit is additionally fibrant, we have that $\mathbb J_+$ is a $\sSet_{\**}$-interval for any $\sSet$-interval $\mathbb J$,
      and then an easy adjunction argument finishes the proof.
\end{proof}

We now have collected all the remaining pieces to prove Theorem \ref{INTRO_MODEL_THM} from Theorem \ref{MODEL_THM}.

\begin{proof}
      [Proof of Theorem \ref{INTRO_MODEL_THM}]
      Propositions \ref{WE_ARE_DK_PROP} and \ref{COH_DK_ARE_WE_PROP},
      combined with Proposition \ref{SSET_COH_PROP} and Corollary \ref{PTSSETCOH_COR},
      imply that when $\V$ satisfies the coherence axiom, 
      the $\F$-(semi)-model structures from Theorem \ref{MODEL_THM} are in fact the $\F$-Dwyer-Kan model structures.
\end{proof}




\begin{remark}
      \label{FIB_ISOFIB_REM}
      When $\V$ satisfies the coherence condition, we also have an additional nice description of fibrations:
      A map $F: \O \to \P$ in $\Op^G_\F(\V)$ is a fibration iff
      $F$ is a local $\F$-fibration such that
      $j^{\**}\pi_0(F^H)$ is an isofibration of 1-categories.
      Indeed, the arguments in \cite[Propositions 2.3 and 2.5]{Ber07b} extend almost as written to the general case,
      notationally replacing $\mathcal H$ with an arbitrary $\V$-interval $\mathbb J$ and
      $\mathscr F$ with $\mathbb A$.
\end{remark}


% \begin{example}
%       Explicitly, when $(\V,\otimes) = (\sSet, \times)$, we get the above description using the quintessential model of an interval object.
%       Let $W_!$ denotes the left adjoint to the homotopy coherent nerve
%       $N_{hc}: \sCat \to \sSet$,
%       and we note that $W_! J$ is a $\sSet$-interval, where $J$ is the nerve of the walking isomorphism $J = N \tilde{[1]}$, by Example \ref{WJ EX}.
      
%       Thus a map $f \in \sOp$ being path-lifting implies in particular that $h c N(f)$ has the right lifting property against $\** \to N \tilde{[1]}$,
%       and since $\tau \circ h c N(-) \simeq \pi_0$ by \cite[Prop. 4.8]{CM11}
%       \todo[inline]{this only works for fibrant objects - does that kill this argument?}
%       this implies that $\pi_0(f)$ has the right lifting property against $\** \to \tilde{[1]}$;
%       that is, $f$ being path-lifting implies $j^{\**}\pi_0(f)$ is an isofibration of categories.    
% \end{example}


\begin{example}
      The category $\Top$ of compactly-generated weak-Hausdorff spaces satisfies all the hypotheses of our Theorem.
      Indeed, $(\Top, \times)$:
      \begin{enumerate}[label = (\roman*)]\itemsep-4pt
      \item is cofibrantly generated (e.g. \cite{Pia91});
      \item is a closed monoidal model category with cofibrant unit (e.g. \cite[Prop. 4.2.11]{Hov99});
      \item has cellular fixed points by \cite{Pia91} (see also e.g. \cite[Lemma 3.18]{Ste16});
      \item has cofibrant symmetric pushout powers since geometric realization is left Quillen and strong monoidal;
      \item is right proper;
      \item has a generating set of intervals, as all objects are fibrant (see \cite[Lemma 2.1]{BM13}); and
      \item is coherent by \cite[Lem. 4.16]{BV73}.
      \end{enumerate}
      Thus, combining Theorem \ref{MODEL_THM} with Remarks \ref{TOP_FULL_REM} and \ref{OPGCV_FULL_REM},
      $\Op^G(\Top)$ has the $\F$-Dwyer-Kan model structure for any $(G, \Sigma)$-family $\F$. % with units.

      However, as is often the case, pointed spaces can be poorly behaved unless one restricts to the category of well-pointed or closedly pointed spaces. % \todo{though these may still be weakly cellular, which could be enough}
\end{example}


% ----------------------------------------------------------------------------------------------------
{\color{OliveGreen} % ---------------------------------------- OLIVE GREEN --------------------
  \begin{example}
        Possible examples:
        \begin{itemize}
        \item $\Gamma$-spaces?
        \item dg-modules with the projective model structure?
        \item simplicial modules over a ring? \cite[\S 3.1.15]{Rez96}, \cite[Example 4.23]{Cav}
        \end{itemize}
  \end{example}

  % \begin{example}
  %       $R$ a commutative ring containing the rational numbers,
  %       $\mathcal A$ the abelian category of projective $R$-modules,
  %       and consider $Ch(\mathcal A)$ with projective model strucutre:

  %       \begin{itemize}
  %       \item cellularity in \cite{Ste16},
  %       \item cofibrant symmetric pushout powers implied by ``freely powered'', proved in \cite[Prop 7.1.4/7]{Lur},
  %       \item right proper?
  %       \item generating set of intervals?
  %       \item cofibrant unit?            
  %       \end{itemize}

  %       More generally, $\mathcal C$ locally presentable quasi-abelian category,
  %       $R$ a commutative monoid object in $\mathcal C$ containing the rational numbers,
  %       and consider $dg_R(\mathcal C)$ with the projective model structure \cite[Prop 2.12]{Wal15};
  %       this also satisfies cofibrant symmetric pushout powers by \cite[Prop 3.4]{Wal15}.
  % \end{example}

  \begin{example}
        Non-examples:
        \begin{itemize}
        \item $\Set$, $\Cat$, (none are cellular)
        \item $\mathsf{Ch}$ (only weakly cellular - can we do something with this?)
        \item any model for $(\infty,1)$-cats (none are right proper)
        \end{itemize}
  \end{example}

  \begin{remark}
        We say one note about the choice of conditions. There are several notions which have a similar form to
        ``cofibrant symmetric pushout powers'' and play a similar role in transferring model structures:
        freely powered of Lurie,
        the commutative monoid axiom of White-Yau,
        symmetric h-monodial of Pavlov-Scholbach,
        and the cofibration hypothesis of Mandell-May-Schwede-Shipley,
        to name just a few.

        It is straightforward to check that cofibrant symmetric pushout powers is a (much) weaker condition than freely powered,
        but slightly stronger than the commutative monoid axiom.
        % The first is obvious.
        % The second follows since the adjunction
        % \[
        %       G/G \cdot (-): \V \leftrightarrows \V^G: (-)/G
        % \]
        % is Quillen, as for all $A \in \V$ and $H \leq G$, $(G/H \cdot A)/G \cong A$
        % (see also \cite[Lemma 4.5.4.11]{Lur})

        \todo[inline]{connection to the others?}
  \end{remark}

  \todo[inline]{how much of \cite{WY} extends via this arrangement?}

} % ---------------------------------------- OLIVE GREEN ----------------------------------------







\fi%

\newpage

\section{Review of revelent model structures}

In this expository section, we recall the main features of the model structures necessary for the later sections.
Full details and discussion can be found in \cite{BP_edss} and \cite{Per_eds}.

\subsection{Equivariant simplicial operads}

\begin{itemize}
\item $\Sigma$-cofibration
\end{itemize}


\subsection{Equivariant dendroidal sets}
\label{EDS_SEC}

We begin with a brief overview of $G$-trees and equivariant dendroidal sets, whose discovery/definition is central and motivating for this entire project.

\begin{definition}
      The category $\Omega_G$ of \textit{$G$-trees} is the category of indecomposable forests with $G$-action.
      Equivalently, an object $T \in \Omega_G$ can be described as any of the following:
      \begin{enumerate}
      \item A collection $T = (T_i)_{i \in R}$ of trees in $\Omega$, together with a compatible $G$-action on the whole system.
      \item A functor $T: G \ltimes R \to \Omega$ for $R = \mathbf R(T)$ the transitive $G$-set of \textit{roots} of $T$.
      \item An induction $T \simeq G \cdot_H T_e$ for some $T_e \in \Omega^H$ and $H \leq G$.
            % \item The quotient $T \simeq (G \cdot T_e) / N$ for $N$ the graph of the homomorphism $H \to \Aut(T_e)$ encoding the $H$-action.
      \end{enumerate}
\end{definition}

We consider the na\'ive presheaf category 
$
\dSet^G = \Fun(G \times \Omega^{op}, \Set),
$
of \textit{equivariant dendroidal sets}.
There is a natural inclusion
\[
      \Omega[-] \colon \Omega_G \to \dSet^G,
      \qquad
      \Omega[T] = \coprod \Omega[T_i] \simeq G \cdot_H \Omega[T_e].
      % \simeq \left(G \cdot \Omega[T_e] \right) / N
\]
extending the Yoneda embedding $\Omega \times G \into \dSet^G$.
Even though the presheaf category is na\'ive, this functor allows us to see genuine equivariant information recorded by $G$-trees
and build a more ``genuine'' model structure.
To begin, we make the following definitions (see \cite[\S 6]{Per_eds}).

\begin{definition}
      For $T \in \Omega_G$, let $\partial \Omega[T]$ denote the \textit{boundary}
      \[
            \partial \Omega[T] = \coprod \partial \Omega[T_i] \simeq G \cdot_H \partial \Omega[T_e].
            % \simeq \left(G \cdot \partial \Omega[T_e] \right) / N.
      \]
      The \textit{boundary inclusions} are maps in $\dSet^G$ of the form
      \[
            \partial \Omega[T] \to \Omega[T] =
            \coprod \big( \partial \Omega[T_i] \to \Omega[T_i]\big) \simeq
            G \cdot_H \big( \partial \Omega[T_e] \to \Omega[T_e] \big).
            % \simeq
            % \left( G \cdot \left( \partial \Omega[T_e] \into \Omega[T_e] \right) \right) / N.
      \]
\end{definition}

\begin{definition}
      Given $T \in \Omega_G$ and $e \in E(T)$, the associated \textit{$G$-inner horn} is the sub presheaf
      \[
            \Lambda^{Ge}[T] \into \partial \Omega[T] \into \Omega[T],
            \qquad
            \Lambda^{Ge}[T] = \amalg \Lambda^{H_i e}{T_i} \simeq G \cdot_H \Lambda^{He}[T_e].
            % \simeq \left(G \cdot \Lambda^{He}[T_e] \right) / N.
      \]
      The \textit{generating $G$-inner horn inclusion} are maps in $\dSet^G$ of the form
      \[
            \Lambda^{Ge}[T] \to \Omega[T]
      \]
      for $T \in \Omega_G$ and $e \in E(T)$.

      A presheaf $X$ is called a \textit{$G$-$\infty$-operad} if $X \to \**$ has the right lifting property with respect to all generating $G$-inner horn inclusions.
\end{definition}

\begin{definition}
      The class of \textit{$G$-normal monomorphisms}
      is the smallest saturated\footnote{A class of maps is \textit{saturated} if it is closed under pushouts, retracts, and transfinite compositions.}
      class of maps containing the boundary inclusions $\partial \Omega[T] \into \Omega[T]$.
      A map is called a \textit{trivial fibration} if it has the right lifting property with respect to the $G$-normal monomorphisms.
      
      The class of \textit{$G$-inner anodyne extensions} is the smallest saturated class of maps containing the generating $G$-inner horn inclusions.
\end{definition}

These two classes of maps form the basis of a model structure on $\dSet^G$.

\begin{theorem}[{\cite[Thm 2.1, Thm 8.22]{Per_eds}}]
      There exists a model structure on $\dSet^G$ such that
      \begin{itemize}
      \item cofibrations are $G$-normal monomorphisms,
      \item fibrant objects are $G$-$\infty$-operads,
      \item fibrations between fibrant objects are precisely the maps $X \to Y$ such that the functors associated to each of the fixed-point homotopy categories $\tau j^{\**}(X^H \to Y^H)$ are categorical fibrations for all $H \leq G$.
      \item weak equivalences are the smallest hypersaturated class\footnote{A class of maps is \textit{hypersaturated} if it is saturated and closed under 2-out-of-3.}
            of maps containing the $G$-inner anodyne extensions and trivial fibrations.
      \end{itemize}
\end{theorem}

More generally, similar model structures exist for any (weak) indexing system.
\begin{definition}
      A \textit{weak indexing system} is a sieve\footnote{
        A subcategory $\mathcal B \subseteq \mathcal C$ is a \textit{sieve} if for all maps $A \to B$ in $\mathcal C$
        with $B \in \mathcal B$, both $A$ and $f$ are in $\mathcal B$.}
      $\Omega_\F \subseteq \Omega_G$.
\end{definition}

Unpacking, let $\Sigma_\F \subseteq \Sigma_G$ denote the image of $\Omega_\F$ under the functor $\mathsf{lr}$.
As $\Sigma_G$ is equivalent to the category $\coprod_n \O_{\mathrm{Gr}_n}$
where $\mathrm{Gr}_n$ is the collection of graph subgroups of $G \times \Sigma_n$ \todo{citation},
$\Sigma_\F$ corresponds to some sub-$(G,\Sigma)$-collection $\F = \set{\F_n}$ of $\mathrm{Gr} = \set{\mathrm{Gr}_n}$
so each $\F_n$ is a family of graph subgroups of $G \times \Sigma_n$.      

Extending the above, a map is \textit{$\F$-normal} (resp. \textit{$\F$-anodyne})
if it is in the smallest saturated class of maps containing the
boundary inclusions (resp. generating $G$-inner horn inclusions) for $T \in \Omega_\F$.
Additionally, a map is a \textit{trivial $\F$-fibration} (resp. \textit{$\F$-$\infty$-operad}) if it has the right lifting property with respect to $\F$-normal maps (resp. $\F$-anodyne maps).

\begin{theorem}[{\cite[\S 9]{Per_eds}}]
      There exists an $\F$-model structure on $\dSet^G$, denoted $\dSet^G_\F$, such that
      cofibrations are $\F$-normal monomorphisms,
      fibrant objects are $\F$-$\infty$-operads,
      fibrations between fibrant objects are precisely the maps $X \to Y$ such that the functors associated to each of the fixed-point homotopy categories $\tau j^{\**}(X^H \to Y^H)$ are categorical fibrations for all $H \leq G$,
      and weak equivalences are the smallest saturated class of maps containing $\F$-inner anodyne extensions and trivial $\F$-fibrations, and is closed via 2-out-of-3.
\end{theorem}

These results will be applied in Proposition \ref{W!_COF_PROP}, the main result necessary to prove that $W_!$ is left Quillen.




















\subsection{Equivariant dendroidal Segal spaces and the joint Bousfield model structure}
\label{JT_SEC}

terminology we need:

We recall three of the model structures on $\mathsf{sdSet}^G = \Set^{\Delta^{op} \times \Omega^{op} \times G}$ used in \cite{BP_edss},
coming from the two different (generalized) Reedy categories.
The simplicial category $\Delta$ is Reedy, and with the model structure on $\dSet^G$ from \cite{Per_eds} this produces the
\textit{simplicial Reedy} model structure on  $(\dSet^G)^{\Delta^{op}}$.
Dually, $\Omega^{op} \times G$ is a generalized Reedy category, and with the usual Kan-Quillen model structure on $\sSet$ produces the
\textit{dendroidal Reedy} model structure on $\sSet^{\Omega^{op} \times G}$.

An organizational feature of \cite{BP_edss} was the use of joint Bousfield localizations:
\begin{definition}[{\cite[Prop. 4.1]{BP_edss}}]
      
\end{definition}

\begin{definition}
      A map $f: X \to Y$ in the \textit{equivariant dendroidal Reedy model structure} on $\mathcal M^{\Omega^{op} \times G}$
      is a weak equivalence iff $f(U): X(U) \to Y(U)$ is a $G$-graph weak equivalence in $\mathcal M^{G \times \Aut(U)}$.
\end{definition}

\begin{notation}
      For $X \in \mathsf{sdSet}^G$, we have unique colimit-preserving functors
      \[
            X_{(-)} \colon \sSet \to \dSet^G,
            \qquad
            X(-) \colon \dSet^G \to \sSet
      \]
      such that $X_{\Delta[n]} = X_n$ and $X(\Omega[U]) = X(U)$ for $n \geq 0$, $U \in \Omega$.
\end{notation}

\begin{lemma}
      \label{COMUOTOHOM_FACTS}
      Facts we need for Proposition \ref{COMUOTOHOM PROP}:
      \begin{enumerate}[label = (\roman*)]
      \item vertical/simplicial and horizontal/dendroidal equivalences are also joint equivalences
      \item weak equivalences in $\mathsf{PreOp}^G$ are detected by $\gamma^{\**}$.
            %
      \item Reedy (co)fibrations are levelwise (co)fibrations. 
            In particular, joint fibrantions are levelwise fibrations in either direction.
            
            More generally, for any Reedy-admissible collection of families $\set{\mathcal F_r}_{r \in \mathbb R}$,
            $f$ a Reedy (co)fibration implies that $f_r$ is a $\mathcal F_r$-(co)fibration in $\mathcal M^{\Aut(r)}_{\mathcal F_r}$. (\cite[Lemmas A.27, A.29]{BP_edss}).
            %
      \item \label{GENREEDY_LBL} The Reedy generating (trivial) cofibrations in $\mathcal M^{\Delta^{op}}$ and $\mathcal M^{G \times \Omega^{op}}$ are
            \[
                  (\partial \Delta[n] \to \Delta[n]) \square i,
                  \qquad
                  (\partial \Omega[T] \to \Omega[T]) \square i,
            \]
            for $n \geq 0$ and $T \in \Omega_G$,
            where $i$ is a generating (trivial) cofibration in $\mathcal M$
            ({\cite[Prop. A.33, Example A.34]{BP_edss}}).
            % More generally, the $R$-Reedy generating (trivial) cofibrations in $\mathcal M^{\Delta^{op} \times R^{op}}$ are
            % \[
            %       (\partial \Delta[n] \to \Delta[n]) \square i'
            %       \qquad
            %       (\Lambda^i[n] \to \Delta[n]) \square i'
            % \]
            % where $i'$ is a generating cofibration for the Reedy model structure on $\mathcal M^{R^{op}}$.
            % ------------------------------
            % ----------
      \item \label{PRODREEDY_LBL} If $\mathbb R$ and $\mathbb S$ are generalized Reedy with
            corresponding Reedy admissible families $\set{\F_r}_{r \in \mathbb R}$, $\set{\F_s}_{s \in \mathbb s}$,
            then the $\set{\F_{r,s}}$-Reedy model structure on $\mathcal M^{\mathbb R \times \mathbb S}$
            can be equivalently obtained by considering the $\set{\F_r}$-Reedy model structure over the $\set{\F_s}$-Reedy model structure on $\mathcal M^{\mathbb S}$ (and vice versa)
            ({\cite[Remark A.31]{BP_edss}}).
            %
      %       Now, fix a group $G$ and a family of subgroups $\mathcal F$.
      % \item $G$ is a Reedy category, with $\set{\mathcal F}$ a Reedy-admissible collection.
      %       Thus we will say $X \in \left(\mathsf{sSet}^{\Delta^{op}}\right)^G_\F$ is \textit{$\mathcal F$-horizontal Reedy fibrant} if
      %       $X \in (\sSet)^{\Delta^{op} \times G}$ is $\set{e \times \mathcal F}$-Reedy fibrant;
      %       equivalently, if $X^H \in \mathsf{sSet}^{\Delta^{op}}$ is horizontally Reedy fibrant for all $H \in \mathcal F$.
            
      %       Moreover, the following model structures agree:
      %       \begin{itemize}
      %       \item the joint model structure on $\mathsf{sSet}^{\Delta^{op} \times G}$,
      %       \item the joint model structure on $\mathsf{s}\left(\mathsf{Set}^{\Delta^{op}}\right)^G$,
      %       \item the $\F$-model structure over the joint model structure on $\left(\mathsf{sSet}^{\Delta^{op}}\right)^G$.
      %       \end{itemize}            
            % ----------
      \item $X \to Y$ in $\mathsf{sdSet}^G$ is a simplicial equivalence iff $X(U) \to Y(U)$ is a $G$-graph equivalence in $\sSet^{G \times \Aut(U)}$ for all $U \in \Omega$ iff $X(\Omega[T]) \to Y(\Omega[T])$ is a Kan equivalence in $\sSet$ for all $T \in \Omega_G$.
            \begin{proof}
                  The first ``iff'' is by definition of the generalized Reedy structure on $\Omega^{op} \times G$.
                  Then, as $T \simeq G \cdot U/ \Gamma$ for some graph subgroup $\Gamma \leq G \times \Aut(U)$,
                  the second ``iff'' follows since $X(\Omega[T]) \simeq X(U)^\Gamma$.
            \end{proof}                  
            % 
      \item \label{SFIB_JEQ_LBL} $X,Y \in \mathsf{sdSet}^G$ simplicially fibrant, then $X \to Y$ is a joint equivalence iff it is a simplicial equivalence
            (\cite[Prop. 4.5(iii), Cor. 4.29(iii)]{BP_edss}).
            % ----------
      \item \label{RJOINTT_LBL} If $X \in (\mathsf{sdSet}^G)^{\Delta^{op}}$ is fibrant in the Reedy-over-joint model structure,
            then for all $T \in \Omega_G$, $X(\Omega[T])$ is horizontal Reedy fibrant in
            $\left((\mathsf{sSet})^{\Delta^{op}}\right)^{\Aut(T)}$.
            \begin{proof}
                  First, we note that for all $U \in \Omega$, $X(U)$ is $G$-graph horizontal Reedy fibrant in
                  $\left((\mathsf{sSet})^{\Delta^{op}}\right)^{G \times \Aut(U)}$;
                  this follows from the Quillen adjuction
                  \[
                        \left( \mathsf{sdSet}_{jt}^G \right)^{\Delta^{op}} \leftrightarrows
                        (\mathsf{sSet})^{\Omega^{op} \times G \times \Delta^{op}} =
                        \left( (\mathsf{sSet})^{\Delta^{op}} \right)^{\Omega^{op} \times G}
                  \]
                  where $(\sSet)^{\mathbb R}$ is given the generalized $\mathbb R$-Reedy model structure over the Kan model structure on $\sSet$.
                  Then, since $T \simeq G \cdot U/ \Gamma$ for some graph subgroup $\Gamma \leq G \times \Aut(U)$,
                  the result follows since $X(\Omega[T]) \simeq X(U)^\Gamma$.
            \end{proof}
            %
      \item \label{JF_VERT_LBL}
            $X \in \mathsf{sSet}^{\Delta^{op}}$ is joint fibrant iff
            $X$ is horizontal Reedy fibrant and all vertex maps $X(m) \to X(0)$ are Kan equivalences in $\mathsf{sSet}$
            (\cite[Prop. 4.24(ii)]{BP_edss}).
            %
      \item \label{DIAG_LBL}
             If $X \in \mathsf{sSet}^{\Delta^{op}}$ is horizontal Reedy firant, then
             $X_n \to X_0$ and hence $X_0 \to \delta^{\**}X$ is a Kan equivalence in $\mathsf{Set}^{\Delta^{op}}$.
             %
      % \item \label{JF_VERTG_LBL}
      %       $X \in \mathsf{sSet}^{\Delta^{op} \times G}$ is joint fibrant iff
      %       $X$ is horizontal Reedy fibrant and all vertex maps $X(m) \to X(0)$ are $\mathcal F$-weak equivalences in $\mathsf{sSet}^G$
      %       (\cite[Prop. 4.24(ii)]{BP_edss}).
      %       %
      % \item \label{DIAGG_LBL} If $X \in \mathsf{sSet}^{\Delta^{op} \times G}$ is horizontal Reedy firant, then
      %       $X_n \to X_0$ and hence $X_0 \to \delta^{\**}X$ is an $\mathcal F$-Kan equivalence in $\mathsf{Set}^{\Delta^{op} \times G}$.
      %       ----------
      \end{enumerate}
\end{lemma}





\subsection{Segal preoperads}
\label{SPREOP_SEC}

Words:
\begin{enumerate}
\item preoperad
\item evaluations $X(A)$, $A \in \mathsf{dSet}^G$
\item complete equivalences
\item $\mathsf{sk}_\eta$
\item $G$-Segal operad
\item mapping spaces
\item DK-equivalences
\end{enumerate}

\begin{theorem}[\cite{BP_edss}]
    The adjunction $\gamma^{\**} \colon \mathsf{PreOp}^G \rightleftarrows \mathsf{sdSet}^G \colon \gamma_{\**}$
    is a Quillen equivalence, where $\gamma^{\**}$ detects and reflects weak equivalences and cofibrations.
\end{theorem}








\section{The tame model structure}
\label{TAME_SEC}

In this section, we build an auxiliary model structure on preoperads, Quillen equivalent to the one recalled in \S \ref{SPREOP_SEC},
as well as prove Proposition \ref{KEYPR PROP},
the main tool which allows us to show that the nerve $\sOp^G \to \mathsf{PreOp}^G$ is an equivalence of homotopy theories.



\begin{definition}
	The \textit{colored tensor product} 
\[
\begin{tikzcd}[row sep = 0, column sep = 40pt]
	\mathsf{PreOp}^G \times \mathsf{sSet} \ar{r}{(-)\otimes_{\mathsf{F}}(-)} &
	\mathsf{PreOp}^G
\end{tikzcd}
\]
is defined by $(X \otimes_{\mathsf{F}} K)(T) = X(T) \times K$
whenever $T$ is a non-linear tree (equivalently, 
$\mathsf{Hom}_{\Omega}(T,\eta)=\emptyset$) and
is defined by the following pushout when $T=[n]$ is linear.
\[
\begin{tikzcd}
	X(\eta) \times K \ar{r} \ar{d} \arrow[dr, phantom, "\ulcorner", very near start]  &
	X(\eta) \ar{d}
\\
	X([n]) \times K \ar{r} & 
	(X \otimes_{\mathfrak{C}} K)([n]) 
\end{tikzcd}
\]
\end{definition}

\begin{remark}
More concisely, $X \otimes_{\mathfrak{C}} K$ is defined by the pushout
\[
\begin{tikzcd}
	\left(\mathsf{sk}_{\eta}X \right) \times K \ar{r} \ar{d} \arrow[dr, phantom, "\ulcorner", very near start]  &
	\mathsf{sk}_{\eta}X \ar{d}
\\
	X \times K \ar{r} & 
	X \otimes_{\mathsf{F}} K 
\end{tikzcd}
\]
\end{remark}


\begin{remark}
For fixed $K \in \mathsf{sSet}$, the functor
$(-) \otimes_{\mathsf{F}} K
\colon \mathsf{PreOp}^G \to \mathsf{PreOp}^G$
preserves all colimits (indeed, it is not hard to build the right adjoint explicitly).

However, for a fixed $X \in \mathsf{PreOp}^G$,
the functor 
$X \otimes_{\mathsf{F}} (-)
\colon \mathsf{sSet} \to \mathsf{PreOp}^G$
does not preserve all colimits.
In particular, this functor can not preserve coproducts since, writing 
$\mathfrak{C} = X(\eta)$ for the $G$-set of objects of $X$,
the image of $X \otimes_{\mathsf{F}} (-)$ is entirely contained in the subcategory
$\mathsf{PreOp}^{G,\mathfrak{C}} \subset
\mathsf{PreOp}^G$
of preoperads with $G$-set of objects $\mathfrak{C}$ and maps which are the identity on objects. 
Instead, one has that the functor 
\[
X \otimes_{\mathsf{F}} (-) \colon
\mathsf{sSet} \to \mathsf{PreOp}^{G,\mathfrak{C}}
\]
does preserve colimits. 
In practice, this means that some standard arguments concerning tensor products can only be applied after adjusting the objects of the relevant preoperads
({\color{red} see later}).
\end{remark}


\begin{remark}\label{COLORTENSGAM REM}
Let $X \to Y$ be any map in $\mathsf{PreOp}^G$
which is the identity on colors and 
$K \in \mathsf{sSet}$. Then the squares below are pushout squares.
Moreover, whenever $K$ is connected the rightmost horizontal maps are isomorphisms.
\[
\begin{tikzcd}
	X \times K \ar{r} \ar{d} 
	\arrow[dr, phantom, "\ulcorner", very near start] &
	\gamma_! \left( X \times K \right) \ar{r} \ar{d} 
	\arrow[dr, phantom, "\ulcorner", very near start] &
	X \otimes_{\mathsf{F}} K \ar{d}
\\
	Y \times K \ar{r} &
	\gamma_! \left( Y \times K \right) \ar{r} &
	Y \otimes_{\mathsf{F}} K
\end{tikzcd}
\]
\end{remark}


\begin{definition}
	Let $f \colon \mathfrak{C} \to \mathfrak{D}$
	be a map of $G$-sets (of colors).
	We define adjoint functors
\[
	f_{!} \colon
	\mathsf{PreOp}^{G,\mathfrak{C}}
\rightleftarrows
	\mathsf{PreOp}^{G,\mathfrak{D}}
	\colon f^{\**}
\]
via the pushout and pullback squares
(note that $\mathsf{sk}_{\eta} f_! A$ depends only on 
$\mathfrak{C}$ while 
$\mathsf{csk}_{\eta} f^{\**} X$ depends only on
$\mathfrak{D}$)
\[
\begin{tikzcd}
	\mathsf{sk}_{\eta} A \ar{r} \ar{d} \arrow[dr, phantom, "\ulcorner", very near start]  &
	\mathsf{sk}_{\eta} f_! A \ar{d}
&&
	f^{\**} X \ar{r} \ar{d} &
	X \ar{d}
\\
	A \ar{r} & 
	f_! A
&&
	\mathsf{csk}_{\eta} f^{\**} X \ar{r} & 
	\mathsf{csk}_{\eta} X
	\arrow[ul, phantom, "\lrcorner", very near start]
\end{tikzcd}
\]
\end{definition}


\begin{definition}
	A $G$-preoperad $X \in \mathsf{PreOp}^G$ is called a \textit{$G$-Segal operad} if, 
	for each $G$-tree $T$,
	the natural map 
	$X\left( \Omega[T] \right) \to 
	X \left( Sc[T] \right)$
	is a Kan equivalence.
\end{definition}

\begin{notation}
Given a $G$-Segal operad $X$ and $G$-corolla $C$, 
$X(\partial \Omega[C])$ is a discrete simplicial set whose elements
are the $C$-signatures $(c_1,\cdots,c_n;c_0)$ of $X$.
The map $X(\Omega[C]) \to X(\partial \Omega[C])$ hence yields
a coproduct decomposition 
\[
X(\Omega[C]) \simeq \coprod_{C\text{-signatures }(c_1,\cdots,c_n;c_0)}
X(c_1,\cdots,c_n;c_0)
\]
\end{notation}


\begin{remark}\label{SEOPDK REM}
Given a $G$-Segal operad $X$, consider a dendroidal Reedy fibrant replacement $X \to \tilde{X}$ such that $X(\eta) \simeq \tilde{X}(\eta)$. 
This means that all maps 
$X(\Omega[T]) \to \tilde{X}(\Omega[T])$ are Kan equivalences,
and moreover, by the following pullback diagram
\[
\begin{tikzcd}
	Z (Sc[T]) \ar{r} \ar{d} &
	\prod_{v \in \boldsymbol{V}_G(T)} Z
	(\Omega[T_v]) \ar{d}
\\
	\prod_{(G/H_i \cdot e_i) \in \boldsymbol{E}_G(T)} 
	\mathfrak{C}^{H_i} \ar{r}  &
	\prod_{v \in \boldsymbol{V}_G(T)}
	\prod_{(G/H_i \cdot e_i) \in \boldsymbol{E}_G(T_v)} 
	\mathfrak{C}^{H_i} 
	\arrow[ul, phantom, "\lrcorner", very near start]
\end{tikzcd}
\]
so are the maps $X(Sc[T]) \to \tilde{X}(Sc[T])$.
This shows that $\tilde{X}$ is also a Segal operad, 
and thus a fibrant object in $\mathsf{PreOp}^G$.

Furthermore, the Kan equivalences 
$X(\Omega[C]) \to \tilde{X}(\Omega[C])$
induce Kan equivalences 
$X(c_1,\cdots,c_n;c_0) \to \tilde{X}(c_1,\cdots,c_n;c_0)$.
It follows that the complete equivalences between Segal operads are precisely the Dwyer-Kan equivalences. 
\end{remark}


\begin{remark}\label{SLIMOD REM}
Noting that for every fibrant 
$\tilde{X} \in \mathsf{PreOp}^G$
any equivalence in $\tilde{X}$ is in the image of a map
$J \to \tilde{X}$, 
a slight modification of the proof of Lemma \ref{INTER_LEM}
shows that for any Segal operad $X$
any equivalence in $X$ is in the image of a countable, contractible
$I \in \mathsf{PreOp}^G$
such that $\eta \amalg \eta \to I$
is a tame cofibration.
\end{remark}




\begin{theorem}
	There is a model structure on 
	$\mathsf{PreOp}^G$,
	called the \textbf{tame model structure},
	such that:
\begin{itemize}
	\item the weak equivalences are the complete equivalences (i.e. detected by inclusion into 
	$\mathsf{sdSet}^G$);
	\item generating cofibrations are given by the maps
	\begin{itemize}
		\item[(TC1)] $G/H \cdot \left(\emptyset \to\Omega[\eta]\right)$ for $H\leq G$;
		\item[(TC2)] $\Omega[C] \otimes_{\mathsf{F}} \left(\partial \Delta[n] \to \Delta[n]\right)$ for $C \in \Sigma_G$, $n \geq 0$;
		\item[(TC3)] 
$\left( Sc[T] \to \Omega[T] \right) 
\square_{\mathsf{F}} 
\left(\partial \Delta[n] \to \Delta[n]\right)$ for $T \in \Omega_G$, $n \geq 0$.
	\end{itemize}
\end{itemize}
Furthermore, one has generating anodyne cofibrations the maps
\begin{itemize}
	\item[(TA1)] $G/H \cdot 
	\left(\Omega[\eta] \to I \right)$ for $H \leq G$,
	and $\Omega[\eta] \to I$ a weak equivalence in $\mathsf{PreOp}$ such that $I(\eta) = \{0,1\}$, $\Omega[\eta] \amalg \Omega[\eta] \to I$ is a tame cofibration, and $I$ is countable;
	\item[(TA2)] $\Omega[C] \otimes_{\mathsf{F}}\left(\Lambda^i[n] \to \Delta[n]\right)$ for $C \in \Sigma_G$, $0 \leq i \leq n$;
	\item[(TA3)] 
$\left( Sc[T] \to \Omega[T] \right) 
\square_{\mathsf{F}} 
\left(\partial \Delta[n] \to \Delta[n]\right)$ for $T \in \Omega_G$, $n \geq 0$.
	\end{itemize}
\end{theorem}


\begin{proof}
	The existence of the model structure will follow by applying J. Smith's theorem \cite[Thm. 1.7]{Bek00}. Conditions c0 and c2 therein are inherited from $\mathsf{sdSet}^G$
	and the technical ``solution set condition'' c3 follows from
	\cite[Prop. 1.15]{Bek00} since weak equivalences are accessible, being the preimage by $\gamma^{\**}$ if the weak equivalences in 
	$\mathsf{sdSet}^G$ 
	(see \cite[Cor. A.2.6.5]{Lur09} and \cite[Cor. A.2.6.6]{Lur09}).
	
	For c1, we must show that any map $X \to Y$ with the right lifting property against (TC1), (TC2), (TC3) is a weak equivalence.
	Writing $f \colon \mathfrak{C} \to \mathfrak{D}$ for the underlying map of colors,
	consider the factorization $X \to f^{\**}Y \to Y$.
	Note that since maps out of (TC1) depend only on objects and both of (TC2) and (TC3) consist of maps which are identities on objects,
	$X \to Y$ will have the right lifting property against (TC1) iff 
	$f^{\**} X \to Y$ does
	and the right lifting property against 
	(TC2) and (TC3) iff $X \to f^{\**}Y$ does.
	
Note now that $f^{\**} Y \to Y$ has the right lifting proper against all maps 
	$\left(\partial \Omega[T] \to \Omega[T] \right) \times \Delta[n]$.
	Indeed, if $T \simeq G/H \cdot \eta$ is a stick, this is precisely the lifting condition agains (TC1), and otherwise it follows automatically since $\left(\partial \Omega[T] \to \Omega[T] \right) \times \Delta[n]$ is the identity on objects.
	Therefore, the levels 
	$\left(f^{\**} Y \right)_n \to Y_n$ are trivial fibrations in 
	$\mathsf{dSet}^G$, showing that 
	$f^{\**} Y \to Y$ is a dendroidal equivalence, 
	and thus a complete equivalence. 
	
	Since the maps in both of (TC2) and (TC3) are the identity on objects, $X \to Y$ has the right lifting property against these maps iff $X \to f^{\**}Y$ does.
The lifting property against (TC2) then says that the maps
$X(\Omega[C]) \to f^{\**} Y (\Omega[C])$
are trivial Kan fibrations for all $G$-corollas $C \in \Sigma_G$,
and thus so are the maps
$X(Sc[T]) \to f^{\**} Y (Sc[T])$ for all $G$-trees $T \in \Omega_G$.
But it then follows from the lifting property against
(TC3) that the maps 
$X(\Omega[T]) \to f^{\**} Y (\Omega[T])$
are trivial Kan fibrations for all $G$-trees,
showing that $X \to f^{\**} Y$ is a simplicial equivalence, and thus a complete equivalence. 
\[
\begin{tikzcd}
	X(\Omega[T]) \ar{r} \ar[->>]{d}{\sim} &
	X(Sc[T]) \ar{r} \ar[->>]{d}{\sim} &
	\prod_{v \in \boldsymbol{V}_G(T)} X(\Omega[T_v])
	\ar[->>]{d}{\sim}
\\
	f^{\**} Y(\Omega[T]) \ar{r} &
	f^{\**} Y(Sc[T]) \ar{r} \ar{d} &
	\prod_{v \in \boldsymbol{V}_G(T)} f^{\**} Y
	(\Omega[T_v]) \ar{d}
	\arrow[ul, phantom, "\lrcorner", very near start]
\\
	&
	\prod_{(G/H_i \cdot e_i) \in \boldsymbol{E}_G(T)} 
	\mathfrak{C}^{H_i} \ar{r}  &
	\prod_{v \in \boldsymbol{V}_G(T)}
	\prod_{(G/H_i \cdot e_i) \in \boldsymbol{E}_G(T_v)} 
	\mathfrak{C}^{H_i} 
	\arrow[ul, phantom, "\lrcorner", very near start]
\end{tikzcd}
\]
This completes the proof of c1, establishing the existence of the tame model structure.

We now turn to the ``further'' claim considering the claimed generating anodyne cofibrations, i.e., 
we wish to show that the maps in 
(TA1), (TA2), (TA3) satisfy the conditions in
Lemma \ref{SEMICOF LEM}.

We first check condition (i).
The case of maps in (TC1) is tautological.
Since $\Lambda^{i}[n]$ is connected, 
the maps in (TA2) have the form
$\gamma_{!} 
\left( \Omega[C] \times
\left( \Lambda^i[n] \to \Delta[n] \right) \right)$,
and are thus weak equivalences thanks to the pushouts
in Remark \ref{COLORTENSGAM REM}.
As for (TA3), it follows from Remark \ref{COLORTENSGAM REM}
that the maps
$\left( Sc[T] \to \Omega[T] \right) \otimes \partial \Delta[n]$
and 
$\left( Sc[T] \to \Omega[T] \right) \otimes \Delta[n]$
are trivial cofibrations, so that the claim follows from a standard pushout and 2-out-of-3 argument.

We now turn to condition (ii).
The lifting condition against (TA3) says that $J$-fibrant objects are such that the maps $X(\Omega[T]) \to X(Sc[T])$
are trivial fibrations, and thus that such $X$ are Segal operads.
Therefore, by Remark \ref{SEOPDK REM} it suffices to check that $J$-fibrations between Segal operads which are also DK equivalences are in fact trivial fibrations, i.e. that they have the right lifting property against the maps in (TC1),(TC2),(TC3).
Given $X \to Y$ a $J$-fibration with $J$-fibrant $Y$,
the lifting property against (TC3) is tautological since 
(TC3) equals (TA3).
Next, the lifting property against (TA2) says that the maps
$X(\Omega[T]) \to f^{\**} Y(\Omega[T])$
are Kan fibrations, and the DK condition says that these are Kan equivalences,
so that we conclude that such maps have the right lifting property against (TC2).
Lastly, given any lifting problem against a map in (TC1),
essential surjectivity and Remark \ref{SLIMOD REM}
produce a lifting problem against a map in (TA1) which has a solution, providing a solution to the original problem.
\end{proof}


{\color{red} To show that maps in (TA3) are normal cofibrations one can use a pushout of projective cofibrant cubes argument.}


\begin{lemma}
      \label{OMEGATTAME_LEM}
      For all $T \in \Omega_G$, $\Omega[T] \in \mathsf{PreOp}^G$ is tame cofibrant.
\end{lemma}
\begin{proof}
      From (TC3) with $n=0$, it suffices to show that $Sc[T]$ is tame cofibrant.
      By (TC1), $\partial\Omega[T]$ is tame cofibrant.
      Now, (TC2) with $n=0$ says that $\partial \Omega[C] \to \Omega[C]$ is a tame cofibration,
      and thus the result follows via the following pushout.
      \[
            \begin{tikzcd}
                  \displaystyle{
                    \coprod_{Gv \in \boldsymbol{V}_G(T)} \partial\Omega[T_{Gv}]
                  }
                  \arrow[d] \arrow[r]
                  &
                  \partial\Omega[T] \arrow[d]
                  \\
                  \displaystyle{
                    \coprod_{Gv \in \boldsymbol{V}_G(T)} \Omega[T_{Gv}]
                  }
                  \arrow[r]
                  &
                  Sc[T]
            \end{tikzcd}
            \]
\end{proof}


The following results are adapted from \cite{JT07} (see Proposition 7.15 therein). 


\begin{proposition}
	A cofibration $A \to B$ is a weak equivalence iff it has the left lifting property against all fibrations between fibrant objects.
\end{proposition}

\begin{proof}
	Let $B \xrightarrow{\sim} \tilde{B}$ be a fibrant replacement and
	let $A \xrightarrow{\sim} \tilde{A} \twoheadrightarrow \tilde{B}$
	be a factorization of the composite $A \to \tilde{B}$ 
	as a trivial cofibration followed by a fibration.
	One then has a lift in the diagram
\[
\begin{tikzcd}
	A \ar{r}{\sim} \ar[>->]{d} & \tilde{A} \ar[->>]{d}
\\
	B \ar{r}{\sim} \ar[dashed]{ru} & \tilde{B}
\end{tikzcd}
\]
where the top and bottom horizontal maps are weak equivalences. 
But then the 2-out-of-6 property for weak equivalences says that all maps are weak equivalences.
\end{proof}


\begin{corollary}\label{SIMPLQUILL COR}
An adjunction 
\[
F \colon \mathcal{C}
	\rightleftarrows
\mathcal{D} \colon G
\]
between model categories is a Quillen adjunction
provided that $F$ preserves cofibrations
and $G$ preserves fibrations between fibrant objects.
\end{corollary}


\begin{lemma}
	Let $A \to B$ be a tame cofibration in $\mathsf{PreOp}^G$, 
	$\mathcal{O} \in \mathsf{sOp}^G$ a $\Sigma$-cofibrant 
	$G$-operad,
	and consider a pushout diagram in $\mathsf{sOp}^G$ of the form
\[
\begin{tikzcd}
	\tau A \ar{r} \ar{d} & \mathcal{O} \ar{d}
\\
	\tau B \ar{r} & \mathcal{P}
\end{tikzcd}
\]
	Then $\mathcal{O} \to \mathcal{P}$ is a $\Sigma$-cofibration and 
\begin{equation}\label{UNITEQUIV EQ}
B \amalg_{A} N \mathcal{O}
	\to 
N \mathcal{P}
\end{equation}
is a weak equivalence.
\end{lemma}

\begin{proof}
	We first consider the case where $A\to B$ is in one of (TC1),(TC2),(TC3). 
	
	The (TC1) case is immediate, 
	since $\mathcal{O} \to \mathcal{O} \amalg G/H \cdot \Omega(\eta)$ is a $\Sigma$-cofibration and
	\eqref{UNITEQUIV EQ}
	is the isomorphism
	$N\mathcal{O} \amalg G/H\cdot \Omega[\eta] \simeq 
	N\left( \mathcal{O} \amalg G/H \cdot \Omega(\eta) \right)$.

	The (TC3) case is also straightforward:
	since $\tau A \to \tau B$ is an isomorphism, one can take 
	$\mathcal{O}=\mathcal{P}$, so that 
	\eqref{UNITEQUIV EQ} becomes a section of the map
	$N \mathcal{O} \to B \amalg_{A} N \mathcal{O}$, which is a trivial cofibration (it is a pushout of $A \to B$),
	and 2-out-of-3 hence implies that \eqref{UNITEQUIV EQ} is a weak equivalence.

	The most interesting case is then (TC2), 
	in which case it is well known that 
	$\mathcal{O} \to \mathcal{P}$ is a $\Sigma$-cofibration and
	each of the levels
$(B \amalg_{A} N \mathcal{O})_n
	\to 
(N \mathcal{P})_n$
for $n \geq 0$
is an equivalence in $\mathsf{dSet}^G$ by (an iteration of)
Proposition \ref{KEYPR PROP}, 
showing that \eqref{UNITEQUIV EQ} is in fact a dendroidal equivalence, and thus also a complete equivalence.

	We now turn to the case of $A \to B$ a general cofibration between cofibrant objects.
	As usual, $A \to B$ is a retract of a transfinite composition of pushouts of generating cofibrations.
	Since the conclusions of the result are invariant under retracts,
	we are free to assume that $A \to B$ is a transfinite composite
\[
A = A_0 \to A_1 \to A_2 \to \cdots \to A_{\beta} \to 
colim_{\beta < \kappa} A_{\beta} = B.
\]
where each map $A_{\beta} \to A_{\beta +1}$ is a pushout of a map in one of (TC1),(TC2),(TC3).

Defining $\mathcal{O}_{\beta}$ by replacing $A \to B$ with $A \to A_{\beta}$ in the pushout,
$\mathcal{O} \to \mathcal{P}$ becomes the transfinite composite of the maps $\mathcal{O}_{\beta} \to \mathcal{O}_{\beta + 1}$
and \eqref{UNITEQUIV EQ} becomes
$
colim_{\beta < \kappa} \left( 
N \mathcal{O} \amalg_{N \tau A} N \tau A_{\beta}
	\to 
N \mathcal{O}_{\beta}
\right)
$.
It thus suffices to show, by induction on $\beta < \kappa$, 
that the maps $\mathcal{O}_{\beta} \to \mathcal{O}_{\beta + 1}$ are $\Sigma$-cofibrations and that the maps 
$N \mathcal{O} \amalg_{N \tau A} N \tau A_{\beta}
	\to 
N \mathcal{O}_{\beta}$
are weak equivalences
(that this last condition suffices follows since
filtered colimits of weak equivalences in $\mathsf{PreOp}^G$ are weak equivalences ({\color{red} add this})).
Consider now the following diagrams.
\[
\begin{tikzcd}
	\tau A \ar{r} \ar{d} & \mathcal{O} \ar{d}
&&
	A_{\beta} \amalg_{A} N \mathcal{O}
	\ar[>->]{r} \ar{d}[swap]{\sim} &
	A_{\beta+1} \amalg_{A} N \mathcal{O}
	\ar{d}[swap]{\sim}
\\
	\tau A_{\beta} \ar{r} \ar{d} & \mathcal{O}_{\beta} \ar{d}
&&
	N \mathcal{O}_{\beta} \ar[>->]{r} &
	A_{\beta+1} \amalg_{A_{\beta}} N \mathcal{O}_{\beta} \ar{d}
\\
	\tau A_{\beta + 1} \ar{r} & \mathcal{O}_{\beta + 1}
&&
	&
	N \mathcal{O}_{\beta+1}
\end{tikzcd}
\]
The induction hypothesis states that
$\mathcal{O} \to \mathcal{O}_{\beta}$ is a $\Sigma$-cofibration and that the map
$A_{\beta} \amalg_A N \mathcal{O} \to \mathcal{O}_{\beta}$ is a weak equivalence.
Therefore, $\mathcal{O}_{\beta}$ is $\Sigma$-cofibrant 
and the both vertical maps marked $\sim$ in the rightmost diagram above are weak equivalences 
(this uses the fact that $\mathsf{PreOp}^G$ is left proper),
and thus the induction step will follow provided that the result holds for
the map $A_{\beta} \to A_{\beta + 1}$ and $\mathcal{O}_{\beta}$.
But $A_{\beta} \to A_{\beta + 1}$ is assumed to be a pushout of a map in (TC1),(TC2),(TC3), in which case the result is already known, and thus noting that the result is clearly invariant under pushouts finishes the proof.
\end{proof}

Setting $A = \emptyset $, $\mathcal{O}= \emptyset$ in the previous result yields the following.

\begin{corollary}\label{KEYEQUIV COR}
	If $B \in \mathsf{PreOp}^G$ is tame cofibrant, then 
	$B \to N \tau B$ is a weak equivalence.
\end{corollary}

\begin{proposition}\label{PREQUIEQUIV PROP}
The adjunction
\[
	\tau \colon \mathsf{PreOp}^G_{\text{tame}}
		\rightleftarrows 
	\mathsf{sOp}^G \colon N
\]
is a Quillen equivalence.
\end{proposition}


\begin{proof}
Firstly, note that $N$ preserves and detects weak equivalences.
Indeed, this follows since all objects in the image of $N$ are Segal operads, so that by Remark \ref{SEOPDK REM} a map in the image of $N$ is a weak equivalence iff it is a Dwyer-Kan equivalence.

Next, we show that this is a Quillen adjunction using Corollary \ref{SIMPLQUILL COR}.
The claim that $\tau$ preserves cofibrations follows since
$\tau$ sends the maps in (TC1) and (TC2) to generating cofibrations of $\mathsf{sOp}^G$ and the maps in (TC3) to isomorphisms.
For the claim that $N$ preserves fibrations between fibrant,
we use a somewhat indirect argument
(though we note that a direct argument is also possible,
by showing that fibrations between fibrant objects in $\mathsf{PreOp}^G$
also satisfy a ``local fibration plus isofibration'' description).
By Corollary \ref{KEYEQUIV COR} and 2-out-of-3, 
one has that for any trivial cofibration between cofibrant objects
$A \to B$, the map $N \tau A \to N \tau B$ is a weak equivalence, and thus so is $\tau A \to \tau B$.
This shows that $\tau$ sends all maps in (TA1),(TA2),(TA3)
to trivial cofibrations, and since these maps detect fibrations between fibrant objects in $\mathsf{PreOp}^G$, 
the standard adjunction argument shows that 
$N$ indeed preserves fibrations between fibrant objects.

For the Quillen equivalence claim, 
let $B \in \mathsf{PreOp}^G$ be tame cofibrant and
$\mathcal{O} \in \mathsf{sOp}^G$ be fibrant.
We must show that the leftmost map below is a weak equivalence iff its adjoint, which is the rightmost composite, is.
\[
	\tau B \to \mathcal{O},
\qquad
	B \xrightarrow{\sim} N \tau B \to N \mathcal{O}
\]
This is immediate from Corollary \ref{KEYEQUIV COR}
and the fact that $N$ preserves and detects weak equivalences.
\end{proof}


\begin{proposition}
	The adjunction 
$W_! \colon \mathsf{dSet}^G 
	\rightleftarrows 
\mathsf{sOp}^G \colon hcN$
	is a Quillen adjunction.
\end{proposition}

{\color{red} HERE}

\begin{proof}
	We again apply Corollary \ref{SIMPLQUILL COR}.
	For the claim that $W_!$ preserves cofibrations,
	it suffices to show this for the generating cofibrations
	$G\cdot_H \left( \partial \Omega[U] \to \Omega[U] \right)$ for $U \in \Omega^H$.
	But this follows since 
	$G \cdot_H \left(W_! \partial \Omega[U] \to W_! \Omega[U] \right)$
	is a pushout of the map
\[
	G \cdot_H \Omega[U - \boldsymbol{E}^{\mathsf{i}}(U)]
\otimes
	\left(
	\partial \left( \Delta[1]^{\times \boldsymbol{E}^{\mathsf{i}}(U) } \right) 
		\to
	\Delta[1]^{\times \boldsymbol{E}^{\mathsf{i}}(U) }
	\right).
\]
Similarly, the map
	$G \cdot_H \left(W_! \Lambda^E[U] \to W_! \Omega[U] \right)$
is a pushout 

%\[
%\begin{tikzcd}
%	G \cdot_H 
%	\Omega[U - \boldsymbol{E}^{\mathsf{i}}(U)] \otimes 
%	\partial \left( \Delta[1]^{\times \boldsymbol{E}^{\mathsf{i}}(U) } \right) 
%	\ar{r} \ar{d} &
%	G \cdot_H W_! \partial \Omega[U] \ar{d}
%\\
%	G \cdot_H 
%	\Omega[U - \boldsymbol{E}^{\mathsf{i}}(U)] \otimes 
%	\Delta[1]^{\times \boldsymbol{E}^{\mathsf{i}}(U) } \ar{r}	&
%	G \cdot_H W_! \Omega[U]
%\end{tikzcd}
%\]

	
\end{proof}





\newpage





\section{Nerves of free extensions are homotopy pushouts}



\subsection{The characteristic edge lemma}

\begin{notation}
Let $X \in \mathsf{dSet}$ be a dendroidal set and 
$x \colon \Omega[T] \to X$
a dendrex.

We will write $\langle x \rangle = x\left(  \Omega[T] \right)$
and refer to
$\langle x \rangle \subseteq X$
as the \emph{principal subpresheaf generated by $x$}.
\end{notation}


\begin{remark}
Note that
$\langle x \rangle = \langle \bar{x} \rangle$
iff $x,\bar{x}$ are both degeneracies of a common non-degenerate dendrex.
In particular, if the chosen representatives $x,\bar{x}$ are both nondegenerate,
there must exist an isomorphism
$\varphi \colon T \xrightarrow{\simeq} \bar{T}$
such that $x= \bar{x} \circ \varphi$.

{\color{red} bla bla something about how this is the same as choosing some nice equivalence classes of dendrices}
\end{remark}



\begin{definition}\label{CHAREDGE DEF}
Bla bla 

\begin{enumerate}
\item[(Ch0.1)] $b_i \colon \Omega[U_i] \to B$ is injective away from
$b_i^{-1}(A_{<i})$
\item[(Ch0.2)]
\[\left\{g \langle \partial^{\**}_E b_i \rangle | g \in G,E \subseteq \Xi^i \right\}
 =
\left\{\langle \partial^{\**}_E b_{g i} \rangle | g \in G,E \subseteq \Xi^{gi} \right\}\]
{\color{blue} this implies 
$g \langle b_i \rangle = \langle b_{gi} \rangle$ since these are the largest elements with respect to inclusion}
\item[(Ch1)] if $V \hookrightarrow U_i$ is outer and $\Xi^i_V = \emptyset$
then $\langle \partial_V^{\**} b_i\rangle \subseteq A$
\item[(Ch2)] if $V \hookrightarrow U_i$ and
$\langle \partial_{V-\Xi^i_V}^{\**} b_i\rangle \subseteq A$
then
$\langle \partial_V^{\**} b_i\rangle \subseteq A_{<i}$
\item[(Ch3)] if $j \not \geq i$, $V \hookrightarrow U_i$ and
$\langle \partial_{V-\Xi^i_V}^{\**} b_i\rangle \subseteq \langle b_j \rangle$
then
$\langle \partial_V^{\**} b_i\rangle \subseteq A_{<i}$
\end{enumerate}
\end{definition}



\begin{lemma}\label{CHAREDGE LEM}
bla bla if conditions in Definition \ref{CHAREDGE DEF} are met then
$A \to B$ is inner anodyne.
\end{lemma}


\subsection{Refactoring}



The following is an equivariant analogue of \cite[Prop. 3.2]{CM13b}

\begin{proposition}\label{KEYPR PROP}
Fix a $G$-set $\mathfrak{C}$ of colors,
and suppose that 
$\mathcal{O} \in \mathsf{Op}^{G,\mathfrak{C}}$
and
$B \in \mathsf{Sym}^{G,\mathfrak{C}}$
are
$\Sigma$-cofibrant.
Then, writing
\begin{equation}\label{PCOPROD EQ}
\mathcal{P} = \mathcal{O} \amalg_{\mathsf{F}} \mathbb{F} B
\end{equation}
for the fixed color coproduct, 
one has that the induced map
\begin{equation}\label{ANODYNEMAP EQ}
N \mathcal{O} \amalg_{\mathfrak{C}} B \to N \mathcal{P}
\end{equation}
is $G$-inner anodyne.
%
%Suppose that $\mathcal{O} \in \mathsf{Op}^{G,\mathfrak{C}}$
%is $\Sigma$-cofibrant.
%Further, let $C \in \Sigma_G$ be any $G$-corolla and consider 
%a pushout in $\mathsf{Op}^{G}$ of the form
%\begin{equation}\label{PUSHOUTPROP EQ}
%\begin{tikzcd}
%	\partial \Omega(C) \ar{r} \ar{d} & \mathcal{O} \ar{d}
%\\
%	\Omega(C) \ar{r} & \mathcal{P}.
%\end{tikzcd}
%\end{equation}
\end{proposition}


The proof of this result will follow from an instance of 
({\color{blue} a rephrased version of})
\cite[Lemma 3.4]{BP_edss}.
Before proving the result, we need to understand $\mathcal{P}$ itself.

Noting that the coproduct \eqref{PCOPROD EQ}
is a particular case of \eqref{OU EQ} with $X=\emptyset$,
Proposition \ref{FILTPUSH PROP}
(note that the condition $X=\emptyset$ implies it is also 
$Q^{in}_T[u] = \emptyset$ in that result)
then implies that  
\begin{equation}\label{PUSHOPPR EQ}
	\mathcal{P}(C) = 
	\coprod_{
	[T] \in \mathsf{Iso}
	\left( \Omega_{\mathfrak{C}}^a \downarrow C \right)
	}
	\left(
		\prod_{v \in V^{ac}(T)} \mathcal{O}(T_v)
	\times
		\prod_{v \in V^{in}(T)} B(T_v)
	\right)
	\cdot_{\mathsf{Aut}_{\Omega^a_{\mathfrak{C}}}(T)} \mathsf{Aut}_{\Sigma_{\mathfrak{C}}}(C)
\end{equation}



The decomposition \eqref{PUSHOPPR EQ}
will be the key to verifying the 
characteristic edge conditions in Definition \ref{CHAREDGE DEF}.
To do so, we will first find it useful to discuss a number of special types of principal subpresheaves of $N \mathcal{P}$, suggested by \eqref{PUSHOPPR EQ}.




\begin{definition}
Let $\langle p \rangle \subseteq N \mathcal{P}$ be 
a principal subpresheaf. 
We say $\langle p \rangle$ is:
\begin{itemize}
\item \emph{unital} if there is a representative
$p\colon \Omega[T] \to N \mathcal{P}$ with $T=\eta$ the stick tree;
\item \emph{reduced} if there is a representative
$p\colon \Omega[T] \to N \mathcal{P}$ with $T \in \Sigma$, i.e. with $T$ a corolla.
%\item \emph{alternating} if $T \in \Omega^a$ is an alternating tree
%and for each active (resp. inert) vertex 
%$T_v \hookrightarrow T$ it is
%$\langle \partial_{T_v}^{\**} p \rangle \subseteq \O$
%(resp. inert $\langle \partial_{T_v}^{\**} p \rangle \subseteq B$);
\end{itemize}
Moreover, recalling that any tree $T$
has an associated corolla $\mathsf{lr}(T)$, 
we abbreviate
$\partial_r p = \partial_{\mathsf{lr}(T)}p$
and call
$\langle \partial_r p \rangle$
the \emph{reduction} of  
$\langle p \rangle$.
\end{definition}



\begin{definition}
Let $\langle p \rangle \subseteq N \mathcal{P}$ be 
a principal subpresheaf. 
We say $\langle p \rangle$ is:
\begin{itemize}
\item \emph{elementary} 
if there is a representative
$p\colon \Omega[T] \to N \mathcal{P}$ such that for each vertex $T_v \hookrightarrow T$,
it is $\langle \partial_{T_v} p\rangle \subseteq \O \amalg_{\mathfrak{C}}B$;
%\item \emph{alternating} if $T \in \Omega^a$ is an alternating tree
%and for each active (resp. inert) vertex 
%$T_v \hookrightarrow T$ it is
%$\langle \partial_{T_v}^{\**} p \rangle \subseteq \O$
%(resp. inert $\langle \partial_{T_v}^{\**} p \rangle \subseteq B$);
\item \emph{canonical} if it is elementary and for any elementary
$\langle q \rangle \subseteq \langle p \rangle$
such that 
$\langle \partial_r q \rangle = \langle \partial_r p \rangle$
it is
$\langle q \rangle = \langle p \rangle$.
%
%
% for any proper inner face
%$T-E \hookrightarrow T, 
%\emptyset \neq E \subseteq \boldsymbol{E}^{\mathsf{i}}(T)$
%one has that $\langle \partial^{\**}_{T-E} p \rangle$ is \emph{not} elementary.
\end{itemize}
Moreover, we say a dendrex $p \colon \Omega[T] \to N \mathcal{P}$
is \emph{alternating} if $T \in \Omega^a$ is an alternating tree
and for each active (resp. inert) vertex 
$T_v \hookrightarrow T$ it is
$\langle \partial_{T_v}^{\**} p \rangle \subseteq \O$
(resp. $\langle \partial_{T_v}^{\**} p \rangle \subseteq B$).
\end{definition}

{\color{green} HERE!}


%\begin{remark}\label{NONDEGNEED REM}
%The ``non-degenerate'' condition in the definition of 
%``canonical'' is almost (but not quite) redundant, since degeneracies in $\Omega$ usually admit sections that are proper inner face maps \cite[Cor. 5.38]{Per_eds},
%with the exceptions being the maps $[n] \to [0]$.
%\end{remark}

\begin{remark}
{\color{blue} the definition of elementary does not care if the representative $p$ is non-degenerate or not} 
\end{remark}


\begin{remark}\label{NONDEGNEED REM}
A degenerate dendrex $p \colon \Omega[T] \to N \mathcal{P}$
will have an inner face which is non-degenerate unless the dendrex is completely degenerate, i.e. it is a composite
$p \colon [n] \to [0] \to N \mathcal{P}$.
\end{remark}



\begin{remark}\label{PUSHOPPRRST REM}
\eqref{PUSHOPPR EQ} 
can be restated as saying that for each reduced principal subpresheaf
$\langle p \rangle \subseteq N \mathcal{P}$
there exists an alternating dendrex $p^a$, \emph{unique up to isomorphism}, 
such that
$\langle \partial_r p^a \rangle = \langle p \rangle$.

More generally, if $\langle p \rangle$
is not reduced, we similarly write
$\langle p^a \rangle = \langle \left(\partial_r p\right)^a \rangle$.
\end{remark}


\begin{remark}\label{ELEMLABEL REM}
If $\langle p \rangle \subseteq N \mathcal{P}$
is elementary, the tree $T$ can be naturally regarded as a
$\{\O,B\}$-labeled tree by labeling $T_v$ according to whether
it is $\langle \partial_{T_v}^{\**} p\rangle \subseteq \O $ or
$\langle \partial_{T_v}^{\**} p\rangle \subseteq B$.

By ({\color{blue} the alternating tree analog of})
\cite[Prop. 5.49]{BP_geo}, there is hence a unique alternating tree $T^a$ together with a tall planar $B$-inert label map $T^a \to T$,
and it then follows that
$\langle \partial_{T^a} p \rangle = \langle p^a \rangle$.
\end{remark}



\begin{lemma}\label{CANCHAR LEM}
A principal subpresheaf $\langle p \rangle$
is canonical iff
$\langle p \rangle = \langle p^a \rangle$
for the alternating dendrex $p^a$ as in Remark \ref{PUSHOPPRRST REM}.
\end{lemma}


\begin{proof}
Consider first the ``only if'' direction. 
Let $p \colon \Omega[T] \to N \mathcal{P}$
be a non-degenerate representative of $\langle p \rangle$ and
$T^a \to T$ as in Remark \ref{ELEMLABEL REM}.
%That $p$ is not degenerate follows since any degeneracy in 
%$\Omega_{\mathfrak{C}}$ has a section that is an inner face 
%(this follows from \cite[Cor. 5.38]{Per_eds}).
%
Writing $T^a \to T' \to T$ for the 
``degeneracy followed by inner face'' factorization, 
$T'$ inherits a $\{\O,B\}$-labeling from $T^a$ and both maps in the factorization are tall planar $B$-inert label maps, 
so that $\langle \partial_{T'} p \rangle$ is again elementary.
But since $\langle p \rangle$ is canonical by assumption, 
it must then be the case that $T^a \to T$ is a degeneracy map,
i.e. that $p^a$
is a degeneracy of $p$.


{\color{red} HERE!}


For the ``if'' direction, we again let $p \colon \Omega[T] \to N \mathcal{P}$
be a non-degenerate representative. 
Since the assumption $\langle p \rangle = \langle p^a \rangle$ guarantees 
that $\langle p \rangle$ is elementary,
one can choose an inner face
$T-E \hookrightarrow T$
such that 
$\langle \partial_{T-E} p \rangle$
is elementary but
$\langle \partial_{T-E'} p \rangle$
is not elementary
for any smaller inner face
$T-E' \hookrightarrow T-E$.


, we consider two cases.
If $\langle p \rangle$
comes from a composite $p\colon [n] \to [0] \to N \mathcal{P}$, 
then $\langle p \rangle$ is non-degenerate iff $n=0$ and the desired statement is trivial.
Otherwise, by Remark \ref{NONDEGNEED REM}
one can necessarily find some inner face 
$T-E$ such that
$\langle \partial_{T-E} p \rangle$
is elementary.
But then $\langle \partial_{T-E} p \rangle$ is the unique principal subpresheaf on which $\langle p^a \rangle$
is degenerate, and the result follows.
\end{proof}


Combining Lemma \ref{CANCHAR LEM} with 
Remark \ref{PUSHOPPRRST REM}
yields the following.


\begin{corollary}
For each reduced principal subpresheaf 
$\langle p \rangle$
there exists a unique 
elementary canonical principal subpresheaf 
$\langle p^c \rangle$
such that
$\langle \partial_r p^c \rangle = \langle p \rangle$.

{\color{red} degeneracies generate the same presheaf as the thing they are degenerate on...}


\end{corollary}




{\color{red} HERE}

\begin{remark}
The inclusion
$\langle \partial_r p \rangle \subseteq \langle p \rangle$ can almost always be realized by a inner face map unless $\partial_r p$ is a unit.
\end{remark}




{\color{red} degeneracies generate the same presheaf as the thing they are degenerate on...}



\begin{lemma}
For each principal subpresheaf 
$\langle p \rangle \subseteq N \mathcal{P}$
there exists a smallest elementary principal subpresheaf
$\langle p^e \rangle \subseteq N \mathcal{P}$
such that
$\langle p \rangle \subseteq \langle p^e \rangle$.

In addition, choosing non-degenerate representatives
$p \colon \Omega[T] \to N \mathcal{P}$,
$p^e \colon \Omega[T^e] \to N \mathcal{P}$,
there is a face map
$T \hookrightarrow T^e$, unique up to isomorphism, such that
$\langle p \rangle = \langle \partial_{T} p^e \rangle$.
Lastly, this face map is an inner face of the form
$T \simeq T^e - E$ for some $E \subseteq \Xi^{p^e}$.
\end{lemma}



\begin{remark}
{\color{red} reduced subpresheaf can't correspond to a unit operation.}
\end{remark}



\subsection{Previous discussion (to be refactored)}


Given a set $\mathfrak{C}$ of colors,
write $\Sigma_{\mathfrak{C}}$ for the groupoid of corollas with edges labeled by colors in $\mathfrak{C}$.

If, in addition, $\mathfrak{C}$ is a $G$-set, 
we write $G \ltimes \Sigma_{\mathfrak{C}}^{op}$ for the larger groupoid obtained via the associated Grothendieck construction.


% SOMETHING?

Writing
$\mathsf{Sym}^{G,\mathfrak{C}} = 
\mathsf{Set}^{G \ltimes \Sigma_{\mathfrak{C}}^{op}}$,
we have a natural monad $\mathbb{F}$ on
$\mathsf{Sym}^{G,\mathfrak{C}}$
whose algebra category we denote by 
$\mathsf{Op}^{G,\mathfrak{C}}$.


\begin{notation}
	Given a $G$-equivariant function 
	$f \colon \mathfrak{C} \to \mathfrak{D}$
	we write
\[
	f_{\**} \colon 
	\mathsf{Sym}^{G,\mathfrak{C}}
	\rightleftarrows
	\mathsf{Sym}^{G,\mathfrak{D}}
	\colon f^{\**}
\qquad
	\check{f}_{\**} \colon 
	\mathsf{Op}^{G,\mathfrak{C}}
	\rightleftarrows
	\mathsf{Op}^{G,\mathfrak{D}}
	\colon f^{\**}
\]
for the standard adjunctions (note the need to distinguish notations for the left adjoints).
\end{notation}


\begin{example}\label{GCORMPA EX}
Given a $G$-corolla $C \in \Sigma_G$, we write $\partial C$ for the set of edges of $C$, which is naturally identified with the set of objects of the associated $G$-operad
$\Omega(C) \in \mathsf{Op}^G$.

One can then regard $\Omega(C) \in \mathsf{Op}^{G,\partial C}$ and, moreover, $\Omega(C)$ is in fact the free operad over the symmetric sequence obtained by removing the units of $\Omega(C)$,
which we denote by
$\Omega'(C) \in \mathsf{Sym}^{G,\partial C}$.

Given the non-equivariant decomposition
$C = C_1 \amalg \cdots \amalg C_k$
with $C_i \in \Sigma$, 
one can naturally regard the $C_i$ as objects of $\Sigma_{\partial C}$.
In fact, one then has an identification
\begin{equation}\label{SOMEIDEN EQ}
	\Omega'(C) \simeq 
	\Sigma_{\partial C}[C_1] \amalg \cdots \amalg \Sigma_{\partial C}[C_k]
\end{equation}
where $\Sigma_{\partial C}[-]$ denotes the representable presheaf in 
$\mathsf{Set}^{\Sigma_{\partial C}^{op}}$.
This claim requires some justification, since a priori the right hand side of \eqref{SOMEIDEN EQ} is an object in $\mathsf{Set}^{\Sigma_{\partial C}^{op}}$,
rather than in $\mathsf{Set}^{G \ltimes \Sigma_{\partial C}^{op}}$,
i.e. we need to describe the action of the additional action arrows
$D \xrightarrow{g} gD$ on this presheaf.
This action is given by the following diagram, where the vertical $g$ arrows simply act on labels, and all horizontal arrows are shuffle arrows (i.e. arrows in $\Sigma_{\partial C}$).
The diagonal $C_g$ arrow corresponds to the structural $G$-action on $C$. It is then straightforward to check that there is a unique dashed shuffle $\tau_g$ as indicated
\[
\begin{tikzcd}
	D \ar{d}[swap]{g} \ar{r}{\sigma} & C_i \ar{rd}{C_g} \ar{d}[swap]{g}
\\
	g D \ar{r}[swap]{g \sigma} & g C_i \ar[dashed]{r}{\simeq}[swap]{\tau_g} & C_{g i}
\end{tikzcd}
\]
and one defines $g_{\**}\colon \Sigma_{\partial C}[C_i] \to \Sigma_{\partial C}[C_{gi}]$
via $\sigma \mapsto \tau_g \circ (g \sigma)$.

Moreover, letting $f \colon \partial C \to \mathfrak{C}$
be a map of colors, 
one obtains $\mathfrak{C}$-corollas $C_i^{f} \in \Sigma_{\mathfrak{C}}$
by coloring each edge $e\in \partial C$ by $f(e) \in \mathfrak{C}$, resulting in a generalized identification 
\begin{equation}\label{SOMEIDENGEN EQ}
	f_{\**} \Omega'(C) \simeq 
	\Sigma_{\mathfrak{C}}[C_1^f] \amalg \cdots \amalg \Sigma_{\mathfrak{C}}[C^f_k]
\end{equation}
Indeed, \eqref{SOMEIDENGEN EQ} follows from the observation
that the Kan extension 
$\mathsf{Lan}_{G \ltimes \Sigma_{\partial C} \to G \ltimes \Sigma_{\mathfrak{C}}}$ 
coincides, after forgetting with the $G$-action arrows,
with the Kan extension
$\mathsf{Lan}_{\Sigma_{\partial C} \to \Sigma_{\mathfrak{C}}}$
(cf. Lemma \ref{REDUCELAN LEM}).
\end{example}


\begin{definition}
      \label{COR_GGRAPH_DEF}
	Given a $\mathfrak{C}$-corolla $C$, 
	a subgroup 
	$\Gamma \leq \mathsf{Aut}_{G \ltimes \Sigma_{\mathfrak{C}}^{op}}(C)$
	is called a \textit{$G$-graph subgroup} if
	$\Gamma \cap \mathsf{Aut}_{\Sigma_{\mathfrak{C}}^{op}}(C) = \**$.
	
	We write $\mathcal{F}^{\Gamma} = \{\mathcal{F}^{\Gamma}_C\}$
	for the collection of families of $G$-graph subgroups.
	
	A $G$-$\mathfrak{C}$-symmetric sequence
	$X \in \mathsf{Sym}^{G,\mathfrak{C}}$
	is called $\Sigma$-cofibrant if each level
	$X(C)$ is $\mathcal{F}^{\Gamma}_C$-cofibrant.
\end{definition}


\begin{remark}
	Write a $\mathfrak{C}$-corolla as $C^f \in \Sigma_{\mathfrak{C}}$,
	where $C \in \Sigma$ is the underlying corolla and
	$f\colon \partial C \to \mathfrak{C}$
	is the coloring.
	A $G$-graph subgroup 
	$\Gamma \leq \mathsf{Aut}_{G \ltimes \Sigma_{\mathfrak{C}}^{op}}(C^f)$ is, under the map 
	$G \ltimes \Sigma_{\mathfrak{C}}^{op} \to
	G \times \Sigma^{op}$,
	identified with a 
	$G$-graph subgroup of 
	$G \times \mathsf{Aut}_{\Sigma^{op}}(C)$,
	i.e., with the graph of a partial antihomomorphism
\[
	G^{op} \geq H^{op} \xrightarrow{(-)^{-1}} H 
	\xrightarrow{\tau_{(-)}} \mathsf{Aut}_{\Sigma^{op}}(C)
\]
	which is subject to the requirement
\[
	f(\tau_h(e)) = h f(e).
\]
Using the $\tau$ automorphisms one can then
\begin{inparaenum}
\item[(i)] equip $C$ with a $H$-action,
so that one can regard $C \in \Sigma^H \subseteq \Sigma_H$;
\item[(ii)] extend $\Sigma_{\mathfrak{C}}[C^f]$ to 
an object in $\mathsf{Set}^{H \ltimes \Sigma_{\mathfrak{C}}}$
by defining the action of the $H$-action arrows $h$ via 
$\sigma \mapsto \tau_{h} \circ (h \sigma)$;
\item[(iii)]
following Example \ref{GCORMPA EX}, one thus has an identification
\[
	f_{\**} \Omega'(C) \simeq \Sigma_{\mathfrak{C}}[C^f]
\]
of objects in $\mathsf{Set}^{H \ltimes \Sigma_{\mathfrak{C}}}$ and therefore an identification 
\[
	f_{\**} \left( G \cdot_H \Omega'(C) \right) \simeq 
	G\cdot_H \Sigma_{\mathfrak{C}}[C^f]
\]
of objects in $\mathsf{Set}^{G \ltimes \Sigma_{\mathfrak{C}}}$.
\end{inparaenum}
\end{remark}


\begin{example}
Let $G = \mathbb{Z}_{/2} = \{\pm 1\}$ and 
$\mathfrak{C} = \{\mathfrak{a}, -\mathfrak{a}, \mathfrak{b}\}$ where we implicitly have
$-\mathfrak{b} = \mathfrak{b}$.
Consider the two $\mathfrak{C}$-corollas 
$C,D \in \Sigma_{\mathfrak{C}}$ below.
\begin{equation}
	\begin{tikzpicture}[auto,grow=up, level distance = 2.2em,
	every node/.style={font=\scriptsize,inner sep = 2pt}]%
		\tikzstyle{level 2}=[sibling distance=3em]%
			\node at (0,0) [font = \normalsize] {$C$}%	
				child{node [dummy] {}%
					child{node {}%
					edge from parent node [swap] {$-\mathfrak{a}$}}%
					child[level distance = 2.9em]{node {}%
					edge from parent node [swap,	near end] {$\mathfrak{b}$}}%
					child[level distance = 2.9em]{node {}%
					edge from parent node [near end] {$\mathfrak{b}$}}%
					child{node {}%
					edge from parent node  {$\mathfrak{a}$}}%
				edge from parent node [swap] {$\mathfrak{b}$}};%
			\node at (7,0) [font = \normalsize] {$D$}%	
				child{node [dummy] {}%
					child{node {}%
					edge from parent node [swap] {$-\mathfrak{a}$}}%
					child[level distance = 2.9em]{node {}%
					edge from parent node [swap,	near end] {$-\mathfrak{a}$}}%
					child[level distance = 2.9em]{node {}%
					edge from parent node [near end] {$\mathfrak{a}$}}%
					child{node {}%
					edge from parent node  {$\mathfrak{a}$}}%
				edge from parent node [swap] {$\mathfrak{b}$}};%
	\end{tikzpicture}%
\end{equation}%
Each of $C,D$ admit exactly two non-trivial $G$-graph subgroups,
which are encoded by the $\mathbb{Z}_{/2}$-actions on the underlying corollas depicted below.
\begin{equation}
	\begin{tikzpicture}[auto,grow=up, level distance = 2.2em,
	every node/.style={font=\scriptsize,inner sep = 2pt}]%
		\tikzstyle{level 2}=[sibling distance=3em]%
			\node at (-1.6,0) [font = \normalsize] {$C_1$}%	
				child{node [dummy] {}%
					child{node {}%
					edge from parent node [swap] {$-a$}}%
					child[level distance = 2.9em]{node {}%
					edge from parent node [swap,	near end] {$c\phantom{b}$}}%
					child[level distance = 2.9em]{node {}%
					edge from parent node [near end] {$b$}}%
					child{node {}%
					edge from parent node  {$a$}}%
				edge from parent node [swap] {$r$}};%
			\node at (1.6,0) [font = \normalsize] {$C_2$}%	
				child{node [dummy] {}%
					child{node {}%
					edge from parent node [swap] {$-a$}}%
					child[level distance = 2.9em]{node {}%
					edge from parent node [swap,	near end] {$-b$}}%
					child[level distance = 2.9em]{node {}%
					edge from parent node [near end] {$b$}}%
					child{node {}%
					edge from parent node  {$a$}}%
				edge from parent node [swap] {$r$}};%
			\node at (5.4,0) [font = \normalsize] {$D_1$}%	
				child{node [dummy] {}%
					child{node {}%
					edge from parent node [swap] {$-a$}}%
					child[level distance = 2.9em]{node {}%
					edge from parent node [swap,	near end] {$-b$}}%
					child[level distance = 2.9em]{node {}%
					edge from parent node [near end] {$b$}}%
					child{node {}%
					edge from parent node  {$a$}}%
				edge from parent node [swap] {$r$}};%
			\node at (8.6,0) [font = \normalsize] {$D_2$}%	
				child{node [dummy] {}%
					child{node {}%
					edge from parent node [swap] {$-b$}}%
					child[level distance = 2.9em]{node {}%
					edge from parent node [swap,	near end] {$-a$}}%
					child[level distance = 2.9em]{node {}%
					edge from parent node [near end] {$b$}}%
					child{node {}%
					edge from parent node  {$a$}}%
				edge from parent node [swap] {$r$}};%
	\end{tikzpicture}%
\end{equation}%

\end{example}




The following is the analogue of \cite[Prop. 3.2]{CM13b}

\begin{proposition}\label{KEYPR PROP}
Suppose that $\mathcal{O} \in \mathsf{Op}^{G,\mathfrak{C}}$
is $\Sigma$-cofibrant.
Further, let $C \in \Sigma_G$ be any $G$-corolla and consider 
a pushout in $\mathsf{Op}^{G}$ of the form
\begin{equation}\label{PUSHOUTPROP EQ}
\begin{tikzcd}
	\partial \Omega(C) \ar{r} \ar{d} & \mathcal{O} \ar{d}
\\
	\Omega(C) \ar{r} & \mathcal{P}.
\end{tikzcd}
\end{equation}
Then the induced map
\begin{equation}\label{ANODYNE MAP}
	\Omega[C] \amalg_{\partial \Omega[C]} N\mathcal{O} \to N\mathcal{P}
\end{equation}
is $G$-inner anodyne.
\end{proposition}

\begin{proof}
The desired claim that \eqref{ANODYNE MAP}
is $G$-inner anodyne will follow by applying the  
\textit{characteristic edge lemma} \cite[Lemma 3.4]{BP_edss},
but we first need some preliminary discussion. 

Let us write $f \colon \partial C \to \mathfrak{C}$
for the induced map of colors.
The first step is to rewrite \eqref{PUSHOUTPROP EQ} as a pushout diagram in $\mathsf{Op}^{G,\mathfrak{C}}$, which can be done by applying $\check{f}_{\**}$
to the leftmost objects in \eqref{PUSHOUTPROP EQ}.
Since
\[
	\check{f}_{\**} \Omega(C) \simeq 
	\check{f}_{\**} \left( \mathbb{F} \Omega'(C) \right) \simeq 
	\mathbb{F} \left(f_{\**}  \Omega'(C) \right)
\]
one has that, writing $C \simeq G \cdot_H C_{\star}$ one has the alternative pushout in $\mathsf{Op}^{G,\mathfrak{C}}$
\begin{equation}
\begin{tikzcd}
	\mathbb{F} ( \emptyset ) \ar{r} \ar{d} & \mathcal{O} \ar{d}
\\
	\mathbb{F} \left( 
	G \cdot_H \Sigma_{\mathfrak{C}}[C^f_{\star}] \right) \ar{r} & \mathcal{P}.
\end{tikzcd}
\end{equation}
Writing $B = G \cdot_H \Sigma_{\mathfrak{C}}[C^f_{\star}]$, one then has
\begin{equation}\label{PUSHOPPR EQ}
	\mathcal{P}(C) = 
	\coprod_{
	[T] \in \mathsf{Iso}
	\left( \Omega_{\mathfrak{C}}^a \downarrow C \right)
	}
	\left(
		\prod_{v \in V^{ac}(T)} \mathcal{O}(T_v)
	\times
		\prod_{v \in V^{in}(T)} B(T_v)
	\right)
	\cdot_{\mathsf{Aut}_{\Omega^a_{\mathfrak{C}}}(T)} \mathsf{Aut}_{\Sigma_{\mathfrak{C}}}(C)
\end{equation}

We now discuss the dendrices of $N \mathcal{P}$. Firstly, recall that, by the strict Segal condition characterization of nerves \cite[Cor. 2.7]{CM13a},
a dendrex $\Omega[T] \to N \mathcal{P}$
is uniquely specified by the tree $T \in \Omega$ together with a choice of operations
$\{p_v \in \mathcal{P}(T_v)\}_{v \in \boldsymbol{V}(T)}$.
Noting that \eqref{PUSHOPPR EQ} implies that the canonical map (of sequences) 
$\mathcal{O} \amalg B \to \mathcal{P}$
is a monomorphism, 
we will say a dendrex $(T,\{p_v\})$ is \textit{elementary}
if all operations $p_v$ are in $\mathcal{O} \amalg B$.
Additionally, an elementary dendrex $(T,\{p_v\})$ is called \textit{alternating} if $T$ is an alternating tree and 
$p_v$ is in $\mathcal{O}$ (resp. $B$) if
$v \in \boldsymbol{V}(T)$ is active (resp. inert).

Given an elementary dendrex $(T,\{p_v\})$ and a map of trees 
$\varphi \colon S \to T$
we will need to know when 
$\varphi^{\**}(T,\{p_v\})$
is again elementary.
Since all maps in $\Omega$ are, uniquely up to isomorphism,
factored as a degeneracy followed by an inner face followed by an outer face, it suffices to discuss each of those cases.
It is straightforward to check that 
$\varphi^{\**}(T,\{p_v\})$ 
is elementary whenever $\varphi$ is a degeneracy or an outer face.
For an inner face
$\varphi \colon T-D \to T$,
noting that a partial composite $p \circ_i q$ of non-unit operations is in $\mathcal{O} \amalg B$ iff both operations are in $\mathcal{O}$,
one sees that $\varphi^{\**}(T,\{p_v\})$ is elementary iff
for each $d \in D$ the adjacent vertices of $T$ are either both labeled by operations of $\mathcal{P}$ or one of them is labeled by an identity.
An elementary dendrex is called \textit{reduced} if it has no such edges. 
In other words, an elementary dendrex is reduced iff none of its inner faces are reduced, so that, in particular, all elementary dendrices admit at least one reduced inner face (note that specifying such a face is somewhat subtle: even when a dendrex is non-degenerate, it may not be enough to collapse edges connecting $\mathcal{O}$ vertices, since this may possibly introduce new identity vertices, resulting in a degenerate vertex).
Note that reduced dendrices are necessarily non-degenerate, but not vice versa.
In fact, a non-degenerate dendrex is reduced iff it has a degeneracy which is an alternating dendrex. 


In what follows, we will find it convenient to work with elementary dendrices that have been suitably ``planarized''.
To do so, fix a subset
$\mathcal{O}^{\mathsf{st}} \amalg B^{\mathsf{st}} 
\subset
\coprod_{C \in \Sigma_{\mathfrak{C}}}
\mathcal{O}(C) \amalg B(C)$
of coset representatives for the $\Sigma$-action,
which we call \textit{standard} representatives.
An elementary dendrex $(T,\{p_v\})$ is then called \textit{standard}
if all $p_v$ are standard (i.e. in 
$\mathcal{O}^{\mathsf{st}} \amalg B^{\mathsf{st}}$).
Moreover, since both $\mathcal{O}$ and $B$ are $\Sigma$-cofibrant/$\Sigma$-free sequences (the former by assumption),
for each elementary simplex $(T,\{p_v\})$,
there is a unique (replanarization) isomorphism
$\varphi \colon T' \to T$ such that
$(T',\{\varphi^{\**}p_v\})$ is standard,
and we write 
$\mathsf{st}(T,\{p_v\}) = \varphi^{\**} (T,\{p_v\}) = (T',\{\varphi^{\**}_v p_v\})$
to denote this.


Note that it now follows from \eqref{PUSHOPPR EQ} that,
for each operation $p \in \mathcal{P}(C)$,
there exists a unique standard alternating dendrex
$b'_p \colon \Omega[T'_p] \to N \mathcal{P}$ and isomorphism 
$C \simeq T'_p - \boldsymbol{E}^{\mathsf{i}}(T'_p)$
such that the composite
\begin{equation}\label{STANDELDE EQ}
	\Omega[C] \simeq
	\Omega[T'_p - \boldsymbol{E}^{\mathsf{i}}(T'_p)] \to 
	\Omega[T'_p] \xrightarrow{b'_p}
	N \mathcal{P}
\qquad
	\Omega[C] \simeq
	\Omega[T_p - \boldsymbol{E}^{\mathsf{i}}(T_p)] \to 
	\Omega[T_p] \xrightarrow{b_p}
	N \mathcal{P}
\end{equation}
is $p$. In fact, due to the correspondence between alternating elementary dendrices and reduced elementary dendrices, 
the analogous claim  also holds 
for the corresponding non-degenerate dendrex 
$b_p \colon \Omega[T_p] \to N \mathcal{P}$.


We can now finally discuss how to apply \cite[Lemma 3.4]{BP_edss}.

Firstly, we need to identify a $G$-poset $I$ and dendrices 
$b_i \colon \Omega[U_i] \to N \mathcal{P}$ for $i \in I$.
Firstly, the underlying set of $I$ is the set of 
non-degenerate standard dendrices of $\mathcal{P}$,
which we abbreviate as
$i = (U_i,\{p_v^i\})$.
The dendrex $b_i \colon \Omega[U_i] \to N \mathcal{P}$ is then tautological, being $i$ itself, but it will preferable to use distinct notations for $i \in I$ and
$b_i \in N \mathcal{P} (U_i)$.
Given $i,j \in I$, we write $i \leq j$ if exists a (in general not  planar) face map
$\varphi \colon U_i \to U_j$
such that $b_i = \varphi^{\**}(b_j)$.
Note that by the uniqueness of standardizations $\varphi$ can only be an isomorphism if $i=j$, showing that $\leq$ indeed satisfies anti-symmetry.
Lastly, we define the $G$-action on $I$ via
\[b_{g i} = \mathsf{st} (g b_i).\]
When reading this formula, note that
$g b_i \in \mathcal{P}(U_i)$
(since this uses the $G$-action
on $\mathcal{P}$),
while $b_{g i} \in \mathcal{P}(U_{g i})$,
where $U_{g i}$ comes with the unique isomorphism
$U_{g i} \xrightarrow{g^{-1}} U_i$ which standardizes $b g_i$.

Lastly, the characteristic edge sets 
$\Xi^i \subseteq \boldsymbol{E}^{\mathsf{i}}(U_i)$ consist of those inner edges such that at least one of the adjacent vertices is mapped to an operation in $B$. Note that, by the discussion above, for an inner face of $\varphi \colon U_i - D \to U_i$
the dendrex $\varphi^{\**}(b_i)$ is elementary iff
$D \cap \Xi^i = \emptyset$.

We now note that it is in fact 
$N \mathcal{P} = 
A \cup \bigcup_{i \in I} b_i\left(\Omega[U_i]\right)$.
Indeed, given an arbitrary (non-elementary) non-degenerate dendrex
$(S,\{p_v\}_{v \in \boldsymbol{V}(T)})$, 
the trees $T_{p_v}$ from \eqref{STANDELDE EQ}
can be regarded as a $S$-substitution datum, which after assembled
yields a non-degenerate standard dendrex 
$\Omega[T] \xrightarrow{b_{\{p_v\}}} N \mathcal{P}$ 
whose image contains $(S,\{p_v\}_{v})$.

We now check the characteristic conditions in \cite[Lemma 3.4]{BP_edss}.

(Ch0.2) is straightforward.

For (Ch1), since outer faces of standard dendrices are again standard, one needs only consider the case 
$\bar{V}=U_i$, or else $\bar{V}$ would be in some $U_j$ for $j<i$.
But if $\bar{V}=U_i$, the assumption in (Ch1) states that
$\Xi^i = \emptyset$, so that $i$ must either be a dendrex where all vertices map to $\mathcal{O}$, i.e. $b_i \in N \mathcal{O}$,
or $U_i$ is a corolla its vertex maps to $B$, i.e. 
$b_i \in B$.
In either case, one has
$b_i \in A = B \cup N\mathcal{O}$, and (Ch1) follows.

To check both (Ch2) and (Ch3), observe first that
$V \hookrightarrow U_i$ will automatically be in $A_{<i}$ if either 
$\bar{V} \neq U_i$ or $T = U_i - D$ with 
$D \not \subseteq \Xi_i$
(since in either case $U_i$ would be in some $U_j$ with $j<i$),
so that one needs only consider the case 
$V=U_i$.

(Ch2) then follows since, except in the trivial cases where $\Xi^i = \emptyset$, the dendrex $b_i(U_i - \Xi^i)$ always contains at least one vertex not in $\mathcal{O} \amalg B$.

For (Ch3), we argue that if 
$b_i(U_i - \Xi^i) \in b_j \left( \Omega[U_j] \right)$
then in fact $i\leq j$.
Writing $\bar{U}_i = U_i - \Xi^i$, the hypothesis is that
\[
\begin{tikzcd}
	\bar{U}_i \ar{d} \ar{r}{\bar{\varphi}} & U_j \ar{r}{b_j} & N \mathcal{P}
\\
	U_i \ar[dashed]{ru}[swap]{\varphi}
\end{tikzcd}
\]
there is a face map $\bar{\varphi}$ as above such that
$b_i(\bar{U}_i) = \bar{\varphi}^{\**}(b_j)$, and the goal is to build $\varphi$ such that $b_i = \varphi^{\**}(b_j)$.
Let $w \in \boldsymbol{V}(\bar{U}_i)$ be a vertex and 
let $p_w$ be the corresponding operation of $\mathcal{P}$.
Then the outer fact $(U_i)_{w}$ is precisely $T_{p_w}$ from
\eqref{STANDELDE EQ}.
On the other hand, letting
$(U_j)_w - D_w \hookrightarrow (U_j)_w$
be any choice of reduced inner face, 
one has that this too is $T_{p_w}$, at least up to a replanarization isomorphism, i.e. one has isomorphims
$(U_i)_{w} \simeq (U_j)_w - D_w$,
compatible with the restrictions of $b_i$, $b_j$.
But then combining these isomorphisms yields the desired $\varphi$,
and (Ch3) follows.

Lastly, we show (Ch0.1).
Given any non-degenerate dendrex
$\Omega[V] \xrightarrow{c} N \mathcal{P}$
by applying \eqref{STANDELDE EQ} to each individual operation
(together with an ``assembly of substitution data'' argument)
one obtains that there exists a unique
non-degenerate standard dendrex 
$b_c \colon \Omega[U_c] \to N \mathcal{P}$,
edge subset $D_c \subseteq \Xi^c$ and isomorphism
$V \simeq U_c - D_c$
such that $b$ equals the composite
\begin{equation}\label{STANDELDEGER EQ}
	V \simeq U_c - D_c
	\hookrightarrow U_c
	\xrightarrow{b_c} N \mathcal{P}
\end{equation}
Recall now that, by the preliminary argument for (Ch2) and (Ch3), 
the non-degenerate dendrices not in 
$b_i^{-1}(A_{< i})$ are precisely the replanarizations of the faces 
$U_i - D$ with $D \subseteq \Xi^i$.
But the uniqueness of the data in \eqref{STANDELDEGER EQ}
implies that all such replanarizations of the $U_i - D$
do indeed have distinct images in $N \mathcal{P}$,
thus establishing (Ch0.1) and finishing the proof.
\end{proof}


\begin{remark}
	In general, injectivity of the map
	$b_i \colon \Omega[U_i] \to N \mathcal{P}$
	will fail in $b_i^{-1}(A_{< i})$.
	Indeed, in general two edges/vertices of $U_i$
	may be assigned the same color/operation of $\mathcal{P}$.
	In fact, injectivity may in general fail even for large outer faces.
\end{remark}


{\color{blue} Bla poset not finite but still projective/injective, which is enough}




















\newpage

\section{The Quillen equivalence}
\label{QE_SEC}

In this section, we synthesize the above results as well as the results of the related papers \cite{BP_geo,BP_edss,Per_eds},
to prove the main theorem of this project, Theorem \ref{QE_THM}.

\subsection{The non-equivariant adjunction}
We recall the adjunction in question.
\todo[inline]{say more. come back once the paper is organized}



\begin{definition}
      \label{OT_DEF}
      Fix $T \in \Omega$.
      The operad $\Omega(T)$ is the free colored operad generated by its edges and vertices
      it has colors $\mathfrak C = \mathbf E(T)$ the set of edges, and
      for any $\mathbf E(T)$-colored sequence $\vect C$, 
      \begin{equation}
            \Omega(T)(\vect C) =
            \begin{cases}
                  \** \qquad \qquad \qquad & \mbox{there exists $C \xrightarrow{\phi} T$ such that $\partial \phi = \vect C$}
                  \\
                  \varnothing & \text{else,}
            \end{cases}
      \end{equation}
      where $\partial \phi$ is the $\mathbf E(T)$-signature $(\phi(1), \phi(2), \dots, \phi(n); \phi(0))$
      given by the (ordered) image of the edges of $C$.
      \todo[inline]{profile of a map? is this used in other places?}
      This extends to a functor $\Omega \into \Op(\Set)$, and has an associated nerve-realization adjunction
      \[
            \Op(\Set) \leftrightarrows \dSet
      \]
      and \cite[Prop. 2.5]{CM11} show that this adjunction is Quillen.
\end{definition}

To enrich this to a Quillen adjunction out of $\sOp$, we need a topological enrichment of this inclusion.
We do this by constructing the free simplicial resolution \footnote{
  Also called the \textit{Godemont} resolution, c.f. \cite[\S 8.3]{BM06}.}
of $\Omega(T)$ as an algebra in the category of \textit{pointed} $\mathbf E(T)$-colored symmetric sequences $\Sym^{\mathbf E(T)}_{+}(\Set)$.
We elaborate on that distinction now.

\begin{definition}
      Given a set of colors $\mathfrak C$, let $\eta_{\mathfrak C} = \eta$ denote the initial $\mathfrak C$-colored operad in $\V$,
      defined for all $\mathfrak C$-sequences $\vect C$ by
      \[
            \eta(\vect C) =
            \begin{cases}
                  1_\V \qquad \qquad & \vect C = (x;x)
                  \\
                  \varnothing & \mbox{else.}
            \end{cases}
      \]
      The category of \textit{pointed} $\mathfrak C$-colored symmetric sequences in $\V$ is the category
      \[
            \Sym^{\mathfrak C}_+(\V) := \eta_{\mathfrak C} \downarrow \Sym^{\mathfrak C}(\V).
      \]
\end{definition}
We should think of pointed symmetric sequences as already having selected "identity" operations in any potential operadic structure.
Indeed, the monad $\mathbb F^{\mathfrak C}$ factors
\begin{equation}
      \label{FC_FAC_EQ}
      \begin{tikzcd}
            \Sym^{\mathfrak C}(\V) \arrow[r, shift left, "{(-) \amalg \eta}"]
            &
            \Sym^{\mathfrak C}_+(\V) \arrow[l, shift left] \arrow[r, shift left, "\mathbb F^{\mathfrak C}_+"]
            &
            \Op^{\mathfrak C}(\V) \arrow[l, shift left]
      \end{tikzcd}
\end{equation}
where $\mathbb F^{\mathfrak C}_+$ is defined on $X \in \Sym^{\mathfrak C}_+(\V)$ to be the pushout in $\Op^{\mathfrak C}(\V)$
\[
      \begin{tikzcd}
            \mathbb F^{\mathfrak C} \eta_{\mathfrak C} \arrow[d] \arrow[r]
            &
            \eta_{\mathfrak C} \arrow[d]
            \\
            \mathbb F^{\mathfrak C} X \arrow[r]
            &
            \mathbb F^{\mathfrak C}_+ X.
      \end{tikzcd}
\]

The following observations are straightforward.
\begin{lemma}
      $\mathbb F^{\mathfrak C}_+$ is left adjoint to the forgetful functor $\Op^{\mathfrak C}(\V) \to \Sym^{\mathfrak C}_+(\V)$,
      and \eqref{FC_FAC_EQ} is a factorization of the original free-forgetful adjunction for $\mathbb F^{\mathfrak C}$. 
\end{lemma}

We can now make the following definition of the $W$-construction.

\begin{definition}
      Given $T \in \Omega$, define $W(T)_\bullet \in \Op(\sSet) \subseteq (\Op^{\mathbf E(T)})^{\Delta^{op}}$ to be the bar construction
      \[
            W(T)_n = \left( \mathbb F^{\mathbf E(T)}_+ \right)^{n+1} \left(\Omega(T)\right),
      \]
      with the simplicial maps the monadic unit and monadic multiplication.
\end{definition}

We unpack this $\mathbf E(T)$-colored operad, culminating in \eqref{WT_EQ}.
First, we consider $\mathbb F^{\mathbf E(T)} \Omega(T)$ and $\mathbb F^{\mathbf E(T)}_+ \Omega(T)$.
We observe that $\mathbb F^{\mathbf E(T)} \Omega(T)(\vect C) = \varnothing$ iff $\Omega(T)(\vect C) = \varnothing$.
Otherwise, if $C \xrightarrow{\phi} T$ is such that $\partial \phi = \vect C$, then
\begin{equation}
      \label{FET_EQ}
      \mathbb F^{\mathbf E(T)}\Omega(T)(\vect C)
      = \set{\mbox{factorizations $C \xrightarrow{\psi_0} S_1 \xrightarrow{\psi_1} T$ of $\phi$ with $\psi_0$ tall}}.
\end{equation}
We note in particular that neither map in the factorization needs to be injective;
specifically, for any $e \in \mathbf E(T)$,
\begin{equation}
      \label{FET_EE_EQ}
      \mathbb F^{\mathbf E(T)}\Omega(T)(e;e) = \mathbb Z_{\geq 0}
\end{equation}
is in bijection with the set of non-negative integers.

\begin{example}
      More generally, we consider the simple case where $T = \mathbb Z_{/2}$, with leaves $\set{a,b}$ and root $r$.
      Then $\mathbb F^{\mathbf E(T)}\Omega(T)(a,b;r) = \mathbb Z_{\geq 0}^{\times \set{a,b,r}}$,
      where the tree $S_1$ associated to the triple $(n_a, n_b, n_r)$ has one node of valence two,
      with root (resp. left and right) path out of that node of length $n_r$ (resp. $n_a$, $n_b$).
      \[
            \begin{tikzpicture}
                  [grow=up,auto,
                  level distance=2em,
                  every node/.style = {font=\footnotesize,solid},
                  dummy/.style={circle,draw,inner sep=0pt,minimum size=1.75mm},
                  emph/.style={edge from parent/.style={black,thin,dotted,draw}},
                  norm/.style={edge from parent/.style={black,thin,draw,solid}}
                  ]
                  \node {$S$}
                  child[norm, level distance=1.5em] {node [dummy] {}
                    child[norm, level distance=1.2em] {node [dummy] {}
                      child[emph]{node [dummy] {} % vertex
                        child[norm] {node [dummy] {}
                          child[emph] {node [dummy] {}
                            child[norm] {edge from parent node [swap] {$b$}}
                          }
                        }
                         child[norm] {node [dummy] {}
                          child[emph] {node [dummy] {}
                            child[norm] {edge from parent node {$a$}}
                          }
                        }
                      }
                    }
                    edge from parent node [swap] {$r$}
                  };
            \end{tikzpicture}
      \]
      where there are $n_r$, $n_b$, and $n_a$ edges on the bottom, right, and left paths respectively.
\end{example}
      
Now, $\mathbb F^{\mathbf E(T)}\eta_{\mathbf E(T)}(\vect C) = \varnothing$ unless $\vect C = (e;e)$ for some $e \in \mathbf E(T)$;
in this case, $\mathbb F^{\mathbf E(T)}\eta(e;e) = \mathbb Z_{\geq 0}$, as in \eqref{FET_EE_EQ}.
% 
If we also denote the unit map by $\eta$, we see that the image of the map
\[
      \mathbb F^{\mathbf E(T)}\eta \xrightarrow{\mathbb F^{\mathbf E(T)}\eta} \mathbb F^{\mathbf E(T)} \Omega(T)
\]
corresponds to all factorizations from \eqref{FET_EQ} with $S_1$ a linear tree and $\phi$ (and hence $\psi_1$) a degeneracy.

Thus in the pushout $\mathbb F^{\mathbf E(T)}_+\Omega(T)$, all the non-injective operations are identified, and we have
\begin{align*}
  \mathbb F^{\mathbf E(T)}_+\Omega(T)(\vect C)
  & = \set{\mbox{factorizations $C \xrightarrow{\psi_0} S_1 \xrightarrow{\psi_1} T$ of $\phi$ with $\psi_1$ injective, $\psi_0$ tall}} \\
  & = \set{\mbox{inner faces $S$ of $T_{\phi(1 2 \dots n) \leq \phi(0)}$}} \\
  & = \set{\mbox{subsets $S \subseteq \mathbf E^i(T_{\phi(1 2 \dots n) \leq \phi(0)})$ of inner edges}}    
\end{align*}
where $1 2 \dots n \leq 0$ is the vertex of $C$ and $T_{\phi(1 2 \dots n) \leq \phi(0)}$ is the associated outer face of $T$.

In a similar fashion,
% ---------------------------------------- OLIVE GREEN ----------------------------------------
{\color{OliveGreen} using concatinations of pushout squares of the form
  \[
        \begin{tikzcd}
              \mathbb F^2 \eta \arrow[r] \arrow[d]
              &
              \mathbb F \eta \arrow[d] \arrow[r]\
              &
              \eta \arrow[d]
              \\
              \mathbb F X \arrow[r]
              &
              \mathbb F( \mathbb F_+ X) \arrow[r]
              &
              \mathbb F_+^2 X,
        \end{tikzcd}
  \]
} % ---------------------------------------- OLIVE GREEN ----------------------------------------
  we see that
\begin{align*}
  \left(\mathbb F^{\mathbf E(T)}\right)^n \Omega(T)(\vect C)
  & = \set{
    \begin{array}{l}
      \mbox{factorizations $C \xrightarrow{\psi_0} S_1 \xrightarrow{\psi_1} S_2 \xrightarrow{\psi_2} \dots \xrightarrow{\psi_{n-1}} S_n \xrightarrow{\psi_n} T$}
      \\
      \mbox{of $\phi$ with $\psi_i$ tall for $i < n$}
    \end{array}
  },
  \\
  \left(\mathbb F^{\mathbf E(T)}_+ \right)^n \Omega(T)(\vect C)
  & = \set{
    \begin{array}{l}
      \mbox{factorizations $C \xrightarrow{\psi_0} S_1 \xrightarrow{\psi_1} S_2 \xrightarrow{\psi_2} \dots \xrightarrow{\psi_{n-1}} S_n \xrightarrow{\psi_n} T$}
      \\
      \mbox{of $\phi$ with $\psi_i$ injective for $i > 1$ and $\psi_i$ tall for $i < n$}
    \end{array}
  }
  \\
  &= \set{\mbox{nested subsets $S_1 \subseteq S_2 \subseteq \dots \subseteq S_n \subseteq \mathbf E^i(T_{\phi(1 2 \dots n) \leq \phi(0)})$ of inner edges}}.
\end{align*}

Finally, for any finite set $A$, we recall that the $n$-simplicies of $\Delta[1]^{\times A}$ correspond to $(n+1)$-strings of nested subsets of $A$.
Thus we have the following description of $W(T)$:

\begin{equation}
      \label{WT_EQ}
      W(T)(\vect C) = 
      \begin{cases}
            \Delta[1]^{\times \mathbf E^i(T_{\underline e \leq e})}, \qquad & \mbox{there exist $C \xrightarrow{\phi} T$ with $\partial \phi = \vect C$}
            \\
            \varnothing & \text{else,}
      \end{cases}
\end{equation}

It is clear that $W(-)$ is functorial. Thus we have the following definition.

\begin{definition}
      The \textit{homotopy coherent dendroidal nerve} and the \textit{$W_!$-construction}
      are the right- and left-adjoints of the nerve-realization adjunction associated to the functor $W: \Omega \into \Op(\sSet)$.
      \begin{equation}
            \label{SOPDSET_EQ}
            hc N \colon \sOp \leftrightarrows
            \dSet \colon W_!
      \end{equation}
\end{definition}
The culmination of extensive work by Cisinski-Moerdijk-Weiss \cite{CM13a,CM13b,CM11,MW09,MW07} is the following.

\begin{theorem}
      \label{CMW_THM}
      $hcN$ and $W_!$ form a Quillen equivalence with respect to the model structures
      which, in particular, are recovered by the $G = \**$ cases of Theorem \ref{MODEL_THM} and \cite[Thm. 2.1]{Per_eds}.
\end{theorem}

\todo[inline]{come back}

\subsection{The equivariant adjunction}

The main goal of this project \cite{BP_geo,BP_edss,Per_eds} is to understand the correct equivariant lifting of Theorem \ref{CMW_THM}.
On the one hand, the construction of the new adjunction is immediate:

\begin{definition}
      Define
      \[
            W_! \colon \dSet^G \rightleftarrows \sOp^G \colon hcN_d
      \]
      to be the
      extension of the adjunction \eqref{SOPDSET_EQ}
      given by post-composition.
\end{definition}

% Equivalently, given decompositions $T \simeq G \cdot_H U$ and $T \simeq G \cdot U / N$, we have
% $W(T) \simeq G \cdot_H W(U)$ and $W(T) \simeq G \cdot W(U)/N$.


On the other hand, the homotopy theories involved are much more involved to capture the genuine equivariance,
as noted in \S \ref{MSC_SEC}, \S \ref{MS_SEC}, \S \ref{EDS_SEC}, and \cite{Per_eds}.

We have the following generalization of \cite[Prop 4.5]{CM11} (see also \cite[Prop. 6.15]{Per_eds}).

\begin{proposition}
      \label{W!_COF_PROP}
      Suppose $\F = \set{\F_n}$ is a weak indexing system.
      Then $W_!: \dSet^G \to \sOp^G$ sends $\F$-normal monomorphisms to cofibrations and inner $\F$-anodyne extensions to trivial $\F$-cofibrations.
\end{proposition}
\begin{proof}
      It suffices to check this on the generating maps.
      To that end, let $T \in \Omega_F$, choose a decomposition $T \simeq G \cdot_H T_e$ and consider the map
      \begin{equation}
            G \cdot_H \left( W_! \partial \Omega[T_e] \xrightarrow{\ i \ } W(T_e) \right)      
      \end{equation}
      in $\sOp^G$.
      If $T_e = \eta$, then $i$ is the canonical map $\varnothing \to \eta$,
      % W_! \partial \Omega[\eta] = \varnothing \to W(\eta) = \eta$,
      and so $G \cdot_H i$ is a generating cofibration from \eqref{IFJF_EQ} in $\sOp^G$.

      Thus we may assume that $|V(T)| \geq 1$.
      In this case, $i$ is an isomorphism on object $H$-sets $\mathbf E(T_e)$,
      and for any $\mathbf E(T_e)$-signature $\vect C$,
      $i(\vect C)$ is the identity unless $\vect C = h \cdot \partial \mathsf{lr}(T_e)$ for some $h \in H$,
      where $\partial \mathsf{lr}(T_e)$ is the signature $(e_1, \dots, e_n; e)$ 
      with $\set{e_1,\dots, e_n}$ the set of leaves of $T_e$ and $e$ the root.
      When $\vect C = \partial \mathsf{lr}(T_e)$, $i(\vect C)$ is the iterated pushout product
      \[
            i(\vect C) = (\partial \Delta[1] \to \Delta[1])^{\square \mathbf E^i(T_e)}.
      \]
      We need that $i(\vect C)$ is a genuine cofibration in $\sSet^{\Aut(\vect C)}_\F$.
      However, $\Aut(\vect C) = \Gamma \leq H \times \Sigma_n$ for $\Gamma \leq H \times \Sigma_n$ the graph subgroup encoding the $H$-action on the set of leaves $\set{e_1, \dots, e_n}$,
      and since $\F$ is a weak indexing system, $\Aut(\vect C) \in \F_n$,
      and thus, as $\F$ is closed under subgroups, $\sSet^{\Aut(\vect C)}_\F = \sSet^{\Aut(\vect C)}_{gen}$.
      Then the claim follows from the fact that monomorphisms are genuine cofibrations,
      {\color{OliveGreen} % ---------- OLIVE GREEN ----------------------------------------
        or using the fact that $\sSet$ has cofibrant symmetric pushout powers,
        so $(\partial \Delta[1] \to \Delta[1])^{\square m}$ is a genuine $\Sigma_m$-cofibration where $m = |\mathbf E^i(T_e)|$,
        and then as restriction is left Quillen on genuine model structures,
        \[
              \mathrm{Res}^{\Sigma_m}_{\Aut(C)}(\partial \Delta[1] \to \Delta[1])^{\square m} = i(\vect C),
              \qquad
              \Aut(\vect C) = \Gamma \xrightarrow{pr} H \xrightarrow{\mathbf E^i(T_e)} \Sigma_m,
        \]
        is a genuine $\Aut(\vect C)$-cofibration.
      } % ---------------------------------------- OLIVE GREEN --------------------
      %
      It is then straightforward to check that $G \cdot_H i$ has the left lifting property against all local trivial $\F$-fibrations in $\sOp^G$.
      {\color{OliveGreen} % ----------------------------------------
        Indeed, we clearly have a lifts on the level of sequences.
        To check that the resulting map is operadic, it suffices to check on composites of the form
        $C_0 \circ (C_1, \dots, C_n) = \mathsf{lr}(T_e)$.
        If $C_j = \mathsf{lr}(T_e)$ for some $j$, then the composite map is just the identity, and there is no content to check.
        If $C_j \neq \mathsf{lr}(T_e)$ for all $j$,
        then this follows by a simple diagram chase on the cube relating the compositions in $W(T)$ and $W_! \partial \Omega(T)$ with those in the source and target of the testing trivial fibration.
      } % COLOR OLIVEGREEN
      Hence $G \cdot_H i$ is an $\F$-cofibration in $\sOp^G$ by Lemma \ref{CAV_4.8}.
      
     
      Similarly, consider a generating $\F$-inner horn inclusion
      \[
            G \cdot_H \left(W_! \Lambda^{H e}[T_e] \xrightarrow{\ j \ } W(T_e) \right).
      \]
      with $T \simeq G \cdot_H T_e \in \Omega_\F$ and $e \in \mathbf E(T_e)$ some inner edge.
      Again, $j$ is bijective on objects and $j(\vect C)$ is the identity unless $\vect C = h \cdot \partial \mathsf{lr}(T_e)$.
      When $\vect C = \partial \mathsf{lr}(T_e)$, $j(\vect C)$ is the iterated pushout product
      \begin{equation}
            j(\vect C) =
            \left(
                  \begin{tikzcd}
                        \partial \left(
                              \Delta[1]^{\times \mathbf E^i(T_e) \setminus H e}
                        \right) \arrow[d]
                        \\
                        \Delta[1]^{\times \mathbf E^i(T_e) \setminus H e}
                  \end{tikzcd}
            \right) \square
            \left(
                  \begin{tikzcd}
                        \set{1} \arrow[d]
                        \\
                        \Delta[1]
                  \end{tikzcd}
            \right)^{\square H e}
            =
            \left(
                  \begin{tikzcd}
                        \partial \Delta[1] \arrow[d]
                        \\
                        \Delta[1]
                  \end{tikzcd}
            \right)^{\square \mathbf E^i(T_e) \setminus H e}
            \square
            \left(
                  \begin{tikzcd}
                        \set{1} \arrow[d]
                        \\
                        \Delta[1]
                  \end{tikzcd}
            \right)^{\square H e}.                              
      \end{equation}
      A similar argument shows that $G \cdot_H j$ has the left lifting property against all local $\F$-fibrations,
      hence is a trivial $\F$-cofibration in $\sOp^G$ by Lemma \ref{CAV_4.3}, as desired. 
\end{proof}

We have the following immediate corollary.
\begin{corollary}
      [{cf. \cite[Prop. 6.15]{Per_eds}, \cite[Cor. 4.6]{CM11}}]
      For any $\F$-fibrant $\O \in \sOp^G$, $h c N \O$ is an $\F$-$\infty$-operad.
\end{corollary}

Another corollary is the following key step.

\begin{proposition}[{cf. \cite[Prop/ 4.9]{CM11}}]
      $W_!: \dSet^G \to \sOp^G$ is left Quillen.
\end{proposition}
\begin{proof}
      As we already know by Proposition \ref{W!_COF_PROP} that $W_!$ preserves cofibrations,
      it suffices by Corollary \ref{SIMPLQUILL COR} to show that $h c N$ preserves fibrations between fibrant objects.
      Suppose $f: \O \to \P$ is an $\F$-fibration between $\F$-fibrant operads in $\sOp^G$.
      Then Proposition \ref{W!_COF_PROP} also implies that $h c N (f)$ is an $\F$-inner-fibration between $\F$-$\infty$-operads.
      By \cite[Thm. 8.22]{Per_eds}, this is an $\F$-fibration in $\dSet^G_\F$ iff $\tau (h c N_d(f)^H) = \tau (h c N_d(f^H))$ is a categorical fibration for all $H \leq G$.
      But by Remark \ref{FIB_ISOFIB_REM}, being a fibration in $\sOp^G_\F$ implies that $\pi_0(f^H)$ is a categorical fibration for all $H \leq G$, 
      and since $\pi_0 = \tau \circ h c N$ by \cite[Prop. 4.8]{CM11}, the result follows.
\end{proof}

\begin{remark}
      One can also show a more powerful equivariant version of \cite[Prop. 4.8]{CM11},
      namely that the following diagram commutes.
      \begin{equation}
            \begin{tikzcd}
                  \sOp^G \arrow[r, "i_{\**}"] \arrow[d, "h c N"']
                  &
                  \sOp_G \arrow[d, "h c N"] \arrow[r, "\pi_0"]
                  &
                  \Op_G \arrow[d, equal]
                  \\
                  \dSet^G \arrow[r, "i_{\**}"]
                  &
                  \dSet_G \arrow[r, "\tau_G"]
                  &
                  \Op_G
            \end{tikzcd}
      \end{equation}
      In particular, this recovers that $\pi_0 \circ (-)^H = \tau \circ h c N \circ (-)^H$ for all $H$,
      but also more subtle information about the interaction of $\pi_0$ and $h c N$ with the graph subgroup fixed points.
      This story, along with colored genuine equivariant operads as well as cofibrancy considerations of the above narrative,
      will be further explored in a sequel. 
\end{remark}

\begin{remark}
      For any $\mathcal F$-tree $T \in \Omega_{\mathcal F}$, $\Omega[T] \in \dSet^G$ is $\mathcal F$-cofibrant/normal,
      and hence $W(T)$ and $\Omega(T)$ are $\mathcal F$-cofibrant in $\sOp^G$ for all $T \in \Omega_{\mathcal F}$.
\end{remark}

come back!



Finally, the following is a reformalized proof of \cite[Thm. 8.14]{CM13b}, extended to the equivariant setting;
the main result is an immediate corollary.

\begin{proposition}\label{COMUOTOHOM PROP}
      % For all weak indexing systems $\F$,
      The (right) derived composite functors in the following diagram commute up to a zigzag of weak equivalences. 
      \[
            \begin{tikzcd}
                  \mathsf{PreOp}^G \ar{d}[swap]{\gamma^{\**}}&
                  \mathsf{sOp}^G \ar{l}[swap]{N} \ar{d}{hcN}
                  \\
                  \mathsf{sdSet}^G &
                  \mathsf{dSet}^G \ar{l}{c_{!}}
            \end{tikzcd}
      \]
\end{proposition}

Note that though $\gamma^{\**}$ and $c_{!}$ are left Quillen, they both preserve all equivalences, 
so that one needs only perform fibrant replacements in 
$\mathsf{sOp}$.

\begin{proof}
      Recall that, given an object $X$ in a model category $\mathcal{M}$, a simplicial frame of $X$ is a fibrant replacement
      $c_!(X) \to \widetilde{X}(\bullet)$ of the constant 
      simplcial object $c_!(X)$ in the Reedy model structure on $\mathcal{M}^{\Delta^{op}}$.
      Moreover, if $X$ was already fibrant one is free to assume that $\widetilde{X}(0) = X$.
      
      Let $\mathcal{O} \in \mathsf{sOp}^G$ be fibrant, 
      choose a (functorial) fibrant simplicial frame
      $\widetilde{\mathcal{O}}(\bullet) \in (\mathsf{sOp}^G)^{\Delta^{op}}$, where we assume $\widetilde{\mathcal{O}} (0) = \mathcal{O}$.
      Next, let 
      $\gamma^{\**} N \widetilde{\mathcal{O}}(\bullet) 
      \to \widetilde{Q}(\bullet)$
      be a Reedy fibrant replacement in  
      $(\mathsf{sdSet}^G)^{\Delta^{op}}$.
      
      We claim that the following is a zigzag of weak equivalences in $\mathsf{sdSet}^G$.
      \begin{equation}\label{BIGZIG EQ}
            \gamma^{\**} N \mathcal{O} \xrightarrow{\sim}
            \widetilde{Q}(0) \xrightarrow{\sim}
            \delta^{\**} \widetilde{Q} \xleftarrow{\sim}
            \widetilde{Q}_0 \xleftarrow{\sim}
            \left(\gamma^{\**} N \widetilde{\mathcal{O}}\right)_0
            \xrightarrow{\sim}
            hcN \widetilde{\mathcal{O}} \xleftarrow{\sim}
            c_{!} hcN \mathcal{O}
      \end{equation}
      
      That the first map is an equivalence is obvious from definition of $\widetilde{Q}$ and the assumption $\widetilde{\mathcal{O}}(0) = \mathcal{O}$.
      
      The second, third, fourth, and fifth maps are all in fact simplicial equivalences.
      For the second and third maps, note first that $\widetilde{\O}$ and hence
      $\widetilde{Q}$ are homotopically constant, in the sense that
      all vertex maps $\widetilde{Q}(m) \to \widetilde{Q}(0)$ are joint equivalences in $\mathsf{sdSet}^G$.
      % 
      Moreover, since the levels $\widetilde{Q}$ are joint fibrant in $\mathsf{sdSet}^G$,
      Lemma \ref{COMUOTOHOM_FACTS} \ref{SFIB_JEQ_LBL} % \cite[Prop. 4.5(iii)]{BP_edss}
      implies that these are in fact simplicial equivalences, % $G$-graph simplicial equivalences,
      i.e. that for each $G$-tree $T \in \Omega_G$, the evaluations % tree $U \in \Omega$, the evaluations
      $\widetilde{Q}(\Omega[T])(m) \to \widetilde{Q}(\Omega[T])(0)$
      are Kan equivalences in $\mathsf{sSet}$.  %$G$-graph Kan equivalences in $\mathsf{sSet}^{G \times \Aut(U)}$.
      % 
      But since $\widetilde{Q}(\Omega[T]) \in \sSet^{\Delta^{op}}$ %$\widetilde{Q}(U) \in \mathsf{sSet}^{\Delta^{op} \times (G \times \Aut(U))}$
      is itself Reedy fibrant by
      Lemma \ref{COMUOTOHOM_FACTS} \ref{RJOINTT_LBL},
      Lemma \ref{COMUOTOHOM_FACTS} \ref{JF_VERT_LBL} % (the proof of) \cite[Prop. 4.24(ii)]{BP_edss} 
      implies that it is in fact joint Reedy fibrant in $\mathsf{sSet}^{\Delta^{op}}$ % $G$-graph joint Reedy fibrant in $\left(\mathsf{sSet}^{\Delta^{op}}\right)^{G \times \Aut(U)}$,
      and hence by Lemma \ref{COMUOTOHOM_FACTS} \ref{DIAG_LBL}, %\cite[Prop. 4.5(iv)]{BP_edss}
      one has Kan equivalences % $G$-graph Kan equivalences
      $\widetilde{Q}(\Omega[T])(0) \xrightarrow{\sim}
      \delta^{\**} \widetilde{Q}(\Omega[T]) \xleftarrow{\sim}
      \widetilde{Q}_0(\Omega[T])$, as desired.
      
      For the fourth equivalence, note that one can write
      \[
            \widetilde{Q}_0(\Omega[T]) = 
            \mathsf{Hom}_{\mathsf{sdSet}^G}(c_!\Omega[T],\widetilde{Q})=
            \mathsf{Hom}_{\mathsf{PreOp}^G}(c_!\Omega[T],\gamma_{\**}\widetilde{Q}),
      \] 
      \[
            \left(\gamma^{\**} N \widetilde{\mathcal{O}}\right)_0(\Omega[T]]) = 
            \mathsf{Hom}_{\mathsf{sdSet}}(c_!\Omega[T],\gamma^{\**} N \widetilde{\mathcal{O}})=
            \mathsf{Hom}_{\mathsf{PreOp}}(c_!\Omega[T], N \widetilde{\mathcal{O}}),
      \]
      in $\sSet$ for each $T \in \Omega_G$. % $\sSet^{G \times \Aut(U)}$ for each $U \in \Omega$.
      The claim now follows by noting that
      $N \widetilde{\mathcal{O}} \to \gamma_{\**} \widetilde{Q}$
      is an equivalence between Reedy fibrant replacements for $N \mathcal O$ in $(\mathsf{PreOp}^G_{tame})^{\Delta^{op}}$
      (as $\gamma^{\**}\gamma_{\**}\tilde Q \to \tilde Q$ is a joint equivalence and $N$, $\gamma_{\**}$ are right Quillen),
      and that $\Omega[T]$ is tame cofibrant by Lemma \ref{OMEGATTAME_LEM}.
      
      For the fifth equivalence, note that
      \[
            \left(\gamma^{\**} N \widetilde{\mathcal{O}}\right)_0(\Omega[T]) = 
            \mathsf{Hom}_{\mathsf{PreOp}^G}(\Omega[T], N \widetilde{\mathcal{O}}) =
            \mathsf{Hom}_{\mathsf{sOp}^G}(\Omega(T),  \widetilde{\mathcal{O}}),
      \]
      \[
            \left(hcN \widetilde{\mathcal{O}} \right)(\Omega[T]) = 
            \mathsf{Hom}_{\mathsf{sOp}^G}(W(T),  \widetilde{\mathcal{O}}),
      \]
      in $\sSet$ for each $T \in \Omega_G$, %$\sSet^{G \times \Aut(U)}$ for each $U \in \Omega$,
      so that the required claim follows since 
      $\widetilde{\mathcal{O}}$ is Reedy fibrant and
      $W(T) \to \Omega(T)$ is an equivalence of cofibrant operads in $\sOp^G$. % a $G$-graph weak equivalence of cofibrant operads in $\sOp^{G \times \Aut(U)}$.

      Finally, for the last map, one needs simply to note that
      $c_! hcN \mathcal{O} = hcN c_! \mathcal{O}$, so that the required claim follows since 
      $c_! \mathcal{O} \to \widetilde{O}$
      is a levelwise equivalence of levelwise fibrant operads
      and $hcN: \sOp^G \to \dSet^G$ is right Quillen.
\end{proof}

\begin{proof}
      [Proof of Theorem \ref{QE_THM}]
      Proposition \ref{COMUOTOHOM PROP} implies that $W_!$ is an equivalence of homotopy theories.
      Since by Proposition \ref{W!_COF_PROP} it is left Quillen,
      the result follows.
\end{proof}


\section{Indexing system analogue results}

As in \cite[\S 6]{BP_edss} and \cite[\S 9]{Per_eds}, we dedicate our final section to 
outlining the variations of the main results from Sections \ref{TAME_SEC} and \ref{QE_SEC} to
the \textit{indexing systems} of Blumberg-Hill \cite{BH15}, or more accurately
the mild generalization of \textit{weak indexing systems} of the authors \cite[\S 9]{Per_eds}, \cite[\S4.4]{BP_geo} (and independently identified by Gutierrez-White \cite{GW}).







For COMUOTOHOM PROP: replace all instances of $\Omega_G$ with $\Omega_{\mathcal F}$ - in particular, $X \to Y$ is a simplicial equivalence in $\mathsf{sdSet}^G_{\mathcal F}$ iff $X(\Omega[T]) \to Y(\Omega[T])$ is a Kan equivalence for all $T \in \Omega_{\mathcal F}$. 



\iffalse%


%%%%%%%%%%%%%%%%%%%%%%%%%%%%%%%%%%%%%%%%%%%%%%%%%%%%%%%%%%%%%%%%%%%%%%%
% -------------------- UP: PETER, DOWN: LUIS  --------------------
%%%%%%%%%%%%%%%%%%%%%%%%%%%%%%%%%%%%%%%%%%%%%%%%%%%%%%%%%%%%%%%%%%%%%%%

\newpage

\section{In $\mathsf{dSet}_G$}

\begin{definition}
      Define the \textit{genuine operadic nerve} $N: \Op_G \to \dSet_G$ by
      \begin{equation}
            N\P(T) = \Hom_{\Op_G}(T, \P)
      \end{equation}
      where we think of $T$ as the operad $T \in \Op^G \into \Op_G$. 
\end{definition}

\begin{remark}
      We note that $N\P \in (SCI)^{\boxslash !}$,
      as $T \in \Op_G$ is a free $\mathbb F_G$-algebra on its vertices.
\end{remark}

\begin{remark}
      We can rephrase the definition of being an $\mathbb F_G$-algebra in terms of $N\P$.
      For $\P \in \Sym_G$ a $G$-symmetric sequence,
      a genuine $G$-operad structure on $\P$ is given by:
      \begin{itemize}
      \item Composition Maps: $ $\\
            maps 
            $N\P(T) \to \P(\mathsf{lr}(T))$
            for all $T \in \Omega_G$.
      \item Naturality under restriction and conjugation: $ $\\
            maps $N\P(T_1) \to N\P(T_0)$
            for all quotient maps $T_0 \to T_1$ in $\Omega_{G,0}$,
            such that the following commutes:
            \begin{equation}
                  \begin{tikzcd}
                        N\P(T_1) \arrow[r] \arrow[d]
                        &
                        \P(\mathsf{lr}(T_1)) \arrow[d]
                        \\
                        N\P(T_0) \arrow[r]
                        &
                        \P(\mathsf{lr}(T_0)).
                  \end{tikzcd}
            \end{equation}
      \item Associativity under $\mathbb F_G$: $ $\\
            maps $N\P(T_1) \to N\P(T_0)$
            for all planar tall maps $T_0 \to T_1$ in $\Omega_G^t$,
            such that the analogus diagram (with the right vertical map the identity) commutes.\footnote{
              As in \cite{BP_geo}, we note that ``associativity'' under $\mathbb F_G$ includes both
              the usual notion of associativity of our composition maps,
              but also unitality;
              this is recorded here by the fact that degeneracies are always planar tall.}
      \end{itemize}
\end{remark}

The above reflects the following result.

\begin{proposition}
      $\Op_G$ is equivalent to the subcategory of $\mathsf{dSet_G}$ spanned by those $X$ such that
      \begin{enumerate}
      \item $X(H/H) = \set{\**}$ for all $H \leq G$.
      \item $X(T) \cong \otimes_{T_v \in V(T)}X(T_v)$. 
      \end{enumerate}
\end{proposition}
\begin{proof}
      The fact that $N\P \in (SCI_{\F})^{\boxslash !}$ is immediate, as remarked above.

      For the reverse direction, we will follow the construction of the homotopy operad as in \cite[\S 6]{MW09},
      replacing their use of inner horn inclusions with \textit{orbital} inner $G$-horn inclusions,
      to show that any $X \in (OHI)^{\boxslash !}$ is in the image of $N$; 
      the result will then follow from \cite[HYPER PROP]{BP_edss}.

      In fact, interpreting all of their pictures are as \textit{orbital} representations of $G$-trees yields that,
      for all $C \in \Sigma_G$
      \begin{itemize}
      \item $\sim_{G e}$ is an equivalence relation on $X(C)$ for all $Ge \in E_G(C)$.
      \item The relations $\sim_{G e}$ and $\sim_{G e'}$ are equal for all $e,e'\in E(C)$.
      \item $[h] \circ [f] = [h \circ f]$ yields a well-defined composition map.
      \end{itemize}
      \todo[inline]{naturality, associativity of composition}
\end{proof}


\newpage




\section{Scratchwork}




\subsection{Semi-cofibrantly generated}


The following codifies a formal argument implicit in the proof of \cite[Thm. 7.19]{CM13b}.

\begin{definition}
Given a set $J$ of maps that admit the small object argument, we say that $X \in \mathcal{M}$ is \textit{$J$-fibrant} if $X \to \**$ has the right lifting property against maps in $J$.

Further, given $D$ a class of maps in $\mathcal{M}$,
we write $D_{J\text{-fib}} \subseteq D$ to denote 
the subclass of maps whose target is $J$-fibrant.
\end{definition}

\begin{lemma}\label{SEMICOF LEM}
	Let $\mathcal{M}$ be a model category with $(C,W,F)$
	the corresponding classes of cofibrations, weak equivalences and fibrations. 
	Further, $J$ be a set of maps admitting the small object argument and such that:
\begin{itemize}
	\item[(i)] $J \subseteq C \cap W$;
	\item[(ii)] 
	$\left(J^{\boxslash} \cap W \right)_{J\text{-fib}}
	\subseteq \left( F \cap W \right)_{J\text{-fib}}$.
\end{itemize}
Then one further has that:
\begin{itemize}
	\item[(a)]
	$\left(\prescript{\boxslash}{}{\left(J^{\boxslash}\right)}\right)_{J\text{-fib}}
	= 
	\left( C \cap W \right)_{J\text{-fib}}$;
	\item[(b)]
	$\left(J^{\boxslash} \right)_{J\text{-fib}}
	= F_{J\text{-fib}}$.
\end{itemize}
\end{lemma}

\begin{remark}
Rephrasing (b), one has that the fibrant objects of $\mathcal{M}$ are precisely the $J$-fibrant objects
and thus that the fibrations between fibrant objects are precisely the $J$-fibrations.
\end{remark}

\begin{proof}
	To check (a), recalling first that 
	$\prescript{\boxslash}{}{\left(J^{\boxslash}\right)}$
	is the saturation of $J$, one has that (i) in fact implies 
	$\prescript{\boxslash}{}{\left(J^{\boxslash}\right)}
		\subseteq C \cap W $.
	For the converse direction, given a trivial cofibration
	$A \to Y$ with $J$-fibrant target,
	form the factorization 
	$A \to X \to Y$ as a 
	$J$-cofibration followed by a $J$-fibration. 
	By the first direction the map $A\to X$ is a weak equivalence, and thus by 2-out-of-3 so is $X \to Y$.
	But then by (ii) the map $X \to Y$ is a trivial fibration, so that the lifting below exists,
	showing that $A \to Y$ is a retract of $A \to X$, and thus also in the saturation $\prescript{\boxslash}{}{\left(J^{\boxslash}\right)}$, 
	as desired.
\[
\begin{tikzcd}
	A \ar[>->]{r}{J} \ar[>->]{d}[swap]{\sim}&
	X \ar[->>]{d}{J}
%& &
%	A \ar[>->]{r}{\sim} \ar[>->]{d}[swap]{\sim}&
%	Y \ar[->>]{d}{\sim}
\\
	Y \ar[equal]{r} \ar[dashed]{ru} & Y
%& &
%	X \ar[equal]{r} \ar[dashed]{ru} & X
\end{tikzcd}
\]

To check (b), one direction is again immediate from (i),
since $J^{\boxslash} \supseteq (C \cap W)^{\boxslash} = F$.
For the converse direction, it suffices to show that 
a $J$-fibration $X\to Y$ with $J$-fibrant target has the right lifting property against trivial cofibrations, as on the left diagram below.
After factoring the bottom horizontal map as a $J$-cofibration followed by a $J$-fibration as on the right diagram, it suffices to shows that a lift $B' \to X$ exists.
But since $B'$ is $J$-fibrant, this follows from (a), which shows that the composite $A \to B \to B'$ is a $J$-cofibration.
\[
\begin{tikzcd}
	A \ar{r} \ar[>->]{d}[swap]{\sim}&
	X \ar[->>]{d}{J}
&&
	A \ar{rr} \ar[>->]{d}[swap]{\sim}&&
	X \ar[->>]{d}{J}
\\
	B \ar{r} \ar[dashed]{ru} & Y
&&
	B \ar[>->]{r}[swap]{J} &
	B' \ar[->>]{r}[swap]{J} \ar[dashed]{ru}
	& Y
\end{tikzcd}
\]
\end{proof}

\begin{remark}
	Analyzing the proof above, one is free to replace the class of fibrant objects with any other class that is compatible with $J$-fibrations, in the sense that if 
	$X \to Y$ is a $J$-fibration and $Y$ is in the class, then so is $X$.
\end{remark}





\subsection{Formalizing some stuff}



\begin{lemma} \label{INTER_LEM}
Let $\mathcal{O} \in \mathsf{sOp}$ and let
$g \colon x \to y$ be an equivalence in $\mathcal{O}$.

Then there exists a countable, cofibrant and contractile $H \in \mathsf{sOp}_{\{0,1\}}$ 
and a map 
$\varphi \colon H \to \mathcal{O}$
such that 
$g$ is in the image of $\varphi$. 
\end{lemma}


\begin{proof}
	We start by considering the case where $\mathcal{O}$ is locally fibrant.
	
	$g$ can be regarded as a map
	$[1] \xrightarrow{g} \mathcal{O}$,
	and one thus likewise gets a map
	$\Delta[1] \xrightarrow{g}  hcN \mathcal{O}$
	which is an equivalence in the 
	$\infty$-category $hcN \mathcal{O}$,
	so that one can find a (non-unique) factorization
	$\Delta[1] \to J \to hcN \mathcal{O}$
	which by adjunction yields a factorization
	$[1]=W_!\Delta[1] \to W_! J \to \mathcal{O}$,
	which establishes the desired claim 
	since $W_! J$ is contractible due to 
	Example \ref{WJ EX}.
	
	For a general $\mathcal{O}$, 
	consider first a local fibrant replacement
	$F \colon \mathcal{O} \to \mathcal{O}'$.
	One can hence find a map 
	$W_! J \to \mathcal{O}'$ such that
	$F(g)$ is in its image. 
	We now factor this map as
	$W_! J \xrightarrow{\sim} H \to \mathcal{O}'$
	where the second map is a local fibration.
	
	One can now form a pullback
\[
\begin{tikzcd}
	\tilde{H} \ar{r} \ar{d} & H' \ar{d}
\\
	\mathcal{O} \ar{r} & \mathcal{O}'
\end{tikzcd}
\]
where $\tilde{H}$ is seen to be contractible since
$\mathsf{sSet}$ is right proper.
	A priori, $\tilde{H}$ will need not be countable nor cofibrant, but this is easily rectified:
	indeed one can show that any countable subcomplex of $\tilde{H}$ is contained in a contractible countable subomplex, 
	yielding a countable contractible subcomplex whose image in $\mathcal{O}$ contains $g$. Lastly, performing a cofibrant replacement of that complex finishes the proof.
\end{proof}



\begin{example}\label{WJ EX}
	Let $J = N \widetilde{[1]}$ be the nerve of the contractible groupoid on two objects.
	
	Then there is an identification
\[
	W_{!} J \simeq \mathbb{F}^{\bullet} \widetilde{[1]}
\]
	where $\mathbb{F}$ denotes the (unital) free operad monad.

	To see this, we start by describing
	$\mathbb{F}^{\bullet} \widetilde{[1]}$.
	Writing $f \colon 0 \to 1$ and 
	$g \colon  1\to 0$ for the non-identity arrows in 
	$\widetilde{[1]}$ (so that $g=f^{-1}$), the $0$-simplices of $\mathbb{F}^{\bullet} \widetilde{[1]}$
	are the alternating words
	$f,g,fg,gf,fgf,gfg,fgfg,gfgf,\cdots$
	in the letters $f$, $g$.
	More generally
	$n$-simplices are given by equipping such alternating words with ``$n$ nested layers of brackets''
	(so that, for example, 
	$\left((f)(gf)\right) 
	\left( (gf) \right)$
	encodes a $2$-simplex).
	Alternatively, given an alternating word of length $l$, such bracketings are encoded by a flag of subsets
	$F_1 \subseteq F_2 \subseteq \cdots 
	\subseteq F_n \subseteq \{1,\cdots,l-1\}$.
	
	To describe $W_{!} J$, we apply the explicit description of the $W_{!} (-)$ construction given in 
	\cite{DS11}.
	Following \cite[Cor. 4.8]{DS11}, the $n$-simplices of $W_{!} J$ are uniquely encoded by a map
\begin{equation}\label{NECMAP EQ}
	N = \Delta^{k_1} \vee \Delta^{k_2} 
	\vee \cdots \vee \Delta^{k_r} 
	\to J 
\end{equation}
which is totally nondegenerate (this means that all simplices $\Delta^{k_i} \to J$ are nondegenerate)
together with a flag of subsets
	$\boldsymbol{J}(N) = 
	G_0 \subseteq 
	G_1 \subseteq \cdots \subseteq
	G_{n-1} \subseteq \boldsymbol{E}^{\mathsf{i}}(N)$.
	Noting that the nondegenerate simplices of $J$ (other than the points $0,1$)
	are themselves identified with alternating words
	$f,g,fg,gf,fgf,gfg,fgfg,gfgf,\cdots$,
	one sees that so is the map \eqref{NECMAP EQ}.
	Therefore, we see that a $n$-simplex of 
	$W_{!} J$ is uniquely, determined by some alternating word of some size $l$ together with a flag  
	$G_0 \subseteq 
	G_1 \subseteq \cdots \subseteq
	G_{n-1} \subseteq \boldsymbol{E}^{\mathsf{i}}(N)
	=\{1,\cdots,l-1\}$, since 
	$G_0$ suffices to recover the domain of 
	\eqref{NECMAP EQ}.
	
	This shows that $W_{!} J$ $\mathbb{F}^{\bullet} \widetilde{[1]}$ indeed have the same simplices.
	The fact that the simplicial operators coincide can be readily checked explicitly with the most interesting case is that of the top differential $d_n$, which in either case is induced by multiplication in 
	$\widetilde{[1]}$ 
	(that this is the case for $W_{!} J$ follows from the description of the simplicial operators in 
	\cite[Cor. 4.4]{DS11} together with the description of the ``flanking'' procedure in \cite[Lemma 4.5]{DS11}).
\end{example}




\subsection{Extra lifts for $\infty$-categories}


\begin{lemma}
The inclusion 
\[[0,1,2] \cup [0,2,3,\cdots,n] \cup 
\Lambda^0[0,1,3,\cdots,n]
 \to \Lambda^{0,2}[n]\]
is built cellularly from inclusions
$\Lambda^0[k] \to \Delta[k]$ with $k<n$.
Moreover, all such cells send $[0,1]$ to $[0,1]$.
\end{lemma}

\begin{proof}
Since $[0,2,3,\cdots,n]$ is in the domain, all missing faces must contain $1$.
Moreover, since the smallest face not containing $2$
that is not in $\Lambda^0[0,1,3,\cdots,n]$ is $[1,3,\cdots,n]$,
which is also the smallest face not in $\Lambda^{0,2}[n]$,
we see that all missing faces must contain $2$ as well.

It now suffices to check that $0$ is characteristic with respect to the missing faces, i.e.
that $12\underline{a}$ is missing iff $012\underline{a}$ is missing, and this is now obvious. 
\end{proof}

\begin{remark}
	The map $\Lambda^{0,2}[n] \to \Delta[n]$ is inner anodyne ($2$ is characteristic).
	
	This observation, together with the previous lemma, are the technical core of the observation that lifts
\[
	\begin{tikzcd}
	\Lambda^{0}[n] \ar{d}  \ar{r} & X
\\
	\Delta[n] \ar[dashed]{ru}
	\end{tikzcd}
\]
exist when $X$ is an $\infty$-category and $[0,1]$ is mapped to an equivalence in $X$.
\end{remark}




\subsection{TBD}


\begin{lemma}
Suppose that a category $\Xi$ has subcategories 
$\Xi^-,\Xi^+$ which contain the isomorphisms and satisfy the unique factorization up to unique isomorphism axiom.

Write $\mathsf{Arr}(\Xi)$ for the arrow category of $\Xi$ and 
$\mathsf{Arr}^{-}(\Xi), \mathsf{Arr}^{+}(\Xi)$
for the full subcategories whose objects are arrows in $\Xi^-,\Xi^+$. 
Then $\mathsf{Arr}^{-}(\Xi)$ (resp. $\mathsf{Arr}^{+}(\Xi)$)
is initial (resp. terminal) in $\mathsf{Arr}(\Xi)$.
\end{lemma}



\begin{proof}
Given $f \in \mathsf{Arr}(\Xi)$, we need to show that there exist diagrams as below, and moreover that all such diagrams are connected. 
Existence follows from the factorization assumption. Moreover, its is straightforward from the ``uniqueness up to isomorphism'' that all connections are connected.
But by factoring the left vertical map in the diagram below, we now see that all such diagrams are connected to a factorization.
\[
\begin{tikzcd}
	\bullet \ar{r}{f} \ar{d}& 
	\bullet \ar{d}
\\
	\bullet \ar{r}[swap]{+} &
	\bullet
\end{tikzcd}
\]
%
%\[
%\begin{tikzcd}
%	\bullet \ar{r}{-} \ar{dd} \ar{rd}[swap]{-} &
%	\bullet \ar{rrr}{+} \ar{rrd}{-} &&&
%	\bullet \ar{dd}
%\\
%	&
%	\bullet \ar{rrd}[swap]{+}  && 
%	\bullet \ar{rd}{+} \ar[dashed]{ll}[swap]{\simeq}
%\\
%	\bullet 	\ar{rrr}[swap]{-} &&&
%	\bullet \ar{r}[swap]{+} & 
%	\bullet
%\end{tikzcd}
%\]
\end{proof}



\begin{lemma}\label{REDUCELAN LEM}
	Suppose that $F \colon \mathcal{C} \to \mathcal{D}$ is a functor in 
	$\mathsf{Cat}^G$.
Then the following square commutes up to natural isomorphism
\[
\begin{tikzcd}[column sep=50pt]
	\mathcal{V}^{G \ltimes \mathcal{C}} 
	\ar{r}{\mathsf{Lan}_{G \ltimes \mathcal{C} \to G \ltimes \mathcal{D}}} \ar{d}[swap]{\mathsf{fgt}}&
	\mathcal{V}^{G \ltimes \mathcal{D}} \ar{d}{\mathsf{fgt}}
\\
	\mathcal{V}^{\mathcal{C}} 
	\ar{r}[swap]{\mathsf{Lan}_{\mathcal{C} \to\mathcal{D}}} &
	\mathcal{V}^{\mathcal{D}}
\end{tikzcd}
\]
\end{lemma}


\begin{proof}
For each $d \in \mathcal{D}$ (recall that $\mathcal{D}$ and $G \ltimes \mathcal{D}$ have the same objects) one has an obvious inclusion 
$\mathcal{C} \downarrow d \to G\ltimes \mathcal{C} \downarrow d$.
Moreover, for each object of $G\ltimes \mathcal{C} \downarrow d$
there is a unique $g \in G$ such that the object is described as a composite
$F(c) \xrightarrow{g} g F(c) \to d$,
were $g F(c) \to d$ can be regarded as an object of $\mathcal{C} \downarrow d$.
One thus has a retraction 
$G \ltimes \mathcal{C} \downarrow d \to \mathcal{C} \downarrow d$
showing that $\mathcal{C} \downarrow d$ is terminal in
$G \ltimes \mathcal{C} \downarrow d$
and finishing the proof. 
\end{proof}




\subsection{The homotopy genuine equivariant operad}


Our goal in this section is to build,
for each $G$-$\infty$-operad $X \in \mathsf{dSet}^G$,
the associated homotopy genuine equivariant operad
$\mathsf{ho} (X)$,
which we will describe as an object in
$\mathsf{dSet}_G$
satisfying a strict Segal condition.


We start with some notation. 
Given a multiset $I$ of edges of a tree $T \in \Omega$
(formally, $I$ is a function 
$I \colon \boldsymbol{E}(T) \to \mathbb{N}_0$),
we write $\sigma^I T \in \Omega$
for the tree obtained by degenerating $T$ once for each edge in $I$.
More explicitly, $\sigma^I T$ is the unique tree such that there is a planar degeneracy
$\pi \colon \sigma^I T \to T$
such that $|\pi^{-1}(e)| = I(e) + 1$.
Moreover,
note that if $T\in \Omega_G$ is a $G$-tree, 
then $\sigma^{I} T \in \Omega_{G}$
can be defined if $I$ is $G$-equivariant
(formally, this means that the multiset $I$ is a composite
$\boldsymbol{E}(T) \to \boldsymbol{E}_G(T)
\to \mathbb{N}_0$).

Our main interest will be in degeneracies of $G$-corollas. Recall that, up to isomorphism, 
a $G$-corolla $C \in \Sigma_G$ is determined the number $0 \leq k$ of leaf orbits
and isotropy subgroups
$H_i \leq H_0 \leq G$ for $0 \leq i \leq k$,
where $H_0$ is the isotropy of a (chosen) root edge.
Pictorially, such a $G$-corolla has the orbital representation given on the left below,
but in this section we will find it more convenient to label edge orbits using coset notation as on the right below,
so that $[e_i] = G e_i$ denotes the $G$-orbit of $e_i$.
\[
\begin{tikzpicture}
[grow=up,auto,level distance=2.3em,every node/.style = {font=\footnotesize},dummy/.style={circle,draw,inner sep=0pt,minimum size=1.75mm}]
	\node at (0,0) [font=\normalsize]{$C$}
		child{node [dummy] {}
			child{
			edge from parent node [swap,near end] {$G/H_k$} node [name=Kn] {}}
			child{
			edge from parent node [near end] {$G/H_1$}
node [name=Kone,swap] {}}
		edge from parent node [swap] {$G/H_0$}
		};
		\draw [dotted,thick] (Kone) -- (Kn) ;
	\node at (5,0) [font=\normalsize]{$C$}
		child{node [dummy] {}
			child{
			edge from parent node [swap,near end] {$[e_k]$} node [name=Kn] {}}
			child{
			edge from parent node [near end] {$[e_1]$}
node [name=Kone,swap] {}}
		edge from parent node [swap] {$[e_0]$}
		};
		\draw [dotted,thick] (Kone) -- (Kn) ;
\end{tikzpicture}
\]
We will then abbreviate $\sigma^i C = \sigma^{[e_i]} C$, and write $e_i$, $e_i'$ for the two edges of $\sigma^i C $ that degenerate the edge $e_i$ of $C$,
with $e_i$ denoting the inner edge and $e'_i$ the outer
edge.
\[
\begin{tikzpicture}
[grow=up,auto,level distance=3em,
every node/.style = {font=\footnotesize},
dummy/.style={circle,draw,inner sep=0pt,minimum size=1.75mm}]
	\node at (0,0) [font=\normalsize]{$\sigma^0 C$}
		child{node [dummy] {}
			child{node [dummy] {}
				child{
				edge from parent node [swap,near end] {$[e_k]$} node [name=Kn] {}}
				child{
				edge from parent node [near end] {$[e_1]$}
node [name=Kone,swap] {}}
			edge from parent node [swap] {$[e_0]$}}
		edge from parent node [swap] {$[e'_0]$}
		};
		\draw [dotted,thick] (Kone) -- (Kn) ;
	\node at (5,0) [font=\normalsize]{$\sigma^i C$}
		child{node [dummy] {}
			child{
			edge from parent node [swap,near end] {$[e_k]$} node [near start,inner sep=1pt,name=Kn] {}}
			child[level distance=3.4em]{node [dummy] {}
				child[level distance=2.7em]{
				edge from parent node [swap] {$[e'_i]$}
}
			edge from parent node [near end,swap] {$[e_i]$}
node [near start,inner sep=1pt,name=Kone,swap] {}
node [near start,inner sep=1pt,name=Kone1] {}}
			child{
			edge from parent node [near end] {$[e_1]$}
node [swap] {}
node [near start,inner sep=1pt,name=Kn1,swap]{}}
		edge from parent node [swap] {$[e_0]$}
		};
		\draw [dotted,thick] (Kone) -- (Kn) ;
		\draw [dotted,thick] (Kone1) -- (Kn1) ;
\end{tikzpicture}
\]
$\sigma^i C$ then has an orbital inner face
$\sigma^i C - [e_i]$ obtained by removing $[e_i]$
as well as an orbital outer face obtained by removing $e'_i$,
which we denote $\sigma^i C - [e'_i]$.
Moreover, note that we have natural identifications
$C = \sigma^i C - [e_i]$,
$C = \sigma^i C - [e'_i]$.

In what follows, we will find it convenient to simplify notation by denoting maps $\Omega[T] \to X$,
where $T \in \Omega_G$ and $X \in \mathsf{dSet}^G$,
simply as $T \to X$.


\begin{definition}
	Let $X \in \mathsf{dSet}^G$ be a $G$-$\infty$-operad and $C$ a $G$-corolla with edge orbits
	$[e_0],\cdots,[e_k]$.
	
	Given two operations 
	$f,g\colon C \rightrightarrows X$,
	we write $f \sim_i g$ if there exists a map
	$H \colon \sigma^i C \to X$ such that
\begin{itemize}
\item $f$ equals the restriction $H|_{\sigma^i C-[e'_i]}$;
\item $g$ equals the restriction $H|_{\sigma^i C-[e_i]}$;
\item the restriction $H|_{\sigma^i [e_i]}$
is the degeneracy $\sigma^i [e_i] \to [e_i] \to C \to X$.
\end{itemize}
\end{definition}


\begin{remark}\label{HOMOTBOUND REM}
	Note that if $f \sim_i g$ then it must be
	$f|_{\partial C} = g|_{\partial C}$.
\end{remark}


\begin{example}\label{EQUIVSIM EX}
	Let $G = \mathbb{Z}_{/2} = \{\pm 1\}$
	and consider the $G$-corolla with orbital and expanded representations as given on the left below.
\[
\begin{tikzpicture}
[grow=up,auto,level distance=2.3em,every node/.style = {font=\footnotesize},dummy/.style={circle,draw,inner sep=0pt,minimum size=1.75mm}]
	\node at (0,0) [font=\normalsize]{$C$}
		child{node [dummy] {}
			child{
			edge from parent node [swap] {$G \cdot e$}
node [name=Kone,swap] {}}
		edge from parent node [swap] {$G/G \cdot r$}
		};
	\node at (3,0) [font=\normalsize]{$C$}
		child{node [dummy] {}
			child{
			edge from parent node [swap,near end] {$-e$} node [name=Kn] {}}
			child{
			edge from parent node [near end] {$e$}
node [name=Kone,swap] {}}
		edge from parent node [swap] {$r$}
		};
	\node at (7,0) [font=\normalsize]{$\sigma^{\{e,-e\}} C$}
		child{node [dummy] {}
			child{node [dummy] {}
				child{
				edge from parent node [swap] {$G \cdot e'$}
node [swap] {}}
			edge from parent node [swap] {$G \cdot e$}
node [swap] {}}
		edge from parent node [swap] {$G/G \cdot r$}
		};
	\node at (10,0) [font=\normalsize]{$\sigma^{\{e,-e\}} C$}
		child{node [dummy] {}
			child{node [dummy] {}
				child{
				edge from parent node [swap] {$-e'$} node {}}
			edge from parent node [swap,near end] {$-e$} node {}}
			child{node [dummy] {}
				child{
				edge from parent node {$e'$}
node [swap] {}}
			edge from parent node [near end] {$e$}
node [swap] {}}
		edge from parent node [swap] {$r$}
		};
\end{tikzpicture}
\]
$C$ then has a single leaf $G$-edge orbit $[e] = G \cdot e$, so that for
$f,g \colon C \to X$ it is
$f \sim_1 g$
if there exists a 
$H \colon \sigma^{\{e,-e\}}C \to X$
such that 
\begin{equation}\label{EQUIVHOMOT EQ}
	f = H|_{\sigma^{\{e,-e\}}C - \{e',-e'\}}
\qquad
	g = H|_{\sigma^{\{e,-e\}}C - \{e,-e\}}
\qquad
	H_{\sigma^e e}, H|_{\sigma^{-e}-e} \text{ are degenerate}.
\end{equation}
It is worthwhile to compare this equivariant relation with the relations obtained if one forgets the $G$-actions. Indeed, while \eqref{EQUIVHOMOT EQ} implicitly assumes that all of $f,g,H$ are $G$-equivariant,
by omitting that assumption one can reinterpret 
\eqref{EQUIVHOMOT EQ}
as defining a relation
$f \sim_{[e]} g$ between not necessarily $G$-equivariant maps $f,g \colon C \to X$.

A priori, $\sim_{[e]}$ relation differs from the 
non-equivariant 
$\sim_{e}$ and $\sim_{-e}$
relations obtained by regarding $C$ as a non-equivariant corolla.
However, for $f,g,H$ as in \eqref{EQUIVHOMOT EQ} one has
\begin{equation}\label{EQUIVSIM EQ}
f = H|_{\sigma^{\{e,-e\}}C - \{e',-e'\}}
\sim_e H|_{\sigma^{\{e,-e\}}C - \{e,-e'\}}
\sim_{-e} H|_{\sigma^{\{e,-e\}}C - \{e,-e\}} =g
\end{equation}
so that, by Lemma \ref{EQUIVI LEM}(b) below one has that
$f \sim_{[e]} g$ in fact implies
$f \sim_{e} g$. Moreover, the converse statement follows immediately by using degeneracies.

More generally, similar considerations show that the $\sim$ relations are compatible with restricting the $G$-actions.
\end{example}


\begin{lemma}\label{EQUIVI LEM}
	Let $X \in \mathsf{dSet}^G$ be a $G$-$\infty$-operad and $C$ a $G$-corolla with edge orbits
	$[e_0],\cdots,[e_k]$. Then:
\begin{itemize}
	\item[(a)] each of the relations $\sim_i$ is an equivalence relation;
	\item[(b)] all the equivalence relations $\sim_i$ coincide.
\end{itemize}
\end{lemma}

\begin{proof}
	We first address (a). 
	
	For the reflexive condition, one can take $H$ to be the degeneracy
	$\sigma^i C \xrightarrow{\sigma^i} C \xrightarrow{f} X$.
	
	For the symmetry and transitive conditions, consider the tree
	$\sigma^{ii} C$, which degenerates $[e_i]$ twice.
\[
\begin{tikzpicture}
[grow=up,auto,level distance=3em,
every node/.style = {font=\footnotesize},
dummy/.style={circle,draw,inner sep=0pt,minimum size=1.75mm}]
	\node at (0,0) [font=\normalsize]{$\sigma^{ii} C$}
		child{node [dummy] {}
			child{
			edge from parent node [swap,near end] {$[e_k]$} node [near start,inner sep=1pt,name=Kn] {}}
			child[level distance=3.4em]{node [dummy] {}
				child[level distance=2.7em]{node [dummy] {}
					child[level distance=2.7em]{
					edge from parent node [swap] {$[e''_i]$}
}
				edge from parent node [swap] {$[e'_i]$}
}
			edge from parent node [near end,swap] {$[e_i]$}
node [near start,inner sep=1pt,name=Kone,swap] {}
node [near start,inner sep=1pt,name=Kone1] {}}
			child{
			edge from parent node [near end] {$[e_1]$}
node [swap] {}
node [near start,inner sep=1pt,name=Kn1,swap]{}}
		edge from parent node [swap] {$[e_0]$}
		};
		\draw [dotted,thick] (Kone) -- (Kn) ;
		\draw [dotted,thick] (Kone1) -- (Kn1) ;
\end{tikzpicture}
\]
Suppose $f \sim_i g$, and let 
$H \colon \sigma^{i} C \to X$ be the associated homotopy.
Define a map 
$\bar{H} \colon \Lambda^{[e_i]}_o[\sigma^{ii} C] \to X$ by
\[
	\bar{H}|_{\sigma^{ii}C - [e''_i]} = H,
		\qquad
	\bar{H}|_{\sigma^{ii}C - [e'_i]} = f \circ \sigma^i,
		\qquad
	\bar{H}|_{\sigma^{ii} [e_i]} = 
	f|_{[e_i]} \circ \sigma^{ii} =
	g|_{[e_i]} \circ \sigma^{ii}.
\]
Since the orbital inner horn inclusion
$\bar{H} \colon \Lambda^{[e_i]}_o[\sigma^{ii} C] \to \Omega[C]$
is $G$-inner anodyne,
$\bar{H}$ admits an extension $\widetilde{H} \colon \sigma^{ii}C \to X$.
The restriction $\bar{H}|_{\sigma^{ii}C - [e_i]}$ then provides the homotopy exhibiting $g \sim_i f$, and symmetry of $\sim_i$ follows.

Next, suppose $f \sim_i g$ and $g \sim_i h$, and let 
$H \colon \sigma^{i} C \to X$ and
$K \colon \sigma^{i} C \to X$ be be the associated homotopies.
Define a map 
$\bar{H} \colon \Lambda^{[e'_i]}_o[\sigma^{ii} C] \to X$ by
\[
	\bar{H}|_{\sigma^{ii}C - [e''_i]} = H,
		\qquad
	\bar{H}|_{\sigma^{ii}C - [e_i]} = K,
		\qquad
	\bar{H}|_{\sigma^{ii} [e_i]} = 
	f|_{[e_i]} \circ \sigma^{ii} =
	g|_{[e_i]} \circ \sigma^{ii} =
	h|_{[e_i]} \circ \sigma^{ii}.
\]
$\bar{H}$ again admits an extension $\widetilde{H} \colon \sigma^{ii}C \to X$, and the restriction $\bar{H}|_{\sigma^{ii}C - [e'_i]}$
provides the homotopy exhibiting $f \sim_i g$, so that transitivity of $\sim_i$.

We next turn to (b). Consider the tree $\sigma^{ij} C$ which degenerates $C$ once along each of $[e_i]$ and $[e_j]$.
\[
\begin{tikzpicture}
[grow=up,auto,level distance=2.75em,
every node/.style = {font=\footnotesize},
dummy/.style={circle,draw,inner sep=0pt,minimum size=1.75mm}]
	\node at (0,0) [font=\normalsize]{$\sigma^{ij} C$}
		child{node [dummy] {}
			child{
			edge from parent node [swap,near end] {$[e_k]$} node [near start,inner sep=1pt,name=Kn] {}}
			child[level distance=3.4em,sibling distance=2em]{node [dummy] {}
				child[level distance=2.7em]{
				edge from parent node [swap] {$[e'_j]$}
}
			edge from parent node [very near end,swap] {$[e_j]$}
node [near start,inner sep=1pt,name=Kone,swap] {}
node [inner sep=1pt,name=Kn2] {}}
			child[level distance=3.4em,sibling distance=2em]{node [dummy] {}
				child[level distance=2.7em]{
				edge from parent node {$[e'_i]$}
}
			edge from parent node [very near end] {$[e_i]$}
node [inner sep=1pt,name=Kone2,swap] {}
node [near start,inner sep=1pt,name=Kone1] {}}
			child{
			edge from parent node [near end] {$[e_1]$}
node [swap] {}
node [near start,inner sep=1pt,name=Kn1,swap]{}}
		edge from parent node [swap] {$[e_0]$}
		};
		\draw [dotted,thick] (Kn) -- (Kone) ;
		\draw [dotted,thick] (Kone1) -- (Kn1) ;
		\draw [dotted,thick] (Kone2) -- (Kn2) ;
\end{tikzpicture}
\]
Suppose $f \sim_i g$ with $H \colon \sigma^{i} C \to X$ the associated homotopy.
Define a map 
$\bar{H} \colon \Lambda^{[e_i]}_o[\sigma^{ij} C] \to X$ by
\[
	\bar{H}|_{\sigma^{ij}C - [e'_j]} = H,
		\qquad
	\bar{H}|_{\sigma^{ij}C - [e_j]} = f \circ \sigma^i,
		\qquad
	\bar{H}|_{\sigma^{ij}C - [e'_i]} = f \circ \sigma^j.
\]
Yet again, $\bar{H}$ admits an extension $\widetilde{H} \colon \sigma^{ij}C \to X$, and the restriction $\bar{H}|_{\sigma^{ij}C - [e_i]}$
provides a homotopy exhibiting $g \sim_j f$. (b) now follows.
\end{proof}

In light of Lemma \ref{EQUIVI LEM},
given $f,g \rightrightarrows C \to X$ with 
$C$ a $G$-corolla and $X$ a $G$-$\infty$-operad,
we will henceforth write $f \sim g$ whenever $f \sim_i g$ for some (and thus all) $i$.
We now extend the $\sim$ relation.

\begin{definition}\label{XTENDSIM DEF}
	Let $T \in \Omega_G$ be a $G$-tree
	and $X \in \mathsf{dSet}^G$ be a 
	$G$-$\infty$-operad.
	
	Given dendrices $x,y\colon T \to X$ we write
	$x \sim y$ if there are equivalences of restrictions
	$x|_{T_v} \sim y|_{T_v}$ for all $G$-vertices
	$v \in \boldsymbol{V}_G(T)$.
	
	Further, we define $\mathsf{ho}(X)(T) = X(T)/\sim$.
\end{definition}

\begin{proposition}
Let $X \in \mathsf{dSet}^G$ be a $G$-$\infty$-operad. Then the assignment 
		$T \mapsto \mathsf{ho}(X)(T)$
		is a contravariant functor in $T \in \Omega_G$, i.e.
		$\mathsf{ho}(X)\in \mathsf{dSet}_G$.
\end{proposition}


\begin{proof}
	It suffices to show that the $\sim$ equivalence relations are compatible with the generating classes of maps in $\Omega_G$, namely
	degeneracies, inner faces, outer faces, and quotient maps.
	
	The cases of degeneracies and outer faces are obvious. In the case of quotients, 
	since any quotient $\bar{T} \to T$ of $G$-trees induces quotients on $G$-vertices, it suffices to consider the case of a quotient
	$\bar{C} \xrightarrow{\pi} C$ of $G$-corollas.
	But it is then straightforward to check that if a homotopy exhibiting $f \sim_0 g$ also induces a homotopy exhibiting 
	$f \circ \pi \sim_0 g \circ \pi$
	(notably, the same needs not be true for the relations $f \sim_i g$ when $0<i$, 
	in which case the exhibiting homotopy 
	may instead exhibit a string of relations 
	$f \circ \pi \sim \cdots \sim g \circ \pi$
	as in \eqref{EQUIVSIM EQ}).

It remains to address the most interesting case,
that inner faces. Since inner faces can be factored as composites of inner faces that collapse a singe inner edge orbit,
it suffices to consider the case of faces
$D \to T$ where $T$ has a single edge edge orbit.
I.e. we can assume that there are $G$-corollas
$C_1$, $C_2$ such that 
$T = C_1 \amalg_{[e_i]} C_2$ and
$D = T - [e_i]$, as illustrated below.
\[
\begin{tikzpicture}
[grow=up,auto,level distance=3em,
every node/.style = {font=\footnotesize},
dummy/.style={circle,draw,inner sep=0pt,minimum size=1.75mm}]
	\node at (0,0) [font=\normalsize]{$C_1$}
		child{node [dummy] {}
			child{
			edge from parent node [swap,near end] {} node [near start,inner sep=1pt,name=Kn] {}}
			child[level distance=3.4em]{node {}
			edge from parent node [near end,swap] {$[e_i]$}
node [near start,inner sep=1pt,name=Kone,swap] {}
node [near start,inner sep=1pt,name=Kone1] {}}
			child{
			edge from parent node [near end] {}
node [swap] {}
node [near start,inner sep=1pt,name=Kn1,swap]{}}
		edge from parent node [swap] {$[e_0]$}
		};
		\draw [dotted,thick] (Kone) -- (Kn) ;
		\draw [dotted,thick] (Kone1) -- (Kn1) ;
	\node at (4,0) [font=\normalsize]{$C_2$}
		child{node [dummy] {}
			child{
			edge from parent node [swap,near end] {} node [name=Kn] {}}
			child{
			edge from parent node [near end] {}
node [name=Kone,swap] {}}
		edge from parent node [swap] {$[e_i]$}
		};
		\draw [dotted,thick] (Kone) -- (Kn) ;
	\node at (9,0) [font=\normalsize]{$T$}
		child{node [dummy] {}
			child{
			edge from parent node [swap,near end] {} node [near start,inner sep=1pt,name=Kn] {}}
			child[level distance=3.4em]{node [dummy] {}
				child{
				edge from parent node [swap,near end] {} node [name=Kn2] {}}
				child{
				edge from parent node [near end] {}
node [name=Kone2,swap] {}}
			edge from parent node [near end,swap] {$[e_i]$}
node [near start,inner sep=1pt,name=Kone,swap] {}
node [near start,inner sep=1pt,name=Kone1] {}}
			child{
			edge from parent node [near end] {}
node [swap] {}
node [near start,inner sep=1pt,name=Kn1,swap]{}}
		edge from parent node [swap] {$[e_0]$}
		};
		\draw [dotted,thick] (Kone) -- (Kn) ;
		\draw [dotted,thick] (Kone1) -- (Kn1) ;
		\draw [dotted,thick] (Kone2) -- (Kn2) ;
\end{tikzpicture}
\]
The claim is now that if
$x,y \colon T \to X$ are such that
$x|_{C_1} \sim y|_{C_1}$ and
$x|_{C_2} \sim y|_{C_2}$
then it is also 
$x|_{D} \sim y|_{D}$.
This will follow from the next two claims:
\begin{itemize}
\item[(i)] if $x,y \colon T \to X$ are such that
$x|_{C_1} = y|_{C_1}$ and
$x|_{C_2} = y|_{C_2}$
then $x|_{D} \sim y|_{D}$;
\item[(ii)]
given $x \colon T \to X$, $f\colon C_1 \to X$ and
$g \colon C_2 \to X$ such that
$f \sim x|_{C_1}$, $g \sim x|_{C_2}$,
there exists
$y \colon T \to X$ such that
$y|_{C_1} = f$, $y|_{C_2} = g$ and
$y|_D = x|_D$.
\end{itemize}
To show (i) and (ii), consider the degeneracies
$\sigma^0 T$ and $\sigma^i T$ pictured below.
\[
\begin{tikzpicture}
[grow=up,auto,level distance=3em,
every node/.style = {font=\footnotesize},
dummy/.style={circle,draw,inner sep=0pt,minimum size=1.75mm}]
	\node at (0,0) [font=\normalsize]{$\sigma^0 T$}
		child{node [dummy] {}
		child{node [dummy] {}
			child{
			edge from parent node [swap,near end] {} node [near start,inner sep=1pt,name=Kn] {}}
			child[level distance=3.4em]{node [dummy] {}
				child{
				edge from parent node [swap,near end] {} node [name=Kn2] {}}
				child{
				edge from parent node [near end] {}
node [name=Kone2,swap] {}}
			edge from parent node [near end,swap] {$[e_i]$}
node [near start,inner sep=1pt,name=Kone,swap] {}
node [near start,inner sep=1pt,name=Kone1] {}}
			child{
			edge from parent node [near end] {}
node [swap] {}
node [near start,inner sep=1pt,name=Kn1,swap]{}}
		edge from parent node [swap] {$[e_0]$}}
		edge from parent node [swap] {$[e'_0]$}
		};
		\draw [dotted,thick] (Kone) -- (Kn) ;
		\draw [dotted,thick] (Kone1) -- (Kn1) ;
		\draw [dotted,thick] (Kone2) -- (Kn2) ;
	\node at (6,0) [font=\normalsize]{$\sigma^i T$}
		child{node [dummy] {}
			child{
			edge from parent node [swap,near end] {} node [near start,inner sep=1pt,name=Kn] {}}
			child[level distance=3.4em]{node [dummy] {}
			child{node [dummy] {}
				child{
				edge from parent node [swap,near end] {} node [name=Kn2] {}}
				child{
				edge from parent node [near end] {}
node [name=Kone2,swap] {}}
			edge from parent node [swap] {$[e'_i]$}}
			edge from parent node [near end,swap] {$[e_i]$}
node [near start,inner sep=1pt,name=Kone,swap] {}
node [near start,inner sep=1pt,name=Kone1] {}}
			child{
			edge from parent node [near end] {}
node [swap] {}
node [near start,inner sep=1pt,name=Kn1,swap]{}}
		edge from parent node [swap] {$[e_0]$}
		};
		\draw [dotted,thick] (Kone) -- (Kn) ;
		\draw [dotted,thick] (Kone1) -- (Kn1) ;
		\draw [dotted,thick] (Kone2) -- (Kn2) ;
\end{tikzpicture}
\]
Given $x,y$ as in (i), one can now build a map
$H \colon \Lambda_o^{[e_i]}[\sigma^0 T] \to X$ by
\[
	H|_{\sigma^0 T - [e_0]} = x,
\qquad
	H|_{\sigma^0 T - [e'_0]} = y,
\qquad
	H|_{\sigma^0 C_1} = 
	x|_{C_1} \circ \sigma^0 = 
	y|_{C_1} \circ \sigma^0.
\]
Letting $\widetilde{H}\colon \sigma^0 T \to X$
be an extension of $H$,
the restriction $H|_{\sigma^0 T - [e_i]}$
provides the desired homotopy 
$x|_{D} \sim y|_{D}$, showing (i).


Lastly, let $x,f,g$ be as in (ii), 
and let
$K \colon \sigma^i C_1 \to X$ exhibit the relation
$f \sim_i x|_{C_1}$
and 
$ \bar{K} \colon \sigma^i C_2 \to X$
exhibit the relation
$x|_{C_2} \sim_i g$ (note the reversed order).
Now build the map
$H \colon \Lambda_o^{[e'_i]}[\sigma^i T] \to X$ by
\[
	H|_{\sigma^i T - [e_i]} = x,
\qquad
	H|_{\sigma^i C_1} = K,
\qquad
	H|_{\sigma^i C_2} = \bar{K}.
\]
Again letting 
$\widetilde{H} \colon \sigma^i T \to X$,
the restriction 
$\widetilde{H}|_{\sigma^i T - [e'_i]}$
provides the required $y \colon T \to X$,
showing (ii) and finishing the proof.
\end{proof}


\begin{corollary}
Let $X \in \mathsf{dSet}^G$ be a $G$-$\infty$-operad. Then
	\begin{itemize}
	\item[(a)] $\mathsf{ho}(X)\in \mathsf{dSet}_G$ is a genuine equivariant operad, i.e. it satisfies the strict right lifting condition against the Segal core inclusions
	$Sc[T] \to \Omega[T]$ for $T \in \Omega_G$;
	\item[(b)] the quotient map
	$\gamma_{\**}X \to \mathsf{ho}(X)$ is the universal map from $\gamma_{\**}X$ to a genuine equivariant operad.
	\end{itemize}
\end{corollary}

\begin{proof}
	Note first that by Remark \ref{HOMOTBOUND REM}
	any map 	$Sc[T] \to \mathsf{ho}(X)$ admits a factorization 
	$Sc[T] \to \gamma_{\**}X \xrightarrow{q} \mathsf{ho}(X)$.
	
	The right lifting property for $\mathsf{ho}(X)$
	against the maps $Sc[T] \to \Omega[T]$
	is then automatic from the lifting property for $X$.

	For strictness,	
	note that Definition \ref{XTENDSIM DEF}
	can be reinterpreted as saying that
	$x,y \colon \Omega[T] \rightrightarrows X$
	give rise to the same point of 
	$\mathsf{ho}(X)$, i.e. 
	the composites 
	$\Omega[T] \rightrightarrows X \xrightarrow{q}
	\mathsf{ho}(X)$ coincide, 
	iff the composites 
	$Sc[T] \to \Omega[T] \rightrightarrows X \xrightarrow{q}
	\mathsf{ho}(X)$ coincide, showing strictness, and thus (a).
		
	For (b), since $\mathsf{ho}(X)$ is a quotient of
	$\gamma_{\**} X$, it suffices to show that any map
	from $F \colon \gamma_{\**}X \to Y$ with $Y$ a genuine equivariant operad must also enforce the $\sim$ relation.
	For a $G$-corolla $C$ and
	$f,g\colon C \to X$ such that 
	$H \colon \sigma^i C \to X$ exhibits
	$f \sim_i g$, 
	the strict lifting condition for $Y$
	shows that the maps
	$F\circ H \colon \sigma^i C \to Y$,
	$f \circ \sigma^i \colon \sigma^i C \to Y$
	must coincide, and thus that
	$F(f)=F(g)$.
	The further claim that $F$ respects equivalences
	of general dendrices $x,y\colon T \rightrightarrows X$
	is immediate from Definition \ref{XTENDSIM DEF}.
\end{proof}











\newpage


% -------------------- APPENDIX --------------------

\appendix


\section{Monad for colored operads}
\label{MONAD_APDX}


Our main goal in this appendix is to establish some necessary technical results concerning the category $\mathsf{Op}(\mathcal{V})$ of colored operads.



\subsection{Colored strings}



Throughout we will let $\Omega$ denote the dendroidal category of trees.
Further, given a tree $T\in \Omega$ we will write 
$E(T)$ for the underlying set of edges.

\begin{definition}
Let $\mathfrak{C}$ be a set of colors.
The category $\Omega_{\mathfrak{C}}$ of $\mathfrak{C}$-colored trees has as objects pairs
$(T,\mathfrak{c}_T\colon E(T) \to \mathfrak{C})$ and a map
$f\colon (T,\mathfrak{c}_T) \to (T',\mathfrak{c}_{T'})$
is a map of underlying trees $f\colon T \to T'$
such that $\mathfrak{c}_T = \mathfrak{c}_{T'} f$.

A map in $\Omega_{\mathfrak{C}}$ is the called \textit{planar/tall} if the underlying map of trees is.

The category $\Omega_{\mathfrak{C}}^n$ of planar strings has as objects strings $T_0 \to T_1 \to \cdots \to T_n$ of maps that are planar and tall and arrows a tuple of compatible isomorphisms of strings.
\end{definition}


The key functors discusses in \cite{BP_geo} then immediately extend to the strings
$\Omega_{\mathfrak{C}}^n$. Namely, one has simplicial operators
\[
d_i \colon \Omega_{\mathfrak{C}}^n \to \Omega_{\mathfrak{C}}^{n-1},
\quad 0 \leq i \leq n
\qquad
s_j \colon \Omega_{\mathfrak{C}}^{n} \to \Omega_{\mathfrak{C}}^{n+1},
\quad -1 \leq j \leq n
\]
which remove (resp. repeat) the $i$-th (resp. $j$-th) tree in the string,
as well as vertex operators
\[
\Omega^{n}_{\mathfrak{C}}
\xrightarrow{\boldsymbol{V}^k}
\Sigma \wr \Omega^{n-k-1}_{\mathfrak{C}}
\]
Lastly, given a map of colors 
$f \colon \mathfrak{C} \to \mathfrak{D}$
one has natural change of color functors
$f_{\**} \colon \Omega^n_{\mathfrak{C}} \to \Omega^n_{\mathfrak{D}}$.



These operators satisfy a number of compatibilities. Firstly, the $d^i$, $s^j$ operators satisfy the usual simplicial identities, 
and the $\boldsymbol{V}^k$ operators are ``additive'' in the sense that
the composite
\begin{equation}\label{VKADD EQ}
	\Omega^{n}_{\mathfrak{C}} \xrightarrow{\boldsymbol{V}^l} 
	\Sigma \wr \Omega^{n-l-1}_{\mathfrak{C}} \xrightarrow{\Sigma \wr \boldsymbol{V}^k}
	\Sigma^{\wr 2} \wr \Omega^{n-k-l-2}_{\mathfrak{C}} \xrightarrow{\sigma^0}
	\Sigma \wr \Omega^{n-k-l-2}_{\mathfrak{C}}.
\end{equation}
equals $\boldsymbol{V}^{k+l+1}$.
The next results list the compatibilities between $d^i$, $s^j$ and $\boldsymbol{V}^k$ operators.

\begin{proposition}\label{CATDIAG PROP}
One has the following diagrams in the $2$-category
$\mathsf{Cat}$.
\begin{itemize}
\item[(i)]
For $0\leq i < k \leq n$ there are $2$-isomorphisms $\pi_{i,k}$ and for $-1 \leq j \leq k \leq n$ there are commutative diagrams
\begin{equation}
\begin{tikzcd}[row sep = tiny, column sep = 35pt]
	\Omega_{\mathfrak{C}}^n
	\arrow{dr}[swap,name=U]{}{V^k} \arrow{dd}[swap]{d^i} &
&
	\Omega_{\mathfrak{C}}^n
	\arrow{dr}{V^k} \arrow{dd}[swap]{s^j} &
\\
	& \Sigma \wr \Omega_{\mathfrak{C}}^{n-k-1}
&
	& \Sigma \wr \Omega_{\mathfrak{C}}^{n-k-1}
\\
	|[alias=V]|
	\Omega_{\mathfrak{C}}^{n-1} \arrow{ur}[swap]{V^{k-1}} &
&
	\Omega_{\mathfrak{C}}^{n+1} \arrow{ur}[swap]{V^{k+1}} &
\arrow[Leftrightarrow, from=V, to=U,shorten >=0.15cm,shorten <=0.15cm
,swap,"\pi_{i,k}"
]
\end{tikzcd}
\end{equation}
\item[(ii)] 
For $-1 \leq k < i \leq n$ and for $-1 \leq k \leq j \leq n$
there are commutative diagrams
\begin{equation}
\begin{tikzcd}[row sep = 10pt, column sep = 35pt]
	\Omega^n_{\mathfrak{C}}
	\arrow{r}[swap,name=U]{}{V^k} \arrow{dd}[swap]{d^i} &
	\Sigma \wr \Omega^{n-k-1}_{\mathfrak{C}} \ar{dd}{d^{i-k-1}}
&
	\Omega^n_{\mathfrak{C}}
	\arrow{r}{V^k} \arrow{dd}[swap]{s^j} &
	\Sigma \wr \Omega^{n-k-1}_{\mathfrak{C}} \ar{dd}{s^{j-k-1}}
\\
\\
	|[alias=V]|
	\Omega^{n-1}_{\mathfrak{C}} \arrow{r}[swap]{V^{k}} &
	\Sigma \wr \Omega^{n-k-2}_{\mathfrak{C}}
&
	\Omega^{n+1}_{\mathfrak{C}} \arrow{r}[swap]{V^{k}} &
	\Sigma \wr \Omega^{n-k}_{\mathfrak{C}}
\end{tikzcd}
\end{equation}
\item[(iii)] 
all $d_i$, $s_j$, $\boldsymbol{V}^k$ and $\pi_{i,k}$
are natural in $\mathfrak{C}$, i.e. for each map of colors
$f \colon \mathfrak{C} \to \mathfrak{D}$ one has commutative diagrams
\[
\begin{tikzcd}[column sep = small, row sep = small]
	\Omega^n_{\mathfrak{C}} \ar{r}{d^i} \ar{dd}[swap]{f_{\**}} &
	\Omega^{n-1}_{\mathfrak{C}} \ar{dd}{f_{\**}}
&
	\Omega^n_{\mathfrak{C}} \ar{r}{s^j} \ar{dd}[swap]{f_{\**}} &
	\Omega^{n+1}_{\mathfrak{C}} \ar{dd}{f_{\**}}
&
	\Omega^n_{\mathfrak{C}} \ar{r}{\boldsymbol{V}^k} \ar{dd}[swap]{f_{\**}} &
	\Sigma \wr \Omega^{n-k-1}_{\mathfrak{C}} \ar{dd}{f_{\**}}
&
	\Omega^n_{\mathfrak{C}}
	\ar{rrrrr}[name=toE]{\boldsymbol{V}^k} \ar{rd}[swap]{d^i} \ar{dd}[swap]{f_{\**}}
	&&&
	&&
	\Sigma \wr \Omega^{n-k-1}_{\mathfrak{C}}  \ar{dd}{f_{\**}}
\\
	&
&
	&
&
	&
&
	&
	|[alias=DBE]|
	\Omega^{n-1}_{\mathfrak{C}} \ar{rrrru}[swap]{\boldsymbol{V}^{k-1}}
\\
	\Omega^n_{\mathfrak{D}} \ar{r}{d^i} &
	\Omega^{n-1}_{\mathfrak{D}}
&
	\Omega^n_{\mathfrak{D}} \ar{r}{s^j} &
	\Omega^n_{\mathfrak{D}}
&
	\Omega^n_{\mathfrak{D}} \ar{r}{\boldsymbol{V}^k} &
	\Sigma \wr \Omega^{n-k-1}_{\mathfrak{D}}
&
	\Omega^n_{\mathfrak{D}} \ar{rrrrr}[name=toB]{\boldsymbol{V}^k} \ar{rd}[swap]{d^i}
	&&&
	&&
	\Sigma \wr \Omega^{n-k-1}_{\mathfrak{D}}
\\
	&
&
	&
&
	&
&
	&
	|[alias=D]| \Omega^{n-1}_{\mathfrak{D}} \ar{rrrru}[swap]{\boldsymbol{V}^{k-1}}
\arrow[Leftrightarrow, from=DBE, to=toE, shorten <=0.15cm,shorten >=0.15cm
,swap,"\pi"
]
	\arrow[Leftrightarrow, from=D, to=toB, shorten <=0.15cm,shorten >=0.15cm,swap,"\pi"]
	\arrow[from=DBE, to=D, crossing over, near start, swap, "f_{\**}"]
\end{tikzcd}
\]
\end{itemize}
Furthermore, the diagrams in (ii) are pullback squares in $\mathsf{Cat}$.
\end{proposition}

The following lists the compatibilities of the $\pi_{i,k}$ isomorphisms, 
which are extensions of the additivity of $\boldsymbol{V}^k$ in \eqref{VKADD EQ} and of the simpicial identities between the $d^i$, $s^j$ operators.



\begin{proposition}\label{CATDIAG2 PROP}
In each of the following items, the two composite natural transformations coincide.
\begin{itemize}
\item[(IT1)]
For $0 \leq i < k $ and $-1 \leq l \leq n-k-1$
\begin{equation}
\begin{tikzcd}[row sep = 20pt, column sep = 25pt]
	|[alias=V]|
	\Omega_{\mathfrak{C}}^{n} \ar{r}{V^{k}}[swap,name=UU]{} \arrow{d}[swap]{d^i}&
	\Sigma \wr \Omega_{\mathfrak{C}}^{n-k-1} \ar{r}{V^l} &
	\Sigma^{\wr 2} \wr \Omega^{n-k-l-2}_{\mathfrak{C}} \ar{r}{\sigma^0} &
	\Sigma \wr \Omega^{n-k-l-2}_{\mathfrak{C}}
&
	\Omega^{n}_{\mathfrak{C}} \ar{r}{V^{k+l+1}}[swap,name=UUU]{} \arrow{d}[swap]{d^i}&
	\Sigma \wr \Omega^{n-k-l-2}_{\mathfrak{C}} &
\\
	|[alias=VV]|
	\Omega^{n-1}_{\mathfrak{C}} \arrow{ur}[swap]{V^{k-1}} & & &
&
	|[alias=VVV]|
	\Omega^{n-1}_{\mathfrak{C}} \arrow{ur}[swap]{V^{k+l}} &
\arrow[Leftrightarrow, from=VV, to=UU,shorten >=0.05cm,shorten <=0.05cm
,swap,"\pi"
]
\arrow[Leftrightarrow, from=VVV, to=UUU,shorten >=0.05cm,shorten <=0.05cm
,swap,"\pi"
]
\end{tikzcd}
\end{equation}

\item[(IT2)]
For $-1 \leq k < i < k + l + 1 \leq n$
\begin{equation}
\begin{tikzcd}[row sep = 20pt, column sep = 25pt]
	\Omega^n_{\mathfrak{C}} \ar{r}{V^k} \ar{d}[swap]{d^i} &
	|[alias=V]|
	\Sigma \wr \Omega^{n-k-1}_{\mathfrak{C}} \ar{r}{V^{l}}[swap,name=UU]{} \arrow{d}[swap]{d^{i-k-1}} &
	\Sigma^{\wr 2} \wr \Omega^{n-k-l-2}_{\mathfrak{C}} \ar{r}{\sigma^0} &
	\Sigma \wr \Omega^{n-k-l-2}_{\mathfrak{C}}
&
	\Omega^{n}_{\mathfrak{C}} \ar{r}{V^{k+l+1}}[swap,name=UUU]{} \arrow{d}[swap]{d^i}&
	\Sigma \wr \Omega^{n-k-l-2}_{\mathfrak{C}} &
\\
	\Omega^{n-1}_{\mathfrak{C}} \ar{r}{V^k} &
	|[alias=VV]|
	\Sigma \wr \Omega^{n-1}_{\mathfrak{C}} \arrow{ur}[swap]{V^{l-1}} & &
&
	|[alias=VVV]|
	\Omega^{n-1}_{\mathfrak{C}} \arrow{ur}[swap]{V^{k+l}} &
\arrow[Leftrightarrow, from=VV, to=UU,shorten >=0.05cm,shorten <=0.05cm
,swap,"\pi"
]
\arrow[Leftrightarrow, from=VVV, to=UUU,shorten >=0.05cm,shorten <=0.05cm
,swap,"\pi"
]
\end{tikzcd}
\end{equation}
\item[(FF1)]
For $0 \leq i < i' < k \leq n$
\begin{equation}
\begin{tikzcd}[row sep = 20pt, column sep = 35pt]
	\Omega^n_{\mathfrak{C}}
	\arrow{dr}[swap,name=U]{}{V^k} \arrow{d}[swap]{d^{i'}} &
&
	\Omega^n_{\mathfrak{C}}
	\arrow{dr}[swap,name=UUU]{}{V^k} \arrow{d}[swap]{d^i} &
\\
	|[alias=V]|
	\Omega^{n-1}_{\mathfrak{C}} \ar{r}[near start,swap]{V^{k-1}}[swap,name=UU]{} \arrow{d}[swap]{d^i}&
	\Sigma \wr \Omega^{n-k-1}_{\mathfrak{C}}
&
	|[alias=VVV]|
	\Omega^{n-1}_{\mathfrak{C}} \ar{r}[near start, swap]{V^{k-1}}[swap,name=UUUU]{} \ar{d}[swap]{d^{i'-1}} &
	\Sigma \wr \Omega^{n-k-1}_{\mathfrak{C}}
\\
	|[alias=VV]|
	\Omega^{n-2}_{\mathfrak{C}} \arrow{ur}[swap]{V^{k-2}} &
&
	|[alias=VVVV]|
	\Omega^{n-2}_{\mathfrak{C}} \arrow{ur}[swap]{V^{k-2}} &
\arrow[Leftrightarrow, from=V, to=U,shorten >=0.05cm,shorten <=0.05cm
,swap,"\pi"
]
\arrow[Leftrightarrow, from=VV, to=UU,shorten >=0.25cm,shorten <=0.05cm
,swap,"\pi"
]
\arrow[Leftrightarrow, from=VVV, to=UUU,shorten >=0.05cm,shorten <=0.05cm
,swap,"\pi"
]
\arrow[Leftrightarrow, from=VVVV, to=UUUU,shorten >=0.25cm,shorten <=0.05cm
,swap,"\pi"
]
\end{tikzcd}
\end{equation}
\item[(FF2)]
For $0 \leq i < k < i' \leq n$
\begin{equation}
\begin{tikzcd}[row sep = 20pt, column sep = 35pt]
	\Omega^n_{\mathfrak{C}}
	\arrow{r}[swap,name=U]{}{V^k} \arrow{d}[swap]{d^{i'}} &
	\Sigma \wr \Omega^{n-k-1}_{\mathfrak{C}} \ar{d}{d^{i'-k-1}}
&
	\Omega^n_{\mathfrak{C}}
	\arrow{dr}[swap,name=UUU]{}{V^k} \arrow{d}[swap]{d^i} &
\\
	|[alias=V]|
	\Omega^{n-1}_{\mathfrak{C}} \ar{r}{V^{k}}[swap,name=UU]{} \arrow{d}[swap]{d^i}&
	\Sigma \wr \Omega^{n-k-2}_{\mathfrak{C}}
&
	|[alias=VVV]|
	\Omega^{n-1}_{\mathfrak{C}} \ar{r}[near start, swap]{V^{k-1}}[swap,name=UUUU]{} \ar{d}[swap]{d^{i'-1}} &
	\Sigma \wr \Omega^{n-k-1}_{\mathfrak{C}} \ar{d}{d^{i'-k-1}}
\\
	|[alias=VV]|
	\Omega^{n-2}_{\mathfrak{C}} \arrow{ur}[swap]{V^{k-1}} &
&
	|[alias=VVVV]|
	\Omega^{n-2}_{\mathfrak{C}} \ar{r}[swap]{V^{k-1}} &
	\Sigma \wr \Omega^{n-k-2}_{\mathfrak{C}}
\arrow[Leftrightarrow, from=VV, to=UU,shorten >=0.05cm,shorten <=0.05cm
,swap,"\pi"
]
\arrow[Leftrightarrow, from=VVV, to=UUU,shorten >=0.05cm,shorten <=0.05cm
,swap,"\pi"
]
\end{tikzcd}
\end{equation}
\item[(DF1)]
For 
%$0 \leq j+1 < i < k +1 \leq n +1$ or 
$-1 \leq j < i \leq k \leq n$
\begin{equation}
\begin{tikzcd}[row sep = 20pt, column sep = 35pt]
	\Omega^{n}_{\mathfrak{C}}
	\arrow{dr}[swap,name=U]{}{V^{k}} \arrow{d}[swap]{s^j} &
&
	\Omega^{n}_{\mathfrak{C}}
	\arrow{dr}[swap,name=UUU]{}{V^{k}} \arrow{d}[swap]{d^{i-1}} &
\\
	|[alias=V]|
	\Omega^{n+1}_{\mathfrak{C}} \ar{r}{V^{k+1}}[swap,name=UU]{} \arrow{d}[swap]{d^i}&
	\Sigma \wr \Omega^{n-k-1}
&
	|[alias=VVV]|
	\Omega^{n-1}_{\mathfrak{C}} \ar{r}[near start, swap]{V^{k-1}}[swap,name=UUUU]{} \ar{d}[swap]{s^j} &
	\Sigma \wr \Omega^{n-k-1}
\\
	|[alias=VV]|
	\Omega^{n}_{\mathfrak{C}} \arrow{ur}[swap]{V^{k}} &
&
	|[alias=VVVV]|
	\Omega^{n}_{\mathfrak{C}} \arrow{ur}[swap]{V^{k}} &
\arrow[Leftrightarrow, from=VV, to=UU,shorten >=0.25cm,shorten <=0.05cm
,swap,"\pi"
]
\arrow[Leftrightarrow, from=VVV, to=UUU,shorten >=0.05cm,shorten <=0.05cm
,swap,"\pi"
]
\end{tikzcd}
\end{equation}
\item[(DF2)]
For $0 \leq j+1 = i \leq k \leq n$ or 
$0 \leq j = i \leq k \leq n$
\begin{equation}
\begin{tikzcd}[row sep = 20pt, column sep = 35pt]
	\Omega^n_{\mathfrak{C}}
	\arrow{dr}[swap,name=U]{}{V^k} \arrow{d}[swap]{s^j} &
&
	\Omega^n_{\mathfrak{C}}
	\arrow{dr}[swap,name=UUU]{}{V^k} \arrow[equal]{dd} &
\\
	|[alias=V]|
	\Omega^{n+1}_{\mathfrak{C}} \ar{r}{V^{k+1}}[swap,name=UU]{} \arrow{d}[swap]{d^i}&
	\Sigma \wr \Omega^{n-k-1}_{\mathfrak{C}}
&
	&
	\Sigma \wr \Omega^{n-k-1}_{\mathfrak{C}}
\\
	|[alias=VV]|
	\Omega^{n}_{\mathfrak{C}} \arrow{ur}[swap]{V^k} &
&
	|[alias=VVVV]|
	\Omega^{n}_{\mathfrak{C}} \arrow{ur}[swap]{V^k} &
\arrow[Leftrightarrow, from=VV, to=UU,shorten >=0.25cm,shorten <=0.05cm
,swap,"\pi"
]
\end{tikzcd}
\end{equation}
\item[(DF3)]
For $0\leq i < j \leq k \leq n$
\begin{equation}
\begin{tikzcd}[row sep = 20pt, column sep = 35pt]
	\Omega^n_{\mathfrak{C}}
	\arrow{dr}[swap,name=U]{}{V^k} \arrow{d}[swap]{s^j} &
&
	\Omega^n_{\mathfrak{C}}
	\arrow{dr}[swap,name=UUU]{}{V^k} \arrow{d}[swap]{d^{i}} &
\\
	|[alias=V]|
	\Omega^{n+1}_{\mathfrak{C}} \ar{r}{V^{k+1}}[swap,name=UU]{} \arrow{d}[swap]{d^i}&
	\Sigma \wr \Omega^{n-k-1}_{\mathfrak{C}}
&
	|[alias=VVV]|
	\Omega^{n-1}_{\mathfrak{C}} \ar{r}[near start, swap]{V^{k-1}}[swap,name=UUUU]{} \ar{d}[swap]{s^{j-1}} &
	\Sigma \wr \Omega^{n-k-1}_{\mathfrak{C}}
\\
	|[alias=VV]|
	\Omega^{n}_{\mathfrak{C}} \arrow{ur}[swap]{V^k} &
&
	|[alias=VVVV]|
	\Omega^{n}_{\mathfrak{C}} \arrow{ur}[swap]{V^k} &
\arrow[Leftrightarrow, from=VV, to=UU,shorten >=0.25cm,shorten <=0.05cm
,swap,"\pi"
]
\arrow[Leftrightarrow, from=VVV, to=UUU,shorten >=0.05cm,shorten <=0.05cm
,swap,"\pi"
]
\end{tikzcd}
\end{equation}
\item[(DF4)]
For $0 \leq i < k \leq j \leq n$
\begin{equation}
\begin{tikzcd}[row sep = 20pt, column sep = 35pt]
	\Omega^n_{\mathfrak{C}}
	\arrow{r}[swap,name=U]{}{V^k} \arrow{d}[swap]{s^j} &
	\Sigma \wr \Omega^{n-k-1}_{\mathfrak{C}} \ar{d}{s^{j-k-1}}
&
	\Omega^n_{\mathfrak{C}}
	\arrow{dr}[swap,name=UUU]{}{V^k} \arrow{d}[swap]{d^i} &
\\
	|[alias=V]|
	\Omega^{n+1}_{\mathfrak{C}} \ar{r}{V^{k}}[swap,name=UU]{} \arrow{d}[swap]{d^i}&
	\Sigma \wr \Omega^{n-k}_{\mathfrak{C}}
&
	|[alias=VVV]|
	\Omega^{n-1}_{\mathfrak{C}} \ar{r}[near start, swap]{V^{k-1}}[swap,name=UUUU]{} \ar{d}[swap]{s^{j-1}} &
	\Sigma \wr \Omega^{n-k-1}_{\mathfrak{C}} \ar{d}{s^{j-k-1}}
\\
	|[alias=VV]|
	\Omega^{n}_{\mathfrak{C}} \arrow{ur}[swap]{V^{k-1}} &
&
	|[alias=VVVV]|
	\Omega^{n}_{\mathfrak{C}} \ar{r}[swap]{V^{k-1}} &
	\Sigma \wr \Omega^{n-k}_{\mathfrak{C}}
\arrow[Leftrightarrow, from=VV, to=UU,shorten >=0.05cm,shorten <=0.05cm
,swap,"\pi"
]
\arrow[Leftrightarrow, from=VVV, to=UUU,shorten >=0.05cm,shorten <=0.05cm
,swap,"\pi"
]
\end{tikzcd}
\end{equation}
\end{itemize}
\end{proposition}




\subsection{The $(-)\wr A$ construction}\label{WRACONST SEC}



One of the key ideas used in \cite{BP_geo} when describing the monad on spans was the use of categories 
$\Omega^n \wr A$ defined by pullbacks diagrams of the form
\begin{equation}\label{WRASAMPLE EQ}
\begin{tikzcd}
	\Omega^n \wr A \ar{r}{\boldsymbol{V}^n} \ar{d} &
	\Sigma \wr A  \ar{d}
\\
	\Omega^n \ar{r}{\boldsymbol{V}^n} &
	\Sigma \wr \Sigma
\end{tikzcd}
\end{equation}
Moreover, these categories are related by analogues of the operators $d^i$, $s^j$, $\boldsymbol{V}^k$, $\pi_{i,k}$
which satisfy all the analogues of the compatibilities 
listed in Propositions \ref{CATDIAG PROP} and \ref{CATDIAG2 PROP}.

In \cite{BP_geo} these analogue operators were built via a somewhat adhoc method, but here we will prefer a more systematic approach which regards the $(-) \wr A$ construction as a sort of $2$-categorical pullback functor. We first introduce the relevant $2$-categories.


\begin{definition}
Let $\mathcal{E} \to \mathcal{B}$ be a split Grothendieck fibration.
We write $\mathsf{Cat}\downarrow^r_{\mathcal{B}} \mathcal{E}$ for the $2$-category such that:
\begin{itemize}
	\item objects are functors $F \colon \mathcal{C} \to \mathcal{E}$; 
	
	\item an $1$-arrow from 
	$F \colon \mathcal{C} \to \mathcal{E}$
	to
	$F' \colon \mathcal{C}' \to \mathcal{E}$
	is a pair $(f,\phi)$
	formed by a functor $f\colon \mathcal{C} \to \mathcal{C}'$ and a natural transformation $\phi \colon F' f \Rightarrow F$ consisting of pullback arrows over $\mathcal{B}$
		\begin{equation}
		\begin{tikzcd}[row sep = tiny, column sep = 35pt]
			\mathcal{C} \arrow{dr}[name=U]{F} \arrow{dd}[swap]{f}
		\\
			& \mathcal{E}
		\\
			|[alias=V]| \mathcal{C}' \arrow{ur}[swap]{F'}
		\arrow[Rightarrow, from=V, to=U,shorten >=0.25cm,shorten <=0.25cm
		,swap,"\phi"
		]
		\end{tikzcd}
		\end{equation}
	\item a $2$-arrow from $(f,\phi)$ to $(f',\phi')$ is a $2$-arrow $\varphi \colon f \to f'$ such that
	$\phi' \circ F' \varphi = \phi$.
		\begin{equation}
		\begin{tikzcd}[column sep = 50pt]
			\mathcal{C} \arrow{dr}[name=U]{F} 
			\arrow[bend right]{dd}[swap]{f}[name=F]{}
			\arrow[bend left]{dd}{f'}[swap,name=FF]{}
			&
		&
			\mathcal{C} \arrow{dr}[name=U2]{F} 
			\arrow[bend right]{dd}[swap]{f}
			&
		\\
			& \mathcal{E}
		&
			& \mathcal{E}
		\\
			\mathcal{C}' \arrow{ur}[swap]{F'}[near start, name=V]{}
			&
		&
			|[alias=V2]| \mathcal{C}' \arrow{ur}[swap]{F'}
			&
		\arrow[Rightarrow, from=V, to=U,shorten >=0.25cm,shorten <=0.25cm
		,swap,"\phi'"
		]
		\arrow[Rightarrow, from=F, to=FF,shorten >=0.0cm,shorten <=0.0cm
		,swap,"\varphi"
		]
		\arrow[Rightarrow, from=V2, to=U2,shorten >=0.25cm,shorten <=0.25cm
		,swap,"\phi"
		]
		\end{tikzcd}
		\end{equation}
\end{itemize}
\end{definition}

Given a map $\rho \colon \mathcal{E} \to \mathcal{F}$
of split Grothendieck fibrations over $\mathcal{B}$,
we now define a pullback $2$-functor on weak right spans
\begin{equation}\label{WSPANPULL EQ}
\rho^{\**} \colon
\mathsf{Cat} \downarrow^r_\mathcal{B} \mathcal{F} 
	\to
\mathsf{Cat} \downarrow^r_\mathcal{B} \mathcal{E}.
\end{equation}

On objects, i.e. functors $F \colon \mathcal{C} \to \mathcal{F}$, one sets 
$\rho^{\**}(\mathcal{C} \to \mathcal{F})=
(\mathcal{C} \times_{\mathcal{F}} \mathcal{E}
\to \mathcal{E})
$.

On $1$-arrows, i.e. pairs 
$(f,\phi \colon F_2 \circ f \Rightarrow F_1)$
as in the bottom of the diagram below
\[
\begin{tikzcd}[column sep = 20pt, row sep = small]
	\mathcal{C}_1 \times_{\mathcal{F}} \mathcal{E} 
	\ar{rrrrr}[name=toE]{}[near end]{E_1} \ar[dashed]{rd}[swap]{\bar{f}} \ar{dd}[swap]{\pi}
	&&&
	&&
	\mathcal{E}  \ar{dd}{\rho}
\\
	&
	|[alias=DBE]|
	\mathcal{C}_2 \times_{\mathcal{F}} \mathcal{E} \ar{rrrru}[swap]{E_2}
\\
	\mathcal{C}_1 \ar{rrrrr}[name=toB]{}[near end]{F_1} \ar{rd}[swap]{f}
	&&&
	&&
	\mathcal{F} 
\\
	&
	|[alias=D]| \mathcal{C}_2 \ar{rrrru}[swap]{F_2}
\arrow[Rightarrow, from=DBE, to=toE, shorten <=0.15cm,shorten >=0.15cm,dashed
,swap,"\bar{\phi}"
]
	\arrow[Rightarrow, from=D, to=toB, shorten <=0.15cm,shorten >=0.15cm,swap,"\phi"]
	\arrow[from=DBE, to=D, crossing over, near start, "\pi"]
\end{tikzcd}
\]
we define $\rho^{\**}(f,\phi)$ as the only possible choice of dashed data
$(\bar{f},\bar{\phi})$
such that $\bar{\phi}$ consists of pullback arrows over $\mathcal{B}$
and the diagram commutes in the sense that
$\pi \bar{f} = f \pi$ and 
$\rho \bar{\phi} = \phi \pi$.
%\begin{equation}
%\begin{tikzcd}[row sep = tiny, column sep = 35pt]
%	\mathcal{C}_1 \arrow{dr}[name=U]{F_1} \arrow{dd}[swap]{f}
%\\
%	& \mathcal{F}
%\\
%	|[alias=V]| \mathcal{C}_2 \arrow{ur}[swap]{F_2}
%\arrow[Rightarrow, from=V, to=U,shorten >=0.25cm,shorten <=0.25cm
%,swap,"\phi"
%]
%\end{tikzcd}
%\end{equation}
Alternatively, one has the explicit formula
\[
\pi^{\**} (f,\phi)=
\left(
	\left( f \pi,
	\left( \phi \pi \right)^{\**} E_1 \right),
	\left( \phi \pi \right)^{\**} E_1 \Rightarrow E_1
\right)
\]

%writing 
%$\pi \colon \mathcal{C}_i \times_{\mathcal{B}} \mathcal{E}
%\to \mathcal{C}_i$
%and
%$E_i \colon \mathcal{C}_i \times_{\mathcal{B}} \mathcal{E}
%\to \mathcal{E}$
%for the projections



Lastly, on a $2$-arrow $\varphi \colon (f,\phi) \Rightarrow (f',\phi')$
as on the bottom of the leftmost diagram below
\begin{equation}\label{PULL2ARR EQ}
\begin{tikzcd}[column sep = 16pt, row sep = 17pt]
	\mathcal{C}_1 \times_{\mathcal{F}} \mathcal{E} 
	\ar{rrrrr}[name=toE]{}[near end]{E_1} 
	\ar[bend left]{rd}[near start,swap,name=FE]{}
	\ar[bend right]{rd}[name=FFE]{} \ar{dd}[swap]{\pi}
	&&&
	&&
	\mathcal{E}  \ar{dd}
&&
	\mathcal{C}_1 \times_{\mathcal{F}} \mathcal{E} 
	\ar{rrrrr}[name=toE2]{}[near end]{E_1} 
	\ar[bend right]{rd}{} \ar{dd}
	&&&
	&&
	\mathcal{E}  \ar{dd}
\\
	&
	|[alias=DBE]|
	\mathcal{C}_2 \times_{\mathcal{F}} \mathcal{E} \ar{rrrru}[swap]{E_2} &&&&
&&
	&
	|[alias=DBE2]|
	\mathcal{C}_2 \times_{\mathcal{F}} \mathcal{E} \ar{rrrru}[swap]{E_2} &&&&
\\
	\mathcal{C}_1 \ar{rrrrr}[name=toB]{}[near end]{F_1} 
	\ar[bend left]{rd}[swap,name=FF]{}
	\ar[bend right]{rd} [name=F]{}
	&&&
	&&
	\mathcal{F} 
&&
	\mathcal{C}_1 \ar{rrrrr}[name=toB2]{}[near end]{F_1} 
	\ar[bend right]{rd}{}
	&&&
	&&
	\mathcal{F} 
\\
	&
	|[alias=D]| \mathcal{C}_2 \ar{rrrru}[swap]{F_2} &&&&
&&
	&
	|[alias=D2]|
	\mathcal{C}_2 \ar{rrrru}[swap]{F_2} &&&&
\arrow[Rightarrow, from=DBE, to=toE, shorten <=0.15cm,shorten >=0.15cm
,swap,"\bar{\phi}'"
]
\arrow[Rightarrow, from=DBE2, to=toE2, shorten <=0.15cm,shorten >=0.15cm
,swap,"\bar{\phi}"
]
\arrow[Rightarrow, from=D, to=toB, shorten <=0.15cm,shorten >=0.15cm,swap,"\phi'"]
\arrow[Rightarrow, from=D2, to=toB2, shorten <=0.15cm,shorten >=0.15cm,swap,"\phi"]
\arrow[Rightarrow, from=F, to=FF, shorten <=0cm,shorten >=0cm,swap,"\varphi"]
\arrow[Rightarrow, from=FFE, to=FE, shorten <=0cm,shorten >=0cm,swap,dashed,"\bar{\varphi}"]
\arrow[from=DBE, to=D, crossing over,near start,"\pi"]
\arrow[from=DBE2, to=D2, crossing over]
\end{tikzcd}
\end{equation}
we define $\rho^{\**}(\varphi)$
as the only choice of dashed $\bar{\varphi}$
such that $\bar{\phi}' \circ E_2\bar{\varphi} = \bar{\phi}$
and $\pi \bar{\varphi} = \varphi \pi$.

%one sets $\pi^{\**} \varphi (c,b,e)$ to be the unique dashed arrow in the left diagram below that lifts $\varphi(c)$.
%\[
%\begin{tikzcd}
%	\left(\phi(c)\right)^{\**} e \ar{rr} \ar[dashed]{rd} &&
%	e
%&
%	B_2 f(c) \ar{rr}{\phi(c)} \ar{rd}[swap]{\varphi(c)} &&
%	b
%\\
%	& \left(\phi'(c)\right)^{\**} e \ar{ru} &
%&
%	& B_2 f'(c) \ar{ru}[swap]{\phi'(c)} &
%\end{tikzcd}
%\]


%The associativity and unitality conditions of $\rho^{\**}$ are straightforward.



%Alternatively, $\pi^{\**}\varphi$ this is the unique dashed natural transformation in the left diagram below such that the left section commutes 
%(meaning that the two natural transformations between the two functors
%$\mathcal{C}_1 \times_{\mathcal{B}} \mathcal{E} 
%\rightrightarrows \mathcal{C}_2$ coincide) and 
%the top composite natural transformation is 
%$\pi^{\**} \phi$.



We are now ready to generalize the $(-) \wr A$
construction from \eqref{WRASAMPLE EQ}.

First, note that using the functor
$\boldsymbol{V}^n \colon \Omega^n_{\mathfrak{C}} \to \Sigma \wr \Sigma_{\mathfrak{C}}$
the categories 
$\Omega^n_{\mathfrak{C}}$ are extended to an object in
$\mathsf{Cat} \downarrow^r_{\Sigma} \Sigma \wr \Sigma_{\mathfrak{C}}$.
Hence, given a functor $A \to \Sigma_{\mathfrak{C}}$
we define 
\begin{equation}\label{WRADEF EQ}
(-) \wr A \colon 
\mathsf{Cat} \downarrow^r_{\Sigma} \Sigma \wr \Sigma_{\mathfrak{C}}
\to
\mathsf{Cat} \downarrow^r_{\Sigma} \Sigma \wr A
\end{equation}
as the pullback \eqref{WSPANPULL EQ} for the map
$\Sigma \wr A \to \Sigma \wr \Sigma_{\mathfrak{C}}$.

As a result, one obtains categories $\Omega_{\mathfrak{C}}^n \wr A$
together with maps 
$\Omega_{\mathfrak{C}}^n \wr A 
\xrightarrow{\boldsymbol{V}^n} \Sigma \wr A$
and simplicial operators $d^i$, $s^j$ between them.
To further obtain vertex functors
$\boldsymbol{V}^k \colon \Omega^n_{\mathfrak{C}} \wr A
\to 
\Sigma \wr \left(\Omega^{n-k-1}_{\mathfrak{C}} \wr A \right)$
we first note that the $\Sigma \wr (-)$ operation can be extended to a $2$-endofunctor
\[
\begin{tikzcd}[row sep = 0pt]
	\mathsf{Cat} \downarrow^r_{\Sigma} \Sigma \wr A \ar{r}{\Sigma \wr (-)} &
	\mathsf{Cat} \downarrow^r_{\Sigma} \Sigma \wr A
\\
	\mathcal{C} \to \Sigma \wr A \ar[mapsto]{r} &
	\Sigma \wr \mathcal{C} \to \Sigma^{\wr 2} \wr A \to \Sigma \wr A 
\end{tikzcd}
\]
from which it follows that one has natural identifications
$\Sigma \wr \left(\Omega^{n}_{\mathfrak{C}} \wr A \right)
\simeq 
\left(\Sigma \wr \Omega^{n}_{\mathfrak{C}}\right) \wr A $
which are readily seen to be compatible with the cosimplicial operators on $\Sigma^{\wr k} \wr (-)$.
As such, we will henceforth suppress parenthesis and write 
simply 
$\Sigma \wr \Omega^{n}_{\mathfrak{C}} \wr A$
to denote 
$\Sigma \wr \left(\Omega^{n}_{\mathfrak{C}} \wr A \right)$,
so that the $2$-functor $(-)\wr A$ in \eqref{WRADEF EQ}
yields further vertex functors 
$\boldsymbol{V}^k \colon \Omega^n_{\mathfrak{C}} \wr A
\to 
\Sigma \wr \Omega^{n-k-1}_{\mathfrak{C}} \wr A$
and natural transformations $\pi_{i,k}$
satisfying all the analogues of the compatibility conditions
in Proposition \ref{CATDIAG PROP}(i)(ii) and Proposition \ref{CATDIAG2 PROP}.

The analogue of Proposition \ref{CATDIAG PROP}(iii) requires an extra argument, and is stated in the following result.



\begin{proposition}\label{SPANPIECE PROP}
A commutative square
\begin{equation}\label{SPANPIECE EQ}
\begin{tikzcd}
	A \ar{d} \ar{r}{f} &  \ar{d} B
\\
	\Sigma_{\mathfrak{C}} \ar{r}[swap]{f} & \Sigma_{\mathfrak{D}}
\end{tikzcd}
\end{equation}
induces natural maps 
$f_{\**} \colon
\Omega_{\mathfrak{C}}^n \wr A \to 
\Omega_{\mathfrak{D}}^n \wr B $
such that the diagrams below commute.
\[
\begin{tikzcd}[column sep = 6pt, row sep = small]
	\Omega^n_{\mathfrak{C}} \wr A \ar{r}{d^i} \ar{dd}[swap]{f_{\**}} &
	\Omega^{n-1}_{\mathfrak{C}} \wr A \ar{dd}{f_{\**}}
&
	\Omega^n_{\mathfrak{C}} \wr A \ar{r}{s^j} \ar{dd}[swap]{f_{\**}} &
	\Omega^{n+1}_{\mathfrak{C}}\wr A \ar{dd}{f_{\**}}
&
	\Omega^n_{\mathfrak{C}} \wr A \ar{r}{\boldsymbol{V}^k} \ar{dd}[swap]{f_{\**}} &
	\Sigma \wr \Omega^{n-k-1}_{\mathfrak{C}} \wr A \ar{dd}{f_{\**}}
&
	\Omega^n_{\mathfrak{C}} \wr A
	\ar{rrrrr}[name=toE]{\boldsymbol{V}^k} \ar{rd}[swap]{d^i} \ar{dd}[swap]{f_{\**}}
	&&&
	&&
	\Sigma \wr \Omega^{n-k-1}_{\mathfrak{C}} \wr A  \ar{dd}{f_{\**}}
\\
	&
&
	&
&
	&
&
	&
	|[alias=DBE]|
	\Omega^{n-1}_{\mathfrak{C}} \wr A \ar{rrrru}[swap]{\boldsymbol{V}^{k-1}}
\\
	\Omega^n_{\mathfrak{D}} \wr B \ar{r}{d^i} &
	\Omega^{n-1}_{\mathfrak{D}} \wr B
&
	\Omega^n_{\mathfrak{D}} \wr B \ar{r}{s^j} &
	\Omega^n_{\mathfrak{D}} \wr B
&
	\Omega^n_{\mathfrak{D}} \wr B \ar{r}{\boldsymbol{V}^k} &
	\Sigma \wr \Omega^{n-k-1}_{\mathfrak{D}} \wr B
&
	\Omega^n_{\mathfrak{D}} \wr B \ar{rrrrr}[name=toB]{\boldsymbol{V}^k} \ar{rd}[swap]{d^i}
	&&&
	&&
	\Sigma \wr \Omega^{n-k-1}_{\mathfrak{D}} \wr B
\\
	&
&
	&
&
	&
&
	&
	|[alias=D]| \Omega^{n-1}_{\mathfrak{D}} \wr B \ar{rrrru}[swap]{\boldsymbol{V}^{k-1}}
\arrow[Leftrightarrow, from=DBE, to=toE, shorten <=0.15cm,shorten >=0.15cm
,swap,"\pi"
]
	\arrow[Leftrightarrow, from=D, to=toB, shorten <=0.15cm,shorten >=0.15cm,swap,"\pi"]
	\arrow[from=DBE, to=D, crossing over, near start, swap, "f_{\**}"]
\end{tikzcd}
\]
\end{proposition}


\begin{proof}
The desired map 
$f_{\**} \colon
\Omega_{\mathfrak{C}}^n \wr A \to 
\Omega_{\mathfrak{D}}^n \wr B $
is obtained immediately by drawing the pullback diagrams defining each term, so we focus on the slightly more interesting claim that the given diagrams commute.
To see this, we first factor \eqref{SPANPIECE EQ} as
\[
\begin{tikzcd}
	A \ar{d} \ar{r} & B \times_{\Sigma_{\mathfrak{D}}} \Sigma_{\mathfrak{C}} \ar{r} \ar{d} &  \ar{d} B
\\
	\Sigma_{\mathfrak{C}} \ar[equal]{r} & \Sigma_{\mathfrak{C}} \ar{r} & \Sigma_{\mathfrak{D}}
\end{tikzcd}
\]
and note that it suffices to prove the result separately for each half.
For the left half, the desired commutativity claims are simply the functoriality of $(-) \wr A$. On the other hand, for the right square the commutativity claims follow by instead noting that all diagrams in 
Proposition \ref{CATDIAG PROP}(iii)
can be regarded as diagrams in the $2$-category
$\mathsf{Cat} \downarrow^r_{\Sigma} \Sigma \wr \Sigma_{\mathfrak{D}}$
(by using the composites 
$\Omega_{\mathfrak{C}}^n \xrightarrow{f_{\**}} \Omega_{\mathfrak{D}}^n \xrightarrow{\boldsymbol{V}^n} \Sigma \wr \Sigma_{\mathfrak{D}}$) 
and then applying the pullback functor $(-) \wr B$. 
\end{proof}



Next, note that thanks to the composite functors 
$\Omega^n_{\mathfrak{C}} \wr A \to \Omega^n_{\mathfrak{C}} 
\xrightarrow{d^{0,\cdots,n}} \Sigma_{\mathfrak{C}}$
one can regard the $\Omega_{\mathfrak{C}}^n \wr (-)$ constructions
as endofunctors on the regular $1$-overcategory
$\mathsf{Cat} \downarrow \Sigma_{\mathfrak{C}}$.


\begin{proposition}\label{ASSOCIDS PROP}
Let $k,l\geq -1$. One has canonical natural identifications 
$\Omega^k_{\mathfrak{C}} \wr \Omega^l_{\mathfrak{C}} \wr A
\simeq 
\Omega^{k+l+1}_{\mathfrak{C}} \wr A $.

Moreover, these identifications are associative in the sense that for any $k,l,m \leq -1$ the iterated composite identifications below coincide.
\[
\Omega^k_{\mathfrak{C}} \wr \Omega^l_{\mathfrak{C}} \wr \Omega^m_{\mathfrak{C}} \wr A
	\simeq 
\Omega^{k+l+1}_{\mathfrak{C}} \wr \Omega^m_{\mathfrak{C}} \wr A
	\simeq 
\Omega^{k+l+m+2}_{\mathfrak{C}} \wr A
\qquad
\Omega^k_{\mathfrak{C}} \wr \Omega^l_{\mathfrak{C}} \wr \Omega^m_{\mathfrak{C}} \wr A
	\simeq 
\Omega^{k}_{\mathfrak{C}} \wr \Omega^{l+m+1}_{\mathfrak{C}} \wr A
	\simeq 
\Omega^{k+l+m+2}_{\mathfrak{C}} \wr A
\]
Moreover, the identifications above further induce the following identifications
\[
d^i \wr \Omega^l \wr A \simeq d^i \wr A
	\quad
\pi_{i,k} \wr \Omega^l \wr A \simeq \pi_{i,k} \wr A
	\quad
s^j \wr \Omega^l \wr A \simeq d^j \wr A
	\quad
\Omega^k \wr d^i \wr A \simeq d^{k+i+1} \wr A
	\quad
\Omega^k \wr s^j \wr A \simeq s^{k+j+1} \wr A
\]
\end{proposition}


\begin{proof}
	The first claim follows by noting that all squares in the diagram below are pullback squares
\[
\begin{tikzcd}
	\Omega^{k+l+1}_{\mathfrak{C}} \wr A \ar{r}{\boldsymbol{V}^k} \ar{d} &
	\Sigma \wr \Omega^{l}_{\mathfrak{C}} \wr A  \ar{d} \ar{r}{\boldsymbol{V}^l} &
	\Sigma^{\wr 2} \wr A \ar{d}
\\
	\Omega^{k+l+1}_{\mathfrak{C}} \ar{r}{\boldsymbol{V}^k} 
	\ar{d}[swap]{d^{k+1,\cdots,k+l+1}} &
	\Sigma \wr \Omega^{l}_{\mathfrak{C}} \ar{r}{\boldsymbol{V}^l}
	\ar{d}{d^{0,\cdots,l}} &
	\Sigma^{\wr 2} \wr \Sigma_{\mathfrak{C}}
\\
	\Omega^{k}_{\mathfrak{C}} \ar{r}{\boldsymbol{V}^k} &
	\Sigma \wr \Sigma_{\mathfrak{C}}
\end{tikzcd}
\]
while associativity follows from the obvious extension of the diagram above.

For the additional identifications, 
those identifications concerning $d^i$ and $\pi_{i,k}$
follow from the left diagram below 
(the bottom section of which commutes by 
Proposition \ref{CATFDIAG2 PROP} (FF2)),
the identification concerning $d^{k+i+1}$ follows from the rightmost diagram, and the identifications concerning 
$s^j$ and $s^{k+j+1}$
follow from obvious analogues of these diagrams.
\[
\begin{tikzcd}[row sep = 10pt]
	\Omega^{k+l+1}_{\mathfrak{C}} \wr A \ar{rrr}[name=TT,swap]{} \ar{rd}[swap]{d^i} \ar{dd} &&&
	\Sigma \wr \Omega^{l}_{\mathfrak{C}} \wr A \ar{dd}
&
	\Omega^{k+l+1}_{\mathfrak{C}} \wr A \ar{rr} \ar{rd}[swap]{d^{k+j+1}} \ar{dd} &&
	\Sigma \wr \Omega^{l}_{\mathfrak{C}} \wr A \ar{rd} \ar{dd}
\\
	&
	|[alias=TD]|
	\Omega^{k+l}_{\mathfrak{C}} \wr A \ar{rru} \ar{dd} &&
&
	&
	\Omega^{k+l}_{\mathfrak{C}} \wr A \ar{rr} \ar{dd} &&
	\Sigma \wr \Omega^{l-1}_{\mathfrak{C}} \wr A 	 \ar{dd}
\\
	\Omega^{k+l+1}_{\mathfrak{C}} \ar{rrr}[name=MT,swap]{} \ar{rd}[swap]{d^i} \ar{dd} &&&
	\Sigma \wr \Omega^{l}_{\mathfrak{C}} \ar{dd}
&
	\Omega^{k+l+1}_{\mathfrak{C}} \ar{rr} \ar{rd}[swap]{d^{k+j+1}} \ar{dd}&&
	\Sigma \wr \Omega^{l}_{\mathfrak{C}} \ar{rd} \ar{dd}
\\
	&
	|[alias=MD]|
	\Omega^{k+l}_{\mathfrak{C}} \ar{rru} \ar{dd} &&
&
	&
	\Omega^{k+l}_{\mathfrak{C}} \ar{rr} \ar{dd} &&
	\Sigma \wr \Omega^{l-1}_{\mathfrak{C}} \ar{dd}
\\
	\Omega^{k}_{\mathfrak{C}} \ar{rrr}[name=DT,swap]{} \ar{rd}[swap]{d^i} &&&
	\Sigma \wr \Sigma_{\mathfrak{C}}
&
	\Omega^{k}_{\mathfrak{C}} \ar{rr} \ar[equal]{rd} &&
	\Sigma \wr \Sigma_{\mathfrak{C}} \ar[equal]{rd}
\\
	&
	|[alias=DD]|
	\Omega^{k-1}_{\mathfrak{C}} \ar{rru} &&
&
	&
	\Omega^{k}_{\mathfrak{C}} \ar{rr} &&
	\Sigma \wr \Sigma_{\mathfrak{C}} 
\arrow[Leftrightarrow, from=TT, to=TD,shorten >=0.05cm,shorten <=0.05cm,
"\pi_{i}"
]
\arrow[Leftrightarrow, from=MT, to=MD,shorten >=0.05cm,shorten <=0.05cm,
"\pi_{i}"
]
\arrow[Leftrightarrow, from=DT, to=DD,shorten >=0.05cm,shorten <=0.05cm,
"\pi_{i}"
]
\end{tikzcd}
\]
\end{proof}




\subsection{The non-equivariant monad} \label{NONEQMON SEC}

\subsubsection{The non-equivariant monad on colored spans}

Our task in this section is to provide a suitable description of
$\mathsf{Op}(\mathcal{V})$.
We will mimic the approach used in \cite{BP_geo},
by first building a monad on a suitable category of spans, which is then transferred to $\mathsf{Sym}(\mathcal{V})$ along the Kan extension adjunction.



\begin{definition}
The category $\mathsf{WSpan}^l(\Sigma_{\bullet}^{op},\mathcal{V})$ has
\begin{itemize}
\item objects given by a choice of a set of colors $\mathfrak{C}$
and a span $\Sigma^{op}_{\mathfrak{C}} \leftarrow A^{op} \rightarrow \mathcal{V}$
\item morphisms given by a choice of a map of colors
$f \colon \mathfrak{C} \to \mathfrak{D}$
together with the data in the following diagram
\begin{equation}\label{COLORSPANMAP EQ}
\begin{tikzcd}[column sep = 20pt]
	\Sigma_{\mathfrak{C}}^{op}
		\ar{d}[swap]{f_{\**}} &
	A^{op}
		\ar{r}[name=U,swap]{} \ar{d} \ar{l} &
	\mathcal{V}	
\\
	\Sigma_{\mathfrak{D}}^{op}
		&
	|[alias=V]|
	B^{op} \ar{l}
		\ar{ru}
\arrow[Leftarrow, from=V, to=U,shorten >=0.05cm,shorten <=0.05cm]
\end{tikzcd}
\end{equation}
\end{itemize}
\end{definition}



\begin{remark}
By definition there is a forgetful functor
$\mathsf{WSpan}^l(\Sigma_{\bullet}^{op},\mathcal{V}) \to \mathsf{F}$
which remembers the set of colors.
Moreover, one easily checks that this is a Grothendieck fibration, with the cartesian arrows being the diagrams \eqref{COLORSPANMAP EQ} where the square is a pullback square and the natural transformation is an identity.
\end{remark}



\begin{remark}\label{LANADJ REM}
Given a $\Sigma^{op}_{\mathfrak{C}} \leftarrow A^{op} \rightarrow \mathcal{V}$
one can the form the left Kan extension
\[
\begin{tikzcd}[column sep = 30pt]
	A^{op}
		\ar{r}[name=U,swap]{}{F} \ar{d} &
	\mathcal{V}	
\\
	|[alias=V]|
	\Sigma_{\mathfrak{C}}^{op} 
		\ar[dashed]{ru}[swap]{\mathsf{Lan} F}
\arrow[Leftarrow, from=V, to=U,shorten >=0.05cm,shorten <=0.05cm]
\end{tikzcd}
\]
and it is straightforward to check that this defines an adjunction
\[
	\mathsf{Lan} \colon
	\mathsf{WSpan}^l(\Sigma_{\bullet}^{op},\mathcal{V}) 
\rightleftarrows
	\mathsf{Sym}(\mathcal{V})
	\colon \iota
\]
where the inclusion $\iota$ sends $\Sigma^{op}_{\mathfrak{C}} \to \mathcal{V}$ 
to the span
$\Sigma^{op}_{\mathfrak{C}} \xleftarrow{=} \Sigma^{op}_{\mathfrak{C}} \to \mathcal{V}$.
Moreover, it is straightforward to check that this is a fibered adjunction over $\mathsf{F}$, meaning that both functors as well as the adjunction unit and counit are compatible with the projection to $\mathsf{F}$ in the obvious way (cf. Definition \ref{FIBMON DEF}).
\end{remark}



\begin{remark}
One can also define a larger category 
$\mathsf{WSpan}^l( - ,\mathcal{V})$
where the categories $\Sigma_{\mathfrak{C}}^{op}$ in the spans (and functors between them) are allowed to be any category (any functor),
in which case left Kan extension defines an adjunction (cf. Remark \ref{SUBCATDOWNL REM})
\[
	\mathsf{Lan} \colon
	\mathsf{WSpan}^l( - ,\mathcal{V}) 
\rightleftarrows
	\mathsf{Cat} \downarrow ^l \mathcal{V}
	\colon \iota
\]
\end{remark}



\begin{definition}\label{NCOLOR DEF}
The monad $N$ on 
$\mathsf{WSpan}^l(\Sigma_{\bullet}^{op},\mathcal{V})$
sends the span 
$\Sigma^{op}_{\mathfrak{C}} \leftarrow A^{op} \to \mathcal{V}$
to the (opposite of the) composite span in
\begin{equation}\label{NCOLOR EQ}
\begin{tikzcd}
	\Omega^0_{\mathfrak{C}} \wr A \ar{r}{\boldsymbol{V}^0} \ar{d} &
	\Sigma \wr A  \ar{d} \ar{r} &
	\Sigma \wr \mathcal{V}^{op} \ar{r}{\otimes} &
	\mathcal{V}^{op}
\\
	\Omega^0_{\mathfrak{C}} \ar{r}{\boldsymbol{V}^0} \ar{d} &
	\Sigma \wr \Sigma_{\mathfrak{C}} 
\\
	\Sigma_{\mathfrak{C}}
\end{tikzcd}
\end{equation}
has monad multiplication
$\mu \colon N N
\Rightarrow 
N$ given by the diagram
\begin{equation}\label{NMONMULTTR EQ}
\begin{tikzcd}
	\Sigma_{\mathfrak{C}} \ar[equal]{d}&
	\Omega^1_{\mathfrak{C}} \wr A \ar{l} \ar{r} \ar{d}[swap]{d^0}&
	\Sigma \wr \Omega^0_{\mathfrak{C}} \wr A \ar{r} &
	|[alias=U]|
	\Sigma^{\wr 2} \wr A \ar{r} \ar{d}[swap]{\sigma^0} &
	\Sigma^{\wr 2} \wr \mathcal{V}^{op} \ar{r}{\otimes} \ar{d}[swap]{\sigma^0} &
	\Sigma \wr \mathcal{V}^{op} \ar{r}{\otimes} &
	|[alias=UU]|
	\mathcal{V}^{op} \ar[equal]{d}
\\
	\Sigma_{\mathfrak{C}} &
	|[alias=V]|
	\Omega^0_{\mathfrak{C}} \wr A \ar{l} \ar{rr} & &
	\Sigma \wr A \ar{r} &
	|[alias=VV]|
	\Sigma \wr \mathcal{V}^{op} \ar{rr}{\otimes} & &
	\mathcal{V}^{op}
\arrow[Leftrightarrow, from=V, to=U,shorten >=0.15cm,shorten <=0.15cm
,swap,"\pi"
]
\arrow[Leftrightarrow, from=VV, to=UU,shorten >=0.15cm,shorten <=0.15cm
]
\end{tikzcd}
\end{equation}
and the unit
$\eta \colon id \Rightarrow N$ given by
\begin{equation}\label{NMONIDTR EQ}
\begin{tikzcd}
	\Sigma_{\mathfrak{C}} \ar[equal]{d} & 
	A \ar{d}[swap]{s^{-1}} \ar{l} \ar[equal]{r} &
	A \ar{d}[swap]{\delta^0} \ar{r} &
	\mathcal{V}^{op} \ar{d}[swap]{\delta^0} \ar[equal]{r} &
	\mathcal{V}^{op} \ar[equal]{d}
\\
	\Sigma_{\mathfrak{C}} &
	\Omega^0_{\mathfrak{C}} \wr A \ar{l} \ar{r} &
	\Sigma \wr A \ar{r} &
	\Sigma \wr \mathcal{V}^{op} \ar{r}{\otimes} &
	\mathcal{V}^{op}
\end{tikzcd}
\end{equation}

The fact that $N$ is functorial follows from Proposition \ref{SPANPIECE PROP}.
\end{definition}





\begin{proposition}\label{MONISMON PROP}
$N$ is a monad on $\mathsf{WSpan}^l(\Sigma_{\bullet}^{op},\mathcal{V})$.
\end{proposition}


\begin{proof}
To check associativity, the functor $\mu N \colon 
N N N
\Rightarrow N N$
is encoded by the diagram
\[
\begin{tikzcd}
	\Omega^2_{\mathfrak{C}} \wr A \ar{r} \ar{d}[swap]{d^0} &
	\Sigma \wr \Omega^1_{\mathfrak{C}} \wr A \ar{r} &
	|[alias=UUU]|
	\Sigma^{\wr 2} \wr \Omega^0_{\mathfrak{C}} \wr A
	\ar{d}[swap]{\sigma^0} \ar{r} &
	\Sigma^{\wr 3} \wr A \ar{d}[swap]{\sigma^0} \ar{r} &
	\Sigma^{\wr 3} \wr \mathcal{V}^{op} \ar{d}[swap]{\sigma^0} \ar{r}{\otimes} &
	\Sigma^{\wr 2} \wr \mathcal{V}^{op} \ar{d}[swap]{\sigma^0} \ar{r}{\otimes} &
	\Sigma \wr \mathcal{V}^{op} \ar{r}{\otimes} & 
	|[alias=UUUU]|
	\mathcal{V}^{op} \ar[equal]{d}
\\
	|[alias=VVV]|
	\Omega^1_{\mathfrak{C}} \wr A \ar{rr} \ar{d}[swap]{d^0} & &
	\Sigma \wr \Omega^0_{\mathfrak{C}} \wr A \ar{r} &
	|[alias=U]|
	\Sigma^{\wr 2} \wr A \ar{r} \ar{d}[swap]{\sigma^0} &
	\Sigma^{\wr 2} \wr \mathcal{V}^{op} \ar{r}{\otimes} \ar{d}[swap]{\sigma^0} &
	|[alias=VVVV]|
	\Sigma \wr \mathcal{V}^{op} \ar{rr}{\otimes} & &
	|[alias=UU]|
	\mathcal{V}^{op} \ar[equal]{d}
\\
	|[alias=V]|
	\Omega^0_{\mathfrak{C}} \wr A \ar{rrr} & & &
	\Sigma \wr A \ar{r} &
	|[alias=VV]|
	\Sigma \wr \mathcal{V}^{op} \ar{rrr}{\otimes} & & &
	\mathcal{V}^{op}
\arrow[Leftrightarrow, from=V, to=U,shorten >=0.15cm,shorten <=0.15cm
,swap,"\pi"
]
\arrow[Leftrightarrow, from=VV, to=UU,shorten >=0.15cm,shorten <=0.15cm
]
\arrow[Leftrightarrow, from=VVV, to=UUU,shorten >=0.15cm,shorten <=0.15cm
,swap,"\pi"
]
\arrow[Leftrightarrow, from=VVVV, to=UUUU,shorten >=0.15cm,shorten <=0.15cm
]
\end{tikzcd}
\]
while the functor
$ N \mu \colon 
N N N
\Rightarrow N N$
is encoded by
\[
\begin{tikzcd}
	\Omega^2_{\mathfrak{C}} \wr A \ar{d}[swap]{d^1} \ar{r} &
	\Sigma \wr \Omega^1_{\mathfrak{C}} \wr A \ar{d}[swap]{d^0} \ar{r} &
	\Sigma^{\wr 2} \wr \Omega^0_{\mathfrak{C}} \wr A \ar{r} &
	|[alias=UUU]|
	\Sigma^{\wr 3} \wr A \ar{d}[swap]{\sigma^1} \ar{r} &
	\Sigma^{\wr 3} \wr \mathcal{V}^{op} \ar{d}[swap]{\sigma^1} \ar{r}{\otimes} &
	\Sigma^{\wr 2} \wr \mathcal{V}^{op} \ar{r}{\otimes} &
	|[alias=UUUU]|
	\Sigma \wr \mathcal{V}^{op} \ar{r}{\otimes} \ar[equal]{d} &
	\mathcal{V}^{op} \ar[equal]{d}
\\
	\Omega^1_{\mathfrak{C}} \wr A \ar{r} \ar{d}[swap]{d^0} &
	|[alias=VVV]|
	\Sigma \wr \Omega^0_{\mathfrak{C}} \wr A \ar{rr} & &
	|[alias=U]|
	\Sigma^{\wr 2} \wr A \ar{r} \ar{d}[swap]{\sigma^0} &
	|[alias=VVVV]|
	\Sigma^{\wr 2} \wr \mathcal{V}^{op} \ar{rr}{\otimes} \ar{d}[swap]{\sigma^0} & &
	\Sigma \wr \mathcal{V}^{op} \ar{r}{\otimes} &
	|[alias=UU]|
	\mathcal{V}^{op} \ar[equal]{d}
\\
	|[alias=V]|
	\Omega^0_{\mathfrak{C}} \wr A \ar{rrr} & & &
	\Sigma \wr A \ar{r} &
	|[alias=VV]|
	\Sigma \wr \mathcal{V}^{op} \ar{rrr}{\otimes} & & &
	\mathcal{V}^{op}
\arrow[Leftrightarrow, from=V, to=U,shorten >=0.15cm,shorten <=0.15cm
,swap,"\pi"
]
\arrow[Leftrightarrow, from=VV, to=UU,shorten >=0.15cm,shorten <=0.15cm
]
\arrow[Leftrightarrow, from=VVV, to=UUU,shorten >=0.15cm,shorten <=0.15cm
,swap,"\pi"
]
\arrow[Leftrightarrow, from=VVVV, to=UUUU,shorten >=0.15cm,shorten <=0.15cm
]
\end{tikzcd}
\]
That the leftmost sections of these diagrams match follows by 
parts (IT1) and (FF1) of Proposition \ref{CATDIAG2 PROP},
while the fact that the rightmost sections coincide follows since
$\mathcal{V}$ is a monoidal category.

The unitality of the monad $N$
follows by a simpler version of the argument above.
\end{proof}



\subsubsection{The non-equivariant monad on symmetric sequences}


We will now use the adjunction
\[
	\mathsf{Lan} \colon
	\mathsf{WSpan}^l(\Sigma_{\bullet}^{op},\mathcal{V}) 
\rightleftarrows
	\mathsf{Sym}(\mathcal{V})
	\colon \iota
\]
from Remark \ref{LANADJ REM} to induce a monad on 
$\mathsf{Sym}(\mathcal{V})$.
To do so, we will verify the conditions in \cite[Proposition 2.27]{BP_geo},
stating that the natural transformations
\[
	\mathsf{Lan} \iota \xrightarrow{\epsilon} id
\qquad
	\mathsf{Lan} N \xrightarrow{\eta} \mathsf{Lan} N \iota \mathsf{Lan}
\]
are natural isomorphisms.

This is clear for $\epsilon$ while for $\eta$ it follows from the following two lemmas, the first of which is proven exactly as \cite[Lemma 2.21]{BP_geo}.


\begin{lemma}\label{FINWRPRODLIM LEM}
If in $\mathcal{V}$
the monoidal product %products
commutes with colimits in each variable, and the leftmost diagram
\begin{equation}\label{WRLAN EQ}
	\begin{tikzcd}[column sep = 4.5em]
	\mathcal{C}^{op} \ar{r}[swap,name=F]{}{G} \ar{d}[swap]{k^{op}} & 
	\mathcal{V} & 
	(\Sigma \wr \mathcal{C})^{op} \ar{d}[swap]{(\Fin_s \wr k)^{op}} 
	\ar{r}[swap,name=FF]{}{(\Sigma \wr G^{op})^{op}} & 
	(\Sigma \wr \mathcal{V}^{op})^{op} \ar{r}{\otimes} &
	\mathcal{V}
\\
	|[alias=D]|\mathcal{D}^{op} \ar{ru}[swap]{H} &
	& 
	|[alias=FD]|(\Sigma \wr \mathcal{D})^{op} 
	\ar{ru}[swap]{(\Sigma \wr H^{op})^{op}}
	\ar[bend right=13]{rru}[swap]{\otimes \circ (\Sigma \wr H^{op})^{op}}
	&
	\arrow[Leftarrow, from=D, to=F,shorten <=0.10cm,"\epsilon"]
	\arrow[Leftarrow, from=FD, to=FF,shorten <=0.10cm]
	\end{tikzcd}
\end{equation}
is a left Kan extension diagram then so is the composite of the rightmost diagram. 
\end{lemma}


The following is a variation of \cite[Lemma 4.28]{BP_geo}
\begin{lemma}\label{LANPULLCOMA LEM}
	Suppose that $\mathcal{V}$ is complete. If the rightmost triangle in 
\[
\begin{tikzcd}
	\Omega_{\mathfrak{C}}^{0} \wr A \ar{r}{V} 
	\ar{d} & 
	\Sigma \wr A  
	\ar{d}  \ar{r}[swap,name=F]{}&
	\mathcal{V}^{op}
\\
	\Omega_{\mathfrak{C}}^{0} \ar{r}[swap]{V} & 
	|[alias=FEG]|\Sigma \wr \Sigma_{\mathfrak{C}} \ar{ru}
\arrow[Rightarrow, from=FEG, to=F,shorten <=0.15cm]
\end{tikzcd}
\]
is a right Kan extension diagram then so is the composite diagram.
\end{lemma}



\begin{proof}
Our proof will be a slightly more formalized version of the proof in \cite[Lemma 4.28]{BP_geo}.

Firstly, note that the composite
\begin{equation}\label{COMPISO EQ}
T \downarrow \Omega^0_{\mathfrak{C}} 
	\to 
\left( T_v \right)_{V(T)} \downarrow \Sigma \wr \Sigma_{\mathfrak{C}}
	\to
\left( T_v \right)_{V(T)} \downarrow_{\Sigma} \Sigma \wr \Sigma_{\mathfrak{C}}
\end{equation}
is an isomorphism. Indeed, the objects of 
$T \downarrow \Omega^0_{\mathfrak{C}}$
are isomorphisms $T \simeq T'$, which are entirely determined by 
a tuple of isomorphisms $\left(T_v \simeq T'_v\right)_{V(T)}$,
which are the objects of the target. 

We now claim that the maps
\begin{equation}\label{COMPISO2 EQ}
T \downarrow \Omega^0_{\mathfrak{C}} \wr A
	\to 
\left( T_v \right)_{V(T)} \downarrow \Sigma \wr A
	\to
\left( T_v \right)_{V(T)} \downarrow_{\Sigma} \Sigma \wr A
\end{equation}
are likewise isomorphisms.
Indeed, writing $G$ (resp. $\bar{G}$) for the composite functors in \eqref{COMPISO EQ}, (resp. \eqref{COMPISO2 EQ}), one has
\[
\bar{G}^{-1}
\left((a_v), (T_v) \xrightarrow{g} (f(a_v))\right)=
\left(\left(\left(\pi_{\Sigma} V G^{-1}(g)\right)^{-1}\right)^{\**} (a_v),
%G^{-1}\left((T_v \xrightarrow{\simeq} f(a_v))_{V(T)}\right),
T \xrightarrow{G^{-1}(g)} \bullet
\right).
\]
Now consider the diagram
\begin{equation}\label{COMPISO3 EQ}
T \downarrow \Omega^0_{\mathfrak{C}} \wr A
	\to 
\left( T_v \right)_{V(T)} \downarrow \Sigma \wr A
	\to
\left( T_v \right)_{V(T)} \downarrow_{\Sigma} \Sigma \wr A
	\to 
\left( T_v \right)_{V(T)} \downarrow \Sigma \wr A.
\end{equation}
The result will follow provided that the first map in \eqref{COMPISO3 EQ} is final. But this now follows since this map is isomorphic to the full composite of \eqref{COMPISO3 EQ}, which is final since \eqref{COMPISO2 EQ} is an isomorphism and the last map in \eqref{COMPISO3 EQ} is known to be final.
\end{proof}



\begin{definition}\label{COLORMON DEF}
The (colored) free operad monad is the monad
$\mathbb{F}$ on $\mathsf{Sym}(\mathcal{V})$
with underlying functor
$\mathbb{F} = \mathsf{Lan} N \iota$ with multiplication and unit given by
\[
	\mathsf{Lan} N \iota \mathsf{Lan} N \iota \xleftarrow{\simeq} 
	\mathsf{Lan} N N \iota \to 
	\mathsf{Lan} N \iota
\qquad
	id \xleftarrow{\simeq} 
	\mathsf{Lan} \iota \to
	\mathsf{Lan} N \iota.
\]
\end{definition}





\subsection{Pushouts of operads}



Our goal in this section will be to understand free operad extensions, i.e. pushouts of the form 
\begin{equation}\label{OU EQ}
            \begin{tikzcd}
                  \mathbb F X \arrow[d, "\mathbb{F}u"'] \arrow[r]
                  &
                  \O \arrow[d]
                  \\
                  \mathbb F Y \arrow[r]
                  &
                  \O[u].
            \end{tikzcd}
\end{equation}
where $u \colon X \to Y$ is a map of symmetric sequences.
Moreover, we will require that \eqref{OU EQ} is a fibered diagram over $\mathsf{F}$, i.e. that all maps therein are the identity on colors.
However, rather than fix the set of colors, 
we will find it convenient to consider all colors simultaneously, 
and note that our constructions are natural 
on \eqref{OU EQ} with respect to change of colors.
More explicitly, this will mean that our work in this section will be natural with regard to commutative diagrams
\begin{equation}\label{COLORCHNAT EQ}
	\begin{tikzcd}
		\mathbb X \arrow[r, "u"',swap] \arrow[d]
	&
		Y \arrow[d]
&
		\mathbb F X \arrow[d] \arrow[r]
	&
		\O \arrow[d]
\\
		X' \arrow[r, "u'"']
	&
		Y'
&
		\mathbb F X' \arrow[r]
	&
		\O'
	\end{tikzcd}
\end{equation}
where all vertical maps induce the same map on objects.


To understand the pushouts \eqref{OU EQ},
we will produce a filtration
\begin{equation}\label{FILT EQ}
      \O = \O_0 \into \O_1 \into \O_2 \into \dots \into \colim_k \O_k = \O[u]
\end{equation}
of the underlying symmetric sequences, i.e. with 
$\mathcal{O}_i \in \mathsf{Sym}(\mathcal{V})$
(all maps in \eqref{FILT EQ} will, again, be the identity on colors).


Writing $\amalg_{\mathsf{F}}$ and $\mathbin{\check\amalg}_{\mathsf{F}}$
for the fibered coproducts in 
$\mathsf{Sym}(\mathcal{V})$ and
$\mathsf{Op}(\mathcal{V})$
(i.e. these are the coproducts within each fiber over $\mathsf{F}$, rather than the coproducts in the overall categories)
the discussion in $(5.3)$ through $(5.7)$ of \cite{BP_geo}
yields that
\begin{align*}
  \O[u]
  &
    \simeq \mathrm{coeq}\left(
          \O \mathbin{\check\amalg_{\mathsf{F}}} \mathbb F X \mathbin{\check\amalg}_{\mathsf{F}} \mathbb F Y \rightrightarrows \O \mathbin{\check\amalg}_{\mathsf{F}} \mathbb F Y
          \right)
  \\
  &
    \simeq \colim_{[l] \in \Delta^{op}} 
    B_l \left( \O, \mathbb F X, \mathbb F X, \mathbb F X, \mathbb FY \right)
  \\
  &
    \simeq \colim_{[l] \in \Delta^{op},[n] \in \Delta^{op}} 
    B_l \left( \mathbb F^{\circ n+1} \O, \mathbb F X, \mathbb F X, \mathbb F X, \mathbb FY \right)
  \\
  &
    \simeq \colim_{[l] \in \Delta^{op},[n] \in \Delta^{op}} 
    \mathsf{Lan} N \circ \left( N^{\circ n} \iota \O \amalg_{\mathsf{F}} \iota X^{\amalg_{\mathsf{F}} 2l +1} \amalg_{\mathsf{F}} \iota Y \right),
    \stepcounter{equation}\tag{\theequation}\label{OU EQ1}
    % =: \colim_{n,l} \mathsf{Lan} \hat N_{n,l}^{(\O,X,Y)},
\end{align*}
where $B_{\bullet}$ denotes the \textit{double bar construction}
with respect to $\mathbin{\check\amalg}_{\mathsf{F}}$,
$\mathbb{F}^{\bullet +1} \mathcal{O}$ denotes the simplicial resolution of $\mathcal{O}$, 
and $N$ is the span monad in Definition \ref{NCOLOR DEF}.
Crucially, we note that colimits over $\Delta^{op}$
are computed by the reflexive coequalizer determined by levels $0$ and $1$, 
so that the colimits in \eqref{OU EQ1}
can be computed in $\mathsf{Sym}(\mathcal{V})$
rather than in $\mathsf{Op}(\mathcal{V})$.


Noting that $N \left(N^{\circ n} \iota \O \amalg_{\mathsf{F}} \iota X^{\amalg_{\mathsf{F}} 2l +1}\amalg_{\mathsf{F}} \iota Y \right)$ denotes a certain span
$\Sigma_{\mathfrak{C}}^{op} \leftarrow 
\left(\Omega^{n,\lambda_l}_{\mathfrak{C}}\right)^{op} \to \mathcal{V}$,
we will then apply (the natural analogue of)
\cite[Prop. 5.37]{BP_geo}
along each simplicial direction
to convert the last line of \eqref{OU EQ1}
into a $\mathsf{Lan}$
over a single span
$\Sigma_{\mathfrak{C}}^{op} \leftarrow 
\left|\Omega^{n,\lambda_l}_{\mathfrak{C}}\right|^{op} \to \mathcal{V}$.

The task of describing 
$\Omega^{n,\lambda_l}_{\mathfrak{C}}$
is similar to the upshot of Proposition \ref{ASSOCIDS PROP}, which shows that
$N^{\circ n+1}$ is naturally calculated using the
$\Omega_{\mathfrak{C}}^{n} \wr (-)$ 
construction.

Indeed, we will do a little more. For $\lambda = \lambda_a \amalg \lambda_i$
a partition of $\set{1,2,\dots,l}$
we will write 
$N^{\times \lambda}$
for the monad (cf. \cite[\S 2.3]{BP_geo}) on 
$\left(\mathsf{WSpan}_l(\Sigma_{\bullet}^{op},\mathcal{V})\right)^{\times l}$
given by
\[
\left(N^{\times \lambda} (A_j)\right)_k = 
\begin{cases}
N(A_k) & \text{if } k\in \lambda_a
\\
A_k & \text{if } k\in \lambda_i
\end{cases}
\]
where we note that $N^{\times \lambda}$
preserves the fibered product
$\left(\mathsf{WSpan}_l(\Sigma_{\bullet}^{op},\mathcal{V})\right)^{\times_{\mathsf{F}} l}$,
i.e. the subcategory consisting of those tuples $(A_j)$ with the same objects in $\mathsf{F}$ (and similarly for maps), and we will slightly abuse notation by also writing 
$N^{\times \lambda}$
for the monad restricted to this subcategory.


\subsubsection{Labeled colored strings}

The categories $\Omega_{\mathfrak C}^{n,s,\lambda}$ in the following definition will then represent
the functors
$N^{\circ s+1} \circ \coprod_{\mathsf{F}} \circ \left(N^{\times \lambda}\right)^{\circ n-s}$.

 

\begin{definition}[{cf. \cite[Defn. 5.10]{BP_geo}}]\label{CLPS DEF}
      Given $-1 \leq s \leq n$, $l \geq 0$, and a partition $\lambda = \lambda_a \amalg \lambda_i$ of $\set{1,2,\dots,l}$,
      define $\Omega_{\mathfrak C}^{n,s,\lambda}$ to have as objects
$n$-planar strings
\begin{equation}
	T_{-1} \xrightarrow{\varphi_0} T_0 \xrightarrow{\varphi_1} T_1 \xrightarrow{\varphi_2} \dots
	T_{s} \xrightarrow{\varphi_{s+1}} T_{s+1} \xrightarrow{\varphi_{s+2}}  \dots
	\xrightarrow{\varphi_n} T_n
\end{equation}
together with $l$-labelings of $T_s, T_{s+1}, \cdots, T_n$
such that
$\varphi_{r}, r>s$ are $\lambda_i$-inert label maps.

Arrows in $\Omega_{\mathfrak C}^{n,s,\lambda}$
are isomorphisms $\left(\pi_r \colon T_r \to T'_r\right)$
such that $\pi_r,r \geq s$ are label maps.

Further, for any $s<0$ of $n<s'$ we write
\[
\Omega_{\mathfrak{C}}^{n,s,\lambda} = \Omega_{\mathfrak{C}}^{n,-1,\lambda}
\qquad
\Omega_{\mathfrak{C}}^{n,s',\lambda} = \Omega_{\mathfrak{C}}^{n}
\]
\end{definition}

We now discuss the functors relating the $\Omega_{\mathfrak{C}}^{n,s,\lambda}$ categories. Firstly, for 
$s \leq s'$ 
and map of labels $g \colon \{1,\cdots,l'\} \to \{1,\cdots,l\}$
such that $\lambda'_a \subseteq g^{-1}\left( \lambda_a\right)$
there are natural functors
\[
\Omega_{\mathfrak{C}}^{n,s,\lambda} \to \Omega_{\mathfrak{C}}^{n,s',\lambda},
\qquad
\Omega_{\mathfrak{C}}^{n,s,\lambda'} \xrightarrow{g_{\**}} \Omega_{\mathfrak{C}}^{n,s,\lambda}.
\]
Second, by keeping track of labels on vertices,
 the usual functors relating the categories 
$\Omega^n_{\mathfrak{C}}$ extend to the categories
$\Omega_{\mathfrak{C}}^{n,s,\lambda}$. Indeed, for 
$k \leq n$
and 
$f \colon \mathfrak{C} \to \mathfrak{D}$ a map of colors
one has functors
\begin{equation}\label{FGTLABEL EQ}
\Omega_{\mathfrak{C}}^{n,s,\lambda} \xrightarrow{V^k} \Sigma \wr\Omega_{\mathfrak{C}}^{n-k-1,s-k-1,\lambda},
\qquad
\Omega_{\mathfrak{C}}^{n,s,\lambda} \xrightarrow{f_{\**}} \Omega_{\mathfrak{D}}^{n,s,\lambda}.
\end{equation}
Lastly, one also has simplicial operators $d_i$, $s_j$, 
but some care is needed with the way these interact with the index $s$. To do so, defining functions $d_i,s_j\colon \mathbb{Z} \to \mathbb{Z}$ by
\begin{equation}\label{SIMPLEXP EQ}
 d_i(s) = 
\begin{cases}
s-1, & i<s
\\
s, & s\leq i
\end{cases}
\qquad
s_j(s) = 
\begin{cases}
s+1, & j<s
\\
s, & s\leq j
\end{cases}
\end{equation}
one has simplicial operators
\[
\Omega_{\mathfrak{C}}^{n,s,\lambda} \xrightarrow{d_i} \Sigma \wr\Omega_{\mathfrak{C}}^{n,d_i(s),\lambda},
\qquad
\Omega_{\mathfrak{C}}^{n,s,\lambda} \xrightarrow{s_j} \Sigma \wr\Omega_{\mathfrak{C}}^{n,s_j(s),\lambda},
\]
for $0\leq i \leq n$ and $-1\leq j \leq n$.
In practice, we will prefer to suppress $s$ from the notation,
and write 
$\Omega_{\mathfrak{C}}^{n,\bullet,\lambda}$ to denote the string of categories 
$\Omega_{\mathfrak{C}}^{n,s,\lambda}$ as a whole.
Lastly, the $\pi_{i,k}$ natural isomorphisms for $i<k$ from Proposition \ref{CATDIAG PROP}
generalize to natural isomorphisms
\begin{equation}
\begin{tikzcd}[row sep = tiny, column sep = 35pt]
	\Omega_{\mathfrak{C}}^{n,s,\lambda}
	\arrow{r}{V^k} \arrow{dd}[swap]{d^i} &
	|[alias=U]|
	 \Sigma \wr \Omega_{\mathfrak{C}}^{n-k-1,s-k-1,\lambda}
	 \ar[equal]{dd}{}
\\
\\
	|[alias=V]|
	\Omega_{\mathfrak{C}}^{n-1,d_i(s),\lambda} \arrow{r}[swap]{V^{k-1}} &
	 \Sigma \wr \Omega_{\mathfrak{C}}^{n-k-1,d_i(s)-k,\lambda}
\arrow[Leftrightarrow, from=V, to=U,shorten >=0.15cm,shorten <=0.15cm
,swap,"\pi_{i,k}"
]
\end{tikzcd}
\end{equation}
(note that the right vertical map is an identity even if
$s-k-1 \neq d_i(s)-k$, since that can only occur if $s\leq i \leq k$, implying that the rightmost terms are both $\Sigma \wr \Omega_{\mathfrak{C}}^{n-k-1,-1,\lambda}$).

\begin{remark}
We now discuss the naturality of the given functors on the categories
$\Omega_{\mathfrak{C}}^{n,s,\lambda}$ just described.
\begin{enumerate}[label=(\roman*)]
\item by keeping track of vertex labels, all the analogues of the properties in Propositions \ref{CATDIAG PROP} and \ref{CATDIAG2 PROP} extend (note that this includes the pullback claims in 
Proposition \ref{CATDIAG PROP}(ii))
\item the change of color functors $f_{\**}$, change of label functors $g_{\**}$, and the forgetful functors in
\eqref{FGTLABEL EQ} are all natural with respect to each other.
\item $d_i$, $s_j$, $\boldsymbol{V}^k$
and $\pi_{i,k}$, are natural with respect to the change of label functors $g_{\**}$ and the forgetful functors in
\eqref{FGTLABEL EQ}, 
in the sense that they satisfy the analogues of the properties in 
Proposition \ref{CATDIAG PROP}(iii) with 
the role of $f_{\**}$ replaced with the latter functors.
\item
For $k \leq s \leq s'$ the following squares are pullback squares
\[
\begin{tikzcd}[column sep = small, row sep = small]
	\Omega^{n,s,\lambda}_{\mathfrak{C}} \ar{r}{\boldsymbol{V}^k} \ar{dd} &
	\Sigma \wr \Omega^{n-k-1,s-k-1,\lambda}_{\mathfrak{C}} \ar{dd}
\\
\\
	\Omega^{n,s',\lambda}_{\mathfrak{C}} \ar{r}[swap]{\boldsymbol{V}^k} &
	\Sigma \wr \Omega^{n-k-1,s'-k-1,\lambda}_{\mathfrak{C}}
\end{tikzcd}
\]
\end{enumerate}
\end{remark}

The following is the main purpose of the 
$\Omega_{\mathfrak{C}}^{n,s,\lambda}$ categories,
adapting the work in \S \ref{WRACONST SEC}.

\begin{definition}[{cf. \cite[Notation 5.24]{BP_geo}}]\label{NA_DEF}
      Given a $l$-tuple of functors
      $\left(A_j \to \Sigma_{\mathfrak C} \right)_{1\leq j \leq l}$,
we write
\begin{equation}\label{WRAJDEF EQ}
(-) \wr (A_j) \colon 
\mathsf{Cat} \downarrow^r_{\Sigma} \Sigma \wr \Sigma_{\mathfrak{C}}^{\amalg l}
\to
\mathsf{Cat} \downarrow^r_{\Sigma} \Sigma \wr \amalg_j A_j
\end{equation}
for the pullback \eqref{WSPANPULL EQ} for the map
$\Sigma \wr \amalg_j A_j \to \Sigma \wr \Sigma_{\mathfrak{C}}^{\amalg l}$.
 
In particular, for all $-1\leq s \leq n$, this defines categories
$\Omega^{n,s,\lambda}_{\mathfrak{C}} \wr (A_j)$ via pullbacks
(note that the $s \leq n$ restriction guarantees that the target of the lower $\boldsymbol{V}^n$ is indeed $\Sigma \wr \Sigma_{\mathfrak{C}}^{\amalg l}$) 
\begin{equation}\label{WRAJSAMPLE EQ}
\begin{tikzcd}
	\Omega^{n,s,\lambda}_{\mathfrak{C}} \wr (A_j) \ar{r}{\boldsymbol{V}^n} \ar{d} &
	\Sigma \wr \amalg_j A_j  \ar{d}
\\
	\Omega^{n,s,\lambda}_{\mathfrak{C}} \ar{r}{\boldsymbol{V}^n} &
	\Sigma \wr \Sigma_{\mathfrak{C}}^{\amalg l}
\end{tikzcd}
\end{equation}
along with analogues of $d_i$ (for $i<n$), $s_j$, $V^k$, $\pi_{i,k}$
and of the forgetful functors in \eqref{FGTLABEL EQ}
(cf. the discussion following \eqref{WRADEF EQ}).
\end{definition}
 

\begin{proposition}\label{SPANPIECEJ PROP}
A tuple of commutative squares
\begin{equation}\label{SPANPIECEJ EQ}
\begin{tikzcd}
	A_j \ar{d} \ar{r}{f} &  \ar{d} B_j
\\
	\Sigma_{\mathfrak{C}} \ar{r}[swap]{f} & \Sigma_{\mathfrak{D}}
\end{tikzcd}
\end{equation}
induces natural maps 
$f_{\**} \colon
\Omega_{\mathfrak{C}}^{n,\bullet,\lambda} \wr (A_j) \to 
\Omega_{\mathfrak{D}}^{n,\bullet,\lambda} \wr (B_j) $.

Similarly, a map of tuples $A_j \to B_{g(j)}$ for 
$g \colon \{1,\cdots,l\} \to \{1,\cdots,l'\}$
induces natural maps 
$g_{\**} \colon
\Omega_{\mathfrak{C}}^{n,\bullet,\lambda} \wr (A_j) \to 
\Omega_{\mathfrak{C}}^{n,\bullet,\lambda'} \wr (B_{j'}) $.

Moreover, both $f_{\**}$ and $g_{\**}$ satisfy the analogues of the commutativity properties in Proposition \ref{SPANPIECE PROP}.
In particular, the diagrams below commute.
\[
\begin{tikzcd}[column sep = 4pt, row sep = small]
	\Omega^{n,\bullet,\lambda}_{\mathfrak{C}} \wr (A_j)
	\ar{rrrrr}[name=toE]{\boldsymbol{V}^k} \ar{rd}[swap]{d^i} \ar{dd}[swap]{f_{\**}}
	&&&
	&&
	\Sigma \wr \Omega^{n-k-1,\bullet,\lambda}_{\mathfrak{C}} \wr (A_j) \ar{dd}{f_{\**}}
&
	\Omega^{n,\bullet,\lambda}_{\mathfrak{C}} \wr (A_j)
	\ar{rrrrr}[name=toE2]{\boldsymbol{V}^k} \ar{rd}[swap]{d^i} \ar{dd}[swap]{f_{\**}}
	&&&
	&&
	\Sigma \wr \Omega^{n-k-1,\bullet,\lambda}_{\mathfrak{C}} \wr (A_j) \ar{dd}{f_{\**}}
\\
	&
	|[alias=DBE]|
	\Omega^{n-1,\bullet,\lambda}_{\mathfrak{C}} \wr (A_j) \ar{rrrru}[swap]{\boldsymbol{V}^{k-1}}
	&&&&
&
	&
	|[alias=DBE2]|
	\Omega^{n-1,\bullet,\lambda}_{\mathfrak{C}} \wr (A_j) \ar{rrrru}[swap]{\boldsymbol{V}^{k-1}}
\\
	\Omega^{n,\bullet,\lambda}_{\mathfrak{D}} \wr (B_j) \ar{rrrrr}[name=toB]{\boldsymbol{V}^k} \ar{rd}[swap]{d^i}
	&&&
	&&
	\Sigma \wr \Omega^{n-k-1,\bullet,\lambda}_{\mathfrak{D}} \wr (B_j)
&
	\Omega^{n,\bullet,\lambda'}_{\mathfrak{C}} \wr (B_{j'}) \ar{rrrrr}[name=toB2]{\boldsymbol{V}^k} \ar{rd}[swap]{d^i}
	&&&
	&&
	\Sigma \wr \Omega^{n-k-1,\bullet,\lambda'}_{\mathfrak{C}} \wr (B_{j'})
\\
	&
	|[alias=D]| \Omega^{n-1,\bullet,\lambda}_{\mathfrak{D}} \wr (B_j) \ar{rrrru}[swap]{\boldsymbol{V}^{k-1}}
	&&&&
&
	&
	|[alias=D2]| \Omega^{n-1,\bullet,\lambda'}_{\mathfrak{C}} \wr (B_{j'}) \ar{rrrru}[swap]{\boldsymbol{V}^{k-1}}
\arrow[Leftrightarrow, from=DBE, to=toE, shorten <=0.15cm,shorten >=0.15cm
,swap,"\pi"
]
	\arrow[Leftrightarrow, from=D, to=toB, shorten <=0.15cm,shorten >=0.15cm,swap,"\pi"]
	\arrow[from=DBE, to=D, crossing over, near start, swap, "f_{\**}"]
\arrow[Leftrightarrow, from=DBE2, to=toE2, shorten <=0.15cm,shorten >=0.15cm
,swap,"\pi"
]
	\arrow[Leftrightarrow, from=D2, to=toB2, shorten <=0.15cm,shorten >=0.15cm,swap,"\pi"]
	\arrow[from=DBE2, to=D2, crossing over, near start, swap, "f_{\**}"]
\end{tikzcd}
\]
\end{proposition}

\begin{proof}
This follows by repeating the argument in the proof of Proposition \ref{SPANPIECE PROP}.
\end{proof}


 
Thanks to the composite functors
$\Omega_{\mathfrak{C}}^{n,s,\lambda} \wr (A_j)
\to \Omega_{\mathfrak{C}}^{n,s,\lambda} 
\to \Omega^{-1,0}_{\mathfrak{C}} = \Sigma_{\mathfrak{C}}$,
we can regard the 
$\Omega_{\mathfrak{C}}^{n,s,\lambda} \wr (-)$
construction as a functor
$\left(\mathsf{Cat}\downarrow \Sigma_{\mathfrak{C}}\right)^{\times l}
\to \mathsf{Cat}\downarrow \Sigma_{\mathfrak{C}}$

\begin{corollary}[{cf. \cite[Cor. 5.32]{BP_geo}}]
Let $-1 \leq k, -1 \leq s \leq n$.
There are natural identifications
\[
	\OC^k \wr \OC^{n,s,\lambda} \wr (A_j) \simeq
	\OC^{n+k+1,s+k+1,\lambda} \wr (A_j),
\qquad
	\OC^{n,s,\lambda} \wr (\OC^k)^{\times \lambda} \wr (A_j) \simeq
	\OC^{n+k+1,s,\lambda} \wr (A_j)	
\]
which are unital and associative in the natural ways.
Moreover, these induce identifications
\[
d^i \wr \Omega^{n,s,\lambda} \wr (A_j) \simeq d^i \wr (A_j)
	\quad
\pi_{i,k} \wr \Omega^{n,s,\lambda} \wr (A_j) \simeq \pi_{i,k} \wr (A_j)
	\quad
s^j \wr \Omega^{n,s,\lambda} \wr (A_j) \wr A \simeq d^j \wr  (A_j)
\]
\[
\Omega^k \wr (d^i) \wr (A_j) \simeq d^{k+i+1} \wr (A_j)
	\quad
\Omega^k \wr (s^j) \wr (A_j) \simeq s^{k+j+1} \wr (A_j)
\]
\end{corollary}


\begin{proof}
Much as in Proposition \ref{ASSOCIDS PROP}, this follows by noting that all squares in the following diagrams are pullback squares.
\begin{equation}\label{LSTRINGS EQ}
\begin{tikzcd}[column sep = 9pt]
		\OC^{n+k+1,s+k+1,\lambda} \arrow{r}{\boldsymbol{V}^k} \wr (A_j)
		\ar{r} \ar{d}
		%\bullet \arrow[dashed]{d} \arrow[dashed]{r}
	&
		\Sigma \wr \OC^{n,s,\lambda} \wr (A_j) \ar{r} \ar{d}
		%\bullet \arrow[dashed]{d} \arrow[dashed]{r}
	&
		\Sigma^{\wr 2} \wr \amalg_j A_j \arrow{r} \ar{d}
	&
		\Sigma \wr \amalg_j A_j \ar{d}
\\
		\OC^{n+k+1,s+k+1,\lambda} \arrow{r}{\boldsymbol{V}^k} \arrow[d]
	&
		\Sigma \wr \OC^{n,s,\lambda} \arrow{r}{\boldsymbol{V}^n} \arrow[d]
	&
		\Sigma^{\wr 2} \wr \Sigma_{\mathfrak C}^{\amalg l} \ar{r}
	&
		\Sigma \wr \Sigma_{\mathfrak C}^{\amalg l}
\\
		\OC^k \arrow{r}{\boldsymbol{V}^k}
	&
		\Sigma \wr \Sigma_{\mathfrak C}
\\ % NEW DIAGRAM ------------------------------
		\OC^{n+k+1,s,\lambda} \arrow{r}{\boldsymbol{V}^n} \wr (A_j)
		\ar{r} \ar{d}
		%\bullet \arrow[dashed]{d} \arrow[dashed]{r}
	&
		\Sigma \wr \amalg \left(\OC^{k}\right)^{\times \lambda} \wr (A_j)
		\ar{r} \ar{d}
		%\bullet \arrow[dashed]{d} \arrow[dashed]{r}
	&
		\Sigma \wr \amalg_j \Sigma \wr A_j \arrow{r} \ar{d}
	&
		\Sigma^{\wr 2} \wr \amalg_j A_j \arrow{r} \ar{d}
	&
		\Sigma \wr \amalg_j A_j \ar{d}
\\
		\OC^{n+k+1,s,\lambda} \arrow{r}{\boldsymbol{V}^n} \arrow[d]
	&
		\Sigma \wr \amalg \left(\OC^{k}\right)^{\times \lambda} \arrow{r}{\boldsymbol{V}^k} \arrow[d]
	&
		\Sigma \wr \amalg_l \Sigma \wr \Sigma_{\mathfrak{C}} \ar{r}
	&
		\Sigma^{\wr 2} \wr \amalg_l \Sigma_{\mathfrak{C}} \ar{r}
	&
		\Sigma \wr \Sigma_{\mathfrak C}^{\amalg l}
\\
		\OC^{n,s,\lambda} \arrow{r}{\boldsymbol{V}^n}
	&
		\Sigma \wr \Sigma_{\mathfrak C}^{\amalg l}
            \end{tikzcd}
      \end{equation}
\end{proof}

\subsubsection{Filtration of free extensions}


We now return to discussing
\eqref{OU EQ1}.
The preceeding discussion shows that
({\color{blue} $N_{n,l}^{(\O,X,Y)}$ notation undiscussed})
\begin{equation}\label{1STRED EQ}
\O[u] \simeq
\mathop{\colim}\limits_{(\Delta \times \Delta)^{op}}
\left(
	\mathsf{Lan}_{\left(\Omega_{\mathfrak C}^{n,\lambda_l} \to \Sigma_{\mathfrak C}\right)^{op}} N_{n,l}^{(\O,X,Y)}
\right)
\end{equation}
where $\lambda_l$ is the partition on 
\[
\langle \langle l \rangle \rangle
=
\{-\infty,-l,\cdots,-1,0,1,\cdots,+\infty\}
\]
such that $\left(\lambda_l\right)_a = \{-\infty\}$.
Moreover, the simplicial operators in the $l$ direction are described by antisymmetric functions $\langle \langle l \rangle \rangle
 \to \langle \langle l' \rangle \rangle
$
which are given by \eqref{SIMPLEXP EQ} on non-negative values.


\begin{proposition}\label{EXTENTREE PROP}
The double simplicial realization
$|\Omega_{\mathfrak C}^{n,\lambda_l}|$,
which we call the \textit{extension tree category}
and denote
$\OC^e$, has as objects the 
$\{\O,X,Y\}$-labeled trees
and as arrows tall maps $\varphi \colon T \to S$ such that
\begin{enumerate}[label=(\roman*)]
\item if $T_v$ has a $X$-label, then 
$S_v \in \Sigma_{\mathfrak{C}}$ and
$S_v \in \Sigma_{\mathfrak{C}}$ has a $X$-label;
\item if $T_v$ has a $Y$-label, then 
$S_v \in \Sigma_{\mathfrak{C}}$ and
$S_v \in \Sigma_{\mathfrak{C}}$ has either a $X$-label or a $Y$-label;
\item if $T_v$ has a $\O$-label, then 
$S_v \in \Sigma_{\mathfrak{C}}$ has only $X$ and $\O$ label;
\end{enumerate}
\end{proposition}


\begin{proof}
This is a direct analogue of \cite[Prop. 5.41]{BP_geo}, and the proof therein carries through without significant changes, so we only sketch the key arguments.
Firstly, it is straightforward to check that for each fixed $l$,
the realization $|\Omega^{n,\lambda_l}|$
in the $n$ direction is the category 
$\Omega^{t,\lambda_l}_{\mathfrak{C}}$
with objects the $\langle \langle l \rangle \rangle$-labeled trees and and maps the tall label maps which are inert on colors other than $-\infty$/$\O$ (cf. \cite[Rem. 5.36]{BP_geo}).
Moreover, maps 
$T \to S$
in 
$\Omega^e_{\mathfrak{C}}$
canonically factor 
as
$T \to T' \to S$
where the first map is a a relabel map (i.e. an underlying isomorphism of trees that simply changes labels) and the second map is a label map, 
so that the result follows from the observation that relabel maps in 
$\Omega^e_{\mathfrak{C}}$
correspond to objects of  
$\Omega^{t,\lambda_1}$
while label maps correspond to maps of
$\Omega^{t,\lambda_0}$.
\end{proof}


(The natural analogue of) \cite[Prop 5.37]{BP_geo}
applied to 
\eqref{1STRED EQ}
(note that the ``natural transformation component of differential operators are isomorphisms for the $n$ direction follows from \eqref{NMONMULTTR EQ} and \eqref{NMONIDTR EQ}''
while in the $l$ direction it follows since the associate maps of tuples (cf. Proposition \ref{SPANPIECEJ PROP}) are the identity in each coordinate) now yields
\begin{equation}\label{2NDRED EQ}
\O[u] \simeq
	\mathsf{Lan}_{\left(\Omega_{\mathfrak C}^{e} \to
	\Sigma_{\mathfrak C}\right)^{op}} N^{(\O,X,Y)}
\end{equation}

We can explain how the desired filtration 
\eqref{FILT EQ} is obtained:
after replacing  
$\Omega_{\mathfrak C}^{e}$
in \eqref{2NDRED EQ}
with a suitable subcategory 
$\widehat{\Omega}_{\mathfrak C}^{e}$,
the filtration will follow from a filtration 
$\widehat{\Omega}_{\mathfrak C}^{e}[\leq k]$
of 
$\widehat{\Omega}_{\mathfrak C}^{e}$ itself.


\begin{definition}
$\widehat{\Omega}_{\mathfrak C}^{e} \hookrightarrow \Omega_{\mathfrak C}^{e}$
is the full subcategory of those labeled trees whose underlying tree is alternating, active nodes are labeled by $\O$ 
and passive nodes are labeled by $X$ or $Y$.

Moreover, writing $|T| = |V^X(T)|+ |V^Y(T)|$,
\begin{enumerate}[label=(\roman*)]
\item $\widehat{\Omega}_{\mathfrak C}^{e}[\leq k]$ (resp. $\widehat{\Omega}_{\mathfrak C}^{e}[k]$)
is the full subcategory of those $T$ with $|T| \leq k$ ($|T|=k$);
\item $\widehat{\Omega}_{\mathfrak C}^{e}[\leq k \setminus Y]$ (resp. $\widehat{\Omega}_{\mathfrak C}^{e}[k \setminus Y]$)
is the further subcategory of those $T$ with $|T|_Y \neq k$.
\end{enumerate}
Subcategories $\OC^a[\leq k], \OC^a[k]$ of $\OC^a$ are defined similarly.
\end{definition}



The following results follow exactly as in the cited result from 
\cite{BP_geo} that they generalize.

\begin{lemma}[{cf. \cite[Cor. 5.53, Lemma 5.58]{BP_geo}}]
\label{LANINT LEM}

	$\widehat\Omega_{\mathfrak C}^e \into 
	\Omega_{\mathfrak C}^e$
	is $\Ran$-initial over $\SC$.
     
	Similarly, $\widehat\Omega_{\mathfrak C}^e[\leq k-1] \into 
\widehat\Omega_{\mathfrak C}^e[\leq k \setminus Y]$
	is $\Ran$-initial over $\SC$.
\end{lemma}

\begin{remark}[{cf. \cite[Remark 5.57]{BP_geo}}]
      \label{OEFIB REM}
      The following diagram
      \begin{equation}
            \begin{tikzcd}
                  \widehat\Omega_{\mathfrak C}^e[k \setminus Y] \arrow[rr, hookrightarrow] \arrow[dr]
                  &&
                  \widehat\Omega_{\mathfrak C}^e[k] \arrow[dl]
                  \\
                  &
                  \Omega_{\mathfrak C}^a[k]
            \end{tikzcd}
      \end{equation}
	is a map of Grothendieck fibrations
	such that fibers over $T \in \Omega_{\mathfrak C}^a[k]$ are the punctured cube and cube categories
      \begin{equation}
            (Y \to X)^{\times V^{in}(T)} - Y^{\times V^{in}(T)},
            \qquad
            (Y \to X)^{\times V^{in}(T)}.
      \end{equation}
	for $V^{in}(T)$ the set of inert vertices.
\end{remark}


We now finally describe the filtration \eqref{FILT EQ}.
\begin{definition}\label{FILTSTAGE DEF}
Let $\O_k$ denote the left Kan extension
\begin{equation}
\begin{tikzcd}
	|[alias = A]|
	\widehat\Omega_{\mathfrak C}^e[\leq k]^{op}
	\arrow[r, "\tilde N"] \arrow[d, "\mathsf{lr}"']
&
	\V
\\
	\SC^{op}
	\arrow[ur, "\O_k"', ""{name = B}]
	\arrow[Rightarrow, from = A, to = B]
\end{tikzcd}
\end{equation}
\end{definition}

Since $\widehat\Omega_{\mathfrak C}^e[\leq 0] \simeq \SC$
and the nerve of $\widehat \Omega_{\mathfrak C}^e$ is the union of the nerves of the $\widehat\Omega_{\mathfrak C}^e[\leq k]$
the desired filtration \eqref{FILT EQ} follows.


Moreover, to prove Theorem \ref{THM1_C}, we will to
understand how each filtration stage is built from the previous one.
To do so, we consider the following diagram, where the left square is a pushout at the level of nerves (cf. \cite[(5.65)]{BP_geo}),
so that after taking $\mathsf{Lan}$
one obtains the right pushout
(that the top right corner is indeed $\O_{k-1}$ follows from Lemma \ref{LANINT LEM})
\begin{equation}\label{FILTLAN EQ}
\begin{tikzcd}
		\widehat\Omega_{\mathfrak C}^e[k \setminus Y] \arrow[r] \arrow[d]
	&
		\widehat\Omega_{\mathfrak C}^e[\leq k \setminus Y] \arrow[d]
&
		\mathsf{Lan}_{\widehat\Omega_{\mathfrak C}^e[k \setminus Y]^{op}} \tilde N \arrow[r] \arrow[d]
	&
		\O_{k-1} \arrow[d]
\\
		\widehat\Omega_{\mathfrak C}^e[k] \arrow[r]
	&
		\widehat\Omega_{\mathfrak C}^e[\leq k]
&
		\mathsf{Lan}_{\widehat \Omega_{\mathfrak C}^e[k]^{op}} \tilde N \arrow[r]
	&
		\O_k
\end{tikzcd}
\end{equation}


Lastly, Remark \ref{OEFIB REM} allows us to 
rewrite the left vertical map of can extensions in \eqref{FILTLAN EQ}


\begin{proposition}[{cf. \cite[Prop. 5.66]{BP_geo}}]
      \label{FILTPUSH PROP}
%      Suppose $\V$ is a closed monoidal category, and fix a $G$-set $\mathfrak C$.
%      For $n \geq 0$, let $\O_n$ denote the $n$-stage of the filtration \eqref{FILT_EQ} in $\Sym^{G, \mathfrak C}(\V)$
%      of the pushout map from \eqref{OU_EQ}.
For each $\mathfrak C$-signature $C \in \Sigma_{\mathfrak C}$, one has a pushout diagram
      \vspace{-10pt}
\begin{align}\label{FILTPUSH EQ}
\begin{tikzcd}[ampersand replacement=\&]
	\mathop{\coprod}\limits_{[T] \in \Iso(C \downarrow \Omega_{\mathfrak C}^a[k])}
	\left(
		\mathop{\bigotimes}\limits_{v \in V^{ac}(T)} \O(T_v) \otimes
		Q^{in}_T[u]
	\right) \cdot_{\Aut_{\OC} (T)} \Aut_{\SC}(C)
		\arrow[r] \arrow[d]
\&[15pt]
	\O_{k-1}(C) \arrow[d]
\\                  
	\mathop{\coprod}\limits_{[T] \in \Iso(C \downarrow \Omega_{\mathfrak C}^a[k])}
	\left(
		\mathop{\bigotimes}\limits_{v \in V^{ac}(T)} \O(T_v) \otimes
		\mathop{\bigotimes}\limits_{v \in V^{in}(T)} Y(T_v)
	\right) \cdot_{\Aut_{\OC}(T)} \Aut_{\SC}(C)
		\arrow[r]
\&
\O_k(C)
\end{tikzcd}
\end{align}
      where $Q^{in}_T[u]$ denotes the source of the pushout-product map
      \begin{equation}
            \mathop{\mathlarger{\mathlarger{\mathlarger{\square}}}}_{v \in V^{in}(T)} u(T_v): Q^{in}_T[u] \to \bigotimes_{v \in V^{in}(T)} Y(T_v).
      \end{equation}
\end{proposition}

\begin{proof}
Computing the left Kan extensions iteratively by first extending to
$G \ltimes \OC^a[k]^{op}$, 
Remark \ref{OEFIB REM} allows us to rewrite 
the leftmost map in \eqref{FILTLAN EQ} as
\begin{equation}\label{FILTLANFIN EQ}
	\mathsf{Lan}{(\OC^a[k] \to \SC)^{op}}\left(
		\bigotimes_{v \in V^{ac}(T)}\O(T_v) \otimes
		\mathop{\mathlarger{\mathlarger{\mathlarger{\square}}}}\limits_{v \in V^{in}(T)} u(T_v)
		\right).
\end{equation}
	The description of the leftmost maps in \eqref{FILTPUSH EQ} follows since the undercategories
	$C \downarrow \OC^a[k]^{op}$ are groupoids.
\end{proof}



\subsubsection{Filtration of free extensions in the equivariant case}


When \eqref{OU EQ} is a $G$-equivariant diagram (note that the maps are still assumed the identity on colors),
naturality with regard to data as in \eqref{COLORCHNAT EQ}
shows that in 
\eqref{2NDRED EQ} the left Kan extension 
is compatible with $G$-equivariance, i.e. it is in fact 
$
\mathsf{Lan}^G \colon 
\left(
\mathsf{WSpan}_l
(\Sigma^{op}_{\bullet},\mathcal{V})
\right)^G
\to
\left(
\mathsf{Fun}
(\Sigma^{op}_{\bullet},\mathcal{V})
\right)^G
$.
Hence, by (the span version of) Lemma \ref{EQUIVFUNCONV LEM},
we further have the alternative formula
\begin{equation}\label{3RDRED EQ}
\O[u] \simeq
	\mathsf{Lan}_{\left(G^{op} \ltimes \Omega_{\mathfrak C}^{e} \to
	G^{op} \ltimes \Sigma_{\mathfrak C}\right)^{op}} N^{(\O,X,Y)}
\end{equation}
and, since the subcategories 
$\widehat{\Omega}_{\mathfrak{C}}^e$,
$\widehat{\Omega}_{\mathfrak{C}}^e[\leq k]$,
$\widehat{\Omega}_{\mathfrak{C}}^e[k]$
are compatible with the $G$-action, by replacing these categories with 
$G^{op} \ltimes \widehat{\Omega}_{\mathfrak{C}}^e$,
$G^{op} \ltimes \widehat{\Omega}_{\mathfrak{C}}^e[\leq k]$,
$G^{op} \ltimes \widehat{\Omega}_{\mathfrak{C}}^e[k]$
in Definition \ref{FILTSTAGE DEF}
we see that the right square in \eqref{FILTLAN EQ}
is naturally a square in 
$\mathsf{Sym}(\mathcal{V})^G$.

Hence, by accounting for equivariance
we further obtain the following analogue of 
Proposition \ref{FILTPUSH PROP}




\begin{proposition}[{cf. \cite[Prop. 5.66]{BP_geo}}]
      \label{FILTPUSHG PROP}
%      Suppose $\V$ is a closed monoidal category, and fix a $G$-set $\mathfrak C$.
%      For $n \geq 0$, let $\O_n$ denote the $n$-stage of the filtration \eqref{FILT_EQ} in $\Sym^{G, \mathfrak C}(\V)$
%      of the pushout map from \eqref{OU_EQ}.
For each $\mathfrak C$-signature $C \in \Sigma_{\mathfrak C}$, one has a pushout diagram
      \vspace{-10pt}
\begin{align}\label{FILTPUSHG EQ}
\begin{tikzcd}[ampersand replacement=\&]
	\mathop{\coprod}\limits_{[T] \in \Iso(C \downarrow G^{op} \ltimes \Omega_{\mathfrak C}^a[k])}
	\left(
		\mathop{\bigotimes}\limits_{v \in V^{ac}(T)} \O(T_v) \otimes
		Q^{in}_T[u]
	\right) \cdot_{\Aut_{G^{op} \ltimes \OC}(T)} \Aut_{G^{op} \ltimes \SC}(C)
		\arrow[r] \arrow[d]
\&
	\O_{k-1}(C) \arrow[d]
\\                  
	\mathop{\coprod}\limits_{[T] \in \Iso(C \downarrow G^{op} \ltimes \Omega_{\mathfrak C}^a[k])}
	\left(
		\mathop{\bigotimes}\limits_{v \in V^{ac}(T)} \O(T_v) \otimes
		\mathop{\bigotimes}\limits_{v \in V^{in}(T)} Y(T_v)
	\right) \cdot_{\Aut_{G^{op} \ltimes \OC}(T)} \Aut_{G^{op} \ltimes \SC}(C)
		\arrow[r]
\&
	\O_k(C)
\end{tikzcd}
\end{align}
\end{proposition}



\begin{remark} 
In the equivariant setting one is free to use either \eqref{FILTPUSH EQ} or \eqref{FILTPUSHG EQ}.
In fact, the coproduct summands of the left maps in \eqref{FILTPUSH EQ} are interchanged by the $G$-action, 
and each coproduct summand in \eqref{FILTPUSHG EQ} is then the coproduct of a $G$-conjugacy class of summands of \eqref{FILTPUSH EQ}.
%
%
%The connection between the two formulas is given by Lemma \ref{REDUCELAN LEM},
though some care is needed.
%Namely, \eqref{PUSHOPPRG EQ} generally features fewer coproduct summands but this is compensated by the inductions
%$(-) \cdot_{\mathsf{Aut}_{G \ltimes \Omega^a_{\mathfrak{C}}}(T)} \mathsf{Aut}_{G \ltimes \Sigma_{\mathfrak{C}}}(C)$,
%which produce more terms than the 
%$(-) \cdot_{\mathsf{Aut}_{\Omega^a_{\mathfrak{C}}}(T)} \mathsf{Aut}_{\Sigma_{\mathfrak{C}}}(C)$
%inductions.
\end{remark}




\begin{remark}
Using the identification
$
\left(
\mathsf{WSpan}_l
\left(\Sigma_{\bullet}^{op},\mathcal{V}\right)
\right)^G
\simeq
\mathsf{WSpan}_l
\left(G \ltimes \Sigma_{\bullet}^{op},\mathcal{V}\right)
$
from (the analogue of) Lemma \ref{EQUIVFUNCONV LEM},
one has that the equivariant monad $N^G$
is described by the following analogue of \eqref{NCOLOR EQ}
(see the discussion in \eqref{RHOPURP EQ} for the discussion of the natural transformation
$G^{op} \ltimes \Sigma \wr (-) \to \Sigma \wr G^{op} \ltimes (-)$)
\begin{equation}\label{NCOLORG EQ}
\begin{tikzcd}
	G^{op} \ltimes \Omega^0_{\mathfrak{C}} \wr A \ar{r}{\boldsymbol{V}^0} \ar{d} &
	G^{op} \ltimes \Sigma \wr A  \ar{d} \ar{r} &
	\Sigma \wr G^{op} \ltimes A  \ar{d} \ar{r} &
	\Sigma \wr \mathcal{V}^{op} \ar{r}{\otimes} &
	\mathcal{V}^{op}
\\
	G^{op} \ltimes \Omega^0_{\mathfrak{C}} \ar{r}{\boldsymbol{V}^0} \ar{d} &
	G^{op} \ltimes \Sigma \wr \Sigma_{\mathfrak{C}} \ar{r} &
	\Sigma \wr G^{op} \ltimes \Sigma_{\mathfrak{C}} 
\\
	G^{op} \ltimes \Sigma_{\mathfrak{C}}
\end{tikzcd}
\end{equation}
This diagram suggests an alternative way to handle the equivariant case which is a little closer in spirit to the genuine equivariant operad work in \cite{BP_geo}.
Rather than using the naturality of filtrations with respect to change of color data as in \eqref{COLORCHNAT EQ},
one could instead regard the composites
$G^{op} \ltimes \OC^n \to 
G^{op} \ltimes \Sigma \wr \OC^{n-k-1} \to
\Sigma \wr G^{op} \ltimes  \OC^{n-k-1}$
as the formal equivariant analogues of the vertex functors.
It is then not hard to verify that all claims in Propositions \ref{CATDIAG PROP} and \ref{CATDIAG2 PROP},
so that all work in \S \ref{NONEQMON SEC} 
and ({\color{blue} fill in section reference})
generalizes by replacing occurrences 
of symbols like $\OC^{n}$ and $A$ with 
$G^{op} \ltimes \OC^{n}$ and $G^{op} \ltimes A$
(we note that, in particular,
in mimicking \eqref{WRADEF EQ}
this leads us to define categories 
$\left(G^{op} \ltimes \OC^{n}\right)
\wr
\left(G^{op} \ltimes A\right)$,
which are canonically isomorphic to the categories
$G^{op} \ltimes \OC^{n}
\wr A$ as in \eqref{NCOLORG EQ}).
\end{remark}


{\color{blue} HERE}



% ------------------------------ homotopy stuff ------------------------------

We end this section by discussing the homotopical nature of this filtration.
Again, the results from \cite{BP_geo} are sufficiently robust.

We say $X \to Y \in \Sym^{G, \mathfrak C}(\V)$ is a \textit{level genuine cofibration} if for each $C \in G \ltimes \SC^{op}$,
$X(C) \to Y(C)$ is a genuine cofibration in $\V^{\Aut(C)}$;
we say $X$ is \textit{level genuine cofibrant} if $\varnothing \to X$ is a level genuine cofibration.

\begin{lemma}[{cf. \cite[5.73]{BP_geo}}] % EXMAINLEM LEM
      \label{EXMAINLEM LEM}
      Suppose $\mathcal{V}$ is a cofibrantly generated closed monoidal model category
      with cellular fixed points and
      with cofibrant symmetric pushout powers (Defn \ref{CSPP_DEF}).
      
      Let $\O \in \mathsf{Sym}^{G,\mathfrak C}(\mathcal{V})$
      be level genuine cofibrant
      and  
      $u: X \to Y$ in $\Sym^{G, \mathfrak C}(\V)$ a level genuine cofibration. 
      Let $T \in G \ltimes \Omega_{\mathfrak C}^a[k]^{op}$ and write $C = \mathsf{lr}(T)$.
      Then
      \begin{equation}\label{EXMAINLEM EQ}
            \left(
                  \bigotimes\limits_{v \in V^{ac}(T)}\P(T_v) \otimes
                  \underset{v \in V^{in}(T)}
                  {\mathlarger{\mathlarger{\mathlarger{\square}}}}
                  u(T_v)
            \right) 
            \mathop{\otimes}\limits_{\mathsf{Aut}(T)} \mathsf{Aut}(C).
      \end{equation}
      is a genuine cofibration in 
      $\mathcal{V}^{\mathsf{Aut}(C)}_{\text{gen}}$,
      where the automorphism groups are either all taken in $G \ltimes \OC$ or all taken in $\Omega_{\mathfrak C}$,
      and which is trivial if $k \geq 1$ and $u$ is trivial.	
\end{lemma}
\begin{proof}
      This follows exactly as in \cite{BP_geo}, as we may take the automorphism groups to live in either
      $\Sigma \wr (G^{op} \ltimes \Sigma_{\mathfrak C})$ or $\Sigma \wr \Sigma_{\mathfrak C}$.
\end{proof}







\subsubsection{Pushouts and change of color}



First note that, by a variation of the arguments in \eqref{OU EQ1}, \eqref{1STRED EQ}, \eqref{2NDRED EQ},
any pushout 
\[
\begin{tikzcd}
	A \ar{r} \ar{d} & \mathcal{O} \ar{d}
\\
	B \ar{r} & \mathcal{P}
\end{tikzcd}
\]
has a description
\begin{align*}
  \mathcal{P}
  &
    \simeq \colim_{[l] \in \Delta^{op}} 
    B_l \left( \O, A, A, A, B \right)
  \\
  &
    \simeq \colim_{[l] \in \Delta^{op},[n] \in \Delta^{op}} 
    B_l \left( \mathbb F^{\circ n+1} \O, \mathbb F^{\circ n+1} A, 
    \mathbb F^{\circ n+1} A, \mathbb F^{\circ n+1} A, \mathbb F^{\circ n+1} B \right)
\\
&
    \simeq \colim_{[l] \in \Delta^{op},[n] \in \Delta^{op}} 
    \mathsf{Lan} N \circ \left( N^{\circ n} \iota \O 
		\amalg_{\mathsf{F}}
	\left( N^{\circ n} \iota X\right)^{\amalg_{\mathsf{F}} 2l +1}
		\amalg_{\mathsf{F}}
	N^{\circ n} \iota Y \right)
\\
&	
	\simeq
	\mathop{\colim}\limits_{(\Delta \times \Delta)^{op}}
\left(
	\mathsf{Lan}_{\left(\Omega_{\mathfrak C}^{n,\lambda^a_l} \to \Sigma_{\mathfrak C}\right)^{op}} N_{n,l}^{(\O,A,B)}
\right)
\\
&	
	\simeq
	\mathsf{Lan}_{\left(\Omega_{\mathfrak C}^{p} \to
	\Sigma_{\mathfrak C}\right)^{op}} N^{(\O,A,B)}
    \stepcounter{equation}\tag{\theequation}\label{OU EQ2}
    % =: \colim_{n,l} \mathsf{Lan} \hat N_{n,l}^{(\O,X,Y)},
\end{align*}
where the partition $\lambda^a_l$ in the fourth line is the active partition with $\left(\lambda^a_l\right)_a = \{1,\cdots,l\}$
and, similarly to Proposition \ref{EXTENTREE PROP},
the double realization
$\OC^p \simeq |\Omega_{\mathfrak C}^{n,\lambda^a_l}|$
%
%in $\mathsf{Op}^{\mathfrak{C}}$ can be described as a left Kan extension
%\[
%\begin{tikzcd}
%	\Omega_{\mathfrak{C}}^{p,op} \ar{rr}{N^{(B,A,\mathcal{O})}} \ar{d}[swap]{\mathsf{lr}} &&
%	\mathcal{V}
%\\
%	\Sigma_{\mathfrak{C}}^{op} \ar[dashed]{rru}[swap]{\mathcal{P}}
%\end{tikzcd}
%\]
%Where $\Omega^p_{\mathfrak{C}}$ 
is the category whose objects are the
$\{\mathcal{O},A,B\}$-labeled trees
%$T \in \Omega_{\mathfrak{C}}$ together with a map
%$V(T) \to \{\mathcal{O},A,B\}$ (i.e. a labeling of the vertices of $T$ by $\{\mathcal{O},A,B\}$)
and whose arrows are tall maps $T \to S$ such that
\begin{enumerate}[label=(\roman*)]
\item if $v \in V(T)$ is $A$-labeled, then all vertices in $S_{v}$ are $A$-labeled
\item if $v \in V(T)$ is $B$-labeled, then all vertices in $S_{v}$ are either $A$-labeled or $B$-labeled
\item if $v \in V(T)$ is $\mathcal{O}$-labeled, then all vertices in $S_{v}$ are either $A$-labeled or $\mathcal{O}$-labeled
\end{enumerate}
Moreover, one has the formula
\begin{equation}\label{NBAO EQ}
N^{(B,A,\mathcal{O})}(T) = 
\left(\bigotimes_{v \in V_B(T)} B(T_v) \right) \otimes
\left(\bigotimes_{v \in V_A(T)} A(T_v) \right) \otimes
\left(\bigotimes_{v \in V_{\mathcal{O}}(T)} \mathcal{O}(T_v) \right)
\end{equation}





\begin{lemma}\label{BASICPUSH LEMMA}
Let $f \colon \mathfrak{C} \hookrightarrow \mathfrak{D}$ be an injective map of colors, $A \to B$ a map in $\mathsf{Op}^{\mathfrak{C}}$
and $\mathcal{O} \in \mathsf{Op}^{\mathfrak{D}}$.

Then if the leftmost diagram in $\mathsf{Op}^{\mathfrak{D}}$ is a pushout, so is the adjoint rightmost diagram in $\mathsf{Op}^{\mathfrak{C}}$.
\[
\begin{tikzcd}
	f_! A \ar{r} \ar{d} & \mathcal{O} \ar{d}
&
	A \ar{r} \ar{d} & f^{\**} \mathcal{O} \ar{d}
\\
	f_! B \ar{r} & \mathcal{P}
&
	B \ar{r} & f^{\**} \mathcal{P}
\end{tikzcd}
\]
\end{lemma}


\begin{proof}[Proof of Lemma \ref{BASICPUSH LEMMA}]

We start by noting that the top composite in the diagram
\[
\begin{tikzcd}
	\Omega_{\mathfrak{C}}^{p,op} \ar{r}{f_{\**}} \ar{d}[swap]{\mathsf{lr}} &
	\Omega_{\mathfrak{D}}^{p,op} \ar{rr}{N^{(f_!B,f_!A,\mathcal{O})}} \ar{d}[swap]{\mathsf{lr}} &&
	\mathcal{V}
\\
	\Sigma_{\mathfrak{C}}^{op} \ar{r}{f_{\**}} &
	\Sigma_{\mathfrak{D}}^{op} %\ar[dashed]{rru}[swap]{\mathcal{P}}
\end{tikzcd}
\]
is $N^{(B,A,f^{\**}\mathcal{O})}$, so that our intended result is that the induced map
\begin{equation}\label{LANISO EQ}
	\mathsf{Lan}_{\Omega_{\mathfrak{C}}^{p,op} \to \Sigma_{\mathfrak{C}}^{op}}
	N^{(B,A,f^{\**}\mathcal{O})}
\xrightarrow{\simeq}
\left(
	\mathsf{Lan}_{\Omega_{\mathfrak{D}}^{p,op} \to \Sigma_{\mathfrak{D}}^{op}}
	N^{(f_!B,f_!A,\mathcal{O})}
\right) \circ f_{\**}
\end{equation}
is an isomorphism. To see this, first let $\widehat{\Omega}^p_{\mathfrak{D}}$
be the full subcategory of $\Omega^p_{\mathfrak{D}}$
such that if 
$v \in V(T)$ is $A$ or $B$-labeled then $T_v \in \Sigma_{\mathfrak C}$.
It follows from \eqref{NBAO EQ} that $N^{(f_! B, f_! A, \mathcal{O})}(T) = \emptyset$ whenever $T \not \in \widehat{\Omega}^p_{\mathfrak{D}}$,
and it is straightforward to check that 
$\widehat{\Omega}^p_{\mathfrak{D}}$
is a sieve of $\Omega^p_{\mathfrak{D}}$, i.e. for any map $T \to S$ in $\Omega^p_{\mathfrak{D}}$ such that $S \in \widehat{\Omega}^p_{\mathfrak{D}}$ then $T \in \widehat{\Omega}^p_{\mathfrak{D}}$.
From this is follows that 
$N^{(f_!B,f_!A,\mathcal{O})}$
is the left Kan extension of its restriction to 
$\widehat{\Omega}^p_{\mathfrak{D}}$, 
so we are free to replace
$\Omega^p_{\mathfrak{D}}$
with
$\widehat{\Omega}^p_{\mathfrak{D}}$
in \eqref{LANISO EQ}.

The result now follows by noting that,
for each $C \in \Sigma_{\mathfrak{C}}$
the inclusion
$(C \downarrow \Omega^p_{\mathfrak{C}})
\to
(C \downarrow \widehat{\Omega}^p_{\mathfrak{D}})
$
has a right adjoint given by the assignment $T \mapsto T - E_{\mathfrak{D} \setminus \mathfrak{C}}(T)$,
where 
$E_{\mathfrak{D} \setminus \mathfrak{C}}(T)$ is the set of edges not in $\mathfrak{C}$ (the fact that this has a natural labeling follows since all the edges being collapsed must connect $\mathcal{O}$-vertices, so that there is no ambiguity as to how to label the vertices of $T - E_{\mathfrak{D} \setminus \mathfrak{C}}(T)$).
\end{proof}


\begin{corollary}\label{FGTPUSH COR}
Suppose that $f,g$ on the left pushout diagram below are both injective on colors.
\[
\begin{tikzcd}
	A \ar{r}{f} \ar{d}[swap]{g} & \mathcal{O} \ar{d}
&
	A \ar{r} \ar{d}[swap]{g} & f^{\**} \mathcal{O} \ar{d}
\\
	B \ar{r}{\tilde{f}} & \mathcal{P}
&
	B \ar{r} & \tilde{f}^{\**} \mathcal{P}
\end{tikzcd}
\]
Then the rightmost diagram is also a pushout diagram.
\end{corollary}

\begin{proof}
This follows by adding to both $A$ and $\mathcal{O}$ a disjoint trivial operad with object set $\mathfrak{C}(B) \setminus \mathfrak{C}(A)$. In doing so one does not alter the pushout, but the map $g$ now becomes a fixed color map, so that the previous result can be applied to the left pushout.
\end{proof}


\begin{corollary}
      \label{LOCALISO_COR}
      Suppose that we have a pushout in $\Op(\V)$ such that $f,g$ are both injective on colors.
      \[
            \begin{tikzcd}
                  A \arrow[d, "g"'] \arrow[r, "f"]
                  &
                  \O \arrow[d]
                  \\
                  B \arrow[r]
                  &
                  \P
            \end{tikzcd}
      \]
      If $A \to B$ is a local isomorphism, then so is $\O \to \P$.
\end{corollary}
\begin{proof}
      A map $g: A \to B$ is a local isomorphism iff $A \to g^{\**}B$ is an isomorphism in $\Op^{\mathfrak C(A)}(\V)$.
      Thus the result follows by applying Corollary \ref{FGTPUSH COR} with the notations of $B$ and $\O$ switched.      
\end{proof}


% {\color{OliveGreen} % ---------------------------------------- OLIVE GREEN ----------------------------------------
%   We end by proving a technical step in the above proof, of greater specificity than Corollary \ref{LOCALISO_COR}.
%   \begin{lemma}
%         \label{LOCALISO_LEM}
%         Suppose that we have a pushout in $\Op(\V)$ such that $f,g$ are both injective on colors.
%         \[
%               \begin{tikzcd}
%                     A \arrow[d, "g"'] \arrow[r, "f"]
%                     &
%                     \O \arrow[d]
%                     \\
%                     B \arrow[r]
%                     &
%                     \P
%               \end{tikzcd}
%         \]
%         If $A \to B$ is a local isomorphism, then so is $\O \to \P$.
%   \end{lemma}
%   \begin{proof}
%         We denote the sets of colors by
%         \[
%               \mathfrak C = \mathfrak C(A),
%               \qquad
%               \mathfrak C' = \mathfrak C(B),
%               \qquad
%               \mathfrak D = \mathfrak C(\O),
%               \qquad
%               \mathfrak D' = \mathfrak C(\P),
%         \]
%         and fix a $\mathfrak D$-signature $C$ for the duration of the proof.
%         We will show that $\O(C) \to \P(C)$ is an isomorphism
%         by following the string of natural isomorphisms below:
%         \begin{equation}
%               \label{LOCALISO_EQ}
%               \P(C) =
%               \lim_{\Omega_{\mathfrak D'}^p \downarrow C} N =
%               \lim_{\hat\Omega_{\mathfrak D'}^p \downarrow C} N =
%               \lim_{\hat \Omega_{\mathfrak D}^p \downarrow C} N =
%               \lim_{\hat\Omega_{\mathfrak D}^{B, \O} \downarrow C} N = 
%               \lim_{\hat\Omega_{\mathfrak D}^{A,\O} \downarrow C} N =
%               \lim_{\Omega_{\mathfrak D} \downarrow C} N =
%               \O(C),
%         \end{equation}
%         where $N = N^{(B,A,\O)}$ and it's restrictions.
        
%         Let us define the sequence of full subcategories of $\Omega_{\mathfrak D}^p$, following Lemma \ref{BASICPUSH LEMMA}, written in \eqref{LOCALISO_EQ}.
%         \begin{itemize}
%         \item Let $\hat \Omega_{\mathfrak D'}^p$
%               be spanned by those trees $\set{B,A,\O}$-labeled trees $T$ in $\Omega_{\mathfrak D'}$ such that
%               if $v \in V(T)$ is $A$-labeled (resp. $B$-labeled, $\O$-labeled), then
%               $T_v \in \Sigma_{\mathfrak C}$ (resp. $\Sigma_{\mathfrak C'}$, $\Sigma_{\mathfrak D}$).
%         \item Let $\hat \Omega_{\mathfrak D}^p \subseteq \hat \Omega_{\mathfrak D'}^p$ be spanned by those objects whose udnerlying colored tree lives in $\Omega_{\mathfrak D} \subseteq \Omega_{\mathfrak D'}$.
%         \item Let $\hat \Omega_{\mathfrak D}^{B,\O} \subseteq \hat \Omega_{\mathfrak D}^p$ be spanned by those objects with no $A$-labeled vertices.
%         \item Let $\hat \Omega_{\mathfrak D}^{A,\O} \subseteq \hat \Omega_{\mathfrak D}^p$ be spanned by those objects with no $B$-labeled vertices.
%         \end{itemize}

%         The first two isomorphisms follow by \eqref{NBAO EQ} and the associated discussion in the proof of Lemma \ref{BASICPUSH LEMMA}.
        
%         Next, given $(T \to C) \in \Omega_{\mathfrak D'}^p \downarrow C$, we note that the roots and leaves of $T$ must in fact be $\mathfrak D$-colored,
%         and thus for all maximal outer faces $U \subseteq T$ whose vertices are all $B$-labeled,
%         the roots and leaves of $U$ must be $\mathfrak C$-colored.
%         Thus if we ``collapse'' all of these subtrees, the resulting tree will not have any colors in $\mathfrak C'$.
%         That is, following a variation on the arguments from \cite[Prop. 5.48, Cor. 5.52]{BP_geo},
%         there exist right retractions
%         \[
%               \hat \Omega_{\mathfrak D'}^{p} \downarrow C \xrightarrow{\mathsf{lr}_B}
%               \hat \Omega_{\mathfrak D'}^{p,a} \downarrow C 
%               =
%               \hat \Omega_{\mathfrak D}^{p,a} \downarrow C \xleftarrow{\mathsf{lr}_B}
%               \hat \Omega_{\mathfrak D}^{p} \downarrow C 
%         \]
%         where $\hat \Omega_{\mathfrak D'}^{p,a} \subseteq \hat \Omega_{\mathfrak D'}^{p}$
%         and $\hat \Omega_{\mathfrak D}^{p,a} \subseteq \hat \Omega_{\mathfrak D}^{p}$
%         are full subcategories spanned by those objects whose underlying tree is alternating with active nodes labeled by $B$ \footnote{
%           More information on this terminology can be found in \cite{BP_geo} near
%           Definitions 5.43, 5.47 and the cited results.}.
%         This gives the third isomorphism in \eqref{LOCALISO_EQ}.

%         Now, consider the construction $(-)_{A \mapsto B}$ which takes a tree $T \in \hat\Omega_{\mathfrak D}^{p}$
%         and changes all $A$-labels to $B$-labels, so $(T)_{A \mapsto B} \in \hat\Omega_{\mathfrak D}^{B,\O}$.
%         This is functorial on maps which preserve $\O$-labels,
%         those maps $T \to S$ such that if $v \in V(T)$ is $\O$-labeled, then so is every vertex in $S_v$.
%         Moreover, there is a map $(T)_{A \mapsto B} \to T$ for all $T$, natural in such maps.

%         We note that any map $T \to S$ with $S \in \hat \Omega_{\mathfrak D}^{B,\O}$ and $T \in \hat\Omega_{\mathfrak D}^p$ factors $T \to T' \to S$ where $T' \to S$ preserves $\O$-labels.
%         Thus, we may follow the proof of \cite[Lemma 5.57]{BP_geo} to show that
%         $\hat\Omega_{\mathfrak D}^{B,\O} \downarrow C$ is initial in $\hat\Omega_{\mathfrak D}^p \downarrow C$, given the fourth isomorphism.
        
%         Finally, the fifth isomorphism follows as $A \to B$ is a local isomorphism,
%         and the last follows from applying \cite[Prop 5.37]{BP_geo} to the free resolution of $\O$.      
%   \end{proof}
% } % ---------------------------------------- OLIVE GREEN ----------------------------------------



\subsection{Free operads and change of color}


Let $f \colon \mathfrak{C} \to \mathfrak{D}$ be an inclusion of colors.
Our goal in this section is to show that the pullback of a free $\mathfrak{D}$-colored operad is a free $\mathfrak{C}$-colored operad, in a natural way.

More precisely, our goal is to identify a functor 
$\widehat{f^{\**}}$ such that the diagram below commutes (up to natural isomorphism).
\begin{equation}\label{HATFST EQ}
\begin{tikzcd}
	\mathsf{Sym}^{\mathfrak{D}} \ar{r}{\mathbb{F}_{\mathfrak{D}}} \ar[dashed]{d}[swap]{\widehat{f^{\**}}} &
	\mathsf{Op}^{\mathfrak{D}} \ar{d}{f^{\**}}
\\
	\mathsf{Sym}^{\mathfrak{C}} \ar{r}[swap]{\mathbb{F}_{\mathfrak{C}}} &
	\mathsf{Op}^{\mathfrak{C}}
\end{tikzcd}
\end{equation}
We now introduce some notation.
We will write 
$\Omega^0_{\mathfrak{C},\mathfrak{D}} \subseteq \Omega^0_{\mathfrak{D}}$
for the full subcategory of those trees whose root and leaves are labeled by $\mathfrak{C}$.
One then has that in the diagram below the square is a pullback square.
\begin{equation}\label{PULLFREEFREE EQ}
\begin{tikzcd}
	\Omega^0_{\mathfrak{C},\mathfrak{D}} \ar{r}{} \ar{d}{} &
	\Omega^0_{\mathfrak{D}} \ar{d}{} \ar{r} &
	\Sigma \wr \Sigma_{\mathfrak{D}} \ar{r} &
	\Sigma \wr \mathcal{V}^{op} \ar{r} &
	\mathcal{V}^{op}
\\
	\Sigma_{\mathfrak{C}} \ar{r}[swap]{} &
	\Sigma_{\mathfrak{D}}
\end{tikzcd}
\end{equation}
Moreover, since $\Sigma_{\mathfrak{C}} \to \Sigma_{\mathfrak{D}}$ is an inclusion of components, it follows that performing a Kan extension to $\Sigma_{\mathfrak{D}}$ and then restricting to $\Sigma_{\mathfrak{C}}$
is equivalent to performing the Kan extension of the top composite.




We next note that there is a natural map 
$\Omega^0_{\mathfrak{C},\mathfrak{D}} \to \Omega^0_{\mathfrak{C}}$
given by $T \mapsto \partial_{\mathfrak{D} \setminus \mathfrak{C}}(T)$, i.e. by collapsing those inner edges not in $\mathfrak{C}$, and we write
$\Omega^{0,\partial}_{\mathfrak{C},\mathfrak{D}}
\subseteq 
\Omega^{0}_{\mathfrak{C},\mathfrak{D}}
$ for the full subcategory mapping to $\Sigma_{\mathfrak{C}}$ under $\partial_{\mathfrak{D} \setminus \mathfrak{C}}$ (more explicitly, elements of $\Omega^{0,\partial}_{\mathfrak{C},\mathfrak{D}}$ are non-stick trees with root and leaves in $\mathfrak{C}$ and inner edges in $\mathfrak{D} \setminus \mathfrak{C}$).
Moreover, noting that for the canonical maps 
$\partial_{\mathfrak{D} \setminus \mathfrak{C}}(T) \to T$
the subtrees $T_v$ for $v \in V(\partial_{\mathfrak{D} \setminus \mathfrak{C}}(T))$
must be in $\Omega^{0,\partial}_{\mathfrak{D}, \mathfrak{C}}$
one obtains a vertex functor 
$V \colon \Omega_{\mathfrak{C},\mathfrak{D}}^{0}
\to \Sigma \wr \Omega_{\mathfrak{C},\mathfrak{D}}^{0,\partial}$.
One then has a diagram
\begin{equation}\label{PULLFREEFREE2 EQ}
\begin{tikzcd}
	\Omega^0_{\mathfrak{C},\mathfrak{D}} \ar{r}{} \ar{d}[swap]{\partial_{\mathfrak{D} \setminus \mathfrak{C}}} &
	\Sigma \wr \Omega^{0,\partial}_{\mathfrak{C},\mathfrak{D}} \ar{d}[swap]{\partial_{\mathfrak{D} \setminus \mathfrak{C}}} \ar{r} &
	\Sigma \wr \Sigma \wr \Sigma_{\mathfrak{D}} \ar{r} &
	\Sigma \wr \Sigma_{\mathfrak{D}} \ar{r} &
	\Sigma \wr \mathcal{V}^{op} \ar{r} &
	\mathcal{V}^{op}
\\
	\Omega_{\mathfrak{C}} \ar{r}[swap]{} \ar{d}[swap]{\mathsf{lr}} &
	\Sigma \wr \Sigma_{\mathfrak{C}}
\\
	\Sigma_{\mathfrak{C}}
\end{tikzcd}
\end{equation}
where the top composite is naturally isomorphic to the top composite in 
\eqref{PULLFREEFREE EQ}, and whose Kan extension can alternatively be computed by first computing the Kan extension
\begin{equation}\label{HATFST EQ}
\begin{tikzcd}
	\Omega_{\mathfrak{C},\mathfrak{D}}^{0,\partial} \ar{r} \ar{d} &
	\Sigma \wr \Sigma_{\mathfrak{D}} \ar{r}{X} & 
	\Sigma \wr \mathcal{V}^{op} \ar{r} & 
	\mathcal{V}^{op}
\\
	\Sigma_{\mathfrak{C}} \ar[dashed]{rrru}[swap]{\widehat{f^{\**}}(X)}
\end{tikzcd}
\end{equation}
and then performing the associated Kan extension along 
$\Omega_{\mathfrak{C}} \to \Sigma_{\mathfrak{C}}$
in \eqref{PULLFREEFREE2 EQ}.
Put together, these observations show the desired claim that
$
f^{\**} \mathbb{F}_{\mathfrak{D}}(X) \simeq
\mathbb{F}_{\mathfrak{C}} \widehat{f^{\**}}(X)
$
as symmetric sequences.

Checking that this isomorphism agrees with the operad structures requires some extra work, which we now briefly sketch.
Firstly, to simplify the discussion concerning Kan extensions,
it is preferable to first prove the analogous result for spans, i.e. that there is 
$\widehat{f^{\**}_s}$ as in the following analogue of 
\eqref{HATFST EQ}.
\begin{equation}\label{HATFSTSP EQ}
\begin{tikzcd}
	\mathsf{WSpan}(\Sigma_{\mathfrak{D}},\mathcal{V}^{op})
	\ar{r}{N_{\mathfrak{D}}} \ar[dashed]{d}[swap]{\widehat{f^{\**}_s}} &
	\mathsf{Alg}_{N_{\mathfrak{D}}}
	\ar{d}{f^{\**}_s}
\\
	\mathsf{WSpan}(\Sigma_{\mathfrak{C}},\mathcal{V}^{op})
	\ar{r}{N_{\mathfrak{C}}} &
	\mathsf{Alg}_{N_{\mathfrak{C}}}
\end{tikzcd}
\end{equation}
Firstly, as a natural adaptation of 
\eqref{HATFST EQ}
one defines $\widehat{f^{\**}_s}$
on the span $\Sigma_{\mathfrak{D}} \rightarrow A \leftarrow \mathcal{V}^{op}$
to be given by 
\[
\begin{tikzcd}
	\Omega_{\mathfrak{C},\mathfrak{D}}^{0,\partial} \wr A
	\ar{r} \ar{d} &
	\Sigma \wr A \ar{r} \ar{d} &
	\Sigma \wr \mathcal{V}^{op} \ar{r} &
	\mathcal{V}^{op}
\\
	\Omega_{\mathfrak{C},\mathfrak{D}}^{0,\partial}
	\ar{r} \ar{d} &
	\Sigma \wr \Sigma_{\mathfrak{D}}
\\
	\Sigma_{\mathfrak{C}}
\end{tikzcd}
\]
where the square defining 
$\Omega^{0,\partial}_{\mathfrak{C},\mathfrak{D}} \wr A$ is a pullback. 

Assuming for the moment that \eqref{HATFSTSP EQ} indeed commutes up to natural isomorphism (i.e. that compatibility with the monad structures as been verified), one then has the string of isomorphisms
\[
f^{\**} \circ \mathbb{F}_{\mathfrak{D}} = 
f^{\**} \circ \mathsf{Lan} \circ N_{\mathfrak{D}} \circ \iota \xleftarrow{\simeq}
\mathsf{Lan} \circ f^{\**}_s \circ N_{\mathfrak{D}} \circ \iota \simeq
\mathsf{Lan} \circ N_{\mathfrak{C}} \circ \widehat{f^{\**}_s} \circ \iota
\xleftarrow{\simeq}
\mathsf{Lan} \circ N_{\mathfrak{C}} \circ \iota \circ \mathsf{Lan}
\circ \widehat{f^{\**}} \circ \iota
=
\mathbb{F}_{\mathfrak{C}} \circ \mathsf{Lan}
\circ \widehat{f^{\**}_s} \circ \iota
=
\mathbb{F}_{\mathfrak{C}}
\circ \widehat{f^{\**}}
\]
where the second step follows since 
$\Sigma_{\mathfrak{C}} \to \Sigma_{\mathfrak{D}}$ is an inclusion of components and the fourth step follows
from Lemmas \ref{FINWRPRODLIM LEM} and \ref{LANPULLCOMA LEM}.

We now tackle the remaining claim of showing that
\eqref{HATFSTSP EQ} commutes up to isomorphism.

We first need to discuss higher string generalizations of the tree categories 
$\Omega^{0}_{\mathfrak{C},\mathfrak{D}}$
and
$\Omega^{0,\partial}_{\mathfrak{C},\mathfrak{D}}$.

Namely, we define full subcategories 
$\Omega^{n,\partial}_{\mathfrak{C},\mathfrak{D}}
\subseteq
\Omega^{0}_{\mathfrak{C},\mathfrak{D}}
\subseteq
\Omega^{n}_{\mathfrak{D}}
$
where the objects of 
$\Omega^{0}_{\mathfrak{C},\mathfrak{D}}$
are strings $T_0 \to T_1 \to \cdots \to T_n$
such that $T_k \in \Omega_{\mathfrak{C}}$ for $k<n$
and the objects of 
$\Omega^{n,\partial}_{\mathfrak{C},\mathfrak{D}}$
satisfy the further restriction that
$T_{k-1}=\partial_{\mathfrak{D} \setminus \mathfrak{C}}(T_k)$.


Note that the vertex and simplicial operators then restrict to maps
\[
	\Omega_{\mathfrak{C},\mathfrak{D}}^{n, \partial} 
	\xrightarrow{V}
	\Sigma \wr \Omega_{\mathfrak{C},\mathfrak{D}}^{n-1, \partial} 
\quad
	n \geq 1
\qquad \qquad
	\Omega_{\mathfrak{C},\mathfrak{D}}^{n, \partial} 
	\xrightarrow{d_i}
	\Omega_{\mathfrak{C},\mathfrak{D}}^{n-1, \partial}
\quad
	n-2 \geq i \geq 0
\qquad \qquad
	\Omega_{\mathfrak{C},\mathfrak{D}}^{n, \partial} 
	\xrightarrow{d_{n-1},\simeq}
	\Omega_{\mathfrak{C},\mathfrak{D}}^{n-1}
\]
where we note that the last map is an isomorphism. 
Indeed, the $\Omega^0_{\mathfrak{C},\mathfrak{D}}$ 
on the top left corner of \eqref{PULLFREEFREE2 EQ} is best viewed as
being $\Omega^{1,\partial}_{\mathfrak{C},\mathfrak{D}}$, so that the leftmost horizontal map is the regular $V$ and the vertical maps are $d_1,d_0$.

The required compatibility with the operadic structure now follows by noting that the composite
$N_{\mathfrak{C}} N_{\mathfrak{C}} \widehat{f^{\**}_s}
\simeq
N_{\mathfrak{C}} f^{\**}_s N_{\mathfrak{D}}
\to
f^{\**}_s N_{\mathfrak{D}}
$
is encoded by the diagram
\[
\begin{tikzcd}[column sep =8pt]
	\Omega^{2,\partial}_{\mathfrak{C},\mathfrak{D}} \wr A \ar{d}[swap]{d^1}{\simeq}  \ar{r} &
	\Sigma \wr \Omega^{1,\partial}_{\mathfrak{C},\mathfrak{D}} \wr A \ar{d}[swap]{d^0}{\simeq} \ar{r} &
	\Sigma^{\wr 2} \wr \Omega^{0,\partial}_{\mathfrak{C},\mathfrak{D}} \wr A \ar{r} &
	|[alias=UUU]|
	\Sigma^{\wr 3} \wr A \ar{d}[swap]{\sigma^1} \ar{r} &
	\Sigma^{\wr 3} \wr \mathcal{V}^{op} \ar{d}[swap]{\sigma^1} \ar{r}{\otimes} &
	\Sigma^{\wr 2} \wr \mathcal{V}^{op} \ar{r}{\otimes} &
	|[alias=UUUU]|
	\Sigma \wr \mathcal{V}^{op} \ar{r}{\otimes} \ar[equal]{d} &
	\mathcal{V}^{op} \ar[equal]{d}
\\
	\Omega^1_{\mathfrak{C},\mathfrak{D}} \wr A \ar{r} \ar{d}[swap]{d^0} &
	|[alias=VVV]|
	\Sigma \wr \Omega^0_{\mathfrak{C},\mathfrak{D}} \wr A \ar{rr} & &
	|[alias=U]|
	\Sigma^{\wr 2} \wr A \ar{r} \ar{d}[swap]{\sigma^0} &
	|[alias=VVVV]|
	\Sigma^{\wr 2} \wr \mathcal{V}^{op} \ar{rr}{\otimes} \ar{d}[swap]{\sigma^0} & &
	\Sigma \wr \mathcal{V}^{op} \ar{r}{\otimes} &
	|[alias=UU]|
	\mathcal{V}^{op} \ar[equal]{d}
\\
	|[alias=V]|
	\Omega^0_{\mathfrak{C},\mathfrak{D}} \wr A \ar{rrr} & & &
	\Sigma \wr A \ar{r} &
	|[alias=VV]|
	\Sigma \wr \mathcal{V}^{op} \ar{rrr}{\otimes} & & &
	\mathcal{V}^{op}
\arrow[Leftrightarrow, from=V, to=U,shorten >=0.15cm,shorten <=0.15cm
,swap,"\pi"
]
\arrow[Leftrightarrow, from=VV, to=UU,shorten >=0.15cm,shorten <=0.15cm
]
\arrow[Leftrightarrow, from=VVV, to=UUU,shorten >=0.15cm,shorten <=0.15cm
,swap,"\pi"
]
\arrow[Leftrightarrow, from=VVVV, to=UUUU,shorten >=0.15cm,shorten <=0.15cm
]
\end{tikzcd}
\]
while the composite
$N_{\mathfrak{C}} N_{\mathfrak{C}} \widehat{f^{\**}_s}
\to
N_{\mathfrak{C}} \widehat{f^{\**}_s}
\simeq
f^{\**}_s N_{\mathfrak{D}}
$
is encoded by the diagram
\[
\begin{tikzcd}[column sep =8pt]
	\Omega^{2,\partial}_{\mathfrak{C},\mathfrak{D}} \wr A \ar{r} \ar{d}[swap]{d^0} &
	\Sigma \wr \Omega^{1,\partial}_{\mathfrak{C},\mathfrak{D}} \wr A \ar{r} &
	|[alias=UUU]|
	\Sigma^{\wr 2} \wr \Omega^{0,\partial}_{\mathfrak{C},\mathfrak{D}} \wr A
	\ar{d}[swap]{\sigma^0} \ar{r} &
	\Sigma^{\wr 3} \wr A \ar{d}[swap]{\sigma^0} \ar{r} &
	\Sigma^{\wr 3} \wr \mathcal{V}^{op} \ar{d}[swap]{\sigma^0} \ar{r}{\otimes} &
	\Sigma^{\wr 2} \wr \mathcal{V}^{op} \ar{d}[swap]{\sigma^0} \ar{r}{\otimes} &
	\Sigma \wr \mathcal{V}^{op} \ar{r}{\otimes} & 
	|[alias=UUUU]|
	\mathcal{V}^{op} \ar[equal]{d}
\\
	|[alias=VVV]|
	\Omega^{1,\partial}_{\mathfrak{C},\mathfrak{D}} \wr A \ar{rr} \ar{d}[swap]{d^0}{\simeq} & &
	\Sigma \wr \Omega^{0,\delta}_{\mathfrak{C},\mathfrak{D}} \wr A \ar{r} &
	|[alias=U]|
	\Sigma^{\wr 2} \wr A \ar{r} \ar{d}[swap]{\sigma^0} &
	\Sigma^{\wr 2} \wr \mathcal{V}^{op} \ar{r}{\otimes} \ar{d}[swap]{\sigma^0} &
	|[alias=VVVV]|
	\Sigma \wr \mathcal{V}^{op} \ar{rr}{\otimes} & &
	|[alias=UU]|
	\mathcal{V}^{op} \ar[equal]{d}
\\
	|[alias=V]|
	\Omega^0_{\mathfrak{C}} \wr A \ar{rrr} & & &
	\Sigma \wr A \ar{r} &
	|[alias=VV]|
	\Sigma \wr \mathcal{V}^{op} \ar{rrr}{\otimes} & & &
	\mathcal{V}^{op}
\arrow[Leftrightarrow, from=V, to=U,shorten >=0.15cm,shorten <=0.15cm
,swap,"\pi"
]
\arrow[Leftrightarrow, from=VV, to=UU,shorten >=0.15cm,shorten <=0.15cm
]
\arrow[Leftrightarrow, from=VVV, to=UUU,shorten >=0.15cm,shorten <=0.15cm
,swap,"\pi"
]
\arrow[Leftrightarrow, from=VVVV, to=UUUU,shorten >=0.15cm,shorten <=0.15cm
]
\end{tikzcd}
\]



\fi%










\bibliography{biblio-new}{}
\bibliographystyle{amsalpha2}



\end{document}


%%% Local Variables:
%%% mode: latex
%%% TeX-master: t
%%% End:
