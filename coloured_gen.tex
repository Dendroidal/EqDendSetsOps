\documentclass[psamsfonts,onesided,10pt
,draft
]{amsart}%{amsart, a4paper, article}


\usepackage[hidelinks]{hyperref}
\hypersetup{
  % colorlinks,
  final,
  pdftitle={Coloured genuine operads,
  pdfauthor={Bonventre, P. and Pereira, L. A.},
  % pdfsubject={Your subject here},
  % pdfkeywords={keyword1, keyword2},
  linktoc=page
}
\usepackage[open=false]{bookmark}

\input{commands.tex}%




% -------- TIKZ ------------------------------------------------------------
\usepackage{tikz}%
\usetikzlibrary[decorations.pathreplacing, arrows, cd, backgrounds, patterns]
\tikzset{%
  treenode/.style = {shape=rectangle, rounded corners,%
    draw, align=center,%
    1    top color=white, bottom color=blue!20},%
  root/.style     = {treenode, font=\Large, bottom color=red!30},%
  env/.style      = {treenode, font=\ttfamily\normalsize},%
  dummy/.style    = {circle,draw,inner sep=0pt,minimum size=2mm},%
  bicolor/.style 2 args={
    dashed,dash pattern=on 6pt off 6pt,->,#1,
    postaction={draw,dashed,dash pattern=on 6pt off 6pt,->,#2,dash phase=6pt}
  }
}%


% --------- SidewaysFigures----------------------------------
\usepackage[figuresright]{rotating}
\newenvironment{amssidewaysfigure}
{\begin{sidewaysfigure}[htpb]
        \vspace*{.5\textwidth}
        \begin{minipage}{\textheight}\centering
            }
            {\end{minipage}\end{sidewaysfigure}
}

% -------- Commands on DRAFT --------------------------

\usepackage{ifdraft}
\ifdraft{
  \color[RGB]{63,63,63}
  % \pagecolor[rgb]{0.5,0.5,0.5}
  \pagecolor[RGB]{220,220,204}
  % \color[rgb]{1,1,1}
}

\usepackage[draft]{showkeys}
\usepackage[obeyDraft]{todonotes}


% ----- What Labels Changed? --------------------------------
\makeatletter

\def\@testdef #1#2#3{%
  \def\reserved@a{#3}\expandafter \ifx \csname #1@#2\endcsname
  \reserved@a  \else
  \typeout{^^Jlabel #2 changed:^^J%
    \meaning\reserved@a^^J%
    \expandafter\meaning\csname #1@#2\endcsname^^J}%
  \@tempswatrue \fi}

\makeatother

% -------Commands--------------------------------------------
\newcommand{\mycircled}[2][none]{%
  \mathbin{
    \tikz[baseline=(a.base)]\node[draw,circle,inner sep=-1.5pt, outer sep=0pt,fill=#1](a){\ensuremath #2\strut};
  }
}
\newcommand{\owr}{\mycircled{\wr}}
\newcommand{\UV}{\underline{\mathcal V}}
\renewcommand{\phi}{\varphi}
\newcommand{\UC}{\underline{\mathfrak C}}



% ------Personal Info------------------------------------
\title{Coloured Genuine Operads}%

\author{Peter Bonventre and Lu\'is Pereria}
\address{}

\date{\today}

\begin{document}

\begin{abstract}
      Things and stuff
\end{abstract}

\maketitle

\tableofcontents

% ----------STUFF-----------------------------------------

\section{Coloured Operads}


\subsection{Non-Equivariant Coloured Operads}


Fix a closed symmetric monoidal category $\V$.

\begin{definition}
      Fix a set $\mathfrak C$ of \textit{colours}.
      A tuple
      $\ksi = (c_1, \ldots, c_n; c_0) \in \mathfrak C^{\times n} \times \mathfrak C$
      is called a \textit{signature} of $\mathfrak C$, and let $|\ksi|$ denote the length $n$
      (so $\ksi \in \mathfrak C^{\times |\ksi| + 1}$).
      
      A \textit{$\mathfrak C$-coloured operad}\footnote{These are also known as \textit{symmetric multicategories}} in $\V$ consists of the following data:
      \begin{enumerate}%\itemsep-4pt
      \item An object $\O(\ksi) \in \V$ for each signature $\ksi$.
      \item For each $c \in \mathfrak C$, a \textit{unit} $1_c \in \O(c;c)$.
      \item For any signature $\ksi \in \mathfrak C^{\times n+1}$ and $\sigma \in \Sigma_n$, a map $\O(\ksi) \to \O(\sigma \cdot \ksi)$,
            where $\Sigma_n$ acts on the left of $\mathfrak C^{\times n+1}$ by acting on the first $n$ coordinates.
            Explicitly, this is a map
            \begin{equation}
                  \O(c_1, \ldots, c_n; c_0) \xrightarrow{\sigma} \O(c_{\sigma^{-1}1}, \ldots, c_{\sigma^{-1}n}; c_0).
            \end{equation}
      \item For any compatible signatures $\ksi = (c_1, \ldots, c_n; c_0)$, $\ksi_i = (c_{i,1}, \ldots, c_{i,m_i}; c_i)$, a \textit{composition} map
            \begin{equation}
                  \O(\ksi) \times \O(\ksi_1) \times \ldots \times \O(\ksi_n) \to \O(c_{1,1}, \ldots, c_{n,m_n}; c_0)
            \end{equation}
      \end{enumerate}
      subject to all the compatibilities you'd expect.

      A map of $\mathfrak C$-coloured operads is a compatible collection of maps
      $\set{\O(\ksi) \to \O'(\ksi)}_{\ksi}$.
      
      Let $\Op^{\mathfrak C}(\V)$ denote the category of $\mathfrak C$-coloured operads in $\V$.
\end{definition}

\begin{definition}
      Given a map $f: \mathfrak C' \to \mathfrak C$ and a $\mathfrak C$-coloured operad $\O$,
      there is a natural $\mathfrak C'$-coloured operad $f^*(\O)$, where
      \begin{equation}
            f^{\**}(\O)(c'_1, \ldots, c'_n; c'_0) = \O(f(c'_1), \ldots, f(c'_n); f(c'_0)).
      \end{equation}

      A \textit{map of coloured operads} $\O' \to \O$ is given by the data of a map of colours $f: \mathfrak C' \to \mathfrak C$,
      and a map of $\mathfrak C'$-coloured operads $\O' \to f^*(\O)$.
      
      Let $\Op(\V)$ denote the category of coloured operads in $\V$.
\end{definition}

\begin{remark}
      The category $\Op(\V)$ is isomorphic to the Grothendieck construction on the functor
      \begin{equation}
            \begin{tikzcd}[row sep = tiny]
                  \mathsf F \arrow[r] & \mathsf{Cat}
                  \\
                  \mathfrak C \arrow[r, mapsto] & \Op^{\mathfrak C}(\V).
            \end{tikzcd}
      \end{equation}
\end{remark}

\begin{notation}
      In previous work \todo{cite}, $\Op(\V)$ has been used to denote \textit{single-coloured} operads specifically; that is, $\set{\**}$-coloured operads.
      For this article, we will write these as $\Op^{\set{\**}}(\V)$. 
\end{notation}


\subsection{Equivariant Coloured Operads}

\begin{definition}
      The category $\Op^G(\V)$ of  \textit{$G$-coloured operads} in $\V$ is the category of
      $G$-objects in $\Op(\V)$.
\end{definition}


\begin{remark}
      Unpacking this definition, we see $\O \in \Op^G(\V)$ consists of the following data:
      \begin{enumerate}
      \item A $G$-set $\mathfrak C$ of colours.
      \item For each signature $\ksi$ of $\mathfrak C$, an object $\O(\ksi) \in \V$.
      \item For each signature $\ksi \in \mathfrak C^{\times n+1}$ and $(g,\sigma) \in G\times \Sigma_n$, a map
            $\O(\ksi) \to \O((g,\sigma)\cdot \ksi)$,
            where $G$ acts on $\mathfrak C^{\times n+1}$ diagonally (across all $n+1$ coordinates), and $\Sigma_n$ acts on the first $n$.
      \item For each $c \in \mathfrak C$, a \textit{unit} $1_c \in \O(c;c)^{G_c}$, where $G_c$ is the stabilizer of $c$.
      \item For compatible signatures $\ksi$, $\ksi_1$, $\ldots$, $\ksi_n$, \textit{composition maps}
            \begin{equation}
                  \O(\ksi) \otimes \O(\ksi_1) \otimes \ldots \otimes \O(\ksi_n) \to \O(\ksi \circ (\ksi_1, \ldots, \ksi_n)),
            \end{equation}
      \end{enumerate}
      such that composition is
      compatible with the $G$-action on each component as well as the appropriate actions of $\Sigma$,
      and is unital and associative. 
\end{remark}

\begin{remark}
      Unlike in the single-coloured case, this is \textit{not} the same as coloured operads in $\V^G$.
      Indeed, objects in $\Op(\V^G)$ have a $G$-fixed set of colours, and each level $\O(\ksi)$ is a full $G$-set
      (though only a partial $\Sigma_{|\ksi|}$-set).
\end{remark}

\begin{definition}
      Given a $G$-set $\mathfrak C$, let $\Op^{G,\mathfrak C}(\V)$ denote the category of \textit{$\mathfrak C$-coloured operads} and maps which are the identity on colours.

      Parallel to the non-equivariant case, $\Op^G(\V)$ is isomorphic to the Grothendieck construction on the functor
            \begin{equation}
            \begin{tikzcd}[row sep = tiny]
                  \mathsf F^G \arrow[r] & \mathsf{Cat}
                  \\
                  \mathfrak C \arrow[r, mapsto] & \Op^{G,\mathfrak C}(\V).
            \end{tikzcd}
      \end{equation}
      
\end{definition}

\subsubsection{Categorical Description}

\begin{definition}
      Given a $G$-set $X$, let $B_XG$ denote the \textit{translation category} of $X$,
      with object set $X$ and morphisms $g: x \to g\cdot x$ for all pairs $(g,x) \in G \times X$.

      We will denote $B_{\set{\**}}G$ by $\mathsf G$.
\end{definition}

\begin{remark}
      We observe that we have a natural diagonal map
      \begin{equation}
            F \times \mathsf G \into \mathsf F \wr \mathsf G,
      \end{equation}
      and so for any functor $F: \mathcal C \to \mathsf F$, we have an induced functor
      $F: \mathcal C \times \mathsf G \to \mathsf F \wr \mathsf G$. 
\end{remark}

\begin{definition}
      Let $\mathfrak C \Sigma$ be the category
      \begin{equation}
            \mathfrak C \Sigma = \coprod\limits_{n\geq 0} B_{\mathfrak C^{\times n} \times \mathfrak C}(G \times \Sigma_n).
      \end{equation}
      We note that $\mathrm{Ob}(\mathfrak C \Sigma)$ is precisely the set of \textit{signatures} in $\mathfrak C$.
      Further, we observe that this is equivalent to the pullback
      \begin{equation}
            \label{CSIGMA_EQ}
            \begin{tikzcd}
                  \mathfrak C \Sigma \arrow[d] \arrow[r, "E"]
                  &
                  \mathsf F \wr B_{\mathfrak C}G \arrow[d]
                  \\
                  \Sigma \times \mathsf G \arrow[r, "E"]
                  &
                  \mathsf F \wr \mathsf G
            \end{tikzcd}
      \end{equation}
      where $E: \Omega \to \mathsf F$ sends a tree to its set of edges.
      \todo[inline]{$B_{\mathfrak C}G = G \ltimes \mathfrak C$}
      
      More generally, let $\mathfrak C \Omega$ be the pullback
      \begin{equation}
            \label{COMEGA_EQ}
            \begin{tikzcd}
                  \mathfrak C \Omega \arrow[d] \arrow[r, "E"]
                  &
                  \mathsf F \wr B_{\mathfrak C}G \arrow[d]
                  \\
                  \Omega \times \mathsf G \arrow[r, "E"]
                  &
                  \mathsf F \wr \mathsf G
            \end{tikzcd}
      \end{equation}

      We have a natural inclusion of categories $\mathfrak C \Sigma \into \mathfrak C \Omega$.
      Moreover, we will called elements of these categories
      \textit{coloured trees} (or \textit{coloured corollas}),
      and denote them by $(T,\mathfrak c)$, where $\mathfrak c: E(T) \to \mathfrak C$ is a map of sets.
\end{definition}

\begin{remark}
      Unpacking definitions, we see that a map $(T, \mathfrak c) \to (S, \mathfrak d)$ is given by
      a map $f: T \to S$ in $\Omega$ and an element $g\in G$,
      such that $g.\mathfrak c(e) = \mathfrak d(f(e))$ for all $e \in E(T)$.
      \begin{equation}
            \begin{tikzcd}
                  E(T) \arrow[r, "f"] \arrow[d, "\mathfrak c"']
                  &
                  E(S) \arrow[d, "\mathfrak d"]
                  \\
                  \mathfrak C \arrow[r, "g"]
                  &
                  \mathfrak C
            \end{tikzcd}
      \end{equation}
      
      In particular, we have maps of the form
      \begin{equation}
            g = (id, g): (T, E(T) \to \mathfrak C) \to (T, E(T) \to \mathfrak C \xrightarrow{g \cdot} \mathfrak C). 
      \end{equation}
\end{remark}

\begin{remark}
      $\mathfrak C \Omega$ is equivalent to the
      Grothendieck construction on the functor
      \begin{equation}
            \begin{tikzcd}[row sep = tiny]
                  \Omega^{op} \times G \arrow[r]
                  &
                  \mathsf{Cat}
                  \\
                  T \arrow[r, mapsto]
                  &
                  \Fun(E(T), \mathfrak C).
            \end{tikzcd}
      \end{equation}
      \todo[inline]{compare with genuine case: RHS equals $\Fun(\Phi(E(T)), \mathfrak C) = \Fun(\Phi(E(G \cdot T)), \mathfrak C))$}
      and a similar result holds for $\mathfrak C \Sigma$. 
\end{remark}

\begin{remark}
      Note that we can replace the $G$-set $\mathfrak C$ with a \textit{coefficient system} $\underline{\mathfrak C}$,
      substituting the rectangle of pullbacks below for \eqref{COMEGA_EQ}
      \begin{equation}
            \begin{tikzcd}
                  \underline{\mathfrak C}\Omega \arrow[d] \arrow[r, "E"]
                  &
                  \mathsf F \wr B_{\mathfrak C(G/e)}G \arrow[r] \arrow[d]
                  &
                  \mathsf F \wr \underline{\mathfrak C} \arrow[d]
                  \\
                  \Omega \times G \arrow[r, "E"]
                  &
                  \mathsf F \wr G \arrow[r]
                  &
                  \mathsf F \wr O_G
            \end{tikzcd}
      \end{equation}
      with $\mathfrak C(G/e) \into \underline{\mathfrak C}$ and $G \into O_G$ the natural inclusions.
      \todo[inline]{compare $B_{\mathfrak C(G/e)}G = G \ltimes \mathfrak C(G/e)$ and $\underline{\mathfrak C} = O_G \ltimes \underline{\mathfrak C}$.}
      In this case, $\underline{\mathfrak C}\Omega = \mathfrak C(G/e)\Omega$.
\end{remark}

Many of the natural functors around $\Omega$ and $\Sigma$ have generalizations to the coloured setting,
which can be built through a straightforward use of the universal property of pullbacks.

\begin{definition}
      We have a natural \textit{vertex} functor
      $V: \mathfrak C \Omega \to \Sigma \wr \mathfrak C \Sigma$,
      as colourings of a tree restrict to colourings of each vertex corolla.

      Similarly, there is a \textit{leaf-root} funct or
      $\mathsf{lr}: \mathfrak C \Omega \to \mathfrak C \Sigma$,
      where the colouring of $\mathsf{lr}(T)$ is a restrict of the colouring of $T$.
\end{definition}

\begin{definition}
      The category $\Sym^{G,\mathfrak C}$ of \textit{symmetric $(G,\mathfrak C)$-sequences} is
      the category of functors $X: \mathfrak C \Sigma^{op} \to \V$.
\end{definition}

\begin{definition}
      Given $X \in \Sym^{G, \mathfrak C}$, let $\mathbb F^{\mathfrak C} X$ denote the left Kan extension below.
      \begin{equation} 
           \begin{tikzcd}
                  \mathfrak C \Omega^{op}
                  \arrow[d, "\mathsf{lr}"']
                  \arrow[r, "V"]
                  &
                  (\Sigma \wr \mathfrak C \Sigma)^{op} \arrow[r, "X"]
                  \arrow[dl, Rightarrow]
                  &
                  (\Sigma \wr \V^{op})^{op} \arrow[r, "\otimes"]
                  &
                  \V
                  \\
                  \mathfrak C \Sigma^{op} \arrow[urrr, "\Lan = \mathbb F^{\mathfrak C} X"']
            \end{tikzcd}
      \end{equation}
\end{definition}


\subsection{Single-Coloured Operads}
We first show that this generalizes the free single-coloured operad monad.
When $\mathfrak C = \set{\**}$, we have
$\mathfrak C \Omega = \Omega \times G$, and similarly
$\mathfrak C \Sigma = \Sigma \times G$.

\begin{notation}
      Given $X \in \mathsf{Cat}(\C, \mathsf{Fun}(\mathcal D, \V))$,
      let $\tilde X$ denote the adjoint functor in the isomorphic category $\mathsf{Cat}(\C \times \mathcal D, \V)$.
\end{notation}

\begin{lemma}
      \label{SPAN_LAN_LEM}
      Conisder the two spans below.
      \begin{equation}
            \begin{tikzcd}
                  \C \arrow[d, "p"] \arrow[r, "X"]
                  &
                  \mathsf{Fun}(\mathcal D, \V)
                  &&
                  \C \times \mathcal D \arrow[d, "p \times \mathsf{id}"] \arrow[r, "\tilde X"]
                  &
                  \V
                  \\
                  \mathcal E
                  &
                  &&
                  \mathcal E \times \mathcal D
            \end{tikzcd}
      \end{equation}
      
      Then $\Lan_p X$ is adjoint to $\Lan_{p \times \mathsf{id}} \tilde X$. 
\end{lemma}
\begin{proof}
      We have
      \begin{align}
        \widetilde{\Lan_p X}(e,d)
        &= (\Lan_p X(e))(d)
          = \left(
          \colim\limits_{\substack{ \C \downarrow e \\ p(c) \to e}} X(c)
        \right)(d)
        = \colim\limits_{\substack{ \C \downarrow e \\ p(c) \to e}}(X(c)(d))
        = \colim\limits_{\substack{ \C \downarrow e \\ p(c) \to e}}(\tilde X(c,d))\\
        &= \colim\limits_{\substack{ \C \times \set{d} \downarrow (e,d) \\ p(c) \to e}}(\tilde X(c,d))
        \cong \colim\limits_{\substack{ \C \times \mathcal D \downarrow (e,d) \\ (p(c),d') \to (e,d)}}(\tilde X(c,d'))
        = \Lan_{p \times \mathsf{id}}\tilde X(c,d),
      \end{align}
      where the isomorphism holds by a straightforward finality argument.
      On maps, a similar argument holds.
\end{proof}

\begin{notation}[\cite{BP17}]
      Let $\mathbb F'$ denote the \textit{free single-coloured operad monad} on $\V$, given by the left Kan extension of the following diagram.
      \begin{equation}
            \begin{tikzcd}
                  \Omega^{op}
                  \arrow[d, "\mathsf{lr}"']
                  \arrow[r, "V"]
                  &
                  (\Sigma \wr \Sigma)^{op} \arrow[r, "X"]
                  \arrow[dl, Rightarrow]
                  &
                  (\Sigma \wr \V^{op})^{op} \arrow[r, "\otimes"]
                  &
                  \V
                  \\
                  \Sigma^{op} \arrow[urrr, "\Lan = \mathbb F' X"']
            \end{tikzcd}
      \end{equation}
\end{notation}

\begin{proposition}
      $\mathbb F^{\set{\**}}$ is a monad, and moreover
      the category of $\mathbb F^{\set{\**}}$-algebras in $\Fun(\Sigma \times G, \V)$ is equivalent to
      the category of $\mathbb F'$-algebras in $\Fun(\Sigma, \V^G)$.
\end{proposition}
\begin{proof}
      Let $\tau: \tilde X \mapsto X$ denote the isomorphism of categories
      $\Fun(\Sigma \times G, \V) \xrightarrow{\tau} \Fun(\Sigma, \V^G)$.
      Then $\mathbb F^{\set{\**}} = \tau^{-1} \mathbb F' \tau$ by \ref{SPAN_LAN_LEM}, and so
      $\mathbb F^{\set{\**}}$ is in fact a monad, and the
      the isomorphism lifts to an isomorphism on the category of algebras.
\end{proof}


\subsection{General Case}

\begin{theorem}
      For every $G$-set $\mathfrak C$, $\mathbb F^{\mathfrak C}$ is a monad, with category of algebras given by $\Op^{G,\mathfrak C}(\V)$. 
\end{theorem}
\begin{proof}
      \todo[inline]{This will be a corollary of Genuine Coloured stuff}
\end{proof}






\newpage

\section{Coloured Genuine Equivariant Operads}

Throughout this section, we will abuse notation, and refer to
a coefficient system and its associated (Grothendieck) category over $O_G$ by the same name.

\subsection{Coloured $G$-Trees}

\newcommand{\CS}{\underline{\mathfrak C} \Sigma_G}
\newcommand{\CO}{\underline{\mathfrak C} \Omega_G}


\begin{definition}
      Let $\underline{\mathfrak C}$ be a $G$-coefficient system of sets.
      Then the category $\underline{\mathfrak C}\Omega_G$ of \textit{$\underline{\mathfrak C}$-coloured $G$-trees}
      is defined to be the pullback below.
      \begin{equation}
            \label{COMEGA_G_EQ}
            \begin{tikzcd}
                  \CO \arrow[d] \arrow[r]
                  &
                  \mathsf F \wr \underline{\mathfrak C} \arrow[d]
                  \\
                  \Omega_G \arrow[r, "E_G"]
                  &
                  \mathsf F \wr O_G
            \end{tikzcd}
      \end{equation}
      The category $\CS$ of $\underline{\mathfrak C}$-coloured corollas is the subcategory defined similarly,
      with $\Omega_G$ replaced with $\Sigma_G$.
\end{definition}


Explicitly, objects of $\CO$ are pairs $(T, \mathfrak c)$ of
a $G$-tree $T$ and
a map $\mathfrak c: E_G(T) \to \underline{\mathfrak C}$ over $O_G$.
That is, each orbit of edges $[e]$ is assigned a ``colour'' $\mathfrak c([e]) \in \underline{\mathfrak C}(G/G_{[e]})$,
where $G_{[e]}$ is the stabilizer in $G$ of $e$.
Morphisms $(T, \mathfrak c) \to (S, \mathfrak d)$
are given by maps of trees $\phi: S \to T$ such that, for every edge orbit $[e]$ of $S$, we have
\begin{equation}
      \mathfrak c([e]) = \phi_{[e]}^{\**}\mathfrak d([\phi(e)]),
\end{equation}
where $\phi_{[e]}: G / G_{[e]} \to G / G_{[\phi(e)]}$ is the map in $O_G$ induced by $\phi$.


\begin{remark}
      Consider the Grothendieck construction on the functor
      \begin{equation}
            \begin{tikzcd}[row sep = tiny]
                  \mathsf F^{G,op} \arrow[r]
                  &
                  \mathsf{Set}
                  \\
                  A \arrow[r, mapsto]
                  &
                  \Set^{O_G^{op}}(\Phi(A), \underline{\mathfrak C}),
            \end{tikzcd}
      \end{equation}
      where $\Phi: \Set^G \to \Set^{O_G^{op}}$ sends a $G$-set $X$ to its fixed-point system $G/H \mapsto X^H$.
      We will denote this by $\mathsf F^G \wr \underline{\mathfrak C}$.
      Then $\CO$ is also isomorphic to the pullback
      \begin{equation}
            \begin{tikzcd}
                  \CO \arrow[d] \arrow[r]
                  &
                  \mathsf F^G \wr \underline{\mathfrak C} \arrow[d]
                  \\
                  \Omega_G \arrow[r, "E"]
                  &
                  \mathsf F^G.
            \end{tikzcd}
      \end{equation}
      \todo[inline]{We note that the class of morphisms in $\mathsf F^G$ in the image of $E$ (restricted to $\Omega_G^0$)
        are those isomorphic to an adjunction counit $G \cdot_H A|_H \to A$.}
      In this case, a colouring is a map $\mathfrak c: \Phi E(T) \to \mathfrak C$ of coefficient systems,
      and morphisms are maps $\phi: T \to S$ such that
      $\mathfrak c(G/H,e) = \mathfrak d(G/H,e)$ for all $e \in E(T)^H$.
      \begin{equation}
            \begin{tikzcd}
                  \Phi A \arrow[rr, "f"] \arrow[dr, "\mathfrak c"']
                  &&
                  \Phi B \arrow[dl, "\mathfrak d"]
                  \\
                  &
                  \underline{\mathfrak C}
            \end{tikzcd}
      \end{equation}
      It is easy to show this is equivalent to requiring that $\mathfrak c(G/G_e,e) = \phi_e^{\**} \mathfrak d(G/G_{\phi(e)}, \phi(e))$).
      
      Similarly, $\underline{\mathfrak C}\Omega_G$ is isomorphic to the Grothendieck construction on the functor
      \begin{equation}
            \begin{tikzcd}[row sep = tiny]
                  \Omega_G^{op} \arrow[r]
                  &
                  \mathsf{Cat}
                  \\
                  T \arrow[r, mapsto]
                  &
                  \Set^{O_G^{op}}(\Phi(E(T)), \underline{\mathfrak C}),
              \end{tikzcd}
        \end{equation}

        $\CS$ can be defined similarly, with the relevant sources restricted to
        $\Sigma_G \subseteq \Omega_G$. 
\end{remark}


\begin{remark}
      $\underline{\mathfrak C}\Omega_G$ is also a root fibration ---
      that is, a split Grothendieck fibration over the orbit category.
      \todo[inline]{cite reading material}
      Formally, as $\mathsf F \wr (-)$ and pullbacks preserve such fibrations, and these are compatible under composition,
      this follows from the natural maps $\underline{\mathfrak C}\Omega_G \to \Omega_G \to O_G$.
      Explicitly, $\underline{\mathfrak C}\Omega_G(G/H)$ has as objects those pairs $(T,\mathfrak c)$ such that
      $T \simeq G \cdot_H T_*$ for $T_{\**} \in \Omega^H$.
      In this case, $E_G(T) \simeq E_H(T_{\**})$, and we have a natural factorization
      \begin{equation}
            \begin{tikzcd}
                  E_G(T) \simeq E_H(T_{\**}) \arrow[r, "\mathfrak c"] \arrow[dr]
                  &
                  \mathfrak C|_{H} \arrow[r, hookrightarrow] \arrow[d]
                  &
                  \mathfrak C \arrow[d]
                  \\
                  &
                  O_H \arrow[r, hookrightarrow]
                  &
                  O_G.
            \end{tikzcd}
      \end{equation}
      Maps $\phi:(T,\mathfrak c) \to (S, \mathfrak d)$ in each fiber are called \textit{colour-fixed},
      as $\mathfrak c([e]) = \mathfrak d([\phi (e)])$ for all $[e] \in E_G(T)$.
      
      Given $q: G/H \to G/K$ in the orbit category,
      the chosen Cartesian maps are the induced root pullback maps $q: q^{\**}T \to T$ on $G$-trees,
      with the colouring of $q^{\**}T$ defined by $(q^{\**}\mathfrak c)([e]) = q^{\**}(\mathfrak c([e]))$.
\end{remark}

\begin{remark}
      We note that any \textit{planar} map of coloured $G$-trees is always colour-fixed.
\end{remark}

\begin{remark}
      A \textit{quotient} map in $\UC \Omega_G$ is any morphism such that the underlying map in $\Omega_G$ is a quotient.
\end{remark}

% \begin{remark}
%       Replacing the bottom-left corner in \eqref{COMEGA_G_EQ} with $\Omega^G$ changes the pullback to the category
%       $\underline{\mathfrak C}\Omega^G$ of ``$\underline{\mathfrak C}$-coloured trees with $G$-action''.
%       \todo[inline]{leaf-root for later?}
%       Any coloured $G$-tree $(T, \mathfrak c)$ with $T = (T_a)_{a \in A}$ has that each $T_a$ is a
%       ``$\underline{\mathfrak C|_{G_a}}$-coloured tree with $G_a$-action.''
% \end{remark}

We have natural inclusions on the left
\begin{equation}
      \begin{tikzcd}
            \underline{\mathfrak C}\Sigma \arrow[d, "\iota"] \arrow[r]
            &
            \underline{\mathfrak C}\Omega \arrow[d]
            &&
            \Sigma \times G \arrow[d] \arrow[r]
            &
            \Omega \times G \arrow[d]
            \\
            \underline{\mathfrak C}\Sigma_G \arrow[r]
            &
            \underline{\mathfrak C}\Omega_G
            &&
            \Sigma_G \arrow[r]
            &
            \Omega_G
      \end{tikzcd}
\end{equation}
which forget to the uncoloured inclusions on the right.
Specifically, $U \mapsto G \cdot U$ and, as $E_G(G \cdot U) = E(U)$, the associated colouring map is simply $\mathfrak c$ again.
On morphisms, $(\phi,g)$ maps to $(\phi)_{G} \circ g$.

\subsection{Planar Strings and Stuff}

Generalizing \cite[Remark 3.78]{BP17}
\todo[inline]{otherwise, have to force on the non-equivariant trees the correct isotropy of their colours. If not, we just see $\Phi\mathfrak C(G/e)$, and not the whole coefficient system.}
\begin{definition}
      Given $(T,\mathfrak c) \in \underline{\mathfrak C}\Omega_G$, a
      \textit{planar (resp. rooted) $T$-substitution datum} is a tuple
      $((U_{v_{Ge}}, \mathfrak c_{v_{Ge}}))_{v_{Ge} \in V_G(T)}$ of $\underline{\mathfrak C}$-coloured $G$-trees along with
      planar (resp. rooted colour-fixed) tall maps
      $T_{v_{Ge}} \to U_{v_{Ge}}$.

      A map of planar (resp. rooted) $T$-substitution data $(U_{v_{Ge}}) \to (V_{v_{Ge}})$ is a compatible tuple of planar (resp. rooted colour-fixed) tall maps $(U_{v_{Ge}} \to V_{v_{Ge}})$. 
\end{definition}

\begin{lemma}[{cf. \cite[Prop. 3.41]{BP17}}]
      Let $(T,\mathfrak c) \in \underline{\mathfrak C}\Omega_G$ be a $\underline{\mathfrak C}$-coloured $G$-tree.
      There are isomorphisms of categories
      \begin{equation}
            \begin{tikzcd}[row sep = 4pt]
                  \mathsf{Sub}_p(T) \arrow[rr, shift left]
                  &&
                  (T, \mathfrak c) \downarrow \underline{\mathfrak C}\Omega_G^{pt,cf}
                  \arrow[ll, shift left]
                  \\
                  (U_{v_{Ge}}) \arrow[rr, mapsto]
                  &&
                  ((T, \mathfrak c) \to \colim_{Sc_G(T)}U_{(-)}).
            \end{tikzcd}
      \end{equation}
      where $\underline{\mathfrak C}\Omega_G^{pt,cf}$ is the category of planar tall maps under $(T, \mathfrak c)$. 
\end{lemma}
\begin{proof}
      This follows as in \cite[Prop. 3.41]{BP17}, going by induction on $n=|V_G(T)|$.
      Let $U_T$ denote the colimit, if it exists.
      If $n$ is 0 or 1, $T$ is terminal in $Sc_G(T)$, and the colouring on $U_T$ is just $\mathfrak c$.
      Otherwise, we have a decomposition $T = R \amalg_{Ge} S$ with $R,S \in \underline{\mathfrak C}\Omega_G$, such that
      the existance of $U_T$ follows from the existance of the pushout below in $\underline{\mathfrak C}\Omega_G^{pt,cf}$.
      \begin{equation}
            \begin{tikzcd}
                  (\eta_{Ge}, \mathfrak c) \arrow[d] \arrow[r]
                  &
                  U_S \arrow[d, dashed]
                  \\
                  U_R \arrow[r, dashed]
                  &
                  U_T
            \end{tikzcd}
      \end{equation}
      Forgetting colours, this is an equivariant grafting diagram, and hence the $G$-tree $U_T$ exists.
      Moreover, we have $E(U_T) = E(U_S) \amalg_{Ge} E(U_R)$, and so we have a well-defined colouring
      \begin{equation}
            \mathfrak c_{U_T}(Gf) =
            \begin{cases}
                  \mathfrak c_{U_R}(Gf) \qquad \qquad & Gf \in E_G(R) \\
                  \mathfrak c_{U_S}(Gf) & Gf \in E_G(S)
            \end{cases}
      \end{equation}
      since the overlap $Ge$ is in $T$, and hence it is dictated that $\mathfrak c_{U_T}(Ge) = \mathfrak c (Ge)$.
\end{proof}

\begin{lemma}[{cf. \cite[Lemma 3.63]{BP17}}]
      $\underline{\mathfrak C}\Omega_G^0 \to \mathsf F_s \wr \underline{\mathfrak C}\Sigma_G$
      sends root pullbacks to pullbacks over $\mathsf F_s \wr O_G$.
\end{lemma}
\begin{proof}
      Exactly as in \textit{loc cite}, with the additional note that
      the colouring of $\psi^{\**}T$ is precisely such that each $(\psi^{\**}T)_{v_{Ge}} \to T_{v_{G\phi(e)}}$
      is a pullback in $\UC \Sigma_G$.  COME BACK
\end{proof}

\begin{definition}
      The category $\UC\Omega_G^n$ of \textit{coloured planar $n$-strings} is the category
      whoses objects are strings
      \begin{equation}
            (T_0,\mathfrak c_0)
            \xrightarrow{\phi_1} (T_1, \mathfrak c_1)
            \xrightarrow{\phi_2} \ldots
            \xrightarrow{\phi_n} (T_n, \mathfrak c_n)
      \end{equation}
      where $(T_i, \mathfrak c_i) \in \UC\Omega_G$ and the $\phi_i$ are all coloured planar tall maps,
      while arrows are commutative diagrams of quotient maps.
\end{definition}
\newpage

\section{In $\mathsf{dSet_G}$}

\begin{definition}
      Define the \textit{genuine operadic nerve} $N: \Op_G \to \dSet_G$ by
      \begin{equation}
            N\P(T) = \Hom_{\Op_G}(T, \P)
      \end{equation}
      where we think of $T$ as the operad $T \in \Op^G \into \Op_G$. 
\end{definition}

\begin{remark}
      We note that $N\P \in (SCI)^{\boxslash !}$,
      as $T \in \Op_G$ is a free $\mathbb F_G$-algebra on its vertices.
\end{remark}

\begin{remark}
      We can rephrase the definition of being an $\mathbb F_G$-algebra in terms of $N\P$.
      For $\P \in \Sym_G$ a $G$-symmetric sequence,
      a genuine $G$-operad structure on $\P$ is given by:
      \begin{itemize}
      \item Composition Maps: $ $\\
            maps 
            $N\P(T) \to \P(\mathsf{lr}(T))$
            for all $T \in \Omega_G$.
      \item Naturality under restriction and conjugation: $ $\\
            maps $N\P(T_1) \to N\P(T_0)$
            for all quotient maps $T_0 \to T_1$ in $\Omega_{G,0}$,
            such that the following commutes:
            \begin{equation}
                  \begin{tikzcd}
                        N\P(T_1) \arrow[r] \arrow[d]
                        &
                        \P(\mathsf{lr}(T_1)) \arrow[d]
                        \\
                        N\P(T_0) \arrow[r]
                        &
                        \P(\mathsf{lr}(T_0)).
                  \end{tikzcd}
            \end{equation}
      \item Associativity under $\mathbb F_G$: $ $\\
            maps $N\P(T_1) \to N\P(T_0)$
            for all planar tall maps $T_0 \to T_1$ in $\Omega_G^t$,
            such that the analogus diagram (with the right vertical map the identity) commutes.\footnote{
              As in \cite{BP17}, we note that ``associativity'' under $\mathbb F_G$ includes both
              the usual notion of associativity of our composition maps,
              but also unitality;
              this is recorded here by the fact that degeneracies are always planar tall.}
      \end{itemize}
\end{remark}

The above reflects the following result.

\begin{proposition}
      $\Op_G$ is equivalent to the subcategory of $\mathsf{dSet_G}$ spanned by those $X$ such that
      \begin{enumerate}
      \item $X(H/H) = \set{\**}$ for all $H \leq G$.
      \item $X(T) \cong \otimes_{T_v \in V(T)}X(T_v)$. 
      \end{enumerate}
\end{proposition}
\begin{proof}
      The fact that $N\P \in (SCI_G)^{\boxslash !}$ is immediate, as remarked above.

      For the reverse direction, we will follow the construction of the homotopy operad as in \cite[\S 6]{MW08},
      replacing their use of inner horn inclusions with \textit{orbital} inner $G$-horn inclusions,
      to show that any $X \in (OHI)^{\boxslash !}$ is in the image of $N$; 
      the result will then follow from \cite[HYPER PROP]{BP18}.

      In fact, interpreting all of their pictures are as \textit{orbital} representations of $G$-trees yields that,
      for all $C \in \Sigma_G$
      \begin{itemize}
      \item $\sim_{G e}$ is an equivalence relation on $X(C)$ for all $Ge \in E_G(C)$.
      \item The relations $\sim_{G e}$ and $\sim_{G e'}$ are equal for all $e,e'\in E(C)$.
      \item $[h] \circ [f] = [h \circ f]$ yields a well-defined composition map. \todo[inline]{come back}
      \end{itemize}
      \todo[inline]{need to show naturality, check associativity of composition}
\end{proof}



\section{Algebras and Examples}

\subsection{Single Coloured}


\bibliography{biblio}{}



\bibliographystyle{abbrv}



\end{document}