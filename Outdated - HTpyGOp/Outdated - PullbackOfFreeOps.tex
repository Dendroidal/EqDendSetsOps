\documentclass[a4paper,10pt
%,draft
%, final
]{article}%

\pdfcompresslevel=0
\pdfobjcompresslevel=0

\usepackage[hidelinks]{hyperref}
\hypersetup{
  % colorlinks,
  final,
  pdftitle={Equivariant Dendroidal Segal Spaces},
  pdfauthor={Bonventre, P. and Pereira, L. A.},
  % pdfsubject={Your subject here},
  % pdfkeywords={keyword1, keyword2},
  linktoc=page
}
%\usepackage[open=false]{bookmark}

\input{commands.tex}%


%-------- Tikz ---------------------------

\usepackage{tikz}%
\usetikzlibrary{matrix,arrows,decorations.pathmorphing,
cd,patterns,calc}
\tikzset{%
  treenode/.style = {shape=rectangle, rounded corners,%
                     draw, align=center,%
                     top color=white, bottom color=blue!20},%
  root/.style     = {treenode, font=\Large, bottom color=red!30},%
  env/.style      = {treenode, font=\ttfamily\normalsize},%
  dummy/.style    = {circle,draw,inner sep=0pt,minimum size=2mm}%
}%

\usetikzlibrary[decorations.pathreplacing]



% ---- Commands on draft --------

\usepackage{ifdraft}
\ifdraft{
  \color[RGB]{63,63,63}
  % \pagecolor[rgb]{0.5,0.5,0.5}
  \pagecolor[RGB]{220,220,204}
  % \color[rgb]{1,1,1}
}


\usepackage[draft]{showkeys}
\usepackage{todonotes}%[obeyDraft]


% ----- Labels Changed? --------

\makeatletter

\def\@testdef #1#2#3{%
  \def\reserved@a{#3}\expandafter \ifx \csname #1@#2\endcsname
  \reserved@a  \else
  \typeout{^^Jlabel #2 changed:^^J%
    \meaning\reserved@a^^J%
    \expandafter\meaning\csname #1@#2\endcsname^^J}%
  \@tempswatrue \fi}

\makeatother


% ---- Commands --------

% new symbols

\newcommand{\mycircled}[2][none]{%
  \mathbin{
    \tikz[baseline=(a.base)]\node[draw,circle,inner sep=-1.5pt, outer sep=0pt,fill=#1](a){\ensuremath #2\strut};
cf.  }
}
\newcommand{\owr}{\mycircled{\wr}}

% replace symbols

\renewcommand{\hat}{\widehat}

% random

\renewcommand{\F}{\mathcal F}
\newcommand{\Q}{\mathcal Q}

\newcommand{\lltimes}{\underline{\ltimes}}


% detecting $\V$-categories:

\newcommand{\I}{\mathbb I}
\newcommand{\J}{\mathbb J}
\renewcommand{\1}{\eta}%{\ensuremath{\mathbb{id}}}

% lazy shortcuts

\newcommand{\SC}{\Sigma_{\mathfrak C}}
\newcommand{\OC}{\Omega_{\mathfrak C}}

\newcommand{\UV}{\underline{\mathcal V}}
\newcommand{\UC}{\underline{\mathfrak C}}


% ---- Title --------

\title{Equivariant Segal operads, simplicial operads, and dendroidal sets}

\author{Peter Bonventre, Lu\'is A. Pereira}%

\date{\today}


% ---- Document body --------

\begin{document}

\maketitle

\begin{abstract}
      Things and stuff
\end{abstract}

\tableofcontents




\subsubsection{Free operads and injective change of color}
\label{FREEOPCOL_SEC}

Let $f \colon \mathfrak{C} \to \mathfrak{D}$ be an inclusion of colors.
Our goal in this section is to show that the pullback of a free $\mathfrak{D}$-colored operad is a free $\mathfrak{C}$-colored operad, in a natural way.
As consequence, we will show that the left adjoint $\check f_!$ is underlying.

More precisely, our goal is to identify a functor 
$\widehat{f^{\**}}$ such that the diagram below commutes (up to natural isomorphism).
\begin{equation}\label{HATFST EQ}
      \begin{tikzcd}
            \mathsf{Sym}^{\mathfrak{D}} \ar{r}{\mathbb{F}_{\mathfrak{D}}} \ar[dashed]{d}[swap]{\widehat{f^{\**}}} &
            \mathsf{Op}^{\mathfrak{D}} \ar{d}{f^{\**}}
            \\
            \mathsf{Sym}^{\mathfrak{C}} \ar{r}[swap]{\mathbb{F}_{\mathfrak{C}}} &
            \mathsf{Op}^{\mathfrak{C}}
      \end{tikzcd}
\end{equation}

We now introduce some notation.
We will write 
$\Omega^0_{\mathfrak{C},\mathfrak{D}} \subseteq \Omega^0_{\mathfrak{D}}$
for the full subcategory of those trees whose root and leaves are labeled by $\mathfrak{C}$.
One then has that in the diagram below the square is a pullback square.
\begin{equation}\label{PULLFREEFREE EQ}
\begin{tikzcd}
	\Omega^0_{\mathfrak{C},\mathfrak{D}} \ar{r}{} \ar{d}{} &
	\Omega^0_{\mathfrak{D}} \ar{d}{} \ar{r} &
	\Sigma \wr \Sigma_{\mathfrak{D}} \ar{r} &
	\Sigma \wr \mathcal{V}^{op} \ar{r} &
	\mathcal{V}^{op}
\\
	\Sigma_{\mathfrak{C}} \ar{r}[swap]{} &
	\Sigma_{\mathfrak{D}}
\end{tikzcd}
\end{equation}
Moreover, since $\Sigma_{\mathfrak{C}} \to \Sigma_{\mathfrak{D}}$ is an inclusion of components, it follows that performing a Kan extension to $\Sigma_{\mathfrak{D}}$ and then restricting to $\Sigma_{\mathfrak{C}}$
is equivalent to performing the Kan extension of the top composite.




We next note that there is a natural map 
$\Omega^0_{\mathfrak{C},\mathfrak{D}} \to \Omega^0_{\mathfrak{C}}$,
defined by $T \mapsto \partial_{\mathfrak{D} \setminus \mathfrak{C}}(T)$,
collapsing those inner edges not in $\mathfrak{C}$, and we write
$\Omega^{0,\partial}_{\mathfrak{C},\mathfrak{D}}
\subseteq 
\Omega^{0}_{\mathfrak{C},\mathfrak{D}}
$ for the full subcategory mapping to $\Sigma_{\mathfrak{C}}$ under $\partial_{\mathfrak{D} \setminus \mathfrak{C}}$ (more explicitly, elements of $\Omega^{0,\partial}_{\mathfrak{C},\mathfrak{D}}$ are non-stick trees with root and leaves in $\mathfrak{C}$ and inner edges in $\mathfrak{D} \setminus \mathfrak{C}$).
Moreover, noting that for the canonical maps 
$\partial_{\mathfrak{D} \setminus \mathfrak{C}}(T) \to T$
the subtrees $T_v$ for $v \in V(\partial_{\mathfrak{D} \setminus \mathfrak{C}}(T))$
must be in $\Omega^{0,\partial}_{\mathfrak{D}, \mathfrak{C}}$
one obtains a vertex functor 
$V \colon \Omega_{\mathfrak{C},\mathfrak{D}}^{0}
\to \Sigma \wr \Omega_{\mathfrak{C},\mathfrak{D}}^{0,\partial}$.
One then has a diagram
\begin{equation}\label{PULLFREEFREE2 EQ}
\begin{tikzcd}
	\Omega^0_{\mathfrak{C},\mathfrak{D}} \ar{r}{} \ar{d}[swap]{\partial_{\mathfrak{D} \setminus \mathfrak{C}}} &
	\Sigma \wr \Omega^{0,\partial}_{\mathfrak{C},\mathfrak{D}} \ar{d}[swap]{\partial_{\mathfrak{D} \setminus \mathfrak{C}}} \ar{r} &
	\Sigma \wr \Sigma \wr \Sigma_{\mathfrak{D}} \ar{r} &
	\Sigma \wr \Sigma_{\mathfrak{D}} \ar{r} &
	\Sigma \wr \mathcal{V}^{op} \ar{r} &
	\mathcal{V}^{op}
\\
	\Omega_{\mathfrak{C}} \ar{r}[swap]{} \ar{d}[swap]{\mathsf{lr}} &
	\Sigma \wr \Sigma_{\mathfrak{C}}
\\
	\Sigma_{\mathfrak{C}}
\end{tikzcd}
\end{equation}
where the top composite is naturally isomorphic to the top composite in 
\eqref{PULLFREEFREE EQ}, and whose Kan extension can alternatively be computed by first computing the Kan extension
\begin{equation}\label{HATFSTDEF EQ}
\begin{tikzcd}
	\Omega_{\mathfrak{C},\mathfrak{D}}^{0,\partial} \ar{r} \ar{d} &
	\Sigma \wr \Sigma_{\mathfrak{D}} \ar{r}{X} & 
	\Sigma \wr \mathcal{V}^{op} \ar{r} & 
	\mathcal{V}^{op}
\\
	\Sigma_{\mathfrak{C}} \ar[dashed]{rrru}[swap]{\widehat{f^{\**}}(X)}
\end{tikzcd}
\end{equation}
and then performing the associated Kan extension along 
$\Omega_{\mathfrak{C}} \to \Sigma_{\mathfrak{C}}$
in \eqref{PULLFREEFREE2 EQ}.

\begin{definition}
      \label{HATFST_DEF}
      Given $X \in \Sym^{\mathfrak D}$, define $\widehat{f^{\**}}(X)$ to be the left Kan extension in \eqref{HATFSTDEF EQ}.
\end{definition}

Put together, these observations show the desired claim that
$
f^{\**} \mathbb{F}_{\mathfrak{D}}(X) \simeq
\mathbb{F}_{\mathfrak{C}} \widehat{f^{\**}}(X)
$
as symmetric sequences.

Checking that this isomorphism agrees with the operad structures requires some extra work, which we now briefly sketch.
Firstly, to simplify the discussion concerning Kan extensions,
it is preferable to first prove the analogous result for spans, i.e. that there is 
$\widehat{f^{\**}_s}$ as in the following analogue of 
\eqref{HATFST EQ}.
\begin{equation}\label{HATFSTSP EQ}
\begin{tikzcd}
	\mathsf{WSpan}(\Sigma_{\mathfrak{D}},\mathcal{V}^{op})
	\ar{r}{N_{\mathfrak{D}}} \ar[dashed]{d}[swap]{\widehat{f^{\**}_s}} &
	\mathsf{Alg}_{N_{\mathfrak{D}}}
	\ar{d}{f^{\**}_s}
\\
	\mathsf{WSpan}(\Sigma_{\mathfrak{C}},\mathcal{V}^{op})
	\ar{r}{N_{\mathfrak{C}}} &
	\mathsf{Alg}_{N_{\mathfrak{C}}}
\end{tikzcd}
\end{equation}
Firstly, as a natural adaptation of 
\eqref{HATFSTDEF EQ}
one defines $\widehat{f^{\**}_s}$
on the span $\Sigma_{\mathfrak{D}} \rightarrow A \leftarrow \mathcal{V}^{op}$
to be given by 
\[
\begin{tikzcd}
	\Omega_{\mathfrak{C},\mathfrak{D}}^{0,\partial} \wr A
	\ar{r} \ar{d} &
	\Sigma \wr A \ar{r} \ar{d} &
	\Sigma \wr \mathcal{V}^{op} \ar{r} &
	\mathcal{V}^{op}
\\
	\Omega_{\mathfrak{C},\mathfrak{D}}^{0,\partial}
	\ar{r} \ar{d} &
	\Sigma \wr \Sigma_{\mathfrak{D}}
\\
	\Sigma_{\mathfrak{C}}
\end{tikzcd}
\]
where the square defining 
$\Omega^{0,\partial}_{\mathfrak{C},\mathfrak{D}} \wr A$ is a pullback. 

Assuming for the moment that \eqref{HATFSTSP EQ} indeed commutes up to natural isomorphism (i.e. that compatibility with the monad structures as been verified), one then has the string of isomorphisms
\[
f^{\**} \circ \mathbb{F}_{\mathfrak{D}} = 
f^{\**} \circ \mathsf{Lan} \circ N_{\mathfrak{D}} \circ \iota \xleftarrow{\simeq}
\mathsf{Lan} \circ f^{\**}_s \circ N_{\mathfrak{D}} \circ \iota \simeq
\mathsf{Lan} \circ N_{\mathfrak{C}} \circ \widehat{f^{\**}_s} \circ \iota
\xleftarrow{\simeq}
\mathsf{Lan} \circ N_{\mathfrak{C}} \circ \iota \circ \mathsf{Lan}
\circ \widehat{f^{\**}} \circ \iota
=
\mathbb{F}_{\mathfrak{C}} \circ \mathsf{Lan}
\circ \widehat{f^{\**}_s} \circ \iota
=
\mathbb{F}_{\mathfrak{C}}
\circ \widehat{f^{\**}}
\]
where the second step follows since 
$\Sigma_{\mathfrak{C}} \to \Sigma_{\mathfrak{D}}$ is an inclusion of components and the fourth step follows
from Lemmas \ref{FINWRPRODLIM LEM} and \ref{LANPULLCOMA LEM}.

We now tackle the remaining claim of showing that
\eqref{HATFSTSP EQ} commutes up to isomorphism.

We first need to discuss higher string generalizations of the tree categories 
$\Omega^{0}_{\mathfrak{C},\mathfrak{D}}$
and
$\Omega^{0,\partial}_{\mathfrak{C},\mathfrak{D}}$.

Namely, we define full subcategories 
$\Omega^{n,\partial}_{\mathfrak{C},\mathfrak{D}}
\subseteq
\Omega^{n}_{\mathfrak{C},\mathfrak{D}}
\subseteq
\Omega^{n}_{\mathfrak{D}}
$
where the objects of 
$\Omega^{n}_{\mathfrak{C},\mathfrak{D}}$
are strings $T_0 \to T_1 \to \cdots \to T_n$
of maps
such that $T_k \in \Omega_{\mathfrak{C}}$ for $k<n$
and the objects of 
$\Omega^{n,\partial}_{\mathfrak{C},\mathfrak{D}}$
satisfy the further restriction that
$T_{n-1}=\partial_{\mathfrak{D} \setminus \mathfrak{C}}(T_n)$.


Note that the vertex and simplicial operators then restrict to maps
\[
	\Omega_{\mathfrak{C},\mathfrak{D}}^{n, \partial} 
	\xrightarrow{V}
	\Sigma \wr \Omega_{\mathfrak{C},\mathfrak{D}}^{n-1, \partial} 
\quad
	n \geq 1
\qquad \qquad
	\Omega_{\mathfrak{C},\mathfrak{D}}^{n, \partial} 
	\xrightarrow{d_i}
	\Omega_{\mathfrak{C},\mathfrak{D}}^{n-1, \partial}
\quad
	n-2 \geq i \geq 0
\qquad \qquad
	\Omega_{\mathfrak{C},\mathfrak{D}}^{n, \partial} 
	\xrightarrow{d_{n-1},\simeq}
	\Omega_{\mathfrak{C},\mathfrak{D}}^{n-1}
\]
where we note that the last map is an isomorphism. 
Indeed, the $\Omega^0_{\mathfrak{C},\mathfrak{D}}$ 
on the top left corner of \eqref{PULLFREEFREE2 EQ} is best viewed as
being $\Omega^{1,\partial}_{\mathfrak{C},\mathfrak{D}}$, so that the leftmost horizontal map is the regular $V$ and the vertical maps are $d_1,d_0$.

The required compatibility with the operadic structure now follows by noting that the composite
$N_{\mathfrak{C}} N_{\mathfrak{C}} \widehat{f^{\**}_s}
\simeq
N_{\mathfrak{C}} f^{\**}_s N_{\mathfrak{D}}
\to
f^{\**}_s N_{\mathfrak{D}}
$
is encoded by the diagram
\[
\begin{tikzcd}[column sep =8pt]
	\Omega^{2,\partial}_{\mathfrak{C},\mathfrak{D}} \wr A \ar{d}[swap]{d^1}{\simeq}  \ar{r} &
	\Sigma \wr \Omega^{1,\partial}_{\mathfrak{C},\mathfrak{D}} \wr A \ar{d}[swap]{d^0}{\simeq} \ar{r} &
	\Sigma^{\wr 2} \wr \Omega^{0,\partial}_{\mathfrak{C},\mathfrak{D}} \wr A \ar{r} &
	|[alias=UUU]|
	\Sigma^{\wr 3} \wr A \ar{d}[swap]{\sigma^1} \ar{r} &
	\Sigma^{\wr 3} \wr \mathcal{V}^{op} \ar{d}[swap]{\sigma^1} \ar{r}{\otimes} &
	\Sigma^{\wr 2} \wr \mathcal{V}^{op} \ar{r}{\otimes} &
	|[alias=UUUU]|
	\Sigma \wr \mathcal{V}^{op} \ar{r}{\otimes} \ar[equal]{d} &
	\mathcal{V}^{op} \ar[equal]{d}
\\
	\Omega^1_{\mathfrak{C},\mathfrak{D}} \wr A \ar{r} \ar{d}[swap]{d^0} &
	|[alias=VVV]|
	\Sigma \wr \Omega^0_{\mathfrak{C},\mathfrak{D}} \wr A \ar{rr} & &
	|[alias=U]|
	\Sigma^{\wr 2} \wr A \ar{r} \ar{d}[swap]{\sigma^0} &
	|[alias=VVVV]|
	\Sigma^{\wr 2} \wr \mathcal{V}^{op} \ar{rr}{\otimes} \ar{d}[swap]{\sigma^0} & &
	\Sigma \wr \mathcal{V}^{op} \ar{r}{\otimes} &
	|[alias=UU]|
	\mathcal{V}^{op} \ar[equal]{d}
\\
	|[alias=V]|
	\Omega^0_{\mathfrak{C},\mathfrak{D}} \wr A \ar{rrr} & & &
	\Sigma \wr A \ar{r} &
	|[alias=VV]|
	\Sigma \wr \mathcal{V}^{op} \ar{rrr}{\otimes} & & &
	\mathcal{V}^{op}
\arrow[Leftrightarrow, from=V, to=U,shorten >=0.15cm,shorten <=0.15cm
,swap,"\pi"
]
\arrow[Leftrightarrow, from=VV, to=UU,shorten >=0.15cm,shorten <=0.15cm
]
\arrow[Leftrightarrow, from=VVV, to=UUU,shorten >=0.15cm,shorten <=0.15cm
,swap,"\pi"
]
\arrow[Leftrightarrow, from=VVVV, to=UUUU,shorten >=0.15cm,shorten <=0.15cm
]
\end{tikzcd}
\]
while the composite
$N_{\mathfrak{C}} N_{\mathfrak{C}} \widehat{f^{\**}_s}
\to
N_{\mathfrak{C}} \widehat{f^{\**}_s}
\simeq
f^{\**}_s N_{\mathfrak{D}}
$
is encoded by the diagram
\[
\begin{tikzcd}[column sep =8pt]
	\Omega^{2,\partial}_{\mathfrak{C},\mathfrak{D}} \wr A \ar{r} \ar{d}[swap]{d^0} &
	\Sigma \wr \Omega^{1,\partial}_{\mathfrak{C},\mathfrak{D}} \wr A \ar{r} &
	|[alias=UUU]|
	\Sigma^{\wr 2} \wr \Omega^{0,\partial}_{\mathfrak{C},\mathfrak{D}} \wr A
	\ar{d}[swap]{\sigma^0} \ar{r} &
	\Sigma^{\wr 3} \wr A \ar{d}[swap]{\sigma^0} \ar{r} &
	\Sigma^{\wr 3} \wr \mathcal{V}^{op} \ar{d}[swap]{\sigma^0} \ar{r}{\otimes} &
	\Sigma^{\wr 2} \wr \mathcal{V}^{op} \ar{d}[swap]{\sigma^0} \ar{r}{\otimes} &
	\Sigma \wr \mathcal{V}^{op} \ar{r}{\otimes} & 
	|[alias=UUUU]|
	\mathcal{V}^{op} \ar[equal]{d}
\\
	|[alias=VVV]|
	\Omega^{1,\partial}_{\mathfrak{C},\mathfrak{D}} \wr A \ar{rr} \ar{d}[swap]{d^0}{\simeq} & &
	\Sigma \wr \Omega^{0,\delta}_{\mathfrak{C},\mathfrak{D}} \wr A \ar{r} &
	|[alias=U]|
	\Sigma^{\wr 2} \wr A \ar{r} \ar{d}[swap]{\sigma^0} &
	\Sigma^{\wr 2} \wr \mathcal{V}^{op} \ar{r}{\otimes} \ar{d}[swap]{\sigma^0} &
	|[alias=VVVV]|
	\Sigma \wr \mathcal{V}^{op} \ar{rr}{\otimes} & &
	|[alias=UU]|
	\mathcal{V}^{op} \ar[equal]{d}
\\
	|[alias=V]|
	\Omega^0_{\mathfrak{C}} \wr A \ar{rrr} & & &
	\Sigma \wr A \ar{r} &
	|[alias=VV]|
	\Sigma \wr \mathcal{V}^{op} \ar{rrr}{\otimes} & & &
	\mathcal{V}^{op}
\arrow[Leftrightarrow, from=V, to=U,shorten >=0.15cm,shorten <=0.15cm
,swap,"\pi"
]
\arrow[Leftrightarrow, from=VV, to=UU,shorten >=0.15cm,shorten <=0.15cm
]
\arrow[Leftrightarrow, from=VVV, to=UUU,shorten >=0.15cm,shorten <=0.15cm
,swap,"\pi"
]
\arrow[Leftrightarrow, from=VVVV, to=UUUU,shorten >=0.15cm,shorten <=0.15cm
]
\end{tikzcd}
\]


This finishes the argument establishing \eqref{HATFST EQ}.

We explore several consequences.
\begin{corollary}
      \label{FTF_COR}
      For any injective $f \colon \mathfrak C \to \mathfrak D$, the following hold.
      \begin{enumerate}[label = (\roman*)]
      \item $f^{\**} \mathbb F_{\mathfrak D} f_! \simeq \mathbb F_{\mathfrak C}$.
      \item $f^{\**} f_! = id_{\Sym_{\mathfrak C}}$ and $f^{\**} \check{f_!} = id_{\Op_{\mathfrak C}}$.
      \item $U \check{f_!} = f_! U$.
      \end{enumerate}
\end{corollary}
\begin{proof}
      For (i), it suffices to show that $\widehat{f^{\**}} f_!$ is the identity,
      which is straightforward from the definition given in \eqref{HATFSTDEF EQ}.
      % $\widehat{f^{\**}} f_! X(\ksi) = \varnothing$ unless $\ksi \in \Sigma_{\mathfrak C}$, in which case
      % $\widehat{f^{\**}} f_! X(\ksi) = X(\ksi)$.
      For (ii), the first half is immediate,
      and using \eqref{CFS_EQ} plus the fact that $f^{\**}$ commutes with reflexive coequalizers\footnote{
        As with $\mathbb F$, this follows as $\otimes$ commuting with colimits in both variables implies it commutes with reflexive coequalizers; see e.g. \cite[Lemma 2.3.2]{Rez96}.},
      the second half follows from part (i).

      For (iii), following (i) and (ii) it is straightforward from the definitions that we have natural isomorphisms
      \[
            \mathbb F_{\mathfrak D} f_! U \mathbb F_{\mathfrak C} \xrightarrow{\simeq} % \mathbb F_{\mathfrak D} f_! U(\simeq)
            \mathbb F_{\mathfrak D} f_! f^{\**} U \mathbb F_{\mathfrak D} f_! \xrightarrow{\simeq} % \mathbb F_{\mathfrak D}(\simeq)
            \mathbb F_{\mathfrak D} U \mathbb F_{\mathfrak D} f_!,
      \]
      so in particular for any $\O \in \Op^{\mathfrak C}$
      % there exists a unique $\mathfrak D$-colored operad structure on $f_! X$
      % such that the canonical map $X \to f_! X$ is a map of operads.
      there is an induced $\mathfrak D$-colored operad structure on $f_! X$,
      which is indeed described by the coequalizer \eqref{CFS_EQ}.
\end{proof}



{\color{OliveGreen} % --------------------------------------------------------------------------------
  \begin{corollary}
        For any $(G, \Sigma)$-family $\F$ and injective map of colors $f \colon \mathfrak C \to \mathfrak D$,
        the map $f^{\**}: \Op^{\mathfrak D}_\F(\V) \to \Op^{\mathfrak C}_\F(\V)$ preserves (trivial) cofibrations between cofibrant objects.
  \end{corollary}
  \todo[inline]{the statement is true as written, however the machinery to prove it entirely has not been developed.}
  \begin{proof}
        We first show $f^{\**}$ preserves generating (trivial) cofibrations.
        To start, let $X$ be a free symmetric sequence
        \[
              X = \Sigma_{\mathfrak D}(-, \ksi)[K] = \Hom_{\Sigma_{\mathfrak D}}(-,\ksi) \cdot_{\Aut(\ksi)} K
        \]
        on some $\mathfrak D$-signature $\ksi$ and $K \in \V^{\Aut(\ksi)}$.
        % From the definition of $\Omega^{0,\partial}_{\mathfrak C, \mathfrak D}$ and $\widehat{f^{\**}}(X)$ from \eqref{HATFSTDEF},
        It is straightforward to check that the only case in which $\widehat{f^{\**}}(X)$ is not the initial symmetric sequence (i.e. constant to the initial object in $\V$)
        is when $\ksi$ is in fact a $\mathfrak C$-signature, in which case
        \begin{equation}
              \label{SDFI_EQ}
              \Sigma_{\mathfrak D}(-,\ksi)[K] = f_!\Sigma_{\mathfrak C}(-,\ksi)[K].
        \end{equation}
        The claim for generating (trivial) cofibrations then follows from Corollary \ref{FTF_COR}(i).

        For the general case, we have that since $\otimes$ commutes with colimits in each variable, it commutes with filtered colimits, in particular transfinite compositions (\cite[Lemma 2.3.3]{Rez96}), and hence $f^{\**}$ preserves transfinite compositions.
        Moreover, $f^{\**}$ preserves pushouts of free symmetric sequences
        $\Sigma_{\mathfrak D}(-,\ksi)[i]$
        with $\ksi \in \Sigma_{\mathfrak C}$ by \eqref{SDFI_EQ}, Corollary \ref{FTF_COR}(iii), and Lemma \ref{BASICPUSH LEMMA}.

        It remains to check that the image under $f^{\**}$ of a pushout over $\Sigma_{\mathfrak D}(-,\ksi)[i]$ with $\ksi \notin \Sigma_{\mathfrak C}$ is a (trivial) cofibration.
        
        {\color{red} can't go on from here yet}
  \end{proof}

  The following example demonstrates that $f^{\**}$ need not preserve all pushouts.
  \begin{example}
        Let $f \colon \set{0} \hookrightarrow \set{0,1}$, and consider the following pushouts in $\Cat^{\set{0,1}}(\sSet)$.
        \[
              \begin{tikzcd}
                    \set{0,1} \arrow[d, "{\mathbb F(-, (1;1))[\partial \Delta[0] \to \Delta[0]]}"'] \arrow[r]
                    &
                    {[1]} \arrow[d]
                    &&& % --------------------
                    \set{0,1} \arrow[d, "{\mathbb F(-, (1;1))[\partial \Delta[0] \to \Delta[0]]}"'] \arrow[r]
                    &
                    \widetilde{[1]} \arrow[d]
                    \\
                    \set{0} \amalg \mathbb N_0\set{1} \arrow[r]
                    &
                    \mathcal P
                    &&& % ----------
                    \set{0} \amalg \mathbb N_0\set{1} \arrow[r]
                    &
                    \tilde{\mathcal P}
              \end{tikzcd}
        \]
        In both cases, $f^{\**}$ applied to the left vertical arrows yields $id_{\varnothing}$.
        However, the pushout on the left is preserved by $f^{\**}$, as
        $f^{\**}([1]) = f^{\**}(\mathcal P) = \**$,
        but the pushout on the right is not preserved, as
        $f^{\**}(\widetilde{[1]}) = \**$ but $f^{\**}(\tilde{\mathcal P}) = \mathbb N_0$.

        \todo[inline]{In this case, the RHS is still a pushout of a free thing, but I don't know how to make this systematic
          - it must preserve cofibrant objects, but Lemma \ref{BASICPUSH LEMMA} only takes care of the first case from \cite[Thm. 1.15]{BM13}.}
  \end{example}
} % ---------------------------------------- OLIVEGREEN ----------------------------------------











\bibliography{biblio-new}{}
\bibliographystyle{amsalpha2}



\end{document}


%%% Local Variables:
%%% mode: latex
%%% TeX-master: t
%%% End:
