\documentclass[a4paper,10pt
%,draft
%, final
]{article}%

\pdfcompresslevel=0
\pdfobjcompresslevel=0

\usepackage[hidelinks]{hyperref}
\hypersetup{
  % colorlinks,
  final,
  pdftitle={Equivariant Dendroidal Segal Spaces},
  pdfauthor={Bonventre, P. and Pereira, L. A.},
  % pdfsubject={Your subject here},
  % pdfkeywords={keyword1, keyword2},
  linktoc=page
}
%\usepackage[open=false]{bookmark}

\input{commands.tex}%


%-------- Tikz ---------------------------

\usepackage{tikz}%
\usetikzlibrary{matrix,arrows,decorations.pathmorphing,
cd,patterns,calc}
\tikzset{%
  treenode/.style = {shape=rectangle, rounded corners,%
                     draw, align=center,%
                     top color=white, bottom color=blue!20},%
  root/.style     = {treenode, font=\Large, bottom color=red!30},%
  env/.style      = {treenode, font=\ttfamily\normalsize},%
  dummy/.style    = {circle,draw,inner sep=0pt,minimum size=2mm}%
}%

\usetikzlibrary[decorations.pathreplacing]



% ---- Commands on draft --------

\usepackage{ifdraft}
\ifdraft{
  \color[RGB]{63,63,63}
  % \pagecolor[rgb]{0.5,0.5,0.5}
  \pagecolor[RGB]{220,220,204}
  % \color[rgb]{1,1,1}
}


\usepackage[draft]{showkeys}
\usepackage{todonotes}%[obeyDraft]


% ----- Labels Changed? --------

\makeatletter

\def\@testdef #1#2#3{%
  \def\reserved@a{#3}\expandafter \ifx \csname #1@#2\endcsname
  \reserved@a  \else
  \typeout{^^Jlabel #2 changed:^^J%
    \meaning\reserved@a^^J%
    \expandafter\meaning\csname #1@#2\endcsname^^J}%
  \@tempswatrue \fi}

\makeatother


% ---- Commands --------

% new symbols

\newcommand{\mycircled}[2][none]{%
  \mathbin{
    \tikz[baseline=(a.base)]\node[draw,circle,inner sep=-1.5pt, outer sep=0pt,fill=#1](a){\ensuremath #2\strut};
cf.  }
}
\newcommand{\owr}{\mycircled{\wr}}

% replace symbols

\renewcommand{\hat}{\widehat}

% random

\renewcommand{\F}{\mathcal F}
\newcommand{\Q}{\mathcal Q}

\newcommand{\lltimes}{\underline{\ltimes}}


% detecting $\V$-categories:

\newcommand{\I}{\mathbb I}
\newcommand{\J}{\mathbb J}
\renewcommand{\1}{\eta}%{\ensuremath{\mathbb{id}}}

% lazy shortcuts

\newcommand{\SC}{\Sigma_{\mathfrak C}}
\newcommand{\OC}{\Omega_{\mathfrak C}}

\newcommand{\UV}{\underline{\mathcal V}}
\newcommand{\UC}{\underline{\mathfrak C}}


% ---- Title --------

\title{Equivariant Segal operads, simplicial operads, and dendroidal sets}

\author{Peter Bonventre, Lu\'is A. Pereira}%

\date{\today}


% ---- Document body --------

\begin{document}

\maketitle

\begin{abstract}
      Things and stuff
\end{abstract}

\tableofcontents





\section{An approach to colored trees via edge data}


\subsection{$2$-overcategories}

\begin{definition}
Let $\mathcal{C}$ be a $2$-category and $c \in \mathcal{C}$ an object.
We write $\mathcal{C} \downarrow^r c$ for the $2$-category such that:
\begin{itemize}
	\item objects are the arrows $\beta \colon b \to c$;
	\item an $1$-arrow from 
	$\beta \colon b \to c$
	to
	$\beta' \colon b' \to c$ 
	is a pair $(f,\phi)$
	formed by a $1$-arrow $f\colon b \to b'$ and a $2$-arrow
	$\phi \colon \beta' f \Rightarrow \beta$
		\begin{equation}
		\begin{tikzcd}[row sep = tiny, column sep = 35pt]
			b \arrow{dr}[name=U]{\beta} \arrow{dd}[swap]{f}
		\\
			& c
		\\
			|[alias=V]| b' \arrow{ur}[swap]{\beta'}
		\arrow[Rightarrow, from=V, to=U,shorten >=0.25cm,shorten <=0.25cm
		,swap,"\phi"
		]
		\end{tikzcd}
		\end{equation}
	\item a $2$-arrow from $(f,\phi)$ to $(f',\phi')$ is a $2$-arrow $\varphi \colon f \to f'$ such that
	$\phi' \circ \varphi \beta' = \phi$.
		\begin{equation}
		\begin{tikzcd}[column sep = 50pt]
			b \arrow{dr}[name=U]{\beta} 
			\arrow[bend right]{dd}[swap]{f}[name=F]{}
			\arrow[bend left]{dd}{f'}[swap,name=FF]{}
			&
		&
			b \arrow{dr}[name=U2]{\beta} 
			\arrow[bend right]{dd}[swap]{f}
			&
		\\
			& c
		&
			& c
		\\
			b' \arrow{ur}[swap]{\beta'}[near start, name=V]{}
			&
		&
			|[alias=V2]| b' \arrow{ur}[swap]{\beta'}
			&
		\arrow[Rightarrow, from=V, to=U,shorten >=0.25cm,shorten <=0.25cm
		,swap,"\phi'"
		]
		\arrow[Rightarrow, from=F, to=FF,shorten >=0.0cm,shorten <=0.0cm
		,swap,"\varphi"
		]
		\arrow[Rightarrow, from=V2, to=U2,shorten >=0.25cm,shorten <=0.25cm
		,swap,"\phi"
		]
		\end{tikzcd}
		\end{equation}
\end{itemize}
\end{definition}

\begin{example}
When $\mathcal{C}$ is a $1$-category, then 
$\mathcal{C} \downarrow^r c$ is the usual overcategory.
\end{example}

\begin{example}
When $\mathcal{C} = \mathsf{Cat}$ is the $2$-category of categories, 
then the underlying $1$-category of
the $2$-category
$\mathsf{Cat} \downarrow^r \mathcal{B}$
coincides with the category of weak right spans
$\mathsf{WSpan}^r(\**,\mathcal{B})$.
\end{example}

\begin{remark}
One can check that when $\mathcal{B}$ is a complete category, one obtains a limit $2$-functor
$\mathsf{lim} \colon \mathsf{Cat} \downarrow^r \mathcal{B} \to \mathcal{B}$.
Note that here the target $\mathcal{B}$ is a $2$-category where all $2$-arrows are identities, so that we are in particular claiming that whenever two $1$-arrows in 
$\mathsf{Cat} \downarrow^r \mathcal{B}$
are connected by a $2$-arrow they induce the same map on limits.
\end{remark}

\begin{notation}\label{SIMPONE NOT}
An $1$-arrow  $(f,\phi) \colon \beta \to \beta'$ in 
$\mathcal{C} \downarrow c$
will be called a 
\textit{simple $1$-arrow}
if $\phi = id_{\beta}$,
i.e. if the arrow exhibits the commutative diagram
$\beta = \beta' f$.
\end{notation}


\begin{remark}\label{SPANLIM REM}
Given a diagram 
$J \xrightarrow{j \mapsto (B_j \colon \mathcal{C}_j \to \mathcal{B})}
\mathsf{Cat} \downarrow^r \mathcal{B} $
and a cone over it, 
i.e. an object
$(B \colon \mathcal{C} \to \mathcal{B}) \in
\left(\mathsf{Cat} \downarrow^r \mathcal{B} \right)$
together with compatible maps 
$(f_j,\phi_j) \colon B \to B_j$,
this will be a limit in 
$\mathsf{Cat} \downarrow^r \mathcal{B}$
provided that
\[
	\mathcal{C} = \lim_{j \in J} \mathcal{C}_j
\qquad
	B = \colim_{j \in J} B_j f_j
\]
where the limit takes place in 
$\mathsf{Cat}$
and the colimit in 
$\mathsf{Fun}(\mathcal{C}, \mathcal{B})$.
\end{remark}


The following is a slight strengthening of \cite[Lemma A.6]{BP_geo}. 

\begin{proposition}
The $1$-category of functors 
$\mathsf{Fun}(\mathcal{C} \ltimes \mathcal{D}_{\bullet} \to \mathcal{V})$ is naturally isomorphic to the $1$-category of lifts
\[
\begin{tikzcd}
& \mathsf{Cat} \downarrow^l \mathcal{V} \ar{d}{\mathsf{fgt}}
\\
\mathcal{C} \ar{r}[swap]{\mathcal{D}_{\bullet}} \ar[dashed]{ru}&
\mathsf{Cat}
\end{tikzcd}
\]
\end{proposition}


\subsection{Edge data}


%Recall the identification
%\[
%\Omega^t \simeq |\Omega^{\bullet}|
%\]
%of the category $\Omega^t$ of trees and tall maps with the realization of the simplicial object in categories with $n$-th level the planar $n$-strings $\Omega^{n}$.
%It then follows that the target functors
%$\Omega^n \to \Omega^t$
%given by 
%$(T_0 \to T_1 \to \cdots \to T_n)
%\mapsto T_n$
%define a simplicial object in 
%$\mathsf{WSpan}^r(\**,\Omega^t)$
%where all simplicial operators other than the top faces $d_n$ are $1$-arrows in
%$\mathsf{WSpan}^r(\**,\Omega^t)$.


Our goal in this section is to show that the string categories $\Omega^n$ in \cite{BP_geo} and the natural functors connecting them can naturally be extended to objects and functors in 
the $2$-category
$\mathsf{Cat} \downarrow^r \mathsf{F}$.

Firstly, let
$\boldsymbol{E} \colon \Omega \to \mathsf{F}$
denote the (forgetful) functor 
sending a planarized tree to the underlying ordered set of edges. 
We then obtain further edge functors 
$\boldsymbol{E} \colon \Omega^n \to \mathsf{F}$
via the formula 
$\boldsymbol{E}(T_0 \to T_1 \to \cdots \to T_n)=
\boldsymbol{E}(T_n)$.
It is immediate that the simplicial operators
$d^i \colon \Omega^n \to \Omega^{n-1}$
for $0 \leq i <n$
and
$s^j \colon \Omega^n \to \Omega^{n+1}$
for $-1 \leq j \leq n$
are compatible with the edge functors,
and can thus be regarded as simple $1$-arrows in 
$\mathsf{Cat} \downarrow^r \mathsf{F}$.
Furthermore, the top face operator
$d^n \colon \Omega^n \to \Omega^{n-1}$ can be extended to a (non-simple) $1$-arrow in 
$\mathsf{Cat} \downarrow^r \mathsf{F}$
via the obvious natural transformation
$\boldsymbol{E}(T_{n-1}) \to \boldsymbol{E}(T_{n})$.
It is straightforward to check that the $d^i$, $s^j$
still satisfy the simplicial identities when regarded as 
$1$-arrows in $\mathsf{Cat} \downarrow^r \mathsf{F}$.

Our next goal is to discuss the interaction of edge data with the vertex functors
$\boldsymbol{V} \colon \Omega^n \to \Sigma \wr \Omega^{n-1}$.
To do so, we must first discuss how the 
$\Sigma \wr (-)$ construction itself interacts with edge data.
It is immediate that the $\Sigma \wr (-)$ construction defines a $2$-functor 
$\mathsf{Cat} \downarrow^r \mathsf{F} \to 
\mathsf{Cat} \downarrow^r \Sigma \wr \mathsf{F}$
and by whiskering with the coproduct 
$\Sigma \wr \mathsf{F} \xrightarrow{\amalg} \mathsf{F}$ 
one obtains a $2$-endofunctor from 
$\mathsf{Cat} \downarrow^r \mathsf{F}$
to itself, which we abusively again call $\Sigma \wr (-)$.
Explicitly, in the more interesting case of $1$-arrows,
the endofunctor $\Sigma \wr (-)$ sends
an $1$-arrow $(f,\phi)$
as on the left below 
to the composite
$\left(\Sigma \wr f, \amalg (\Sigma \wr \phi)\right)$
on the right.

\begin{equation}
\begin{tikzcd}[row sep = 10pt, column sep = 40pt]
	\mathcal{C}_1 \arrow{dr}[name=U]{E_1} \arrow{dd}[swap]{f} &
&
	\Sigma \wr \mathcal{C}_1 \arrow{dr}[name=UU]{\Sigma\wr E_1} \arrow{dd}[swap]{\Sigma \wr f} &
\\
	& \mathsf{F}
&
	& \Sigma \wr \mathsf{F} \ar{r}{\amalg} &
	\mathsf{F}
\\
	|[alias=V]| \mathcal{C}_2 \arrow{ur}[swap]{E_2} &
&
	|[alias=VV]| \Sigma \wr \mathcal{C}_2 \arrow{ur}[swap]{\Sigma\wr E_2} &
\arrow[Rightarrow, from=V, to=U,shorten >=0.25cm,shorten <=0.25cm
,swap,"\phi"
]
\arrow[Rightarrow, from=VV, to=UU,shorten >=0.25cm,shorten <=0.25cm
,swap,"\Sigma \wr \phi"
]
\end{tikzcd}
\end{equation}
By iterating the $\Sigma \wr (-)$ endofunctor, one
then has that the cosimplicial operators
$\delta^i \colon \Sigma^{\wr n} \wr \mathcal{C} \to 
\Sigma^{\wr n+1} \wr \mathcal{C}
$
for $0\leq i \leq n$
and 
$\sigma^i \colon \Sigma^{\wr n+2} \wr \mathcal{C} \to 
\Sigma^{\wr n+1} \wr \mathcal{C}
$
for $0\leq i \leq n$
can also be regarded as simple $1$-arrows in 
$\mathsf{Cat} \downarrow^r \mathsf{F}$.
Indeed, after unpacking definitions this follows from the following diagrams, where all squares are known to commute.
\[
\begin{tikzcd}
	\Sigma^{\wr n} \wr \mathcal{C} \ar{r} \ar{d}[swap]{\delta^i} &
	\Sigma^{\wr n} \wr \mathsf{F} \ar{r}{\amalg^{\circ n}} \ar{d}[swap]{\delta^i} &
	\mathsf{F} \ar[equal]{d}
&
	\Sigma^{\wr n+2} \wr \mathcal{C} \ar{r} \ar{d}[swap]{\sigma^i} &
	\Sigma^{\wr n+2} \wr \mathsf{F} \ar{r}{\amalg^{\circ n+2}} \ar{d}[swap]{\sigma^i} &
	\mathsf{F} \ar[equal]{d}
\\
	\Sigma^{\wr n+1} \wr \mathcal{C} \ar{r} &
	\Sigma^{\wr n+1} \wr \mathsf{F} \ar{r}{\amalg^{\circ n+1}} &
	\mathsf{F}
&
	\Sigma^{\wr n+1} \wr \mathcal{C} \ar{r} &
	\Sigma^{\wr n+1} \wr \mathsf{F} \ar{r}{\amalg^{\circ n+1}} &
	\mathsf{F}
\end{tikzcd}
\]

\begin{remark}\label{COSPULL REM}
Any commuting square in $\mathsf{Cat} \downarrow^r \mathsf{F}$
which is an underlying pullback of categories and such that two opposing arrows are simple $1$-arrows is a pullback in square in 
$\mathsf{Cat} \downarrow^r \mathsf{F}$.
In particular, for any $1$-arrow in $\mathsf{Cat} \downarrow^r \mathsf{F}$
with underlying functor
$\mathcal{C} \to \mathcal{D}$, the following squares are pullback squares in $\mathsf{Cat} \downarrow^r \mathsf{F}$.
\[
\begin{tikzcd}
	\Sigma^{\wr n} \wr \mathcal{C} \ar{r} \ar{d}[swap]{\delta^i} &
	\Sigma^{\wr n} \wr \mathcal{D} \ar{d}[swap]{\delta^i}
&
	\Sigma^{\wr n+2} \wr \mathcal{C} \ar{r} \ar{d}[swap]{\sigma^i} &
	\Sigma^{\wr n+2} \wr \mathcal{D}  \ar{d}[swap]{\sigma^i}
\\
	\Sigma^{\wr n+1} \wr \mathcal{C} \ar{r} &
	\Sigma^{\wr n+1} \wr \mathcal{D}
&
	\Sigma^{\wr n+1} \wr \mathcal{C} \ar{r} &
	\Sigma^{\wr n+1} \wr \mathcal{D}
\end{tikzcd}
\]
\end{remark}



We now obtain that for $0 \leq n$ the vertex functors
\[
\begin{tikzcd}[row sep = 0]
	\Omega^n \ar{r}{\boldsymbol{V}} & 
	\Sigma \wr \Omega^{n-1}
\\
	T_0 \to T_1 \to \cdots T_n \ar[mapsto]{r} &
	\left(T_{1,v} \to \cdots \to T_{n,v} \right)_{v \in V(T_0)}
\end{tikzcd}
\]
can be extended to $1$-arrows in 
$\mathsf{Cat} \downarrow^r \mathsf{F}$
via the natural transformations
\[
\coprod_{v \in V(T_0)} \boldsymbol{E}(T_{n,v})
\to
\boldsymbol{E}(T_n).
\]

Just as in \cite[\S 3.4]{BP_geo},
we inductively define further functors 
$\boldsymbol{V}^k \colon \Omega^n \to \Sigma \wr \Omega^{n-k-1}$
for 
$0 \leq k \leq n$
by setting $\boldsymbol{V}^0 = \boldsymbol{V}$
and letting 
$\boldsymbol{V}^{k+1}$
be the composites
\begin{equation}\label{VK1 EQ}
	\Omega^{n+1} \xrightarrow{\boldsymbol{V}} 
	\Sigma \wr \Omega^n \xrightarrow{\Sigma \wr \boldsymbol{V}^k}
	\Sigma^{\wr 2} \wr \Omega^n \xrightarrow{\sigma^0}
	\Sigma \wr \Omega^n.
\end{equation}
In addition, it is convenient to also set 
$\boldsymbol{V}^{-1} = \delta^0 \colon \Omega^n \to \Sigma \wr \Omega^n$.
Note that the formula
$
\boldsymbol{V}^{k+1} = 
\sigma^0 (\Sigma \wr \boldsymbol{V}^{k}) \boldsymbol{V}
$
given by \eqref{VK1 EQ} still holds for $k=-1$.

It will be useful have alternative descriptions 
of the $1$-arrows $\boldsymbol{V}^k$.
Given a string $T_0 \to \cdots \to T_n$ in $\Omega^n$,
let $V(T_0 \to \cdots \to T_n)$ denote the underlying set of $V(T_n)$ together with the total order induced lexicographically by the string of maps 
$V(T_n) \to \cdots V(T_0)$.

\begin{proposition}\label{VKDEF PROP}
Let $0 \leq k \leq n$.
The $1$-arrow 
$\boldsymbol{V}^k \colon \Omega^n \to \Sigma \wr \Omega^{n-k-1}$ in 
$\mathsf{Cat} \downarrow^r \mathsf{F}$
is given by the functor
$\boldsymbol{V}^k(T_0\to \cdots \to T_n)=
\left(T_{k+1,v} \to \cdots \to T_{n,v} \right)_{v \in V(T_0 \to \cdots \to T_k)}$
and the natural transformation
$
\coprod_{v \in V(T_0 \to \cdots \to T_k)} \boldsymbol{E}(T_{n,v})
\to
\boldsymbol{E}(T_n)
$.

Furthermore, for any $-1\leq k \leq n$ and $-1 \leq l \leq n$, the composite
\begin{equation}\label{VKGEN EQ}
	\Omega^{n+1} \xrightarrow{\boldsymbol{V}^l} 
	\Sigma \wr \Omega^n \xrightarrow{\Sigma \wr \boldsymbol{V}^k}
	\Sigma^{\wr 2} \wr \Omega^n \xrightarrow{\sigma^0}
	\Sigma \wr \Omega^n.
\end{equation}
equals $\boldsymbol{V}^{k+l+1}$.
\end{proposition}

\begin{proof}
The first claim follows by induction on $k$ together with the observation that
$V(T_0 \to T_1 \to \cdots \to T_n) =
\coprod_{v \in V(T_0)}
V(T_{1,v} \to \cdots \to T_{n,v})$
as ordered sets, so that one has
\begin{align*}
\sigma^0 
\left( \left( T_{k+1,v} \to \cdots \to T_{n,v}
\right)_{v \in V\left(T_{1,w} \to \cdots \to T_{k,w}\right)}
\right)_{w \in V(T_0)}
	= &
\left( T_{k+1,v} \to \cdots \to T_{n,v}
\right)_{v \in \underset{w \in V(T_0)}{\coprod} V\left(T_{1,w} \to \cdots \to T_{k,w}\right)}
\\
= &
\left( T_{k+1,v} \to \cdots \to T_{n,v}
\right)_{v \in V(T_0 \to \cdots T_k)}.
\end{align*}
For the second claim, the case $l=-1$ is a consequence of
the cosimplicial identities, the $l=0$ case is tautological and the $l>0$ cases follow inductively from the calculation
\begin{align*}
\sigma^0 (\Sigma \wr \boldsymbol{V}^k)\boldsymbol{V}^{l+1} = &
\sigma^0 (\Sigma \wr \boldsymbol{V}^k) \sigma^0 (\Sigma \wr \boldsymbol{V}^l)\boldsymbol{V} =
\sigma^0 \sigma^0 (\Sigma^{\wr 2} \wr \boldsymbol{V}^k)  (\Sigma \wr \boldsymbol{V}^l)\boldsymbol{V}=
\sigma^0 \sigma^1 (\Sigma^{\wr 2} \wr \boldsymbol{V}^k)  (\Sigma \wr \boldsymbol{V}^l)\boldsymbol{V}
\\
=&
\sigma^0 (\Sigma \wr \sigma^0) (\Sigma^{\wr 2} \wr \boldsymbol{V}^k)  (\Sigma \wr \boldsymbol{V}^l)\boldsymbol{V}
=
\sigma^0 (\Sigma \wr (\sigma^0 (\Sigma \wr \boldsymbol{V}^k)  \boldsymbol{V}^l))\boldsymbol{V}
=\sigma^0 (\Sigma \wr \boldsymbol{V}^{k+l+1})\boldsymbol{V} =
\boldsymbol{V}^{k+l+2}
\end{align*}
\end{proof}

We end this section by discussing the compatibilities between the $1$-arrows in $\mathsf{Cat} \downarrow^r \mathsf{F}$ determined by the functors
$d^i\colon \Omega^n \to \Omega^{n-1}$,
$s^j\colon \Omega^n \to \Omega^{n+1}$ and
$V^k \colon \Omega^n \to \Sigma \wr \Omega^{n-k-1}$.


\begin{proposition}\label{CATFDIAG PROP}
One has the following diagrams in the $2$-category
$\mathsf{Cat} \downarrow^r \mathsf{F}$.
\begin{itemize}
\item[(i)]
For $0\leq i < k \leq n$ there are $2$-isomorphisms $\pi_{i,k}$ and for $-1 \leq j \leq k \leq n$ there are commutative diagrams
\begin{equation}
\begin{tikzcd}[row sep = tiny, column sep = 35pt]
	\Omega^n
	\arrow{dr}[swap,name=U]{}{V^k} \arrow{dd}[swap]{d^i} &
&
	\Omega^n
	\arrow{dr}{V^k} \arrow{dd}[swap]{s^j} &
\\
	& \Sigma \wr \Omega^{n-k-1}
&
	& \Sigma \wr \Omega^{n-k-1}
\\
	|[alias=V]|
	\Omega^{n-1} \arrow{ur}[swap]{V^{k-1}} &
&
	\Omega^{n+1} \arrow{ur}[swap]{V^{k+1}} &
\arrow[Leftrightarrow, from=V, to=U,shorten >=0.15cm,shorten <=0.15cm
,swap,"\pi_{i,k}"
]
\end{tikzcd}
\end{equation}
\item[(ii)] 
For $-1 \leq k < i \leq n$ and for $-1 \leq k \leq j \leq n$
there are commutative diagrams
\begin{equation}
\begin{tikzcd}[row sep = 10pt, column sep = 35pt]
	\Omega^n
	\arrow{r}[swap,name=U]{}{V^k} \arrow{dd}[swap]{d^i} &
	\Sigma \wr \Omega^{n-k-1} \ar{dd}{d^{i-k-1}}
&
	\Omega^n
	\arrow{r}{V^k} \arrow{dd}[swap]{s^j} &
	\Sigma \wr \Omega^{n-k-1} \ar{dd}{s^{j-k-1}}
\\
\\
	|[alias=V]|
	\Omega^{n-1} \arrow{r}[swap]{V^{k}} &
	\Sigma \wr \Omega^{n-k-2}
&
	\Omega^{n+1} \arrow{r}[swap]{V^{k}} &
	\Sigma \wr \Omega^{n-k}
\end{tikzcd}
\end{equation}
\end{itemize}
Furthermore, the diagrams in (ii) are pullback squares in $\mathsf{Cat} \downarrow^r \mathsf{F}$.
\end{proposition}


\begin{proof}
The $2$-isomorphisms $\pi_{i,k}$ are the permutation isomorphisms (i.e. pullback arrows over $\Sigma$) 
encoded by the unordered isomorphism
$V(T_0 \to \cdots \widehat{T_i} \cdots \to T_k) \simeq V(T_0 \to \cdots \to T_k)$.

The remaining claims in (i) and (ii) are straightforward, though we note that in addition to checking that the diagrams commute in $\mathsf{Cat}$,
the fact that these are diagrams in 
$\mathsf{Cat} \downarrow^r \mathsf{F}$
requires checking the commutativity of a diagram in 
$\mathsf{Fun}(\Omega^n,\mathsf{F})$. In the case of the $\pi_{i,k}$ diagram, after unpacking definitions,
one needs to check the commutativity of the leftmost diagram below, which simplifies to the equivalent rightmost diagram, which clearly commutes.
\[
\begin{tikzcd}[column sep=5pt]
	\boldsymbol{E}(T_{0} \to \cdots \to T_{n})
	 &
	\underset{v \in V(T_0 \to \cdots \to T_k)}{\coprod} \boldsymbol{E}(T_{k+1,v} \to \cdots \to T_{n,v}) \ar{l}
&
	\boldsymbol{E}(T_{n})
	 &
	\underset{v \in V(T_0 \to \cdots \to T_k)}{\coprod} \boldsymbol{E}(T_{n,v}) \ar{l}
\\
	\boldsymbol{E}(T_{0} \to \cdots \widehat{T_i} \cdots \to T_{n})
 \ar{u} &
	\underset{v \in V(T_0 \to \cdots \widehat{T_i} \cdots \to T_k)}{\coprod} \boldsymbol{E}(T_{k+1,v} \to \cdots \to T_{n,v})
	\ar{u}[swap]{\simeq} \ar{l}
&
	\boldsymbol{E}(T_{n}) \ar[equal]{u} &
	\underset{v \in V(T_0 \to \cdots \widehat{T_i} \cdots \to T_k)}{\coprod} \boldsymbol{E}(T_{n,v})
	\ar{u}[swap]{\simeq} \ar{l}
\end{tikzcd}
\]
The claims concerning the other diagrams follow analogously.

For the pullback claim, note first that the $k=-1$ case, where $\boldsymbol{V}^{-1} = \delta^0$,
is an instance of Remark \ref{COSPULL REM}.
Moreover, combining the inductive description of the functors
$\boldsymbol{V}^{k}$ in \eqref{VK1 EQ} with Remark \ref{COSPULL REM} and the observation that the endofunctor $\Sigma \wr (-)$
preserves pullback squares,
one reduces to the cases with $k=0$.
Moreover since $d^i$ for $i<n$ and any $s^j$ 
are simple $1$-arrows, those cases follow by the first half of Remark \ref{COSPULL REM}.
For the remaining case of $d^i$ with $i=n$, the simplicial identities and the $i<n$ cases reduce to the case of $d^1$ with $n=1$.
The remaining claim is then that for each $1$-string
$T_0 \to T_1$ there is a natural isomorphism
\[
	\boldsymbol{E}(T_1) \simeq
	colim\left(
	\boldsymbol{E}(T_0) \leftarrow 
	\coprod_{v \in \boldsymbol{V}(T_0)} \boldsymbol{E}(T_{0,v}) \to 
	\coprod_{v \in \boldsymbol{V}(T_0)} \boldsymbol{E}(T_{1,v})
	\right)
\]
and this is clear from the discussion of substitution data.
\end{proof}

Lastly, the natural transformations $\pi_{i,k}$
satisfy a number of iterative and simplical relations, listed in the following result.


\begin{proposition}\label{CATFDIAG2 PROP}
In each of the following items, the two composite natural transformations coincide.
\begin{itemize}
\item[(IT1)]
For $0 \leq i < k $ and $-1 \leq l \leq n-k-1$
\begin{equation}
\begin{tikzcd}[row sep = 20pt, column sep = 25pt]
	|[alias=V]|
	\Omega^{n} \ar{r}{V^{k}}[swap,name=UU]{} \arrow{d}[swap]{d^i}&
	\Sigma \wr \Omega^{n-k-1} \ar{r}{V^l} &
	\Sigma^{\wr 2} \wr \Omega^{n-k-l-2} \ar{r}{\sigma^0} &
	\Sigma \wr \Omega^{n-k-l-2}
&
	\Omega^{n} \ar{r}{V^{k+l+1}}[swap,name=UUU]{} \arrow{d}[swap]{d^i}&
	\Sigma \wr \Omega^{n-k-l-2} &
\\
	|[alias=VV]|
	\Omega^{n-1} \arrow{ur}[swap]{V^{k-1}} & & &
&
	|[alias=VVV]|
	\Omega^{n-1} \arrow{ur}[swap]{V^{k+l}} &
\arrow[Leftrightarrow, from=VV, to=UU,shorten >=0.05cm,shorten <=0.05cm
,swap,"\pi"
]
\arrow[Leftrightarrow, from=VVV, to=UUU,shorten >=0.05cm,shorten <=0.05cm
,swap,"\pi"
]
\end{tikzcd}
\end{equation}

\item[(IT2)]
For $-1 \leq k < i < k + l + 1 \leq n$
\begin{equation}
\begin{tikzcd}[row sep = 20pt, column sep = 25pt]
	\Omega^n \ar{r}{V^k} \ar{d}[swap]{d^i} &
	|[alias=V]|
	\Sigma \wr \Omega^{n-k-1} \ar{r}{V^{l}}[swap,name=UU]{} \arrow{d}[swap]{d^{i-k-1}} &
	\Sigma^{\wr 2} \wr \Omega^{n-k-l-2} \ar{r}{\sigma^0} &
	\Sigma \wr \Omega^{n-k-l-2}
&
	\Omega^{n} \ar{r}{V^{k+l+1}}[swap,name=UUU]{} \arrow{d}[swap]{d^i}&
	\Sigma \wr \Omega^{n-k-l-2} &
\\
	\Omega^{n-1} \ar{r}{V^k} &
	|[alias=VV]|
	\Sigma \wr \Omega^{n-1} \arrow{ur}[swap]{V^{l-1}} & &
&
	|[alias=VVV]|
	\Omega^{n-1} \arrow{ur}[swap]{V^{k+l}} &
\arrow[Leftrightarrow, from=VV, to=UU,shorten >=0.05cm,shorten <=0.05cm
,swap,"\pi"
]
\arrow[Leftrightarrow, from=VVV, to=UUU,shorten >=0.05cm,shorten <=0.05cm
,swap,"\pi"
]
\end{tikzcd}
\end{equation}
\item[(FF1)]
For $0 \leq i < i' < k \leq n$
\begin{equation}
\begin{tikzcd}[row sep = 20pt, column sep = 35pt]
	\Omega^n
	\arrow{dr}[swap,name=U]{}{V^k} \arrow{d}[swap]{d^{i'}} &
&
	\Omega^n
	\arrow{dr}[swap,name=UUU]{}{V^k} \arrow{d}[swap]{d^i} &
\\
	|[alias=V]|
	\Omega^{n-1} \ar{r}[near start,swap]{V^{k-1}}[swap,name=UU]{} \arrow{d}[swap]{d^i}&
	\Sigma \wr \Omega^{n-k-1}
&
	|[alias=VVV]|
	\Omega^{n-1} \ar{r}[near start, swap]{V^{k-1}}[swap,name=UUUU]{} \ar{d}[swap]{d^{i'-1}} &
	\Sigma \wr \Omega^{n-k-1}
\\
	|[alias=VV]|
	\Omega^{n-2} \arrow{ur}[swap]{V^{k-2}} &
&
	|[alias=VVVV]|
	\Omega^{n-2} \arrow{ur}[swap]{V^{k-2}} &
\arrow[Leftrightarrow, from=V, to=U,shorten >=0.05cm,shorten <=0.05cm
,swap,"\pi"
]
\arrow[Leftrightarrow, from=VV, to=UU,shorten >=0.25cm,shorten <=0.05cm
,swap,"\pi"
]
\arrow[Leftrightarrow, from=VVV, to=UUU,shorten >=0.05cm,shorten <=0.05cm
,swap,"\pi"
]
\arrow[Leftrightarrow, from=VVVV, to=UUUU,shorten >=0.25cm,shorten <=0.05cm
,swap,"\pi"
]
\end{tikzcd}
\end{equation}
\item[(FF2)]
For $0 \leq i < k < i' \leq n$
\begin{equation}
\begin{tikzcd}[row sep = 20pt, column sep = 35pt]
	\Omega^n
	\arrow{r}[swap,name=U]{}{V^k} \arrow{d}[swap]{d^{i'}} &
	\Sigma \wr \Omega^{n-k-1} \ar{d}{d^{i'-k-1}}
&
	\Omega^n
	\arrow{dr}[swap,name=UUU]{}{V^k} \arrow{d}[swap]{d^i} &
\\
	|[alias=V]|
	\Omega^{n-1} \ar{r}{V^{k}}[swap,name=UU]{} \arrow{d}[swap]{d^i}&
	\Sigma \wr \Omega^{n-k-2}
&
	|[alias=VVV]|
	\Omega^{n-1} \ar{r}[near start, swap]{V^{k-1}}[swap,name=UUUU]{} \ar{d}[swap]{d^{i'-1}} &
	\Sigma \wr \Omega^{n-k-1} \ar{d}{d^{i'-k-1}}
\\
	|[alias=VV]|
	\Omega^{n-2} \arrow{ur}[swap]{V^{k-1}} &
&
	|[alias=VVVV]|
	\Omega^{n-2} \ar{r}[swap]{V^{k-1}} &
	\Sigma \wr \Omega^{n-k-2}
\arrow[Leftrightarrow, from=VV, to=UU,shorten >=0.05cm,shorten <=0.05cm
,swap,"\pi"
]
\arrow[Leftrightarrow, from=VVV, to=UUU,shorten >=0.05cm,shorten <=0.05cm
,swap,"\pi"
]
\end{tikzcd}
\end{equation}
\item[(DF1)]
For 
%$0 \leq j+1 < i < k +1 \leq n +1$ or 
$-1 \leq j < i \leq k \leq n$
\begin{equation}
\begin{tikzcd}[row sep = 20pt, column sep = 35pt]
	\Omega^{n}
	\arrow{dr}[swap,name=U]{}{V^{k}} \arrow{d}[swap]{s^j} &
&
	\Omega^{n}
	\arrow{dr}[swap,name=UUU]{}{V^{k}} \arrow{d}[swap]{d^{i-1}} &
\\
	|[alias=V]|
	\Omega^{n+1} \ar{r}{V^{k+1}}[swap,name=UU]{} \arrow{d}[swap]{d^i}&
	\Sigma \wr \Omega^{n-k-1}
&
	|[alias=VVV]|
	\Omega^{n-1} \ar{r}[near start, swap]{V^{k-1}}[swap,name=UUUU]{} \ar{d}[swap]{s^j} &
	\Sigma \wr \Omega^{n-k-1}
\\
	|[alias=VV]|
	\Omega^{n} \arrow{ur}[swap]{V^{k}} &
&
	|[alias=VVVV]|
	\Omega^{n} \arrow{ur}[swap]{V^{k}} &
\arrow[Leftrightarrow, from=VV, to=UU,shorten >=0.25cm,shorten <=0.05cm
,swap,"\pi"
]
\arrow[Leftrightarrow, from=VVV, to=UUU,shorten >=0.05cm,shorten <=0.05cm
,swap,"\pi"
]
\end{tikzcd}
\end{equation}
\item[(DF2)]
For $0 \leq j+1 = i \leq k \leq n$ or 
$0 \leq j = i \leq k \leq n$
\begin{equation}
\begin{tikzcd}[row sep = 20pt, column sep = 35pt]
	\Omega^n
	\arrow{dr}[swap,name=U]{}{V^k} \arrow{d}[swap]{s^j} &
&
	\Omega^n
	\arrow{dr}[swap,name=UUU]{}{V^k} \arrow[equal]{dd} &
\\
	|[alias=V]|
	\Omega^{n+1} \ar{r}{V^{k+1}}[swap,name=UU]{} \arrow{d}[swap]{d^i}&
	\Sigma \wr \Omega^{n-k-1}
&
	&
	\Sigma \wr \Omega^{n-k-1}
\\
	|[alias=VV]|
	\Omega^{n} \arrow{ur}[swap]{V^k} &
&
	|[alias=VVVV]|
	\Omega^{n} \arrow{ur}[swap]{V^k} &
\arrow[Leftrightarrow, from=VV, to=UU,shorten >=0.25cm,shorten <=0.05cm
,swap,"\pi"
]
\end{tikzcd}
\end{equation}
\item[(DF3)]
For $0\leq i < j \leq k \leq n$
\begin{equation}
\begin{tikzcd}[row sep = 20pt, column sep = 35pt]
	\Omega^n
	\arrow{dr}[swap,name=U]{}{V^k} \arrow{d}[swap]{s^j} &
&
	\Omega^n
	\arrow{dr}[swap,name=UUU]{}{V^k} \arrow{d}[swap]{d^{i}} &
\\
	|[alias=V]|
	\Omega^{n+1} \ar{r}{V^{k+1}}[swap,name=UU]{} \arrow{d}[swap]{d^i}&
	\Sigma \wr \Omega^{n-k-1}
&
	|[alias=VVV]|
	\Omega^{n-1} \ar{r}[near start, swap]{V^{k-1}}[swap,name=UUUU]{} \ar{d}[swap]{s^{j-1}} &
	\Sigma \wr \Omega^{n-k-1}
\\
	|[alias=VV]|
	\Omega^{n} \arrow{ur}[swap]{V^k} &
&
	|[alias=VVVV]|
	\Omega^{n} \arrow{ur}[swap]{V^k} &
\arrow[Leftrightarrow, from=VV, to=UU,shorten >=0.25cm,shorten <=0.05cm
,swap,"\pi"
]
\arrow[Leftrightarrow, from=VVV, to=UUU,shorten >=0.05cm,shorten <=0.05cm
,swap,"\pi"
]
\end{tikzcd}
\end{equation}
\item[(DF4)]
For $0 \leq i < k \leq j \leq n$
\begin{equation}
\begin{tikzcd}[row sep = 20pt, column sep = 35pt]
	\Omega^n
	\arrow{r}[swap,name=U]{}{V^k} \arrow{d}[swap]{s^j} &
	\Sigma \wr \Omega^{n-k-1} \ar{d}{s^{j-k-1}}
&
	\Omega^n
	\arrow{dr}[swap,name=UUU]{}{V^k} \arrow{d}[swap]{d^i} &
\\
	|[alias=V]|
	\Omega^{n+1} \ar{r}{V^{k}}[swap,name=UU]{} \arrow{d}[swap]{d^i}&
	\Sigma \wr \Omega^{n-k}
&
	|[alias=VVV]|
	\Omega^{n-1} \ar{r}[near start, swap]{V^{k-1}}[swap,name=UUUU]{} \ar{d}[swap]{s^{j-1}} &
	\Sigma \wr \Omega^{n-k-1} \ar{d}{s^{j-k-1}}
\\
	|[alias=VV]|
	\Omega^{n} \arrow{ur}[swap]{V^{k-1}} &
&
	|[alias=VVVV]|
	\Omega^{n} \ar{r}[swap]{V^{k-1}} &
	\Sigma \wr \Omega^{n-k}
\arrow[Leftrightarrow, from=VV, to=UU,shorten >=0.05cm,shorten <=0.05cm
,swap,"\pi"
]
\arrow[Leftrightarrow, from=VVV, to=UUU,shorten >=0.05cm,shorten <=0.05cm
,swap,"\pi"
]
\end{tikzcd}
\end{equation}
\end{itemize}
\end{proposition}





\subsection{Pullback functors}


Recall \cite[Prop. 2.7]{BP_geo}
that if
$\pi \colon \mathcal{E} \to \mathcal{B}$ is 
a split Grothendieck fibration, 
then so is the map of functor categories
$\mathcal{E}^{\mathcal{C}} \to \mathcal{B}^{\mathcal{C}}$
for any category $\mathcal{C}$.
Explicitly, given functors $E \colon \mathcal{C} \to \mathcal{E}$,
$B',B \colon \mathcal{C} \to \mathcal{B}$
such that $B=\pi \circ E$
and a natural transformation
$\varphi \colon B' \Rightarrow B$,
the pullback functor
$\varphi^{\**} E \colon \mathcal{C} \to \mathcal{E}$
is described on objects by
$\left(\varphi^{\**} E\right) (c) =
\left(\varphi(c)\right)^{\**} (E(c))$
and on arrows $f\colon c \to \bar{c}$
as the unique dashed arrow in the leftmost diagram below which makes that diagram commute and lifts $B'(f)$. 
\[
\begin{tikzcd}
	\varphi^{\**}E(c) \ar{r} 
	\ar[dashed]{d}[swap]{\varphi^{\**}E (f)} &
	E(c) \ar{d}{E(f)}
&&
	B'(c) \ar{d}[swap]{B'(f)} \ar{r} &
	B(c) \ar{d}{B(f)}
\\
	\varphi^{\**}E(\bar{c}) \ar{r} &
	E(\bar{c}) 
&&
	B'(\bar{c}) \ar{r} &
	B(\bar{c})
\end{tikzcd}
\]

Given a split Grothendieck fibration 
$\pi^{\**} \colon \mathcal{E} \to \mathcal{B}$,
we now define a pullback functor on weak right spans
\begin{equation}\label{WSPANPULL EQ}
\pi^{\**} \colon
\mathsf{Cat} \downarrow^r \mathcal{B} 
	\to
\mathsf{Cat} \downarrow^r \mathcal{E} 
\end{equation}
as follows.

On objects, i.e. functors $B\colon \mathcal{C} \to \mathcal{B}$, one sets 
$\pi^{\**}(\colon \mathcal{C} \to \mathcal{B})=
(\mathcal{C} \times_{\mathcal{B}} \mathcal{E}
\to \mathcal{E})
$.

On $1$-arrows, i.e. pairs 
$(f,\phi \colon B_2 \circ f \Rightarrow B_1)$
as in the diagram below
\begin{equation}
\begin{tikzcd}[row sep = tiny, column sep = 35pt]
	\mathcal{C}_1 \arrow{dr}[name=U]{B_1} \arrow{dd}[swap]{f}
\\
	& \mathcal{B}
\\
	|[alias=V]| \mathcal{C}_2 \arrow{ur}[swap]{B_2}
\arrow[Rightarrow, from=V, to=U,shorten >=0.25cm,shorten <=0.25cm
,swap,"\phi"
]
\end{tikzcd}
\end{equation}
and writing 
$\pi \colon \mathcal{C}_i \times_{\mathcal{B}} \mathcal{E}
\to \mathcal{C}_i$
and
$E_i \colon \mathcal{C}_i \times_{\mathcal{B}} \mathcal{E}
\to \mathcal{E}$
for the projections, one sets
\[
\pi^{\**} (f,\phi)=
\left(
	\left( f \pi,
	\left( \phi \pi \right)^{\**} E_1 \right),
	\left( \phi \pi \right)^{\**} E_1 \Rightarrow E_1
\right)
\]
or, put another way, 
$\pi^{\**}(f,\phi)$
is characterized as the unique choice of dashed data in the diagram
\[
\begin{tikzcd}[column sep = small, row sep = small]
	\mathcal{C}_1 \times_{\mathcal{B}} \mathcal{E} 
	\ar{rrrrr}[name=toE]{E_1} \ar[dashed]{rd} \ar{dd}
	&&&
	&&
	\mathcal{E}  \ar{dd}
\\
	&
	|[alias=DBE]|
	\mathcal{C}_2 \times_{\mathcal{B}} \mathcal{E} \ar{rrrru}[swap]{E_2}
\\
	\mathcal{C}_1 \ar{rrrrr}[name=toB]{B_1} \ar{rd} 
	&&&
	&&
	\mathcal{B} 
\\
	&
	|[alias=D]| \mathcal{C}_2 \ar{rrrru}[swap]{B_2}
\arrow[Rightarrow, from=DBE, to=toE, shorten <=0.15cm,shorten >=0.15cm,dashed
%,swap,"\pi_i"
]
	\arrow[Rightarrow, from=D, to=toB, shorten <=0.15cm,shorten >=0.15cm,swap,"\phi"]
	\arrow[from=DBE, to=D, crossing over]
\end{tikzcd}
\]
such that the side faces commute, the top natural transformation consists of pullback arrows for $\pi \colon \mathcal{E} \to \mathcal{B}$, and the total diagram commutes, in the sense that the two composite natural transformations $B_2 i \pi \Rightarrow \pi E_1$ coincide.


Lastly, on a $2$-arrow $\varphi \colon (f,\phi) \Rightarrow (f',\phi')$
one sets $\pi^{\**} \varphi (c,b,e)$ to be the unique dashed arrow in the left diagram below that lifts $\varphi(c)$.
\[
\begin{tikzcd}
	\left(\phi(c)\right)^{\**} e \ar{rr} \ar[dashed]{rd} &&
	e
&
	B_2 f(c) \ar{rr}{\phi(c)} \ar{rd}[swap]{\varphi(c)} &&
	b
\\
	& \left(\phi'(c)\right)^{\**} e \ar{ru} &
&
	& B_2 f'(c) \ar{ru}[swap]{\phi'(c)} &
\end{tikzcd}
\]
Alternatively, $\pi^{\**}\varphi$ this is the unique dashed natural transformation in the left diagram below such that the left section commutes 
(meaning that the two natural transformations between the two functors
$\mathcal{C}_1 \times_{\mathcal{B}} \mathcal{E} 
\rightrightarrows \mathcal{C}_2$ coincide) and 
the top composite natural transformation is 
$\pi^{\**} \phi$.
\begin{equation}\label{PULL2ARR EQ}
\begin{tikzcd}[column sep = small, row sep = 17pt]
	\mathcal{C}_1 \times_{\mathcal{B}} \mathcal{E} 
	\ar{rrrrr}[name=toE]{E_1} 
	\ar[bend left]{rd}[near start,swap,name=FE]{}
	\ar[bend right]{rd}[name=FFE]{} \ar{dd}
	&&&
	&&
	\mathcal{E}  \ar{dd}
&&
	\mathcal{C}_1 \times_{\mathcal{B}} \mathcal{E} 
	\ar{rrrrr}[name=toE2]{E_1} 
	\ar[bend right]{rd}{} \ar{dd}
	&&&
	&&
	\mathcal{E}  \ar{dd}
\\
	&
	|[alias=DBE]|
	\mathcal{C}_2 \times_{\mathcal{B}} \mathcal{E} \ar{rrrru}[swap]{E_2} &&&&
&&
	&
	|[alias=DBE2]|
	\mathcal{C}_2 \times_{\mathcal{B}} \mathcal{E} \ar{rrrru}[swap]{E_2} &&&&
\\
	\mathcal{C}_1 \ar{rrrrr}[name=toB]{B_1} 
	\ar[bend left]{rd}[swap,name=FF]{}
	\ar[bend right]{rd} [name=F]{}
	&&&
	&&
	\mathcal{B} 
&&
	\mathcal{C}_1 \ar{rrrrr}[name=toB2]{B_1} 
	\ar[bend right]{rd}{}
	&&&
	&&
	\mathcal{B} 
\\
	&
	|[alias=D]| \mathcal{C}_2 \ar{rrrru}[swap]{B_2} &&&&
&&
	&
	|[alias=D2]|
	\mathcal{C}_2 \ar{rrrru}[swap]{B_2} &&&&
\arrow[Rightarrow, from=DBE, to=toE, shorten <=0.15cm,shorten >=0.15cm
%,swap,"\pi_i"
]
\arrow[Rightarrow, from=DBE2, to=toE2, shorten <=0.15cm,shorten >=0.15cm
%,swap,"\pi_i"
]
\arrow[Rightarrow, from=D, to=toB, shorten <=0.15cm,shorten >=0.15cm,swap,"\phi'"]
\arrow[Rightarrow, from=D2, to=toB2, shorten <=0.15cm,shorten >=0.15cm,swap,"\phi"]
\arrow[Rightarrow, from=F, to=FF, shorten <=0cm,shorten >=0cm,swap,"\varphi"]
\arrow[Rightarrow, from=FFE, to=FE, shorten <=0cm,shorten >=0cm,swap,dashed]
\arrow[from=DBE, to=D, crossing over]
\arrow[from=DBE2, to=D2, crossing over]
\end{tikzcd}
\end{equation}

The associativity and unitality conditions of $\pi^{\**}$ are straightforward.

We will find it convenient to generalize the pullback functor
construction \eqref{WSPANPULL EQ}.
Firstly, note that the image of $\pi^{\**}$ in \eqref{WSPANPULL EQ}
lands in the $2$-subcategory
$\mathsf{Cat} \downarrow^r_{\mathcal{B}} \mathcal{E}
\subset \mathsf{Cat} \downarrow^r \mathcal{E}$ 
containing only those $1$-arrows $(f,\phi)$
such that $\phi$ consists of pullback arrows (but otherwise containig all objects and all $2$-arrows between the specified $1$-arrows).
It is then straightforward to check that if
$\rho \colon \mathcal{E} \to \mathcal{F}$ is a map of split Grothendieck fibrations over $\mathcal{B}$, then the construction in this section generalizes to a pullback functor 
\begin{equation}\label{WSPANPULL2 EQ}
\rho^{\**} \colon
\mathsf{Cat} \downarrow^r_{\mathcal{B}} \mathcal{F} 
	\to
\mathsf{Cat} \downarrow^r_{\mathcal{B}} \mathcal{E}.
\end{equation}



\begin{remark}\label{SIGMANAT REM}
Suppose that in \eqref{PULL2ARR EQ} one has that:
\begin{inparaenum}
\item[(i)] the map
$\mathcal{C}_2 \times_{\mathcal{B}} \mathcal{E}
\to \mathcal{C}_2$
is a map of split Grothendieck fibrations over a base $\mathcal{B}'$;
\item[(ii)]
that $\varphi$ consists of pullback arrows over $\mathcal{B}'$;
\item[(iii)]
and the map
$E_2 \colon \mathcal{C}_2 \times_{\mathcal{B}} \mathcal{E}
\to \mathcal{E}$
sends pullback arrows over $\mathcal{B}'$
to pullback arrows over $\mathcal{B}$.
\end{inparaenum}

Assumptions (i) and (ii) then guarantee the existence of 
a unique natural transformation
$\tilde{\varphi} \colon \tilde{f}_1
\Rightarrow \pi^{\**}f_2$
formed by pullback arrows over $\mathcal{B'}$
and such that
$\pi \tilde{\varphi} = \varphi \pi$,
and we claim that this natural transformation coincides with 
$\pi^{\**} \varphi \colon \pi^{\**}f_1 \Rightarrow \pi^{\**}f_2$.

Indeed, by the universal property of $\pi^{\**} \varphi$
it suffices to verify that the pair
$(\tilde{f}_1,\pi^{\**}\phi' \circ E_2 \tilde{\varphi})$
fulfills the universal property of the pair
$(\pi^{\**}f_1,\pi^{\**}\phi)$. 
And since (iii) guarantees that
$\pi^{\**}\phi' \circ E_2 \tilde{\varphi}$
consists of pullback arrows over $\mathcal{B}$, this follows since
\[
	\pi \left( \pi^{\**}\phi' \circ E_2 \tilde{\varphi}
	\right)
=
	\pi ( \pi^{\**}\phi') \circ \pi E_2 \tilde{\varphi}
=
	\phi' \pi \circ B_2 \pi \tilde{\varphi}
=
	\phi' \pi \circ B_2 \varphi \pi
=
	\left(\phi' \circ B_2 \varphi \right) \pi
=
	\phi \pi.
\]
\end{remark}



When $\mathcal{B}$ is cocomplete, then all limits in 
$\mathsf{Cat} \downarrow^r \mathcal{B}$
have the form described in Remark \ref{SPANLIM REM}.
However, since we will be interested in cases where 
$\mathcal{B}$ possesses only some limits,
we will refer to limits as in the previous remark as 
\textit{standard limits}.

In the next result recall that a functor 
$\pi \colon \mathcal{E} \to \mathcal{B}$
is said to \textit{reflect colimits} if any cone diagram in $\mathcal{E}$ which becomes a colimit diagram after projection to $\mathcal{B}$ was already a colimit diagram in $\mathcal{E}$
and to \textit{lift colimits} if any diagram in $\mathcal{E}$ whose projection to $\mathcal{B}$ admits a colimit in $\mathcal{B}$ also admits a compatible
colimit in $\mathcal{E}$.

Lastly, for a Grothendieck fibration 
$\mathcal{E} \to \mathcal{B}$, 
we write $\mathcal{E}_{\mathcal{B}}
\subseteq \mathcal{E}$
for the wide subcategory consisting only of pullback arrows. Note that 
$\mathcal{E}_{\mathcal{B}} \to \mathcal{B}$
is then a split Grothendieck fibration with discrete fibers.


\begin{proposition}\label{PRESSTLIM PROP}
Suppose $\mathcal{E} \to \mathcal{B}$ 
is a split Grothendieck fibration such that 
$\mathcal{E}_{\mathcal{B}} \to \mathcal{B}$ 
reflects and lifts colimits
%and $\mathcal{E}_{\mathcal{B}} \to \mathcal{E}$ preserves colimits
. Then
\begin{equation}
\pi^{\**} \colon
\mathsf{Cat} \downarrow^r \mathcal{B} 
	\to
\mathsf{Cat} \downarrow^r \mathcal{E} 
\end{equation}
preserves standard limits.
\end{proposition}


\begin{proof}
Consider a standard limit diagram in
$\mathsf{Cat} \downarrow^r \mathcal{B} $
as described in Remark \ref{SPANLIM REM}.
The key claim is the identification
$
\mathcal{C} \times_{\mathcal{B}} \mathcal{E}
=
\lim_{j \in J}
\mathcal{C}_j \times_{\mathcal{B}} \mathcal{E}
$.

A functor $\mathcal{F} \to \lim_{j \in J}
\mathcal{C}_j \times_{\mathcal{B}} \mathcal{E}$
is characterized by functors
$C_j\colon \mathcal{F} \to \mathcal{C}_j$
and 
$E_j\colon \mathcal{F} \to \mathcal{E}$
such that
$B_j C_j = \pi E_j$
and, for each arrow $k \colon j \to j'$
inducing a map 
$(f_k,\phi_k) \colon \mathcal{C}_j \to \mathcal{C}_{j'}$,
one has
$\left(\phi_k C_j\right)^{\**} E_j = E_{j'}$
and
$f_k C_j = C_{j'}$.
The $C_j$ then induce an unique compatible functor
$C \colon \mathcal{F} \to \mathcal{C} = 
\lim_{j \in J}
\mathcal{C}_j$ and,
writing 
$E = \colim_{j \in J} E_j$ for the colimit in 
$\mathsf{Fun}(\mathcal{F},\mathcal{E}_{\mathcal{B}})$
lifting
the colimit
$BC = \colim_{j \in J} B_jC_j$
in 
$\mathsf{Fun}(\mathcal{F},\mathcal{B})$.

By combining the two functors $C$ and $E$, one then obtains the unique functor  
$(C,E) \colon \mathcal{F} \to 
\mathcal{C} \times_{\mathcal{B}} \mathcal{E}$
compatible with the given functors
$(C_j,E_j) \colon \mathcal{F} \to 
\mathcal{C}_j \times_{\mathcal{B}} \mathcal{E}$.
To see this, note first that compatibility follows since the maps 
$E_j \to E$ are pullbacks over $\phi_j C$ so that
$E_j = \left( \phi_j C \right)^{\**} E$
(due to the colimit being taken in $\mathcal{E}_{\mathcal{B}}$). 
On the other hand, uniqueness follows since the colimit $E$ described above is \textit{strictly} unique. Indeed, if $E'$ were another such colimit, the canonical isomorphism $E'\simeq E$ would project to the identity 
morphism for $BC$, and since pullbacks over an identity are again identities, this would mean $E'=E$.
\end{proof}


\begin{corollary}\label{COLORCOR COR}
Let $\mathfrak{C}$ be a set of colors and define
$\Omega^n_{\mathfrak{C}}$ for $n \geq -1$ via the pullback
\[
\begin{tikzcd}
	\Omega^n_{\mathfrak{C}} \ar{r}{\boldsymbol{E}} \ar{d} & \mathsf{F} \wr \mathfrak{C} \ar{d}
\\
	\Omega^n \ar{r}{\boldsymbol{E}} & \mathsf{F} 
\end{tikzcd}
\]
Then one has functors 
$d^i \colon \Omega^n_{\mathfrak{C}} \to \Omega^{n-1}_{\mathfrak{C}}$ for $0 \leq i \leq n$,
$s^j \colon \Omega^n_{\mathfrak{C}} \to \Omega^{n+1}_{\mathfrak{C}}$ for $-1 \leq j \leq n$, 
and
$\boldsymbol{V}^k \colon \Omega^n_{\mathfrak{C}} \to 
\Sigma \wr \Omega^{n-k-1}_{\mathfrak{C}}$
for $-1 \leq k \leq n$
and natural isomorphisms
$\pi_{i,k} \colon \boldsymbol{V}^{k-1} d^i
\xrightarrow{\simeq} \boldsymbol{V}^{k}$
for $0\leq i < k \leq n$
\begin{equation}
\begin{tikzcd}[row sep = tiny, column sep = 35pt]
	\Omega^n_{\mathfrak{C}}
	\arrow{dr}[swap,name=U]{}{V^k} \arrow{dd}[swap]{d^i} \\
	& \Sigma \wr \Omega^{n-k-1}_{\mathfrak{C}}
\\
	|[alias=V]|
	\Omega^{n-1}_{\mathfrak{C}} \arrow{ur}[swap]{V^{k-1}} \arrow[Leftrightarrow, from=V, to=U,shorten >=0.15cm,shorten <=0.15cm
,swap,"\pi_{i,k}"
]
\end{tikzcd}
\end{equation}
satisfying the analogues of 
\eqref{VKGEN EQ} and
Propositions \ref{CATFDIAG PROP} and \ref{CATFDIAG2 PROP}.

Lastly, the natural transformations $\pi_{i,k}$
consist of pullback arrows over $\Sigma$.
\end{corollary}


\begin{proof}
	The result follows by applying the 
	$2$-categorical pullback functor
	\eqref{WSPANPULL EQ} to the Grothendieck fibration
	$\pi_{\mathfrak{C}}^{\**} \colon \mathsf{Fin} \wr \mathfrak{C} \to \mathsf{Fin}$.
	Note that the existence of pullback squares
\begin{equation}\label{AMALGPULL EQ}
\begin{tikzcd}
	\Sigma \wr \mathsf{F} \wr \mathfrak{C} \ar{r}{\amalg} \ar{d} &
	\mathsf{F} \wr \mathfrak{C} \ar{d}
\\
	\Sigma \wr \mathsf{F} \ar{r}{\amalg} &
	\mathsf{F}
\end{tikzcd}
\end{equation}
shows that there is a canonical identification of $2$-functors
$\pi_{\mathfrak{C}}^{\**} \Sigma \wr (-) \simeq
\Sigma \wr \pi_{\mathfrak{C}}^{\**} (-)$,
which is readily seen to also induce natural identifications
$\pi_{\mathfrak{C}}^{\**} \delta^i \simeq 
\delta^i \pi_{\mathfrak{C}}^{\**}$ and
$\pi_{\mathfrak{C}}^{\**} \sigma^i \simeq
\sigma^i \pi_{\mathfrak{C}}^{\**}$.

All claims follow directly from $2$-functoriality of 
$\pi_{\mathfrak{C}}^{\**}$ except
for the additional ``pullback claim'' in Proposition \ref{CATFDIAG PROP}, which follows by applying Proposition \ref{PRESSTLIM PROP}, 
and the claim that $\pi_{i,k}$ consists of pullback arrows over $\Sigma$, 
which follows from Remark \ref{SIGMANAT REM}
(with condition (iii) therein following 
since the upper arrow in 
\eqref{AMALGPULL EQ}
maps $\Sigma$-pullback arrows to $\mathsf{F}$-pullback arrows.
\end{proof}




\begin{remark}
Suppose that in the pullback square below one has that:
\begin{inparaenum}
	\item[(i)] $\rho$ is a map of split Grothendieck fibrations over a base $\mathcal{B}$;
	\item[(ii)] $\mathcal{E}$ is a a split Grothendieck fibration over a base $\mathcal{A}$;
	\item[(iii)] $F$ sends pullback arrows over $\mathcal{A}$ to pullback arrows over $\mathcal{B}$.
\end{inparaenum}
\begin{equation}\label{PULLGROTH EQ}
\begin{tikzcd}
	\mathcal{E} \times_{\mathcal{D}} \mathcal{C} \ar{r} \ar{d} &
	\mathcal{C} \ar{d}{\rho}
\\
	\mathcal{E} \ar{r}[swap]{F} &
	\mathcal{D}
\end{tikzcd}
\end{equation}
One can then show that the map 
$\mathcal{E} \times_{\mathcal{D}} \mathcal{C} \to \mathcal{E}$
is a map of split fibrations over $\mathcal{A}$.
Indeed, it is straightforward to verify that 
one can define the pullback arrows in 
$\mathcal{E} \times_{\mathcal{D}} \mathcal{C}$
to be those arrows whose coordinate components are all pullback arrows.
Alternatively, by writing
$\mathcal{E} \simeq \left( \mathcal{A}^{op} \ltimes \mathcal{E}_{\bullet}^{op} \right)^{op}$
as the Grothendieck construction of a functor
$\mathcal{A}^{op} \xrightarrow{\mathcal{E}_{\bullet}} \mathsf{Cat}$,
condition (iii) guarantees that $\mathcal{E}_{\bullet}$
factors as in the left triangle below
\[
\begin{tikzcd}
	\mathcal{A}^{op} \ar{r} \ar[bend right=10]{rrd}[swap]{\mathcal{E}_{\bullet}} &
	\mathsf{Cat} \downarrow^r_{\mathcal{B}} \mathcal{D}
	\ar{r}{\rho^{\**}} \ar{rd}[swap,near start]{\mathsf{fgt}}[name=D]{} &
	|[alias=C]|
	\mathsf{Cat} \downarrow^r_{\mathcal{B}} \mathcal{C}
	\ar{d}{\mathsf{fgt}}
\\
	& &
	\mathsf{Cat}
\arrow[Rightarrow, from=C, to=D,shorten >=0.05cm,shorten <=0.05cm]
\end{tikzcd}
\]
and $\mathcal{E} \times_{\mathcal{D}} \mathcal{C}$
is then the (contravariant) Grothendieck construction for the top right composite.
\end{remark}



\begin{lemma}
Suppose that in \eqref{PULLGROTH EQ}
one has that conditions (i),(ii),(iii)
hold and that in addition:
\begin{inparaenum}
\item[(iv)] the fibers of $\mathcal{E} \to \mathcal{A}$ are groupoids (equivalently, any arrow mapping to an identity is an isomorphism);
\item[(v)]
for all $e \in \mathcal{E}$
the induced composite functors
\begin{equation}\label{COMPISO EQ}
e \downarrow_{\mathsf{r}} \mathcal{E}
	\hookrightarrow
e \downarrow \mathcal{E}
	\to 
F(e) \downarrow \mathcal{D}
	\to
F(e) \downarrow_{\mathsf{r}} \mathcal{D}
\end{equation}
are isomorphisms.
\end{inparaenum}

Then for all 
$e \in \mathcal{E}$
the induced composite functors
\begin{equation}\label{COMPISO2 EQ}
e \downarrow_{\mathsf{r}}
\mathcal{E} \times_{\mathcal{D}} \mathcal{C}
	\hookrightarrow
e \downarrow
\mathcal{E} \times_{\mathcal{D}} \mathcal{C}
	\to 
F(e) \downarrow \mathcal{C}
	\to
F(e) \downarrow_{\mathsf{r}} \mathcal{C}
\end{equation}
are again isomorphisms.
\end{lemma}


\begin{proof}
Let us write $G$ (resp. $\bar{G}$) for the composite functors in \eqref{COMPISO EQ}
(resp. \eqref{COMPISO2 EQ}), as well as 
$\pi_{\mathcal{B}} \colon \mathcal{D} \to \mathcal{B}$
for the projection.

One then has 
$G \left(e \xrightarrow{f} e' \right)=
\left(
F(e) \to \left( \pi_{\mathcal{B}} F(f)\right)^{\**} F(e')
\right)
$
and 
\[
\bar{G}\left((e',c'),e \xrightarrow{f} e' \right) =
\left( \left( \pi_{\mathcal{B}} F(f)\right)^{\**} c', F(e) \to \left( \pi_{\mathcal{B}} F(f)\right)^{\**} F(e')\right)
\]
Assumption (iv) then implies that the arrows
$f$ in $\mathcal{E}$ are invertible
and thus so are the arrows 
$\pi_{\mathcal{B}} F(f)$ in $\mathcal{B}$,
while assumption (v) says that $G$ is invertible.
It is then straightforward to check that 
$\bar{G}^{-1}$ is given on objects by 
\[
\bar{G}^{-1}
\left(c, F(e) \xrightarrow{g} \rho(c) \right)
=
\left(
	\left(
	\left(
	\pi_{\mathcal{B}} F G^{-1}(g)
	\right)^{-1}
\right)^{\**} c, e \xrightarrow{G^{-1}(g)} \bullet
\right)
\]
\end{proof}



\begin{corollary}
Suppose that the following pullback square satisfies conditions 
(i),(ii),(iii),(iv),(v)
listed previously.
\begin{equation}\label{PULLGROTH EQ}
\begin{tikzcd}
	\mathcal{E} \times_{\mathcal{D}} \mathcal{C} \ar{r}{\bar{F}} \ar{d}[swap]{\bar{\rho}} &
	\mathcal{C} \ar{d}{\rho}
\\
	\mathcal{E} \ar{r}[swap]{F} &
	\mathcal{D}
\end{tikzcd}
\end{equation}
Then, for any functor 
$H \colon \mathcal{C} \to \mathcal{V}$ with $\mathcal{V}$ a complete category one has a natural isomorphism
\[
	\left(\mathsf{Ran}_{\rho} H\right) \circ F
\xrightarrow{\simeq}
	\mathsf{Ran}_{\bar{\rho}} \left( H \circ \bar{F} \right)
\]
\end{corollary}


\begin{proof}
	It suffices to show that, for each $e \in \mathcal{E}$ the induced functor
	$e \downarrow \mathcal{E} \to F(e) \downarrow \mathcal{D}$ is initial.
	And since initial functors satisfy the hypersaturation cancellation property, it suffices to show that the composite
	$e \downarrow_{\mathsf{r}} \mathcal{E} \to F(e) \downarrow \mathcal{D}$
	is initial.
	
	Our assumptions then imply that this functor is naturally isomorphic to the composite
	$e \downarrow_{\mathsf{r}} \mathcal{E} \to 
	F(e) \downarrow \mathcal{D} \to 
	F(e) \downarrow_{\mathsf{r}} \mathcal{D} \to
	F(e) \downarrow \mathcal{D}$, which is initial since 
	\eqref{COMPISO2 EQ} is an isomorphism and
	$F(e) \downarrow_{\mathsf{r}} \mathcal{D} \to
	F(e) \downarrow \mathcal{D}$
	is initial. Noting that natural isomorphisms preserve initiality of functors finishes the proof. 
\end{proof}




\subsection{Vertex data}


Much like the edge functors
$\boldsymbol{E} \colon \Omega^n_{\mathfrak{C}} \to \mathsf{F} \wr \mathfrak{C}$
can be regarded as encoding objects of
$\mathsf{Cat} \downarrow^r \mathsf{F} \wr \mathfrak{C}$,
we will find it useful to regard the vertex funtors
$\boldsymbol{V}^n \colon \Omega^n \to \Sigma \wr \Sigma_{\mathfrak{C}}$
as encoding objects of 
$\mathsf{Cat} \downarrow^r_{\Sigma} \Sigma \wr \Sigma_{\mathfrak{C}}$.
Much as before, 
one has an endofunctor $\Sigma \wr (-)$ on
$\mathsf{Cat} \downarrow^r_{\Sigma} \Sigma \wr \Sigma_{\mathfrak{C}}$
which sends an object
$\mathcal{C} \to \Sigma \wr \Sigma_{\mathfrak{C}}$
to the composite
$\Sigma \wr \mathcal{C} \to 
\Sigma^{\wr 2} \wr \Sigma_{\mathfrak{C}}
\xrightarrow{\sigma^0}
\Sigma \wr \Sigma_{\mathfrak{C}}$.


It then follows by \eqref{VKGEN EQ} and Proposition \ref{CATFDIAG PROP}(i) that the functors
$\boldsymbol{V}^k \colon \Omega^n_{\mathfrak{C}} \to \Sigma \wr \Omega^{n-k-1}_{\mathfrak{C}}$
for $-1\leq k \leq n$
and $s^j \colon \Omega^n \to \Omega^{n+1}$
for $-1 \leq j \leq n$
can be regarded as simple $1$-arrows of 
$\mathsf{Cat} \downarrow^r_{\Sigma} \Sigma \wr \Sigma_{\mathfrak{C}}$,
while again by Proposition \ref{CATFDIAG PROP}(i)
the pair $(d^i,\pi_{i,n})$ for $0 \leq i < n$
\begin{equation}
\begin{tikzcd}[row sep = tiny, column sep = 35pt]
	\Omega^n_{\mathfrak{C}}
	\arrow{dr}[swap,name=U]{}{V^n} \arrow{dd}[swap]{d^i} \\
	& \Sigma \wr \Sigma_{\mathfrak{C}}
\\
	|[alias=V]|
	\Omega^{n-1}_{\mathfrak{C}} \arrow{ur}[swap]{V^{n-1}} \arrow[Leftrightarrow, from=V, to=U,shorten >=0.15cm,shorten <=0.15cm
,swap,"\pi_{i,n}"
]
\end{tikzcd}
\end{equation}
can be regarded as a (non-simple) $1$-arrow of 
$\mathsf{Cat} \downarrow^r_{\Sigma} \Sigma \wr \Sigma_{\mathfrak{C}}$.

It now follows from 
Proposition \ref{CATFDIAG2 PROP} (IT1) and (IT2)
that the diagrams in Proposition \ref{CATFDIAG PROP}(i)
for $0 \leq i < k \leq n$ and $-1\leq j\leq k \leq n$
and in Proposition \ref{CATFDIAG PROP}(ii)
for $-1\leq k < i < n$ (note that this differs from the original condition)
and $-1\leq k \leq j \leq n$
extend to diagrams in 
$\mathsf{Cat} \downarrow^r_{\Sigma} \Sigma \wr \Sigma_{\mathfrak{C}}$,
which then satisfy all applicable relations in
Proposition \ref{CATFDIAG2 PROP}. 



\begin{remark}
It is also possible to discuss edge and vertex data simultaneously. Namely the diagram
\begin{equation}
\begin{tikzcd}[row sep = 20pt, column sep = 35pt]
	\Omega^n_{\mathfrak{C}}
	\arrow{r} \arrow{rd}[name=U]{} &
	|[alias=V]|
	\Sigma \wr \Sigma_{\mathfrak{C}} \ar{d}
\\
	& \mathsf{F} \wr \mathfrak{C}
\arrow[Rightarrow, from=V, to=U,shorten >=0.05cm,shorten <=0.05cm
]
\end{tikzcd}
\end{equation}
can be regarded as extending the category 
$\Omega^n_{\mathfrak{C}}$
to an object of the iterated $2$-overcategory
$\left(\mathsf{Cat} \downarrow^r \mathsf{F} \wr \mathfrak{C} \right) \downarrow^r (\Sigma \wr \Sigma_{\mathfrak{C}} \to \mathsf{F} \wr \mathfrak{C})$.
One can then check that the operators
$d^i \colon \Omega^n_{\mathfrak{C}} \to \Omega^{n-1}_{\mathfrak{C}}$ for $0\leq i <n$,
$s^j \colon \Omega^n_{\mathfrak{C}} \to \Omega^{n+1}_{\mathfrak{C}}$ for $-1 \leq j \leq n$
and
$\boldsymbol{V}^k \colon \Omega^n_{\mathfrak{C}} \to \Sigma \wr \Omega^{n-k-1}_{\mathfrak{C}}$ for $-1 \leq k \leq n$ also extend to this iterated $2$-overcategory.

However, our purposes can be achieved by treating edge and vertex data separately,
and we hence will not require this more complex perspective.
\end{remark}




\subsection{Iterated $2$-overcategories}


We will find it useful to discuss iterations of the 
$2$-overcategory construction described above.
Given a $1$-arrow in $\mathsf{Cat}$ (i.e. a functor) $\mathcal{B} \to \mathsf{F}$, we write
$\mathsf{Cat} \downarrow^r (\mathcal{B} \to \mathsf{F})$
as an abbreviation for the iterated $2$-overcategory
$\left(\mathsf{Cat} \downarrow^r \mathsf{F} \right) \downarrow^r (\mathcal{B} \to \mathsf{F}) $.
Unpacking definitions, the objects of 
$\mathsf{Cat} \downarrow^r (\mathcal{B} \to \mathsf{F})$
are quadruples $(\mathcal{C}, B, F, \gamma)$ as below,
\begin{equation}
	\begin{tikzcd}[row sep = tiny, column sep = 35pt]
		&
		|[alias=V]|
		\mathcal{B} \arrow{dd} 
	\\
		\mathcal{C} \arrow{ur}{B} 
		\arrow{dr}[swap,name=U]{F} & 
	\\
		&
		\mathsf{F}
	\arrow[Rightarrow, from=V, to=U,shorten >=0.25cm,shorten <=0.25cm
	,"\gamma"
	]
	\end{tikzcd}
\end{equation}
a $1$-arrow from 
$(\mathcal{C}, B, F, \gamma)$ to 
$(\mathcal{C}', B', F', \gamma')$
is a triple $(f,\varphi, \epsilon)$ as below such that then composite natural transformations in the two diagrams coincide,
\begin{equation}
\begin{tikzcd}[column sep = 50pt,row sep = 50pt]
	|[alias=VV]|
	\mathcal{C}' \ar{r}{B'} &
	|[alias=V]|
	\mathcal{B} \ar{d} &
&
	|[alias=VVVV]|
	\mathcal{C}' \ar{r}{B'} \ar{rd}[swap,near end]{F'}[name=UUU]{} &
	|[alias=VVV]|
	\mathcal{B} \ar{d} &
\\
	\mathcal{C} \ar{r}[swap]{F}[swap,name=U]{} \ar{u}{f} \ar{ru}[near start,swap]{B}[name=UU]{}&
	\mathsf{F} &
&
	\mathcal{C} \ar{r}[swap]{F}[name=UUUU]{} \ar{u}{f} &
	\mathsf{F} &
	\arrow[Rightarrow, from=V, to=U, shorten >=0.25cm,shorten <=0.25cm
	,"\gamma"
	]
	\arrow[Rightarrow, from=VV, to=UU, shorten >=0.15cm,shorten <=0.15cm,
	swap,"\varphi"
	]
	\arrow[Rightarrow, from=VVV, to=UUU, shorten >=0.15cm,shorten <=0.15cm
	,"\gamma'"
	]
	\arrow[Rightarrow, from=VVVV, to=UUUU, shorten >=0.25cm,shorten <=0.25cm,
	swap
	,"\epsilon"
	]
\end{tikzcd}
\end{equation}
and a natural transformation from  
$(f,\varphi, \epsilon)$ to
$(f',\varphi', \epsilon)'$
is a natural transformation 
$\rho \colon f \to f'$ such that
$\varphi' \circ \rho B' = \varphi$ and
$\epsilon' \circ \rho F' = \epsilon$









\section{Monad on equivariant spans}

In this section, we explain how to adapt the monad in the previous section when in the presence of a $G$-action on the color set $\mathfrak{C}$, the key obstacle being that of adapting the vertex functors
$\boldsymbol{V}^k \colon
\Omega_{\mathfrak{C}}^n \to 
\Sigma \wr \Omega_{\mathfrak{C}}^{n-k-1}$.

Firstly, recall that given a category $\mathcal{C}$ with a $G$-action, 
i.e. an object of $\mathsf{Cat}^G$, one can form the category
$G \ltimes \mathcal{C}$, which is obtained from $\mathcal{C}$ by formally adding ``action arrows'' 
$c \to g c$.
Moreover, this construction is readily seen to define a 
$2$-functor
$G \ltimes (-) \colon
\mathsf{Cat}^G \to \mathsf{Cat}$.

The $G \ltimes (-)$ construction interacts with the $\Sigma \wr (-)$ construction as follows. There is a natural transformation
\[
\begin{tikzcd}
	\mathsf{Cat}^G 
	\ar[bend left]{r}{G \ltimes \Sigma \wr (-)}[swap,name=T]{}
	\ar[bend right]{r}[swap]{\Sigma \wr G \ltimes (-)}[name=D]{} &
	\mathsf{Cat}
\arrow[Rightarrow, from=T, to=D,shorten >=0.05cm,shorten <=0.05cm,
"\rho"
]
\end{tikzcd}
\]
whose constituent functors
$G \ltimes \Sigma \wr \mathcal{C}
\xrightarrow{\rho_C}
\Sigma \wr G \ltimes \mathcal{C}$
are the identity on objects and the arrows of $\Sigma \wr \mathcal{C}$
and send an action arrow
$(c_i) \xrightarrow{g} (g c_i)$ to the diagonal action arrow
$(c_i \xrightarrow{g} g c_i)$.
It can be readily checked that $\rho$ is also compatible with natural transformations, in the sense that for 
$\phi$ a natural transformation in 
$\mathsf{Cat}^G$, it is
$\rho (G \ltimes \Sigma \wr \phi) = 
(\Sigma \wr G \ltimes \phi) \rho$
(alternatively, and regarding a $2$-category as a category enriched over categories, this says that $\rho$ is an enriched natural transformation).

Moreover, $\rho$ is compatible with the natural transformations
$\sigma^0\colon \Sigma \wr \Sigma \wr (-) \Rightarrow \Sigma \wr (-)$
and 
$\delta^0 \colon id \Rightarrow \Sigma \wr (-)$
in the sense that the following squares commute
\begin{equation}\label{GSIGMACOMP EQ}
\begin{tikzcd}
	G \ltimes \Sigma \wr \Sigma \wr \mathcal{C} \ar{r}{\rho} \ar{d}[swap]{\sigma^0} &
	\Sigma \wr G \ltimes  \Sigma \wr \mathcal{C} \ar{r}{\rho} &
	\Sigma \wr \Sigma \wr G \ltimes \mathcal{C} \ar{d}{\sigma^0}
&
	G \ltimes \mathcal{C} \ar[equal]{r} \ar{d}[swap]{\delta^0} &
	G \ltimes \mathcal{C} \ar{d}{\delta^0}
\\
	G \ltimes \Sigma \wr \mathcal{C} \ar{rr}{\rho} &&
	\Sigma \wr G \ltimes \mathcal{C}
&
	G \ltimes \Sigma \wr \mathcal{C} \ar{r}{\rho} &
	\Sigma \wr G \ltimes \mathcal{C}
\end{tikzcd}
\end{equation}


When the color set $\mathfrak{C}$ is a $G$-set, the categories 
$\Omega^n_{\mathfrak{C}}$ and operators
$d^i \colon \Omega^n_{\mathfrak{C}} \to \Omega^{n-1}_{\mathfrak{C}}$,
$s^j \colon \Omega^n_{\mathfrak{C}} \to \Omega^{n-1}_{\mathfrak{C}}$,
$\boldsymbol{V}^k \colon
\Omega_{\mathfrak{C}}^n \to 
\Sigma \wr \Omega_{\mathfrak{C}}^{n-k-1}$
all naturally lie in $\mathsf{Cat}^G$.
We then again write
$d^i \colon G \ltimes \Omega^n_{\mathfrak{C}} \to G \ltimes \Omega^{n-1}_{\mathfrak{C}}$,
$s^j \colon G \ltimes \Omega^n_{\mathfrak{C}} \to G \ltimes \Omega^{n-1}_{\mathfrak{C}}$ for the induced functors,
and 
$\boldsymbol{V}^k \colon
G \ltimes \Omega_{\mathfrak{C}}^n \to 
\Sigma \wr G \ltimes  \Omega_{\mathfrak{C}}^{n-k-1}$
for the induced composite
\[
G \ltimes \Omega_{\mathfrak{C}}^n 
\xrightarrow{G \ltimes \boldsymbol{V}^k } 
G \ltimes \Sigma \wr \Omega_{\mathfrak{C}}^{n-k-1}
\xrightarrow{\rho}
\Sigma \wr G \ltimes  \Omega_{\mathfrak{C}}^{n-k-1}
\]
Lastly, for $0 \leq i <k$ we again write 
$\pi_i \colon \boldsymbol{V}^{k-1} d^i
\Rightarrow 
\boldsymbol{V}^{k}$
for the natural transformation obtained by whiskering with $\rho$.


\begin{proposition}\label{ANOAN PROP}
The analogues of the identities in \eqref{VKGEN EQ} and 
Proposition \ref{CATFDIAG2 PROP}
all hold for the categories $G \ltimes \Omega^n_{\mathfrak{C}}$.
\end{proposition}


\begin{proof}
Most claims follow directly from the original versions 
by whiskering with $\rho$, with the only exceptions being the analogues of \eqref{VKGEN EQ} and (IT1),(IT2) which also require the use of
\eqref{GSIGMACOMP EQ}.

All arguments are similar, and we thus present only the argument for the analogue (IT2), which states that the composite natural transformation in the following diagram matches $\pi_i$.
\[
\begin{tikzcd}[column sep = 5pt]
	G \ltimes \Omega^n_{\mathfrak{C}}
		\ar{r} \ar{d} &
	G \ltimes \Sigma \wr \Omega^{n-k-1}_{\mathfrak{C}}
		\ar{r} \ar{d} &
	\Sigma \wr G \ltimes \Omega^{n-k-1}_{\mathfrak{C}}
		\ar{r}[name=U,swap]{} \ar{d} &
	\Sigma \wr G \ltimes \Sigma \wr \Omega^{n-k-l-2}_{\mathfrak{C}}		
		\ar{r} &
	\Sigma \wr \Sigma \wr G \ltimes \Omega^{n-k-l-2}_{\mathfrak{C}}
		\ar{r} &
	\Sigma \wr G \ltimes \Omega^{n-k-l-2}_{\mathfrak{C}}
\\
	G \ltimes \Omega^{n-1}_{\mathfrak{C}}
		\ar{r} &
	G \ltimes \Sigma \wr \Omega^{n-k-2}_{\mathfrak{C}}
		\ar{r} &
	|[alias=V]|
	\Sigma \wr G \ltimes \Omega^{n-k-2}_{\mathfrak{C}}
		\ar{ru}
\arrow[Leftrightarrow, from=V, to=U,shorten >=0.05cm,shorten <=0.05cm]
\end{tikzcd}
\]
By naturality of $\rho$ this matches the composite natural transformation in the diagram
\[
\begin{tikzcd}[column sep = 5pt]
	G \ltimes \Omega^n_{\mathfrak{C}}
		\ar{r} \ar{d} &
	G \ltimes \Sigma \wr \Omega^{n-k-1}_{\mathfrak{C}}
		\ar{r}[name=U,swap]{} \ar{d} &
	G \ltimes \Sigma \wr \Sigma \wr \Omega^{n-k-l-2}_{\mathfrak{C}}		
		\ar{r} &
	\Sigma \wr G \ltimes \Sigma \wr \Omega^{n-k-l-2}_{\mathfrak{C}}
		\ar{r} &
	\Sigma \wr \Sigma \wr G \ltimes \Omega^{n-k-l-2}_{\mathfrak{C}}
		\ar{r} &
	\Sigma \wr G \ltimes \Omega^{n-k-l-2}_{\mathfrak{C}}
\\
	G \ltimes \Omega^{n-1}_{\mathfrak{C}}
		\ar{r} &
	|[alias=V]|
	G \ltimes \Sigma \wr \Omega^{n-k-2}_{\mathfrak{C}}
		\ar{ru}
\arrow[Leftrightarrow, from=V, to=U,shorten >=0.05cm,shorten <=0.05cm]
\end{tikzcd}
\]
and by \eqref{GSIGMACOMP EQ} this further matches the composite in the diagram
\[
\begin{tikzcd}[column sep = 5pt]
	G \ltimes \Omega^n_{\mathfrak{C}}
		\ar{r} \ar{d} &
	G \ltimes \Sigma \wr \Omega^{n-k-1}_{\mathfrak{C}}
		\ar{r}[name=U,swap]{} \ar{d} &
	G \ltimes \Sigma \wr \Sigma \wr \Omega^{n-k-l-2}_{\mathfrak{C}}		
		\ar{r} &
	G \ltimes \Sigma \wr \Omega^{n-k-l-2}_{\mathfrak{C}}
		\ar{r} &
	\Sigma \wr G \ltimes \Omega^{n-k-l-2}_{\mathfrak{C}}
\\
	G \ltimes \Omega^{n-1}_{\mathfrak{C}}
		\ar{r} &
	|[alias=V]|
	G \ltimes \Sigma \wr \Omega^{n-k-2}_{\mathfrak{C}}
		\ar{ru}
\arrow[Leftrightarrow, from=V, to=U,shorten >=0.05cm,shorten <=0.05cm]
\end{tikzcd}
\]
which indeed matches $\pi_i$ by the original claim in (IT2).
\end{proof}



For any $G$-set $\mathfrak{C}$ we write
$\mathsf{WSpan}_{G}^r(G \ltimes \Sigma_{\mathfrak{C}}, \mathcal{V}^{op})
\subseteq
\mathsf{WSpan}^r(G \ltimes \Sigma_{\mathfrak{C}}, \mathcal{V}^{op})$
for the full $2$-subcategory with objects those spans that can be written as
$G \ltimes \Sigma_{\mathfrak{C}}
\leftarrow G \times A \to 
\mathcal{V}^{op}$, where we assume that the leftmost map is induced by a map in $\mathsf{Cat}^G$. Likewise, we assume that arrows are induced by maps in $\mathsf{Cat}^G$.


It follows from \eqref{OMEGAADEF EQ}
that whenever $A \to \Sigma_{\mathfrak{C}}$
is a map in $\mathsf{Cat}^G$
then $\Omega^0_{\mathfrak{C}} \wr A$
is again in $\mathsf{Cat}^G$.
In fact, a little more is true, with \eqref{WRA2FUN EQ}
generalizing to a $G$-equivariant $2$-functor
\begin{equation}\label{WRAG2FUN EQ}
\mathsf{Cat}^G \downarrow^r_{\Sigma} \Sigma \wr \Sigma_{\mathfrak{C}}
	\xrightarrow{(-)\wr A}
\mathsf{Cat}^G \downarrow^r_{\Sigma} \Sigma \wr A.
\end{equation}
Moreover, all the conclusions of 
Proposition \ref{ASSOCIDS PROP} still hold
(indeed, it suffices to check that the first identification respects $G$-actions).

Lastly, note that one has a diagram of pullback squares
(so that the total square is again a pullback square)
\begin{equation}
\begin{tikzcd}
	G \ltimes \Omega^0_{\mathfrak{C}} \wr A \ar{r} \ar{d} &
	G \ltimes \Sigma \wr A \ar{d} \ar{r}
&
	\Sigma \wr G \ltimes A \ar{d}
\\
	G \ltimes \Omega^0_{\mathfrak{C}} \ar{r}
&
	G \ltimes \Sigma \wr \Sigma_{\mathfrak{C}} \ar{r}
&
	\Sigma \wr G \ltimes \Sigma_{\mathfrak{C}}
\end{tikzcd}
\end{equation}



We now define a monad $N_{\mathfrak{C}}$ on 
$\mathsf{WSpan}_{G}^r(G \ltimes \Sigma_{\mathfrak{C}}, \mathcal{V}^{op})$
by setting 
$N_{\mathfrak{C}}
(G \ltimes \Sigma_{\mathfrak{C}}
\leftarrow G \times A \to 
\mathcal{V}^{op})
$
to be the composite span in the diagram
\[
\begin{tikzcd}
	G \ltimes \Omega^0_{\mathfrak{C}} \wr A \ar{r}{\boldsymbol{V}^0} \ar{d} &
	\Sigma \wr G \ltimes A  \ar{d} \ar{r} &
	\Sigma \wr \mathcal{V}^{op} \ar{r}{\otimes} &
	\mathcal{V}^{op}
\\
	G \ltimes \Omega^0_{\mathfrak{C}} \ar{r}{\boldsymbol{V}^0} \ar{d} &
	\Sigma \wr G \times \Sigma_{\mathfrak{C}} 
\\
	G \ltimes \Sigma_{\mathfrak{C}}
\end{tikzcd}
\]
while multiplication and unit are defined in analogy with \eqref{NMONMULT EQ} and \eqref{NMONID EQ} via
\begin{equation}\label{NMONMULTG EQ}
\begin{tikzcd}
	G \ltimes \Sigma_{\mathfrak{C}} \ar[equal]{d}&
	G \ltimes \Omega^1_{\mathfrak{C}} \wr A \ar{l} \ar{r} \ar{d}[swap]{d^0}&
	\Sigma \wr G \ltimes \Omega^0_{\mathfrak{C}} \wr A \ar{r} &
	|[alias=U]|
	\Sigma^{\wr 2} \wr G \ltimes A \ar{r} \ar{d}[swap]{\sigma^0} &
	\Sigma^{\wr 2} \wr \mathcal{V}^{op} \ar{r}{\otimes} \ar{d}[swap]{\sigma^0} &
	\Sigma \wr \mathcal{V}^{op} \ar{r}{\otimes} &
	|[alias=UU]|
	\mathcal{V}^{op} \ar[equal]{d}
\\
	G \ltimes \Sigma_{\mathfrak{C}} &
	|[alias=V]|
	G \ltimes \Omega^0_{\mathfrak{C}} \wr A \ar{l} \ar{rr} & &
	\Sigma \wr G \ltimes A \ar{r} &
	|[alias=VV]|
	\Sigma \wr \mathcal{V}^{op} \ar{rr}{\otimes} & &
	\mathcal{V}^{op}
\arrow[Leftrightarrow, from=V, to=U,shorten >=0.15cm,shorten <=0.15cm
,swap,"\pi"
]
\arrow[Leftrightarrow, from=VV, to=UU,shorten >=0.15cm,shorten <=0.15cm
]
\end{tikzcd}
\end{equation}
and 
\begin{equation}\label{NMONIDG EQ}
\begin{tikzcd}
	G \ltimes \Sigma_{\mathfrak{C}} \ar[equal]{d} & 
	G \ltimes A \ar{d}[swap]{s^{-1}} \ar{l} \ar[equal]{r} &
	G \ltimes A \ar{d}[swap]{\delta^0} \ar{r} &
	\mathcal{V}^{op} \ar{d}[swap]{\delta^0} \ar[equal]{r} &
	\mathcal{V}^{op} \ar[equal]{d}
\\
	G \ltimes \Sigma_{\mathfrak{C}} &
	G \ltimes \Omega^0_{\mathfrak{C}} \wr A \ar{l} \ar{r} &
	\Sigma \wr G \ltimes A \ar{r} &
	\Sigma \wr \mathcal{V}^{op} \ar{r}{\otimes} &
	\mathcal{V}^{op}
\end{tikzcd}
\end{equation}

The following generalization of 
Proposition \ref{MONISMON PROP}
follows from the same proof,
by using Proposition \ref{ANOAN PROP}.

\begin{proposition}
$N_{\mathfrak{C}}$ defines a monad on 
$\mathsf{WSpan}_{G}^r(G \ltimes \Sigma_{\mathfrak{C}}, \mathcal{V}^{op})$.
\end{proposition}



\section{Leftovers}



\subsection{Transferring monad from spans color by color}
\[
\begin{tikzcd}
	\mathsf{WSpan}^r(\Sigma_{\mathfrak{C}}^{op},\mathcal{V}) &
	\mathsf{Fun}(\Sigma_{\mathfrak{C}}^{op},\mathcal{V})
	\ar{l}[swap]{\iota}
\\
	\mathsf{WSpan}^r(\Sigma_{\mathfrak{D}}^{op},\mathcal{V}) \ar{u}{f^{\**}}&
	\mathsf{Fun}(\Sigma_{\mathfrak{D}}^{op},\mathcal{V})
	\ar{l}[swap]{\iota} \ar{u}[swap]{f^{\**}}
\end{tikzcd}
\]

The identity
$f^{\**} \iota = \iota f^{\**}$
induces a map
$\mathsf{Lan} f^{\**} \to f^{\**} \mathsf{Lan}$
and thus a 
composite
\[
\mathsf{Lan} N_{\mathfrak{C}} \iota f^{\**} = 
\mathsf{Lan} N_{\mathfrak{C}} f^{\**} \iota \to 
\mathsf{Lan} f^{\**} N_{\mathfrak{D}} \iota \to
f^{\**} \mathsf{Lan} N_{\mathfrak{D}} \iota
\]

\[
\begin{tikzcd}[column sep=5pt]
	\mathsf{Lan} N_{\mathfrak{C}} \iota \mathsf{Lan} N_{\mathfrak{C}} \iota f^{\**} \ar{rr} &&
	\mathsf{Lan} N_{\mathfrak{C}} \iota \mathsf{Lan} f^{\**} N_{\mathfrak{D}} \iota \ar{r} &
	\mathsf{Lan} N_{\mathfrak{C}} f^{\**} \iota \mathsf{Lan}  N_{\mathfrak{D}} \iota \ar{rr} &&
	f^{\**} \mathsf{Lan} N_{\mathfrak{D}} \iota \mathsf{Lan}  N_{\mathfrak{D}} \iota 
\\
	\mathsf{Lan} N_{\mathfrak{C}} N_{\mathfrak{C}} \iota f^{\**} \ar[equal]{r} \ar{u}{\simeq} \ar{d} &
	\mathsf{Lan} N_{\mathfrak{C}} N_{\mathfrak{C}} f^{\**} \iota  \ar{r} \ar{d} &
	\mathsf{Lan} N_{\mathfrak{C}} f^{\**} N_{\mathfrak{D}} \iota \ar[equal]{r} \ar{u}{\simeq} &
	\mathsf{Lan} N_{\mathfrak{C}} f^{\**}  N_{\mathfrak{D}} \iota \ar{r} \ar{u} &
	\mathsf{Lan} f^{\**} N_{\mathfrak{D}}  N_{\mathfrak{D}} \iota \ar{r} \ar{d} &
	f^{\**} \mathsf{Lan} N_{\mathfrak{D}}  N_{\mathfrak{D}} \iota \ar{u}{\simeq} \ar{d}
\\
	\mathsf{Lan} N_{\mathfrak{C}} \iota f^{\**} \ar[equal]{r} &
	\mathsf{Lan} N_{\mathfrak{C}} f^{\**} \iota  \ar{rrr} 	&&&
	\mathsf{Lan} f^{\**} N_{\mathfrak{D}} \iota \ar{r} &
	f^{\**} \mathsf{Lan} N_{\mathfrak{D}}  \iota 
\end{tikzcd}
\]



\subsection{Left adjoint for induced adjunction on algebras}



\begin{proposition}
Let 
$L \colon \mathcal{C}
\rightleftarrows
\mathcal{D} \colon R$
be an adjuntion, 
$T$ a monad on $\mathcal{D}$,
$\bar{T}$ a monad on $\mathcal{C}$,
and 
$\bar{T} \Rightarrow RTL$
a map of monads.

Then the adjoint map
$L\bar{T} \Rightarrow TL$
is a map of right $\bar{T}$-modules and there is an adjuntion
\[\bar{L} \colon \mathsf{Alg}_{\bar{T}}(\mathcal{C})
\rightleftarrows
\mathsf{Alg}_{T}(\mathcal{D}) \colon R\]
there the left adjoint is given by the reflexive coequalizer
\begin{equation}\label{LEFADJ EQ}
\bar{L}(c) = 
\colim (TL\bar{T} c \rightrightarrows TL c)
\end{equation}
where othe maps come from the multiplication 
$\bar{T}c \to c$ and the right $\bar{T}-$module structure of $TL$.
\end{proposition}

\begin{proof}
	The claim that $L\bar{T} \Rightarrow TL$
	is a right $\bar{T}$-module map is a simple adjunction exercise.
	
	For the formula \eqref{LEFADJ EQ},
	note first that maps of $T$-algebras
	$TL\bar{T}c \to d$ and $TLc \to d$
	are adjoint to maps
	$\bar{T} c \to R d$ and $c \to R d$
	in $\mathcal{D}$.
	Compatibility with multiplication $\bar{T}c\to c$
	says that the first map is the composite
	$\bar{T}c \to c \to R d$.
	
	For compatibility with the right $\bar{T}$-module structure,
	the composite
	$TL\bar{T}c \to TTLc \to TLc \to d$
	in $T$-algebras is adjoint to the composite
	$L\bar{T}c \to TLc \to Td \to d$
	in $\mathcal{D}$ which is adjoint to the composite
	$\bar{T}c \to RTLc \to RTd \to Rd$
	in $\mathcal{C}$.
	But this last composite equals the composite
	$\bar{T}c \to RTLRd \to RTd \to Rd$
	and thus (by definition of the $\bar{T}$-algebra structure on $Rd$), the composite 
	$\bar{T}c \to \bar{T}Rd \to Rd$.

	In other words, the coequalizer conditions state precisely that $c \to Rd$ is a $\bar{T}$-algebra map, and the result follows.
\end{proof}









\subsection{Fiber monads (leftovers)}

\begin{lemma}
Suppose $T$ is a fiber monad for 
$\mathcal{C} \to \mathcal{D}$.
Then if $f\colon d \to d'$ is an isomorphism in $\mathcal{D}$,
the associated map
$T_d f^{\**} \to f^{\**}T_{d'}$
is a natural isomorphism.
\end{lemma}

\begin{proof}
It is sufficient to argue for the existence of both a left and a right inverse. Moreover, it suffices to show that those inverses exist after precomposition and postcomposition with $(f^{-1})^{\**}$. This now follows by noting that the composites below are both identities.
\[T_d f^{\**}(f^{-1})^{\**} \to 
f^{\**} T_{d'} (f^{-1})^{\**} \to
f^{\**}(f^{-1})^{\**} T_d
\qquad 
T_{d'} (f^{-1})^{\**}f^{\**} \to 
(f^{-1})^{\**} T_{d} f^{\**} \to
(f^{-1})^{\**} f^{\**}T_{d'}
\]
\end{proof}






Our next goal is to make this discussion more explicit in the case where
$I=G$ is a group.
Namely, given a $G$-object $d \in \mathcal{D}^G$,
we wish for a more explicit description of the fiber
$\left(\mathcal{C}^G\right)_d$ of $\mathcal{C}^G$ over $d$,
as well as of the fiber monad $(T^G)_d$.
To do so, we will find it helpful to slightly abuse notation by writing $\mathcal{C}_d$ for the fiber of $\mathcal{C}$ over the underlying object $d \in \mathcal{D}$ and $T_d$ for the fiber monad.

Writing $d\xrightarrow{g} d$ for the $G$-action on $d$, one obtains a  $G^{op}$-action on $T_d$ via the pullback functors $g^{\**}$.
Additionally, we write $EG$ for the contractible groupoid with objects the elements of $G$, and note that $EG$ has an obvious 
$G^{op}$-action given by right multiplication of the objects.

It is then straightforward to check that the objects of the equivariant functor category
\begin{equation}\label{TWISTEDFIX EQ}
\mathsf{Fun}^{G^{op}}(EG,\mathcal{C}_d)
\end{equation}
are determined by ``twisted $G$-objects'' in $\mathcal{C}_d$, i.e. objects 
$c\in \mathcal{C}_d$
together with action maps 
$b_g \colon c \to g^{\**}c$
which are associative in the sense that the composite
$c \xrightarrow{b_g} g^{\**}c \xrightarrow{g^{\**}b_h} g^{\**} h^{\**} c = (hg)^{\**}c$ equals $b_{hg}$
and unital in the sense that
$b_e \colon c \xrightarrow{=} e^{\**}c = c$
is the identity
(indeed, the only difference is that the
objects of \ref{TWISTEDFIX EQ} explicitly include the data
of the conjugate maps $g^{\**}b_h$).

One now has the following.



\begin{proposition}
There are natural identifications
\[
\left(\mathcal{C}^G\right)_d \simeq
\mathsf{Fun}^{G^{op}}(EG,\mathcal{C}_d)
\]
Further, the monad $\left(T^G\right)_d$
is identified with the assignment
$(c \to g^{\**}c) \mapsto 
(T_d c \to T_dg^{\**}c \xrightarrow{\simeq} g^{\**}T_d c)$.
\end{proposition}



\begin{proof}
An object of $\left(\mathcal{C}^G\right)_d$
consists of an object $c \in \mathcal{C}_d$ together with action maps
$a_g\colon c \to c$ which lift the action maps $d \xrightarrow{g} d$.
Factoring these maps
as $c \xrightarrow{b_g} g^{\**}c \to c$
yields the corresponding map of $\mathsf{Fun}^{G^{op}}(EG,\mathcal{C}_d)$,
and one can readily check that the associativity and unit conditions on $a_g$ correspond to those on $b_g$.

Lastly, the claim concerning $\left(T^G\right)_d$ simply unpacks definitions, by noting that
$T_d(a_g)$ is the composite
\[
T_d c \xrightarrow{T_d b_g}
T_d g^{\**} c \xrightarrow{\simeq} 
g^{\**} T_d c  \to T_d c 
\]
\end{proof}






\section{Old proof of nerve of free extensions result}





The following is the analogue of \cite[Prop. 3.2]{CM13b}

\begin{proposition}\label{KEYPR PROP}
Suppose that $\mathcal{O} \in \mathsf{Op}^{G,\mathfrak{C}}$
is $\Sigma$-cofibrant.
Further, let $C \in \Sigma_G$ be any $G$-corolla and consider 
a pushout in $\mathsf{Op}^{G}$ of the form
\begin{equation}\label{PUSHOUTPROP EQ}
\begin{tikzcd}
	\partial \Omega(C) \ar{r} \ar{d} & \mathcal{O} \ar{d}
\\
	\Omega(C) \ar{r} & \mathcal{P}.
\end{tikzcd}
\end{equation}
Then the induced map
\begin{equation}\label{ANODYNE MAP}
	\Omega[C] \amalg_{\partial \Omega[C]} N\mathcal{O} \to N\mathcal{P}
\end{equation}
is $G$-inner anodyne.
\end{proposition}

\begin{proof}
The desired claim that \eqref{ANODYNE MAP}
is $G$-inner anodyne will follow by applying the  
\textit{characteristic edge lemma} \cite[Lemma 3.4]{BP_edss},
but we first need some preliminary discussion. 

Let us write $f \colon \partial C \to \mathfrak{C}$
for the induced map of colors.
The first step is to rewrite \eqref{PUSHOUTPROP EQ} as a pushout diagram in $\mathsf{Op}^{G,\mathfrak{C}}$, which can be done by applying $\check{f}_{\**}$
to the leftmost objects in \eqref{PUSHOUTPROP EQ}.
Since
\[
	\check{f}_{\**} \Omega(C) \simeq 
	\check{f}_{\**} \left( \mathbb{F} \Omega'(C) \right) \simeq 
	\mathbb{F} \left(f_{\**}  \Omega'(C) \right)
\]
one has that, writing $C \simeq G \cdot_H C_{\star}$ one has the alternative pushout in $\mathsf{Op}^{G,\mathfrak{C}}$
\begin{equation}
\begin{tikzcd}
	\mathbb{F} ( \emptyset ) \ar{r} \ar{d} & \mathcal{O} \ar{d}
\\
	\mathbb{F} \left( 
	G \cdot_H \Sigma_{\mathfrak{C}}[C^f_{\star}] \right) \ar{r} & \mathcal{P}.
\end{tikzcd}
\end{equation}
Writing $B = G \cdot_H \Sigma_{\mathfrak{C}}[C^f_{\star}]$, one then has
\begin{equation}\label{PUSHOPPR EQ}
	\mathcal{P}(C) = 
	\coprod_{
	[T] \in \mathsf{Iso}
	\left( \Omega_{\mathfrak{C}}^a \downarrow C \right)
	}
	\left(
		\prod_{v \in V^{ac}(T)} \mathcal{O}(T_v)
	\times
		\prod_{v \in V^{in}(T)} B(T_v)
	\right)
	\cdot_{\mathsf{Aut}_{\Omega^a_{\mathfrak{C}}}(T)} \mathsf{Aut}_{\Sigma_{\mathfrak{C}}}(C)
\end{equation}

We now discuss the dendrices of $N \mathcal{P}$. Firstly, recall that, by the strict Segal condition characterization of nerves \cite[Cor. 2.7]{CM13a},
a dendrex $\Omega[T] \to N \mathcal{P}$
is uniquely specified by the tree $T \in \Omega$ together with a choice of operations
$\{p_v \in \mathcal{P}(T_v)\}_{v \in \boldsymbol{V}(T)}$.
Noting that \eqref{PUSHOPPR EQ} implies that the canonical map (of sequences) 
$\mathcal{O} \amalg B \to \mathcal{P}$
is a monomorphism, 
we will say a dendrex $(T,\{p_v\})$ is \textit{elementary}
if all operations $p_v$ are in $\mathcal{O} \amalg B$.
Additionally, an elementary dendrex $(T,\{p_v\})$ is called \textit{alternating} if $T$ is an alternating tree and 
$p_v$ is in $\mathcal{O}$ (resp. $B$) if
$v \in \boldsymbol{V}(T)$ is active (resp. inert).

Given an elementary dendrex $(T,\{p_v\})$ and a map of trees 
$\varphi \colon S \to T$
we will need to know when 
$\varphi^{\**}(T,\{p_v\})$
is again elementary.
Since all maps in $\Omega$ are, uniquely up to isomorphism,
factored as a degeneracy followed by an inner face followed by an outer face, it suffices to discuss each of those cases.
It is straightforward to check that 
$\varphi^{\**}(T,\{p_v\})$ 
is elementary whenever $\varphi$ is a degeneracy or an outer face.
For an inner face
$\varphi \colon T-D \to T$,
noting that a partial composite $p \circ_i q$ of non-unit operations is in $\mathcal{O} \amalg B$ iff both operations are in $\mathcal{O}$,
one sees that $\varphi^{\**}(T,\{p_v\})$ is elementary iff
for each $d \in D$ the adjacent vertices of $T$ are either both labeled by operations of $\mathcal{P}$ or one of them is labeled by an identity.
An elementary dendrex is called \textit{reduced} if it has no such edges. 
In other words, an elementary dendrex is reduced iff none of its inner faces are reduced, so that, in particular, all elementary dendrices admit at least one reduced inner face (note that specifying such a face is somewhat subtle: even when a dendrex is non-degenerate, it may not be enough to collapse edges connecting $\mathcal{O}$ vertices, since this may possibly introduce new identity vertices, resulting in a degenerate vertex).
Note that reduced dendrices are necessarily non-degenerate, but not vice versa.
In fact, a non-degenerate dendrex is reduced iff it has a degeneracy which is an alternating dendrex. 


In what follows, we will find it convenient to work with elementary dendrices that have been suitably ``planarized''.
To do so, fix a subset
$\mathcal{O}^{\mathsf{st}} \amalg B^{\mathsf{st}} 
\subset
\coprod_{C \in \Sigma_{\mathfrak{C}}}
\mathcal{O}(C) \amalg B(C)$
of coset representatives for the $\Sigma$-action,
which we call \textit{standard} representatives.
An elementary dendrex $(T,\{p_v\})$ is then called \textit{standard}
if all $p_v$ are standard (i.e. in 
$\mathcal{O}^{\mathsf{st}} \amalg B^{\mathsf{st}}$).
Moreover, since both $\mathcal{O}$ and $B$ are $\Sigma$-cofibrant/$\Sigma$-free sequences (the former by assumption),
for each elementary simplex $(T,\{p_v\})$,
there is a unique (replanarization) isomorphism
$\varphi \colon T' \to T$ such that
$(T',\{\varphi^{\**}p_v\})$ is standard,
and we write 
$\mathsf{st}(T,\{p_v\}) = \varphi^{\**} (T,\{p_v\}) = (T',\{\varphi^{\**}_v p_v\})$
to denote this.


Note that it now follows from \eqref{PUSHOPPR EQ} that,
for each operation $p \in \mathcal{P}(C)$,
there exists a unique standard alternating dendrex
$b'_p \colon \Omega[T'_p] \to N \mathcal{P}$ and isomorphism 
$C \simeq T'_p - \boldsymbol{E}^{\mathsf{i}}(T'_p)$
such that the composite
\begin{equation}\label{STANDELDE EQ}
	\Omega[C] \simeq
	\Omega[T'_p - \boldsymbol{E}^{\mathsf{i}}(T'_p)] \to 
	\Omega[T'_p] \xrightarrow{b'_p}
	N \mathcal{P}
\qquad
	\Omega[C] \simeq
	\Omega[T_p - \boldsymbol{E}^{\mathsf{i}}(T_p)] \to 
	\Omega[T_p] \xrightarrow{b_p}
	N \mathcal{P}
\end{equation}
is $p$. In fact, due to the correspondence between alternating elementary dendrices and reduced elementary dendrices, 
the analogous claim  also holds 
for the corresponding non-degenerate dendrex 
$b_p \colon \Omega[T_p] \to N \mathcal{P}$.


We can now finally discuss how to apply \cite[Lemma 3.4]{BP_edss}.

Firstly, we need to identify a $G$-poset $I$ and dendrices 
$b_i \colon \Omega[U_i] \to N \mathcal{P}$ for $i \in I$.
Firstly, the underlying set of $I$ is the set of 
non-degenerate standard dendrices of $\mathcal{P}$,
which we abbreviate as
$i = (U_i,\{p_v^i\})$.
The dendrex $b_i \colon \Omega[U_i] \to N \mathcal{P}$ is then tautological, being $i$ itself, but it will preferable to use distinct notations for $i \in I$ and
$b_i \in N \mathcal{P} (U_i)$.
Given $i,j \in I$, we write $i \leq j$ if exists a (in general not  planar) face map
$\varphi \colon U_i \to U_j$
such that $b_i = \varphi^{\**}(b_j)$.
Note that by the uniqueness of standardizations $\varphi$ can only be an isomorphism if $i=j$, showing that $\leq$ indeed satisfies anti-symmetry.
Lastly, we define the $G$-action on $I$ via
\[b_{g i} = \mathsf{st} (g b_i).\]
When reading this formula, note that
$g b_i \in \mathcal{P}(U_i)$
(since this uses the $G$-action
on $\mathcal{P}$),
while $b_{g i} \in \mathcal{P}(U_{g i})$,
where $U_{g i}$ comes with the unique isomorphism
$U_{g i} \xrightarrow{g^{-1}} U_i$ which standardizes $b g_i$.

Lastly, the characteristic edge sets 
$\Xi^i \subseteq \boldsymbol{E}^{\mathsf{i}}(U_i)$ consist of those inner edges such that at least one of the adjacent vertices is mapped to an operation in $B$. Note that, by the discussion above, for an inner face of $\varphi \colon U_i - D \to U_i$
the dendrex $\varphi^{\**}(b_i)$ is elementary iff
$D \cap \Xi^i = \emptyset$.

We now note that it is in fact 
$N \mathcal{P} = 
A \cup \bigcup_{i \in I} b_i\left(\Omega[U_i]\right)$.
Indeed, given an arbitrary (non-elementary) non-degenerate dendrex
$(S,\{p_v\}_{v \in \boldsymbol{V}(T)})$, 
the trees $T_{p_v}$ from \eqref{STANDELDE EQ}
can be regarded as a $S$-substitution datum, which after assembled
yields a non-degenerate standard dendrex 
$\Omega[T] \xrightarrow{b_{\{p_v\}}} N \mathcal{P}$ 
whose image contains $(S,\{p_v\}_{v})$.

We now check the characteristic conditions in \cite[Lemma 3.4]{BP_edss}.

(Ch0.2) is straightforward.

For (Ch1), since outer faces of standard dendrices are again standard, one needs only consider the case 
$\bar{V}=U_i$, or else $\bar{V}$ would be in some $U_j$ for $j<i$.
But if $\bar{V}=U_i$, the assumption in (Ch1) states that
$\Xi^i = \emptyset$, so that $i$ must either be a dendrex where all vertices map to $\mathcal{O}$, i.e. $b_i \in N \mathcal{O}$,
or $U_i$ is a corolla its vertex maps to $B$, i.e. 
$b_i \in B$.
In either case, one has
$b_i \in A = B \cup N\mathcal{O}$, and (Ch1) follows.

To check both (Ch2) and (Ch3), observe first that
$V \hookrightarrow U_i$ will automatically be in $A_{<i}$ if either 
$\bar{V} \neq U_i$ or $T = U_i - D$ with 
$D \not \subseteq \Xi_i$
(since in either case $U_i$ would be in some $U_j$ with $j<i$),
so that one needs only consider the case 
$V=U_i$.

(Ch2) then follows since, except in the trivial cases where $\Xi^i = \emptyset$, the dendrex $b_i(U_i - \Xi^i)$ always contains at least one vertex not in $\mathcal{O} \amalg B$.

For (Ch3), we argue that if 
$b_i(U_i - \Xi^i) \in b_j \left( \Omega[U_j] \right)$
then in fact $i\leq j$.
Writing $\bar{U}_i = U_i - \Xi^i$, the hypothesis is that
\[
\begin{tikzcd}
	\bar{U}_i \ar{d} \ar{r}{\bar{\varphi}} & U_j \ar{r}{b_j} & N \mathcal{P}
\\
	U_i \ar[dashed]{ru}[swap]{\varphi}
\end{tikzcd}
\]
there is a face map $\bar{\varphi}$ as above such that
$b_i(\bar{U}_i) = \bar{\varphi}^{\**}(b_j)$, and the goal is to build $\varphi$ such that $b_i = \varphi^{\**}(b_j)$.
Let $w \in \boldsymbol{V}(\bar{U}_i)$ be a vertex and 
let $p_w$ be the corresponding operation of $\mathcal{P}$.
Then the outer fact $(U_i)_{w}$ is precisely $T_{p_w}$ from
\eqref{STANDELDE EQ}.
On the other hand, letting
$(U_j)_w - D_w \hookrightarrow (U_j)_w$
be any choice of reduced inner face, 
one has that this too is $T_{p_w}$, at least up to a replanarization isomorphism, i.e. one has isomorphims
$(U_i)_{w} \simeq (U_j)_w - D_w$,
compatible with the restrictions of $b_i$, $b_j$.
But then combining these isomorphisms yields the desired $\varphi$,
and (Ch3) follows.

Lastly, we show (Ch0.1).
Given any non-degenerate dendrex
$\Omega[V] \xrightarrow{c} N \mathcal{P}$
by applying \eqref{STANDELDE EQ} to each individual operation
(together with an ``assembly of substitution data'' argument)
one obtains that there exists a unique
non-degenerate standard dendrex 
$b_c \colon \Omega[U_c] \to N \mathcal{P}$,
edge subset $D_c \subseteq \Xi^c$ and isomorphism
$V \simeq U_c - D_c$
such that $b$ equals the composite
\begin{equation}\label{STANDELDEGER EQ}
	V \simeq U_c - D_c
	\hookrightarrow U_c
	\xrightarrow{b_c} N \mathcal{P}
\end{equation}
Recall now that, by the preliminary argument for (Ch2) and (Ch3), 
the non-degenerate dendrices not in 
$b_i^{-1}(A_{< i})$ are precisely the replanarizations of the faces 
$U_i - D$ with $D \subseteq \Xi^i$.
But the uniqueness of the data in \eqref{STANDELDEGER EQ}
implies that all such replanarizations of the $U_i - D$
do indeed have distinct images in $N \mathcal{P}$,
thus establishing (Ch0.1) and finishing the proof.
\end{proof}


\begin{remark}
	In general, injectivity of the map
	$b_i \colon \Omega[U_i] \to N \mathcal{P}$
	will fail in $b_i^{-1}(A_{< i})$.
	Indeed, in general two edges/vertices of $U_i$
	may be assigned the same color/operation of $\mathcal{P}$.
	In fact, injectivity may in general fail even for large outer faces.
\end{remark}









\bibliography{biblio-new}{}
\bibliographystyle{amsalpha2}



\end{document}


%%% Local Variables:
%%% mode: latex
%%% TeX-master: t
%%% End:
