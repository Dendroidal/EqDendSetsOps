\documentclass[a4paper,10pt
%,draft
%, final
]{article}%

\pdfcompresslevel=0
\pdfobjcompresslevel=0

\usepackage[hidelinks]{hyperref}
\hypersetup{
  % colorlinks,
  final,
  pdftitle={Equivariant Dendroidal Segal Spaces},
  pdfauthor={Bonventre, P. and Pereira, L. A.},
  % pdfsubject={Your subject here},
  % pdfkeywords={keyword1, keyword2},
  linktoc=page
}
%\usepackage[open=false]{bookmark}

\input{commands.tex}%


%-------- Tikz ---------------------------

\usepackage{tikz}%
\usetikzlibrary{matrix,arrows,decorations.pathmorphing,
cd,patterns,calc}
\tikzset{%
  treenode/.style = {shape=rectangle, rounded corners,%
                     draw, align=center,%
                     top color=white, bottom color=blue!20},%
  root/.style     = {treenode, font=\Large, bottom color=red!30},%
  env/.style      = {treenode, font=\ttfamily\normalsize},%
  dummy/.style    = {circle,draw,inner sep=0pt,minimum size=2mm}%
}%

\usetikzlibrary[decorations.pathreplacing]



% ---- Commands on draft --------

\usepackage{ifdraft}
\ifdraft{
  \color[RGB]{63,63,63}
  % \pagecolor[rgb]{0.5,0.5,0.5}
  \pagecolor[RGB]{220,220,204}
  % \color[rgb]{1,1,1}
}


\usepackage[draft]{showkeys}
\usepackage{todonotes}%[obeyDraft]


% ----- Labels Changed? --------

\makeatletter

\def\@testdef #1#2#3{%
  \def\reserved@a{#3}\expandafter \ifx \csname #1@#2\endcsname
  \reserved@a  \else
  \typeout{^^Jlabel #2 changed:^^J%
    \meaning\reserved@a^^J%
    \expandafter\meaning\csname #1@#2\endcsname^^J}%
  \@tempswatrue \fi}

\makeatother


% ---- Commands --------

% new symbols

\newcommand{\mycircled}[2][none]{%
  \mathbin{
    \tikz[baseline=(a.base)]\node[draw,circle,inner sep=-1.5pt, outer sep=0pt,fill=#1](a){\ensuremath #2\strut};
cf.  }
}
\newcommand{\owr}{\mycircled{\wr}}

% replace symbols

\renewcommand{\hat}{\widehat}

% random

\renewcommand{\F}{\mathcal F}
\newcommand{\Q}{\mathcal Q}

\newcommand{\lltimes}{\underline{\ltimes}}


% detecting $\V$-categories:

\newcommand{\I}{\mathbb I}
\newcommand{\J}{\mathbb J}
\renewcommand{\1}{\eta}%{\ensuremath{\mathbb{id}}}

% lazy shortcuts

\newcommand{\SC}{\Sigma_{\mathfrak C}}
\newcommand{\OC}{\Omega_{\mathfrak C}}

\newcommand{\UV}{\underline{\mathcal V}}
\newcommand{\UC}{\underline{\mathfrak C}}

\usepackage{harpoon}
\newcommand{\vect}[1]{\text{\overrightharp{\ensuremath{#1}}}}


% ---- Title --------

\title{Equivariant Segal operads, simplicial operads, and dendroidal sets}

\author{Peter Bonventre, Lu\'is A. Pereira}%

\date{\today}


% ---- Document body --------

\begin{document}

\maketitle

\begin{abstract}
      Things and stuff
\end{abstract}

\tableofcontents








\subsection{Remainders from \S 1}




The majority of this research,
beginning with \cite{SS00} and \cite{BM03} and including e.g. \cite{BM07,Mur11,Whi14,Wh16,BB17,PS18,WY18},
has focused on establishing axiomatic frameworks --- i.e., sets of hypotheses on some model category $\V$ ---
such that, in particular, the category $\Op(\V)$ of operads enriched in $\V$
has a model structure lifted from $\V$.
In these cases, $\O \to \P$ is a weak equivalence or fibration iff $\O(n) \to \P(n)$ is so in $\V$ for all $n \geq 0$.
% the categories of $\O$-algebras has a model structure lifted from $\V$ for some (possibly many-colored) operad $\O$ in $\V$.
% In these cases, if $\O$ has a set $\mathfrak C$ of colors, a map $X \to Y$ of $\O$-algebras is a weak equivalence or fibration in the lifted model structure
% iff $X(c) \to Y(c)$ is so in $\V$ for all colors $c$ of $\O$.



These results often also endow lifted model structures on categories $\Op_{\mathfrak C}(\V)$ of so-called colored operads, otherwise known as multicategories, with a fixed set $\mathfrak C$ of colors.
% The category of $\mathfrak C$-colored operads is itself an algebra over an operad whose colors are $\mathfrak C$-signatures
% $\vect C = (c_1,\dots, c_n;c_0)$, $(n+1)$-tuples of elements of $\mathfrak C$.
% As such, the above frameworks establish model structures on each category $\Op_{\mathfrak C}(\V)$.
One may then ask whether these assemble into a single model structure on all operads.
This task has received significantly less attention,
but began with simplicial categories with Dwyer and Kan \cite{DK80}, formalized in Bergner's work \cite{Ber07b}, and by \cite{CM13b,Cav} has been extended to operads.



One common aspect of all of the above cited results is that the theories they produce are ``naive'':
the levels $\O(n)$ have an action of the symmetric group $\Sigma_n$ on $n$ letters,
but the model structures above ignore the isotropy of the operations.
% Does the Rezk model structure detect algebras?
This prevents these theories from detecting necessary features in the equivariant context.
\todo[inline]{$N_\infty$-buisness}




In this paper, we develop an axiomatic system which allows us to build model structures with
weak equivalences which detect isotropy coming from a collection of families.

Equivarinat model structures have only been considered for particular categories, namely spaces and simplicial sets (\cite[Thm. 3.1]{GW18}, \cite[Prop. 3.3.12]{Rez96}).

- It's not just what Stephan does, there is in fact a ton of interaction between the equivariance and the algebra of operads, which for example one doesn't see when only working with categories.



We build model structures on the category of $G$-objects in (symmetric, colored) operads.





\todo[inline]{this is way too technical. introduce things for a bit first!}

This seemingly small addition exposes a number of new challenges.
%
First, the non-equivariant homotopy theories do not take into account any of the isotropy information already present in symmetric operads,
so macherinery must be build to actively detect this behavior.
%
Second, the constituent pieces themselves are more complicated to describe.
For example, producing answers to the basic question
``what is the natural category of $G$-operads with a single fixed set of colors/objects, and how can these pieces interact?''
demands a certain amount of sublty.

As opposed to facing these questions individually,
in this paper we have chosen a perspective which provides answers to them all formally and simultaneously.
The theme runs throughout the entirety of this article:
whenever possible, we construct/package algebraic and homotopy theoretic data universally/abstractly across all relevant fields, and only piece this information apart afterwards.

This comes in three main examples:
\begin{itemize}
	\item fibered categories/adjunctions/monads
	\item families of subgroups of groupoids
	\item fibered Yoneda Lemma / universal symmetric sequences
\end{itemize}

There are several advantages to this perspective:
\begin{itemize}
	\item formal properties do not need to be checked by hand for each particular case
	\item definitions become particular cases of general phenomenon, and often we can recover/compare different perspectives formally by moving across fibers
	\item proofs are streamlined due to the vast amount of formality.
\end{itemize}

Morever, this is \textit{required} to discuss the equivariant setting - ``fiberwise'' discussions are only really fiberwise after forgetting the $G$-action. The $G$-action moves all the pieces around.

The most compelling example of this is the definitions of symmetric sequences and operads.

\todo[inline]{come back}

\subsubsection*{Lit review}

Each of the following starts with a category $\V$ which is a cofibrantly generated, closed symmetric monoidal model category $\V$,
and shows with some additional assumptions that, in particular, $\Op_{\mathfrak C}(\V)$ can be edowed with the model structure transfered from $\Sym_{\mathfrak C}(\V)$ for all sets $\mathfrak C$.

\begin{center}
	\begin{tabular}[h]{c | c | c}
		\textbf{Source} & \textbf{assumptions} & \textbf{conclusions} \\ \hline
		\cite{Mur11} & monoid axiom & non-symmetric colored operads are admissible
		\\
		\cite{BM03}, \cite{BM07} & \begin{tabular}[t]{@{}c@{}} cofibrant unit, sym mon fib replacement, \\ (cocomm) coalg interval \end{tabular} & (symmetric) colored operads are admissible
		\\
		\cite{WY18} & $\spadesuit$ axiom & symmetric colored operads are admissible
		\\
		\cite{BB17} & monoid axiom & tame polynomial monads are admissible
		\\
		\cite{PS18} & (symmetric) $h$-monoidal plus technical & (symmetric) colored operads are admissible
	\end{tabular}
\end{center}

Assembling these into a single model category $\Cat_\bullet(\V)$, $\Op_\bullet(\V)$, requires a significant amount of additional work.
Moreover, there is a flexibility over precisely what the defining features of the model structure should be.
\begin{center}
	\begin{tabular}[h]{c | c | c}
		\textbf{Source} & \textbf{assumptions} & \textbf{model structure} \\ \hline
		\cite{Ber07b},\cite{CM13b} & $\V = \sSet$ & \begin{tabular}[t]{@{}c@{}} weak equivs: DK-equivs \\ fibrations: isofibs \end{tabular}
		\\
		\cite{BM13},\cite{Cav} & ``adequate'' plus technical & \begin{tabular}[t]{@{}c@{}} fibrant objects: locally fibrant \\ trivial fibs: surj local trivial fibs \\ \textit{weak equivalences: DK-equivs} \end{tabular}
		\\
		\cite{Mur15} & monoid axiom & \begin{tabular}[t]{@{}c@{}} weak equivs: DK-equivs \\ trivial fibs: surj local trivial fibs \end{tabular}
		\\
		\cite{Sta14} & \begin{tabular}[t]{@{}c@{}} monoid axiom, simplicial \\ weak equivs closed under transfinite (lazy) 
		\end{tabular}
		&
		\begin{tabular}[t]{@{}c@{}} weak equivs: DK-equivs \\ fibs: isofibs \end{tabular}
	\end{tabular}
\end{center}




\subsubsection*{Remainders}



\begin{remark}
	We have inclusion-forgetful fibered adjunctions
	\begin{equation}\label{JSTAR_CAT_EQ}
	\begin{tikzcd}
	\Cat^{G}(\V)
	\arrow[shift left]{r}{j_!}
	\arrow[d, "{(-)^H}"']
	&
	\Op^{G}(\V)
	\arrow[shift left]{l}{j^{\**}}
	\arrow{d}{(-)^H}
	\\
	\Cat(\V)
	\arrow[shift left]{r}{j_!}
	&
	\Op(\V)
	\arrow[shift left]{l}{j^{\**}}
	\end{tikzcd}
	\end{equation}
	where $j^{\**}$ forgets all non-unary operations and
	$j_!$ is the inclusion $\Cat^G(\V) = \Op^G_\bullet(\V) \downarrow \eta \into \Op^G_\bullet(\V)$,
	and both adjoints are compatible the $H$-fixed point functors.
\end{remark}



\begin{remark}
	We note that if $\V$ satisfies the global monoid axiom and has cofibrant symmetric pushout powers,
	then in particular it satisfies condition $\spadesuit$ of \cite[Thm. 6.1.1]{WY18}:
	First note that
	\[
	X \otimes_{\Sigma_n} (\Sigma_n / H \cdot j) = X/H \cdot j \in \mathcal J^{\otimes} \subseteq \mathcal W
	\]
	for any $H \leq \Sigma_n$ and $j \in \mathcal J$ a generating trivial cofibration of $\V$.
	But since $\otimes$ commutes with colimits in each variable
	and $u^{\square n}$ is a trivial cofibration in $\V^{\Sigma_n}$ for any trivial cofibration $u$ of $\V$,
	the claim follows.
	
	As we mentioned at the begining of this section, the result of \cite[Thm. 6.1.1]{WY18} is insufficent for our purposes,
	as it only includes the case where
	the $(G,\Sigma)$-family $\F$ consists only of the trivial subgroup in each level.
\end{remark}






\bibliography{biblio-new}{}
\bibliographystyle{amsalpha2}



\end{document}


%%% Local Variables:
%%% mode: latex
%%% TeX-master: t
%%% End:
