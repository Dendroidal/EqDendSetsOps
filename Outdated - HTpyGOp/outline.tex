\documentclass[a4paper,10pt
% ,draft
]{article}%

\usepackage[hidelinks]{hyperref}
\hypersetup{
  % colorlinks,
  final,
  pdftitle={Equivariant Dendroidal Segal Spaces},
  pdfauthor={Bonventre, P. and Pereira, L. A.},
  % pdfsubject={Your subject here},
  % pdfkeywords={keyword1, keyword2},
  linktoc=page
}
%\usepackage[open=false]{bookmark}

\input{commands.tex}%


%-------- Tikz ---------------------------

\usepackage{tikz}%
\usetikzlibrary{matrix,arrows,decorations.pathmorphing,
cd,patterns,calc}
\tikzset{%
  treenode/.style = {shape=rectangle, rounded corners,%
                     draw, align=center,%
                     top color=white, bottom color=blue!20},%
  root/.style     = {treenode, font=\Large, bottom color=red!30},%
  env/.style      = {treenode, font=\ttfamily\normalsize},%
  dummy/.style    = {circle,draw,inner sep=0pt,minimum size=2mm}%
}%

\usetikzlibrary[decorations.pathreplacing]



% ---- Commands on draft --------

\usepackage{ifdraft}
\ifdraft{
  \color[RGB]{63,63,63}
  % \pagecolor[rgb]{0.5,0.5,0.5}
  \pagecolor[RGB]{220,220,204}
  % \color[rgb]{1,1,1}
}


\usepackage[draft]{showkeys}
\usepackage{todonotes}%[obeyDraft]


% ----- Labels Changed? --------

\makeatletter

\def\@testdef #1#2#3{%
  \def\reserved@a{#3}\expandafter \ifx \csname #1@#2\endcsname
  \reserved@a  \else
  \typeout{^^Jlabel #2 changed:^^J%
    \meaning\reserved@a^^J%
    \expandafter\meaning\csname #1@#2\endcsname^^J}%
  \@tempswatrue \fi}

\makeatother


% ---- Commands --------

\newcommand{\mycircled}[2][none]{%
  \mathbin{
    \tikz[baseline=(a.base)]\node[draw,circle,inner sep=-1.5pt, outer sep=0pt,fill=#1](a){\ensuremath #2\strut};
cf.  }
}
\newcommand{\owr}{\mycircled{\wr}}
\newcommand{\UV}{\underline{\mathcal V}}

\newcommand{\UC}{\underline{\mathfrak C}}

\renewcommand{\F}{\mathcal F}
\newcommand{\I}{\mathbb I}
\newcommand{\J}{\mathbb J}
\newcommand{\Q}{\mathcal Q}
\renewcommand{\1}{\ensuremath{\mathbb{id}}}
\renewcommand{\hat}{\widehat}
\newcommand{\lltimes}{\underline{\ltimes}}

% ----- Enumerate ----------
\renewcommand\labelenumi{(\theenumi)}

% ---- Title --------

\title{To be determined}%

\author{Peter Bonventre, Lu\'is A. Pereira}%

\date{\today}


% ---- Document body --------

\begin{document}

\maketitle

\begin{abstract}
      Things and stuff
\end{abstract}

\tableofcontents


\section{Introduction}

Blah

\subsection{Main Results}

\textbf{Operadic model structures:}

Let $(\V, \otimes)$ denote $(\sSet, \times)$ or $(\sSet_{\**}, \wedge)$.

\begin{theorem}
      For all $\mathfrak C$, there exists a model structure on the category $\Op^{G, \mathfrak C}(\V)$ of
      $\mathfrak C$-colored operads where weak equivalences are detected on all graph subgroups.
\end{theorem}

\begin{theorem}
      There exists an $\F$-model structure on the category $\Op^G(\V) = \Op(\V)^G$
      where weak equivalences are ``graph Dwyer-Kan equivalences''.
\end{theorem}

\textbf{Tame:}

\begin{theorem}
      There exists a model structure $\mathsf{PreOp}^G_t$ on the category of $G$-preoperads,
      along with Quillen equivalences
      \begin{equation}
            \mathsf{PreOp}^G \leftrightarrows \mathsf{PreOp}^G_t \rightleftarrows \sOp^G
      \end{equation}
\end{theorem}

\textbf{Conclusion:}

\begin{theorem}
      The Quillen adjunction $h c N: \sOp^G \leftrightarrows \dSet^G : W_!$ is a Quillen equivalence.
\end{theorem}

Moreover, $\F$-model structures exist for all indexing families $\F$,
and similar results hold throughout if we replace $(\V, \otimes)$ with any category satisfying several axioms.


\section{Preliminaries}

\subsection{Grothendieck fibrations and wreath products}

\begin{enumerate}
\item section 2.1 wreath products
\end{enumerate}

\subsection{category of trees}

\begin{enumerate}
\item recall $\Omega$, $\Omega_G$
\end{enumerate}

\subsection{some model structure stuff}

\begin{enumerate}
\item section 7.2 semi-cofibrantly generated
\item Thm 7.40 and Cor 7.41 [SIMPLQUILL COR]      
\end{enumerate}

\section{Colored operads}

Intro and stuff

\subsection{operads and symmetric sequences}

\begin{enumerate}
\item Section 3.1 non-equivariant colored sequences and operads
\item Section 3.2 equivariant colored sequences and operads [combine with first part of Section 7.6]
\item statement of existance of monad
\end{enumerate}

\subsection{Color-fixed model structure}

\begin{enumerate}
\item existance of filtration
\item section 4.2 model structures
\end{enumerate}

\section{The homotopy theory of colored operads}

- all of section 5:
\subsection{intro and stuff}
\subsection{trivial cofibrations}
\subsection{2-out-of-2}
\subsection{Dwyer-Kan}

\section{Segal preoperads}

\subsection{Review}

\begin{enumerate}
\item model structures on $\mathsf{sdSet}^G$, $\mathsf{PreOp}^G$
\end{enumerate}

\subsection{Main lemma}

\begin{enumerate}
\item section 7.6 (recall Section 3.2)
\end{enumerate}

\subsection{tame model structure}

\begin{enumerate}
\item first half of 7.7
\end{enumerate}


\section{Comparison}

\begin{enumerate}
\item second half of 7.7
\item (equivariant) Prop 7.5
\item proof of main theorem
\end{enumerate}



\appendix

\section{homotopy in a general model category}

\begin{enumerate}
\item section 2.1
\end{enumerate}


\section{Construction of colored operads}

\subsection{Preliminaries}

\begin{enumerate}
\item section 2.2 stuff about $G \ltimes (-)$
\item section 8.1 2-overcategories
\item section 8.3 Pullback Functors
\item section 7.5 TBD
\end{enumerate}

\subsection{edge data}

\begin{enumerate}
\item section 8.2 edge data
\end{enumerate}

\subsection{monad on spans}

\subsection{equivariant monad}

\subsection{the filtration}

- section on the filtration


\subsection{change of color}
- section 9.2 bla


\section{left over sections?}

\begin{enumerate}
\item Setion 6.1 htpy gen eq op
\item Section 7.1 colored simplicial tensors and cotensors
\item Section 7.4 extra lifts of infty-cats
\end{enumerate}


\end{document}
