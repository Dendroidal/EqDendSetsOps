\documentclass[a4paper,10pt
,draft
%, final
]{article}%


% ---- Commands on draft --------

\usepackage[dvipsnames]{xcolor}% adds colors
\usepackage{ifdraft}
\ifdraft{
  \color[RGB]{63,63,63}
  \pagecolor[RGB]{220,220,204}
  \usepackage[notref]{showkeys}
  \usepackage{geometry}
  \usepackage{todonotes}
}
{
  \usepackage[margin=1in]{geometry}
  \usepackage[fontsize=12pt]{scrextend}
  \usepackage[disable]{todonotes}
}

\pdfcompresslevel=0
\pdfobjcompresslevel=0

%\usepackage{xr-hyper}
\usepackage[pagebackref, colorlinks, citecolor=PineGreen, linkcolor=PineGreen]{hyperref}
\hypersetup{
  final,
  pdftitle={Equivariant dendroidal sets and simplicial operads},
  pdfauthor={Bonventre, P. and Pereira, L. A.},
  linktoc=page
}

%\externaldocument*[HGOP-]{HtpyGOp} % cite using names from other half

\usepackage{amsmath, amsthm}% {amsfonts, amssymb}

% ------ New Characters --------------------------------------

\usepackage[latin1]{inputenc}%
\usepackage{MnSymbol}
\DeclareMathAlphabet\mathbb{U}{msb}{m}{n}
%\usepackage{stmaryrd}
%\usepackage{upgreek}
\usepackage{mathrsfs}
% \usepackage[T1]{fontenc}
% \usepackage[english]{babel}
% \usepackage{fouriernc}
% \DeclareMathAlphabet{\mathscr}{U}{mathrsfs}{m}{n}`
\usepackage{pifont}
\newcommand{\cmark}{\text{\ding{51}}}
\newcommand{\xmark}{\text{\ding{55}}}

\usepackage[normalem]{ulem}%
\usepackage{dsfont}%
\usepackage{bbm}%


%----- Enumerate ---------------------------------------------
% \usepackage{paralist} % for inparaenum
% \usepackage{enumerate}%
\usepackage[inline,shortlabels]{enumitem}%


% ---------- Page Typesetting ----------
\usepackage[final]{microtype}
\usepackage{relsize}

%-------- Tikz ---------------------------

\usepackage{tikz}%
\usetikzlibrary{matrix,arrows,decorations.pathmorphing,
cd,patterns,calc}
\tikzset{%
  % treenode/.style = {shape=rectangle, rounded corners, draw, align=center, font=\footnotesize,
  %                    top color=white, bottom color=blue!20},%
  % root/.style     = {treenode, font=\Large, bottom color=red!30},%
  % env/.style      = {treenode, font=\ttfamily\normalsize},%
  dummy/.style    = {circle,draw,inner sep=0pt,minimum size=2mm}%
}%

\usetikzlibrary[decorations.pathreplacing]




% ----- Labels Changed? --------

\makeatletter

\def\@testdef #1#2#3{%
  \def\reserved@a{#3}\expandafter \ifx \csname #1@#2\endcsname
  \reserved@a  \else
  \typeout{^^Jlabel #2 changed:^^J%
    \meaning\reserved@a^^J%
    \expandafter\meaning\csname #1@#2\endcsname^^J}%
  \@tempswatrue \fi}

\makeatother

%%%%%%%%%%%%%%%%%%%%%%%%% INTERNAL REFERENCES %%%%%%%%%%%%%%%%%%%%%%%%%%%%%%%%%%%

\numberwithin{equation}{section} 
\numberwithin{figure}{section}

\usepackage{mathtools}
\mathtoolsset{showonlyrefs,showmanualtags} % Only number equations which are referenced with eqref


% ------- New Theorems/ Definition/ Names-----------------------

 % \theoremstyle{plain} % bold name, italic text
\newtheorem{theorem}[equation]{Theorem}%
\newtheorem*{theorem*}{Theorem}%
\newtheorem{lemma}[equation]{Lemma}%
\newtheorem{proposition}[equation]{Proposition}%
\newtheorem{corollary}[equation]{Corollary}%
\newtheorem{conjecture}[equation]{Conjecture}%
\newtheorem*{conjecture*}{Conjecture}%
\newtheorem{claim}[equation]{Claim}%

%%%%%% Fancy Numbering for Theorems
\newtheorem{innercustomgeneric}{\customgenericname}
\providecommand{\customgenericname}{}
\newcommand{\newcustomtheorem}[2]{%
  \newenvironment{#1}[1]
  {%
   \renewcommand\customgenericname{#2}%
   \renewcommand\theinnercustomgeneric{##1}%
   \innercustomgeneric
  }
  {\endinnercustomgeneric}
}

\newcustomtheorem{customthm}{Theorem}
\newcustomtheorem{customcor}{Corollary}
%%%%%%%%%%%%%

\theoremstyle{definition} % bold name, plain text
\newtheorem{definition}[equation]{Definition}%
\newtheorem*{definition*}{Definition}%
\newtheorem{example}[equation]{Example}%
\newtheorem{remark}[equation]{Remark}%
\newtheorem{notation}[equation]{Notation}%
\newtheorem{convention}[equation]{Convention}%
\newtheorem{assumption}[equation]{Assumption}%
\newtheorem{exercise}{Exercise}%


% %%%%%%%%%%%%%%%%%%%%%%%%%%%%%%%%%%%%%%%%%%%%%%%%%%%%%%%%%%%%%%%%%%%%%%%%%%%%%%%%
% ------------------------------ COMMANDS ------------------------------

% ---------- macros

\newcommand{\set}[1]{\left\{#1\right\}}%
\newcommand{\sets}[2]{\left\{ #1 \;|\; #2\right\}}%
\newcommand{\longto}{\longrightarrow}%
\newcommand{\into}{\hookrightarrow}%
\newcommand{\onto}{\twoheadrightarrow}%

\usepackage{harpoon}
\newcommand{\vect}[1]{\text{\overrightharp{\ensuremath{#1}}}}


% ---------- operators

\newcommand{\Sym}{\ensuremath{\mathsf{Sym}}}%
\newcommand{\Fin}{\mathsf{F}}%
\newcommand{\Set}{\ensuremath{\mathsf{Set}}}
\newcommand{\Top}{\ensuremath{\mathsf{Top}}}
\newcommand{\sSet}{\ensuremath{\mathsf{sSet}}}%
\newcommand{\Cat}{\mathsf{Cat}}
\newcommand{\sCat}{\mathsf{sCat}}
\newcommand{\Op}{\mathsf{Op}}%
\newcommand{\sOp}{\ensuremath{\mathsf{sOp}}}%
\newcommand{\sSym}{\ensuremath{\mathsf{sSym}}}%
\newcommand{\fgt}{\ensuremath{\mathsf{fgt}}}%
\newcommand{\dSet}{\mathsf{dSet}}
\newcommand{\Fun}{\mathsf{Fun}}
\newcommand{\Fib}{\mathsf{Fib}}
\newcommand{\Alg}{\mathsf{Alg}}
\newcommand{\Kl}{\mathsf{Kl}}



\DeclareMathOperator{\hocmp}{hocmp}%
\DeclareMathOperator{\cmp}{cmp}%
\DeclareMathOperator{\hofiber}{hofiber}%
\DeclareMathOperator{\fiber}{fiber}%
\DeclareMathOperator{\hocofiber}{hocof}%
\DeclareMathOperator{\hocof}{hocof}%
\DeclareMathOperator{\holim}{holim}%
\DeclareMathOperator{\hocolim}{hocolim}%
\DeclareMathOperator{\colim}{colim}%
\DeclareMathOperator{\Lan}{Lan}%
\DeclareMathOperator{\Ran}{Ran}%
\DeclareMathOperator{\Map}{Map}%
\DeclareMathOperator{\Id}{Id}%
\DeclareMathOperator{\mlf}{mlf}%
\DeclareMathOperator{\Hom}{Hom}%
\DeclareMathOperator{\Ho}{Ho}
\DeclareMathOperator{\Aut}{Aut}%
\DeclareMathOperator{\Stab}{Stab}
\DeclareMathOperator{\Iso}{Iso}
\DeclareMathOperator{\Ob}{Ob}

% ---------- shortcuts

\newcommand{\F}{\ensuremath{\mathcal F}}
\newcommand{\V}{\ensuremath{\mathcal V}}
\newcommand{\Q}{\ensuremath{\mathcal Q}}
\renewcommand{\O}{\ensuremath{\mathcal O}}
\renewcommand{\P}{\ensuremath{\mathcal P}}
\newcommand{\C}{\ensuremath{\mathcal C}}
\newcommand{\A}{\ensuremath{\mathcal A}}

\newcommand{\del}{\partial}%

\newcommand{\ki}{\chi}
\newcommand{\ksi}{\xi}
\newcommand{\Ksi}{\Xi}

\newcommand{\lltimes}{\underline{\ltimes}}

% detecting $\V$-categories:

\newcommand{\I}{\mathbb I}
\newcommand{\J}{\mathbb J}
\newcommand{\1}{\ensuremath{\mathbbm 1}}%{\ensuremath{\mathbb{id}}} %\eta

% lazy shortcuts

\newcommand{\SC}{\Sigma_{\mathfrak C}}
\newcommand{\OC}{\Omega_{\mathfrak C}}
\newcommand{\UV}{\underline{\mathcal V}}
\newcommand{\UC}{\underline{\mathfrak C}}











% %%%%%%%%%%%%%%%%%%%%%%%%%%%%%%%%%%%%%%%%%%%%%%%%%%%%%%%%%%%%%%%%%%%%%%%%%%%%%%%%%%%%%%%%%%%%%%%%%%%%
% ------------------------------ MAIN BODY ------------------------------

% ---- Title --------

\title{Equivariant dendroidal sets and simplicial operads}

\author{Peter Bonventre, Lu\'is A. Pereira}%

\date{\today}


% ---- Document body --------

\begin{document}

\maketitle

\begin{abstract}
      Things and stuff
\end{abstract}












\subsection{The $W$ construction}

To extend $\tau \colon \dSet \rightleftarrows \Op \colon N$ from \eqref{DSETADJ_EQ} to a Quillen adjunction out of $\sOp$,
we need a topological enrichment of the inclusion $\Omega \into \Op$ from Definition \ref{OT_DEF}.
This is given in Definition \ref{WT_DEF}, with an explicit description following in Proposition \ref{WT_PROP}.

Briefly, $W(T)$ is the free simplicial resolution\footnote{Also called the \textit{Godement} resolution, c.f. \cite[\S 8.3]{BM06}.}
of $\Omega(T)$ as an algebra in the category of pointed $\mathbf E(T)$-colored symmetric sequences $\Sym^{\mathbf E(T)}_{\eta}(\Set)$.
In order to say more, we will need to use the results from Appendix \ref{AMALGMON_SEC}, and consider
two orthogonal factorizations of the free operad monad $\mathbb F$.

\subsubsection{Factorizations of the free operad monad}

Let $\V$ be a closed symmetric monoidal category.
The free operad monad $\mathbb F \colon \Sym_\bullet(\V) \to \Op_\bullet(\V)$ formally add two types of data to a given symmetric sequence:
units and compositions.
In fact, these correspond to two separate and independent monads, $P$ and $\bar{\mathbb F}$, on $\Sym_\bullet(\V)$.
The first ``adds units''.
\begin{definition}
      Let $P$ be the endofunctor on $\Sym_{\bullet}(\V)$
      which sends a symmetric sequence $(\mathfrak C, X)$ to $(\mathfrak C, X \amalg^{\mathfrak C} \eta_{\mathfrak C})$,
      where we highlight that the coproduct is taken in $\Sym_{\mathfrak C}(\V)$ and not $\Sym_{\bullet}(\V)$.
      This has an obvious monad structure via the fold map.      
\end{definition}

Algebras over $P$ are pointed symmetric sequences.
\begin{definition}
      The category of \textit{pointed} symmetric sequences in $\V$, denoted $\Sym_{\bullet,\eta}(\V)$, is the cateogry with
      \begin{itemize}
      \item objects given by triples $(\mathfrak C, X, \eta_{\mathfrak C} \xrightarrow{1_X} X)$ with
            $\mathfrak C \in \Set$ a set of colors, $X \in \Sym_{\mathfrak C}(\V)$ a $\mathfrak C$-colored symmetric sequence, and
            $\eta_{\mathfrak C}$ the intial $\mathfrak C$-colored operad.
      \item arrows $(\mathfrak C, X, 1_X) \to (\mathfrak D, Y, 1_Y)$ given by
            a map $(\mathfrak C, X) \to (\mathfrak D, Y)$ of symmetric sequences as in \eqref{SYMSEQMAP_EQ} such that
            the pasting diagrams below coincide.
            \[
                  \begin{tikzcd}
                        \Sigma_{\mathfrak C}^{op} \arrow[d, "\eta_{\mathfrak C}"'] \arrow[r, equal]
                        &
                        \Sigma_{\mathfrak C}^{op} \arrow[d, "X"'] \arrow[r, "\phi"]
                        &
                        \Sigma_{\mathfrak D}^{op} \arrow[d, "Y"]
                        & % ----------
                        \Sigma_{\mathfrak C}^{op} \arrow[d, "\eta_{\mathfrak C}"'] \arrow[r, "\phi"]
                        &
                        \Sigma_{\mathfrak D}^{op} \arrow[d, "\eta_{\mathfrak d}"'] \arrow[r, equal]
                        &
                        \Sigma_{\mathfrak D}^{op} \arrow[d, "Y"]
                        \\
                        \V \arrow[r, equal] \arrow[ur, Rightarrow, "1_X"]
                        &
                        \V \arrow[r, equal] \arrow[ur, Rightarrow]
                        &
                        \V
                        & % ----------
                        \V \arrow[r, equal]
                        &
                        \V \arrow[r, equal] \arrow[ur, Rightarrow, "1_Y"]
                        &
                        \V
                  \end{tikzcd}
            \]
      \end{itemize}

      There is a natural (split) Grothendieck fibration $\Sym_{\bullet,\eta}(\V) \to \Set$, factoring through $\Sym_\bullet(\V)$,
      with fiber over $\mathfrak C \in \Set$ given by
      the undercategories $\Sym_{\mathfrak C, \eta}(\V) := \eta_{\mathfrak C} \downarrow \Sym_{\mathfrak C}(\V)$.
\end{definition}


If we ignore units when adding formal composites, the resulting algebraic objects are non-unital operads.
\begin{definition}
      A \textit{non-unital operad} $\O \in \Op_{\bullet, nu}(\V)$ is an operad without units or unitality conditions; that is, 
      a symmetric sequence $(\mathfrak C, \O) \in \Sym_{\bullet}(\V)$,
      equipped with an associative and $\Sigma$-equivariant composition law.
      Specifically, we have all ``partial'' composition laws\footnote{
        We note that for unital operads, whether we define them via partial composition laws or total composition laws makes no difference. The same does not hold for non-unital operads.}
      \[
            \O(n) \times_{\mathfrak C^{\times \lambda}}
            \prod_{i \in \lambda} \O(k_i) \longto \O\left(n-|\lambda|+\sum_{i \in \lambda}k_i\right)
      \]
      for any subset $\lambda \subseteq \set{1,2,\dots,n}$.

      Non-unital operads and maps which change colors and respect the composition operation yield
      a category $\Op_{\bullet,nu}(\V)$ fibered over $\Set$.      
\end{definition}

The monad $\bar{\mathbb F}$ on $\Sym_\bullet(\V)$ encoding non-unital operads
is constructed analogously to $\mathbb F$ as in \cite[Appendix A]{BP_HGOP}, % {MONAD_APDX}
with one key difference:
for any set of colors $\mathfrak C$,
the fundamental category $\Omega_{\mathfrak C}^0$ of $\mathfrak C$-colored trees and isomorphisms
is replaced with the subcategory $\Omega_{\mathfrak C}^{\textrm{nu},0}$ of non-stick trees (i.e, $\vect T \in \Omega_{\mathfrak C}$ with $|\boldsymbol{V}(T)| \geq 1$) and isomorphisms.
This has the effect of replacing the categories $\Omega_{\mathfrak C}^n$ of $n$-strings of \textit{planar tall maps} --- i.e. composites of inner faces and degeneracies --- 
with the subcategory $\Omega_{\mathfrak C}^{\textrm{nu},n}$ of
non-stick trees and $n$-strings of just inner face maps:
a planar tall map $T \to S$ is equivalent to substitution data $(T_v \to S_v)_{v \in V(T)}$ by \cite[Prop. 3.41]{BP_geo},
and if we forbid $S_v$ from being a stick,
we remove the possibility of $T \to S$ to be a degeneracy.

As observed in \cite{BP_geo} \todo{find a more specific citation}, degeneracies and sticks precisely record the unit and the unitality conditions for operads.
Thus, the results from \cite[Appendix A]{BP_HGOP} % \ref{MONAD_APDX}
may be adapted mutatis mutandis;
in particular, \eqref{FREEOP_EQ}, Definition \ref{COLORMON DEF}, and Proposition \ref{ASSOCIDS PROP} yield the following.

\begin{definition}
      The free non-unital operad monad $\bar{\mathbb F}$ on $\Sym_\bullet(\V)$
      sends a symmetric sequence $X \colon \Sigma_{\mathfrak C}^{op} \to \V$ in $\Sym_{\bullet}(\V)$ to the left Kan extension below.
      \[
            \begin{tikzcd}
                  \left(\Omega_{\mathfrak C}^{\text{nu},0}\right)^{op} \arrow[r, hookrightarrow] \arrow[d, "\mathsf{lr}"]
                  &
                  \Omega_{\mathfrak C}^{0,op} \arrow[r,"V"] \arrow[dl, Rightarrow,shorten >=0.4cm,shorten <=0.4cm]
                  &
                  \left(\Sigma \wr \Sigma_{\mathfrak C}\right)^{op} \arrow[r, "X"] 
                  &
                  \left(\Sigma \wr \V^{op}\right)^{op} \arrow[r, "\otimes"]
                  &
                  \V
                  \\
                  \Sigma_{\mathfrak C}^{op} \arrow[urrrr, "\Lan = \bar{\mathbb F} X"']
            \end{tikzcd}
      \]
\end{definition}
\begin{lemma}
      The iterated monad $\bar{\mathbb F}^{\circ n}X$ is given by the the following left Kan extension.
      \begin{equation}
            \label{FREENUOP_EQ}
            \begin{tikzcd}
                  \left(\Omega_{\mathfrak C}^{\text{nu},n}\right)^{op} \arrow[r, hookrightarrow] \arrow[d, "\mathsf{lr}"]
                  &
                  \Omega_{\mathfrak C}^{n,op} \arrow[r,"{(V^0)^{\circ n+1}}"]
                  \arrow[dl, Rightarrow, shorten >=0.4cm,shorten <=0.4cm]
                  &[10pt]
                  \left(\Sigma ^{\wr n+1} \wr \Sigma_{\mathfrak C}\right)^{op} \arrow[r, "{(\sigma^0)^{\circ n+1}}"]
                  &[10pt]
                  \left(\Sigma \wr \Sigma_{\mathfrak C}\right)^{op} \arrow[r, "X"]
                  &
                  \left(\Sigma \wr \V^{op}\right)^{op} \arrow[r, "\otimes"]
                  &
                  \V
                  \\
                  \Sigma_{\mathfrak C}^{op} \arrow[urrrrr, start anchor = east, end anchor = south west, "\Lan = \bar{\mathbb F}^{\circ n} X"']
            \end{tikzcd}
      \end{equation}
\end{lemma}




Now, we apply Appendix \ref{AMALGMON_SEC} to $P$ and $\bar{\mathbb F}$.
First, we observe the following by hand.
\begin{lemma}
      \label{PBARFCOMB_LEM}
      As monads, $\mathbb F = P \bar{\mathbb F}$.
      Moreover, the monad $P$ and $\bar{\mathbb F}$ on $\Sym_{\bullet}(\V)$ can be combined as in Definition \ref{AMALGMON DEF}:
      The natural transformations $\mathbb P \Rightarrow \mathbb F$ and $\bar{\mathbb F} \Rightarrow \mathbb F$
      on $(\mathfrak C, X)$ in $\Sym_\bullet(\V)$ are simply
      \[
            X \amalg^{\mathfrak C} \eta_{\mathfrak C} \xrightarrow{\eta \amalg e} \mathbb F X,
            \qquad
            \bar{\mathbb F} X \hookrightarrow \bar{\mathbb F} X \amalg^{\mathfrak C} \eta_{\mathfrak C}
      \]
      with $\eta$ the adjunction counit and $e$ the inclusion of the units.      
\end{lemma}
% \begin{proof}
%       \includegraphics[width=\textwidth]{P-F-combined.pdf}
% \end{proof}

\begin{proposition}
      The forgetful functor $\Op_\bullet(\V) \to \Sym_{\bullet, \eta}(\V)$ has a left adjoint $\mathbb F_\eta$.
      Moreover, for $Y \in \Sym_{\mathfrak C, \eta}(\V)$,
      the following is a pushout diagram in $\Op_{\mathfrak C}(\V)$.
      \begin{equation}
            \label{FETA_EQ}
            \begin{tikzcd}
                  \mathbb F \eta_{\mathfrak C} \arrow[d] \arrow[r]
                  &
                  \eta_{\mathfrak C} \arrow[d]
                  \\
                  \mathbb F Y \arrow[r]
                  &
                  \mathbb F_{\eta} X
            \end{tikzcd}
      \end{equation}
\end{proposition}
\begin{proof}
      As $\mathbb P$ and $F = \mathbb F$ both preserve reflexive coequalizers\footnote{
        For $\mathbb F$, this follows as $\otimes$ commuting with colimits in both variables implies it commutes with reflexive coequalizers; see e.g. \cite[Lemma 2.3.2]{Rez96}.},
      Corollary \ref{ALGLEFTEX_COR} and \eqref{FRMONDES EQ} imply that
      $\mathbb F_\eta$ exists, 
      and sends a pointed symmetric sequence $Y \colon \Sigma_{\mathfrak C}^{op} \to \V$ in $\Sym_{\bullet, \eta}(\V)$
      to the coequalizer of the following in $\Sym_{\mathfrak C}(\V)$
      \begin{equation}
            \begin{tikzcd}
                  \mathbb F P Y = \mathbb F(Y \amalg \1_{\mathfrak C}) \arrow[r, shift left, "f_1"] \arrow[r, shift right, "f_2"']
                  &
                  \mathbb F Y
            \end{tikzcd}
      \end{equation}
      with
      \begin{equation}
            \label{FPYCOEQ_EQ}
            f_1 = \mathbb F(Y \amalg \1 \xrightarrow{\ id,e \ } Y),
            \qquad
            f_2 \colon \mathbb F(Y \amalg \1) \xrightarrow{\eta_{\bar{\mathbb F}} \amalg id}
            \mathbb F(\bar{\mathbb F} Y \amalg \1) = \mathbb F^2 Y \xrightarrow{\mu} \mathbb FY.
      \end{equation}
      
      For the moreover claim, we note that $f_2$ has another decomposition, namely
      \[
            \mathbb F (Y \amalg \eta) = P\bar{\mathbb F} Y \mathbin{\hat{\amalg}} \mathbb F_\eta \xrightarrow{\eta_{\bar{\mathbb F}} \amalg id}
            P\bar{\mathbb F} \bar{\mathbb F} Y \mathbin{\hat{\amalg}} \mathbb F \eta \xrightarrow{\mu_{\bar{\mathbb F}} \amalg (id \circ \mu)}
            F Y \amalg \eta = \mathbb F Y,
      \]
      where $\hat{\amalg}$ denotes the coproduct in $\mathbb F$-algebras
      (as $P \bar{\mathbb F} \bar{\mathbb F} \to \mathbb F Y$, by Proposition \ref{ALTMULT PROP},
      and both $\mathbb F \eta \to \eta$ and the unique map $\eta \to \O$ are all maps of $\mathbb F$-algebras).

      But then this equalizier is of form
      $
      \O \mathbin{\hat{\amalg}} \P \rightrightarrows \O
      $
      with both maps the identity on the $\O$ component.
      As in any category, this is isomorphic to the coequalizer of the interesting pieces
      $\P \rightrightarrows \O$,
      and thus $F_P Y$ is the coequalizer below of $\mathbb F$-algebras.
      \[
            \mathbb F_\eta \overset{e \circ \mu}{\underset{\mathbb F(e)} \rightrightarrows} \mathbb F Y \xrightarrow{\mathrm{coeq}} F_P Y
      \]
      Converting this coequalizer into a pushout yields the diagram below.
      \begin{equation}
            \label{FETA2_EQ}
            \begin{tikzcd}
                  \mathbb F \eta \mathbin{\hat{\amalg}} \mathbb F \eta \arrow[r] \arrow[d, "\nabla"']
                  &
                  \mathbb F Y \arrow[d]
                  \\
                  \mathbb F \eta \arrow[r]
                  &
                  \mathbb F_\eta Y
            \end{tikzcd}
      \end{equation}
      Finally, one can check that \eqref{FETA_EQ} and \eqref{FETA2_EQ} satisfy the same universal property, finishing the proof.
      the result follows.
\end{proof}

Summarizing, we have the following.
\begin{proposition}
      \label{FFACT_PROP}
      We have the following commuting square (up to natural isomorphism) of left adjoints.
      \begin{equation}
            \begin{tikzcd}
                  \Sym_\bullet(\V) \arrow[d, "P"'] \arrow[r, "\bar{\mathbb F}"]
                  &
                  \Op_{\bullet, nu}(\V) \arrow[d, "\mathbb P"]
                  \\
                  \Sym_{\bullet, \eta}(\V) \arrow[r, "\mathbb F_\eta"]
                  &
                  \Op_\bullet(\V)
            \end{tikzcd}
      \end{equation}
\end{proposition}


\subsubsection{The $W$-construction}

We now restrict to the case $\V = \sSet$ and define $W(T)$.

\begin{definition}
      \label{WT_DEF}
      For any $T \in \Omega_G$, we define $W(T) \in \sOp_{\boldsymbol{E}(T)}^G \subseteq (\Op_{\boldsymbol{E}(T)}^G)^{\Delta^{op}}$
      to be the simplicial object in operads with levels
      \[
            W(T)_n := \mathbb F_\eta^{\circ n+1}\Omega(T)
      \]
      and natural simplicial structure maps.
\end{definition}

\begin{proposition}
      \label{WT_PROP}
      For any $T \in \Omega_G$ and $\boldsymbol{E}(T)$-signature $\vect C = (e_1, \dots, e_k; e) = (\underline{e}; e)$, we have
      \[
            W(T)(\vect C) = \Delta[1]^{\times \boldsymbol{E}^i(T_{\underline{e} \leq e})}
      \]
      is the (nerve of the) poset of inner faces of $T_{e_1 e_2 \dots e_k \leq e}$.
\end{proposition}
\begin{proof}
      Applying Corollary \ref{ALGLEFTEX_COR} from the appendix to Lemma \ref{PBARFCOMB_LEM},
      as well as Definition \ref{OT_DEF}, yields
      \begin{equation}
            \label{WT_EQ}
            W(T)_n = \mathbb F_\eta^{\circ n+1} \Omega(T)
            = \mathbb F_\eta^{\circ n+1} \mathbb P \bar{\mathbb F} \Sigma_\tau[T]
            = P \bar{\mathbb F}^{\circ n+2}\Sigma_\bullet[T].
      \end{equation}
      (where the functors $\mathbb P, \mathbb F_\eta, \bar{\mathbb F}$ denote their underlying endomorphisms on $\Sym_{\bullet}$).
      Now, \eqref{FREENUOP_EQ} says
      \[
            \bar{\mathbb F}^{\circ n+2} \Sigma_\bullet[T] = \Lan_{(\Omega_\bullet^{r,n+1}\to \Sigma_\bullet)^{op}} \bar N^{T},
      \]
      with
      \[
            \left(\bar N^{T}\right)^{op} \colon
            \Omega_{\boldsymbol E(T)}^{r, n+1} \xrightarrow{\boldsymbol{V}}
            \Sigma^{\wr n+2} \wr \Sigma_{\boldsymbol E(T)} \longto
            \Sigma \wr \Sigma_{\boldsymbol E(T)} \xrightarrow{\Sigma_\bullet[T]}
            \Sigma \wr \V^{op} \xrightarrow{\otimes}
            \V^{op}.
      \]
      But $\bar N^T$ evaluated on an $(n+1)$-string of inner faces
      $T_0 \hookrightarrow T_1 
      \hookrightarrow T_2 
      \hookrightarrow \cdots
      \hookrightarrow T_{n+1}$
      is given by a point $\**$
      if $T_{n+1} \simeq T$
      and by $\emptyset$ otherwise.
      The evaluation of $W(T)(\vect C)$ follows.
\end{proof}



















\bibliography{biblio3}{}
\bibliographystyle{amsalpha2}



\end{document}

