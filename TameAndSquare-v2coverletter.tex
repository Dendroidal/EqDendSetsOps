\documentclass[a4paper,10pt
%,draft
]{article}%



% ---- Commands on draft --------

\usepackage[dvipsnames]{xcolor}% adds colors
\usepackage{ifdraft}
\ifdraft{
	\color[RGB]{63,63,63}
	\pagecolor[RGB]{220,220,204}
	\usepackage[notref]{showkeys}
	\usepackage{todonotes}
}
%{
%	\usepackage[disable]{todonotes}
%}



\pdfcompresslevel=0
\pdfobjcompresslevel=0

\usepackage{xr-hyper}
\usepackage[pagebackref, colorlinks, citecolor=PineGreen, linkcolor=PineGreen]{hyperref}
\hypersetup{
	final,
	pdftitle={Edits to ``Equivariant dendroidal sets and simplicial operads''},
	pdfauthor={Bonventre, P. and Pereira, L. A.},
	linktoc=page
}

\externaldocument[TAS-]{TameAndSquare_v2} % cite using names from other half
\externaldocument[OC-]{OneColor} % cite using names from other half
\externaldocument{AllColors-v2}


\usepackage{amsmath, amsthm}% {amsfonts, amssymb}


% ------ New Characters --------------------------------------

\usepackage[latin1]{inputenc}%
\usepackage{MnSymbol}


\usepackage[normalem]{ulem}% underlining
%\usepackage{dsfont}% double strike-through
\usepackage{bbm}% more bb


\DeclareMathAlphabet\mathbb{U}{msb}{m}{n}
\usepackage{upgreek}
\usepackage{mathrsfs}

\usepackage[normalem]{ulem}


%----- Enumerate ---------------------------------------------
\usepackage[inline,shortlabels]{enumitem}% % can use \begin{enumerate*} for inparaenum
\setenumerate{label=(\roman*)}



% ---------- Page Typesetting ----------
\usepackage[final]{microtype}
\usepackage{relsize}
\usepackage{geometry}


%-------- Tikz ---------------------------

\usepackage{tikz}%
\usetikzlibrary{matrix,arrows,decorations.pathmorphing,
cd,patterns,calc}
\tikzset{%
  treenode/.style = {shape=rectangle, rounded corners,%
                     draw, align=center,%
                     top color=white, bottom color=blue!20},%
  root/.style     = {treenode, font=\Large, bottom color=red!30},%
  env/.style      = {treenode, font=\ttfamily\normalsize},%
  dummy/.style    = {circle,draw,inner sep=0pt,minimum size=2mm}%
}%

\usetikzlibrary[decorations.pathreplacing]


% ----- Labels Changed? --------

\makeatletter

\def\@testdef #1#2#3{%
  \def\reserved@a{#3}\expandafter \ifx \csname #1@#2\endcsname
  \reserved@a  \else
  \typeout{^^Jlabel #2 changed:^^J%
    \meaning\reserved@a^^J%
    \expandafter\meaning\csname #1@#2\endcsname^^J}%
  \@tempswatrue \fi}

\makeatother


%%%%%%%%%%%%%%%%%%%%%%%%% INTERNAL REFERENCES %%%%%%%%%%%%%%%%%%%%%%%%%%%%%%%%%%%

\numberwithin{equation}{section} 
\numberwithin{figure}{section}

\usepackage{mathtools}
\mathtoolsset{showonlyrefs,showmanualtags} % Only number equations which are referenced with eqref


% ------- New Theorems/ Definition/ Names-----------------------

 % \theoremstyle{plain} % bold name, italic text
\newtheorem{theorem}[equation]{Theorem}%
\newtheorem*{theorem*}{Theorem}%
\newtheorem{lemma}[equation]{Lemma}%
\newtheorem{proposition}[equation]{Proposition}%
\newtheorem{corollary}[equation]{Corollary}%
\newtheorem{conjecture}[equation]{Conjecture}%
\newtheorem*{conjecture*}{Conjecture}%
\newtheorem{claim}[equation]{Claim}%

%%%%%% Fancy Numbering for Theorems
\newtheorem{innercustomgeneric}{\customgenericname}
\providecommand{\customgenericname}{}
\newcommand{\newcustomtheorem}[2]{%
  \newenvironment{#1}[1]
  {%
   \renewcommand\customgenericname{#2}%
   \renewcommand\theinnercustomgeneric{##1}%
   \innercustomgeneric
  }
  {\endinnercustomgeneric}
}

\newcustomtheorem{customthm}{Theorem}
\newcustomtheorem{customcor}{Corollary}
%%%%%%%%%%%%%

\theoremstyle{definition} % bold name, plain text
\newtheorem{definition}[equation]{Definition}%
\newtheorem*{definition*}{Definition}%
\newtheorem{example}[equation]{Example}%
\newtheorem{remark}[equation]{Remark}%
\newtheorem{notation}[equation]{Notation}%
\newtheorem{convention}[equation]{Convention}%
\newtheorem{assumption}[equation]{Assumption}%
\newtheorem{exercise}{Exercise}%


% %%%%%%%%%%%%%%%%%%%%%%%%%%%%%%%%%%%%%%%%%%%%%%%%%%%%%%%%%%%%%%%%%%%%%%%%%%%%%%%%
% ------------------------------ COMMANDS ------------------------------

% ---------- macros

\newcommand{\set}[1]{\left\{#1\right\}}%
\newcommand{\sets}[2]{\left\{ #1 \;|\; #2\right\}}%
\newcommand{\longto}{\longrightarrow}%
\newcommand{\into}{\hookrightarrow}%
\newcommand{\onto}{\twoheadrightarrow}%

\usepackage{harpoon}
\newcommand{\vect}[1]{\text{\overrightharp{\ensuremath{#1}}}}


% ---------- operators

\newcommand{\Sym}{\ensuremath{\mathsf{Sym}}}%
\newcommand{\Fin}{\mathsf{F}}%
\newcommand{\Set}{\ensuremath{\mathsf{Set}}}
\newcommand{\Top}{\ensuremath{\mathsf{Top}}}
\newcommand{\sSet}{\ensuremath{\mathsf{sSet}}}%
\newcommand{\Cat}{\mathsf{Cat}}
\newcommand{\sCat}{\mathsf{sCat}}
\newcommand{\Op}{\mathsf{Op}}%
\newcommand{\sOp}{\ensuremath{\mathsf{sOp}}}%
\newcommand{\fgt}{\ensuremath{\mathsf{fgt}}}%
\newcommand{\dSet}{\mathsf{dSet}}
\newcommand{\Fun}{\mathsf{Fun}}
\newcommand{\Fib}{\mathsf{Fib}}
\newcommand{\Alg}{\mathsf{Alg}}
\newcommand{\Kl}{\mathsf{Kl}}



\DeclareMathOperator{\hocmp}{hocmp}%
\DeclareMathOperator{\cmp}{cmp}%
\DeclareMathOperator{\hofiber}{hofiber}%
\DeclareMathOperator{\fiber}{fiber}%
\DeclareMathOperator{\hocofiber}{hocof}%
\DeclareMathOperator{\hocof}{hocof}%
\DeclareMathOperator{\holim}{holim}%
\DeclareMathOperator{\hocolim}{hocolim}%
\DeclareMathOperator{\colim}{colim}%
\DeclareMathOperator{\Lan}{Lan}%
\DeclareMathOperator{\Ran}{Ran}%
\DeclareMathOperator{\Map}{Map}%
\DeclareMathOperator{\Id}{Id}%
\DeclareMathOperator{\mlf}{mlf}%
\DeclareMathOperator{\Hom}{Hom}%
\DeclareMathOperator{\Ho}{Ho}
\DeclareMathOperator{\Aut}{Aut}%
\DeclareMathOperator{\Stab}{Stab}
\DeclareMathOperator{\Iso}{Iso}
\DeclareMathOperator{\Ob}{Ob}

% ---------- shortcuts

\newcommand{\F}{\ensuremath{\mathcal F}}
\newcommand{\V}{\ensuremath{\mathcal V}}
\newcommand{\Q}{\ensuremath{\mathcal Q}}
\renewcommand{\O}{\ensuremath{\mathcal O}}
\renewcommand{\P}{\ensuremath{\mathcal P}}
\newcommand{\C}{\ensuremath{\mathcal C}}
\newcommand{\A}{\ensuremath{\mathcal A}}
\newcommand{\G}{\ensuremath{\mathcal G}}

\newcommand{\del}{\partial}%

\newcommand{\ki}{\chi}
\newcommand{\ksi}{\xi}
\newcommand{\Ksi}{\Xi}

\newcommand{\lltimes}{\underline{\ltimes}}

% detecting $\V$-categories:

\newcommand{\I}{\mathbb I}
\newcommand{\J}{\mathbb J}
\newcommand{\1}{\ensuremath{\mathbbm 1}}%{\ensuremath{\mathbb{id}}} %\eta

% lazy shortcuts

\newcommand{\SC}{\Sigma_{\mathfrak C}}
\newcommand{\OC}{\Omega_{\mathfrak C}}
\newcommand{\UV}{\underline{\mathcal V}}
\newcommand{\UC}{\underline{\mathfrak C}}










% %%%%%%%%%%%%%%%%%%%%%%%%%%%%%%%%%%%%%%%%%%%%%%%%%%%%%%%%%%%%%%%%%%%%%%%%%%%%%%%%%%%%%%%%%%%%%%%%%%%%
% ------------------------------ MAIN BODY ------------------------------

% ---- Title --------

\title{Edits to ``Equivariant dendroidal sets and simplicial operads''}

\author{Peter Bonventre, Lu\'is A. Pereira}%

% \date{\today}


\begin{document} 
  
\maketitle
 



\section{General comments}

The referee report included 40 line-by-line comments, 
all of which consisted of
typos,
suggestions for minor changes to wording,
or requests for a little clarification or extra detail in proofs.
All of these have been addressed in the following sections.
The numbering of the items below refer to the numbering in the referee report,
while the numbering/citations for referenced results/articles refer to the numbering/naming in the new version of the article.

      

\section{Non-changes / Comments}

\begin{enumerate}
\item[(10)] Defining the homotopy colimit of a diagram to be the colimit of the cofibrant replacement of the diagram
        in the projective model structure, this equivalence is standard.
        (We note that any particular construction of the homotopy colimit will \textit{not} necessarily satisfy this equivalence, as these normally depend on the starting diagrams being levelwise cofibrant. However, we are not using any specific construction of these homotopy colimits.)
        % If $\mathcal D$-shaped colimits are homotopy colimits, then $F \simeq F'$ implies $QF \simeq QF'$,
        % the latter of which are levelwise cofibrant, so
        % $\colim F \simeq \mathop{coholim} F \colim QF \simeq \colim QF' \simeq \colim F'$.
        % Conversely, since cofibrant replacements in the projective model structure in functor categories 
        % are in particular levelwise weak equivalences,
        % we have that filtered colimits are equivalence to the colimit of their cofibrant-replaced diagram.
\item[(15)] Regarding the comparison of $\Omega[K,T]$ and $\Omega[T] \otimes_{\mathfrak C_\bullet} K$,
        we further specified the reference, from [CM13b, \S 7.1] to the particular pushout square (7.1.5) in their definition.
        When $X = \Omega[T]$, their pushout square and ours, \eqref{TAS-PREOPTENS EQ}, agree (though the diagonal entries are swapped).
\item[(25)] An update to [BPb] should include cofibrant generation in the statement of Theorem A.
        This is established in Propositions 3.2, 3.10, 3.18 of [BPb].
\item[(42)] Directly before Definition \ref{TAS-CHAREDGE DEF}, we note that the descending chain condition will be discussed in Remark \ref{TAS-DCC REM}.
        We feel that the statement of the descending chain condition is more useful in the context of this remark,
        where we compare it to the assumption it replaces,
        than in the statement of Definition \ref{TAS-CHAREDGE DEF}, which is already quite involved.
\item[(35)] No change requested.
        As a counter-example to the use of [CM13b, Lemma 3.4] in [CM13b, Lemma 3.5], using their language we have that
        if $q \colon \Omega[T] \to N_d(P[f])^{(k+1)}$ is in $U_{n+1}^{(k)}$, then the composite
        $s \colon \Omega[\mathsf{lr}(T)] \to \Omega[T] \xrightarrow{q} N_d(P[f])^{(k+1)}$
        also does not factor through $A_n$,
        but $s$ does not satisfy the hypotheses of [CM13b, Lemma 3.4].
        We note that in our language, $s$ is not an elementary dendrex. 
\end{enumerate}

\section{Changes}

\begin{enumerate}
\item[(2)] Added a footnote to clarify the term ``strictly'' when describing the unique factorization of maps in $\Omega$ above Notation 2.2.
\item[(4)] Clarified the discussion of componentwise planar quotient maps in Remark \ref{TAS-PULLBACK_REM}.
\item[(5)] Added a citation to [Per18, Prop 8.8 and Thm 8.22] before the statement of Theorem \ref{TAS-DSETGMOD THM}.
        We feel that the actual statement of Proposition 8.8 would be more confusing then helpful, as it refers to the technical $J$-fibrant condition, as opposed to the more refined conditions described here.
\item[(7)] Clarified in Notation \ref{TAS-SDSETEVAL_NOT} that presheaves under $c_!$ are constant along the simplicial direction.
\item[(8)] Added to the proof of Theorem \ref{TAS-JB_THM} appropriate citations in Hirschhorn showing that the two starting Reedy model structures are left proper and cellular.
\item[(9)] Added a definition of the pushout-product $\square$ to the end of Notation \ref{TAS-SDSETEVAL_NOT}.
\item[(13)] To clarify that in general the natural map in \ref{TAS-SEGCOLCHAR_REM} is not a weak equivalence, the equivalent conditions from Definition \ref{TAS-SEGCOLCHAR DEF} have been pulled into a separate remark.
\item[(14)] Added brief definition of a precategory to Remark \ref{TAS-HOISNERVENON REM}.
\item[(17)] Added an introduction the fibered pushout product notation $\square_{\mathfrak C_\bullet}$ in Definition \ref{TAS-FIBERTENSOR_DEF}
\item[(19)] Ensured that weak equivalences in the joint model structure are always referred to as \textit{joint equivalences}.
        % Changed ``weak equivalence'' to ``joint equivalence'' in: Lemma \ref{TAS-TAMETRIVFIB LEM}, proof of Theorem \ref{TAS-TAMEMS_THM}...
        % Changed "joint/complete equivalence" to "joint equivalence" in: proof of Lemma \ref{TAS-SLIMOD LEM}, Remark \ref{TAS-TC2TC3REP REM}, proof of Lemma \ref{TAS-TAMETRIVFIB LEM}...
\item[(23)] Replaced all instances of ``$\mathfrak C$-signatures'' with ``$\mathfrak C$-profiles'', adding a note before Definition \ref{TAS-SSYM DEF} that the former nomenclature was used in earlier versions of [BPa] and [BPb].
        (Both can be found in the literature, though ``profile'' is more common.)
\item[(26)] Added Proposition \ref{TAS-AUXSYM PROP} describing the auxillary model structure on $\mathsf{sSym}^G_{\mathfrak C}$ and the family $\mathcal F^\Gamma$ more explicitly.
\item[(27)] Added Remark \ref{TAS-MSLIST_REM} to \S 1.1 to list each of the model structures recalled in this paper, as well as the location of their definitions.
\item[(28)] Equations 4.49 and 5.14 have been converted to Definition \ref{TAS-WU_DEF} and Example \ref{TAS-ALTTREES_EX} to alleviate the numbering pressure on these wide equations.
\item[(32)] Expanded Remark \ref{TAS-PUSHOPPRRST REM} to indicate how \eqref{TAS-PUSHOPPR EQ} implies the various statements therein.
\item[(39)] ``Parallel'' simply meant ``with the same source and target'', though we now find the emphasis confusing; we have removed the word ``parallel'' and replaced $\rightrightarrows$ with $\to$ in Definition \ref{TAS-HOEQUIVS DEF}.         
\item[(40)] Updated references, and updated the cited numbering to match the updated papers.
\end{enumerate}



\section{Minor changes and typos}
 
\begin{enumerate}
\item[(1)] Added reference to [BP20] before Table 1.
\item[(3)] Corrected ``$\phi \colon U \to V$'' to ``$\phi \colon S \to T$'' in the statements of Propositions \ref{TAS-NP_TREEFACT_PROP} and \ref{TAS-TREEFACT_PROP}.
\item[(6)] Corrected the typo in Diagram \eqref{TAS-TAUFUNCTS EQ} by replacing $\iota^{\**}$ with $\iota_!$.
\item[(11)] Clarified that $U$ is not a linear tree by correcting the typo and rearranging slightly.
\item[(12)] Corrected the typo, so $X \in \mathsf{PreOp}^G$ instead of $\mathsf{sdSet}^G$.
\item[(16)] Switched the order in which we introduce $X \to Y$ and $K$.
\item[(18)] Replaced ``$1 \leq n$'' with ``$n \geq 1$
\item[(20)] Rearranged first sentence in Lemma \ref{TAS-SLIMOD LEM}.
\item[(21)] Removed second ``for'' in second sentence of the proof of Lemma \ref{TAS-SLIMOD LEM}.
\item[(22)] Replaced ``a left $G$-action of $\mathfrak C$-profiles'' with ``a left $G$-action on the set of $\mathfrak C$-profiles''
\item[(24)] Replaced ``if must be'' with ``we must have'' in sentence in second paragraph after Remark \ref{TAS-SYMGCPRESH REM}.
\item[(29)] Removed extraneous ``to'' in the last sentence of Notation 5.4.
\item[(31)] Modified phrasing, replacing ``it is'' with other language
\item[(33)] Replaced ``an unique'' with ``a unique'' in Remark \ref{TAS-ELEMLABEL REM}.
\item[(34)] Corrected typo in Corollary \ref{TAS-ISODIFCL COR}.
\item[(36)] Removed doubled ``the'' in Remark \ref{TAS-WIS_FAMILY_REM}.
\item[(37)] Removed punctuation from any diagrams, and changed preceding sentences accordingly.
\item[(38)] Replaced ``a $\F$-'' with ``an $\F$'' throughout the paper.
\end{enumerate}

\section{Other changes}
\begin{itemize}
\item Updated the wording in Remark \ref{TAS-SYMGCPRESH REM}.
\end{itemize}




\end{document} 




%%% Local Variables:
%%% mode: latex
%%% TeX-master: t
%%% End:
