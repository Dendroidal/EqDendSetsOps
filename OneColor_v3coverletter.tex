\documentclass[a4paper,10pt
%,draft
]{article}%



% ---- Commands on draft --------

\usepackage[dvipsnames]{xcolor}% adds colors
\usepackage{ifdraft}
\ifdraft{
	\color[RGB]{63,63,63}
	\pagecolor[RGB]{220,220,204}
	\usepackage[notref]{showkeys}
	\usepackage{todonotes}
}
%{
%	\usepackage[disable]{todonotes}
%}



\pdfcompresslevel=0
\pdfobjcompresslevel=0

\usepackage{xr-hyper}
\usepackage[pagebackref, colorlinks, citecolor=PineGreen, linkcolor=PineGreen]{hyperref}
\hypersetup{
	final,
	pdftitle={Edits to ``Homotopy theory of equivariant operads with fixed colors v2''},
	pdfauthor={Bonventre, P. and Pereira, L. A.},
	linktoc=page
}

\externaldocument[TAS-]{TameAndSquare_v2} % cite using names from other half
\externaldocument[OC-]{OneColor_v3} % cite using names from other half
\externaldocument{AllColors-v2}


\usepackage{amsmath, amsthm}% {amsfonts, amssymb}


% ------ New Characters --------------------------------------

\usepackage[latin1]{inputenc}%
\usepackage{MnSymbol}


\usepackage[normalem]{ulem}% underlining
%\usepackage{dsfont}% double strike-through
\usepackage{bbm}% more bb


\DeclareMathAlphabet\mathbb{U}{msb}{m}{n}
\usepackage{upgreek}
\usepackage{mathrsfs}

\usepackage[normalem]{ulem}


%----- Enumerate ---------------------------------------------
\usepackage[inline,shortlabels]{enumitem}% % can use \begin{enumerate*} for inparaenum
\setenumerate{label=(\roman*)}



% ---------- Page Typesetting ----------
\usepackage[final]{microtype}
\usepackage{relsize}
\usepackage{geometry}


%-------- Tikz ---------------------------

\usepackage{tikz}%
\usetikzlibrary{matrix,arrows,decorations.pathmorphing,
cd,patterns,calc}
\tikzset{%
  treenode/.style = {shape=rectangle, rounded corners,%
                     draw, align=center,%
                     top color=white, bottom color=blue!20},%
  root/.style     = {treenode, font=\Large, bottom color=red!30},%
  env/.style      = {treenode, font=\ttfamily\normalsize},%
  dummy/.style    = {circle,draw,inner sep=0pt,minimum size=2mm}%
}%

\usetikzlibrary[decorations.pathreplacing]


% ----- Labels Changed? --------

\makeatletter

\def\@testdef #1#2#3{%
  \def\reserved@a{#3}\expandafter \ifx \csname #1@#2\endcsname
  \reserved@a  \else
  \typeout{^^Jlabel #2 changed:^^J%
    \meaning\reserved@a^^J%
    \expandafter\meaning\csname #1@#2\endcsname^^J}%
  \@tempswatrue \fi}

\makeatother


%%%%%%%%%%%%%%%%%%%%%%%%% INTERNAL REFERENCES %%%%%%%%%%%%%%%%%%%%%%%%%%%%%%%%%%%

\numberwithin{equation}{section} 
\numberwithin{figure}{section}

\usepackage{mathtools}
\mathtoolsset{showonlyrefs,showmanualtags} % Only number equations which are referenced with eqref


% ------- New Theorems/ Definition/ Names-----------------------

 % \theoremstyle{plain} % bold name, italic text
\newtheorem{theorem}[equation]{Theorem}%
\newtheorem*{theorem*}{Theorem}%
\newtheorem{lemma}[equation]{Lemma}%
\newtheorem{proposition}[equation]{Proposition}%
\newtheorem{corollary}[equation]{Corollary}%
\newtheorem{conjecture}[equation]{Conjecture}%
\newtheorem*{conjecture*}{Conjecture}%
\newtheorem{claim}[equation]{Claim}%

%%%%%% Fancy Numbering for Theorems
\newtheorem{innercustomgeneric}{\customgenericname}
\providecommand{\customgenericname}{}
\newcommand{\newcustomtheorem}[2]{%
  \newenvironment{#1}[1]
  {%
   \renewcommand\customgenericname{#2}%
   \renewcommand\theinnercustomgeneric{##1}%
   \innercustomgeneric
  }
  {\endinnercustomgeneric}
}

\newcustomtheorem{customthm}{Theorem}
\newcustomtheorem{customcor}{Corollary}
%%%%%%%%%%%%%

\theoremstyle{definition} % bold name, plain text
\newtheorem{definition}[equation]{Definition}%
\newtheorem*{definition*}{Definition}%
\newtheorem{example}[equation]{Example}%
\newtheorem{remark}[equation]{Remark}%
\newtheorem{notation}[equation]{Notation}%
\newtheorem{convention}[equation]{Convention}%
\newtheorem{assumption}[equation]{Assumption}%
\newtheorem{exercise}{Exercise}%


% %%%%%%%%%%%%%%%%%%%%%%%%%%%%%%%%%%%%%%%%%%%%%%%%%%%%%%%%%%%%%%%%%%%%%%%%%%%%%%%%
% ------------------------------ COMMANDS ------------------------------

% ---------- macros

\newcommand{\set}[1]{\left\{#1\right\}}%
\newcommand{\sets}[2]{\left\{ #1 \;|\; #2\right\}}%
\newcommand{\longto}{\longrightarrow}%
\newcommand{\into}{\hookrightarrow}%
\newcommand{\onto}{\twoheadrightarrow}%

\usepackage{harpoon}
\newcommand{\vect}[1]{\text{\overrightharp{\ensuremath{#1}}}}


% ---------- operators

\newcommand{\Sym}{\ensuremath{\mathsf{Sym}}}%
\newcommand{\Fin}{\mathsf{F}}%
\newcommand{\Set}{\ensuremath{\mathsf{Set}}}
\newcommand{\Top}{\ensuremath{\mathsf{Top}}}
\newcommand{\sSet}{\ensuremath{\mathsf{sSet}}}%
\newcommand{\Cat}{\mathsf{Cat}}
\newcommand{\sCat}{\mathsf{sCat}}
\newcommand{\Op}{\mathsf{Op}}%
\newcommand{\sOp}{\ensuremath{\mathsf{sOp}}}%
\newcommand{\fgt}{\ensuremath{\mathsf{fgt}}}%
\newcommand{\dSet}{\mathsf{dSet}}
\newcommand{\Fun}{\mathsf{Fun}}
\newcommand{\Fib}{\mathsf{Fib}}
\newcommand{\Alg}{\mathsf{Alg}}
\newcommand{\Kl}{\mathsf{Kl}}



\DeclareMathOperator{\hocmp}{hocmp}%
\DeclareMathOperator{\cmp}{cmp}%
\DeclareMathOperator{\hofiber}{hofiber}%
\DeclareMathOperator{\fiber}{fiber}%
\DeclareMathOperator{\hocofiber}{hocof}%
\DeclareMathOperator{\hocof}{hocof}%
\DeclareMathOperator{\holim}{holim}%
\DeclareMathOperator{\hocolim}{hocolim}%
\DeclareMathOperator{\colim}{colim}%
\DeclareMathOperator{\Lan}{Lan}%
\DeclareMathOperator{\Ran}{Ran}%
\DeclareMathOperator{\Map}{Map}%
\DeclareMathOperator{\Id}{Id}%
\DeclareMathOperator{\mlf}{mlf}%
\DeclareMathOperator{\Hom}{Hom}%
\DeclareMathOperator{\Ho}{Ho}
\DeclareMathOperator{\Aut}{Aut}%
\DeclareMathOperator{\Stab}{Stab}
\DeclareMathOperator{\Iso}{Iso}
\DeclareMathOperator{\Ob}{Ob}

% ---------- shortcuts

\newcommand{\F}{\ensuremath{\mathcal F}}
\newcommand{\V}{\ensuremath{\mathcal V}}
\newcommand{\Q}{\ensuremath{\mathcal Q}}
\renewcommand{\O}{\ensuremath{\mathcal O}}
\renewcommand{\P}{\ensuremath{\mathcal P}}
\newcommand{\C}{\ensuremath{\mathcal C}}
\newcommand{\A}{\ensuremath{\mathcal A}}
\newcommand{\G}{\ensuremath{\mathcal G}}

\newcommand{\del}{\partial}%

\newcommand{\ki}{\chi}
\newcommand{\ksi}{\xi}
\newcommand{\Ksi}{\Xi}

\newcommand{\lltimes}{\underline{\ltimes}}

% detecting $\V$-categories:

\newcommand{\I}{\mathbb I}
\newcommand{\J}{\mathbb J}
\newcommand{\1}{\ensuremath{\mathbbm 1}}%{\ensuremath{\mathbb{id}}} %\eta

% lazy shortcuts

\newcommand{\SC}{\Sigma_{\mathfrak C}}
\newcommand{\OC}{\Omega_{\mathfrak C}}
\newcommand{\UV}{\underline{\mathcal V}}
\newcommand{\UC}{\underline{\mathfrak C}}










% %%%%%%%%%%%%%%%%%%%%%%%%%%%%%%%%%%%%%%%%%%%%%%%%%%%%%%%%%%%%%%%%%%%%%%%%%%%%%%%%%%%%%%%%%%%%%%%%%%%%
% ------------------------------ MAIN BODY ------------------------------

% ---- Title --------

\title{Edits to ``Homotopy theory of equivariant operads with fixed colors v2''}

\author{Peter Bonventre, Lu\'is A. Pereira}%

% \date{\today}


\begin{document} 
  
\maketitle


% \subsection*{Editor comments:}

There were three remaining comments from the editors and referee regarding our submission.

% \begin{enumerate}
% \item[(0)] Members of TJM editorial board suggest that you add one or two sentences in the introduction to better explain your motivation, especially in relation with the construction of a Quillen equivalence between a model category of equivariant dendroidal sets and a model category of genuine equivariant operads with varying colors, which is an equivariant counterpart of results obtained by Cisinski and Moerdijk.
% \item[(1)] It is arguable, but in Definition 2.22 (line 1), you have written ``Let $\pi \colon C \to B$, $\pi' \colon D \to B$ be functors \dots''.
%         The added symbol `` ' `` in the notation of the second functor  ``$\pi '$'' should be removed if you adopt the convention to denote all generic fibration by the same notation ``$\pi$'' (which is the case in the equation satisfied by L and R in Definition 2.22).

%         By the way, it seems that the referee has been confused by the phrasing
%         ``Let $\pi \colon C \to B$, $\pi \colon D \to B$ be functors with a common target.''
%         You should actually specify or at least indicate your convention to adopt the same notation for all fibrations, e.g. you may write
%         ``Let C and D be categories, endowed with a functor with a common target $\pi \colon C \to B$, $\pi \colon D \to B$
%         (and which we denote by $\pi$ in both cases by an abuse of notation)''
%         or something similar.
% \item[(2)] The referee suggested that
%         ``the statement of Prop 4.21 can be made more clear. Where below are the model structures? See also statement of 4.25.''
%         You have carried out the requested revision in the statement of Prop. 4.21, but have missed the statement of Prop. 4.25, where the same revision has to be done.
% \end{enumerate}

% \subsection*{Responses}

We have incorporated these suggestions as below:

\begin{enumerate}
\item[(0)] Added a sentence to the top of page 3 stating explicitly that results in [BPb] depend on the fixed color results already being established. \\
        These dependencies, which appear throughout [BPb, Section 3], feature in the verification of all the conditions of [Hov99, Theorem 2.1.19], and are the motivation for this article as a particular piece of our program.
        The motivation for the program itself is articulated more completely in the introduction to the concluding paper [BPa].
\item[(1)] This extraneous `` ' `` is a typo from an earlier edit; we have removed it. 
        Additionally, we've updated the language in Definition \ref{OC-FIBADJ DEF} to increase clarity.
\item[(2)] Replaced ``below'' with ``in \eqref{OC-RESGEN_EQ}'' in the statement of Proposition \ref{OC-RESGEN PROP}.
\end{enumerate}





\end{document} 




%%% Local Variables:
%%% mode: latex
%%% TeX-master: t
%%% End:
