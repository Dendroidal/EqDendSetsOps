\documentclass[a4paper,10pt
%,draft
%, final
]{article}%

\pdfcompresslevel=0
\pdfobjcompresslevel=0

\usepackage[hidelinks]{hyperref}
\hypersetup{
  % colorlinks,
  final,
  pdftitle={Equivariant Dendroidal Segal Spaces},
  pdfauthor={Bonventre, P. and Pereira, L. A.},
  % pdfsubject={Your subject here},
  % pdfkeywords={keyword1, keyword2},
  linktoc=page
}
%\usepackage[open=false]{bookmark}

\input{commands.tex}%


%-------- Tikz ---------------------------

\usepackage{tikz}%
\usetikzlibrary{matrix,arrows,decorations.pathmorphing,
cd,patterns,calc}
\tikzset{%
  treenode/.style = {shape=rectangle, rounded corners,%
                     draw, align=center,%
                     top color=white, bottom color=blue!20},%
  root/.style     = {treenode, font=\Large, bottom color=red!30},%
  env/.style      = {treenode, font=\ttfamily\normalsize},%
  dummy/.style    = {circle,draw,inner sep=0pt,minimum size=2mm}%
}%

\usetikzlibrary[decorations.pathreplacing]



% ---- Commands on draft --------

\usepackage{ifdraft}
\ifdraft{
  \color[RGB]{63,63,63}
  % \pagecolor[rgb]{0.5,0.5,0.5}
  \pagecolor[RGB]{220,220,204}
  % \color[rgb]{1,1,1}
}


\usepackage[draft]{showkeys}
\usepackage{todonotes}%[obeyDraft]


% ----- Labels Changed? --------

\makeatletter

\def\@testdef #1#2#3{%
  \def\reserved@a{#3}\expandafter \ifx \csname #1@#2\endcsname
  \reserved@a  \else
  \typeout{^^Jlabel #2 changed:^^J%
    \meaning\reserved@a^^J%
    \expandafter\meaning\csname #1@#2\endcsname^^J}%
  \@tempswatrue \fi}

\makeatother


% ---- Commands --------

% new symbols

\newcommand{\mycircled}[2][none]{%
  \mathbin{
    \tikz[baseline=(a.base)]\node[draw,circle,inner sep=-1.5pt, outer sep=0pt,fill=#1](a){\ensuremath #2\strut};
cf.  }
}
\newcommand{\owr}{\mycircled{\wr}}

% replace symbols

\renewcommand{\hat}{\widehat}

% random

\renewcommand{\F}{\mathcal F}
\newcommand{\Q}{\mathcal Q}

\newcommand{\lltimes}{\underline{\ltimes}}


% detecting $\V$-categories:

\newcommand{\I}{\mathbb I}
\newcommand{\J}{\mathbb J}
\renewcommand{\1}{\eta}%{\ensuremath{\mathbb{id}}}

% lazy shortcuts

\newcommand{\SC}{\Sigma_{\mathfrak C}}
\newcommand{\OC}{\Omega_{\mathfrak C}}

\newcommand{\UV}{\underline{\mathcal V}}
\newcommand{\UC}{\underline{\mathfrak C}}


% ---- Title --------

\title{Equivariant Segal operads, simplicial operads, and dendroidal sets}

\author{Peter Bonventre, Lu\'is A. Pereira}%

\date{\today}


% ---- Document body --------

\begin{document}

\maketitle

\begin{abstract}
      Things and stuff
\end{abstract}

\tableofcontents



\subsection{Colored trees and a monadic description of equivariant operads}
\label{COMEGA_SEC}





\begin{remark}
      For much of this section, $\mathsf F$ will often denote not just $\mathsf F$ itself, but also any of the subcategories
      $\Sigma$, $\mathsf F_i$, or $\mathsf F_s$.
      We will specify precisely when one of the categories is needed.
\end{remark}

\begin{definition}
      Given a category $\mathcal C$ over $\mathsf F$, let $\mathcal C_{\mathfrak C}$ denote the pullback
      \begin{equation}
            \begin{tikzcd}
                  \mathcal C_{\mathfrak C} \arrow[r] \arrow[d]
                  &
                  \mathsf F \wr \mathfrak C \arrow[d]
                  \\
                  \mathcal C \arrow[r]
                  &
                  \mathsf F
            \end{tikzcd}
      \end{equation}
\end{definition}

\begin{remark}
      We observe that the above definition does not conflict with our notation $\Sigma_{\mathfrak C}$ from the previous section,
      as $\SC$ is isomorphic to the pullback on the left below,
      where $E: \Sigma \to \mathsf F$ sends $\underline{n}$ to $\underline{n+1}$.
      % and the second arrow is the diagonal functor from Remark \ref{WR_DIAG_REM}. 
      \begin{equation}
            \label{COMEGA_B_EQ}
            \begin{tikzcd}
                  \SC \arrow[r, "E"] \arrow[d]
                  &
                  \mathsf F \wr \mathfrak C \arrow[d]
                  & %
                  \OC \arrow[d] \arrow[r, "E"]
                  &
                  \mathsf F \wr \mathfrak C \arrow[d]
                  & %
                  G^{op} \ltimes \OC \arrow[d] \arrow[r, "E"]
                  &
                  G^{op} \ltimes (\mathsf F \wr \mathfrak C) \arrow[d] \arrow[r]
                  &
                  \mathsf F \wr (G^{op} \ltimes \mathfrak C) \arrow[d]
                  \\
                  \Sigma \arrow[r, "E"]
                  &
                  \mathsf F
                  & %
                  \Omega \arrow[r, "E"]
                  &
                  \mathsf F
                  & %
                  G^{op} \times \Omega \arrow[r, "E"]
                  &
                  G^{op} \times \mathsf F \arrow[r]
                  &
                  \mathsf F \wr G^{op}
            \end{tikzcd}
      \end{equation}
\end{remark}

\begin{definition}
      Let $\Omega_{\mathfrak C}$ denote the middle pullback in \eqref{COMEGA_B_EQ},
      where $E: \Omega \to \mathsf F$ sends a tree $U$ to its set $E(U)$ of edges.
      We note that $G^{op} \ltimes \OC$ is given by the right pullback,
      as $G^{op} \ltimes (-)$ preserves pullbacks and the rightmost square is a pullback
      by Lemmas \ref{GL_PULL_LEM} and \ref{GD_PULL_LEM},
      and a similar pair of squares yields $G^{op} \ltimes \SC$. 

      We have a natural inclusion of categories $\Sigma_{\mathfrak C} \into \Omega_{\mathfrak C}$,
      and as such we will called elements of these categories
      \textit{colored corollas} and \textit{colored trees},
      and denote them by $(U,\mathfrak c)$, where $\mathfrak c: E(U) \to \mathfrak C$ is a map of sets,
      or just by $U$ if there is no confusion.
\end{definition}

Unpacking definitions (cf. Remark \ref{G_GR_REM}), we see that a map $(U, \mathfrak c) \to (V, \mathfrak d)$ in $G^{op} \ltimes \Omega_{\mathfrak C}$
is given by
a map $f: U \to V$ in $\Omega$ and an element $g\in G$,
such that $\mathfrak c(f(e)) = g.\mathfrak d(e)$ for all $e \in E(V)$.
\begin{equation}
      \begin{tikzcd}
            E(U) \arrow[d, "\mathfrak c"'] \arrow[r, "f"]
            &
            E(V) \arrow[d, "\mathfrak d"]
            \\
            \mathfrak C \arrow[r, "g"] 
            &
            \mathfrak C
      \end{tikzcd}
\end{equation}

In particular, we have maps of the form
\begin{equation}
      g = (id, g): g^{\**} U \to U
      % (U, E(U) \xrightarrow{\mathfrak c} \mathfrak C \xrightarrow{g \cdot} \mathfrak C)
      % \to
      % (U, E(U) \xrightarrow{\mathfrak c} \mathfrak C).
\end{equation}



Now, many of the natural functors around $\Omega$ and $\Sigma$ have generalizations to the colored setting,
which can be built through a straightforward use of the universal property of pullbacks.

\begin{definition}
      We have natural \textit{vertex} functors
      \begin{equation}
            \OC \to \Sigma \wr \SC,
            \qquad
            % G \ltimes \OC^{op} =
            % (G^{op} \ltimes \OC)^{op} \xrightarrow{V}
            % (G^{op} \ltimes (\Sigma \wr \SC))^{op} \to
            % (\Sigma \wr (G^{op} \ltimes \SC))^{op} =
            % (\Sigma \wr (G \ltimes \SC^{op})^{op})^{op},
            G^{op} \ltimes \OC \to G^{op} \ltimes (\Sigma \wr \SC) \to \Sigma \wr (G^{op} \wr \SC),
      \end{equation}
      as colorings of a tree restrict to colorings of each vertex corolla.

      Similarly, there is a \textit{leaf-root} functor
      $\mathsf{lr}: G \ltimes \Omega_{\mathfrak C}^{op} \to G \ltimes \Sigma_{\mathfrak C}^{op}$,
      where the coloring of $\mathsf{lr}(T)$ is a restrict of the coloring of $T$.
\end{definition}

\begin{remark}
      Equivalently, the first map is given by $\pi_{\mathfrak C}^{\**}$ applied to the original vertex functor
      $V: \Omega \to \Sigma \wr \Sigma$;
      see {\color{red} LATER SECTIONS}).
\end{remark}

With these definitions in place, we make the following definition.

\begin{definition}
      Given $X \in \Sym^{G, \mathfrak C}$, let $\mathbb F^{\mathfrak C} X$ denote the left Kan extension below.
      \begin{equation} 
            \begin{tikzcd}
                  G \ltimes \Omega_{\mathfrak C}^{op}
                  \arrow[d, "\mathsf{lr}"']
                  \arrow[r, "V"]
                  &
                  (\Sigma \wr (G \ltimes \Sigma_{\mathfrak C}^{op})^{op})^{op} \arrow[r, "\Sigma \wr X"]
                  \arrow[dl, Rightarrow]
                  &
                  (\Sigma \wr \V^{op})^{op} \arrow[r, "\otimes"]
                  &
                  \V
                  \\
                  G \ltimes \Sigma_{\mathfrak C}^{op} \arrow[urrr, "\Lan = \mathbb F^{\mathfrak C} X"']
            \end{tikzcd}
      \end{equation}
\end{definition}

\begin{theorem}
      \label{FC_MONAD_PROP}
      For all $G$-sets $\mathfrak C$,
      $\mathbb F^{\mathfrak C}$ is a monad on $\Sym^{G, \mathfrak C}(\V)$,
      with category of algebras $\Op^{G, \mathfrak C}(\V)$.
\end{theorem}

A complete analysis of this functor, including the proof of Theorem \ref{FC_MONAD_PROP},
can be found in Appendix \ref{MONAD_APDX}.

To end this section, we show that this definition is consistent,
in that when $\mathfrak C = \**$, this is isomorphic to the free single-colored operad monad from \cite[{Eq. (4.1)}]{BP_geo}:

% Note that when $\mathfrak C = \set{\**}$, we have
% $G \ltimes \Omega_{\mathfrak C} = \Omega \times G$ and 
% $G \ltimes \Sigma_{\mathfrak C} = \Sigma \times G$.


\begin{notation}[\cite{BP_geo}]
      Let $\mathbb F'$ denote the \textit{free single-colored operad monad} on $\V$, given by the left Kan extension of the following diagram.
      \begin{equation}
            \begin{tikzcd}
                  \Omega^{op}
                  \arrow[d, "\mathsf{lr}"']
                  \arrow[r, "V"]
                  &
                  (\Sigma \wr \Sigma)^{op} \arrow[r, "X"]
                  \arrow[dl, Rightarrow]
                  &
                  (\Sigma \wr \V^{op})^{op} \arrow[r, "\otimes"]
                  &
                  \V
                  \\
                  \Sigma^{op} \arrow[urrr, "\Lan = \mathbb F' X"']
            \end{tikzcd}
      \end{equation}
\end{notation}

First, one lemma.

\begin{notation}
      Given a functor $X : \C \to \mathsf{Fun}(\mathcal D, \V)$,
      let $\tilde X$ denote the adjoint $\tilde X: \C \times \mathcal D \to \V$.
\end{notation}

\begin{lemma}
      \label{SPAN_LAN_LEM}
      Conisder the two spans below.
      \begin{equation}
            \begin{tikzcd}
                  \C \arrow[d, "p"] \arrow[r, "X"]
                  &
                  \mathsf{Fun}(\mathcal D, \V)
                  &&
                  \C \times \mathcal D \arrow[d, "p \times \mathsf{id}"] \arrow[r, "\tilde X"]
                  &
                  \V
                  \\
                  \mathcal E
                  &
                  &&
                  \mathcal E \times \mathcal D
            \end{tikzcd}
      \end{equation}
      
      Then $\Lan_p X$ is adjoint to $\Lan_{p \times \mathsf{id}} \tilde X$. 
\end{lemma}
\begin{proof}
      Using the pointwise description of the Kan extension, we have
      \begin{align}
        \widetilde{\Lan_p X}(e,d)
        &= (\Lan_p X(e))(d)
          = \left(
          \colim\limits_{\substack{ \C \downarrow e \\ p(c) \to e}} X(c)
        \right)(d)
        = \colim\limits_{\substack{ \C \downarrow e \\ p(c) \to e}}(X(c)(d))
        = \colim\limits_{\substack{ \C \downarrow e \\ p(c) \to e}}(\tilde X(c,d))\\
        &= \colim\limits_{\substack{ \C \times \set{d} \downarrow (e,d) \\ p(c) \to e}}(\tilde X(c,d))
        \cong \colim\limits_{\substack{ \C \times \mathcal D \downarrow (e,d) \\ (p(c),d') \to (e,d)}}(\tilde X(c,d'))
        = \Lan_{p \times \mathsf{id}}\tilde X(c,d),
      \end{align}
      where the isomorphism holds by a straightforward finality argument.
      On maps, a similar argument holds.
\end{proof}

\begin{proposition}
      \label{TEST_PROP}
      $\mathbb F^{\set{\**}}$ is a monad, and moreover
      the category of $\mathbb F^{\set{\**}}$-algebras in $\mathsf{Fun}(G \times \Sigma^{op}, \V)$ is equivalent to
      the category of $\mathbb F'$-algebras in $\mathsf{Fun}(\Sigma^{op}, \V^G)$.
\end{proposition}
\begin{proof}
      Let $\tau: \tilde X \mapsto X$ denote the isomorphism of categories
      $\mathsf{Fun}(G \times \Sigma^{op}, \V) \xrightarrow{\tau} \mathsf{Fun}(\Sigma^{op}, \V^G)$.
      Then $\mathbb F^{\set{\**}} = \tau^{-1} \mathbb F' \tau$ by \ref{SPAN_LAN_LEM}, and so
      $\mathbb F^{\set{\**}}$ is in fact a monad, and
      the isomorphism lifts to an isomorphism on the category of algebras.
\end{proof}





% \subsection{Monad - placeholder subsection}

% \todo[inline]{merge with Luis work}

% We define the monad on $\mathsf{WSpan}^r_{\Cat \downarrow^r \mathsf F}(\Sigma^{op}, \V)$.
% \begin{definition}
%       Let $\mathcal A$ be in this category. This consists of the data of a category $\mathcal A$ and the following diagram.
%       \begin{equation}
%             \begin{tikzcd}
%                   |[alias = U]| \Sigma \arrow[dr, "E"']
%                   &
%                   \mathcal A \arrow[d, "\epsilon"'{name = V}, ""{name = D}] \arrow[l, "\lambda"'] \arrow[r]
%                   &
%                   |[alias = C]| \V \arrow[dl, "\varnothing"]
%                   \\
%                   & \mathsf F
%                   \arrow[Rightarrow, from = U, to = V, "\alpha"]
%                   \arrow[Rightarrow, from = C, to = D, "\beta"']
%             \end{tikzcd}
%       \end{equation}
%       Then we define $N(\mathcal A)$ to be the span given by the pullback
%       \begin{equation}
%             \begin{tikzcd}
%                   \Omega \wr \mathcal A \arrow[r] \arrow[d]
%                   &
%                   \mathsf F \wr \mathcal A \arrow[d] \arrow[r]
%                   &
%                   \mathsf F \wr \V \arrow[r, "\otimes"]
%                   &
%                   \V
%                   \\
%                   \Omega \arrow[d] \arrow[r, "V"]
%                   &
%                   \mathsf F \wr \Sigma
%                   \\
%                   \Sigma
%             \end{tikzcd}
%       \end{equation}
%       with the map over $\mathsf F$ given by the pushout in $\Fun(\Omega \wr \mathcal A, \mathsf F)$ of the natural transformations
%       \begin{equation}
%             E(T) \Leftarrow \coprod_{v \in V(T)} (\lambda(a_v)+1) \Rightarrow \coprod_{v \in V(T)}\epsilon(a_v)
%       \end{equation}
%       on the objects $(T, (a_v)) \in \Omega \wr \mathcal A$,
%       where the maps between these functors are given by the following diagram
%       (and in particular, we have a map $\lambda(a_v) + 1 \to E(T_v)$).
%       \begin{equation}
%             \begin{tikzcd}
%                   \Omega \wr \mathcal A \arrow[r, equal] \arrow[d, equal]
%                   &
%                   \Omega \wr \mathcal A \arrow[d] \arrow[r, "V"]
%                   &
%                   \mathsf F \wr \mathcal A \arrow[d, "\lambda"] \arrow[r, "\epsilon"]
%                   &
%                   \mathsf F \wr \mathsf F \arrow[r, "\amalg"] \arrow[d, equal]
%                   &
%                   \mathsf F \arrow[d, equal]
%                   \\
%                   \Omega \wr \mathcal A \arrow[d, equal] \arrow[r, "\lambda"]
%                   &
%                   \Omega \wr \Sigma \arrow[d, equal] \arrow[r, "V"]
%                   &
%                   \mathsf F \wr \Sigma \arrow[r, "{(-)+1}"] \arrow[ur, Rightarrow, "\alpha"] \arrow[drr, Rightarrow] 
%                   &
%                   \mathsf F \wr \mathsf F \arrow[r, "\amalg"]
%                   &
%                   \mathsf F \arrow[d, equal]
%                   \\
%                   \Omega \wr \mathcal A \arrow[r, "\lambda"]
%                   &
%                   \Omega \wr \Sigma \arrow[r, "\simeq"]
%                   &
%                   \Omega \arrow[rr, "E"] \arrow[u, "V"]
%                   &&
%                   \mathsf F
%             \end{tikzcd}
%       \end{equation}
      
%       \todo[inline]{come back: check that this gives maps over $\mathsf F$}
% \end{definition}

% \begin{lemma}
%       $\Omega \wr \mathcal A$ is the strict 2-pullback in $\Cat \downarrow^r \mathsf F$.
% \end{lemma}
% \begin{proof}
%       This follows immediately from Remark \ref{SPANLIM REM}.
% \end{proof}

% \begin{proposition}
%       $N$ is a monad on $\mathsf{WSpan}^r_{\Cat \downarrow^r \mathsf F}(\Sigma^{op}, \V)$.
% \end{proposition}
% \begin{proof}
%       \todo[inline]{come back}
% \end{proof}



% % \begin{proof}
% %       This follows from the work in {\color{red} LATER SECTIONS}:
% %       in particular, $\mathbb F^{\mathfrak C} = \Lan \circ N^{\mathfrak C} \circ \iota$,
% %       where $N^{\mathfrak C}$ is the operation on spans...
% %       The fact that all the necessarily diagrams, as shown in \cite{BP_geo}, commute/agree follows by breaking those diagrams in half:
% %       the left sides never mention $\V$, and we can show that the generalized diagrams commute/agree by applying $\pi^{\**}$ to the diagrams in \cite{BP_geo},
% %       where $\pi: \mathsf F \wr (G \ltimes \mathfrak C) \to \mathsf F$,
% %       and using the fact that this is a functor which preserves standard limits;
% %       the right sides only deal with the permutative structure of our underlying category $\V$.
% %       These are connected by trivially commuting information about our $(G, \mathfrak C)$-symmetric sequence $X$.
% %       \todo[inline]{come back: this doesn't actually say anything yet.}
% % \end{proof}





\bibliography{biblio-new}{}
\bibliographystyle{amsalpha2}



\end{document}


%%% Local Variables:
%%% mode: latex
%%% TeX-master: t
%%% End:
