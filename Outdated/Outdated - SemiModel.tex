\documentclass[a4paper,10pt
%,draft
%, final
]{article}%

\pdfcompresslevel=0
\pdfobjcompresslevel=0

\usepackage[hidelinks]{hyperref}
\hypersetup{
  % colorlinks,
  final,
  pdftitle={Equivariant Dendroidal Segal Spaces},
  pdfauthor={Bonventre, P. and Pereira, L. A.},
  % pdfsubject={Your subject here},
  % pdfkeywords={keyword1, keyword2},
  linktoc=page
}
%\usepackage[open=false]{bookmark}

\input{commands.tex}%


%-------- Tikz ---------------------------

\usepackage{tikz}%
\usetikzlibrary{matrix,arrows,decorations.pathmorphing,
cd,patterns,calc}
\tikzset{%
  treenode/.style = {shape=rectangle, rounded corners,%
                     draw, align=center,%
                     top color=white, bottom color=blue!20},%
  root/.style     = {treenode, font=\Large, bottom color=red!30},%
  env/.style      = {treenode, font=\ttfamily\normalsize},%
  dummy/.style    = {circle,draw,inner sep=0pt,minimum size=2mm}%
}%

\usetikzlibrary[decorations.pathreplacing]



% ---- Commands on draft --------

\usepackage{ifdraft}
\ifdraft{
  \color[RGB]{63,63,63}
  % \pagecolor[rgb]{0.5,0.5,0.5}
  \pagecolor[RGB]{220,220,204}
  % \color[rgb]{1,1,1}
}


\usepackage[draft]{showkeys}
\usepackage{todonotes}%[obeyDraft]


% ----- Labels Changed? --------

\makeatletter

\def\@testdef #1#2#3{%
  \def\reserved@a{#3}\expandafter \ifx \csname #1@#2\endcsname
  \reserved@a  \else
  \typeout{^^Jlabel #2 changed:^^J%
    \meaning\reserved@a^^J%
    \expandafter\meaning\csname #1@#2\endcsname^^J}%
  \@tempswatrue \fi}

\makeatother


% ---- Commands --------

% new symbols

\newcommand{\mycircled}[2][none]{%
  \mathbin{
    \tikz[baseline=(a.base)]\node[draw,circle,inner sep=-1.5pt, outer sep=0pt,fill=#1](a){\ensuremath #2\strut};
cf.  }
}
\newcommand{\owr}{\mycircled{\wr}}

% replace symbols

\renewcommand{\hat}{\widehat}

% random

\renewcommand{\F}{\mathcal F}
\newcommand{\Q}{\mathcal Q}

\newcommand{\lltimes}{\underline{\ltimes}}


% detecting $\V$-categories:

\newcommand{\I}{\mathbb I}
\newcommand{\J}{\mathbb J}
\renewcommand{\1}{\eta}%{\ensuremath{\mathbb{id}}}

% lazy shortcuts

\newcommand{\SC}{\Sigma_{\mathfrak C}}
\newcommand{\OC}{\Omega_{\mathfrak C}}

\newcommand{\UV}{\underline{\mathcal V}}
\newcommand{\UC}{\underline{\mathfrak C}}


% ---- Title --------

\title{Equivariant Segal operads, simplicial operads, and dendroidal sets}

\author{Peter Bonventre, Lu\'is A. Pereira}%

\date{\today}


% ---- Document body --------

\begin{document}

\maketitle

\begin{abstract}
      Things and stuff
\end{abstract}

\tableofcontents






\section{Semi-model structures}

Weaker versions of our main results hold in more generality.
Namely, if we only desire semi-model structures on $\Op^G_{\mathfrak C}$ and $\Op^G$, we may
remove the global monoid axiom condition.
\todo[inline]{is this entirely true?}


\subsection{Semi-model categories}

We recall the definition of a semi-model category, following \cite{Spi01,Wh16}.

\begin{definition}
	Given a class of maps $ I$ in a category $\mathcal D$, we let
	\begin{enumerate}[label = (\roman*)]
		\item \textit{$ I$-inj} denote the class of maps with the right lifting property with respect to $ I$.
		\item \textit{$ I$-cof} denote the class of maps with the left lifting property with respect to $ I$-inj.
		\item \textit{$ I$-cell} denote the class of maps of transfinite compositions of pushouts of maps in $ I$.
	\end{enumerate}
	
	Now suppose $\mathcal D$ has an initial object.
	Given a second class of maps $ J$, let \textit{$ J$-ccd} denote
	the subclass of $ J$-cell with $ I$-cofibrant domains
	(i.e. $x \to y$ in $ J$-cell is in $ J$-ccd iff the map $\varnothing \to x$ is in $ I$-cof).
\end{definition}

\begin{definition}
	Suppose $\mathcal D$ is a category closed under all small limits and colimits.
	% Given an adjunction $F \colon \mathcal M \rightleftarrows \colon U \mathcal D$ with $\mathcal M$ a model category,
	A \textit{semi-model structure} on $\mathcal D$ consists of
	subcateogries of weak equivalences, cofibrations, and fibrations, such that:
	\begin{enumerate}[label = (\roman*)]
		% \item $U$ preserves fibrations and trivial fibrations.
		\item Weak equivalences are closed under two-out-of-three, and all three chosen subcategories are closed under retracts.
		\item Every map in $\mathcal D$ can be functorially factored into a cofibration followed by a trivial fibration.
		Every map in $\mathcal D$ whose domain is \textbf{cofibrant} in $\mathcal D$ can be functorially factored into a trivial cofibration followed by a fibration.
		\item Cofibrations in $\mathcal D$ have the left lifting property with respect to trivial fibrations.
		Trivial cofibrations in $\mathcal D$ whose domain is \textbf{cofibrant} in $\mathcal D$ have the left lifting property with respect to fibrations.
		% \item The initial object in $\mathcal D$ is cofibrant in $\mathcal D$.
		\item Fibrations and trivial fibrations are closed under pullback.
	\end{enumerate}
	
	A \textit{semi-model category} is such a category $\mathcal D$ equipped with a semi-model structure.
	
	A semi-model category $\mathcal D$ is \textit{cofibrantly generated} if there exist sets of maps $ I$, $ J$ such that
	\begin{enumerate}[label = (\roman*)]
		\item $ I$-inj is the class of trivial cofibrations, and $ J$-inj is the class of fibrations; and
		\item the domains of $ I$ are small relative to $ I$-cell, and
		\item the domains of $ J$ are small relative to $ J$-ccd.
	\end{enumerate}
\end{definition}

\begin{remark}
	In \cite{Spi01,Wh16}, these semi-model structures come equipped with an adjunction $\mathcal M \rightleftarrows \mathcal D$ with $\mathcal M$ a model category,
	with the added condition that $U$ preserves (trivial) fibrations.
	Our definition is a special case, where $\mathcal M$ is the terminal category with its unique model strucutre,
	and the left adjoint is the inclusion of the initial object.
	
	In all of the examples from White and Spitzweck, these adjunctions actually induce the semi-model structure on $\mathcal D$;
	however, this extra data is not necessary to generalize much of \cite{Hov99} to this context.
\end{remark}

The following is immediate from the Retract Argument \cite[Lemma 1.1.9]{Hov99}.
\begin{lemma}
	A map is a trivial fibration (resp. cofibration) iff it has the right (resp. left) lifting property with respect to cofibrations (resp. trivial fibrations).
	
	A map with cofibrant source is a fibration (resp. trivial cofibration) iff it has the right (resp. left) lifting property with respect to trivial cofibrations (resp. fibrations).
\end{lemma}

The existence result \cite[Theorem 2.1.19]{Hov99} immediately generalizes to the semi-model categorical context,
as the (proof of the) small object argument \cite[Lemma 2.1.14]{Hov99} implies that
if the domains of $J$ are small relative to $J$-ccd,
then any map in $\mathcal D$ with cofibrant source may be factored into a map in $J$-ccd followed by a map in $J$-inj.
\begin{theorem}\label{SEMIMS_THM}
	Suppose $\mathcal D$ is a category with all small limits and colimits.
	Suppose $\mathcal W$ is a subcategory of $\mathcal D$, and $I$ and $J$ are sets of maps of $\mathcal D$.
	Then there is a cofibrantly generated semi-model structure on $\mathcal D$ with
	$I$ (resp. $J$) the set of generating (trivial) cofibrations, and weak equivalences $\mathcal W$ iff:
	\begin{enumerate}[label = (\roman*)]
		\item $\mathcal W$ is closed under two-out-of-three and retracts.
		\item The domains of $I$ (resp. $J$) are small relative to $I$-cell (resp. $J$-ccd).
		\item $J\text{-ccd} \subseteq \mathcal W \cap I\text{-cof}$.
		\item $I$-inj $\subseteq$ $\mathcal W \cap J$-inj.
		\item Either $\mathcal W \cap I\text{-cof} \subseteq J\textit{-cof}$ or $\mathcal W \cap J\text{-inj} \subseteq I\text{-inj}.$.
	\end{enumerate}
\end{theorem}


\subsection{Results}

The main results Theorem \ref{THMI} and Theorem \ref{THMII}, adapted to the semi-model structure context, follow with only minimal changes.
We highlight these in this section.

First, we stengthen ``admits all equivariant model structures'' to ``acyclic cofibrant fixed points'' for the first result.
\begin{theorem}
	\label{THM1_S}
	Suppose $(\V,\otimes)$
	\begin{enumerate}[label = (\roman*)]
		\item is a cofibrantly generated model category;
		\item is a closed monoidal model category with cofibrant unit;
		\item has ayclic cofibrant fixed points;
		\item has cofibranty symmetric pushout powers.
	\end{enumerate}
	Then for any $(G, \Sigma)$-family $\F$,
	the $\F$-semi-model structure exists on $\Op^G_{\mathfrak C}(\V)$. 
\end{theorem}
\begin{proof}
	Following \cite[Thm. 2.2.2]{WY18} it suffices to consider show that free extensions $\O \to \O[u]$ as in \eqref{OURE EQ} with $\O$ cofibrant in $\mathsf{Op}^G_{\mathfrak C}$ are weak equivalences.
	The proof for $\V = \sSet$ would then suffice, provided we could show that $\O$ must also be cofibrant in $\Sym^G_{\mathfrak C}(\V)$.
	This follows by induction on the cell complex decomposition of $\O$,
	with the base case stating that the initial operad $\mathbb{F}\emptyset$ is cofibrant in $\mathsf{Sym}^G_{\mathfrak{C}}$
	following from the assumption that $\mathcal{V}$ has a cofibrant unit
	and the induction step following from the already established cofibrancy of the maps $\O \to \O[u]$ in $\mathsf{Sym}^G_{\mathfrak{C}}$.
\end{proof}

We record a particular consequence of the above.
\begin{corollary}
	\label{LGC_COR}
	Suppose $(\V, \otimes)$ satisfies the hypotheses of Theorem \ref{THM1_S},
	let $\mathfrak C$ be a $G$-set, and $f: \O \to \P$ a map in $\Op^{G,\mathfrak C}(\V)$.
	If $f$ is a (trivial) cofibration in $\Op^{G, \mathfrak C}_\F(\V)$ for some $(G, \Sigma)$-family $\F$,
	and $\O$ is level $\F$-cofibrant, then
	$f(\vect C)$ is a (trivial) $\F$-cofibration in $\V^{\Aut(\vect C)}_{\F_{\vect C}}$ for all $\mathfrak C$-signatures $\vect C \in G \ltimes \Sigma_{\mathfrak C}$.
\end{corollary}


\begin{corollary}
	\label{CATV_MC_COR}
	Suppose $(\V, \otimes)$ satisfies the hypotheses of Theorem \ref{THM1_S},
	Then we have induced (semi)-model structures on $\Cat^{G}_{ \mathfrak C}(\V) = \Op^{G}_{\mathfrak C}(\V) \downarrow \**$,
	so in particular we have cofibrant replacements in $\Cat_{\mathfrak C}(\V)$ for any set $\mathfrak C$.
\end{corollary}

% \begin{remark}
%       \label{TOP_FULL_REM}
%       The category $\Op^{G,\mathfrak C}(\Top)$ actually has full model structures lifted from $\Sym^{G, \mathfrak C}_\F(\Top)$
%       for any $(G, \Sigma)$-family $\F$,
%       using an argument analogous to \cite[Thm. 3.1]{GW18}.
% \end{remark}


\textbf{THMII}


We note that Proposition \ref{LOCALTCHAR PROP} is formulated to be compatible with semi-model structures.
If the fibers $\mathsf{Op}^G_{\mathfrak{C}}(\V)$ have full model structures one can replace the role of
``$\mathcal{J}_{\mathfrak{C}}$'' and ``$\mathcal{I}_{\mathfrak{C}}$''
with simply ``trivial cofibrations'' and ``cofibrations''.

\todo[inline]{come back}

\begin{remark}
	In \cite{BM13}, the Interval Cofibrancy Theorem does not appear to use the full power of ``adequate'' model categories -
	specifically, it appears that $\otimes$-perfectness is not applied,
	and thus it suffices to assume that the category of algebras over any non-symmetric operad has the induced model structure.
	It appears this too can be weakened:
	in particular, the proof never factors a map into a trivial cofibration followed by a weak equivalence,
	and only uses cofibrant replacement or pushouts of cofibrations between cofibrant objects.
	These operations all make sense in the language of a semi-model category.
	Thus we conclude that Interval Cofibrancy Theorem holds whenever the category of algebras over any non-symmetric operad has the induced semi-model structure. 
	
	A similar analysis holds true for the $W$-construction from \cite{BM06}.
	
	This allows us to immediately conclude that, for any $\V$ satisfying the hypotheses of Theorem \ref{THM1_S}:
	\begin{enumerate}[label=(\roman*)]
		\item (Virtual) equivalence of objects (Definition \ref{EQUIV_DEF}) is an equivalence relation.
		\item Transfinite composites of $\F$-essentially surjective maps in $\Op^G_\bullet(\V)$ are $\F$-essentially surjective.
		\item $\V$ satifies the coherence axiom.
	\end{enumerate}
\end{remark}

The two majors steps in the proof of Theorem \ref{THMII} are that  relative $\mathcal J_\F$-cells are $\F$-weak equivalences, and 2-out-of-3.
The proofs of Proposition \ref{J-CELL_PROP} and Corollary \ref{ALBEETA COR} may be modified accordingly.

\begin{proposition}
	Relative $\mathcal J_\F$-cells with $\F$-cofibrant source are $\F$-weak equivalences.
\end{proposition}

\begin{proof}
	The proof of Proposition \ref{J-CELL_PROP} implies that for any generating trivial cofibration,
	the pushout is either a trivial $\F$-cofibration or a composite of an trivial $\F$-cofibration with an essentially surjective local isomorphism.
	If we started with a cofibrant source, each pushout map will have a cofibrant source,
	and hence by Corollary \ref{LGC_COR} each of these maps will be a level trivial $\F$-cofibration.
	As level genuine trivial cofibrations and essentially surjective maps are closed under transfinite composition, the result holds.
\end{proof}

\begin{proposition}
	Suppose that $\V$ has cofibrant pushout power and acyclic cofibrant fixed points,
	and $\P, \vect B, \vect C, \Lambda$ as in Corollary \ref{ALBEETA COR}.
	Then the conclusions of Corollary \ref{ALBEETA COR} hold.
\end{proposition}
\begin{proof}
	Using a zig-zag of weak equivlances, we may assume $\P$ is $\F$-bifibrant.
	But then $\P$ is level genuine cofibrant by Corollary \ref{LGC_COR},
	and hence $\P(\vect B)^\Lambda$ is cofibrant.
	Thus we replace the use of the 
\end{proof}

The result of \S 4 follows as in the regular model structure case.

\begin{theorem}
	\label{THMII_S}
	Let $(\V,\otimes)$ denote a closed symmetric monoidal category such that
	$(\V, \otimes)$:
	\begin{enumerate}[label = (\roman*)]
		\item is a cofibrantly generated model category;
		\item is a closed monoidal model category with cofibrant unit;
		\item admits all equivariant model structures and has acyclic cofibrant fixed points;
		\item has cofibranty symmetric pushout powers.
	\end{enumerate}
	
	Let $\F = \set{\F_n}_{n \geq 0}$ a $(G, \Sigma)$-family which has enough units.
	Then there exists a semi-model structure on $\Op^G_\bullet(\V)$, denoted $\Op^G_{\bullet,\F}(\V)$,
	with weak equivalences and fibrations defined as in Theorem \ref{THMII}.
	% such that
	% a map $F \colon \O \to \O'$ is a weak equivalence if
	% \begin{enumerate}[label = (\alph*)]
	% \item $\O \to F^{\**}\O'$ is a weak equivalence in $\Op^G_{\mathfrak C(\O), \F}(\V)$ and
	% \item $\pi_0(j^{\**}F^H)$ is essentially surjective for all $H \in \F_1$.
	% \end{enumerate}
\end{theorem}


\todo[inline]{come back}








\bibliography{biblio-new}{}
\bibliographystyle{amsalpha2}



\end{document}


%%% Local Variables:
%%% mode: latex
%%% TeX-master: t
%%% End:
