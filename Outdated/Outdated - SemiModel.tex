\documentclass[a4paper,10pt
%,draft
%, final
]{article}%

\pdfcompresslevel=0
\pdfobjcompresslevel=0

\usepackage[hidelinks]{hyperref}
\hypersetup{
  % colorlinks,
  final,
  pdftitle={Equivariant Dendroidal Segal Spaces},
  pdfauthor={Bonventre, P. and Pereira, L. A.},
  % pdfsubject={Your subject here},
  % pdfkeywords={keyword1, keyword2},
  linktoc=page
}
%\usepackage[open=false]{bookmark}

\input{commands.tex}%


%-------- Tikz ---------------------------

\usepackage{tikz}%
\usetikzlibrary{matrix,arrows,decorations.pathmorphing,
cd,patterns,calc}
\tikzset{%
  treenode/.style = {shape=rectangle, rounded corners,%
                     draw, align=center,%
                     top color=white, bottom color=blue!20},%
  root/.style     = {treenode, font=\Large, bottom color=red!30},%
  env/.style      = {treenode, font=\ttfamily\normalsize},%
  dummy/.style    = {circle,draw,inner sep=0pt,minimum size=2mm}%
}%

\usetikzlibrary[decorations.pathreplacing]



% ---- Commands on draft --------

\usepackage{ifdraft}
\ifdraft{
  \color[RGB]{63,63,63}
  % \pagecolor[rgb]{0.5,0.5,0.5}
  \pagecolor[RGB]{220,220,204}
  % \color[rgb]{1,1,1}
}


\usepackage[draft]{showkeys}
\usepackage{todonotes}%[obeyDraft]


% ----- Labels Changed? --------

\makeatletter

\def\@testdef #1#2#3{%
  \def\reserved@a{#3}\expandafter \ifx \csname #1@#2\endcsname
  \reserved@a  \else
  \typeout{^^Jlabel #2 changed:^^J%
    \meaning\reserved@a^^J%
    \expandafter\meaning\csname #1@#2\endcsname^^J}%
  \@tempswatrue \fi}

\makeatother


% ---- Commands --------

% new symbols

\newcommand{\mycircled}[2][none]{%
  \mathbin{
    \tikz[baseline=(a.base)]\node[draw,circle,inner sep=-1.5pt, outer sep=0pt,fill=#1](a){\ensuremath #2\strut};
cf.  }
}
\newcommand{\owr}{\mycircled{\wr}}

% replace symbols

\renewcommand{\hat}{\widehat}

% random

\renewcommand{\F}{\mathcal F}
\newcommand{\Q}{\mathcal Q}

\newcommand{\lltimes}{\underline{\ltimes}}


% detecting $\V$-categories:

\newcommand{\I}{\mathbb I}
\newcommand{\J}{\mathbb J}
\renewcommand{\1}{\eta}%{\ensuremath{\mathbb{id}}}

% lazy shortcuts

\newcommand{\SC}{\Sigma_{\mathfrak C}}
\newcommand{\OC}{\Omega_{\mathfrak C}}

\newcommand{\UV}{\underline{\mathcal V}}
\newcommand{\UC}{\underline{\mathfrak C}}

\usepackage{harpoon}
\newcommand{\vect}[1]{\text{\overrightharp{\ensuremath{#1}}}}


% ---- Title --------

\title{Equivariant Segal operads, simplicial operads, and dendroidal sets}

\author{Peter Bonventre, Lu\'is A. Pereira}%

\date{\today}


% ---- Document body --------

\begin{document}

\maketitle

\begin{abstract}
      Things and stuff
\end{abstract}

\tableofcontents





\section{Semi-model structures}
\label{SMS_SEC}

Our main results Theorems \ref{THMI} and \ref{THMIII} can be weakened to hold in broader settings.
Namely, if we only desire semi-model structures on $\Op^G_{\mathfrak C}(\V)$ and $\Op_\bullet^G(\V)$, we may
remove the global monoid axiom condition (as long as we mildly strengthen an assumption on $\V^G$).

\subsection{Semi-model categories}

We briefly recall the definition of a semi-model category, following \cite{Spi01,Wh16}.

\begin{notation}[cf. Notation \ref{CELL_NOT}]
      Given a class of maps $\mathcal I$ in a category $\mathcal D$, let
      $\mathcal I$-inj denote the class of maps with the right lifting property with respect to $\mathcal I$.
      % $\mathcal I$-cof is then equal to the class of maps with the left lifting property with respect to $\mathcal I$-inj.
      
      Suppose further that $\mathcal D$ has an initial object, and fix a second class of maps $\mathcal J$.
      We let $\mathcal J$-ccd denote
      the subclass of $\mathcal J$-cell with $\mathcal I$-cofibrant domains
      (i.e. $x \to y$ in $\mathcal J$-cell is in $\mathcal J$-ccd iff the map $\varnothing \to x$ is in $\mathcal I$-cof).
\end{notation}

Let $\mathcal D$ be a category with all small limits and colimits.
A \textit{semi-model structure} on $\mathcal D$ consists of
subcateogries of weak equivalences, cofibrations, and fibrations,
which satisfy all the axioms of a Quillen model category with the following two modifications:
\begin{itemize}
\item Every map in $\mathcal D$ whose domain is \textbf{cofibrant} in $\mathcal D$
      can be functorially factored into a trivial cofibration followed by a fibration.
\item Trivial cofibrations in $\mathcal D$ whose domain is \textbf{cofibrant} in $\mathcal D$
      have the left lifting property with respect to fibrations.
\end{itemize}

A semi-model category $\mathcal D$ is \textit{cofibrantly generated} if there exist sets of maps $\mathcal I$, $\mathcal J$ such that
\begin{enumerate}[label=(\roman*)]
\item $\mathcal I$-inj is the class of trivial cofibrations, and $\mathcal J$-inj is the class of fibrations;
\item the domains of $\mathcal I$ are small relative to $\mathcal I$-cell;
\item the domains of $\mathcal J$ are small relative to $\mathcal J$-ccd.
\end{enumerate}



% \begin{definition}
% 	Suppose $\mathcal D$ is a category closed under all small limits and colimits.
% 	% Given an adjunction $F \colon \mathcal M \rightleftarrows \colon U \mathcal D$ with $\mathcal M$ a model category,
% 	A \textit{semi-model structure} on $\mathcal D$ consists of
% 	subcateogries of weak equivalences, cofibrations, and fibrations, such that:
% 	\begin{enumerate}[label = (\roman*)]
% 		% \item $U$ preserves fibrations and trivial fibrations.
% 		\item Weak equivalences are closed under two-out-of-three, and all three chosen subcategories are closed under retracts.
% 		\item Every map in $\mathcal D$ can be functorially factored into a cofibration followed by a trivial fibration.
% 		Every map in $\mathcal D$ whose domain is \textbf{cofibrant} in $\mathcal D$ can be functorially factored into a trivial cofibration followed by a fibration.
% 		\item Cofibrations in $\mathcal D$ have the left lifting property with respect to trivial fibrations.
% 		Trivial cofibrations in $\mathcal D$ whose domain is \textbf{cofibrant} in $\mathcal D$ have the left lifting property with respect to fibrations.
% 		% \item The initial object in $\mathcal D$ is cofibrant in $\mathcal D$.
% 		\item Fibrations and trivial fibrations are closed under pullback.
% 	\end{enumerate}
	
% 	A \textit{semi-model category} is such a category $\mathcal D$ equipped with a semi-model structure.
	
% 	A semi-model category $\mathcal D$ is \textit{cofibrantly generated} if there exist sets of maps $\mathcal I$, $\mathcal J$ such that
% 	\begin{enumerate}[label = (\roman*)]
% 		\item $\mathcal I$-inj is the class of trivial cofibrations, and $\mathcal J$-inj is the class of fibrations; and
% 		\item the domains of $\mathcal I$ are small relative to $\mathcal I$-cell, and
% 		\item the domains of $\mathcal J$ are small relative to $\mathcal J$-ccd.
% 	\end{enumerate}
% \end{definition}

\begin{remark}
      This definition is actually a special case of the definitions from \cite{Spi01,Wh16},
      where the necessary adjunction $\mathcal D \leftrightarrows \mathcal M$
      has $\mathcal M$ as the terminal category, the right adjoint is the unique map, and the left adjoint is the inclusion of the initial object.
% 	In \cite{Spi01,Wh16}, these semi-model structures come equipped with an adjunction $\mathcal M \rightleftarrows \mathcal D$ with $\mathcal M$ a model category,
% 	with the added condition that $U$ preserves (trivial) fibrations.
% 	Our definition is a special case, where $\mathcal M$ is the terminal category with its unique model strucutre,
% 	and the left adjoint is the inclusion of the initial object.
	
% 	In all of the examples from White and Spitzweck, these adjunctions actually induce the semi-model structure on $\mathcal D$;
% 	however, this extra data is not necessary to generalize much of \cite{Hov99} to this context.
\end{remark}

Semi-model structures shares several of the nice properties of model categories:
\begin{enumerate}[label=(\roman*)]
\item There is a well-defined notion of cofibrant replacement: $\varnothing \rightarrowtail x_c \overset{\sim}{\twoheadrightarrow} x$.
\item Over- and under-categories $\mathcal D \downarrow x$, $x \downarrow \mathcal D$ inhert semi-model structures.
\item A map is a trivial fibration (resp. cofibration) iff it has the right (resp. left) lifting property with respect to cofibrations (resp. trivial fibrations).
\item A map with cofibrant source is a fibration (resp. trivial cofibration) iff it has the right (resp. left) lifting property with respect to trivial cofibrations (resp. fibrations).
\end{enumerate}
These last two follow from the Retract Argument \cite[Lemma 1.1.9]{Hov99}.

Moreover, the existence result \cite[Theorem 2.1.19]{Hov99} generalizes to the semi-model categorical context:
The (proof of the) small object argument \cite[Lemma 2.1.14]{Hov99} implies that
if the domains of $\mathcal J$ are small relative to $\mathcal J$-ccd,
then any map in $\mathcal D$ with cofibrant source may be factored into a map in $\mathcal J$-ccd followed by a map in $J$-inj.
\begin{theorem}\label{SEMIMS_THM}
	Suppose $\mathcal D$ is a category with all small limits and colimits.
	Suppose $\mathcal W$ is a subcategory of $\mathcal D$, and $I$ and $J$ are sets of maps of $\mathcal D$.
	Then there is a cofibrantly generated semi-model structure on $\mathcal D$ with
	$\mathcal I$ (resp. $\mathcal J$) the set of generating (trivial) cofibrations, and weak equivalences $\mathcal W$ iff:
	\begin{enumerate}[label = (\roman*)]
		\item $\mathcal W$ is closed under two-out-of-three and retracts.
		\item The domains of $\mathcal I$ (resp. $\mathcal J$) are small relative to $\mathcal I$-cell (resp. $\mathcal J$-ccd).
		\item $\mathcal J\text{-ccd} \subseteq \mathcal W \cap \mathcal I\text{-cof}$.
		\item $\mathcal I$-inj $\subseteq$ $\mathcal W \cap \mathcal J$-inj.
		\item Either $\mathcal W \cap \mathcal I\text{-cof} \subseteq \mathcal J\textit{-cof}$
                      or $\mathcal W \cap \mathcal J\text{-inj} \subseteq \mathcal I\text{-inj}.$.
	\end{enumerate}
\end{theorem}


\subsection{Semi-model categorical results}

Semi-model categorical adaptations of Theorems \ref{THMI} and \ref{THMII}, written as Theorems \ref{THMI_S} and \ref{THMIII_S} below,
follow from the previously given proofs with minimal changes.
We highlight the important alterations below.

\begin{customthm}{I-S}
      \label{THM1_S}
      Suppose $(\V,\otimes)$ satisfies conditions $(i),(ii),(iii),(iv)$ from Theorem \ref{THMI}.
      Then for any $(G, \Sigma)$-family $\F$,
      there exists a semi-model structure on $\Op_{\mathfrak C}^G(\V)$ such that a map $\O \to \P$ is a weak equivalence (resp. fibration) if the maps in \eqref{THMI_EQ} are so in $\V$
      for all $\mathfrak C$-signatures $\vect C$ and $\Lambda \in \F$ which stabilize $\vect C$.
\end{customthm}
\begin{proof}
      Following \cite[Thm. 2.2.2]{WY18} it suffices to consider show that free extensions $\O \to \O[u]$ as in \eqref{OURE EQ} with $\O$ cofibrant in $\mathsf{Op}^G_{\mathfrak C}(\V)$ are weak equivalences.
      The proof given in \S \ref{OPC_MS_SEC} suffices
      provided we can show that $\O$ must also be cofibrant in $\Sym^G_{\mathfrak C}(\V)$,
      as an application of Proposition \ref{RESGEN PROP} would then imply 
      $n_k^{\O,X,Y}$ is a genuine trivial cofibration in $\V^{G \ltimes \Omega_{\mathfrak C}^{a,op}[k]}$.
      This claim follows by induction on the cell complex decomposition of $\O$,
      with the base case stating that the initial operad $\mathbb{F}\emptyset$ is cofibrant in $\mathsf{Sym}^G_{\mathfrak{C}}(\V)$
      following from the assumption that $\mathcal{V}$ has a cofibrant unit
      and the induction step following from the already established cofibrancy of the maps $\O \to \O[u]$ in $\mathsf{Sym}^G_{\mathfrak{C}}(\V)$.
\end{proof}

We have the following analogue of Lemma \ref{GOTC_LEM}.
\begin{corollary}
      \label{LGC_COR}
      $\F$-(trivial) cofibrations in $\Op_{\mathfrak C}^G(\V)$ with cofibrant source are
      underlying equivariant (trivial) $\F$-cofibrations in $\Sym_{\mathfrak C}^G(\V)$.
      % Suppose $(\V, \otimes)$ satisfies the hypotheses of Theorem \ref{THM1_S},
      % let $\mathfrak C$ be a $G$-set, and $f: \O \to \P$ a map in $\Op^{G,\mathfrak C}(\V)$.
      % If $f$ is a (trivial) cofibration in $\Op^{G, \mathfrak C}_\F(\V)$ for some $(G, \Sigma)$-family $\F$,
      % and $\O$ is level $\F$-cofibrant, then
      % $f(\vect C)$ is a (trivial) $\F$-cofibration in $\V^{\Aut(\vect C)}_{\F_{\vect C}}$ for all $\mathfrak C$-signatures $\vect C \in G \ltimes \Sigma_{\mathfrak C}$.
\end{corollary}

\begin{remark}
      As before, Theorem \ref{I_S} implies the existence of induced semi-model structures on
      $\Cat^{G}_{ \mathfrak C}(\V) = \Op^{G}_{\mathfrak C}(\V) \downarrow \**$;
      in particular we have cofibrant replacements in $\Cat_{\mathfrak C}(\V)$ for any set $\mathfrak C$.
\end{remark}

We note that the proof of Theorem \ref{THMII} works as written when $\Op_{\mathfrak C, \F}^G(\V)$ only has a semi-model structure.
Thus we have the following strengthening.
\begin{customthm}{II}\label{THMII-S}
	Suppose $\V$ satisfies conditions $(i),(ii),(iii),(iv)$ from Theorem \ref{THMI},
	and let $\F$ be a pseudo indexing system (Definition \ref{PIS_DEF}).
	If $\O \in \Op^G_\F(\V)$ is $\F$-cofibrant, then the underlying symmetric sequence is $\F$-cofibrant in $\Sym^G_\F(\V)$.
\end{customthm}



Moving on to the semi-model structure on the total category $\Op_{\bullet, \F}^G(\V)$,
we note Proposition \ref{LOCALTCHAR PROP} applies as written, and thus
Definition \ref{OPGENCOF DEF} defines generating (trivial) cofibrations in the semi-model context as well.
% is formulated to be compatible with semi-model structures.
% If the fibers $\mathsf{Op}^G_{\mathfrak{C}}(\V)$ have full model structures one can replace the role of
% ``$\mathcal{J}_{\mathfrak{C}}$'' and ``$\mathcal{I}_{\mathfrak{C}}$''
% with simply ``trivial cofibrations'' and ``cofibrations''.

Now, the two major steps in the proof of Theorem \ref{THMIII} are that
\begin{itemize}
\item relative $\mathcal J_\F$-cells are $\F$-weak equivalences (Proposition \ref{J-CELL_PROP}), and
\item weak equivalences satisfy 2-out-of-3 (Proposition \ref{2OUTOF3 PROP}).
\end{itemize}
The proof of these results and the discussion surrounding them may be modified accordingly. %  Corollary \ref{ALBEETA COR} 

First, we contend with the Interval Cofibrancy Theorem \ref{INTCOF THM}.
\begin{remark}
      As noted in \S \ref{TRIVCOF_SEC}, the hypotheses needed for the proof of the Interval Cofibrancy Theorem in \cite{BM13}
      are the existence of several model structures.
      This in fact may be further weakened:
      the proof never factors a map into a trivial cofibration followed by a weak equivalence,
      and only uses cofibrant replacement or pushouts of cofibrations between cofibrant objects.
      These operations all make sense in the language of a semi-model category,
      and thus the Interval Cofibrancy Theorem and Interval Amalgamation Lemma \ref{AMALGLEM LEM}
      hold whenever the canonical semi-model structures on $\Cat_{\mathfrak C}(\V)$ exist;
      in particular, the hypotheses of Theorem \ref{THMI_S} suffice.
\end{remark}

% \begin{theorem}
%       \label{INTCOF-S THM}
%       Let $(\V, \otimes)$ be a cofibrantly generated monoidal model category for which
%       the canonical semi-model structures on $\Cat_{\mathfrak C}(\V)$ exist.
%       If $\mathbb J \in \Cat_{\set{0,1}}(\V)$ is cofibrant then
%       $\mathbb J(0,0)$ is cofibrant in $\Cat_{\set{0}}(\V)$.
% \end{theorem}

This is sufficient to prove the semi-model structure analogue of Proposition \ref{J-CELL_PROP}.

\begin{proposition}
      Suppose $(\V,\otimes)$ satisfies the hypotheses from Theorem \ref{THMI_S}.
      Then maps in the saturation of $(TC1),(TC2)$ with with $\F$-cofibrant source are $\F$-weak equivalences.
\end{proposition}
\begin{proof}
	The proof of Proposition \ref{J-CELL_PROP} implies that for any generating trivial $\F$-cofibration,
	the pushout is either a trivial $\F$-cofibration or a composite of an trivial $\F$-cofibration with an essentially surjective local isomorphism.
	If we begin with a cofibrant source, each pushout map will have a cofibrant source,
	and hence by Corollary \ref{LGC_COR} each of these maps will be a level trivial $\F$-cofibration.
        As level genuine trivial cofibrations and essentially surjective maps are closed under transfinite composition by Lemma \ref{TRANSCOMP_ES_LEM},
        the result holds.
\end{proof}



Next, we consider equivalences of objects.
We note that, in a semi-model category, an arbitrary object $x$ need not have a map to it's fibrant replacement,
and instead there exists a zig-zag of weak equivalences
\[
      x \overset{\sim}{\twoheadleftarrow} x_c \overset{\sim}{\rightarrowtail} x_f.
\]
However, this is sufficent for our analysis, as most often we are considering maps from a cofibrant object $\mathbb J$ into our arbitrary category $\mathcal C$.
In particular, we have that the definitions of (virtually, homotopy) equivalent make sense in the semi-model category $\Cat(\V)$,
and that they are nested as in Remark \ref{EQUIVNEST_REM}.
The rest of \S \ref{EQUIVOBJ_SEC} follows as written.

Now, the technical work in \S \ref{HMTYEQ SEC} for finishing Proposition \ref{2OUTOF3 PROP} culminates in Corollary \ref{ALBEETA COR}.
In particular, the proof of this result explicitly uses both the monoid axiom and
condition $(vii)$ from Theorem \ref{THMIII}, a strengthening of our assumptions on the model categories $\V^G$.
In the semi-model structure context, we no longer use the monoid axiom,
but we will require an mildly stronger assumption on the model categories $\V^G$:
\begin{itemize}
\item[(vii')] For any group $G$, fixed points $(-)^G \colon \V^G \to \V$ send genuine (trivial) cofibrations to (trivial) cofibrations.
\end{itemize}

\begin{proposition}
      \label{ALBEETA_S COR}
      Suppose that $\V$ satisfies conditions $(i),(ii),(iii),(iv)$ from Theorem \ref{THMI} and $(vii')$.
      Let $\P, \vect B, \vect C, \Lambda$ be as in Corollary \ref{ALBEETA COR}.
      Then the conclusions of Corollary \ref{ALBEETA COR} hold.
\end{proposition}
\begin{proof}
      The proof as written works in this more general context, provided that we can show that
      $\P(\vect B)^{\Lambda} \otimes (\mathbb C^{\otimes n})^\Lambda \to \P(\vect B)^{\Lambda}$
      is a weak equivalence.
      
      Using a zig-zag of weak equivalences, we may assume $\P$ is bifibrant.
      But then $\P$ is level genuine cofibrant by Corollary \ref{LGC_COR},
      and hence $\P(\vect B)^\Lambda$ is cofibrant by assumption.
      The claim then follows since $\V$ satisfies the pushout product axiom.
\end{proof}

\begin{remark}
      The proof of Proposition \ref{WEAKCELL PROP} implies that
      weak acyclic cellular fixed points (Definition \ref{WEAKCELL DEF})
      suffices for the equivariant assumption $(vii)$ in Corollary \ref{ALBEETA COR}.
      Similarly, if $(-)^H$ additionally satisfies the analogous assumptions
      replacing $\mathcal J_G$ and $\mathcal J$-cof with $\mathcal I_G$ and $\mathcal I$-cof,
      then the equivariant assumption $(vii')$ in Proposition \ref{ALBEETA_S COR} holds.      
\end{remark}

The remaining parts of \S \ref{HMTYEQ SEC} follow as written.
Thus, replacing the use of \cite[Thm. 2.1.19]{Hov99} with Theorem \ref{SEMIMS_THM},
we may conclude the semi-model structure analogue of our main result.

\begin{customthm}{III-S}
      \label{THMIII_S}
      For any $(G,\Sigma)$-family $\F$ with enough units,
      an $\F$-semi-model structure on $\Op^G(\V)$ with weak equivalences/trivial fibrations determined as in
      \eqref{THMIII1ST EQ}, \eqref{THMIII2ND EQ} exists provided
      $(\V, \otimes)$ satisfies conditions $(i),(ii),(iii),(iv)$ in Theorem \ref{THMI_S} and additionally
      \begin{enumerate}[label = (\roman*)]
            \setcounter{enumi}{5}
      \item $\V$ is right proper
      \item[(vii')] for any group $G$, fixed points $(-)^G \colon \V^G \to \V$ send genuine (trivial) cofibrations to (trivial) cofibrations;
      \item[(viii)] $(\V, \otimes)$ has a generating set of intervals.
      \end{enumerate}
\end{customthm}









\newpage


\section{Semi-model structures}
\label{SMS_SEC}

Our main results Theorems \ref{THMI} and \ref{THMIII} can be weakened to hold in broader settings.
Namely, if we only desire semi-model structures on $\Op^G_{\mathfrak C}(\V)$ and $\Op_\bullet^G(\V)$, we may
remove the global monoid axiom condition (as long as we mildly strengthen an assumption on $\V^G$).

\subsection{Semi-model categories}

We briefly recall the definition of a semi-model category, following \cite{Spi01,Wh16}.

\begin{notation}[cf. Notation \ref{CELL_NOT}]
	Given a class of maps $\mathcal I$ in a category $\mathcal D$, let
	$\mathcal I$-inj denote the class of maps with the right lifting property with respect to $\mathcal I$.
	% $\mathcal I$-cof is then equal to the class of maps with the left lifting property with respect to $\mathcal I$-inj.
	
	Suppose further that $\mathcal D$ has an initial object, and fix a second class of maps $\mathcal J$.
	We let $\mathcal J$-ccd denote
	the subclass of $\mathcal J$-cell with $\mathcal I$-cofibrant domains
	(i.e. $x \to y$ in $\mathcal J$-cell is in $\mathcal J$-ccd iff the map $\varnothing \to x$ is in $\mathcal I$-cof).
\end{notation}

Let $\mathcal D$ be a category with all small limits and colimits.
A \textit{semi-model structure} on $\mathcal D$ consists of
subcateogries of weak equivalences, cofibrations, and fibrations,
which satisfy all the axioms of a Quillen model category with the following two modifications:
\begin{itemize}
	\item Every map in $\mathcal D$ whose domain is \textbf{cofibrant} in $\mathcal D$
	can be functorially factored into a trivial cofibration followed by a fibration.
	\item Trivial cofibrations in $\mathcal D$ whose domain is \textbf{cofibrant} in $\mathcal D$
	have the left lifting property with respect to fibrations.
\end{itemize}

A semi-model category $\mathcal D$ is \textit{cofibrantly generated} if there exist sets of maps $\mathcal I$, $\mathcal J$ such that
\begin{enumerate}[label=(\roman*)]
	\item $\mathcal I$-inj is the class of trivial cofibrations, and $\mathcal J$-inj is the class of fibrations;
	\item the domains of $\mathcal I$ are small relative to $\mathcal I$-cell;
	\item the domains of $\mathcal J$ are small relative to $\mathcal J$-ccd.
\end{enumerate}



% \begin{definition}
% 	Suppose $\mathcal D$ is a category closed under all small limits and colimits.
% 	% Given an adjunction $F \colon \mathcal M \rightleftarrows \colon U \mathcal D$ with $\mathcal M$ a model category,
% 	A \textit{semi-model structure} on $\mathcal D$ consists of
% 	subcateogries of weak equivalences, cofibrations, and fibrations, such that:
% 	\begin{enumerate}[label = (\roman*)]
% 		% \item $U$ preserves fibrations and trivial fibrations.
% 		\item Weak equivalences are closed under two-out-of-three, and all three chosen subcategories are closed under retracts.
% 		\item Every map in $\mathcal D$ can be functorially factored into a cofibration followed by a trivial fibration.
% 		Every map in $\mathcal D$ whose domain is \textbf{cofibrant} in $\mathcal D$ can be functorially factored into a trivial cofibration followed by a fibration.
% 		\item Cofibrations in $\mathcal D$ have the left lifting property with respect to trivial fibrations.
% 		Trivial cofibrations in $\mathcal D$ whose domain is \textbf{cofibrant} in $\mathcal D$ have the left lifting property with respect to fibrations.
% 		% \item The initial object in $\mathcal D$ is cofibrant in $\mathcal D$.
% 		\item Fibrations and trivial fibrations are closed under pullback.
% 	\end{enumerate}

% 	A \textit{semi-model category} is such a category $\mathcal D$ equipped with a semi-model structure.

% 	A semi-model category $\mathcal D$ is \textit{cofibrantly generated} if there exist sets of maps $\mathcal I$, $\mathcal J$ such that
% 	\begin{enumerate}[label = (\roman*)]
% 		\item $\mathcal I$-inj is the class of trivial cofibrations, and $\mathcal J$-inj is the class of fibrations; and
% 		\item the domains of $\mathcal I$ are small relative to $\mathcal I$-cell, and
% 		\item the domains of $\mathcal J$ are small relative to $\mathcal J$-ccd.
% 	\end{enumerate}
% \end{definition}

\begin{remark}
	This definition is actually a special case of the definitions from \cite{Spi01,Wh16},
	where the necessary adjunction $\mathcal D \leftrightarrows \mathcal M$
	has $\mathcal M$ as the terminal category, the right adjoint is the unique map, and the left adjoint is the inclusion of the initial object.
	% 	In \cite{Spi01,Wh16}, these semi-model structures come equipped with an adjunction $\mathcal M \rightleftarrows \mathcal D$ with $\mathcal M$ a model category,
	% 	with the added condition that $U$ preserves (trivial) fibrations.
	% 	Our definition is a special case, where $\mathcal M$ is the terminal category with its unique model strucutre,
	% 	and the left adjoint is the inclusion of the initial object.
	
	% 	In all of the examples from White and Spitzweck, these adjunctions actually induce the semi-model structure on $\mathcal D$;
	% 	however, this extra data is not necessary to generalize much of \cite{Hov99} to this context.
\end{remark}

Semi-model structures shares several of the nice properties of model categories:
\begin{enumerate}[label=(\roman*)]
	\item There is a well-defined notion of cofibrant replacement: $\varnothing \rightarrowtail x_c \overset{\sim}{\twoheadrightarrow} x$.
	\item Over- and under-categories $\mathcal D \downarrow x$, $x \downarrow \mathcal D$ inhert semi-model structures.
	\item A map is a trivial fibration (resp. cofibration) iff it has the right (resp. left) lifting property with respect to cofibrations (resp. trivial fibrations).
	\item A map with cofibrant source is a fibration (resp. trivial cofibration) iff it has the right (resp. left) lifting property with respect to trivial cofibrations (resp. fibrations).
\end{enumerate}
These last two follow from the Retract Argument \cite[Lemma 1.1.9]{Hov99}.

Moreover, the existence result \cite[Theorem 2.1.19]{Hov99} generalizes to the semi-model categorical context:
The (proof of the) small object argument \cite[Lemma 2.1.14]{Hov99} implies that
if the domains of $\mathcal J$ are small relative to $\mathcal J$-ccd,
then any map in $\mathcal D$ with cofibrant source may be factored into a map in $\mathcal J$-ccd followed by a map in $J$-inj.
\begin{theorem}\label{SEMIMS_THM}
	Suppose $\mathcal D$ is a category with all small limits and colimits.
	Suppose $\mathcal W$ is a subcategory of $\mathcal D$, and $I$ and $J$ are sets of maps of $\mathcal D$.
	Then there is a cofibrantly generated semi-model structure on $\mathcal D$ with
	$\mathcal I$ (resp. $\mathcal J$) the set of generating (trivial) cofibrations, and weak equivalences $\mathcal W$ iff:
	\begin{enumerate}[label = (\roman*)]
		\item $\mathcal W$ is closed under two-out-of-three and retracts.
		\item The domains of $\mathcal I$ (resp. $\mathcal J$) are small relative to $\mathcal I$-cell (resp. $\mathcal J$-ccd).
		\item $\mathcal J\text{-ccd} \subseteq \mathcal W \cap \mathcal I\text{-cof}$.
		\item $\mathcal I$-inj $\subseteq$ $\mathcal W \cap \mathcal J$-inj.
		\item Either $\mathcal W \cap \mathcal I\text{-cof} \subseteq \mathcal J\textit{-cof}$
		or $\mathcal W \cap \mathcal J\text{-inj} \subseteq \mathcal I\text{-inj}.$.
	\end{enumerate}
\end{theorem}


\subsection{Semi-model categorical results}

Semi-model categorical adaptations of Theorems \ref{THMI} and \ref{THMII}, written as Theorems \ref{THMI_S} and \ref{THMIII_S} below,
follow from the previously given proofs with minimal changes.
We highlight the important alterations below.

\begin{customthm}{I-S}
	\label{THMI_S}
	Suppose $(\V,\otimes)$ satisfies conditions $(i),(ii),(iii),(iv)$ from Theorem \ref{THMI}.
	Then for any $(G, \Sigma)$-family $\F$,
	there exists a semi-model structure on $\Op_{\mathfrak C}^G(\V)$ such that a map $\O \to \P$ is a weak equivalence (resp. fibration) if the maps in \eqref{THMI_EQ} are so in $\V$
	for all $\mathfrak C$-signatures $\vect C$ and $\Lambda \in \F$ which stabilize $\vect C$.
\end{customthm}
\begin{proof}
	Following \cite[Thm. 2.2.2]{WY18} it suffices to consider show that free extensions $\O \to \O[u]$ as in \eqref{OURE EQ} with $\O$ cofibrant in $\mathsf{Op}^G_{\mathfrak C}(\V)$ are weak equivalences.
	The proof given in \S \ref{OPC_MS_SEC} suffices
	provided we can show that $\O$ must also be cofibrant in $\Sym^G_{\mathfrak C}(\V)$,
	as an application of Proposition \ref{RESGEN PROP} would then imply 
	$n_k^{\O,X,Y}$ is a genuine trivial cofibration in $\V^{G \ltimes \Omega_{\mathfrak C}^{a,op}[k]}$.
	This claim follows by induction on the cell complex decomposition of $\O$,
	with the base case stating that the initial operad $\mathbb{F}\emptyset$ is cofibrant in $\mathsf{Sym}^G_{\mathfrak{C}}(\V)$
	following from the assumption that $\mathcal{V}$ has a cofibrant unit
	and the induction step following from the already established cofibrancy of the maps $\O \to \O[u]$ in $\mathsf{Sym}^G_{\mathfrak{C}}(\V)$.
\end{proof}

We have the following analogue of Lemma \ref{GOTC_LEM}.
\begin{corollary}
	\label{LGC_COR}
	$\F$-(trivial) cofibrations in $\Op_{\mathfrak C}^G(\V)$ with cofibrant source are
	underlying equivariant (trivial) $\F$-cofibrations in $\Sym_{\mathfrak C}^G(\V)$.
	% Suppose $(\V, \otimes)$ satisfies the hypotheses of Theorem \ref{THM1_S},
	% let $\mathfrak C$ be a $G$-set, and $f: \O \to \P$ a map in $\Op^{G,\mathfrak C}(\V)$.
	% If $f$ is a (trivial) cofibration in $\Op^{G, \mathfrak C}_\F(\V)$ for some $(G, \Sigma)$-family $\F$,
	% and $\O$ is level $\F$-cofibrant, then
	% $f(\vect C)$ is a (trivial) $\F$-cofibration in $\V^{\Aut(\vect C)}_{\F_{\vect C}}$ for all $\mathfrak C$-signatures $\vect C \in G \ltimes \Sigma_{\mathfrak C}$.
\end{corollary}

\begin{remark}
	As before, Theorem \ref{THMI_S} implies the existence of induced semi-model structures on
	$\Cat^{G}_{ \mathfrak C}(\V) = \Op^{G}_{\mathfrak C}(\V) \downarrow \**$;
	in particular we have cofibrant replacements in $\Cat_{\mathfrak C}(\V)$ for any set $\mathfrak C$.
\end{remark}

We note that the proof of Theorem \ref{THMII} works as written when $\Op_{\mathfrak C, \F}^G(\V)$ only has a semi-model structure.
Thus we have the following strengthening.
\begin{customthm}{II}\label{THMII-S}
	Suppose $\V$ satisfies conditions $(i),(ii),(iii),(iv)$ from Theorem \ref{THMI},
	and let $\F$ be a pseudo indexing system (Definition \ref{PIS_DEF}).
	If $\O \in \Op^G_\F(\V)$ is $\F$-cofibrant, then the underlying symmetric sequence is $\F$-cofibrant in $\Sym^G_\F(\V)$.
\end{customthm}



Moving on to the semi-model structure on the total category $\Op_{\bullet, \F}^G(\V)$,
we note Proposition \ref{LOCALTCHAR PROP} applies as written, and thus
Definition \ref{OPGENCOF DEF} defines generating (trivial) cofibrations in the semi-model context as well.
% is formulated to be compatible with semi-model structures.
% If the fibers $\mathsf{Op}^G_{\mathfrak{C}}(\V)$ have full model structures one can replace the role of
% ``$\mathcal{J}_{\mathfrak{C}}$'' and ``$\mathcal{I}_{\mathfrak{C}}$''
% with simply ``trivial cofibrations'' and ``cofibrations''.

Now, the two major steps in the proof of Theorem \ref{THMIII} are that
\begin{itemize}
	\item relative $\mathcal J_\F$-cells are $\F$-weak equivalences (Proposition \ref{J-CELL_PROP}), and
	\item weak equivalences satisfy 2-out-of-3 (Proposition \ref{2OUTOF3 PROP}).
\end{itemize}
The proof of these results and the discussion surrounding them may be modified accordingly. %  Corollary \ref{ALBEETA COR} 

First, we contend with the Interval Cofibrancy Theorem \ref{INTCOF THM}.
\begin{remark}
	As noted in \S \ref{TRIVCOF_SEC}, the hypotheses needed for the proof of the Interval Cofibrancy Theorem in \cite{BM13}
	are the existence of several model structures.
	This in fact may be further weakened:
	the proof never factors a map into a trivial cofibration followed by a weak equivalence,
	and only uses cofibrant replacement or pushouts of cofibrations between cofibrant objects.
	These operations all make sense in the language of a semi-model category,
	and thus the Interval Cofibrancy Theorem and Interval Amalgamation Lemma \ref{AMALGLEM LEM}
	hold whenever the canonical semi-model structures on $\Cat_{\mathfrak C}(\V)$ exist;
	in particular, the hypotheses of Theorem \ref{THMI_S} suffice.
\end{remark}

% \begin{theorem}
%       \label{INTCOF-S THM}
%       Let $(\V, \otimes)$ be a cofibrantly generated monoidal model category for which
%       the canonical semi-model structures on $\Cat_{\mathfrak C}(\V)$ exist.
%       If $\mathbb J \in \Cat_{\set{0,1}}(\V)$ is cofibrant then
%       $\mathbb J(0,0)$ is cofibrant in $\Cat_{\set{0}}(\V)$.
% \end{theorem}

This is sufficient to prove the semi-model structure analogue of Proposition \ref{J-CELL_PROP}.

\begin{proposition}
	Suppose $(\V,\otimes)$ satisfies the hypotheses from Theorem \ref{THMI_S}.
	Then maps in the saturation of $(TC1),(TC2)$ with with $\F$-cofibrant source are $\F$-weak equivalences.
\end{proposition}
\begin{proof}
	The proof of Proposition \ref{J-CELL_PROP} implies that for any generating trivial $\F$-cofibration,
	the pushout is either a trivial $\F$-cofibration or a composite of an trivial $\F$-cofibration with an essentially surjective local isomorphism.
	If we begin with a cofibrant source, each pushout map will have a cofibrant source,
	and hence by Corollary \ref{LGC_COR} each of these maps will be a level trivial $\F$-cofibration.
	As level genuine trivial cofibrations and essentially surjective maps are closed under transfinite composition by Lemma \ref{TRANSCOMP_ES_LEM},
	the result holds.
\end{proof}



Next, we consider equivalences of objects.
We note that, in a semi-model category, an arbitrary object $x$ need not have a map to it's fibrant replacement,
and instead there exists a zig-zag of weak equivalences
\[
x \overset{\sim}{\twoheadleftarrow} x_c \overset{\sim}{\rightarrowtail} x_f.
\]
However, this is sufficent for our analysis, as most often we are considering maps from a cofibrant object $\mathbb J$ into our arbitrary category $\mathcal C$.
In particular, we have that the definitions of (virtually, homotopy) equivalent make sense in the semi-model category $\Cat(\V)$,
and that they are nested as in Remark \ref{EQUIVNEST_REM}.
The rest of \S \ref{EQUIVOBJ_SEC} follows as written.

Now, the technical work in \S \ref{HMTYEQ SEC} for finishing Proposition \ref{2OUTOF3 PROP} culminates in Corollary \ref{ALBEETA COR}.
In particular, the proof of this result explicitly uses both the monoid axiom and
condition $(vii)$ from Theorem \ref{THMIII}, a strengthening of our assumptions on the model categories $\V^G$.
In the semi-model structure context, we no longer use the monoid axiom,
but we will require an mildly stronger assumption on the model categories $\V^G$:
\begin{itemize}
	\item[(vii')] For any group $G$, fixed points $(-)^G \colon \V^G \to \V$ send genuine (trivial) cofibrations to (trivial) cofibrations.
\end{itemize}

\begin{proposition}
	\label{ALBEETA_S COR}
	Suppose that $\V$ satisfies conditions $(i),(ii),(iii),(iv)$ from Theorem \ref{THMI} and $(vii')$.
	Let $\P, \vect B, \vect C, \Lambda$ be as in Corollary \ref{ALBEETA COR}.
	Then the conclusions of Corollary \ref{ALBEETA COR} hold.
\end{proposition}
\begin{proof}
	The proof as written works in this more general context, provided that we can show that
	$\P(\vect B)^{\Lambda} \otimes (\mathbb C^{\otimes n})^\Lambda \to \P(\vect B)^{\Lambda}$
	is a weak equivalence.
	
	Using a zig-zag of weak equivalences, we may assume $\P$ is bifibrant.
	But then $\P$ is level genuine cofibrant by Corollary \ref{LGC_COR},
	and hence $\P(\vect B)^\Lambda$ is cofibrant by assumption.
	The claim then follows since $\V$ satisfies the pushout product axiom.
\end{proof}

\begin{remark}
	The proof of Proposition \ref{WEAKCELL PROP} implies that
	weak acyclic cellular fixed points (Definition \ref{WEAKCELL DEF})
	suffices for the equivariant assumption $(vii)$ in Corollary \ref{ALBEETA COR}.
	Similarly, if $(-)^H$ additionally satisfies the analogous assumptions
	replacing $\mathcal J_G$ and $\mathcal J$-cof with $\mathcal I_G$ and $\mathcal I$-cof,
	then the equivariant assumption $(vii')$ in Proposition \ref{ALBEETA_S COR} holds.      
\end{remark}

The remaining parts of \S \ref{HMTYEQ SEC} follow as written.
Thus, replacing the use of \cite[Thm. 2.1.19]{Hov99} with Theorem \ref{SEMIMS_THM},
we may conclude the semi-model structure analogue of our main result.


\begin{customthm}{III-S}
	\label{THMIII_S}
	For any $(G,\Sigma)$-family $\F$ with enough units,
	an $\F$-semi-model structure on $\Op^G(\V)$ with weak equivalences/trivial fibrations determined as in
	\eqref{THMIII1ST EQ}, \eqref{THMIII2ND EQ} exists provided
	$(\V, \otimes)$ satisfies conditions $(i),(ii),(iii),(iv)$ in Theorem \ref{THMI_S} and additionally
	\begin{enumerate}[label = (\roman*)]
		\setcounter{enumi}{5}
		\item $\V$ is right proper
		\item[(vii')] for any group $G$, fixed points $(-)^G \colon \V^G \to \V$ send genuine (trivial) cofibrations to (trivial) cofibrations;
		\item[(viii)] $(\V, \otimes)$ has a generating set of intervals.
	\end{enumerate}
\end{customthm}












\bibliography{biblio-new}{}
\bibliographystyle{amsalpha2}



\end{document}


%%% Local Variables:
%%% mode: latex
%%% TeX-master: t
%%% End:
