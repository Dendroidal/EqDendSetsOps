
% -------------------- Translation Categories --------------------

\begin{example}
      \label{G_GR_REM}
      Given a group $G$, let $BG$ denote the category which encodes \textit{left} $G$-actions (so $h \circ g = h g$).
      For any left $G$-set $A$, the \textit{translation category} or \textit{action groupoid}
      has object set $A$
      and morphism set all pairs $(g,a): a \to g.a$ for all $(g,a) \in G \times A$
      (equivalently, $\Hom(a,b) = \sets{g \in G}{b = g.a}$).
      %
      More generally, if $\mathcal C$ is a left $G$-category, the \textit{translation category}
      has object set $\mathrm{Ob}(\mathcal C)$
      and morphism set all tuples $(g,a,b,f)$ with $g \in G$, $a,b\in \mathcal C$, and $f: g.a \to b$.
      %
      These are both isomorphic to the Grothendieck construction $BG \ltimes \mathcal C$
      on the functor $BG \to \Cat$ defining the $G$-action on $\mathcal C$.
      We will mildly abuse notation and denote these categories by simply $G \ltimes \mathcal C$
      \footnote{
        The action groupoid for a $G$-set $A$ is often denoted $B_A G$, generalizing the notation for the category $B G = B_{\**} G$.}.
      
      Further, we note that if $\mathcal C$ has an action by other groups $\Sigma$ which commute with the action by $G$,
      then the iterated Grothendieck constructions above are associative and commutative.
\end{example}

\begin{remark}
      \label{GOP_REM}
      If $\mathcal C$ is a left $G$-category, then we write
      $G^{op} \ltimes \mathcal C$ to denote the Grothendieck construction on the functor
      $BG^{op} \xrightarrow{(-)^{-1}} BG \xrightarrow{\mathcal C} \Cat$.

      Moreover, as $\mathcal C^{op}$ is also a left $G$-category, it is easy to check that
      $(G \ltimes \mathcal C)^{op} \simeq G^{op} \ltimes \mathcal C^{op}$,
      using the above convention.
\end{remark}



% -------------------- OTHER STUFF --------------------


% First, we record a basic result.

% \begin{lemma}
%       \label{PB_GR_LEM}
%       If $\mathcal C \to \mathcal D$ is a Grothendieck fibration, then the pullback
%       \begin{equation}
%             \begin{tikzcd}
%                   \mathcal A \arrow[d] \arrow[r]
%                   &
%                   \mathcal C \arrow[d]
%                   \\
%                   \mathcal B \arrow[r, "F"]
%                   &
%                   \mathcal D
%             \end{tikzcd}
%       \end{equation}
%       is isomorphic to the Grothendieck construction on
%       \begin{equation}
%             \label{PB_GR_EQ}
%             % \begin{tikzcd}[row sep = tiny]
%             %       \mathcal B^{op} \arrow[r]
%             %       &
%             %       \mathsf{Cat}
%             %       \\
%             %       b \arrow[r, mapsto]
%             %       &
%             %       \mathcal C_{F(b)},
%             % \end{tikzcd}
%             \mathcal B^{op} \longto \Cat,
%             \qquad \qquad
%             b \longmapsto \mathcal C_{F(b)},
%       \end{equation}
%       where $\mathcal C_{d}$ is the fiber over $d \in \mathcal D$.
% \end{lemma}
% \begin{proof}
%       The fact that the map \eqref{PB_GR_EQ} is a functor follows from $\mathcal C \to \mathcal D$ being a fibration;
%       the rest follows by unpacking definitions.
% \end{proof}



The following lemmas have straightforward proofs.

\begin{lemma}
      \label{GD_PULL_LEM}
      For any functor $\mathcal C \to \mathcal D$ of $G$-categories, the squares below are Cartesian.
      \begin{equation}
            \begin{tikzcd}
                  \mathcal C \arrow[d] \arrow[r]
                  &
                  G \ltimes \mathcal C \arrow[d]
                  &&
                  G \ltimes (\Sigma \wr \mathcal C) \arrow[d] \arrow[r]
                  &
                  \Sigma \wr (G \ltimes \mathcal C) \arrow[d]
                  \\
                  \mathcal D \arrow[r]
                  &
                  G \ltimes \mathcal D
                  &&
                  G \ltimes (\Sigma \wr \mathcal D) \arrow[r]
                  &
                  \Sigma \wr (G \ltimes \mathcal D)
            \end{tikzcd}
      \end{equation}
      Moreover, the square on the left lifts Kan extensions.
\end{lemma}

\begin{lemma}
      \label{GL_PULL_LEM}
      The functor $G \ltimes (-): \Cat^{G} \to \mathsf{Fib}(G)$ preserves pullbacks and coproducts.
\end{lemma}

\begin{lemma}
      \label{GL_GR_LEM}
      $G \ltimes (-)$ preserves Grothendieck fibrations, and in fact the fibers remain constant.
\end{lemma}
\begin{proof}
      A straightforward diagram chase shows that if $f$ is a Cartesian arrow in $\mathcal E$ over $\mathcal B$,
      then for any $g\in G$, $f \circ g$ is a Cartesian arrow in $G \ltimes \mathcal E$ over $G \ltimes \mathcal B$.
      The result follows immediately.
\end{proof}

\begin{lemma}
      \label{GL_RANINIT_LEM}
      $G \ltimes (-)$ preserves Ran-initiality:
      if $\mathcal C \to \mathcal D$ is a functor of $G$-categories over another $G$-category $\mathcal E$
      which is Ran-initial, then so is $G \ltimes \mathcal C \to G \ltimes \mathcal D$ over $G \ltimes \mathcal E$.
\end{lemma}
\begin{proof} 
      This follows from the unique description of any arrow in $G \ltimes \mathcal E$ as an element of $G$ plus an arrow in $\mathcal E$. 
\end{proof}

\begin{remark}
      For $G$-categories $\mathcal C$ and $\mathcal D$, we have equivalences of categories
      \begin{equation}
            \Fun^G(\mathcal C, \mathcal D)
            \simeq \Fun_{\Fib(G)}(G \ltimes \mathcal C, G \ltimes \mathcal D)
            \simeq \Fun_{\Fib(G^{op})}(G^{op} \ltimes \mathcal C, G^{op} \ltimes \mathcal D).
      \end{equation}
\end{remark}



\subsection{2-overcategories and pullback functors}

Include \S 8.1: 2-overcategories and \S 8.3: Pullback functors.

Random lemmas:

\begin{lemma}
      Suppose $F: \mathcal C \to \mathcal D$ is a simple 1-arrow in $\Cat \downarrow^r \mathcal B$.
      Consider the following cube in $\Cat$, where the bottom, top, and right faces are all (strict) pullbacks.
      \begin{equation}
            \begin{tikzcd}[row sep = small, column sep = small]
                  \pi^{\**} \mathcal C \arrow[rr] \arrow[dr] \arrow[dd]
                  &&
                  \mathcal E \arrow[dr, "\pi"] \arrow[dd, equal]
                  \\
                  &
                  \mathcal C \arrow[rr, crossing over]
                  &&
                  \mathcal B \arrow[dd, equal]
                  \\
                  \pi^{\**} \mathcal D \arrow[rr] \arrow[dr]
                  &&
                  \mathcal E \arrow[dr, "\pi"]
                  \\
                  &
                  \mathcal D \arrow[uu, crossing over, leftarrow] \arrow[rr]
                  &&
                  \mathcal B
            \end{tikzcd}
      \end{equation}
      Then the left square is also a pullback.
\end{lemma}
\begin{proof}
      We check that $\pi^{\**}\mathcal C$ has the correct universal property: for any category $Z$, we have
      \begin{align*}
        \Hom(Z,\mathcal C) \times_{\Hom(Z,\mathcal D)} \Hom(Z, \pi^{\**} \mathcal C)
        & =
          \Hom(Z, \mathcal C) \times_{\Hom(Z, \mathcal D)}\left(
          \Hom(Z, \mathcal D) \times_{\Hom(Z, \mathcal B)} \Hom(Z, \mathcal E)
          \right)
          \\
        & =
          \Hom(Z, \mathcal C) \times_{\Hom(Z, \mathcal B)}\Hom(Z, \mathcal E)
          = \Hom(Z, \pi^{\**}\mathcal C).
      \end{align*}
\end{proof}

\begin{corollary}
      \label{PI_GFIB_COR}
      If a simple 1-arrow $F:\mathcal C \to \mathcal D$ in $\Cat \downarrow^r \mathcal B$
      is an underlying Grothendieck fibration,
      then so is $\pi^{\**}F$.
\end{corollary}
\begin{proof}
      As Grothendieck fibrations are preserved by pullbacks, this follows from the previous lemma.
\end{proof}

\begin{lemma}
      \label{RANINIT_PULL_LEM}
      Suppose we have a triangle
      \begin{equation}
            \begin{tikzcd}
                  \mathcal C \arrow[rr, hookrightarrow] \arrow[dr, "p"] \arrow[ddr, bend right, "\delta"', ""{near end, name = V}]
                  &&
                  \mathcal D \arrow[dl, "p"'] \arrow[ddl, bend left, "\delta", ""'{near end, name = B}]
                  \\
                  &
                  |[alias = A]| \mathcal E \arrow[d, "\epsilon"]
                  \\
                  &
                  \mathsf F
                  \arrow[Rightarrow, from = A, to = B, "\Phi"]
                  \arrow[Rightarrow, from = A, to = V, "\Phi"']
            \end{tikzcd}
      \end{equation}
      in $\Cat \downarrow^r \mathsf F$,
      such that $\mathcal C \into \mathcal D$ is an inclusion of a subcategory by a simple 1-arrow.
      If $\mathcal C \to \mathcal D$ is $\Ran$-initial over $\mathcal E$,
      then $\mathcal C_{\mathfrak C} \to \mathcal D_{\mathfrak C}$ is $\Ran$-initial over $\mathcal E_{\mathfrak C}$.
\end{lemma}
\begin{proof}
      This is just a matter of unpacking definitions.
      Fixing some $(e, \epsilon(e) \xrightarrow{t} \mathfrak C)$ in $\mathcal E_{\mathfrak C}$,
      and $\ki = \begin{cases}
            (d, \delta(d) \xrightarrow{r} \mathfrak C)
            \\
            (e \longto p(d))
      \end{cases}$
      in $(e,t) \downarrow \mathcal D_{\mathfrak C}$,
      so in particular the triangle below commutes.
      \begin{equation}
            \begin{tikzcd}[row sep = small,column sep = tiny]
                  \epsilon(e) \arrow[rr] \arrow[ddr]
                  &&
                  \epsilon p(d) \arrow[d, "\Phi"]
                  \\
                  &&
                  \delta(d) \arrow[dl]
                  \\
                  &
                  \mathfrak C
            \end{tikzcd}
      \end{equation}
      We must show that $\left((e,t) \downarrow \mathcal C_{\mathfrak C} \right) \downarrow \ki$ is non-empty and connected.

      First, by hypothesis we know there exists $c \in \mathcal C$, $c \xrightarrow{f} d$, and $e \to p(c)$ such that the obvious triangle commutes.
      Now, the object 
      $\begin{cases}
            (c, \delta(c) \xrightarrow{\delta(f)} \delta(d) \to \mathfrak C)
            \\
            (e \longto p(c)
      \end{cases}$
      in $(e,t) \downarrow \mathcal C_{\mathfrak C}$ clearly maps to $\ki$.

      Second, any such object over $\ki$ must factor this way, and hence the connectedness of $(e \downarrow C) \downarrow (d,r)$
      implies the desired connectedness.
      \begin{equation}
            \begin{tikzcd}
                  &
                  \epsilon(e) \arrow[dr] \arrow[dl]
                  &
                  && %                  
                  &
                  \epsilon(e) \arrow[dr] \arrow[dl] \arrow[dd]
                  \\
                  \epsilon p(c) \arrow[rr, "{\epsilon p (f)}"] \arrow[d, "\Phi"]
                  &&
                  \epsilon p(d) \arrow[d, "\Phi"]
                  && %
                  \epsilon p (c) \arrow[dr, "{\epsilon p (f)}"'] \arrow[dd, "\Phi"'] \arrow[rr, dashed]
                  &&
                  \epsilon p (c') \arrow[dl, "{\epsilon p (f')}"] \arrow[dd, "\Phi"]
                  \\
                  \delta(c) \arrow[rr, "{P\delta(f)}"] \arrow[dr, "{r \circ \delta(f)}"']
                  &&
                  \delta(d) \arrow[dl, "r"]
                  && %
                  &
                  \epsilon p (d) \arrow[dd]
                  \\
                  &
                  \mathfrak C
                  &
                  && %
                  \delta (c) \arrow[dr, "{\delta(f)}"] \arrow[ddr] \arrow[rr, dashed]
                  &&
                  \delta(c') \arrow[dl, "{\delta(f')}"'] \arrow[ddl]
                  \\
                  &&&& %
                  &
                  \delta(d) \arrow[d]
                  \\
                  &&&& %
                  &
                  \mathfrak C
            \end{tikzcd}
      \end{equation}
\end{proof}

