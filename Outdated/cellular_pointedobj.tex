
\section{Random Stuff}

{\color{OliveGreen} % -------------------------------------------------- OLIVE GREEN ----------------------------------------
  \subsection{Categories of pointed objects}
  \label{PT_SEC}

  \todo[inline]{not really needed - none of these arguments are too difficult, and they are not needed for the narrative}

  To prove this, we take a brief digression on categories of pointed objects in a category,
  culminating with Corollar \ref{PT_MODEL_COR}

  In this short section, we show that, in some cases,
  if $\Op^G_\F(\V)$ has the (semi)-model structure from Theorem \ref{MODEL_THM}, then so does $\Op^G_\F(\V_{\**})$.


  Given $(\V, \otimes)$, the category of pointed objects $\V_{\**} = \** \downarrow \V$ with $\**$ terminal
  is also monoidal via a smash product (see e.g. the proof of \cite[Prop. 4.2.9]{Hov99}).

  % \begin{definition}
  %       [{\cite[Construction 4.19]{EM09}}]
  %       Given $X, Y \in \V_{\**}$, define $X \wedge Y$ to be the pushout
  %       \begin{equation}
  %             \begin{tikzcd}
  %                   X \otimes \** \amalg \** \otimes Y \arrow[d] \arrow[r]
  %                   &
  %                   X \otimes Y \amalg X \otimes Y \arrow[r]
  %                   &
  %                   X \otimes Y \arrow[d]
  %                   \\
  %                   \** \arrow[rr]
  %                   &&
  %                   X \wedge Y.
  %             \end{tikzcd}
  %       \end{equation}
  %       This recovers the smash product of pointed spaces and simplicial sets.
  % \end{definition}

  We recall the following.

  \begin{lemma}
        If $\V \rightleftarrows \mathcal W$ is a strong symmetric monoidal adjunction, then
        the induced pair $\Cat^{\mathfrak C}(\V) \rightleftarrows \Cat^{\mathfrak C}(\mathcal W)$ is an adjunction.
  \end{lemma}

  \begin{lemma}
        The adjunction
        \[
              (-) \amalg \**: (\V,\otimes) \rightleftarrows (\V_{\**}, \wedge) : \mathrm{fgt}
        \]
        is strong symmetric monoidal.

        If $\V$ is a model category, then so is $\V_{\**}$, where $f \in \V_{\**}$ is a weak equivalence (resp. fibration, cofibration) iff
        $U(f)$ is so in $\V$ % \cite[Lemma 1.1.8]{Hov99}.
  \end{lemma}

  Many properties of $\V$ descend to $\V_{\**}$. 

  \begin{lemma}
        \label{PT_PROP_LEM}
        The following hold. 
        \begin{enumerate*}[label = (\roman*)]
        \item If $\V$ is cofibrantly generated, then so is $\V_{\**}$ \cite[Lemma 2.1.21]{Hov99},
        \item If $\V$ is a monoidal model category with $1_\V = \**$ and $1_\V$ is cofibrant, then $\V_{\**}$ is a monoidal model category \cite[Lemma 4.2.9]{Hov99}
        \item If $\V$ has a cofibrant unit and $\**$ is cofibrant, then $\V_{\**}$ has a cofibrant unit $1_\V amalg \**$.
        \item left or right proper \cite{Hir}
        \item If $\V$ has cofibrant symmetric pushout powers, then so does $\V_{\**}$ (since pushouts in $\V_{\**}$ are computed in $\V$).
        \end{enumerate*}
  \end{lemma}

  \begin{corollary}
        If $\Op^{\mathfrak C}(\V)$ has the (semi)-model structure from Theorem \ref{THM1_C}, then so does $\Op^{\mathfrak C}(\V_{\**})$.     
  \end{corollary}

  \begin{lemma}
        Suppose $\V$ is a cofibrantly generated monoidal model category with cofibrant unit and cofibrant symmetric pushout powers.
        Then
        $\Cat^{\mathfrak C}(\V) \rightleftarrows \Cat^{\mathfrak C}(\V_{\**})$ is a Quillen pair of (semi)-model categories.
  \end{lemma}

  We would like $\V_{\**}$ to additionally inherit cellular fixed points from $\V$.
  While this will \textit{not} necessarily be the case, something weaker is always true.
  \begin{definition}
        A model category $\mathcal M$ has \textit{weakly cellular fixed points} if
        $\mathcal M$ satisfies conditions $(i)$ and $(ii)$ of cellularity (Defn. \ref{CELLFP_DEF}),
        and additionally the composite functor $(G/K \cdot (-))^H$ sends generating (trivial) cofibrations to (trivial) cofibrations.

        It is clear that all model categories will cellular fixed points have weakly cellular fixed points.
  \end{definition}

  \begin{remark}
        By \cite[Rem. 2.8]{Ste16}, $\mathcal M$ having weakly cellular fixed points is sufficient for the 
        $\F$-model structure on $\mathcal M^G$ to exist for any family $\F$ of subgroups of $G$.
  \end{remark}

  \begin{remark}
        \label{WEAKCELL_REM}
        Conveniently, at no place in our proof of Theorem \ref{MODEL_THM} (including importantly Theorem I and II from \cite{BP_geo})
        did we ever use the full strength of cellular fixed points,
        and instead we only ever required the existance of the $\F$-model structures.
        The reason we (and \cite{Ste16}) use cellular fixed points is that the full condition (iii) is what allows us to say anything meaningful about the \textit{cofibrant} objects in our category (see, e.g. \cite[Prop. 6.56 and Lemma 6.59]{BP_geo}).
  \end{remark}


  \begin{lemma}
        \label{PT_CELL_LEM}
        Suppose $\V$ is a  cofibrantly generated monoidal model category with weakly cellular fixed points.
        Then $\V_{\**}$ has weakly cellular fixed points.
        If in addition $\V$ has cellular fixed points and all monomorphisms are cofibrations,
        \footnote{
          In fact, we only need that all possible maps $\** \to X$ are cofibrations.
        }.
        then $\V_{\**}$ also has cellular fixed points.
  \end{lemma}
  \begin{proof}
        We note that $(-)^H \circ U = U \circ (-)^H$,
        and that colimits in $\V_{\**}$ can be computed in $\V$ as long as we add an initial object to the indexing diagram (and send it to the zero object in $\V_{\**}$).
        Thus, for any directed poset $\mathcal D$, with $\mathcal D^\triangleleft$ denoting $\mathcal D$ with a free initial object,
        we have the following diagram, where all but the left side of the square are known to commute.
        \begin{equation}
              \begin{tikzcd}[row sep = tiny, column sep = tiny]
                    \Fun(\mathcal D, V^G_{\**}) \arrow[r]
                    &
                    \Fun_{\**}(\mathcal D^\triangleleft, \V^G_{\**}) \arrow[rr] \arrow[dd] \arrow[dr]
                    &&
                    \Fun_{\**}(\mathcal D^\triangleleft, \V^G) \arrow[dd] \arrow[dr]
                    \\
                    &&
                    \Fun_{\**}(\mathcal D^\triangleleft, \V_{\**}) \arrow[rr, crossing over]
                    &&
                    \Fun_{\**}(\mathcal D^\triangleleft, \V) \arrow[dd]
                    \\
                    &
                    \V^G_{\**} \arrow[rr] \arrow[dr]
                    &&
                    \V^G \arrow[dr]
                    \\
                    &&
                    \V_{\**} \arrow[rr] \arrow[uu, leftarrow, crossing over]
                    &&
                    \V
              \end{tikzcd}
        \end{equation}
        Hence $(i)$ of cellularity (Defn. \ref{CELLFP_DEF}) follows.
        
        Moreover, since pushouts and cofibrations are computed in $\V$, $(ii)$ follows immediately.
        
        For $(iii)$, the general case follows since
        the generating (trivial) cofibrations of $\V_{\**}$ are just $f_+$ for $f$ a generating (trivial) cofibration in $\V$,
        and the following equation holds.
        \begin{equation}
              (G/K \cdot_{\**} (-)_+)^H \cong \left( (G/K \cdot (-))_+ \right)^H \cong (G/K \cdot (-))^H_+
        \end{equation}

        With the additional hypothesis, we note that for $A \in \Set$ and $X \in \V_{\**}$, the canonical copowering $A \cdot_{\**} X$ in $\V_{\**}$ is given by the diagram on the left below.
        Thus, using $(ii)$ for $\V$ and our assumption that $\** \to X$ is a cofibration, $(iii)$ follows by the diagram on the right,
        where both the front and back panels are pushouts, the back by definition, and the front by our assumption that every $\** \to X$ is a cofibration (as these are always categorical monomorphisms) and cellularity condition $(ii)$ for $\V$.      
        \begin{equation}
              \begin{tikzcd}[row sep = tiny]
                    A \cdot \** \arrow[d] \arrow[r]
                    &
                    A \cdot X \arrow[d]
                    & % ----------
                    (G/K)^H \cdot \** \arrow[dd] \arrow[rr] \arrow[dr, "\cong"']
                    &[-15pt]&[-15pt]
                    (G/K)^H \cdot X \arrow[dd] \arrow[dr, "\cong"]
                    \\
                    \** \arrow[r]
                    &
                    A \cdot_{\**} X
                    & % ----------
                    &
                    (G/K \cdot \**)^H \arrow[rr, crossing over]
                    &&[-15pt]
                    (G/K \cdot X)^H \arrow[dd]
                    \\
                    && % ----------
                    \** \arrow[rr] \arrow[dr, equal]
                    &&
                    (G/K)^H \cdot_{\**} X \arrow[dr, dashed, "\cong"]
                    \\
                    && % ----------
                    & \**^H \arrow[rr] \arrow[uu, leftarrow, crossing over]
                    &&
                    (G/K \cdot_{\**} X)^H
              \end{tikzcd}
        \end{equation}
  \end{proof}

  To complete Theorems \ref{MODEL_THM} and \ref{INTRO_MODEL_THM} for $\V_{\**}$, we need to check the existance of $\V_{\**}$-intervals and the inheritance of the coherence axiom.

  \begin{proposition}
        \label{VPT_INT_PROP}
        Suppose $\V$ is a cofibrantly generated monoidal model category with cofibrant unit and cofibrant symmetric pushout powers,
        and moreover suppose $(-)_+$ preserves all weak equivalences.
        Then $\J_+$ is a $\V_{\**}$-interval for any $\V$-interval $\J$.
  \end{proposition}
  \begin{proof}
        Given the composite on the left below, we produce the composite on the right
        \begin{equation}
              \begin{tikzcd}
                    \varnothing \arrow[r, rightarrowtail]
                    &
                    \J \arrow[r, "\simeq"]
                    &
                    \I_f
                    & % ----------
                    \** \arrow[r, rightarrowtail]
                    &
                    \J_+ \arrow[r]
                    &
                    (\I_f)_+ \arrow[r, dashed, "\simeq"]
                    &
                    (\I_+)_f
              \end{tikzcd}
        \end{equation}
        by applying $(-)_+$ and using the lifting from the square
        \begin{equation}
              \label{IFP_IPF_EQ}
              \begin{tikzcd}
                    \I_+ \arrow[d, rightarrowtail, "\simeq"'] \arrow[r, rightarrowtail, "\simeq"]
                    &
                    (\I_+)_f \arrow[d, twoheadrightarrow]
                    \\
                    (\I_f)_+ \arrow[ur, dashed, "\simeq"] \arrow[r]
                    &
                    \**.
              \end{tikzcd}
        \end{equation}
        Since $(-)_+$ also preserves all weak equivalences, $\J_+$ is a $\V_{\**}$-interval, as desired.
  \end{proof}

  \begin{proposition}
        \label{PT_COH_PROP}
        Suppose $\V$ is a cofibrantly generated monoidal model category with cofibrant unit and cofibrant symmetric pushout powers,
        and moreover suppose $(-)_+$ preserves all weak equivalences.
        If $\V$ satisfies the coherence axiom, then so does $\V_{\**}$.
  \end{proposition}
  \begin{proof}
        Fix a map $\alpha:\mathbb A_+ \to \mathcal C_f$ detecting a homotopy equivalence in $\Cat(\V_{\**})$.
        By coherence of $\V$, it's adjoint map $\tilde \alpha$
        factors through a natural cofibration $\mathbb A \rightarrowtail \J$.
        Applying $(-)_+$ yields the diagram below, for which the top arrow is again $\alpha$ by adjointness.
        \begin{equation}
              \begin{tikzcd}
                    \mathbb A_+ \arrow[d, rightarrowtail] \arrow[r, "\tilde{\alpha}_+"]
                    &
                    (U \mathcal C_f)_+ \arrow[r]
                    &
                    \mathcal C_f
                    \\
                    \J_+ \arrow[ur, dashed]
              \end{tikzcd}
        \end{equation}
        By Proposition \ref{VPT_INT_PROP}, it now suffices to show that $(-)_+$ preserves natural cofibration.
        This too holds by the following diagram, where the bottom right horizontal map is from \eqref{IFP_IPF_EQ}.
        \begin{equation}
              \begin{tikzcd}
                    \mathbb A_+ \arrow[d, rightarrowtail] \arrow[r]
                    &
                    \I_+ \arrow[d, rightarrowtail, "\simeq"] \arrow[r, equal]
                    &
                    \I_+ \arrow[d, rightarrowtail, "\simeq"]
                    \\
                    \J_+ \arrow[r, "\simeq"]
                    &
                    (\I_f)_+ \arrow[r, "\simeq"]
                    &
                    (\I_+)_f
              \end{tikzcd}
        \end{equation}
  \end{proof}

  \begin{lemma}
        \label{P_P_WE_LEM}
        Suppose $\V$ is a cofibrantly generated monoidal model category with cofibrant unit and cofibrant symmetric pushout powers,
        and moreover suppose $\V$ is left proper and $\**$ is cofibrant in $\V$.
        Then $(-)_+: \V \to \V_{\**}$, and hence $\Cat^{\mathfrak C}(\V) \to \Cat^{\mathfrak C}(\V_{\**})$, preserves all weak equivalences.
  \end{lemma}
  \begin{proof}
        This follows by considering the pushout square below.
        \begin{equation}
              \begin{tikzcd}
                    A \arrow[r, rightarrowtail] \arrow[d, "\simeq"']
                    &
                    A \amalg \** \arrow[d, "\simeq"]
                    \\
                    B \arrow[r, rightarrowtail]
                    &
                    B \amalg \**
              \end{tikzcd}
        \end{equation}
  \end{proof}

  \begin{corollary}
        \label{PT_MODEL_COR}
        Suppose $\V$ satisfies the hypotheses of Theorem \ref{MODEL_THM} (resp. Theorem \ref{INTRO_MODEL_THM}),
        possibly replacing ``cellular'' with ``weakly cellular'',
        and moreover that $\V$ is left proper and $1_\V = \**$.
        Let $\F$ be any $(G, \Sigma)$-family.
        Then $\Op^G(\V_{\**})$ has the $\F$-(semi)-model structure (resp. $\F$-Dwyer-Kan (semi)-model structure).
  \end{corollary}
  \begin{proof}
        Combine Lemmas \ref{PT_PROP_LEM} and \ref{P_P_WE_LEM}, Remark \ref{WEAKCELL_REM}, and Proposition \ref{PT_COH_PROP}.
  \end{proof}

} % ---------------------------------------- OLIVE GREEN ----------------------------------------


\subsection{Pullback functor and the geometric realization}

In this short section, we will prove that the pullback 2-functor $\pi^{\**}$ and geometric realization $|-|$ commute.

\begin{remark}
      \label{REAL_CAT_REM}
      We recall that $|\mathcal C_\bullet|$ has object set $\Ob(\mathcal C_0)$,
      arrows generated by
      maps $f_0: a_0 \to \bar a_0$ in $\mathcal C_0$ and objects $a_1: d_1(a_1) \xrightarrow{a_1} d_0(a_1)$ in $\mathcal C_1$,
      and relations given by information in just $\mathcal C_0$, $\mathcal C_1$, and $\mathcal C_2$:
      \begin{enumerate}[label = (\roman*)]\setcounter{enumi}{-1}
      \item $[f_0] \circ [g_0] = [f_0 \circ g_0]$,
      \item $[id_{a_0}] = [\ ]_{a_0}$,
      \item $[s_0a_0] = [id_{a_0}]$,
      \item for all $f_1: a_1 \to b_1$ in $\mathcal C_1$, the following diagram commutes.
            \begin{equation}
                  \begin{tikzcd}
                        d_1 a_1 \arrow[r, "a_1"] \arrow[d, "d_1 f_1"']
                        &
                        d_0 a_1 \arrow[d, "d_0 f_1"]
                        \\
                        d_1 b_1 \arrow[r, "b_1"]
                        &
                        d_0 b_1
                  \end{tikzcd}
            \end{equation}
      \item for all $a_2 \in \mathcal C_2$, the following diagram commutes.
            \begin{equation}
                  \begin{tikzcd}
                        d_1 d_2 a_2 = d_1 d_1 a_2 \arrow[dr, "d_2 a_2"'] \arrow[rr, "d_1 a_2"]
                        &&
                        d_0 d_1 a_2 = d_0 d_0 a_2
                        \\
                        &
                        d_0 d_2 a_2 = d_1 d_0 a_2 \arrow[ur, "d_0 a_2"]
                  \end{tikzcd}
            \end{equation}
      \end{enumerate}

      See \cite[Remark A.5]{BP_geo} for more details.
\end{remark}

\begin{proposition}
      \label{PIREAL_PROP}
      Let $E_\bullet: \mathcal C_\bullet \to \mathcal B$ be a simplicial object in $\Cat \downarrow^r \mathcal B$
      such that the natural transformation component $\phi_0$ of the structure map $d_0: \mathcal C_1 \to \mathcal C_0$
      is an isomorphism.

      Then we have a map out of the realization $E: |\mathcal C_\bullet| \to \mathcal B$,
      acting by $E_0$ on objects and defined on the generating arrows as follows:
      $f_0 \in \mathcal C_0$ goes to $E_0(f_0)$, and $a_1 \in \mathcal C_1$ goes to the composite
      \begin{equation}
            E_0 d_0(a_1) \xrightarrow{\phi_1} E_1(a_1) \xrightarrow{\phi_0^{-1}} E_0 d_0(a_1).
      \end{equation}
      Moreover, for any (split) Grothendieck fibration $\pi: \mathcal E \to \mathcal B$,
      the map $|\pi^{\**} \mathcal C_\bullet| \to \pi^{\**} |\mathcal C_\bullet|$ is an isomorphism.
\end{proposition}
\begin{proof}
      This result is not challenging, but requires carefully checking all the pieces fit together.
      
      For the first claim, it suffices to check that the map described above is well-defined with respect to the relations on the set of arrows in $|\mathcal C_\bullet|$, and this is not hard to verify directly.
      {\color{OliveGreen}
        Conditions (0) - (ii) are straightforward, $(iii)$ follows as $\phi_0$ and $\phi_1$ are natural transformations,
        and $(iv)$ follows as the following diagram commutes by the simplicial identities.
        \begin{equation}
              \begin{tikzcd}
                    E_0 d_1 d_2 a_2 \arrow[d, equal] \arrow[r, "\phi_1"]
                    &
                    E_1 d_2 a_2 \arrow[dd, "\phi_2"]
                    &
                    E_0 d_0 d_2 a_2 \arrow[d, equal] \arrow[l, "\phi_0"]
                    \\
                    E_0 d_1 d_1 a_2 \arrow[d, "\phi_1"]
                    &&
                    E_0 d_1 d_0 a_2 \arrow[d, "\phi_1"]
                    \\
                    E_1 d_1 a_2 \arrow[r, "\phi_1"]
                    &
                    E_2 a_2
                    &
                    E_1 d_0 a_2 \arrow[l, "\phi_0"]
                    \\
                    &&
                    E_0 d_0 d_0 a_2 \arrow[u, "\phi_0"] \arrow[d, equal]
                    \\
                    E_0 d_0 d_1 a_2 \arrow[uu, "\phi_0"] \arrow[r, "\phi_0"]
                    &
                    E_1 d_1 a_2 \arrow[uu, "\phi_1"]
                    &
                    E_0 d_0 d_1 a_2 \arrow[l, "\phi_0"]
              \end{tikzcd}
        \end{equation}
      }

      Now, both $|\pi^{\**} \mathcal C_\bullet|$ and $\pi^{\**}|\mathcal C_\bullet|$ (which we now know is well-defined)
      have $\Ob(\mathcal C_0) \times_{\Ob(\mathcal B)} \Ob(\mathcal E)$ as their set of objects.
      For maps, we see the category $|\pi^{\**} \mathcal C_\bullet|$ has arrows generated by
      \begin{enumerate}[label = (\Roman*)]
      \item arrows in $\pi^{\**} \mathcal C_0$, namely pairs of arrows $(f_0,\alpha)$ compatible over $\mathcal B$, and
      \item formal arrows $a_1 \wedge e$ for each object $(a_1,e)$ in $\pi^{\**}\mathcal C_1$, which are of the form
            \begin{equation}
                  d_1(a_1,e) = (d_1(a_1), (\phi_1\pi)^{\**}e) \xrightarrow{(a_1,e)} (d_0(a_1), (\phi_0 \pi)^{\**} e).
            \end{equation}
      \end{enumerate}
      These are subject to conditions analogous to those from Remark \ref{REAL_CAT_REM}.
      {\color{OliveGreen}
        \begin{enumerate}[label = (\roman*)]\setcounter{enumi}{-1}
        \item $[(f_0, \alpha)] \circ [(f_0', \alpha')] = [(f_0 f_0', \alpha \alpha')]$,
        \item $[(id_{a_0}, id_e)] = [\ ]_{(a_0,e)}$,
        \item $[s_0 a_0 \wedge e] = [(id_{a_0}, id_e)]$,
        \item For all $(f_1, \alpha): (a_1, e) \to (b_1, \bar e)$ in $\pi^{\**} \mathcal C_1$,
              the following diagram commutes.
              \begin{equation}
                    \begin{tikzcd}
                          d_1(a_1, e) = (d_1 a_1, \phi_1^{\**} e) \arrow[d, "{d_1(f_1, \alpha) = (d_1 f_1, \phi_1^{\**} \alpha)}"'] \arrow[r, "a_1 \wedge e"]
                          &
                          d_0(a_1,e)  (d_0 a_1, \phi_0^{\**} e) \arrow[d, "{d_0(f_1, \alpha) = (d_0 f_1, \phi_0^{\**} \alpha)}"]
                          \\
                          d_1(b_1, \bar e) = (d_1 b_1, \phi_1^{\**} \bar e) \arrow[r, "b_1 \wedge \bar e"]
                          &
                          d_0(b_1, \bar e) = (d_0 b_1, \phi_0^{\**} \bar e).
                    \end{tikzcd}
              \end{equation}
        \item For all $(a_2, e) \in \pi^{\**} \mathcal C_2$, the following diagram commutes.
              \begin{equation}
                    \begin{tikzcd}
                          d_1 d_2 (a_2, e) = d_1 d_1 (a_2, e) \arrow[rr, "d_1 a_2 \wedge \phi_1^{\**} e"] \arrow[dr, "d_2 a_2 \wedge \phi_2^{\**} e"]
                          &&
                          d_0 d_1(a_2, e) = d_0 d_0(a_2, e)
                          \\
                          &
                          d_0 d_2 (a_2, e) = d_1 d_0 (a_2, e) \arrow[ur, "d_1 a_2 \wedge \phi_1^{\**} e"]
                    \end{tikzcd}
              \end{equation}
        \end{enumerate}
      }

      Conversely, maps in $\pi^{\**}|\mathcal C_\bullet|$ are simply pairs of arrows $(f, \alpha)$ in $|\mathcal C_\bullet| \times_{\mathcal B} \mathcal E$.

      We have a natural map $F: |\pi^{\**} \mathcal C_\bullet| \to \pi^{\**} |\mathcal C_\bullet|$, which is the identity on objects,
      and acts as follows on the generating arrows:
      $(f_0, \alpha)$ gets mapped to $([f_0], \alpha)$, while
      $a_1 \wedge e$ gets mapped to $([a_1], \phi_1^{\**} e \to e \to \phi_0^{\**} e)$
      (where here we are again using the $\phi_0$ is invertible).
      It is straightforward to verify that this map is well-defined.      

      We will use the fact that $\pi$ is a split Grothendieck fibration in order to build an inverse $G$ of the functor $F$.
      Namely, given any word $W: a_0 \to b_0$ of composable generating arrows from $\pi^{\**}|\mathcal C_{\**}|$,
      any map $\alpha: \bar e \to e$ lifting $E([W])$ has an induced factorization into
      a map $W^{\**}\alpha$ over $E_0(a_0)$, followed by a composite of cartesian arrows $\alpha_i$, one for each letter $w$ in $W$.

      Now, we send $([W],\alpha)$ to the composite of $|W|+1$ arrows defined as follows.
      If $w_i = (a_0 \xrightarrow{c_0} b_0)$ and the target of $\alpha_i$ is $e$, then the $(i+1)$-st arrow is $(f_0, f_0^{\**}<e>)$.
      If $w_i = a_1$, then the $(i+1)$-st arrow is $a_1 \wedge (\phi_0^{\-1})^{\**} e$.
      Finally, we precompose with $(id_{a_0}, W^{\**}\alpha)$.

      As an example, suppose we have a word $W$ equal to the top row below, so $E(W)$ is as in the second row,
      and an arrow $\alpha: \bar e \to e$ in $\mathcal E$ such that $\pi(\alpha) = E(W)$.
      \begin{equation}
            \begin{tikzcd}[column sep = small]
                  W:
                  &[-20pt]
                  &[20pt]
                  a_0 \arrow[r, "f_0"]
                  &[20pt]
                  b_0 = d_1 a_1 \arrow[rr, "a_1"]
                  &&
                  d_0 a_1
                  \\
                  E(W):
                  &
                  &
                  E_0 a_0 \arrow[r, "E_0 f_0"] % \arrow[d, equal]
                  &
                  E_0 b_0 = E_0 d_1 a_1 \arrow[r, "\phi_1"]
                  &
                  E_1 a_1 \arrow[r, "{\phi_0^{-1}}"]
                  &
                  E_0 d_0 a_1 % \arrow[d, equal]
                  % \\
                  % \pi(\bar e) \arrow[r, equal]
                  % &
                  % \pi(\bar e) \arrow[rrr, "{\pi(\alpha)}"]
                  % &&&
                  % \pi(e)
                  \\
                  &
                  \bar e \arrow[r, dashed, "{\exists ! W^{\**} \alpha}"]
                  &
                  f_0^{\**} (\phi_0^{-1} \phi_1)^{\**} e \arrow[r]
                  &
                  (\phi_0^{-1} \phi_1)^{\**} e \arrow[rr]
                  &&
                  e
                  \\
                  G([W], \alpha):
                  &
                  (a_0, \bar e) \arrow[r, "{(id, W^{\**} \alpha)}"]
                  &
                  (a_0, f_0^{\**} (\phi_0^{-1} \phi_1)^{\**} e) \arrow[r, "{(f_0, f_0^{\**})}"]
                  &
                  (b_0 = d_1 a_1, (\phi_0^{-1} \phi_1)^{\**} e) \arrow[rr, "a_1 \wedge \phi_0^{-1,\**} e"]
                  &&
                  (d_0 a_1, e)
            \end{tikzcd}
      \end{equation}
      The word $E(W)$ induces a factorization of $\alpha$ into two chosen cartesian maps, along with a unique map $W^{\**}\alpha$ determined by the cartesian maps.
      With this factorization, the last line displays the image of the arrow $([W], \alpha)$ in $\pi^{\**}|\mathcal C_\bullet|$. 
      
      Now, it is clear that these functors are invertible, assuming $G$ is in fact functorial.
      We begin by showing that $G$ behaves well under composition.
      It suffices to show that these end maps $W^{\**}\alpha$ ``commute'' with the more structured maps.
      We have two cases, one for each type of generating arrow.
      First, suppose we have a word/map $W = f_0: a_0 \to b_0$ in $\mathcal C_0$, and two maps
      \begin{equation}
            \bar{\bar e} \xrightarrow{\beta} \bar e \xrightarrow{\alpha} e            
      \end{equation}
      in $\mathcal E$, with $\alpha \in \mathcal E_{E_0 a_0}$ and $\pi(\alpha\beta) = E_0 f_0$.
      We must show that the composites
      \begin{align*}
        (a_0, \bar{\bar e}) & \xrightarrow{(id, W^{\**}\beta)}
                              (a_0, f_0^{\**} \bar e) \xrightarrow{(f_0, f_0^{\**})}
                              (b_0, \bar e) \xrightarrow{(id, \alpha)}
                              (b_0, e)
        \\
        (a_0, \bar{\bar e}) & \xrightarrow{(id, W^{\**}(\alpha\beta))}
                              (a_0, f_0^{\**} e) \xrightarrow{(f_0, f_0^{\**})}
                              (b_0, e)
      \end{align*}
      agree. However, these are all arrows of type (I), and so the composites are both $(f_0, \alpha\beta)$.

      Second, suppose we have a word/map $W = a_1$, with $\alpha$, $\beta$ as before such that $\alpha \in \mathcal E_{E_0 d_0 a_1}$ and $\pi(\alpha\beta) = E_0 a_1$.
      Then we have the following diagram
      \begin{equation}
            \begin{tikzcd}[column sep = large]
                  (d_1 a_1, \bar{\bar e}) \arrow[r, "{(id, W^{\**} \beta)}"] \arrow[dr, "{(id, W^{\**}(\alpha\beta))}"']
                  &
                  (d_1 a_1, (\phi_0^{-1} \phi_1)^{\**} \bar e) \arrow[d, "{(id, (\phi_0^{-1} \phi_1)^{\**} \alpha)}"] \arrow[r, "a_1 \wedge \phi_0^{-1,\**} \bar e"]
                  &
                  (d_0 a_1, \bar e) \arrow[d, "{(id, \alpha)}"]
                  \\
                  &
                  (d_1 a_1, (\phi_0^{-1} \phi_1)^{\**} e) \arrow[r, "a_1 \wedge \phi_0^{-1,\**} e"']
                  &
                  (d_0 a_1, e).
            \end{tikzcd}
      \end{equation}
      This commutes by relation (iii) in $|\pi^{\**} \mathcal C_\bullet|$ on the map $(id_{a_1}, \phi_0^{-1,\**}\alpha)$,
      as desired.

      It remains to check that $G$ is independent of the choice of word representative. We will show that the relations all hold in the image of $G$.
      (i) through (iii) are straightforward from the fact the composition works as above.
      For (iii), given $f_1: a_1 \to b_1$, we have two word decompositions
      \begin{equation}
            W:
            \quad 
            d_1 a_1 \xrightarrow{a_1} d_0 a_0 \xrightarrow{d_0 f_1} d_0 b_1,
            \qquad \qquad
            V:
            \quad
            d_1 a_1 \xrightarrow{d_1 f_1} d_1 b_1 \xrightarrow{b_1} d_0 b_1
      \end{equation}
      whose images under $E$ are equal, and we must check that for any $\alpha: \bar e \to e$ over this image,
      the images under $G$ are equal:
      \begin{equation}
            \label{IMUG_EQ}
            \begin{tikzcd}[row sep = small]
                  (d_1 a_1, \bar e) \arrow[r, "{(id, W^{\**}\alpha)}"]
                  &[15pt]
                  (d_1 a_1, (\phi_0^{-1}\phi_1)^{\**} (d_0f_1)^{\**} e) \arrow[r, "{a_1 \wedge \phi_0^{-1,\**} (d_0 f_1)^{\**} e}"]
                  &[35pt]
                  (d_0 a_1, (d_0 f_1)^{|**} e) \arrow[r, "{(d_0f_1, d_0 f_1^{\**})}"]
                  &[20pt]
                  (d_0 b_1, e)
                  \\
                  (d_1 a_1, \bar e) \arrow[r, "{(id, V^{\**}\alpha)}"]
                  &
                  (d_1 a_1, (d_1f_1)^{\**} (\phi_0^{-1} \phi_1)^{\**} e) \arrow[r, "{(d_1 f_1, d_1 f_1^{\**})}"]
                  &
                  (d_1 b_1, (\phi_0^{-1} \phi_1)^{\**} e) \arrow[r, "{b_1 \wedge \phi_0^{-1, \**} e}"]
                  &
                  (d_0 b_1, e).
            \end{tikzcd}
      \end{equation}

      But we note that $d_1 f_1^{\**} = \phi_1^{\**} \beta$ with $\beta = \phi_0^{-1,\**} (d_0 f_1)^{\**}$, as indicated by the following diagram, using relation (iii) from $|\mathcal C_\bullet|$.
      \begin{equation}
            \begin{tikzcd}[column sep = tiny, row sep = small]
                  (d_1 f_1)^{\**} (\phi_0^{-1} \phi_1)^{\**} e \arrow[rr, equal] \arrow[dd, mapsto] \arrow[dr, dashed, "\exists !"]
                  &&
                  (\phi_0^{\-1} \phi_1)^{\**} (d_0 f_1)^{\**} e \arrow[rr] \arrow[dd, mapsto] \arrow[dr, dashed, "\exists !"]
                  &&
                  \phi_0^{-1,\**} (d_0 f_1)^{\**} e \arrow[rr] \arrow[dd, mapsto] \arrow[dr, dashed, "\exists !", "\beta"']
                  &&
                  (d_0 f_1)^{\**} e \arrow[dr] \arrow[dd, mapsto]
                  \\
                  &
                  (\phi_0^{-1} \phi_1)^{\**} \arrow[rr, equal, crossing over]
                  &&
                  (\phi_0^{-1} \phi_1)^{\**} e \arrow[rr, crossing over]
                  &&
                  \phi_0^{-1,\**} e \arrow[rr, crossing over]
                  &&
                  e \arrow[dd, mapsto]
                  \\
                  E_0 d_1 a_1 \arrow[rr, equal] \arrow[dr]
                  &&
                  E_0 d_1 a_1 \arrow[dr] \arrow[rr]
                  &&
                  E_1 a_1 \arrow[rr] \arrow[dr]
                  &&
                  E_0 d_0 a_1 \arrow[dr]
                  \\
                  &
                  E_0 d_1 b_1 \arrow[uu, mapsfrom, crossing over] \arrow[rr, equal]
                  &&
                  E_0 d_1 b_1 \arrow[uu, mapsfrom, crossing over] \arrow[rr]
                  &&
                  E_1 b_1 \arrow[uu, mapsfrom, crossing over] \arrow[rr]
                  &&
                  E_0 d_0 b_1 
            \end{tikzcd}
      \end{equation}
      Hence, by relation (iii) for $|\pi^{\**} \mathcal C_\bullet|$, the maps in \eqref{IMUG_EQ} agree.

      Relation (iv) holds by a similar argument. Thus $G$ is a well-defined inverse to the map $F$, showing $|\pi^{\**} \mathcal C_\bullet|$ and $\pi^{\**}|\mathcal C_\bullet|$ are isomorphic.      
      % Moreover, since $\pi$ is a fibration, any choice of factorization of the arrow $f$ into its generating pieces
      % must in turn produce a factorization of the map $\alpha$ (up to isomorphism on the source).
      % Thus, we have that arrows here are generated by
      % \begin{enumerate}[label = (\roman*)]
      % \item pairs $(f_0: a_0 \to \bar a_0, (E_0(f_0))^{\**}e \to e)$ with $p(e) = E_0(a_0)$.
      % \item pairs $(a_1:d_1(a_1) \to d_0(a_1), \phi_1^{\**}(\phi_0^{-1})^{\**} e \to (\phi_0^{-1})^{\**}e \to e)$ with $p(e) = E_0d_0(a_1)$.
      % \end{enumerate}

      % Thus, it suffices to show that
      % (i) all maps are (isomorphic to one) of the form $E_0(f_0)^{\**}e \to e$, and
      % (ii) all objects $e$ are (isomorphic to one) of the form $(\phi_0 \pi)^{\**} \bar e$.

      % (ii) follows since $\phi_0$ is invertible, so $e = (\phi_0\pi)^{\**} (\phi_0^{-1}\pi)^{\**} e$.
      % (i).... i don't know yet.
      % \todo[inline]{come back}.
\end{proof}





\subsection{Realization and $G\ltimes -$}

\begin{proposition}
      \label{GWRREAL_PROP}
For $\mathcal{C}_{\bullet} \in \mathsf{Cat}^{G\times \Delta^{op}}$
one has a natural identification
\[|G \ltimes \mathcal{C}_{\bullet}| \simeq G \ltimes |\mathcal{C}_{\bullet}|.\]
\end{proposition}

\begin{proof}
	This follows directly by comparing the generating arrows and relations, with the only interesting case being that of the relations
\[
\begin{tikzcd}
	d^1(c_1) \ar{r}{c_1} \ar{d}[swap]{g} &
	d^0(c_1) \ar{d}{g}
\\
	d^1(g c_1) \ar{r}{g c_1} &
	d^0(g c_1)	
\end{tikzcd}
\]
for $c_1 \in \mathcal{C}_1$ which in
$|G \ltimes \mathcal{C}_{\bullet}|$ follow from the existence of the arrow $c_1 \xrightarrow{g} g c_1$ in $G \ltimes \mathcal{C}_1$ 
and in $G \ltimes |\mathcal{C}_{\bullet}|$ follow from an action relation.
\end{proof}


