\documentclass[a4paper,10pt
%,draft
%, final
]{article}%

\pdfcompresslevel=0
\pdfobjcompresslevel=0

\usepackage[hidelinks]{hyperref}
\hypersetup{
  % colorlinks,
  final,
  pdftitle={Equivariant Dendroidal Segal Spaces},
  pdfauthor={Bonventre, P. and Pereira, L. A.},
  % pdfsubject={Your subject here},
  % pdfkeywords={keyword1, keyword2},
  linktoc=page
}
%\usepackage[open=false]{bookmark}

\input{commands.tex}%


%-------- Tikz ---------------------------

\usepackage{tikz}%
\usetikzlibrary{matrix,arrows,decorations.pathmorphing,
cd,patterns,calc}
\tikzset{%
  treenode/.style = {shape=rectangle, rounded corners,%
                     draw, align=center,%
                     top color=white, bottom color=blue!20},%
  root/.style     = {treenode, font=\Large, bottom color=red!30},%
  env/.style      = {treenode, font=\ttfamily\normalsize},%
  dummy/.style    = {circle,draw,inner sep=0pt,minimum size=2mm}%
}%

\usetikzlibrary[decorations.pathreplacing]



% ---- Commands on draft --------

\usepackage{ifdraft}
\ifdraft{
  \color[RGB]{63,63,63}
  % \pagecolor[rgb]{0.5,0.5,0.5}
  \pagecolor[RGB]{220,220,204}
  % \color[rgb]{1,1,1}
}


\usepackage[draft]{showkeys}
\usepackage{todonotes}%[obeyDraft]


% ----- Labels Changed? --------

\makeatletter

\def\@testdef #1#2#3{%
  \def\reserved@a{#3}\expandafter \ifx \csname #1@#2\endcsname
  \reserved@a  \else
  \typeout{^^Jlabel #2 changed:^^J%
    \meaning\reserved@a^^J%
    \expandafter\meaning\csname #1@#2\endcsname^^J}%
  \@tempswatrue \fi}

\makeatother


% ---- Commands --------

% new symbols

\newcommand{\mycircled}[2][none]{%
  \mathbin{
    \tikz[baseline=(a.base)]\node[draw,circle,inner sep=-1.5pt, outer sep=0pt,fill=#1](a){\ensuremath #2\strut};
cf.  }
}
\newcommand{\owr}{\mycircled{\wr}}

% replace symbols

\renewcommand{\hat}{\widehat}

% random

\renewcommand{\F}{\mathcal F}
\newcommand{\Q}{\mathcal Q}

\newcommand{\lltimes}{\underline{\ltimes}}


% detecting $\V$-categories:

\newcommand{\I}{\mathbb I}
\newcommand{\J}{\mathbb J}
\renewcommand{\1}{\eta}%{\ensuremath{\mathbb{id}}}

% lazy shortcuts

\newcommand{\SC}{\Sigma_{\mathfrak C}}
\newcommand{\OC}{\Omega_{\mathfrak C}}

\newcommand{\UV}{\underline{\mathcal V}}
\newcommand{\UC}{\underline{\mathfrak C}}


% ---- Title --------

\title{Equivariant Segal operads, simplicial operads, and dendroidal sets}

\author{Peter Bonventre, Lu\'is A. Pereira}%

\date{\today}


% ---- Document body --------

\begin{document}

\maketitle

\begin{abstract}
      Things and stuff
\end{abstract}

\tableofcontents



\section{Adjunctions between algebras over monads}




\begin{proposition}\label{GENMONS PROP}
If $T \Rightarrow F$ is a monad map,
the forgetful functor
$U \colon \mathsf{Alg}_F \to \mathsf{Alg}_T$
is the right adjoint in an adjunction such that
\begin{enumerate}[label=(\roman*)]
\item on a $T$-algebra $X$ with multiplication $TX \xrightarrow{m} X$,
the left adjoint is given by the coequalizer of $F$-algebras
\begin{equation}\label{LEFTADJFOR EQ}
coeq \left( F T X 
\overset{\mu}{\underset{Fm}{\rightrightarrows}}
F X\right)
\end{equation}
\item the adjunction unit
is the natural map (induced by $T \Rightarrow F$)
\[
T \simeq coeq \left( T T X 
\overset{\mu}{\underset{Tm}{\rightrightarrows}}
T X\right)
	\to
coeq \left( F T X 
\overset{\mu}{\underset{Fm}{\rightrightarrows}}
F X\right)
\]
\item 
the adjunction counit, on a $F$-algebra $Y$ with multiplication 
$FY \xrightarrow{m} Y$, is the natural map (induced by $m$)
\[
coeq \left( F T Y 
\overset{\mu}{\underset{Fm}{\rightrightarrows}}
F Y\right)
\to
Y
\]
\end{enumerate}
\end{proposition}

\begin{proof}
Since existence of the left adjoint is equivalent to the representability
of the functors
$\mathsf{Alg}_F \xrightarrow{\mathsf{Alg}_T(X,U(-))} \mathsf{Set}$
 for each $X \in \mathsf{Alg}_T$
and, for $A\in \mathcal{C}$ and $Y \in \mathsf{Alg}_F$,
one has natural isomorphisms
\begin{equation}\label{ALGADJ EQ}
\mathsf{Alg}_{T}(TA,Y) 
\simeq \mathcal{C}(A,Y)
 \simeq \mathsf{Alg}_{F}(FA,Y)
\end{equation}
one has that the left adjoint of $U$ can be defined
to send the free $T$-algebra
$T A$ to the free $F$-algebra $FA$
and likewise for free maps.
%
Moreover, the maps 
$\mathsf{Alg}_{T}(TA,Y)
\to
\mathsf{Alg}_{T}(TTA,Y)$
and
$\mathsf{Alg}_{F}(FA,Y)
\to
\mathsf{Alg}_{F}(FTA,Y)$
induced by the multiplications
$\mu \colon TT \Rightarrow T$,
$\mu \colon F T \Rightarrow F$
correspond via \eqref{ALGADJ EQ}
to the two horizontal composites in \eqref{MUCOMPADJ EQ} below.
Here the left maps are induced by functoriality of $T$ and $F$, the middle maps by $\bar{F} \Rightarrow F$ (so that the square commutes by naturality) and the right map by the algebra action $FY \to Y$.
\begin{equation}\label{MUCOMPADJ EQ}
\begin{tikzcd}[row sep=0]
	&
	\C(TX,TY) \ar{rd}
\\
	\C(X,Y) \ar{ru} \ar{rd}
&&
	\C(TX,FY) \ar{r}
&
	\C(TX,Y)
\\
&
	\C(FX,FY) \ar{ru}
\end{tikzcd}
\end{equation}
We have thus shown that the left adjoint sends the 
(non-free) map of free algebras
$TTX \xrightarrow{\mu}TX$
to 
$FTX \xrightarrow{\mu} FX$.


But now recalling that any $T$-algebra $X$ with multiplication
$TX \xrightarrow{m} X$
is canonically described as the coequalizer
$X = coeq \left( T T X \overset{\mu}{\underset{T m}{\rightrightarrows}} T X\right)$,
it follows that
\eqref{LEFTADJFOR EQ}
indeed represents $\mathsf{Alg}_T(X,U(-))$, which proves (i).

Lastly, for (ii) and (iii), we have that the adjunction naturally identifies the maps below
\[
X =
 coeq \left( T T X \overset{\mu}{\underset{T m}{\rightrightarrows}} T X\right) \to UY
\qquad
coeq \left( F T X \overset{\mu}{\underset{F m}{\rightrightarrows}} F X\right) \to Y
\]
so (ii) and (iii) follow, respectively by choosing 
the right map to be the identity on \eqref{LEFTADJFOR EQ} and
left map to be the identity on $UY$.
\end{proof}







\begin{definition}
Let $\iota \colon K \to L$ be a fully faithful inclusion.
An $\iota$-absolute colimit is a cone 
$I^+ \xrightarrow{c} L$ which:
\begin{itemize}
\item $c$ is a colimit cone in $L$;
\item the restriction
$c|_I$ is in $K$;
\item 
\[
L(\iota(-),c(+)) \simeq colim_{i\in I} L(\iota(-),c_i)
= colim_{i\in I} K(-,c_i)
\]
\end{itemize}
\end{definition}


\begin{remark}
If $l \in L$ is given by an $\iota$-absolute colimit 
$l \simeq colim_i{k_i}$ then 
\[\left(\mathsf{Lan}_{\iota}F\right)(l) = 
\int^{k\in K} F(k) \cdot L(\iota(k),l)=
\int^{k\in K} F(k) \cdot \left(colim_i K(k,k_i) \right)
=
\colim_i \left( \int^{k\in K} F(k) \cdot K(k,k_i) \right)
=
\colim_i F(k_i)
\]
In other words, $\iota$-absolute coequalizers have the property that they are preserved by the left Kan extension along $\iota$ of any functor from $\iota$.
\[
\begin{tikzcd}
	K \ar{r}{F} \ar{d}[swap]{\iota} &
	\mathcal{C}
\\
	L \ar{ru}[swap]{\mathsf{Lan}_{\iota} F}
\end{tikzcd}
\]
Note that this remark also allows for the case where $\iota$ is the identity, in which case we recover the notion of absolute coequalizer and the familiar fact that absolute coequalizers are preserved by any functor.
\end{remark}


\begin{remark}
For any $l\in L$ which can be described by an $\iota$-absolute colimit
it follows from the previous remark that $\left(\mathsf{Lan}_{\iota} \iota\right)(l)=l$
\[
\begin{tikzcd}
	K \ar{r}{\iota} \ar{d}[swap]{\iota} &
	L
\\
	L \ar{ru}[swap]{\mathsf{Lan}_{\iota} \iota}
\end{tikzcd}
\]
\end{remark}


\begin{example}
The previous remark can easily fail for colimits of diagrams in $K$ that are not absolute.
For instance, let $C_2 = \mathbb{Z}_{/2}$ be the field
with two elements and $K = \{C_2\}$ the subcategory containing only the $1$-dimensional vector space.

Then $\left(\mathsf{Lan}_{\iota}\iota \right)(C_2 \oplus C_2)
= \colim_{\left(C_2 \to C_2 \oplus C_2\right) \in K \downarrow C_2 \oplus C_2} C_2$
Analizing the category $K \downarrow C_2 \oplus C_2$
we get that this is the colimit
\[
\begin{tikzcd}
	& C_2 \ar{rd}{0} \ar{d}{0} \ar{ld}[swap]{0}&
\\
C_2 & C_2 & C_2
\end{tikzcd}
\]
which is $C_2\oplus C_2 \oplus \oplus C_2$.

More generally, 
$\left(\mathsf{Lan}_{\iota}\iota \right)(C_2^{\oplus n}) = C_2^{\oplus 2^n-1}$, the number of non-zero points of $C_2^{\oplus n}$
(or, with an eye to the case of a general field, the number of directions).
\end{example}




\begin{remark}
For each $X \in \mathsf{Alg}_{T}$
there is a canonical reflexive coequalizer
$coeq\left(TTX \rightrightarrows {TX}\right) \simeq X$.

Moreover, this coequalizer has the property of being
$\iota$-absolute,
meaning that it induces ``partial Yoneda'' isomorphisms
\[
coeq \left(
\mathsf{Kl}_T(-,TX) \rightrightarrows
\mathsf{Kl}_T(-,X)
\right)
=
coeq \left(
\mathsf{Alg}_T(\iota(-),TTX) \rightrightarrows
\mathsf{Alg}_T(\iota(-),TX)
\right)
\simeq
\mathsf{Alg}_T(\iota(-),X)
\]
Indeed, these are repackaging of the isomorphism
\[
coeq \left(
\mathcal{C}(-,TTX) \rightrightarrows
\mathcal{C}(-,TX)
\right)
\simeq
\mathcal{C}(-,X)
\]
which exists since the canonical coequalizer becomes split upon forgetting to $\mathcal{C}$.
\end{remark}




\begin{proposition}
Let $P,\bar{F}$ be monads on $\mathcal{C}$ as in Definition \ref{AMALGMON DEF}.

Then for each $\bar{F}$-algebra 
$X \in \mathsf{Alg}_{\bar{F}}(\mathcal{C})$
with multiplication 
$\bar{F}X \xrightarrow{m} X$
one has an $\bar{F}$-algebra structure on $PX$ given by
$\bar{F}P X \xrightarrow{\tau_X} P \bar{F} X \xrightarrow{Pm} X$.

Moreover, the underlying functor and monad strucutre of $P$
lift to a monad $\mathsf{Alg}_{\bar{F}}(\mathcal{C})$.
\end{proposition}


\begin{proof}
Write $m_P$ for the prescribed $\bar{F}$-algebra structure on $PX$.

To check associativity of $m_P$, consider the left diagram in 
\eqref{CHECKMON EQ} below, which commutes by the second half of Proposition \ref{ALTMULT EQ} and because $X$ is a $\bar{F}$-algebra.
Associativity is the claim that the composites 
$\bar{F} \bar{F} P X \xrightarrow{\bar{F} m_P} \bar{F} PX
\xrightarrow{m_P} PX$
and
$\bar{F} \bar{F} P X \xrightarrow{\mu_{PX}} \bar{F} PX
\xrightarrow{m_P} PX$
coincide.
But the second composite is the left bottom composite in \eqref{CHECKMON EQ}
while the first composite is 
\[
\bar{F} \bar{F} P X \xrightarrow{\bar{F} \tau_X}
\bar{F} P \bar{F} X \xrightarrow{\bar{F}P m}
\bar{F} P X \xrightarrow{\tau_X}
P \bar{F} X \xrightarrow{Pm}
PX
\]
which matches the top right composite in \eqref{CHECKMON EQ},
so associativity follows. Unitality of $m_P$ follows similarly using the second half of Lemma \ref{ALTMULT PROP}(iii).

For the moreover claim, that $P$ sends algebra maps to algebra maps follows from naturality of $\tau$, so it remains only to show that,
when applied to $\bar{F}$-algebras, the natural transformations 
$\mu \colon PP \Rightarrow P$ and $\eta \colon I \Rightarrow P$
consist of $\bar{F}$-algebra maps.
For $\mu \colon PP \Rightarrow P$ this follows from the right diagram in 
\eqref{CHECKMON EQ}, which commutes by the first half of Proposition \ref{ALTMULT PROP}(ii), while the $\eta \colon I \Rightarrow P$
case follows similarly using the first half of 
Proposition \ref{ALTMULT PROP}(iii).
\begin{equation}\label{CHECKMON EQ}
\begin{tikzcd}[column sep = 1.55em]
	\bar{F} \bar{F} P X \ar{r}{\bar{F}\tau_X} \ar{d}[swap]{\mu_{PX}}
&
	\bar{F} P \bar{F} X \ar{r}{\tau_{\bar{F}X}}
&
	P \bar{F} \bar{F} X \ar{r}{P\bar{F}m} \ar{d}[swap]{P\mu_X}
&
	P \bar{F} X \ar{d}{Pm}
&%
	\bar{F} P P X \ar{r}{\tau_{PX}} \ar{d}[swap]{\mu_{PX}}
&
	P \bar{F} P X \ar{r}{P \tau_{X}}
&
	P P \bar{F} X \ar{r}{P Pm} \ar{d}[swap]{\mu_{\bar{F}X}}
&
	PP X \ar{d}{\mu_X}
\\
	\bar{F} P X \ar{rr}{\tau_X}
&&
	P \bar{F} X \ar{r}{Pm}
&
	PX
&%
	\bar{F} P X \ar{rr}{\tau_X}
&&
	P \bar{F} X \ar{r}{Pm}
&
	PX
\end{tikzcd}
\end{equation}
\end{proof}





\begin{proposition}
There is an adjunction
$\mathsf{Alg}_{\bar{F}} \rightleftarrows \mathsf{Alg}_{F}$
where
\begin{enumerate}[label=(\roman*)]
\item the left adjoint is given by $P$ on both objects and arrows, 
i.e. for any map $f \colon X \to Y$ of $\bar{F}$-algebras, the left adjoint gives
$Pf \colon Px \to PY$;
\item if the $\bar{F}$-algebra $X$ has multiplication
$\bar{F}X \xrightarrow{m} X$
then the $F$-algebra $PX$ has multiplication
$FPX = P\bar{F}P X \xrightarrow{P\tau_X}
PP\bar{F}X \xrightarrow{\mu_{\bar{F}X}} P\bar{F} X
\xrightarrow{Pm} X$;
\item the adjunction unit is given by $\eta \colon I \Rightarrow P$;
\item the adjunction counit, on a $F$-algebra $Z$ with multiplication
$FZ \xrightarrow{m} Z$,
is given by
$PZ \rightarrow FZ \xrightarrow{m}Z$.
\end{enumerate}
\end{proposition}


\begin{proof}
We have already shown elsewhere that,
on a $\bar{F}$-algebra with multiplication 
$\bar{F}X \xrightarrow{m} X$,
the left adjoint must be given by
the coequalizer
\begin{equation}\label{FBARFCOEQ EQ}
coeq 
 \left(
 \begin{tikzcd}[column sep=8pt]
	F \bar{F} X
	\ar[shift left=.3em]{r}{\mu}
	\ar[shift right=.3em]{r}[swap]{Fm} &
	FX
 \end{tikzcd}
 \right)
=
coeq 
 \left(
 \begin{tikzcd}[column sep=14pt]
	P \bar{F} \bar{F} X
	\ar[shift left=.3em]{r}{P\mu}
	\ar[shift right=.3em]{r}[swap]{P \bar{F}m} &
	P \bar{F} X
 \end{tikzcd}
 \right)
\end{equation}
if it exists.
But the coequalizer 
$coeq \left( \bar{F} \bar{F} X \overset{\mu}{\underset{\bar{F}m}{\rightrightarrows}} \bar{F} X\right)$
is split after forgetting to $\mathcal{C}$,
and hence so is the coequalizer above, which is just obtained by applying $P$ throughout.

But since forgetting to $\mathcal{C}$
creates coequalizers that become split upon forgetting,
it follows that \eqref{FBARFCOEQ EQ} must indeed be $PX$
with some $F$-algebra structure, so (i) follows.

For (ii), writing $m_F$ for the yet unknown $F$-algebra structure on $PX$, note that the diagram below must commute
(due to the creation claim concerning the coequalizer)
\begin{equation}
\begin{tikzcd}[column sep = 1.55em]
	FPX 	\ar{r} 
&
	FFX 	\ar{r}{FPm} \ar{d}[swap]{\mu}
&
	FPX   \ar{d}{m_F}
\\
&
	P \bar{F} X \ar{r}[swap]{P\mu}
&
	P \bar{F} X
\end{tikzcd}
\end{equation}
and (ii) follows.

(iii) is immediate from the corresponding part of Proposition \ref{GENMONS PROP}.

For (iv), this follows since the bottom right triangle in the diagram below must commute (here $m_{\bar{F}}$ is the $\bar{F}$-algebra structure obtained by restricting the $F$-algebra structure).
\begin{equation}
\begin{tikzcd}[column sep = 3em]
	PY \ar{r}{P\eta} \ar{rd}[swap]{P\eta}
&
	PFX 	\ar{r}{Pm} 
&
	P Y   \ar{d}
\\
&
	P \bar{F} Y \ar{r}[swap]{m} \ar{u}
	\ar{ru}[swap]{P m_{\bar{F}}}
&
	Y
\end{tikzcd}
\end{equation}
\end{proof}








\bibliography{biblio-new}{}
\bibliographystyle{amsalpha2}



\end{document}


%%% Local Variables:
%%% mode: latex
%%% TeX-master: t
%%% End:
