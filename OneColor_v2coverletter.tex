\documentclass[a4paper,10pt
%,draft
]{article}%



% ---- Commands on draft --------

\usepackage[dvipsnames]{xcolor}% adds colors
\usepackage{ifdraft}
\ifdraft{
	\color[RGB]{63,63,63}
	\pagecolor[RGB]{220,220,204}
	\usepackage[notref]{showkeys}
	\usepackage{todonotes}
}
%{
%	\usepackage[disable]{todonotes}
%}



\pdfcompresslevel=0
\pdfobjcompresslevel=0

\usepackage{xr-hyper}
\usepackage[pagebackref, colorlinks, citecolor=PineGreen, linkcolor=PineGreen]{hyperref}
\hypersetup{
	final,
	pdftitle={Edits to ``Homotopy theory of equivariant operads with fixed colors''},
	pdfauthor={Bonventre, P. and Pereira, L. A.},
	linktoc=page
}

\externaldocument[TAS-]{TameAndSquare_v2} % cite using names from other half
\externaldocument[OC-]{OneColor} % cite using names from other half
\externaldocument{AllColors-v2}


\usepackage{amsmath, amsthm}% {amsfonts, amssymb}


% ------ New Characters --------------------------------------

\usepackage[latin1]{inputenc}%
\usepackage{MnSymbol}


\usepackage[normalem]{ulem}% underlining
%\usepackage{dsfont}% double strike-through
\usepackage{bbm}% more bb


\DeclareMathAlphabet\mathbb{U}{msb}{m}{n}
\usepackage{upgreek}
\usepackage{mathrsfs}

\usepackage[normalem]{ulem}


%----- Enumerate ---------------------------------------------
\usepackage[inline,shortlabels]{enumitem}% % can use \begin{enumerate*} for inparaenum
\setenumerate{label=(\roman*)}



% ---------- Page Typesetting ----------
\usepackage[final]{microtype}
\usepackage{relsize}
\usepackage{geometry}


%-------- Tikz ---------------------------

\usepackage{tikz}%
\usetikzlibrary{matrix,arrows,decorations.pathmorphing,
cd,patterns,calc}
\tikzset{%
  treenode/.style = {shape=rectangle, rounded corners,%
                     draw, align=center,%
                     top color=white, bottom color=blue!20},%
  root/.style     = {treenode, font=\Large, bottom color=red!30},%
  env/.style      = {treenode, font=\ttfamily\normalsize},%
  dummy/.style    = {circle,draw,inner sep=0pt,minimum size=2mm}%
}%

\usetikzlibrary[decorations.pathreplacing]


% ----- Labels Changed? --------

\makeatletter

\def\@testdef #1#2#3{%
  \def\reserved@a{#3}\expandafter \ifx \csname #1@#2\endcsname
  \reserved@a  \else
  \typeout{^^Jlabel #2 changed:^^J%
    \meaning\reserved@a^^J%
    \expandafter\meaning\csname #1@#2\endcsname^^J}%
  \@tempswatrue \fi}

\makeatother


%%%%%%%%%%%%%%%%%%%%%%%%% INTERNAL REFERENCES %%%%%%%%%%%%%%%%%%%%%%%%%%%%%%%%%%%

\numberwithin{equation}{section} 
\numberwithin{figure}{section}

\usepackage{mathtools}
\mathtoolsset{showonlyrefs,showmanualtags} % Only number equations which are referenced with eqref


% ------- New Theorems/ Definition/ Names-----------------------

 % \theoremstyle{plain} % bold name, italic text
\newtheorem{theorem}[equation]{Theorem}%
\newtheorem*{theorem*}{Theorem}%
\newtheorem{lemma}[equation]{Lemma}%
\newtheorem{proposition}[equation]{Proposition}%
\newtheorem{corollary}[equation]{Corollary}%
\newtheorem{conjecture}[equation]{Conjecture}%
\newtheorem*{conjecture*}{Conjecture}%
\newtheorem{claim}[equation]{Claim}%

%%%%%% Fancy Numbering for Theorems
\newtheorem{innercustomgeneric}{\customgenericname}
\providecommand{\customgenericname}{}
\newcommand{\newcustomtheorem}[2]{%
  \newenvironment{#1}[1]
  {%
   \renewcommand\customgenericname{#2}%
   \renewcommand\theinnercustomgeneric{##1}%
   \innercustomgeneric
  }
  {\endinnercustomgeneric}
}

\newcustomtheorem{customthm}{Theorem}
\newcustomtheorem{customcor}{Corollary}
%%%%%%%%%%%%%

\theoremstyle{definition} % bold name, plain text
\newtheorem{definition}[equation]{Definition}%
\newtheorem*{definition*}{Definition}%
\newtheorem{example}[equation]{Example}%
\newtheorem{remark}[equation]{Remark}%
\newtheorem{notation}[equation]{Notation}%
\newtheorem{convention}[equation]{Convention}%
\newtheorem{assumption}[equation]{Assumption}%
\newtheorem{exercise}{Exercise}%


% %%%%%%%%%%%%%%%%%%%%%%%%%%%%%%%%%%%%%%%%%%%%%%%%%%%%%%%%%%%%%%%%%%%%%%%%%%%%%%%%
% ------------------------------ COMMANDS ------------------------------

% ---------- macros

\newcommand{\set}[1]{\left\{#1\right\}}%
\newcommand{\sets}[2]{\left\{ #1 \;|\; #2\right\}}%
\newcommand{\longto}{\longrightarrow}%
\newcommand{\into}{\hookrightarrow}%
\newcommand{\onto}{\twoheadrightarrow}%

\usepackage{harpoon}
\newcommand{\vect}[1]{\text{\overrightharp{\ensuremath{#1}}}}


% ---------- operators

\newcommand{\Sym}{\ensuremath{\mathsf{Sym}}}%
\newcommand{\Fin}{\mathsf{F}}%
\newcommand{\Set}{\ensuremath{\mathsf{Set}}}
\newcommand{\Top}{\ensuremath{\mathsf{Top}}}
\newcommand{\sSet}{\ensuremath{\mathsf{sSet}}}%
\newcommand{\Cat}{\mathsf{Cat}}
\newcommand{\sCat}{\mathsf{sCat}}
\newcommand{\Op}{\mathsf{Op}}%
\newcommand{\sOp}{\ensuremath{\mathsf{sOp}}}%
\newcommand{\fgt}{\ensuremath{\mathsf{fgt}}}%
\newcommand{\dSet}{\mathsf{dSet}}
\newcommand{\Fun}{\mathsf{Fun}}
\newcommand{\Fib}{\mathsf{Fib}}
\newcommand{\Alg}{\mathsf{Alg}}
\newcommand{\Kl}{\mathsf{Kl}}



\DeclareMathOperator{\hocmp}{hocmp}%
\DeclareMathOperator{\cmp}{cmp}%
\DeclareMathOperator{\hofiber}{hofiber}%
\DeclareMathOperator{\fiber}{fiber}%
\DeclareMathOperator{\hocofiber}{hocof}%
\DeclareMathOperator{\hocof}{hocof}%
\DeclareMathOperator{\holim}{holim}%
\DeclareMathOperator{\hocolim}{hocolim}%
\DeclareMathOperator{\colim}{colim}%
\DeclareMathOperator{\Lan}{Lan}%
\DeclareMathOperator{\Ran}{Ran}%
\DeclareMathOperator{\Map}{Map}%
\DeclareMathOperator{\Id}{Id}%
\DeclareMathOperator{\mlf}{mlf}%
\DeclareMathOperator{\Hom}{Hom}%
\DeclareMathOperator{\Ho}{Ho}
\DeclareMathOperator{\Aut}{Aut}%
\DeclareMathOperator{\Stab}{Stab}
\DeclareMathOperator{\Iso}{Iso}
\DeclareMathOperator{\Ob}{Ob}

% ---------- shortcuts

\newcommand{\F}{\ensuremath{\mathcal F}}
\newcommand{\V}{\ensuremath{\mathcal V}}
\newcommand{\Q}{\ensuremath{\mathcal Q}}
\renewcommand{\O}{\ensuremath{\mathcal O}}
\renewcommand{\P}{\ensuremath{\mathcal P}}
\newcommand{\C}{\ensuremath{\mathcal C}}
\newcommand{\A}{\ensuremath{\mathcal A}}
\newcommand{\G}{\ensuremath{\mathcal G}}

\newcommand{\del}{\partial}%

\newcommand{\ki}{\chi}
\newcommand{\ksi}{\xi}
\newcommand{\Ksi}{\Xi}

\newcommand{\lltimes}{\underline{\ltimes}}

% detecting $\V$-categories:

\newcommand{\I}{\mathbb I}
\newcommand{\J}{\mathbb J}
\newcommand{\1}{\ensuremath{\mathbbm 1}}%{\ensuremath{\mathbb{id}}} %\eta

% lazy shortcuts

\newcommand{\SC}{\Sigma_{\mathfrak C}}
\newcommand{\OC}{\Omega_{\mathfrak C}}
\newcommand{\UV}{\underline{\mathcal V}}
\newcommand{\UC}{\underline{\mathfrak C}}










% %%%%%%%%%%%%%%%%%%%%%%%%%%%%%%%%%%%%%%%%%%%%%%%%%%%%%%%%%%%%%%%%%%%%%%%%%%%%%%%%%%%%%%%%%%%%%%%%%%%%
% ------------------------------ MAIN BODY ------------------------------

% ---- Title --------

\title{Edits to ``Homotopy theory of equivariant operads with fixed colors''}

\author{Peter Bonventre, Lu\'is A. Pereira}%

% \date{\today}


\begin{document} 
  
\maketitle
 



\section{General comments}

The referee report included 9 particular comments,
all of which consisted of
typos,
suggestions for minor changes to wording,
or requests for a little clarification or extra grammatical clarity.
All of these have been addressed in the following sections.
The numbering of the items below refer to the order of the bullets in the referee report,
while the numbering/citations for referenced results/articles refer to the numbering/naming in the new version of the article.

\section*{Non-changes}
      
\begin{enumerate}
\item[(1)] We believe that using $\pi$ as the name for a generic fibration here does not add confusion, but instead illuminates the symmetry of the definition of a fibered adjunction, such as in the second display equation in Definition 2.22.
        Additionally, in key examples (e.g. Proposition 2.30, Remark A.21)
        the maps $\pi$ are in fact the same (up to a forgetful functor).
\end{enumerate}

\section*{Changes}
\begin{enumerate}
\item[(2)] Replaced ``abstract nonsense'' with ``categorical argument''. Our choice is meant to refer to the style of argument in the proofs, not just the statements of the results.
\item[(3)] Replaced ``less'' with ``fewer'' in Remark 3.49.
\item[(4),(6)] Replaced ``iff'' with ``if and only if'' in Proposition 4.17 and Definition 4.23.
\item[(5)] The notation $\mathcal V^{\mathcal G}_{\mathcal F}$ is defined in Definition 4.13.
        We've replaced ``below'' with ``in (4.22)''.
\item[(7)] Clarified the first sentence outlining the proof of Proposition 4.34.
\item[(8)] Editted the sentence starting with ``i.e.'' in the proof of Proposition 4.34.
\item[(9)] As the statement of Theorem I is quite long (as is this paper) and is promenantly displayed in the section titled ``Main Results'', we do not feel it would be beneficial to write out the whole statement again outside the introduction.
        However, for additional clarity, we have added discussion to the proof of Theorem I indicating the applications of each of the assumptions in Theorem I, especially at the beginning of the proof, to provide context and references.
\end{enumerate}


\section*{Other changes}
\begin{itemize}
\item Removed extraneous ``are'' from the last sentence of the abstract.
\item Hyphenated ``single-colored'' throughout the paper.
\item Removed ``objects'' from ``colors/objects'' in the paragraph above (1.3).
\item Replaced ``$\mathfrak C$-signatures'' with ``$\mathfrak C$-profiles'' to provide uniform nomenclature across the different papers in this project.
\item Updated the wording in the paragraph above (1.5).
\item Removed extraneous ``note that'' throughout the paper.
\item Corrected typo ``fiver'' to ``fiber'' in Remark 2.11.
\item Added reference for further discussion at the end of Remark 2.18.
\item Updated wording of the analysis of the composite map above the diagram in the proof of Proposition 2.24.
\item Replaced ``as'' with ``to be'' in Notation 3.13.
\item Reformatted inline enumerate items in the proof of Proposition 3.35.
\item Added a reference for the definition of the pushout product to the discussion after Proposition 4.25.
\item Updated wording in Remark 4.27.
\item Specified that $G \times \Sigma^{op}$ is a groupoid in Definition 5.1.
\item Added an explicit description of the weak equivalences and fibrations to Definition 5.4.
\item Added necessary hypotheses to Corollaries 5.6 and 5.15, and added the base category $\V$ to the notation in all the model structured discussed therein.
\item Updated bibliography and references, mainly to [BP21].
\end{itemize}


\end{document} 




%%% Local Variables:
%%% mode: latex
%%% TeX-master: t
%%% End:
