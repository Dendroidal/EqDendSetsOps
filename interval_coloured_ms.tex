\documentclass[psamsfonts,onesided,10pt,letterpaper]{amsart}%

%\usepackage{hyperref}

\input{commands.tex}%

%\usepackage{fullpage}



\usepackage{showkeys}%
\usepackage{todonotes}

\author{Peter Bonventre, Lu\'is Alexandre Pereira}%
\title{Coloured Model Structure Using Interval Objects}%
\date{\today}



%-------- TIKZ -----------------------------------------
\usepackage{tikz}%
\usetikzlibrary{matrix,arrows,decorations.pathmorphing,
cd,patterns,calc}
\tikzset{%
  treenode/.style = {shape=rectangle, rounded corners,%
                     draw, align=center,%
                     top color=white, bottom color=blue!20},%
  root/.style     = {treenode, font=\Large, bottom color=red!30},%
  env/.style      = {treenode, font=\ttfamily\normalsize},%
  dummy/.style    = {circle,draw,inner sep=0pt,minimum size=2mm}%
}%

\usetikzlibrary[decorations.pathreplacing]

% -------------------- new commands --------------------


\renewcommand{\C}{\ensuremath{\mathfrak{C}}}
\renewcommand{\fc}{\ensuremath{\mathfrak{c}}}
\renewcommand{\F}{\mathcal F}
\newcommand{\FF}{\ensuremath{\mathbb{F}}}
\renewcommand{\H}{\ensuremath{\mathbb{H}}}
\newcommand{\I}{\ensuremath{\mathbb{I}}}
\newcommand{\J}{\ensuremath{\mathbb{J}}}
\renewcommand{\1}{\ensuremath{\mathbb{id}}}
\renewcommand{\P}{\ensuremath{\mathcal{P}}}
\newcommand{\Q}{\ensuremath{\mathcal{Q}}}
% \newcommand{\V}{\mathcal V}
\newcommand{\Vsigma}{\ensuremath{\V^{\Stab(\sigma)}_{\F_\sigma}}}
\newcommand{\N}{\mathbb N}


\newcommand{\floor}[1]{\left \lfloor #1 \right \rfloor}
%\newcommand{\dasharrowdbl}{\dashrightarrow\mathrel{\mkern-14mu}\dashrightarrow}
%\newcommand{\xdasharrowdbl}[2][]{%
%  \xdashrightarrow[#1]{#2}\m}
%\newcommand{\tall}[2]{\ensuremath{\begin{tikzcd} #1 \arrow[r,dashed,twoheadrightarrow] & #2 \end{tikzcd}}}
\newcommand{\tall}{\ensuremath{\Rightarrow}}%\rightsquigarrow
\newcommand{\Tall}{\ensuremath{\Downarrow}}%\rightsquigarrow

\newcommand{\T}{\ensuremath{\mathbb{T}}}

\newcommand{\iso}{\ensuremath{\mathsf{Iso}}}
\newcommand{\alt}{\ensuremath{\mathsf{alt}}}


% \newcommand{\Cat}{\mathsf{Cat}}
% \newcommand{\Op}{\mathsf{Op}}
%\renewcommand{\Sym}{\mathsf{Sym}}
% \newcommand{\Stab}{\mathrm{Stab}}

\newcommand{\Omeganc}{\ensuremath{\Omega_{(n-1,1)-\mathfrak c}}}
\newcommand{\Omeganbin}{\ensuremath{\Omega_{(n-1,1)-bin}}}
\newcommand{\Omeganalt}{\ensuremath{\Omega_{(n-1,1)-alt}}}


\setenumerate[0]{label=(\arabic*)}
\newcommand{\primed}[1]{$#1 '$}

\begin{document}		\maketitle%

%\tableofcontents

\section{Introduction}%
%$\dashrightarrow$
%$\dasharrowdbl$
%$\twoheadrightarrow$
%$\tall{A,B}$
%\begin{tikzcd} A \arrow[r,dashed,twoheadrightarrow] & B \end{tikzcd}

\[
	\begin{tikzpicture}[grow=up, every node/.style = {font=\footnotesize},level distance = 2em]%
	\tikzstyle{level 2}=[sibling distance=3em]%
	\tikzstyle{level 3}=[sibling distance=1em]%
		\node {}%
			child{node [dummy,label=-10:$\varphi$] {}%
				child{node [dummy,label=right:$\psi_3$] {}}%
				child{node [dummy,label=left:$\psi_2$] {}%
					child{}%
					child{}%
					child{}%
				}%
				child{node [dummy,label=left:$\psi_1$] {}%
					child{node {} }%
					child{node {} }%
				}%
			};%
	\end{tikzpicture}%
\]%


We recall a particular description of operads and their equivariant generalizations, and define a new category of ``genuine equivariant operads'' which forget to highly structured coefficient systems of (non-equivariant) operads. We define model structures on both of these categories, and show that there are natural functors realizing a Quillen equivalence.


% \begin{definition}
%       For a category $\V$, a {\em Hopf interval object} in is a cofibrant Hopf object $\H\in\V$ for which there exists a factorization
%       $\begin{tikzcd} 
%             I\coprod I \arrow[r,rightarrowtail] & \H \arrow[r, twoheadrightarrow, "\simeq"] & I
%       \end{tikzcd}$ 
%       of the fold map $I$, the unit in $\V$. We can $\H$ {\em cocommutative} if its Hopf structure is. 
% \end{definition}

\begin{definition}
      % Similarly but different:
      let $\I$ be the $\V$-category with objects $\set{0,1}$ with $\I(0,0) = \I(0,1) = \I(1,0) = \I(1,1) = 1_V$. A {\em $\V$-interval} is a cofibrant object in $\V\Cat_{\set{0,1}}$ (with the transfered model structure) weakly equivalent to $\I$. A set $\mathcal{G}$ of $\V$-intervals is {\em generating} if all $\V$-intervals $\J$ can be obtained as a retract of a trivial extension of an element in $\mathcal{G}$ in $\V\Cat_{\set{0,1}}$:
\[
\begin{tikzcd}
  \mathbb{G} \arrow[r,rightarrowtail, "\simeq"] & \mathbb{K} \arrow[r,yshift=-.3em, "r"'] & \mathbb{J} \arrow[l,yshift=.3em, "i"']
\end{tikzcd}
\]
\end{definition}

\subsection{Assumptions}

We assume the following conditions on the enriching category $\V$:

\begin{enumerate}
\item $\V$ strongly cofibrantly generated monoidal model category;
\item $G\V$ and $\mathsf{Cat}^G(\V)$ have cellular fixed point functors for all finite groups $G$;
% \item $\V$ satisfies transfer for symmetric operads (in particular if $\V$ has a cofibrant unit, has a cocommutative Hopf interval object (in particular, cartesian closed \cite{BM03}), and admits a monoidal fibrant replacement functor \cite{BM07});
\item $\V$ admits a symmetric monoidal fibrant replacement functor which commutes with $H$-fixed points: $R(X)^H = R(X^H)$. 
\end{enumerate}

Additionally, we may assume
\begin{enumerate}
%\item $\V^G_\F$ has a symmetric monoidal fibrant replacement functor for any finite group $G$ and family of subgroups $\F$;
\item $\V Cat$ has a generating set of intervals;
\item $\V$ is right proper
\item $W$ (to be defined later) is closed under transfinite compositions.
\end{enumerate}

\begin{enumerate}
\item We fix a (random) total ordering on the elements of $G$ and of every $\Sigma_n$.
\end{enumerate}

Most of the results will not need all of these assumptions, but this is the most we could possibly require (I think). I will label each result by what is needed (if possible).

Let $I$ and $J$ be the sets of generating cofibrations and generating trivial cofibrations of $\V$.

 \todo[inline]{STUFF}

\newpage
\section{Equivariant operads}

STuff

\subsection{Operads of equivariant operads}

Given a $G$-set $\C$, we define a colored operad $\Op_\C$ whose algebras are $G$-operads with colors $\C$.

The colors of $\Op_\C$ are pairs $(C,\fc)$ with $C$ a corolla, and $\fc: E(T) \to \C$ a set map.

Morphisms are generated by two types of maps:
\begin{enumerate}
\item $(T,\fc,\sigma,\set{\phi_i},\tau)\in \Op_\C((C_1,\fc_1), \ldots, (C_n,\fc_n); (C_0,\fc_0))$, where
  \begin{itemize}
  \item $T$ is a tree with $n$ vertices;
  \item $\fc: E(T)\to \C$ is a set math with $\fc(L(T)) = \fc_0(L(C_0))$ and $\fc(r) = \fc_0(r)$;
  \item $\sigma: \set{1,\ldots, n}\to V(T)$ a bijection such that, if $C[\sigma(i)]$ is the corolla in $T$ connectd to the vertex $\sigma(i)$, we have isomorphisms of colored corollas $\phi_ii: (C[\sigma(i)], \fc|_{C[\sigma(i)]})\xrightarrow{\cong} (C_i, \fc_i)$ for all $i\in \set{1,\ldots, n}$;
  \item $\tau: L(C_0)\to L(T)$ a bijection such that $\fc\tau = \fc_0$;
  \end{itemize}
\item $g\in \Op_\C((C,\fc); (C,g\fc))$
\end{enumerate}
modulo the relation $(T,\fc,\sigma,\set{\phi_i},\tau) = (T,\fc\pi^{-1},\pi\sigma,\set{\pi\phi_i},\pi\tau)$ for all $\pi\in \mathrm{Aut}(T)$.

Composition is defined as follows:
\begin{enumerate}
\item $[T,\fc,\sigma,\set{\phi_i},\tau] \circ_j [S,\mathfrak{d}, s, \set{\psi}, t] = [T\circ_{\sigma(j)}S, \fc', \sigma', \set{\phi_i}\amalg\set{\psi_{\sigma' i}}, \tau']$ where, if $T$ has $n$ vertices and $S$ and $m$ vertices, we have defined
\[
\fc'(e) = \begin{cases}\fc(e) & e\in T\\ \mathfrak{d}(e) & e\in S\end{cases} \qquad \mbox{and} \qquad 
\sigma'(i) =
\begin{cases}
  \sigma(i) & i< j\\
  s(i-j)+j & j\leq i < j+m\\
  \sigma(i-m)+m & i\geq j+m
\end{cases}
\]
and $\tau'$ synthesizes $\tau$ and $t$ (if neceesary) to retain the identification information on the leaves;
\item $g\circ h = gh$;
\item composition between $g$ and $[T]$ is free modulo
\[
[T,\fc,\sigma,\set{\phi_i},\tau] \circ (g,\ldots,g) = g\circ [T,g\fc,\sigma,\set{\phi_i}, \tau].
\]
\end{enumerate}

\newpage
\section{Model stucture on the category of equivariant operads}

\begin{proposition}[\cite{Ste16}, 2.6]
  Assume $\V$ is cofibrantly generated model category such that $\V^G$ has cellular fixed points for all finite groups $G$, and let $\F$ be any family of subgroups of $G$. Then $\V^G$ has the $\F$-model structure; that is, $\V^G$ has a cofibrantly generated model structure where weak equivalences and fibrations are defined by $(-)^H$ for all $H\in \F$. In particular, the generating cofibrations $I_\F = \sets{G/H\otimes i}{i\in I,\ H\in \F}$ and the generating trivial cofibrations $J_\F = \sets{G/H\otimes j}{j\in J,\ H\in \F}$. 
\end{proposition}

Now, let $G$ be a finite group, and suppose we have an {\em indexing family} $\F = \set{\F_n}$ of families of subgroups of $\set{G\times \Sigma_n}$ (c.f. \cite{BH15}). 

Let $\C$ be a $G$-set, of {\em colors}, and $Seq(\C)$ be the set of {\em signatures} in $\C$, defined by
$\sets{
  (a_1,\ldots, a_n; a)\in \C^n\times\C
}
{n\in\mathbb{N}}$.
For each $n$, the set of signatures of length $n+1$ have an action by $G\times \Sigma_n$, where $G$ acts on all components, and $\Sigma_n$ acts on all but the last one. 


\begin{definition}
  Let $\Op^{G,\mathfrak C}(\V)$ be the category of $\C$-colored $\V$-operads.
\end{definition}

For each $G$-set $\C$, we have a monoidal free-forgetful adjunction as below;
\begin{equation}
      \begin{tikzcd}
            fgt: \Op^{G,\mathfrak C}(\V) \arrow[r, shift right]
            &
            \V^{\Sym^{G,\mathfrak C}(\V)} \simeq
            \displaystyle\prod_{n \in \N} \V^{B_{\mathfrak C^{\times n+1}}(G \times \Sigma_n)}
            \arrow[l, shift right]
            \arrow[r, shift right]
            &
            \displaystyle\prod_{n\in\N}\displaystyle\prod_{\ksi\in \C^{\times n+1}}\V^{\Stab(\ksi)}
            \arrow[l, shift right]
            : \FF_\C
      \end{tikzcd}
\end{equation}

Given a signature $\ksi$ of length $n+1$, define $\F_\ksi$ to be the family of subgroups $\Lambda\in \F_n$ such that $\Lambda\leq \Stab(\ksi)$.
\begin{corollary}
  For any such $\ksi\in Seq(\C)$ and family $\F_n$, $\V^{\Stab(\ksi)}$ has the $\F_\ksi$-model structure.
\end{corollary}

\begin{definition}
      Suppose $\V$ has
      and fix a $G$-set $\mathfrak C$.
      Given $\F = \set{\F_n}$ a collection of families $\F_n$ of graph subgroups of $G \times \Sigma_n$,
      and a $\mathfrak C$-signature $\ksi \in \mathfrak C^{\times n+1}$,
      let $\F_\ksi$ denote those subgroups in $\F_n$ which live in $\Stab(\ksi)$.

      Then the $\F$-model structure on the category $\Op^{G,\mathfrak C}_\F(\V)$, denoted $\Op^{G,\mathfrak C}_\F(\V)$,
      is the model structure, if it exists, transferred from the free-forgetful adjunction above,
      where $\V^{\Stab(\ksi)}$ is endowed with the $\F_\ksi$-model structure.
      
      such that. Th
\end{definition}
\begin{proposition}
      If $\V$ has {\color{red} GOOD PROPERTIES}
      (at least cellular, and maybe
      $\V^{\Stab(\ksi)}_{\F_\ksi}$ has a symmetric monoidal fibrant replacement functor which commutes with fixed points),
      then this model structure exists.
\end{proposition}
\begin{proof}
      Other paper.
\end{proof}
% \begin{proof}
%       We use the same arguments as in \cite{BM03}, Theorem 3.2. In particular, we know that $fgt$ preserves filtered colimits. Moreover, given $\O\in \Op^{G,\mathfrak C}(\V)$, define $\tilde \O$ by $\tilde \O(\ksi) = R\O(\ksi)$, where $R$ is the fibrant replacement functor in $\V$; since $R$ commutes with fixed points, this is in fact a levelwise fibrant replacement ($(\O_f)^H = (\O^H)_f$). Further, since $R$ is symmetric monoidal, the operad structure on $\O$ induces one of $\tilde \O$.
%       {\color{red}
%         Lastly, we need a functorical path object in $\Op^{G,\mathfrak C}(\V)$. Since $\V$ cartesian monoidal, given any cocommutative Hopf interval object $\H$ , and fibrant $\O$, we have 
%         \[
%               \begin{tikzcd}
%                     \O \cong \O^I \arrow[r,"\simeq"] & \O^\H \arrow[r,twoheadrightarrow] & \O^{I\coprod I} \cong \O\times \O
%               \end{tikzcd}
%         \]
%         for which the first arrow (resp. second arrow) is a cofibration (resp. fibration) by the pushout-product axiom of monoidal model categories, since $I$ is cofibrant. Lastly, these are (maps of) operads since $\V$ cartesian closed.
%       } % color red
%       \todo[inline]{the red above is quite wrong.}
% \end{proof}

\begin{definition}
      Let $\mathsf{Op}^G(\V)$ denote the category of $G$-objects in $\mathsf{Op}(\V)$.
      Equivalently, this is the category of $\V$-operads with any $G$-set of colors,
      where a map $F: \O\to \P$ is defined by a pair $(F_0,F)$ with
      $F_0: \C(\O)\to \C(\P)$ a map of colours and
      $F: \O\to F^*\P$ a map of $\C(\O)$-coloured operads, where $F^*P(\ksi) = P(F_0(\ksi))$.
      Also, {\color{red} Grothendieck}.
\end{definition}

We have another free-forgetful adjunction $j^*: \mathsf{Op}^G(\V) \leftrightarrow \mathsf{Cat}^G(\V): j_!$, and note that $j^*$ commutes with taking $H$-fixed points for all $H\leq G$;
\[
\begin{tikzcd}
      \mathsf{Op}^G(\V) \arrow[d, "(-)^H"']
      \arrow[r, shift right, "j^*"']
      &
      \mathsf{Cat}^G(\V) \arrow[l, shift right, swap, "j_!"] \arrow[d, "(-)^H"]
      \\
      \Op(\V) \arrow[r, shift right, "j^*"']
      &
      \Cat(\V) \arrow[l, shift right, swap, "j_!"]
\end{tikzcd}
\]

\begin{definition}
      A functor $F: \mathcal C \to \mathcal D$ is
      \begin{enumerate}
      \item \textit{path-lifting}
            if it has the right lifting property against all maps
            $\1 \to \H$
            where $\1$ is the initial category and $\H$ is a $\V$-interval.
      \item \textit{essentially surjective}
            if for any object $d: \1 \to \mathcal D$,
            there is an object $c: \1 \to \mathcal C$
            and a $\V$-interval $\J$ fitting in to the commuting diagram below.
            \begin{equation}
                  \begin{tikzcd}
                        \1 \arrow[rr, "c"] \arrow[dr, "i_0"]
                        &&
                        \mathcal C \arrow[dd, "F"]
                        \\
                        &
                        \J \arrow[dr]
                        \\
                        \1 \arrow[ur, " i_1"] \arrow[rr,"b"]
                        &&
                        \mathcal D
                  \end{tikzcd}
            \end{equation}
      \end{enumerate}
\end{definition}

\begin{definition}
  We call a map $F: \O \to \P$ in $\mathsf{Op}^G(\V)$
  \begin{itemize}
  \item a {\em local fibration} (resp. {\em local weak equivalence}) if
        $F(\ksi): \O(\ksi)\to \P(F(\ksi))$
        is a fibration (resp. weak equivalence) in $\V^{\Stab(\ksi)}_{\F_\ksi}$ for all $\ksi\in \C(\O)^{\times n+1}$ and all $n$;
  \item a {\em local trivial fibration} if both a local fibration and a local weak equivalence;
  \item {\em essentially surjective} (resp. {\em path lifting}) if $j^*F^H$ is essentially surjective (resp. path lifting) in $\Cat(\V)$ for all $H\leq G$;
  \item a {\em fibration} if both path-lifting and a local fibration
  \item a {\em weak equivalence} if both essentially surjective and a local weak equivalence.
  \end{itemize}

  Moreover, foreshadowing, we call such a map
  a \textit{(trivial) cofibration} if it has the left lifting property against all trivial fibrations (resp. fibrations).
\end{definition}

\subsection{Generating (Trivial) Cofibrations and Local Fibrations}

We generalize and combine efforts from \cite{CM13b, BM13, Cav14}.

Fix a graph subgroup $\Gamma$ of $G \times \Sigma_n$, and $X \in \V^\Gamma$.
Define $C_\Gamma[X]$ to be the ``free operad with stabilizer $\Gamma$ generated by $X$''.
Specifically, this operad has colours $\mathfrak C_\Gamma := G \cdot_\Gamma \underline{n+1}$.
Now, letting $\ksi_0$ denote the signature $([e,1],[e,2],\dots,[e,n];[e,0])$,
we define
\begin{equation}
      C_\Gamma[X](\ksi) =
      \begin{cases}
            (g,\sigma)^{\**} X \qquad \qquad & \ksi = (g,\sigma).\ksi_0
            \\
            \varnothing & \mbox{otherwise,}
      \end{cases}
\end{equation}
where $g \in G$ and $\sigma \in \Sigma_n$ are chosen to be the \textit{minimal} elements in those groups with this property.

It is straightforward that the operad $\mathfrak C_\Gamma[X]$ has the universal property
\begin{equation}
      \Hom_{\Op^G(\V)}(C_\Gamma[X], \O) = \mathop\prod\limits_{\zeta \in (\mathfrak C(\O)^{\times n+1})^\Gamma}\Hom_{\V^\Gamma}(X, \O(\zeta)).
\end{equation}

Define $I_{loc}$ to be the set $\set{C_{\Gamma}[i_\gamma]}$ which runs over all
graph subgroups $\Gamma$ of $G \times \Sigma_n$ and
generating cofibrations $i_\gamma$ in $\V^\Gamma_{\F_n}$;
similarly let $J_{loc}$ denote the set $\set{C_{\Gamma}[j_\gamma]}$
with $j_\gamma$ the generating trivial cofibrations.

The universal property makes the following immediate.
\begin{corollary}[c.f. \cite{Cav14} Section 4.2, \cite{CM13b} 1.16]
  $\O\to \P$ is a local (trivial) fibration {\sc iff} $\O\to \P$ has the right lifting property against $J_{loc}$ (resp. $I_{loc}$).
\end{corollary}

Now, define $I_{G}:= I_{loc} \mathbin{\cup} \set{\varnothing \to G/H \cdot \1}_{H\leq G}$
and
$J_{G} := J_{loc} \mathbin{\cup} \set{G/H \cdot (\1 \to \J)}_{H\leq G,\ \J\in\mathbb{G}}$
where again $\1$ is the initial $\V$-category (thought as an operad), and $\mathbb{G}$ is a generating set of $\V$-intervals. 

\begin{lemma}
  \label{CAV_4.8}
  [c.f. \cite{Cav14} 4.8, \cite{BM13} 2.3, \cite{CM13b} 1.18]
  A map $F$ in $\mathsf{Op}^G(\V)$ is a trivial fibration {\sc iff} $F$ is a local trivial fibration such that $F^H$ is surjective on $H$-fixed colors for all $H\leq G$ {\sc iff} $F$ has the right lifting property against $I_{G}$.. 
\end{lemma}
\begin{proof}
      By definition, $F$ is a trivial fibration {\sc iff}
      it is a local trivial fibration such that $j^*F^H$ is path-lifting and essentially surjective for all $H\leq G$.
      Thus, \cite{Cav14} 4.8 immediately implies the first step.
      Moreover, right lifting against $I_{loc}$ is identical to being a local trivial fibration, while
      lifting against $\varnothing \to G/H\otimes \1$ precisely say that $F^H$ is surjective on colors;
      combining these observations yields the result.
\end{proof}

\begin{lemma}
  [c.f. \cite{CM13b} 1.20, \cite{Cav14} Section 4.3]
  $F$ has right lifting against $J_{G}$ {\sc iff} $F$ is a fibration.
\end{lemma}
\begin{proof}
      Again, lifting against $J_{loc}$ is identical to being a local fibration, while lifting against $G/H \cdot (\1 \to \J)$
      is equivalent to $F^H$ lifting against $\1 \to \J$, which is true exactly when it is path lifting by \cite{Cav14}. 
\end{proof}

\begin{lemma}
  \label{POINT_4_LEMMA}
  [c.f. \cite{CM13b} 1.19]
  $J_{G}\mbox{-cof} \subseteq I_{G}\mbox{-cof}$; that is, trivial cofibrations are cofibrations.
\end{lemma}
\begin{proof}
      It suffices to show that if $F$ has (right) lifting against $I_{G}$, it has lifting aginst $J_{G}$.
      Obviously, a local trivial fibration is a local fibration.
      On the other hand, by locality, any cofibration in $\mathsf{Op}^{G, \mathfrak C}(\V)$ for any $G$-set $\C$
      is a cofibration when considered in $\mathsf{Op}^G(\V)$.
      Hence, since $G/H \cdot (\1 \to \1 \amalg \1)$ is in $I_{G}\mbox{-cof}$, the composite
      \begin{equation}
            \begin{tikzcd}
                  G/H \cdot \1 \arrow[r, rightarrowtail]
                  &
                  G/H \cdot (\1 \amalg \1) \arrow[r, rightarrowtail]
                  &
                  G/H \cdot \J 
            \end{tikzcd}
      \end{equation}
      is in $I_{G}\mbox{-cof}$.
      Thus $J_G \subseteq I_G\mbox{-cof}$, implies the result.
\end{proof}

\subsection{Trivial cofibrations are weak equivalences}

\begin{lemma}
  The transfinite composition of essentially surjective maps is essentially surjective.
\end{lemma}
\begin{proof}
      Since taking fixed points commutes with filtered colimits, they commute with transfinite composition,
      and hence by \cite[4.17]{Cav14}, we are done.
\end{proof}

\begin{lemma}
  \label{J-CELL_LEMMA}
  [{c.f. \cite[4.20]{Cav14}}]
  If weak equivalences are closed under transfinite compositions, then relative $J_{G}$-cells are weak equivalences. 
\end{lemma}
\begin{proof}
      Since local weak equivalences are closed under transfinite composition by assumption, and
      essentially surjective maps are closed under transfinite composition by the above lemma,
      it suffices to prove that the pushout of a map $j\in J_{G}$ is a weak equivalence.
      If $j\in J_{loc}$, then since colimits in $\mathsf{Op}^G(\V)$ are computed in $\Op(\V)$,
      and since by \cite{Cav14} pushouts of this form can be computed fiberwise
      (that is, after shifting the operads into a single color),
      we are computing the pushout of a trivial cofibration in $\mathsf{Op}^{G,\C_{\ksi}}(\V)$,
      where $\C_{\ksi}$ is the $G$-set generated by the colors in the given signature $\ksi$.
      By the existance of the transferred model structure, this is again a trivial cofibration.
      Hence, the pushout is a local weak equivalence in $\mathsf{Op}^G(\V)$ which is the identity on colors,
      and hence a weak equivalence itself.
      
      Now, supppose $j$ is the map $G/H \cdot (\1 \to \J)$ for $\J$ a $\V$-interval.
      As in \cite{Cav14}, we can split this pushout into a composition of two pushouts
      \begin{equation}
            \begin{tikzcd}
                  G/H \cdot \1 \arrow[r] \arrow[d, "G/H \cdot \phi"']
                  % \arrow[dr,phantom, yshift=.1em, xshift=.5em, "\lrcorner" near end]
                  &
                  X \arrow[d,"\phi'"]
                  \\
                  G/H \cdot \J_{\set{0,0}} \arrow[r] \arrow[d, "G/H \cdot \psi"']
                  % \arrow[dr,phantom, yshift=.1em, xshift=.5em, "\lrcorner" near end]
                  &
                  X' \arrow[d,"\psi'"]
                  \\
                  G/H \cdot \J \arrow[r]
                  &
                  Y
            \end{tikzcd}
      \end{equation}
      where $\J_{\set{0,0}}$ is the full subcategory of $\J$ spanned by the object $O$.
      It suffices to show both $\psi'$ and $\phi'$ are local weak equivalences which are essentially surjective on fixed points. 
      
      Now, we know from \cite{Cav14} that $\psi$ is injective on colors and fully-faithful
      (that is, induces an isomorphism in $\Op^{\set{0}}(\V)$),
      and hence $G/H \cdot \psi$ is also injective on colors and fully-faithful as a map in $\Op(\V)$.
      Since colimits are created non-equivariantly, by \cite[Prop B.22]{Cav14} and the remark thereafter,
      we conclude that $\psi'$ is fully faithful in $\Op(\V)$, and hence is an isomorphism in $\Op(\V)_{\C(X')}$.
      But $\psi'$ is a $G$-map, and hence it is an isomorphism in $\mathsf{Op}^{G, \C(X')}(\V)$ as well,
      and hence is a local weak equivalence in $\mathsf{Op}^G(\V)$. 
      
      Moreover, we observe that $\C(Y) = \C(X') \amalg (G/H \times \set{1})$.
      Thus, if $x \in \C(Y)^K$ for $K \leq G$ is in $\C(X')$, we have essential surjectivity trivially:
      \begin{equation}
            \begin{tikzcd}
                  \1 \arrow[rrr, "x"] \arrow[dr, " i_0"]
                  &&&
                  (X')^K \arrow[dd, "\psi'"]
                  \\
                  &
                  \J \arrow[r]
                  &
                  \1 \arrow[dr, "x"]
                  \\
                  \1 \arrow[ur, " i_1"] \arrow[rrr,"x"]
                  &&&
                  (Y)^K
            \end{tikzcd}
      \end{equation}
      Lastly, if we consider any orbit of the new object $1\in \C(Y)$,
      there is an associated object $0 \in \C(X')$ such that the essentially surjectivity diagram
      is the same as the pushout diagram for $\psi$:
      \begin{equation}
            \begin{tikzcd}
                  G/H\otimes \1 \arrow[rr,"0"] \arrow[dr, "G/H\otimes i_0"]
                  &&
                  X' \arrow[dd, "\psi'"]
                  \\
                  &
                  G/H\otimes \J \arrow[dr, "\mbox{\large $\lrcorner$}" near end]
                  \\
                  G/H\otimes \1 \arrow[ur, "G/H\otimes i_1"] \arrow[rr, "1"]
                  &&
                  Y
            \end{tikzcd}
      \end{equation}
      Hence $\psi'$ is essentially surjective and a local weak equivalence, hence a weak equivalence in $\mathsf{Op}^G(\V)$. 

      Similarly, when considering $\phi'$, \cite[4.20]{Cav14} again says that pushouts of this form are created in $\Op(\V)_{\C(X)}$ as the pushout
      \begin{equation}
            \begin{tikzcd}
                  p_! (G/H \cdot \1) \arrow[r, "p"] \arrow[d, "p_! (G/H \cdot \phi)"']
                  &
                  X \arrow[d,"\phi'"]
                  \\
                  p_! (G/H \cdot \J_{\set{0,0}}) \arrow[r]
                  &
                  Y
            \end{tikzcd}
      \end{equation}
      In particular, this implies $\phi'$ is bijective on objects, and hence essentially surjective.
      Moreover, as $\phi$ is a trivial cofibration in $\Op(\V)$ by \cite{Cav14},
      $p_! (G/H \cdot \phi)$ is a trivial cofibration in $\Op^{G,\C(X)}(\V)$.
      Thus $\phi'$ is a trivial cofibration in $\mathsf{Op}^{G,\C(X)}(\V)$,
      and thus is a local weak equivalence in $\Op^G(\V)$.

      Hence both $\phi'$ and $\psi'$ are weak equivalences in $\mathsf{Op}^G(\V)$, so the result is proved.
\end{proof}

\todo[inline]{come back}

\subsection{2-out-of-3}

We recall some equivalence relations on objects in a $\V$-category \cite{Cav14, BM13}:
\begin{definition}
  Given $\mathcal{C}$ in  $\Cat(\V)$ and $a,b\in\mathrm{Ob}(\mathcal C)$, we say $a$ and $b$ are
  \begin{itemize}
  \item {\em equivalent} if there exists a map $\gamma: \J \to \mathcal C$ such that
        $\gamma i_0 = a$ and $\gamma i_1 = b$
        for some $\V$-interval $\J$;
  \item {\em virtually equivalent} if $a$ and $b$ are equivalent in some fibrant replacement
        $\mathcal C_f$ of $\mathcal C$ in $\Cat^{\mathrm{Ob}(\mathcal C)}(\V)$;
  \item {\em homotopy equivalent} if there exist maps $\alpha: 1_\V \to \mathcal C_f(a,b)$ and $\beta: 1_\V\to \mathcal C_f(b,a)$ such that $\beta\alpha$ and $\alpha\beta$ and homotopic to the identity arrows $1_V\to \mathcal C_f(a,a)$ and $1_V\to\mathcal C_f(b,b)$, respectively. Equivalently, if $a$ and $b$ are isomorphic in the 1-category $\pi_0\C_f = \mathrm{Ho}\V(1_\V, \mathcal C_f)$.
  \end{itemize}
\end{definition}

We equivariantize these definitions to $G\Cat(\V)$ and $\mathsf{Op}^G(\V)$:
\begin{definition}
  Given $\mathcal{C}\in G\Cat(\V)$ and $a,b\in \mathrm{Ob}(\mathcal{C})$, we say $a$ and $b$ are
\begin{itemize}
\item {\em equivalent} if $\Stab_G(a) = \Stab_G(b) =: H$ and are equivalent in $\mathcal{C}^H$;
\item {\em virtually equivalent} if they are equivalent in some fibrant replacement $\mathcal{C}_f$ of $\mathcal{C}$ in $G\Cat(\V)_{\mathrm{Ob}(\mathcal{C})}$;
\item {\em homotopy equivalent} if $\Stab_G(a) = \Stab_G(b) =: H$ and are homotopy equivalent in $\mathcal{C}^H$. 
\end{itemize}
For an operad $\O\in \mathsf{Op}^G(\V)$ and $a,b\in \C(\O)$, we say $a$ and $b$ are {\em equivalent} (resp. {\em virtually equivalent}, {\em homotopy equivalent}) if they are so in $j^*\O$. 
\end{definition}

The following three lemmas follow directly from the proofs of their non-equivariant counterparts:
\begin{lemma}
  [c.f. \cite{Cav14} 4.10]
  Equivalence and virtual equivalence define equivalence relations on $\C(\O)$. \qed
\end{lemma}
\begin{lemma}
  [c.f. \cite{Cav14} 4.13, \cite{BM13} 2.11]
  Virtually equivalent colors are homotopy equivalent. 
\end{lemma}
\begin{lemma}
  [c.f. \cite{Cav14} 4.12, \cite{BM13} 2.10]
  If $\V$ is right proper, then all virtual equivalent colors are equivalent. 
\end{lemma}

\begin{lemma}
  [c.f. \cite{Cav14} 4.11 and \cite{BM13} 2.9]
  Any local weak equivalence $F: \O\to \P$ in $\mathsf{Op}^G(\V)$ reflects virtual weak equivalences.
\end{lemma}
\begin{proof}
  As in the non-equivariant case, $F$ being a local weak equivalence implies we have a local trivial fibration $F': \O_f\to \P_f$. Thus, for $\Stab(a) = \Stab(b) =: H$, any virtual equivalence $\J \to \P_f^H$ between colors $F'(a) = F(a)$ and $F'(b) = F(b)$ in the image of $F^H$ lifts to one $\J\to \O_f^H$ between $a$ and $b$ (in particular, since the fibration has source $\O_f^H$, the source colors $a$ and $b$ have stabilizer at least $H$, and since their images have stabilizer exactly $H$, so do they). 
\end{proof}

\subsubsection{Equivalences between levels}

We would like to generalize \cite[4.14 and 4.15]{Cav14}, which state that
$\O(\ksi)$ and $\O(\ksi')$ are equivalent in $\V$ if $\ksi$ and $\ksi'$ are related by a string of weak equivalences of colors,
implying that weak equivalences satisfy the 2-out-of-3 property.
However, the notion of weak equivalence we use takes place in $\Vsigma$ as opposed to $\V$,
and the colors can be interchanged in interesting ways via action by $G$.
Thus, we will need to change an entire orbit worth of colors in order to create such a homotopy equivalence. 

To that end, fix $\O_f$ fibrant in $\mathsf{Op}^G(\V)$ with colors $\C$.
Supppose $c_1$ and $d_1$ are homotopy equivalent in $\C$, each with stabilizer $H$,
via maps $\alpha: 1_\V \to \O_f^H(c_1,d_1)$ and $\beta: 1_\V \to \O_f^H(d_1,c_1)$.
Further, let $\ksi = (c_1,c_2,\ldots, c_n;c)$ be a signature in $\C(\O_f)$, with $K := \Stab(c)$.
\todo[inline]{come back}
We define $R\subseteq \set{1,2,\ldots, n}$ to be the set of all $r$ such that $c_r = k_r c_r$ for some $k_r\in K$. Moreover, define:
\begin{itemize}
\item $H_r := k_rHk_r^{-1}$ (if $i\not\in R$, $H_i := H$);
\item for each $i\in \set{1,\ldots, n}$, $d_i := k_id_1$ if $i\in R$, and $c_i$ otherwise; and
\item $\alpha_r$ and $\beta_r$ by post-composition of $\alpha$ or $\beta$ by the action of $k_r$ (if $i\not\in R$, let $\alpha_i$ and $\beta_i$ be the identity map on $c_i = d_i$).  
\end{itemize}
Note that each of these is independent of the choice of $k_r\in k_r H$ such that $k_r c_1 = c_r$. 

Note that $\Stab(c_r) = \Stab(d_r)$, and moreover the pair $(\alpha_r,\beta_r)$ realizes a homotopy equivalence between $c_r$ and $d_r$ (post composition by an isomorphism remains an isomorphism in $\mathrm{Ho}\V$). 

\begin{lemma}
  With the above notation, $\Stab_{G\times \ksi_n}(c_1,\ldots, c_n;c) = \Stab_{G\times\ksi_n}(d_1,\ldots, d_n;d)$. 
\end{lemma}
\begin{proof}
  Suppose $(k,\pi)\in \Stab_{G\times \ksi_n}(c_1,\ldots, c_n;c)$, so that $kc_{\pi^{-1}(i)}= c_i$ for all $i$. We need to show $kd_{\pi^{-1}(i)} = d_i$. We first note that $\pi$ acts n $R$ and $\set{1,\ldots, n}\setminus R$ independently. Thus, if $i=r\in R$ we have $k c_{\pi^{-1}(r)} = c_r$, and hence $k k_{\pi^{-1}r}c_1 = k_r c_1$, so $k_r^{-1} k k_{\pi^{-1}r} =:h_r \in H$. Hence 
\[
k d_{\pi^{-1}(r)} = k k_{\pi^{-1}(r)} d_1 = k_i h_r d_1 = k_i d_1 = d_i,
\]
as desired. On the other hand, if $i\not\in R$, then $k d_{\pi^{-1}(i)} = k c_{\pi^{-1}(i)} = c_i = d_i$. 
\end{proof}

With the same notation as above, let $\otimes \hat\alpha_r$ be the composition
\[
\begin{tikzcd}
  \O_f(c_1,\ldots, c_n;c) \cong \O_f(c_1,\ldots, c_n;c) \otimes 1_V^{\otimes n} \arrow[r, "1\otimes \alpha_1 \otimes\ldots\otimes\alpha_n"] & \O_f(c_1,\ldots,c_n;c) \otimes \O_f^{H_1}(d_1;c_1)\otimes\ldots\otimes\O_f^{H_n}(d_n;c_n) \arrow[r,"\circ"] & \O(d_1,\ldots,d_n;c).
\end{tikzcd}
\]

\begin{lemma}
  The map $\otimes\hat\alpha_r$ descends to $\Lambda$ fixed points for any subgroup $\Lambda\leq \Stab(c_1,\ldots, c_n;c) = \Stab(d_1,\ldots,d_n;c)$.
\end{lemma}
\begin{proof}
  Since the composition structure maps of $\O_f$ are natural in $G\times\ksi_n$, we just need to show that $\otimes \hat\alpha_r$ is preserved by all $(k,\pi)\in Stab(c_1,\ldots, c_n;c)\subseteq G\times \ksi_n$. But we observe this directly:
\[
(k,\pi).(\otimes \hat\alpha_r) = \otimes k\hat\alpha_{\pi^{-1}r} = \otimes k k_{\pi^{-1}r}\hat\alpha = \otimes k_r h_r\hat\alpha = \otimes k_r\hat\alpha = \otimes \hat\alpha_r.
\]
\end{proof}

Since each $\alpha$ is a homotopy equivalence, any ``legal application'' of the map $\otimes \hat\alpha_r$ is a homotopy equivalence in $\V$. Since this homotopy equivalence descends equivariantly to any $\Lambda$-fixed points, it in particular a homotopy equivalence between $\O_f(c_1,\ldots,c_n;c)$ and $\O_f(d_1,\ldots,d_n;c)$ in $\Vsigma$.

In particular, we have:
\begin{proposition}
  \label{CAV_4.14_PROP}
  [c.f. \cite{Cav14} 4.14]
  Let $\O$ be any operad in $\mathsf{Op}^G(\V)$, $\ksi = (c_1,\ldots,c_n;c)\in Seq(\C(\O))$, $K = \Stab_G(c)$, and suppose $(c_i,d_i)$ and $(c,d)$ are pairs of homotopy equivalent colors. Then there exists a zig-zag of weak equivalences in $\Vsigma$ between $\O(\ksi)$ and $\O(\ksi')$, where $\ksi' = (d_1,\ldots, d_n;d)$ with
  \begin{itemize}
  \item $R\subseteq \set{1,\ldots,n}$ those $r$ such that $c_r\in Kc_i$; in particular, choose $k_r\in K$ with $k_r c_i = c_r$ (if $i\not\in R$, let $k_i = 1$); and
  \item $d_i = k_i c_i$.
  \end{itemize}
Moreover, any functor $F:\O\to \P$ induces a functorial zig-zag of weak equivalences between $\P(F(\ksi))$ and $\P(F(\ksi'))$.
\end{proposition}
\begin{proof}
  The above lemmas and discussion say there exists a homotopy equivalence in $\Vsigma$ between $\O_f(c_1,\ldots, c_n;c)$ and $\O_f(d_1,\ldots, d_n;c)$. Moreover, if $(\alpha',\beta')$ realize the homotopy equivalence between $c$ and $d$, then the analogously defined $\hat\alpha'$ descends to all fixed points (as $K\times\ksi_n$ acts trivally on $\O_f^K(d;c)$). Therefore $\hat\alpha'$ is a homotopy equivalence where ever it is defined, and hence $\O_f(c_1,\ldots, c_n;c)$ is homotopy equivalent in $\Vsigma$ to $\O_f(d_1,\ldots, d_n;d)$, and thus weakly equivalent. Finally, the fibrant replacement weak equivalences $\O(\ksi)\to\O_f(\ksi)$ and $\O(\ksi')\to \O_f(\ksi')$ complete the zig-zag.

The second statement follows identically as in the non-equivariant case found in \cite{Cav14} 4.14.
\end{proof}

\begin{proposition}
  \label{CAV_4.15_PROP}
  [c.f. \cite{Cav14} 4.15]
  The class of weak equivalences in $\mathsf{Op}^G(\V)$ satisfies the 2-out-of-3 condition.
\end{proposition}
\begin{proof}
  Essential surjectiving holds in all cases since check it reduces to just checking multiple instances of the non-equivariant case, where it holds via (\cite{Cav14}, 4.15). Now let $\O \xrightarrow{F} \P \xrightarrow{L} \Q$ be a composition of maps in $\mathsf{Op}^G(\V)$. In each case, we just need to check that the odd map out is a local weak equivalence.
  \begin{description}
  \item[Case I:] $F$ and $L$ are weak equivalences.\\
    This is trivial: $\O(\ksi)^\Gamma \simeq \P(F(\ksi))^\Gamma \simeq \Q(LF(\ksi))^\Gamma$.
  \item[Case II:] $L$ and $FL$ are weak equivalences.\\
    This too is trivial by 2-out-of-3. 
  \item[Case III:] $F$ and $LF$ are weak equivalences.\\
    Given $\tau = (d_1,\ldots,d_n;d)\in Seq(\C(\P))$, let $K = \Stab(d)$, and pick a set of representatives ${d_r}_{r\in R}$ for the quotient ${d_1,\ldots, d_r}/K$; thus, for each $i\not\in R$, we have some $r\in R$ and $k_{r,i}\in K$ such that $d_i = k_{r,i}d_r$. By the essential surjectivity of $F$, there exist $c_r\in \C(\O)$ such that $\Stab(c_r) = \Stab(d_r)$ and $F(c_r)$ is equivalent - hence homotopy equivalent - to $d_r$. Similarly, there exists $c\in \C(\O)$ such that $\Stab(c) = \Stab(d)$ and $F(c)$ and $d$ are homotopy equivalent. 

Now, extend the set $\set{c_r}_{r\in R}$ to a signature $(c_1,\ldots, c_n;c)$ via $c_i = k_{r,i}c_r$; thus $F(c_i)$ is homotopy equivalent to $d_i$ via $k_{r,i}\gamma_r$, where $\gamma_r$ realizes the homotopy equivalence between $F(c_r)$ and $d_r$. In particular, these homotopy equivalences are coherent, as in the proof of \ref{CAV_4.14_PROP}. Therefore, we have a diagram of the form
\[
\begin{tikzcd}
  \O(c_1,\ldots, c_n;c) \arrow[r, "(1)"] & \P(F(c_1),\ldots, F(c_n); F(c)) \arrow[d,dash, "(3)"] \arrow[r, "(2)"] & \Q(LF(c_1),\ldots, LF(c_n);LF(c)) \arrow[d, dash, "(4)"] \\
  & \P(d_1,\ldots, d_n;d) \arrow[r, "(5)"] & \Q(L(d_1),\ldots, L(d_n); L(d))
\end{tikzcd}
\]
where ``$(1)$'' is a weak equivalence (in $\Vsigma$) by assumption, ``$(2)$'' is a weak-equivalence by 2-out-of-3 in $\Vsigma$, ``$(3)$'' and ``$(4)$'' are zig-zags of a weak equivalence by \ref{CAV_4.14_PROP}. Thus ``$(5)$'' is a weak equivalence again by 2-out-of-3 in $\Vsigma$. Since the initial $\ksi$ was arbitrary, we have that $L$ is a local weak equivalence, as desired.
  \end{description}
\end{proof}

\subsection{Model structure}

\begin{theorem}
  \label{MODEL_THEOREM}
  Suppose $\V$ is a cofibrantly generated monoidal model category such that the unit is cofibrant, the model structure is right proper, there exists a set $\mathbb{G}$ of generating $\V$-interval, is cartesian closed, plus for all finite groups $G$, $G\Cat(\V)$ and $\V^G$ have cellular fixed-point functors which commute with a symmetric monoidal fibrant replacement functor in $\V$. If the class of weak equivalences of $\mathsf{Op}^G(\V)$ is closed under transfinite compositions, then there exists a cofibrantly generated model structure on $\mathsf{Op}^G(\V)$ with fibrations, weak equivalences, generating cofibrations, and generating trivial cofibrations as described above.
\end{theorem}
\begin{proof}
  Since $\mathsf{Op}^G(\V)$ is complete and cocomplete, it suffices to prove (following \cite{Hov98} Theorem 2.1.19) that:
  \begin{enumerate}
  \item the class of weak equivalences has the 2-out-of-3 property and is closed under retracts;
  \item the domains of $I_{G}$ (resp. $J_{G}$) are small relative to $I_{G}$-cell (resp. $J_{G}$-cell);
  \item $I_{G}$-inj $= W\cap J_{G}$-inj;
  \item $J_{G}$-cell $\subseteq W\cap I_{G}$-cof.
  \end{enumerate}
  Point (1) follows from \ref{CAV_4.15_PROP} and the fact that if $L$ is a retract of $F$, $L^H$ is a retract of $F^H$. Point (2) follows since colimits in $\mathsf{Op}^G(\V)$ are created in $\Op(\V)$, and it holds non-equivariantly. Point (3) follows from \ref{CAV_4.8}. Point (4) follows from \ref{POINT_4_LEMMA} and \ref{J-CELL_LEMMA}.
\end{proof}





\newpage
\bibliography{biblio}{}

\bibliographystyle{alpha}





\end{document}









