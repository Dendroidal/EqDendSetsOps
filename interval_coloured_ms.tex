\documentclass[psamsfonts,oneside,10pt,letterpaper
,draft
]{amsart}%

% \usepackage{hyperref}

\input{commands.tex}%

% \usepackage{fullpage}




\author{Peter Bonventre, Lu\'is Alexandre Pereira}%
\title{Coloured Model Structure Using Interval Objects}%
\date{\today}



% -------- TIKZ -----------------------------------------
\usepackage{tikz}%
\usetikzlibrary{matrix,arrows,decorations.pathmorphing,
  cd,patterns,calc}
\tikzset{%
  treenode/.style = {shape=rectangle, rounded corners,%
    draw, align=center,%
    top color=white, bottom color=blue!20},%
  root/.style     = {treenode, font=\Large, bottom color=red!30},%
  env/.style      = {treenode, font=\ttfamily\normalsize},%
  dummy/.style    = {circle,draw,inner sep=0pt,minimum size=2mm}%
}%

\usetikzlibrary[decorations.pathreplacing]

% -------------------- new commands --------------------


\renewcommand{\C}{\ensuremath{\mathfrak{C}}}
\renewcommand{\fc}{\ensuremath{\mathfrak{c}}}
\renewcommand{\F}{\mathcal F}
\newcommand{\FF}{\ensuremath{\mathbb{F}}}
\renewcommand{\H}{\ensuremath{\mathbb{H}}}
\newcommand{\I}{\ensuremath{\mathbb{I}}}
\newcommand{\J}{\ensuremath{\mathbb{J}}}
\renewcommand{\1}{\ensuremath{\mathbb{id}}}
\renewcommand{\P}{\ensuremath{\mathcal{P}}}
\newcommand{\Q}{\ensuremath{\mathcal{Q}}}
% \newcommand{\V}{\mathcal V}
\newcommand{\Vsigma}{\ensuremath{\V^{\Stab(\sigma)}_{\F_\sigma}}}
\newcommand{\N}{\mathbb N}


\newcommand{\floor}[1]{\left \lfloor #1 \right \rfloor}
% \newcommand{\dasharrowdbl}{\dashrightarrow\mathrel{\mkern-14mu}\dashrightarrow}
% \newcommand{\xdasharrowdbl}[2][]{%
% \xdashrightarrow[#1]{#2}\m}
% \newcommand{\tall}[2]{\ensuremath{\begin{tikzcd} #1 \arrow[r,dashed,twoheadrightarrow] & #2 \end{tikzcd}}}
\newcommand{\tall}{\ensuremath{\Rightarrow}}%\rightsquigarrow
\newcommand{\Tall}{\ensuremath{\Downarrow}}%\rightsquigarrow

\newcommand{\T}{\ensuremath{\mathbb{T}}}

\newcommand{\iso}{\ensuremath{\mathsf{Iso}}}
\newcommand{\alt}{\ensuremath{\mathsf{alt}}}

\DeclareMathOperator{\Ho}{Ho}

% \newcommand{\Cat}{\mathsf{Cat}}
% \newcommand{\Op}{\mathsf{Op}}
% \renewcommand{\Sym}{\mathsf{Sym}}
% \newcommand{\Stab}{\mathrm{Stab}}

\newcommand{\Omeganc}{\ensuremath{\Omega_{(n-1,1)-\mathfrak c}}}
\newcommand{\Omeganbin}{\ensuremath{\Omega_{(n-1,1)-bin}}}
\newcommand{\Omeganalt}{\ensuremath{\Omega_{(n-1,1)-alt}}}


\setenumerate[0]{label=(\arabic*)}
\newcommand{\primed}[1]{$#1 '$}




% -------- COMMANDS ON DRAFT--------------------------

\usepackage{ifdraft}
\ifdraft{
  \color[RGB]{63,63,63}
  % \pagecolor[rgb]{0.5,0.5,0.5}
  \pagecolor[RGB]{220,220,204}
  % \color[rgb]{1,1,1}
}


\usepackage[draft]{showkeys}
\usepackage{todonotes}%[obeyDraft]








\begin{document}		\maketitle%

\tableofcontents

\section{Introduction}%
% $\dashrightarrow$
% $\dasharrowdbl$
% $\twoheadrightarrow$
% $\tall{A,B}$
% \begin{tikzcd} A \arrow[r,dashed,twoheadrightarrow] & B \end{tikzcd}

% \[
%       \begin{tikzpicture}[grow=up, every node/.style = {font=\footnotesize},level distance = 2em]%
%             \tikzstyle{level 2}=[sibling distance=3em]%
%             \tikzstyle{level 3}=[sibling distance=1em]%
%             \node {}%
%             child{node [dummy,label=-10:$\varphi$] {}%
%               child{node [dummy,label=right:$\psi_3$] {}}%
%               child{node [dummy,label=left:$\psi_2$] {}%
%                 child{}%
%                 child{}%
%                 child{}%
%               }%
%               child{node [dummy,label=left:$\psi_1$] {}%
%                 child{node {} }%
%                 child{node {} }%
%               }%
%             };%
%       \end{tikzpicture}%
% \]%





\subsection{Assumptions}

For our main result, we will need the following assumptions on our enriching category $\V$:

\begin{enumerate}
\item $\V$  cofibrantly generated monoidal model category; % strongly cofibrantly gen?
\item $G\V$ and $\mathsf{Cat}^G(\V)$ have cellular fixed point functors for all finite groups $G$;
      % \item $\V$ satisfies transfer for symmetric operads (in particular if $\V$ has a cofibrant unit, has a cocommutative Hopf interval object (in particular, cartesian closed \cite{BM03}), and admits a monoidal fibrant replacement functor \cite{BM07});
\item STUFF SO THAT COLOUR-FIXED MODEL STRUCTURES EXIST (including $\Cat(\V)$, $\Op(\V)$ have model structures) 
      \footnote{This will likely include at least the following:
      $\V$ admits a symmetric monoidal fibrant replacement functor which commutes with $H$-fixed points: $R(X)^H = R(X^H)$.}
\item $\Cat(\V)$ has a generating set of intervals;
\item $\V$ is right proper
\item The class of weak equivalences in $\Op^G(\V)$ (to be defined later) is closed under transfinite compositions.
\end{enumerate}

Most of the results will not need all of these assumptions.

Let $I$ and $J$ be the sets of generating cofibrations and generating trivial cofibrations of $\V$.

\begin{notation}
      We fix a (random) total ordering on the elements of $G$ and of every $\Sigma_n$.
\end{notation}

Stuff.

\newpage
\section{Equivariant operads}

Stuff.
Not used here.
\todo[inline]{other paper}

\subsection{Operads of equivariant operads}

Given a $G$-set $\C$, we define a colored operad $\Op_\C$ whose algebras are $G$-operads with colors $\C$.

The colors of $\Op_\C$ are pairs $(C,\fc)$ with $C$ a corolla, and $\fc: E(T) \to \C$ a set map.

Morphisms are generated by two types of maps:
\begin{enumerate}
\item $(T,\fc,\sigma,\set{\phi_i},\tau)\in \Op_\C((C_1,\fc_1), \ldots, (C_n,\fc_n); (C_0,\fc_0))$, where
      \begin{itemize}
      \item $T$ is a tree with $n$ vertices;
      \item $\fc: E(T)\to \C$ is a set math with $\fc(L(T)) = \fc_0(L(C_0))$ and $\fc(r) = \fc_0(r)$;
      \item $\sigma: \set{1,\ldots, n}\to V(T)$ a bijection such that, if $C[\sigma(i)]$ is the corolla in $T$ connectd to the vertex $\sigma(i)$, we have isomorphisms of colored corollas $\phi_ii: (C[\sigma(i)], \fc|_{C[\sigma(i)]})\xrightarrow{\cong} (C_i, \fc_i)$ for all $i\in \set{1,\ldots, n}$;
      \item $\tau: L(C_0)\to L(T)$ a bijection such that $\fc\tau = \fc_0$;
      \end{itemize}
\item $g\in \Op_\C((C,\fc); (C,g\fc))$
\end{enumerate}
modulo the relation $(T,\fc,\sigma,\set{\phi_i},\tau) = (T,\fc\pi^{-1},\pi\sigma,\set{\pi\phi_i},\pi\tau)$ for all $\pi\in \mathrm{Aut}(T)$.

Composition is defined as follows:
\begin{enumerate}
\item $[T,\fc,\sigma,\set{\phi_i},\tau] \circ_j [S,\mathfrak{d}, s, \set{\psi}, t] = [T\circ_{\sigma(j)}S, \fc', \sigma', \set{\phi_i}\amalg\set{\psi_{\sigma' i}}, \tau']$ where, if $T$ has $n$ vertices and $S$ and $m$ vertices, we have defined
      \[
            \fc'(e) = \begin{cases}\fc(e) & e\in T\\ \mathfrak{d}(e) & e\in S\end{cases} \qquad \mbox{and} \qquad 
            \sigma'(i) =
            \begin{cases}
                  \sigma(i) & i< j\\
                  s(i-j)+j & j\leq i < j+m\\
                  \sigma(i-m)+m & i\geq j+m
            \end{cases}
      \]
      and $\tau'$ synthesizes $\tau$ and $t$ (if neceesary) to retain the identification information on the leaves;
\item $g\circ h = gh$;
\item composition between $g$ and $[T]$ is free modulo
      \[
            [T,\fc,\sigma,\set{\phi_i},\tau] \circ (g,\ldots,g) = g\circ [T,g\fc,\sigma,\set{\phi_i}, \tau].
      \]
\end{enumerate}

\newpage
\section{Model stuctures on colour-fixed equivariant operads}

\todo[inline]{OTHER PAPER}

\begin{proposition}[{\cite[2.6]{Ste16}}]
      Assume $\V$ is cofibrantly generated model category such that $\V^G$ has cellular fixed points for all finite groups $G$,
      and let $\F$ be any family of subgroups of $G$.
      Then $\V^G$ has the $\F$-model structure, a cofibrantly generated model structure where
      weak equivalences and fibrations are defined by $(-)^H$ for all $H\in \F$.
      Moreover, the generating (trivial) cofibrations are given by
      $I_\F = \sets{G/H \cdot i}{i\in I,\ H\in \F}$ and
      $J_\F = \sets{G/H\otimes j}{j\in J,\ H\in \F}$ respecitvely. 
\end{proposition}

Now, let $G$ be a finite group, and suppose we have an {\em indexing family} $\F = \set{\F_n}$ of families of subgroups of $\set{G\times \Sigma_n}$ (c.f. \cite{BH15}). 

Let $\C$ be a $G$-set, of {\em colors}, and $Seq(\C)$ be the set of {\em signatures} in $\C$, defined by
$\sets{
  (a_1,\ldots, a_n; a)\in \C^n\times\C
}
{n\in\mathbb{N}}$.
For each $n$, the set of signatures of length $n+1$ have an action by $G\times \Sigma_n$, where $G$ acts on all components, and $\Sigma_n$ acts on all but the last one. 


\begin{definition}
      Let $\Op^{G,\mathfrak C}(\V)$ be the category of $\C$-colored $\V$-operads.
\end{definition}

For each $G$-set $\C$, we have a monoidal free-forgetful adjunction as below;
\begin{equation}
      \begin{tikzcd}
            fgt: \Op^{G,\mathfrak C}(\V) \arrow[r, shift right]
            &
            \V^{\Sym^{G,\mathfrak C}(\V)} \simeq
            \displaystyle\prod_{n \in \N} \V^{B_{\mathfrak C^{\times n+1}}(G \times \Sigma_n)}
            \arrow[l, shift right]
            \arrow[r, shift right]
            &
            \displaystyle\prod_{n\in\N}\displaystyle\prod_{\ksi\in \C^{\times n+1}}\V^{\Stab(\ksi)}
            \arrow[l, shift right]
            : \FF_\C
      \end{tikzcd}
\end{equation}

Given a signature $\ksi$ of length $n+1$, define $\F_\ksi$ to be the family of subgroups $\Lambda\in \F_n$ such that $\Lambda\leq \Stab(\ksi)$.
\begin{corollary}
      For any $\mathfrak C$-signature $\ksi$ and family $\F_n$, $\V^{\Stab(\ksi)}$ has the $\F_\ksi$-model structure.
\end{corollary}

\begin{definition}
      Suppose $\V$ has
      and fix a $G$-set $\mathfrak C$.
      Given $\F = \set{\F_n}$ a collection of families $\F_n$ of graph subgroups of $G \times \Sigma_n$,
      and a $\mathfrak C$-signature $\ksi \in \mathfrak C^{\times n+1}$,
      let $\F_\ksi$ denote those subgroups in $\F_n$ which live in $\Stab(\ksi)$.

      Then the $\F$-model structure on the category $\Op^{G,\mathfrak C}_\F(\V)$, denoted $\Op^{G,\mathfrak C}_\F(\V)$,
      is the model structure, if it exists, transferred from the free-forgetful adjunction above,
      where $\V^{\Stab(\ksi)}$ is endowed with the $\F_\ksi$-model structure.
\end{definition}
\begin{proposition}
      If $\V$ has {\color{red} GOOD PROPERTIES}
      (at least cellular, and maybe
      $\V^{\Stab(\ksi)}_{\F_\ksi}$ has a symmetric monoidal fibrant replacement functor which commutes with fixed points),
      then this model structure exists.
\end{proposition}
\begin{proof}
      Other paper.
\end{proof}
% \begin{proof}
%       We use the same arguments as in \cite{BM03}, Theorem 3.2. In particular, we know that $fgt$ preserves filtered colimits. Moreover, given $\O\in \Op^{G,\mathfrak C}(\V)$, define $\tilde \O$ by $\tilde \O(\ksi) = R\O(\ksi)$, where $R$ is the fibrant replacement functor in $\V$; since $R$ commutes with fixed points, this is in fact a levelwise fibrant replacement ($(\O_f)^H = (\O^H)_f$). Further, since $R$ is symmetric monoidal, the operad structure on $\O$ induces one of $\tilde \O$.
%       {\color{red}
%       Lastly, we need a functorical path object in $\Op^{G,\mathfrak C}(\V)$. Since $\V$ cartesian monoidal, given any cocommutative Hopf interval object $\H$ , and fibrant $\O$, we have 
%       \[
%             \begin{tikzcd}
%                   \O \cong \O^I \arrow[r,"\simeq"] & \O^\H \arrow[r,twoheadrightarrow] & \O^{I\coprod I} \cong \O\times \O
%             \end{tikzcd}
%       \]
%       for which the first arrow (resp. second arrow) is a cofibration (resp. fibration) by the pushout-product axiom of monoidal model categories, since $I$ is cofibrant. Lastly, these are (maps of) operads since $\V$ cartesian closed.
%     } % color red
%       \todo[inline]{the red above is quite wrong.}
% \end{proof}

\begin{remark}
      {\color{red} FROM HERE ON AFTER, WE ASSUME THAT $\Cat(\V)$ and $\Op^{G,\mathfrak C}(\V)$ have the induced model structures.}
\end{remark}

\begin{definition}
      Let $\mathsf{Op}^G(\V)$ denote the category of $G$-objects in $\mathsf{Op}(\V)$.
      Equivalently, this is the category of $\V$-operads with any $G$-set of colors,
      where a map $F: \O\to \P$ is defined by a pair $(F_0,F)$ with
      $F_0: \C(\O)\to \C(\P)$ a map of colours and
      $F: \O\to F^*\P$ a map of $\C(\O)$-coloured operads, where $F^*P(\ksi) = P(F_0(\ksi))$.
      Also, {\color{red} Grothendieck}.
\end{definition}

\newpage

\section{Model category for all colours}

We have another free-forgetful adjunction $j^*: \mathsf{Op}^G(\V) \leftrightarrow \mathsf{Cat}^G(\V): j_!$, and note that $j^*$ commutes with taking $H$-fixed points for all $H\leq G$;
\[
      \begin{tikzcd}
            \mathsf{Op}^G(\V) \arrow[d, "(-)^H"']
            \arrow[r, shift right, "j^*"']
            &
            \mathsf{Cat}^G(\V) \arrow[l, shift right, swap, "j_!"] \arrow[d, "(-)^H"]
            \\
            \Op(\V) \arrow[r, shift right, "j^*"']
            &
            \Cat(\V) \arrow[l, shift right, swap, "j_!"]
      \end{tikzcd}
\]


% \begin{definition}
%       For a category $\V$, a {\em Hopf interval object} in is a cofibrant Hopf object $\H\in\V$ for which there exists a factorization
%       $\begin{tikzcd} 
%             I\coprod I \arrow[r,rightarrowtail] & \H \arrow[r, twoheadrightarrow, "\simeq"] & I
%       \end{tikzcd}$ 
%       of the fold map $I$, the unit in $\V$. We can $\H$ {\em cocommutative} if its Hopf structure is. 
% \end{definition}

\begin{definition}
      % Similarly but different:
      let $\I$ be the $\V$-category with objects $\set{0,1}$ with $\I(0,0) = \I(0,1) = \I(1,0) = \I(1,1) = 1_V$. A {\em $\V$-interval} is a cofibrant object in $\V\Cat_{\set{0,1}}$ (with the transfered model structure) weakly equivalent to $\I$. A set $\mathcal{G}$ of $\V$-intervals is {\em generating} if all $\V$-intervals $\J$ can be obtained as a retract of a trivial extension of an element in $\mathcal{G}$ in $\V\Cat_{\set{0,1}}$:
      \[
            \begin{tikzcd}
                  \mathbb{G} \arrow[r,rightarrowtail, "\simeq"] & \mathbb{K} \arrow[r,yshift=-.3em, "r"'] & \mathbb{J} \arrow[l,yshift=.3em, "i"']
            \end{tikzcd}
      \]
\end{definition}

\begin{definition}
      We recall, a functor $F: \mathcal C \to \mathcal D$ is
      \begin{enumerate}
      \item \textit{path-lifting}
            if it has the right lifting property against all maps
            $\1 \to \H$
            where $\1$ is the $\V$-category representing a single object
            (so $\1$ has one object, and mapping object is the tensor unit $1_\V$ of $\V$),
            and $\H$ is a $\V$-interval.
      \item \textit{essentially surjective}
            if for any object $d: \1 \to \mathcal D$,
            there is an object $c: \1 \to \mathcal C$
            and a map $\J \to \mathcal D$ out of a $\V$-interval fitting in to the commuting diagram below.
            \begin{equation}
                  \begin{tikzcd}
                        \1 \arrow[rr, dashed, "c"] \arrow[dr, "i_0"]
                        &&
                        \mathcal C \arrow[dd, "F"]
                        \\
                        &
                        \J \arrow[dr, dashed]
                        \\
                        \1 \arrow[ur, " i_1"] \arrow[rr,"b"]
                        &&
                        \mathcal D
                  \end{tikzcd}
            \end{equation}
      \end{enumerate}
\end{definition}

\begin{definition}
      We call a map $F: \O \to \P$ in $\mathsf{Op}^G(\V)$
      \begin{itemize}
      \item a {\em local fibration} (resp. {\em local weak equivalence}) if
            $F(\ksi): \O(\ksi)\to \P(F(\ksi))$
            is a fibration (resp. weak equivalence) in $\V^{\Stab(\ksi)}_{\F_\ksi}$ for all $\ksi\in \C(\O)^{\times n+1}$ and all $n$;
      \item a {\em local trivial fibration} if both a local fibration and a local weak equivalence;
      \item {\em essentially surjective} (resp. {\em path lifting}) if $j^*F^H$ is essentially surjective (resp. path lifting) in $\Cat(\V)$ for all $H\leq G$;
      \item a {\em fibration} if both path-lifting and a local fibration
      \item a {\em weak equivalence} if both essentially surjective and a local weak equivalence.
      \end{itemize}

      Moreover, foreshadowing, we call such a map
      a \textit{(trivial) cofibration} if it has the left lifting property against all trivial fibrations (resp. fibrations).
\end{definition}

\subsection{Generating (Trivial) Cofibrations and Local Fibrations}

We generalize and combine efforts from \cite{CM13b, BM13, Cav14}.

Fix a graph subgroup $\Gamma$ of $G \times \Sigma_n$, and $X \in \V^\Gamma$.
Define $C_\Gamma[X]$ to be the ``free operad with stabilizer $\Gamma$ generated by $X$''.
Specifically, this operad has colours $\mathfrak C_\Gamma := G \cdot_\Gamma \underline{n+1}$.
Now, letting $\ksi_0$ denote the signature $([e,1],[e,2],\dots,[e,n];[e,0])$,
we define
\begin{equation}
      C_\Gamma[X](\ksi) =
      \begin{cases}
            (g,\sigma)^{\**} X \qquad \qquad & \ksi = (g,\sigma).\ksi_0
            \\
            \varnothing & \mbox{otherwise,}
      \end{cases}
\end{equation}
where $g \in G$ and $\sigma \in \Sigma_n$ are chosen to be the \textit{minimal} elements in those groups with this property,
and $\varnothing$ is the initial object in $\V$.

It is straightforward that the operad $\mathfrak C_\Gamma[X]$ has the universal property
\begin{equation}
      \Hom_{\Op^G(\V)}(C_\Gamma[X], \O) = \mathop\prod\limits_{\zeta \in (\mathfrak C(\O)^{\times n+1})^\Gamma}\Hom_{\V^\Gamma}(X, \O(\zeta)).
\end{equation}

Define $I_{loc}$ to be the set $\set{C_{\Gamma}[i_\gamma]}$ which runs over all
graph subgroups $\Gamma$ of $G \times \Sigma_n$ and
generating cofibrations $i_\gamma$ in $\V^\Gamma_{\F_n}$;
similarly let $J_{loc}$ denote the set $\set{C_{\Gamma}[j_\gamma]}$
with $j_\gamma$ the generating trivial cofibrations.

The universal property makes the following immediate.
\begin{corollary}[{cf. \cite[\S 4.2]{Cav14}, \cite[1.16]{CM13b}}]
      $\O\to \P$ is a local (trivial) fibration {\sc iff}
      $\O\to \P$ has the right lifting property against $J_{loc}$ (resp. $I_{loc}$).
\end{corollary}

Now, define $I_{G}:= I_{loc} \mathbin{\cup} \set{\varnothing \to G/H \cdot \1}_{H\leq G}$
and
$J_{G} := J_{loc} \mathbin{\cup} \set{G/H \cdot (\1 \to \J)}_{H\leq G,\ \J\in\mathbb{G}}$
where again $\1$ is the initial $\V$-category (thought as an operad), and $\mathbb{G}$ is a generating set of $\V$-intervals. 

\begin{lemma}
      [{cf. \cite[4.8]{Cav14}, \cite[2.3]{BM13}, \cite[1.18]{CM13b}}]
      \label{CAV_4.8}
      A map $F$ in $\mathsf{Op}^G(\V)$ is a trivial fibration {\sc iff} $F$ is a local trivial fibration such that $F^H$ is surjective on $H$-fixed colors for all $H\leq G$ {\sc iff} $F$ has the right lifting property against $I_{G}$.. 
\end{lemma}
\begin{proof}
      By definition, $F$ is a trivial fibration {\sc iff}
      it is a local trivial fibration such that $j^*F^H$ is path-lifting and essentially surjective for all $H\leq G$.
      Thus, \cite[4.8]{Cav14} immediately implies the first step.
      Moreover, right lifting against $I_{loc}$ is identical to being a local trivial fibration, while
      lifting against $\varnothing \to G/H\otimes \1$ precisely say that $F^H$ is surjective on colors;
      combining these observations yields the result.
\end{proof}

\begin{lemma}
      [{cf. \cite[1.20]{CM13b}, \cite[\S 4.3]{Cav14}}]
      $F$ has right lifting against $J_{G}$ {\sc iff} $F$ is a fibration.
\end{lemma}
\begin{proof}
      Again, lifting against $J_{loc}$ is identical to being a local fibration, while lifting against $G/H \cdot (\1 \to \J)$
      is equivalent to $F^H$ lifting against $\1 \to \J$, which is true exactly when it is path lifting by \cite{Cav14}. 
\end{proof}

\begin{lemma}
      [{cf. \cite[1.19]{CM13b}}]
      \label{POINT_4_LEMMA}
      $J_{G}\mbox{-cof} \subseteq I_{G}\mbox{-cof}$; that is, trivial cofibrations are cofibrations.
\end{lemma}
\begin{proof}
      It suffices to show that if $F$ has (right) lifting against $I_{G}$, it has lifting aginst $J_{G}$.
      Obviously, a local trivial fibration is a local fibration.
      On the other hand, by locality, any cofibration in $\mathsf{Op}^{G, \mathfrak C}(\V)$ for any $G$-set $\C$
      is a cofibration when considered in $\mathsf{Op}^G(\V)$.
      Hence, since $G/H \cdot (\1 \to \1 \amalg \1)$ is in $I_{G}\mbox{-cof}$, the composite
      \begin{equation}
            \begin{tikzcd}
                  G/H \cdot \1 \arrow[r, rightarrowtail]
                  &
                  G/H \cdot (\1 \amalg \1) \arrow[r, rightarrowtail]
                  &
                  G/H \cdot \J 
            \end{tikzcd}
      \end{equation}
      is in $I_{G}\mbox{-cof}$.
      Thus $J_G \subseteq I_G\mbox{-cof}$, implies the result.
\end{proof}

\subsection{Trivial cofibrations are weak equivalences}

\begin{lemma}
      The transfinite composition of essentially surjective maps is essentially surjective.
\end{lemma}
\begin{proof}
      Since taking fixed points commutes with filtered colimits, they commute with transfinite composition,
      and hence by \cite[4.17]{Cav14}, we are done.
\end{proof}

\begin{lemma}
      \label{J-CELL_LEMMA}
      [{c.f. \cite[4.20]{Cav14}}]
      If weak equivalences are closed under transfinite compositions, then relative $J_{G}$-cells are weak equivalences.
      \todo[inline]{may need cofibrant unit. take another look}
\end{lemma}
\begin{proof}
      Since local weak equivalences are closed under transfinite composition by assumption, and
      essentially surjective maps are closed under transfinite composition by the above lemma,
      it suffices to prove that the pushout of a map $j\in J_{G}$ is a weak equivalence.
      If $j\in J_{loc}$, then since colimits in $\mathsf{Op}^G(\V)$ are computed in $\Op(\V)$,
      and since by \cite{Cav14} pushouts of this form can be computed fiberwise
      (that is, after shifting the operads into a single color),
      we are computing the pushout of a trivial cofibration in $\mathsf{Op}^{G,\C_{\ksi}}(\V)$,
      where $\C_{\ksi}$ is the $G$-set generated by the colors in the given signature $\ksi$.
      By the existance of the transferred model structure, this is again a trivial cofibration.
      Hence, the pushout is a local weak equivalence in $\mathsf{Op}^G(\V)$ which is the identity on colors,
      and hence a weak equivalence itself.
      
      Now, supppose $j$ is the map $G/H \cdot (\1 \to \J)$ for $\J$ a $\V$-interval.
      As in \cite{Cav14}, we can split this pushout into a composition of two pushouts
      \begin{equation}
            \begin{tikzcd}
                  G/H \cdot \1 \arrow[r] \arrow[d, "G/H \cdot \phi"']
                  % \arrow[dr,phantom, yshift=.1em, xshift=.5em, "\lrcorner" near end]
                  &
                  X \arrow[d,"\phi'"]
                  \\
                  G/H \cdot \J_{\set{0,0}} \arrow[r] \arrow[d, "G/H \cdot \psi"']
                  % \arrow[dr,phantom, yshift=.1em, xshift=.5em, "\lrcorner" near end]
                  &
                  X' \arrow[d,"\psi'"]
                  \\
                  G/H \cdot \J \arrow[r]
                  &
                  Y
            \end{tikzcd}
      \end{equation}
      where $\J_{\set{0,0}}$ is the full subcategory of $\J$ spanned by the object $O$.
      It suffices to show both $\psi'$ and $\phi'$ are local weak equivalences which are essentially surjective on fixed points. 
      
      Now, we know from \cite{Cav14} that $\psi$ is injective on colors and fully-faithful
      (that is, induces an isomorphism in $\Op^{\set{0}}(\V)$),
      and hence $G/H \cdot \psi$ is also injective on colors and fully-faithful as a map in $\Op(\V)$.
      Since colimits are created non-equivariantly, by \cite[Prop B.22]{Cav14} and the remark thereafter,
      we conclude that $\psi'$ is fully faithful in $\Op(\V)$, and hence is an isomorphism in $\Op^{\C(X')}(\V)$.
      But $\psi'$ is a $G$-map, and hence it is an isomorphism in $\mathsf{Op}^{G, \C(X')}(\V)$ as well,
      and hence is a local weak equivalence in $\mathsf{Op}^G(\V)$. 
      
      Moreover, we observe that $\C(Y) = \C(X') \amalg (G/H \times \set{1})$.
      Thus, if $x \in \C(Y)^K$ for $K \leq G$ is in $\C(X')$, we have essential surjectivity trivially:
      \begin{equation}
            \begin{tikzcd}
                  \1 \arrow[rrr, "x"] \arrow[dr, " i_0"]
                  &&&
                  (X')^K \arrow[dd, "\psi'"]
                  \\
                  &
                  \J \arrow[r]
                  &
                  \1 \arrow[dr, "x"]
                  \\
                  \1 \arrow[ur, " i_1"] \arrow[rrr,"x"]
                  &&&
                  (Y)^K
            \end{tikzcd}
      \end{equation}
      Lastly, if we consider any orbit of the new object $1\in \C(Y)$,
      there is an associated object $0 \in \C(X')$ such that the essentially surjectivity diagram
      is the same as the pushout diagram for $\psi$:
      \begin{equation}
            \begin{tikzcd}
                  G/H \cdot \1 \arrow[rr,"0"] \arrow[dr, "G/H \cdot i_0"]
                  &&
                  X' \arrow[dd, "\psi'"]
                  \\
                  &
                  G/H \cdot \J \arrow[dr, "\mbox{\large $\lrcorner$}" near end]
                  \\
                  G/H \cdot \1 \arrow[ur, "G/H\otimes i_1"] \arrow[rr, "1"]
                  &&
                  Y
            \end{tikzcd}
      \end{equation}
      Hence $\psi'$ is essentially surjective and a local weak equivalence, hence a weak equivalence in $\mathsf{Op}^G(\V)$. 

      Similarly, when considering $\phi'$, \cite[4.20]{Cav14} again says that pushouts of this form are created in $\Op(\V)_{\C(X)}$ as the pushout
      \begin{equation}
            \begin{tikzcd}
                  p_! (G/H \cdot \1) \arrow[r, "p"] \arrow[d, "p_! (G/H \cdot \phi)"']
                  &
                  X \arrow[d,"\phi'"]
                  \\
                  p_! (G/H \cdot \J_{\set{0,0}}) \arrow[r]
                  &
                  Y
            \end{tikzcd}
      \end{equation}
      In particular, this implies $\phi'$ is bijective on objects, and hence essentially surjective.
      Moreover, as $\phi$ is a trivial cofibration in $\Op(\V)$ by \cite{Cav14},
      $p_! (G/H \cdot \phi)$ is a trivial cofibration in $\Op^{G,\C(X)}(\V)$.
      Thus $\phi'$ is a trivial cofibration in $\mathsf{Op}^{G,\C(X)}(\V)$,
      and thus is a local weak equivalence in $\Op^G(\V)$.

      Hence both $\phi'$ and $\psi'$ are weak equivalences in $\mathsf{Op}^G(\V)$, so the result is proved.
\end{proof}


\subsection{2-out-of-3}

\begin{definition}
      If $\V$ has intervals, then in any $\V$-category $\mathcal C$,
      we say that two arrows  $f,g: \1 \to \mathcal C(x,y)$ are \textit{homotopic}
      if there exists a factorization of the form below, with $\mathbb J$ a $\V$-interval.
      \begin{equation}
            \begin{tikzcd}
                  1_\V \amalg 1_\V \arrow[rr, "{(f, g)}"] \arrow[dr, "{(id_0,id_0)}"']
                  &&
                  \mathcal C(x,y)
                  \\
                  &
                  \mathbb J(0,0) \arrow[ur, dashed]
            \end{tikzcd}
      \end{equation}
\end{definition}

We recall some equivalence relations on objects in a $\V$-category \cite{Cav14, BM13}:
\begin{definition}
      Given $\mathcal{C}$ in  $\Cat(\V)$ and $a,b\in\mathrm{Ob}(\mathcal C)$, we say $a$ and $b$ are
      \begin{itemize}
      \item {\em equivalent} if there exists a map $\gamma: \J \to \mathcal C$ such that
            $\gamma i_0 = a$, $\gamma i_1 = b$
            for some $\V$-interval $\J$;
      \item {\em virtually equivalent} if $a$ and $b$ are equivalent in some fibrant replacement
            $\mathcal C_f$ of $\mathcal C$ in $\Cat^{\mathrm{Ob}(\mathcal C)}(\V)$;
      \item {\em homotopy equivalent} if $a$ and $b$ are isomorphic in the unenriched category $\pi_0 \mathcal C_f$
            for some $\mathcal C_f$,
            where $\pi_0: \Cat(\V) \to \Cat$ doesn't change the objects, but has $\pi_0\mathcal C(x,y) = \Ho(\V)(1_\V,\C(x,y))$
            (equivalently, there exist maps
            $\alpha: 1_\V \to \mathcal C_f(a,b)$ and $\beta: 1_\V\to \mathcal C_f(b,a)$ such that
            $\beta\alpha$ and $\alpha\beta$ and homotopic\footnote{
              Recall that the factorizations in a model category give a notion of \textit{(left) homotopy} $\sim$,
              with $\Ho(\V)(X,Y) = \V(QX,RY)/\sim$.}
            to the identity arrows
            $1_V\to \mathcal C_f(a,a)$ and $1_V \to \mathcal C_f(b,b)$, respectively.)
      \end{itemize}
\end{definition}

Equivariantly, we have the following:
\begin{definition}
      Given $\mathcal{C}\in \Cat^G(\V)$ and $a,b\in \mathrm{Ob}(\mathcal{C})$, we say $a$ and $b$ are
      \begin{itemize}
      \item {\em equivalent} if $\Stab_G(a) = \Stab_G(b) =: H$ and are equivalent in $\mathcal{C}^H$;
      \item {\em virtually equivalent} if they are equivalent in some fibrant replacement
            $\mathcal{C}_f$ of $\mathcal{C}$ in $\Cat^{G, \mathrm{Ob}(\mathcal C)}(\V)$;
      \item {\em homotopy equivalent} if $\Stab_G(a) = \Stab_G(b) =: H$ and they are homotopy equivalent in $\mathcal{C}^H$. 
      \end{itemize}
      For an operad $\O\in \mathsf{Op}^G(\V)$ and $a,b\in \C(\O)$, we say $a$ and $b$ are
      {\em equivalent} (resp. {\em virtually equivalent}, {\em homotopy equivalent}) if they are so in $j^*\O$. 
\end{definition}

The following three lemmas follow directly from the proofs of their non-equivariant counterparts:
\begin{lemma}
      [{cf. \cite[4.10]{Cav14}}]
      Equivalence and virtual equivalence define equivalence relations on $\C(\O)$. \qed
\end{lemma}
\begin{lemma}
      [{cf. \cite[4.13]{Cav14}, \cite[2.11]{BM13}}]
      Virtually equivalent colors are homotopy equivalent. 
\end{lemma}
\begin{lemma}
      [{cf. \cite[4.12]{Cav14}, \cite[2.10]{BM13}}]
      \label{RIGHTPROPER_LEM}
      If $\V$ is right proper, then all virtual equivalent colors are equivalent. 
\end{lemma}

\begin{lemma}
      [{cf. \cite[4.11]{Cav14}, \cite[2.9]{BM13}}]
      Any local weak equivalence $F: \O\to \P$ in $\mathsf{Op}^G(\V)$ reflects virtual weak equivalences.
\end{lemma}
\begin{proof}
      As in the non-equivariant case, $F$ being a local weak equivalence implies
      we have a local trivial fibration $F': \O_f\to \P_f$.
      Thus, for $\Stab(a) = \Stab(b) =: H$, any virtual equivalence $\J \to \P_f^H$ between colors
      $F'(a) = F(a)$ and $F'(b) = F(b)$
      lifts to one $\J \to \O_f^H$ between $a$ and $b$
      (in particular,
      % since the fibration has source $\O_f^H$,
      the source colors $a$ and $b$ have stabilizer at least $H$, and since their images have stabilizer exactly $H$, so do they). 
\end{proof}

\subsubsection{Equivalences between levels}

We would like to generalize \cite[4.14 and 4.15]{Cav14}, which state that
$\O(\ksi)$ and $\O(\ksi')$ are equivalent in $\V$ if $\ksi$ and $\ksi'$ are related by a string of weak equivalences of colors,
as this would imply that weak equivalences satisfy the 2-out-of-3 property.
However, the relevant notion of weak equivalence on colours for this paper lives in $\Vsigma$ as opposed to $\V$,
and moreover the colors can be interchanged via the action of $G$.
Thus, we will need to be flexible and change an entire orbit worth of colors in order to create the desired homotopy equivalence. 

\begin{proposition}
      [{c.f. \cite[4.14]{Cav14}}]
      \label{CAV_4.14_PROP}
      Given $\O \in \Op^G(\V)$ with colors $\C$, $\ksi = (c_1,\dots,c_n;c)$ a signature in $\C$, and $K = \Stab(c)$.
      Moreover, suppose that $c_1$ and $d_1$ are homotopy equivalent.
      Then there exists a zig-zag of weak equivalences in $\V^{\Stab(\ksi)}_{\F_\ksi}$ between
      $\O(\ksi)$ and $\O(\theta)$, where
      $\theta = (d_1,\ldots, d_n; c)$, with the colors $d_i$ defined as follows:
      
      Let $\lambda \subseteq \underline{n} = \set{1,2,\ldots, n}$ denote
      the set of all $i$ such that $c_i = k_i \cdot c_1$ for some $k_i\in K$;
      if $i \notin \lambda$, let $k_i$ denote the identity element of $G$.
      Further, for all $i \in \underline{n}$, define
      \begin{equation}
            \label{DCOLORS_EQ}
            d_i =
            \begin{cases}
                  k_i \cdot d_1 \qquad \qquad & i \in \lambda
                  \\
                  c_i & \mbox{otherwise.}
            \end{cases}
      \end{equation}

      Moreover, any functor $F: \O \to \P$ induces a functorial zig-zag of weak equivalences between
      $\P(F(\ksi)$ and $\P(F(\theta)$.
\end{proposition}
\begin{proof}
      Without loss of generality, we may assume $\O$ is fibrant,
      as the fibrant replacement weak equivalences can always be added to any zig-zag.

      Denote the stabilizer of $c_1$ (and hence $d_1$) by $H$, and so we have maps
      $\alpha: 1_\V \to \O^H(c_1,d_1)$ and $\beta: 1_\V \to \O^H(d_1,c_1)$
      realizing their homotopy equivalence.
      For each $i \in \underline{n}$, define
      \begin{equation}
            \label{WEAKEQCOLORS_EQ}
            H_i =
            \begin{cases}
                  k_i H k_i^{-1} \qquad & i \in \lambda
                  \\
                  H & \mbox{else,}
            \end{cases}
            \qquad
            \qquad 
            \alpha_i =
            \begin{cases}
                  1_\V \xrightarrow{\alpha} \O^H_f(c_1;d_1) \xrightarrow{k_i} \O^H_f(c_i;d_i) \qquad & i \in \lambda
                  \\
                  1_\V \xrightarrow{id} \O^H_f(c_i;d_i) & \mbox{else},
            \end{cases}
      \end{equation}
      and $\beta_i$ similarly.
      Note that all of these --- $d_i$, $H_i$, $\alpha_i$, and $\beta_i$ --- are independent of the choice of $k_i\in k_i H$.
      Further, $\Stab(c_i) = \Stab(d_i)$, and
      the pair $(\alpha_i,\beta_i)$ realizes a homotopy equivalence between $c_i$ and $d_i$.

      We firstly claim that $\Stab_{G \times \Sigma_n}(\ksi) = \Stab_{G \times \Sigma_n}(\theta)$.
      To that end, suppose $(k,\pi)\in \Stab_{G\times \Sigma_n}(\ksi)$, so that $k \cdot c_{\pi^{-1}(i)} = c_i$ for all $i$.
      % We need to show $k d_{\pi^{-1}(i)} = d_i$.
      Thus $\pi$ must act on $\lambda$ and $\underline{n} \setminus \lambda$ independently,
      so for $i \in \lambda$ we have $k \cdot c_{\pi^{-1}(i)} = c_i$, or
      $k k_{\pi^{-1}(i)} \cdot c_1 = k_i \cdot c_1$, and so
      $k_i^{-1} k k_{\pi^{-1}(i)} =:h_i \in H$. Hence 
      \begin{equation}
            k \cdot d_{\pi^{-1}(i)} = k k_{\pi^{-1}(i)} \cdot d_1 = k_i h_r \cdot d_1 = k_i \cdot d_1 = d_i,
      \end{equation}
      as desired.
      On the other hand, if $i \not \in \lambda$, then
      $k \cdot d_{\pi^{-1}(i)} = k \cdot c_{\pi^{-1}(i)} = c_i = d_i$.
      The reverse direction is analogous.
      
      Now, let $\otimes \hat \beta_i$ and $\otimes \hat \alpha_i$ be the composites
      \begin{equation}
            \begin{tikzcd}[row sep = tiny]
                  \otimes \hat \beta_i :
                  % \O(c_1,\dots,c_n;c)
                  % =
                  \O(\ksi)
                  \cong
                  \O(\ksi) \otimes 1_\V^{\otimes n} \arrow[rr, "1 \otimes\ \bigotimes_i \beta_i"]
                  &&
                  \O(\ksi) \otimes \bigotimes_i \O^{H_i}(d_i;c_i) \arrow[r,"\circ"]
                  &
                  % \O(d_1,\dots,d_n;c)
                  % =
                  \O(\theta)
                  \\
                  \otimes \hat \alpha_i : 
                  % \O(d_1,\dots,d_n;c)
                  % =
                  \O(\theta)
                  \cong
                  \O(\theta) \otimes 1_\V^{\otimes n} \arrow[rr, "1 \otimes\ \bigotimes_i \alpha_i"]
                  &&
                  \O(\theta) \otimes \bigotimes_i \O^{H_i}(c_i;d_i) \arrow[r,"\circ"]
                  &
                  % \O(c_1,\dots,c_n;c)
                  % =
                  \O(\ksi).
            \end{tikzcd}
      \end{equation}

      We secondly claim that the maps $\otimes \hat \alpha_i$ and $\otimes \hat \beta_i$ descend to $\Lambda$ fixed points 
      for any subgroup $\Lambda \leq \Stab(\ksi) = \Stab(\theta)$.
      Indeed, since the composition structure maps of $\O$ are natural in $G$ and $\Sigma$,
      it suffices to show that $\otimes \hat \beta_i$ is preserved by $(k,\pi)\in Stab(\ksi)$.
      But we observe this directly:
      \begin{equation}
            (k, \pi) . (\otimes \hat \beta_i)
            =
            \otimes k \hat\beta_{\pi^{-1}(i)}
            =
            \otimes k k_{\pi^{-1}i} \hat \beta
            =
            \otimes k_i h_i \hat \beta
            =
            \otimes k_i \hat \beta
            =
            \otimes \hat \beta_i.
      \end{equation}
      The result for $\otimes \hat \alpha_i$ is analogous.
      
      Since all $\alpha_i$ and $\beta_i$ are homotopy equivalences,
      this second claim implies that the composites $\otimes \hat \beta_i$ and $\otimes \hat \alpha_i$
      induce isomorphisms in the homotopy category of $\V^{\Stab(\ksi)}_{\F_\ksi}$,
      and hence these are weak equivalences between $\O(\ksi)$ and $\O(\theta)$ in $\V^{\Stab(\ksi)}_{\F_\ksi}$,
      as desired.

      The moreover follows exactly as in \cite{Cav14}.
\end{proof}

\begin{remark}
      \label{CAV_4.14_REM}
      The above proof also works to show an analogous result when given colors $c_j$ and $d_j$ that are homotopy equivalent.
      Furthermore, if we are given $c$ and $d$ homotopy equivalent, we may just take $\theta = (c_1,\dots, c_n;d)$,
      and the same result holds (as $K\times\Sigma_n$ acts trivially on $\O^K(d;c)$).
\end{remark}


% To that end, fix $\O_f$ fibrant in $\mathsf{Op}^G(\V)$ with colors $\C$.
% Supppose $c_1$ and $d_1$ are homotopy equivalent in $\C$, each with stabilizer $H$,

% Further, let $\ksi = (c_1,c_2,\dots, c_n;c)$ be a signature in $\C(\O_f)$, with $K := \Stab(c)$.

% We define $\lambda \subseteq \underline{n} = \set{1,2,\ldots, n}$ to be
% the set of all $i$ such that $c_i = k_i \cdot c_1$ for some $k_i\in K$;
% if $i \notin \lambda$, let $k_i$ denote the identity element of $G$.
% Moreover, for all $i \in \underline{n}$, define
% \begin{equation}
%       \label{WEAKEQCOLORS_EQ}
%       H_i =
%       \begin{cases}
%             k_i H k_i^{-1} \qquad & i \in \lambda
%             \\
%             H & \mbox{else,}
%       \end{cases}
%       \qquad
%       d_i =
%       \begin{cases}
%             k_i \cdot d_1 \qquad & i \in \lambda
%             \\
%             c_i & \mbox{else},
%       \end{cases}
%       \qquad 
%       \alpha_i =
%       \begin{cases}
%             1_\V \xrightarrow{\alpha} \O^H_f(c_1;d_1) \xrightarrow{k_i} \O^H_f(c_i;d_i) \qquad & i \in \lambda
%             \\
%             1_\V \xrightarrow{id} \O^H_f(c_i;d_i) & \mbox{else},
%       \end{cases}
% \end{equation}
% and $\beta_i$ similarly.
% Note that each of these is independent of the choice of $k_i\in k_i H$.
% Further, $\Stab(c_i) = \Stab(d_i)$, and
% the pair $(\alpha_i,\beta_i)$ realizes a homotopy equivalence between $c_i$ and $d_i$.
% %as post-composition by an isomorphism remains an isomorphism in $\mathrm{Ho}\V$.
% We denote the new signature $\theta = (d_1,\cdots,d_n;c)$.

% \begin{lemma}
%       With the above notation, $\Stab_{G \times \Sigma_n}(\ksi) = \Stab_{G \times \Sigma_n}(\theta)$. 
% \end{lemma}
% \begin{proof}
%       Suppose $(k,\pi)\in \Stab_{G\times \Sigma_n}(\ksi)$, so that $k \cdot c_{\pi^{-1}(i)} = c_i$ for all $i$.
%       % We need to show $k d_{\pi^{-1}(i)} = d_i$.
%       Thus $\pi$ must act on $R$ and $\underline{n} \setminus R$ independently,
%       so $i \in \lambda$ we have $k \cdot c_{\pi^{-1}(i)} = c_i$, or
%       $k k_{\pi^{-1}(i)} \cdot c_1 = k_i \cdot c_1$, and so
%       $k_i^{-1} k k_{\pi^{-1}(i)} =:h_i \in H$. Hence 
%       \begin{equation}
%             k \cdot d_{\pi^{-1}(i)} = k k_{\pi^{-1}(i)} \cdot d_1 = k_i h_r \cdot d_1 = k_i \cdot d_1 = d_i,
%       \end{equation}
%       as desired.
%       On the other hand, if $i \not \in \lambda$, then
%       $k \cdot d_{\pi^{-1}(i)} = k \cdot c_{\pi^{-1}(i)} = c_i = d_i$.

%       The converse is similar.
% \end{proof}

% With the same notation as above, let $\otimes \hat \beta_i$ and $\otimes \hat \alpha_i$ be the composites
% \begin{equation}
%       \begin{tikzcd}[row sep = tiny]
%             \otimes \hat \beta_i :
%             % \O_f(c_1,\dots,c_n;c)
%             % =
%             \O_f(\ksi)
%             \cong
%             \O_f(\ksi) \otimes 1_\V^{\otimes n} \arrow[rr, "1 \otimes\ \bigotimes_i \beta_i"]
%             &&
%             \O_f(\ksi) \otimes \bigotimes_i \O_f^{H_i}(d_i;c_i) \arrow[r,"\circ"]
%             &
%             % \O_f(d_1,\dots,d_n;c)
%             % =
%             \O_f(\theta)
%             \\
%             \otimes \hat \alpha_i : 
%             % \O_f(d_1,\dots,d_n;c)
%             % =
%             \O_f(\theta)
%             \cong
%             \O_f(\theta) \otimes 1_\V^{\otimes n} \arrow[rr, "1 \otimes\ \bigotimes_i \alpha_i"]
%             &&
%             \O_f(\theta) \otimes \bigotimes_i \O_f^{H_i}(c_i;d_i) \arrow[r,"\circ"]
%             &
%             % \O_f(c_1,\dots,c_n;c)
%             % =
%             \O_f(\ksi).
%       \end{tikzcd}
% \end{equation}

% \begin{lemma}
%       The map $\otimes \hat \beta_i$ descends to $\Gamma$ fixed points for any subgroup $\Lambda\leq \Stab(\ksi) = \Stab(\theta)$.
% \end{lemma}
% \begin{proof}
%       Since the composition structure maps of $\O_f$ are natural in $G$ and $\Sigma$,
%       it suffices to show that $\otimes \hat \beta_i$ is preserved by \textit{all} $(k,\pi)\in Stab(\ksi)$.
%       But we observe this directly:
%       \begin{equation}
%             (k, \pi) . (\otimes \hat \beta_i)
%             =
%             \otimes k \hat\beta_{\pi^{-1}(i)}
%             =
%             \otimes k k_{\pi^{-1}i} \hat \beta
%             =
%             \otimes k_i h_i \hat \beta
%             =
%             \otimes k_i \hat \beta
%             =
%             \otimes \hat \beta_i.
%       \end{equation}
% \end{proof}

% Since all $\alpha_i$ and $\beta_i$ are homotopy equivalences,
% the above lemma implies that the composites $\otimes \hat \beta_i$ and $\otimes \hat \alpha_i$
% induce an isomorphism in the homotopy category of $\V^{\Stab(\ksi)}_{\F_\ksi}$,
% and hence these are weak equivalences between $\O_f(\ksi)$ and $\O_f(\theta)$ in $\V^{\Stab(\ksi)}_{\F_\ksi}$.

% This discussion yields the following.
% \begin{corollary}
%       [{c.f. \cite[4.14]{Cav14}}]
%       \label{CAV_4.14_PROP}
%       Let $\O$ be any operad in $\mathsf{Op}^G(\V)$,
%       $\ksi = (c_1,\ldots,c_n;c)$ a signature in $\C(\O)$,
%       $K = \Stab_G(c)$,
%       and suppose either $(c_j,d_j)$ or $(c,d)$ is a pair of homotopy equivalent colors.
%       Then there exists a zig-zag of weak equivalences in $\V^{\Stab(\ksi)}_{\F_\ksi}$ between $\O(\ksi)$ and $\O(\theta)$,
%       where $\theta = (d_1,\dots,d_n;c)$ or $(c_1,\dots,c_n;d)$
%       with $d_i$ defined as in \eqref{WEAKEQCOLORS_EQ}.
%       % where $\ksi' = (d_1,\ldots, d_n;d)$ with
%       % \begin{itemize}
%       % \item $R\subseteq \set{1,\ldots,n}$ those $r$ such that $c_r\in Kc_i$; in particular, choose $k_r\in K$ with $k_r c_i = c_r$ (if $i\not\in R$, let $k_i = 1$); and
%       % \item $d_i = k_i c_i$.
%       % \end{itemize}
%       Moreover, any functor $F:\O\to \P$ induces a functorial zig-zag of weak equivalences between $\P(F(\ksi))$ and $\P(F(\ksi'))$.
% \end{corollary}
% \begin{proof}
%       In the first case, without loss of generality we may take $j = 1$,
%       and then the result follows by the above discussion.
%       Similarly, if the homotopy equivalence between $c$ and $d$ is realized by the pair of maps $\alpha_0$ and $\beta_0$,
%       then the analogous composites $\hat \alpha_0$ and $\hat \beta_0$ descend to all fixed points
%       (as $K\times\ksi_n$ acts trivially on $\O_f^K(d;c)$),
%       and hence we again have a weak equivalence
%       $\O_f(c_1,\ldots, c_n;c) \to \O_f(c_1,\ldots, c_n;d)$
%       in $\V^{\Stab{\ksi}}_{\F_\ksi}$.
      
%       Lastly, the fibrant replacement weak equivalences $\O(\ksi)\to\O_f(\ksi)$ and $\O(\ksi')\to \O_f(\ksi')$ complete the zig-zag.
 
%      The second statement follows identically as in the non-equivariant case found in \cite[4.14]{Cav14}.
% \end{proof}
 
\begin{proposition}
      [{c.f. \cite[4.15]{Cav14}}]
      \label{CAV_4.15_PROP}
      The class of weak equivalences in $\mathsf{Op}^G(\V)$ satisfies the 2-out-of-3 condition.
\end{proposition}
\begin{proof}
      Essential surjectiving holds in all cases since it reduces to checking multiple instances of the non-equivariant case,
      where it holds via \cite[4.15]{Cav14}.
      Now let $\O \xrightarrow{F} \P \xrightarrow{L} \Q$ be a composition of maps in $\mathsf{Op}^G(\V)$.
      If $F$ and $L$ are weak equivalences,
      the composite is obviously a local weak equivalence:
      $\O(\ksi)^\Gamma \simeq \P(F(\ksi))^\Gamma \simeq \Q(LF(\ksi))^\Gamma$.
      If $L$ and $FL$ are weak equivalences,
      then $F$ is by 2-out-of-3 in each $\V^{\Stab(\ksi)}_{\F_\ksi}$.

      Lastly, suppose $F$ and $LF$ are weak equivalences.
      Given a signature $\theta = (d_1,\ldots,d_n;d$) in $\C(\P)$, let $K = \Stab(d)$,
      Now, let $\Lambda = \lambda_1 \amalg \dots \amalg \lambda_r$ denote the partition of $\underline{n}$
      where $i < j$ are in the same class iff there exists $k_{i,j} \in K$ such that $d_j = k_{i,j} \cdot d_i$.
      Define $R \subseteq \underline{n}$ to be the subset of minimal representatives in each class,
      and $H_r$ the stabilizer in $G$ of $c_r$.

      By the essential surjectivity of $F$, there exist $c_r \in \C(\O)$ such that
      $\Stab(c_r) = \Stab(d_r)$ and $F(c_r)$ is equivalent, and hence homotopy equivalent, to $d_r$.
      Similarly, there exists $c\in \C(\O)$ such that $\Stab(c) = \Stab(d)$ with $F(c)$ and $d$ homotopy equivalent. 

      Now, we extend the set $\set{c_r}_{r\in R}$ to a signature $(c_1,\ldots, c_n;c)$ by defining $c_j = k_{r,j} \cdot c_r$
      (again, these are independent of the choice of $k_{r,j} \in k_{r,j}H_r$).

      Consequently, $F(c_i)$ is homotopy equivalent to $d_i$ via $k_{r,i}\gamma_r$,
      where $\gamma_r$ realizes the homotopy equivalence between $F(c_r)$ and $d_r$ for $i \in \lambda_r$.
      % In particular, these homotopy equivalences are coherent, as in the proof of \ref{CAV_4.14_PROP}.
      We have a diagram of the form
      \begin{equation}
            \label{TWOOFTHREE_EQ}
            \begin{tikzcd}
                  \O(c_1,\ldots, c_n;c) \arrow[r, "(1)"]
                  &
                  \P(F(c_1),\ldots, F(c_n); F(c)) \arrow[d,dash, "(3)"] \arrow[r, "(2)"]
                  &
                  \Q(LF(c_1),\ldots, LF(c_n);LF(c)) \arrow[d, dash, "(4)"]
                  \\
                  &
                  \P(d_1,\ldots, d_n;d) \arrow[r, "(5)"]
                  &
                  \Q(L(d_1),\ldots, L(d_n); L(d)).
            \end{tikzcd}
      \end{equation}
      $(1)$ is a weak equivalence in $\V^{\Stab(\ksi)}_{\F_\ksi}$ by assumption,
      $(2)$ is a weak-equivalence by 2-out-of-3 in $\V^{\Stab(\ksi)}_{\F_\ksi}$, and
      $(3)$ and $(4)$ are zig-zags of weak equivalences by iterating applications of \ref{CAV_4.14_PROP} or \ref{CAV_4.14_REM},
      as each application only changes the colours in a particular partition class.
      As these zig-zags are functorial, the above diagram commutes.
      Thus $(5)$ is a weak equivalence again by 2-out-of-3 in $\V^{\Stab(\ksi)}_{\F_\ksi}$, and hence
      $L$ is a local weak equivalence, as desired.
\end{proof}


\subsection{Model structure}

\begin{theorem}
      \label{MODEL_THEOREM}
      Suppose $\V$ is a cofibrantly generated monoidal model category such that
      \begin{enumerate}
      \item the model structure has cellular fixed-point functors,
            % which commute with a symmetric monoidal fibrant replacement functor in $\V$.
      \item STUFF NEEDED SO THAT THE COLOUR-FIXED MODEL STRUCTURE EXISTS,
      \item the unit is cofibrant
            \footnote{Needed for \ref{J-CELL_LEMMA} (I think).} \todo{where else?},
      \item the model structure is right proper
            \footnote{Needed for \ref{RIGHTPROPER_LEM}.},
      \item there exists a set $\mathbb{G}$ of generating $\V$-intervals
            \footnote{Needed for \ref{CAV_4.8}, others.}, and
      \item the class of weak equivalences in $\mathsf{Op}^G(\V)$ is closed under transfinite compositions.
      \end{enumerate}
      Then there exists a cofibrantly generated model structure on $\mathsf{Op}^G(\V)$ with
      fibrations, weak equivalences, generating cofibrations, and generating trivial cofibrations as described above.
\end{theorem}
\begin{proof}
      Since $\mathsf{Op}^G(\V)$ is complete and cocomplete, it suffices to prove (following \cite{Hov98} Theorem 2.1.19) that:
      \begin{enumerate}
      \item the class of weak equivalences has the 2-out-of-3 property and is closed under retracts;
      \item the domains of $I_{G}$ (resp. $J_{G}$) are small relative to $I_{G}$-cell (resp. $J_{G}$-cell);
      \item $I_{G}$-inj $= W\cap J_{G}$-inj;
      \item $J_{G}$-cell $\subseteq W\cap I_{G}$-cof.
      \end{enumerate}
      (1) follows from \ref{CAV_4.15_PROP} and the fact that if $L$ is a retract of $F$, $L^H$ is a retract of $F^H$.
      (2) follows since colimits in $\mathsf{Op}^G(\V)$ are created in $\Op(\V)$, and it holds non-equivariantly.
      (3) follows from \ref{CAV_4.8}.
      (4) follows from \ref{POINT_4_LEMMA} and \ref{J-CELL_LEMMA}.
\end{proof}





\bibliography{biblio}{}

\bibliographystyle{alpha}





\end{document}









