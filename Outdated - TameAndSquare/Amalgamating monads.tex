\documentclass[a4paper,10pt
%,draft
%, final
]{article}%

\pdfcompresslevel=0
\pdfobjcompresslevel=0

\usepackage[hidelinks]{hyperref}
\hypersetup{
  % colorlinks,
  final,
  pdftitle={Equivariant Dendroidal Segal Spaces},
  pdfauthor={Bonventre, P. and Pereira, L. A.},
  % pdfsubject={Your subject here},
  % pdfkeywords={keyword1, keyword2},
  linktoc=page
}
%\usepackage[open=false]{bookmark}

\input{commands.tex}%


%-------- Tikz ---------------------------

\usepackage{tikz}%
\usetikzlibrary{matrix,arrows,decorations.pathmorphing,
cd,patterns,calc}
\tikzset{%
  treenode/.style = {shape=rectangle, rounded corners,%
                     draw, align=center,%
                     top color=white, bottom color=blue!20},%
  root/.style     = {treenode, font=\Large, bottom color=red!30},%
  env/.style      = {treenode, font=\ttfamily\normalsize},%
  dummy/.style    = {circle,draw,inner sep=0pt,minimum size=2mm}%
}%

\usetikzlibrary[decorations.pathreplacing]



% ---- Commands on draft --------

\usepackage{ifdraft}
\ifdraft{
  \color[RGB]{63,63,63}
  % \pagecolor[rgb]{0.5,0.5,0.5}
  \pagecolor[RGB]{220,220,204}
  % \color[rgb]{1,1,1}
}


\usepackage[draft]{showkeys}
\usepackage{todonotes}%[obeyDraft]


% ----- Labels Changed? --------

\makeatletter

\def\@testdef #1#2#3{%
  \def\reserved@a{#3}\expandafter \ifx \csname #1@#2\endcsname
  \reserved@a  \else
  \typeout{^^Jlabel #2 changed:^^J%
    \meaning\reserved@a^^J%
    \expandafter\meaning\csname #1@#2\endcsname^^J}%
  \@tempswatrue \fi}

\makeatother


% ---- Commands --------

% new symbols

\newcommand{\mycircled}[2][none]{%
  \mathbin{
    \tikz[baseline=(a.base)]\node[draw,circle,inner sep=-1.5pt, outer sep=0pt,fill=#1](a){\ensuremath #2\strut};
cf.  }
}
\newcommand{\owr}{\mycircled{\wr}}

% replace symbols

\renewcommand{\hat}{\widehat}

% random

\renewcommand{\F}{\mathcal F}
\newcommand{\Q}{\mathcal Q}

\newcommand{\lltimes}{\underline{\ltimes}}


% detecting $\V$-categories:

\newcommand{\I}{\mathbb I}
\newcommand{\J}{\mathbb J}
\renewcommand{\1}{\eta}%{\ensuremath{\mathbb{id}}}

% lazy shortcuts

\newcommand{\SC}{\Sigma_{\mathfrak C}}
\newcommand{\OC}{\Omega_{\mathfrak C}}

\newcommand{\UV}{\underline{\mathcal V}}
\newcommand{\UC}{\underline{\mathfrak C}}

\newcommand{\Kl}{\underline{\mathsf{Kl}}}
\newcommand{\Alg}{\underline{\mathsf{Alg}}}


% ---- Title --------

\title{Equivariant Segal operads, simplicial operads, and dendroidal sets}

\author{Peter Bonventre, Lu\'is A. Pereira}%

\date{\today}


% ---- Document body --------

\begin{document}

\maketitle

\begin{abstract}
      Things and stuff
\end{abstract}

\tableofcontents






\section{Amalgamating monads}
\label{AMALGMON_SEC}

\subsection{Combinable monads}
\label{COMBMON_SEC}

We establish our setting for this section.

\begin{definition}\label{AMALGMON DEF}
	Let $P$ and $\bar{F}$ be two monads on the same category $\mathcal{C}$.
	We say \emph{$P$ and $\bar{F}$ can be combined} if there exists 
	a monad structure on the composite 
	$F = P \bar{F}$ such that
	\begin{enumerate}[label=(\roman*)]
		\item $P \eta_{\bar{F}} \colon P \Rightarrow P \bar{F} = F$
		is a map of monads
		\item $\eta_P \bar{F} \colon \bar{F} \Rightarrow P \bar{F} = F$
		is a map of monads
		\item the composite below is the identity
		\[
		\begin{tikzcd}[column sep=3em]
		P \bar{F} \ar[Rightarrow]{r}{P \eta_{\bar{F}} \eta_P \bar{F}}
		&
		P \bar{F} P \bar{F} \ar[equal]{r}
		&
		FF  \ar[Rightarrow]{r}{\mu_F}
		&
		F \ar[equal]{r}
		&
		P \bar{F}
		\end{tikzcd}
		\]
	\end{enumerate}
\end{definition}


\begin{remark}\label{SHORTCONV REM}
	To ease notation, in our diagrams we will follow the following conventions:
	\begin{itemize}
		\item by default, unlabeled maps
		% of the form
		$P \Rightarrow F$, $\bar{F} \Rightarrow F$ refer to the maps in
		Definition \ref{AMALGMON DEF}(i)(ii);
		\item we write $\mu$ to denote any of the monad multiplications
		$PP \Rightarrow P$,
		$\bar{F}\bar{F} \Rightarrow \bar{F}$,
		$FF \Rightarrow F$,
		as well as for the left/right action multiplications
		$PF \Rightarrow F$,
		$FP \Rightarrow F$,
		$\bar{F} F \Rightarrow F$,
		$F \bar{F} \Rightarrow F$
		induced by Definition \ref{AMALGMON DEF}(i)(ii)
		(recall that, by definition,
		$PF \Rightarrow F$ is the composite
		$PF \Rightarrow FF \overset{\mu}{\Rightarrow} F$,
		and similarly for the other action multiplications).
		
		Therefore, to avoid ambiguity, we always write the 
		source of a $\mu$ labeled map as a composite of two functors and the target as a single functor
		(for example, this means we write 
		$\mu\colon FF \Rightarrow F$ for the multiplication of $F$,
		but \emph{never} write this multiplication as $\mu \colon P \bar{F} P \bar{F} \Rightarrow P \bar{F}$);
		\item we write $\eta$ for either of the units 
		$I \Rightarrow P$,
		$I \Rightarrow \bar{F}$,
		$I \Rightarrow F$.
	\end{itemize}
	Note that, following these conventions, 
	Definition \ref{AMALGMON DEF}(iii)
	says that 
	$P\bar{F} \Rightarrow FF \overset{\mu}{\Rightarrow} F$
	is the identity.
\end{remark}


\begin{definition}\label{TWISTMAP DEF}
	Given $P,\bar{F},F$ as in Definition \ref{AMALGMON DEF}, 
	the twist map $\tau \colon \bar{F} P \Rightarrow P \bar{F}$ is the composite
	\[
	\begin{tikzcd}
	\bar{F} P \ar[Rightarrow]{r}{}
	&
	FF  \ar[Rightarrow]{r}{\mu}
	&
	F 
	\end{tikzcd}
	\]
	We note that in our diagrams $\tau$ is written as either
	$\tau \colon \bar{F} P \Rightarrow P \bar{F}$ or as
	$\tau \colon \bar{F} P \Rightarrow F$.
\end{definition}


\begin{lemma}\label{MODSTRCOMRE LEM}
	\begin{enumerate}[label=(\roman*)]
		\item the monad maps
		$P \Rightarrow F$ and 
		$\bar{F} \Rightarrow F$ coincide with the composites
		\[
		\begin{tikzcd}
		P \ar[Rightarrow]{r}{\eta P} &
		\bar{F} P \ar[Rightarrow]{r}{\tau} &
		F 
		&
		\bar{F} \ar[Rightarrow]{r}{\bar{F} \eta} &
		\bar{F} P \ar[Rightarrow]{r}{\tau} &
		F 
		\end{tikzcd}
		\]
		\item
		the multiplications $\mu \colon PF \Rightarrow F$ and
		$\mu \colon F\bar{F} \Rightarrow F$
		are also given by
		\[
		\begin{tikzcd}
		PP\bar{F} \ar[Rightarrow]{r}{\mu \bar{F}} &
		P \bar{F} 
		&
		P \bar{F} \bar{F} \ar[Rightarrow]{r}{P \mu} &
		P \bar{F} 
		\end{tikzcd}
		\]
		\item
		the multiplications
		$\mu \colon F P \Rightarrow F$ and
		$\mu \colon \bar{F} F \Rightarrow F$ are also given by the composites
		\[
		\begin{tikzcd}
		P\bar{F} P \ar[Rightarrow]{r}{P \tau} &
		P F \ar[Rightarrow]{r}{\mu} &
		F 
		&
		\bar{F} P \bar{F} \ar[Rightarrow]{r}{\tau \bar{F}} &
		F \bar{F} \ar[Rightarrow]{r}{\mu} &
		F  
		\end{tikzcd}
		\]
		\item the following composites both coincide with $\mu\mu \colon PP\bar{F}\bar{F} \Rightarrow P \bar{F}$
		\[
		\begin{tikzcd}
		P F \bar{F} \ar[Rightarrow]{r}{\mu \bar{F}} &
		F \bar{F} \ar[Rightarrow]{r}{\mu} &
		F 
		%&
		%	PP\bar{F}\bar{F} \ar[Rightarrow]{r}{\mu \mu} &
		%	P \bar{F}  
		&
		P F \bar{F} \ar[Rightarrow]{r}{P \mu} &
		\bar{F} F \ar[Rightarrow]{r}{\mu} &
		F 
		\end{tikzcd}
		\]
	\end{enumerate}
\end{lemma}



\begin{proof}
	Since all of (i),(ii),(iii) are symmetric, we show only the first half of each. (i),(ii),(iii) will follow from the following diagrams.
	\begin{equation}\label{MODSTRCOMRE EQ}
	\begin{tikzcd}[column sep=1.9em]
	P \ar[Rightarrow]{d}
	\ar[Rightarrow]{r}{\eta P}
	&
	\bar{F}P \ar[Rightarrow]{d} \ar[Rightarrow]{rd}{\tau}
	&
	&%
	P P \bar{F} \ar[Rightarrow]{d}[swap]{\mu \bar{F}}
	\ar[Rightarrow]{r}
	&
	FFF \ar[Rightarrow]{d}[swap]{\mu F}
	\ar[Rightarrow]{r}{F \mu}
	&
	FF \ar[Rightarrow]{d}{\mu}
	&%
	P \bar{F} P  \ar[Rightarrow]{d}[swap]{P \tau}
	\ar[Rightarrow]{r}
	&
	FFF \ar[Rightarrow]{d}[swap]{F \mu}
	\ar[Rightarrow]{r}{\mu F}
	&
	FF \ar[Rightarrow]{d}{\mu}
	\\
	F \ar[Rightarrow]{r}[swap]{\eta F}
	&
	FF \ar[Rightarrow]{r}[swap]{\mu}
	&
	F
	&%
	P \bar{F} \ar[Rightarrow]{r}
	&
	FF \ar[Rightarrow]{r}[swap]{\mu}
	&
	F
	&%
	P F \ar[Rightarrow]{r}
	&
	FF \ar[Rightarrow]{r}[swap]{\mu}
	&
	F
	\end{tikzcd}
	\end{equation}
	For (i), consider the left diagram in \eqref{MODSTRCOMRE EQ},
	where the square commutes since
	$\bar{F} \to P$ is a monad map (more precisely, it is thus compatible with monad units) and the triangle commutes by definition of $\tau$.
	Since the bottom composite is the identity (due to $F$ being a monad), (i) follows.
	
	For (ii), consider the center diagram in \eqref{MODSTRCOMRE EQ},
	whose left (resp. right) half commutes since
	$P \Rightarrow F$ is a monad map (resp. $F$ is a monad).
	But by Definition \ref{AMALGMON DEF}(iii), the lower composite therein
	is the identity and the top composite is the natural map
	$PF \Rightarrow FF$, so (ii) follows.
	
	For (iii), consider the right diagram in \eqref{MODSTRCOMRE EQ},
	whose left (resp. right) half commutes by definition of $\tau$
	(resp. since $F$ is a monad).
	The bottom composite therein is the multiplication
	$\mu \colon F \bar{F} \Rightarrow F$
	while, by Definition \ref{AMALGMON DEF}(iii), the top composite is
	the natural map $FP \Rightarrow FF$, so (iii) follows.
	
	
	For (iv), part (i) allows us to rewrite 
	the two composites as
	\[
	\begin{tikzcd}
	PP\bar{F}\bar{F} \ar[Rightarrow]{r}{\mu \bar{F}\bar{F}} &
	P \bar{F} \bar{F} \ar[Rightarrow]{r}{P\mu} &
	P \bar{F}
	&
	PP\bar{F}\bar{F} \ar[Rightarrow]{r}{PP \mu } &
	P P \bar{F} \ar[Rightarrow]{r}{\mu \bar{F}} &
	P \bar{F}
	\end{tikzcd}
	\]
	which are both $\mu\mu\colon PP\bar{F}\bar{F} \Rightarrow P \bar{F}$.
\end{proof}


\begin{proposition}\label{ALTMULT PROP}
	\begin{enumerate}[label=(\roman*)]
		\item
		the multiplication
		$\mu \colon FF \Rightarrow F$ 
		can also be described as the composite
		\begin{equation}\label{ALTMULT EQ}
		\begin{tikzcd}
		P \bar{F} P \bar{F} \ar[Rightarrow]{r}{P \tau \bar{F}}
		&
		P P \bar{F} \bar{F}   \ar[Rightarrow]{r}{\mu \mu}
		&
		P \bar{F} 
		\end{tikzcd}
		\end{equation}
		\item
		the following diagrams commute
		\begin{equation}\label{DOUBLETWIST EQ}
		\begin{tikzcd}
		\bar{F} P P \ar[Rightarrow]{r}{\tau P} 
		\ar[Rightarrow]{d}[swap]{\bar{F} \mu}
		&
		P \bar{F} P \ar[Rightarrow]{r}{P\tau}
		&
		P P \bar{F} \ar[Rightarrow]{d}{\mu \bar{F}}
		&%%
		\bar{F} \bar{F} P  \ar[Rightarrow]{r}{\bar{F} \tau}
		\ar[Rightarrow]{d}[swap]{\mu P}
		&
		\bar{F} P \bar{F}  \ar[Rightarrow]{r}{\tau \bar{F}}
		&
		P \bar{F} \bar{F} \ar[Rightarrow]{d}{P \mu}
		\\
		\bar{F} P \ar[Rightarrow]{rr}{\tau} 
		&&
		P \bar{F}
		&%%
		\bar{F} P \ar[Rightarrow]{rr}{\tau} 
		&&
		P \bar{F}
		\end{tikzcd}
		\end{equation}
		\item
		the following diagrams commute
		\begin{equation}\label{UNITTWIST EQ}
		\begin{tikzcd}
		\bar{F} \ar[Rightarrow]{d}[swap]{\bar{F} \eta}
		\ar[Rightarrow]{rd}{\eta \bar{F}}
		&
		&%%
		P \ar[Rightarrow]{d}[swap]{\eta P}
		\ar[Rightarrow]{rd}{P \eta}
		&
		\\
		\bar{F} P \ar[Rightarrow]{r}{\tau} 
		&
		P \bar{F}
		&%%
		\bar{F} P \ar[Rightarrow]{r}{\tau} 
		&
		P \bar{F}
		\end{tikzcd}
		\end{equation}
	\end{enumerate}
\end{proposition}


\begin{proof}
	Note first that (iii) is simply 
	a rephrasing of Lemma \ref{MODSTRCOMRE LEM}(i), which we include here
	for the sake of Remark \ref{ALTMULT REM}.
	(i) and (ii) will follow from the following diagrams.
	\begin{equation}\label{ALTMPR EQ}
	\begin{tikzcd}
	P \bar{F} P \bar{F} \ar[Rightarrow]{d} 
	\ar[Rightarrow]{r}{P \tau \bar{F}}
	&
	P F \bar{F}  \ar[Rightarrow]{r}{\mu \bar{F}} \ar[Rightarrow]{d}
	&
	F \bar{F} \ar[Rightarrow]{r}{\mu}
	\ar[Rightarrow]{d}
	&
	F \ar[equal]{d}
	&&%
	\bar{F} P P \ar[Rightarrow]{rr}{\tau P}
	\ar[Rightarrow]{rd}{}
	\ar[Rightarrow]{dd}[swap]{\bar{F} \mu}
	&&
	F P \ar[Rightarrow]{d}
	\\
	FFFF \ar[Rightarrow]{r}{F \mu F}
	\ar[Rightarrow]{d}[swap]{\mu \mu}
	&
	FFF \ar[Rightarrow]{r}{\mu F}
	&
	FF \ar[Rightarrow]{r}{\mu}
	&
	F \ar[equal]{d}
	&&%
	&
	FFF \ar[Rightarrow]{d}[swap]{F \mu}
	\ar[Rightarrow]{r}{\mu F}
	&
	FF \ar[Rightarrow]{d}{\mu}
	\\
	FF \ar[Rightarrow]{rrr}{\mu}
	&&&
	F
	&&%
	\bar{F} P \ar[Rightarrow]{r}
	&
	FF \ar[Rightarrow]{r}[swap]{\mu}
	&
	F
	\end{tikzcd}
	\end{equation}
	For (i), consider the left diagram in \eqref{ALTMPR EQ}, where the top left square commutes by definition of the twist $\tau$,
	the remaining top squares commute by definition of the action multiplication,
	and the bottom rectangle commutes due to $F$ being a monad.
	Definition \ref{AMALGMON DEF}(iii)
	then says that the left vertical composite therein is the identity,
	while by Lemma \ref{MODSTRCOMRE LEM}(iv) the top composite is 
	\eqref{ALTMULT EQ}, so (i) follows.
	
	For (ii), by symmetry it suffices to show the first half.
	Consider now the right diagram in \eqref{ALTMPR EQ},
	where the top right section commutes by definition of $\tau$,
	the bottom left section commutes since $P \Rightarrow F$ is a monad map,
	and the bottom right square commutes since $F$ is monad.
	But now the bottom composite therein is $\tau$ by definition
	while, by Lemma \ref{MODSTRCOMRE LEM}(iii),
	the top right outer composite from $\bar{F}PP$ to $F$ matches the top right outer composite in (the left side of) \eqref{DOUBLETWIST EQ},
	so (ii) follows.
\end{proof}



\begin{remark}\label{ALTMULT REM}
	\eqref{ALTMULT EQ} shows that the monad structure on $F=P\bar{F}$ is determined by monoidal structures on $P$ and $\bar{F}$ together with  the twist map $\tau$.
	Moreover, \eqref{DOUBLETWIST EQ} and \eqref{UNITTWIST EQ} provide necessary conditions on $P,\bar{F},\tau$
	for the formula \eqref{ALTMULT EQ}
	to describe a monoidal structure on $F=P\bar{F}$
	that satisfies the conditions in 
	Definition \ref{AMALGMON DEF}.
	In fact, the conditions \eqref{DOUBLETWIST EQ}, \eqref{UNITTWIST EQ}
	turn out to also be sufficient.
	Indeed, it is not hard to check 
	that \eqref{DOUBLETWIST EQ} implies that either of the two iterated multiplications
	$FFF \Rightarrow F$ induced by \eqref{ALTMULT EQ}
	match the map 
	$P\bar{F} P\bar{F} P \bar{F} \Rightarrow 
	PPP \bar{F} \bar{F} \bar{F} \Rightarrow P \bar{F}$,
	where the first map is a (unique) combination of twists $\tau$
	and the second map is given by the iterated multiplications for $P,\bar{F}$.
	Similarly, \eqref{UNITTWIST EQ} implies that the maps
	$P \Rightarrow F$, $\bar{F} \Rightarrow F$
	are maps of monads.
\end{remark}



\subsection{Kleisli categories}

Given a map of monads $T \Rightarrow F$ (for example, $P \Rightarrow F$ or $\bar F \Rightarrow F$), one can explore the existence and properties of various free and forgtful functors.
Somewhat surprisingly, the most natural and general setting for these comparisons is not the entire category of algebras,
but instead in the Kleisli category of free algebras and all algebra maps between them.

\begin{definition}[{\cite{Kl65}}]
	Let $T$ be a monad on a category $\mathcal C$.
	The \textit{Kleisli category} $\mathsf{Kl}_T$ is the category defined as follows:
	it has the same objects as $\mathcal{C}$,
	and arrows from $A$ to $B$,
	which we denote $A \rightsquigarrow B$,
	are given by arrows $A \to TB$ in $\mathcal{C}$.
	Composition
	$A \rightsquigarrow B \rightsquigarrow C$
	of underlying maps 
	$A \xrightarrow{f} TB$,
	$B \xrightarrow{g} TC$
	given by the composite
	$A \xrightarrow{f} TB \xrightarrow{Tg} TTB \xrightarrow{\mu} TB$.
	
	We say an arrow
	$A \rightsquigarrow B$ in $\mathsf{Kl}_F$ is a 
	\emph{free arrow}
	if the underlying map has a factorization
	$A \xrightarrow{f} B \xrightarrow{\eta} TB$.
	In particular, the unit arrow $A \rightsquigarrow A$ in $\Kl_T$
	is the free arrow $A \xrightarrow{=} A \xrightarrow{\eta} TA$.
\end{definition}

\begin{remark}\label{KLEISLIDEF REM}
	% Letting $T$ be a monad on the category $\mathcal{C}$,
	% recall that the Kleisli category
	% $\mathsf{Kl}_T$ is the category with:
	% the same objects as $\mathcal{C}$;
	% arrows from $A$ to $B$,
	% which we denote $A \rightsquigarrow B$,
	% the underlying arrows $A \to TB$ in $\mathcal{C}$;
	% composition
	% $A \rightsquigarrow B \rightsquigarrow C$
	% of underlying maps 
	% $A \xrightarrow{f} TB$,
	% $B \xrightarrow{g} TC$
	% given by the composite
	% $A \xrightarrow{f} TB \xrightarrow{Tg} TTB \xrightarrow{\mu} TB$.
	% Moreover,
	Justifying our intuitive description above,
	there is a canonical fully faithful inclusion
	\[
	\begin{tikzcd}[row sep = 0pt]
	\Kl_T \ar[hookrightarrow]{r}{\iota} &
	\Alg_T
	\\
	A \ar[mapsto]{r} &
	TA
	\\
	(A \xrightarrow{f} TB) \ar[mapsto]{r} &
	(TA \xrightarrow{Tf} TTB \xrightarrow{\mu} TB) 
	\end{tikzcd} 
	\]
	which identifies $\mathsf{Kl}_T$ with the full subcategory of
	$\mathsf{Alg}_T$ spanned by the free algebras.
	%
	% Moreover, we say an arrow
	% $A \rightsquigarrow B$ in $\mathsf{Kl}_F$ is a 
	% \emph{free arrow}
	% if the underlying map has a factorization
	% $A \xrightarrow{f} B \xrightarrow{\eta} TB$.
	Moreover, an arrow is free iff
	% This is equivalent to
	its image under $\iota$
	is the free map
	$TA \xrightarrow{Tf} TB$.
	% Note that free arrows provide a canonical functor
	% $\mathcal C \to \Kl_T$ and that the unit arrow $A \rightsquigarrow A$ in $\Kl_T$
	% is the free arrow $A \xrightarrow{=} A \xrightarrow{\eta} TA$.
	In particular, the free functor $\mathcal C \to \Alg_T$ factors through $\Kl_T$ via free arrows.
\end{remark}


To start our analysis,
for any map of monads $\psi \colon T \to F$,
one has a well known forgetful functor 
$\mathsf{Alg}_F \xrightarrow{\psi^{\**}} \mathsf{Alg}_T$
sending a $F$-monad $Y$ with multiplication
$FY \xrightarrow{m} Y$
to a $T$-monad with the same underlying object $Y$
and multiplication given by
$TY \xrightarrow{\psi} FY \xrightarrow{m} Y$.

A dual situation holds for Kleisli categories:
\begin{lemma}
	\label{KLEISLIPUSH LEM}
	Given a map of monads $\psi \colon T \to F$,
	there is a pushforward functor
	$\mathsf{Kl}_T \xrightarrow{\psi_!} \mathsf{Kl}_F$,
	defined as the identity on objects,
	and sending the map $A \rightsquigarrow B$
	given by the underlying map $A \xrightarrow{f} TB$
	to the map $\psi_!A \rightsquigarrow \psi_!B$
	given by the underlying composite
	$A \xrightarrow{f} TB \xrightarrow{\psi} FB$.
\end{lemma}
\begin{proof}
	To check $\psi_!$ respects composition of arrows, let
	$A \rightsquigarrow B$,
	$B \rightsquigarrow C$
	be maps in $\Kl_T$ with
	underlying maps
	$A \xrightarrow{f} TB$,
	$B \xrightarrow{f} TC$.
	Then $\psi_!$ of the composite 
	$A \rightsquigarrow B \rightsquigarrow C$
	is given by the top right composite in \eqref{CHECKFUN EQ}
	while the composite of 
	$\psi_!A \rightsquigarrow \psi_! B$,
	$\psi_! B \rightsquigarrow \psi_! C$
	is given by the bottom  left composite in \eqref{CHECKFUN EQ}.
	\begin{equation}\label{CHECKFUN EQ}
	\begin{tikzcd}[column sep = 1.55em]
	A \ar{r}{f} 
	&
	TB \ar{r}{Tg} \ar{d}[swap]{\psi}
	&
	TTC \ar{rr}{\mu}  \ar{d}[swap]{\psi T}
	&&
	TC \ar{d}{\psi}
	\\
	&
	FB \ar{r}{Fg}
	&
	FTB \ar{r}{F\psi}
	&
	FFC \ar{r}{\mu}
	&
	FC
	\end{tikzcd}
	\end{equation}
	Lastly, note that $\psi_!$ maps the free arrow 
	$A \xrightarrow{f} B \xrightarrow{\eta} TB$
	to the free arrow
	$A \xrightarrow{f} B \xrightarrow{\eta} FB$,
	so that in particular it preserves unit arrows.
\end{proof}




\begin{proposition}\label{PARTADJ PROP}
	The functors in Remark \ref{KLEISLIDEF REM} and Lemma \ref{KLEISLIPUSH LEM}
	fit into a (non-commutative) square
	\begin{equation}\label{PARTADJ EQ}
	\begin{tikzcd}
	\mathsf{Kl}_{T} \ar{r}{\psi_!} \ar[hookrightarrow]{d}[swap]{\iota}&
	\mathsf{Kl}_{F} \ar[hookrightarrow]{d}{\iota} \arrow[dl, phantom, "\times"]
	\\
	\mathsf{Alg}_{T} &
	\mathsf{Alg}_{F}  \ar{l}{\psi^{\**}}.
	\end{tikzcd}
	\end{equation}
	and there are isomorphisms as below, natural in $A \in \mathsf{Kl}_T$, $Y \in \mathsf{Alg}_F$.
	\begin{equation}\label{PARTADJISO EQ}
	\mathsf{Alg}_T(\iota A, \psi^{\**} Y) \simeq
	\mathsf{Alg}_F(\iota \psi_! A,Y)
	\end{equation}
\end{proposition}
Informally, this result says that
$\psi_!$ is a ``partial left adjoint'' to $\psi^{\**}$
along the fully-faithful inclusion 
$\mathsf{Kl}_T \overset{\iota}{\hookrightarrow} \mathsf{Alg}_T$,
motivating our choice of notation.


\begin{proof}
	The isomorphisms \eqref{PARTADJISO EQ} are the composites of the isomorphisms
	\begin{equation}\label{PARTADJPROOF EQ}
	\mathsf{Alg}_T(T A, \psi^{\**} Y) \simeq
	\mathcal{C}(A,Y) \simeq
	\mathsf{Alg}_F(FA,Y),
	\end{equation}
	so it remains to show naturality.
	Naturality with respect to $Y \in \Alg_F$ is clear.
	However, naturality with respect to $A \in \Kl_T$
	requires care, since a priori \eqref{PARTADJPROOF EQ}
	is only functorial with respect to free maps.
	But now note that 
	$\varphi \colon FA \to Y$
	and 
	$\bar{\varphi} \colon TA \to \psi^{\**} Y$
	are identified precisely if the two composites
	$A \to Y$ in the left diagram below coincide (morever, both triangles therein commute, with commutativity of the right triangle
	inherited from commutatity of the full diagram, due $TA$ being free $T$-algebra).
	Given an arrow $\bar{A} \rightsquigarrow A$ in $\Kl_T$
	given by an underlying map $\bar{A} \xrightarrow{f} TA$, 
	we must then show that the two outer composites
	$\bar{A} \to Y$ in the rightmost diagram coincide.
	\begin{equation}\label{ADJOINTCORR EQ}
	\begin{tikzcd}[column sep = 1.55em]
	A \ar{r}{\eta} \ar{rd}[swap]{\eta}
	&
	T A \ar{r}{\bar{\varphi}} \ar{d}[swap]{\psi}
	&
	Y 
	&&%%
	\bar{A} \ar{r}{\eta} \ar{rd}[swap]{\eta}
	&
	T \bar{A} \ar{r}{\iota f} \ar{d}[swap]{\psi}
	&
	T A \ar{r}{\bar{\varphi}} \ar{d}[swap]{\psi}
	&
	Y 
	\\
	&
	FA \ar{ru}[swap]{\varphi}
	&
	&&%%
	&
	F \bar{A} \ar{r}[swap]{\iota (\psi_! f)}
	&
	FA \ar{ru}[swap]{\varphi}
	&
	\end{tikzcd}
	\end{equation}
	It hence suffices to check that the two composites
	$\bar{A} \to FA$ in that diagram coincide and, after unpacking definitions, this follows from the commutativity of the following diagram.
	\begin{equation}
	\begin{tikzcd}[column sep = 1.55em]
	\bar{A} \ar{r}{\eta} \ar{rd}[swap]{\eta} 
	&
	T\bar{A} \ar{r}{Tf} \ar{d}[swap]{\psi}
	&
	TTA \ar{rr}{\mu}  \ar{d}[swap]{\psi T}
	&&
	TA \ar{d}{\psi}
	\\
	&
	F \bar{A} \ar{r}{Ff}
	&
	FTA \ar{r}{F\psi}
	&
	FFA \ar{r}{\mu}
	&
	FA
	\end{tikzcd}
	\end{equation}
\end{proof}

Necessary and sufficient conditions to extend this partial adjoint to a full adjoint are given below.
\begin{corollary}
	\label{ALGLEFTEX_COR}
	Fix a map of monads $\psi \colon T \to F$.
	\begin{enumerate}[label = (\roman*)]
		\item The left adjoint $\psi_!$ of $\psi^{\**} \colon \Alg_F \to \Alg_T$ exists iff the coequalizers of $F$-algebras
		\[
		\psi_!(X):= \mathop{coeq}\left( FTX \overset{\mu}{\underset{Fm}{\rightrightarrows}} FX \right)
		\]
		exist for all $X \in \mathsf{Alg}_T$.
		\item If $\psi_!$ exists, then the triangle of left adjoints below commutes.
		\[
		\begin{tikzcd}
		\mathcal C \arrow[d, "T"'] \arrow[r, "F"]
		&
		\mathsf{Alg}_F
		\\
		\mathsf{Alg}_T \arrow[ur, "{\psi_!}"']
		\end{tikzcd}
		\]           
	\end{enumerate}
\end{corollary}
\begin{proof}
	The existence of $\psi_!$ is equivalent to the representability of the functors 
	$\Alg_T(X,\psi^{\**}(-))\colon \Alg_{F} \to \mathsf{Set}$
	for each $X \in \Alg_T$,
	so that Proposition \ref{PARTADJ PROP} implies that these functors are representable whenever $X \simeq FA$ is free.
	Hence, since any $X \in \Alg_T$
	with multiplication $m \colon TX \to X$
	is canonically described by the coequalizer % of free $T$-algebras
	$coeq \left(
	T T X 
	\overset{\mu}{\underset{Fm}{\rightrightarrows}}
	T X
	\right)$,
	(i) result follows.
	
	Part (ii) follows since the associated triangle of right adjoints commutes.
\end{proof}


% \begin{remark}\label{ALGLEFTEX REM}
% Recall that the existence of a left adjoint $\psi_!$ to 
% $\psi^{\**} \colon \Alg_F \to \Alg_T$
% is equivalent to the representability of the functors 
% $\Alg_T(X,\psi^{\**}(-))\colon \Alg_{F} \to \mathsf{Set}$
% for each $X \in \Alg_T$,
% so that Proposition \ref{PARTADJ PROP} implies that these functors are representable whenever $X \simeq FA$ is free.

% Hence, since any $X \in \Alg_T$
% with multiplication $m \colon TX \to X$
% is canonically described by the coequalizer % of free $T$-algebras
% $coeq \left( T T X 
% \overset{\mu}{\underset{Fm}{\rightrightarrows}}
% T X\right)$
% it follows that the left adjoint 
% $\psi_! \colon \Alg_F \to \Alg_T$
% exists iff the coequalizers 
% $coeq \left( F T X 
% \overset{\mu}{\underset{Fm}{\rightrightarrows}}
% F X\right)
% $
% of $F$-algebras exist for all $X \in \Alg_T$, in which case it is
% $\psi_!(X)=coeq \left( F T X 
% \overset{\mu}{\underset{Fm}{\rightrightarrows}}
% F X\right)$.
% \end{remark}


\begin{remark}\label{ADJSRMON REM}
	Beck's monadicity theorem (cf. \cite[VI.7 Thm. 1]{McL},\cite[Thm. 5.5.1]{Ri17})
	characterizes monadic adjunctions $L\colon \mathcal{C} \rightleftarrows \mathcal{D}\colon R$
	as those where $R$ creates $R$-split coequalizers. 
	
	Hence, should the adjunction 
	$\psi_! \colon \Alg_T \rightleftarrows \Alg_F \colon \psi^{\**}$
	in Corollary \ref{ALGLEFTEX_COR} exist,
	it readily follows from Beck's theorem that this adjunction is monadic
	(since in 
	$\mathcal{C} \rightleftarrows
	\mathsf{Alg}_T \rightleftarrows
	\Alg_F$
	both the left and total adjunctions are monadic).
	However, since $\psi_!$ is defined using coequalizers of $F$-algebras,
	it is in general not possible to give a simple description of the adjunction monad
	$F_T = \psi^{\**} \psi_!$.
	%
	%Nonetheless, should $T,F$ preserve reflexive coequalizers in $\mathcal{C}$,
	%one has that reflexive coequalizers in algebras
	%are underlying coequalizers in $\mathcal{C}$
	%\cite[Thm. 5.6.5]{Ri17}.
	%Therefore, by manipulating the coequalizers one can show that the multiplication for $F_T$ the induced map on vertical coequalizers for the diagram
	%\[
	%\begin{tikzcd}[column sep =40pt]
	%	FTFT X
	%	\ar[shift left=.3em]{d}{\mu\mu}
	%	\ar[shift right=.3em]{d}[swap]{F\mu m}
	%	\ar{r}{\mu FT \circ \mu T}
	%&
	%	FTX
	%	\ar[shift left=.3em]{d}{\mu}
	%	\ar[shift right=.3em]{d}[swap]{F m}
	%\\
	%	FF X
	%	\ar{r}{\mu}
	%&
	%	F X
	%\end{tikzcd}
	%\]
\end{remark}


\subsection{Adjoints for combinable monads}

Returning to our setup from \S \ref{COMBMON_SEC},
we have the existence of a right adjoint to $\psi_!$ on Kleisli categories for a particular map of monads $\psi$.

\begin{definition}
	Let $P,\bar{F}$ and $F = P \bar{F}$ be as in 
	Definition \ref{AMALGMON DEF},
	and write
	\[
	\psi = P\eta_{\bar F} \colon P \Rightarrow P \bar{F} = F.
	\]
	Define the functor
	$\psi^{\**} \colon \Kl_F \to \Kl_P$
	on objects by
	$\psi^{\**}A = \bar{F} A$,
	and sending the arrow
	$A \rightsquigarrow B$
	with underlying map
	$A \xrightarrow{f} FB$
	to the arrow 
	$\bar{F} A \rightsquigarrow \bar{F} B$
	with underlying map the composite
	$\bar{F} A \xrightarrow{\bar{F}f} \bar{F} FB
	=
	\bar{F}P \bar{F} B \xrightarrow{\tau \bar{F}}
	P \bar{F} \bar{F} B \xrightarrow{P \mu}
	P\bar{F}B$
	or, equivalently (cf. Lemma \ref{MODSTRCOMRE LEM}(iii)(ii)),
	$\bar{F} A \xrightarrow{\bar{F}f} \bar{F} FB
	\xrightarrow{\mu} FB = P\bar{F}B$.
\end{definition}

\begin{proposition}\label{PARTADJP PROP}
	Let $P,\bar{F}$ and $F = P \bar{F}$ be as above.
	% Definition \ref{AMALGMON DEF},
	% and write $\psi = P\eta \colon P \Rightarrow P \bar{F} = F$.
	%
	% Then there is a functor
	% $\psi^{\**} \colon \Kl_F \to \Kl_P$
	% given on objects by
	% $\psi^{\**}A = \bar{F} A$
	% and sending the arrow
	% $A \rightsquigarrow B$
	% with underlying arrow 
	% $A \xrightarrow{f} FB$
	% to the arrow 
	% $\bar{F} A \rightsquigarrow \bar{F} B$
	% with underlying arrow the composite
	% $\bar{F} A \xrightarrow{\bar{F}f} \bar{F} FB
	% =
	% \bar{F}P \bar{F} B \xrightarrow{\tau \bar{F}}
	% P \bar{F} \bar{F} B \xrightarrow{P \mu}
	% P\bar{F}B$
	% or, equivalently (cf. Lemma \ref{MODSTRCOMRE LEM}(iii)(ii)),
	% $\bar{F} A \xrightarrow{\bar{F}f} \bar{F} FB
	% \xrightarrow{\mu} FB = P\bar{F}B$.
	%
	In the diagram
	\begin{equation}\label{PARTADJP EQ}
	\begin{tikzcd}
	\mathsf{Kl}_{P} 
	\ar[shift left=0.25em]{r}{\psi_!} 
	\ar[hookrightarrow]{d}[swap]{\iota}
	&
	\mathsf{Kl}_{F} 
	\ar[shift left=0.25em]{l}{\psi^{\**}}
	\ar[hookrightarrow]{d}{\iota}
	\\
	\mathsf{Alg}_{P} &
	\mathsf{Alg}_{F}  \ar{l}{\psi^{\**}}
	\end{tikzcd}
	\end{equation}
	one has that:
	\begin{enumerate}[label=(\roman*)]
		\item the top horizontal maps are adjoint,
		with adjuntion unit
		$A \rightsquigarrow \psi^{\**} \psi_! A$
		for $A \in \Kl_P$
		and counit
		$\psi_! \psi^{\**} B \rightsquigarrow B$
		for $B \in \Kl_F$
		given respectively by 
		\begin{equation}\label{UNITCOUNIT EQ}
		A \xrightarrow{\eta} 
		\bar{F} A \xrightarrow{\eta \bar{F}}
		P \bar{F} A
		\qquad
		\mbox{and}
		\qquad
		\bar{F} B \xrightarrow{\eta \bar{F}}
		P \bar{F} B =
		F B.
		\end{equation}
		\item the composites
		$
		\mathsf{Kl}_F \xrightarrow{\psi^{\**}} 
		\mathsf{Kl}_P \xrightarrow{\iota} 
		\mathsf{Alg}_P
		$
		and
		$
		\mathsf{Kl}_F \xrightarrow{\iota}
		\mathsf{Alg}_F  \xrightarrow{\psi^{\**}} 
		\mathsf{Alg}_P
		$
		coincide.
	\end{enumerate}
\end{proposition}



\begin{proof}
	We first address (ii).
	On objects,
	after unpacking notation
	$\psi^{\**} \iota B$
	is the $P$-algebra
	$F B$ with multiplication
	$PFB \rightarrow FFB \xrightarrow{\mu} B$
	while 
	$ \iota \psi^{\**} B$
	is the $P$-algebra
	$P \bar{F} B $
	with multiplication
	$P P \bar{F} B \xrightarrow{\mu \bar{F}} P \bar{F} B$,
	and these coincide by Lemma \ref{MODSTRCOMRE LEM}(ii).
	On a map
	$B \rightsquigarrow \bar{B}$
	with underlying arrow $B \xrightarrow{f} F \bar{B}$,
	applying $\psi^{\**} \iota$
	gives the map
	$F B \xrightarrow{F f} FF\bar{B} \xrightarrow{\mu} F \bar{B}$
	while applying $\iota \psi^{\**}$
	gives the map
	$P \bar{F} B \xrightarrow{P \bar{F} f} 
	P \bar{F} F \bar{B} = P \bar{F} P \bar{F} \bar{B}
	\xrightarrow{P \tau \bar{F}}
	P P \bar{F} \bar{F} \bar{B}
	\xrightarrow{PP \mu}
	P P \bar{F} \bar{B} 
	\xrightarrow{\mu \bar{F}} P\bar{F} \bar{B}$,
	and these coincide by 
	Proposition \ref{ALTMULT PROP}(i).
	
	
	We now address the remaining claims. 
	Recalling that the $\iota$ functors are fully faithful,
	$\psi^{\**} \colon \Kl_F \to \Kl_P$
	can be thought of as a restriction of
	$\psi^{\**} \colon \Alg_F \to \Alg_P$. 
	Hence, the implicit claim that $\psi^{\**} \colon \Kl_F \to \Kl_P$
	is a functor is inherited from 
	$\psi^{\**} \colon \Alg_F \to \Alg_P$ being a functor.
	And, similarly, the adjunction claim in (i)
	is inherited from the ``partial adjunction'' claim in 
	Proposition \ref{PARTADJP PROP}.
	
	
	Lastly, to describe the unit 
	(resp. counit)
	we must find the adjoint
	to 
	$\psi_! A \xrightarrow{=} \psi_! A$
	(resp. $\psi^{\**} B \xrightarrow{=} \psi^{\**} B$)
	which corresponds under $\iota$ to the identity
	$FA = FA$
	(resp. $P\bar{F} B = FB$).
	By the string of identifications 
	\eqref{PARTADJPROOF EQ}
	the adjoint is given 
	by the top (resp. bottom) composite on the left
	(resp. diagram) below
	(cf. \eqref{ADJOINTCORR EQ}).
	\begin{equation}
	\begin{tikzcd}[column sep = 1.55em]
	A \ar{r}{\eta} \ar{rd}[swap]{\eta}
	&
	P A \ar[dashed]{r}{\psi} \ar{d}[swap]{\psi}
	&
	FA
	&&%%
	\bar{F} B \ar{r}{\eta} \ar{rd}[swap]{\eta}
	&
	P \bar{F} B \ar[equal]{r} \ar{d}[swap]{\psi}
	&
	F B 
	\\
	&
	FA \ar[equal]{ru}
	&
	&&%%
	&
	F\bar{F} B \ar[dashed]{ru}[swap]{\mu}
	&
	\end{tikzcd}
	\end{equation}
	That these composites match the maps in
	\eqref{UNITCOUNIT EQ}
	follows since $\psi\colon P \Rightarrow F$ is a map of monads. 
\end{proof}



\begin{remark}\label{MONALGP REM}
	By Proposition \ref{PARTADJP PROP}
	one has an adjunction monad $\psi^{\**} \psi_!$
	on $\Kl_P$ which is given on objects by $\bar{F}$.
	Moreover the first half of
	\eqref{UNITCOUNIT EQ} shows that the unit
	$I \Rightarrow \psi^{\**} \psi_!$
	consists of the free arrows on the unit $I \Rightarrow \bar{F}$
	while, 
	by applying the definition of $\psi^{\**}$ to the second half of
	\eqref{UNITCOUNIT EQ}
	gives that the monad multiplication
	$\psi^{\**} \psi_!\psi^{\**} \psi_! \Rightarrow \psi^{\**} \psi_!$
	is given by the left right composite below.
	But hence, by the top right composite, 
	this monad multiplication consists of the free arrows on
	the monad multiplication $\bar{F} \bar{F} \Rightarrow \bar{F}$. 
	\begin{equation}
	\begin{tikzcd}[column sep = 1.55em]
	\bar{F} \bar{F} A \ar{d}[swap]{\bar{F}\eta \bar{F}}
	\ar[equal]{r} 
	&
	\bar{F} \bar{F} A 
	\ar{r}{\mu} \ar{d}[swap]{\eta \bar{F} \bar{F}}
	&
	\bar{F} A   \ar{d}[swap]{\eta \bar{F}}
	\\
	\bar{F} P \bar{F} A \ar{r}[swap]{\tau \bar{F}}
	&
	P \bar{F} \bar{F} A \ar{r}[swap]{P\mu}
	&
	P \bar{F} A
	\end{tikzcd}
	\end{equation}
	As an aside, we note that one can further check that the top adjunction in \eqref{PARTADJ EQ} is a Kleisli adjunction, i.e.
	that it identifies $\Kl_F$ as the Kleisli category over 
	$\Kl_P$ for the monad $\psi^{\**} \psi_!$.
\end{remark}


\begin{remark}
	Should $\psi^{\**}\colon \Alg_F \to \Alg_P$ admit a left adjoint, 
	one has that in the diagram
	\begin{equation}\label{PARTADJDB EQ}
	\begin{tikzcd}
	\mathsf{Kl}_{P} 
	\ar[shift left=0.25em]{r}{\psi_!} 
	\ar[hookrightarrow]{d}[swap]{\iota}
	&
	\mathsf{Kl}_{F} 
	\ar[shift left=0.25em]{l}{\psi^{\**}}
	\ar[hookrightarrow]{d}{\iota}
	\\
	\mathsf{Alg}_{P} 
	\ar[shift left=0.25em]{r}{\psi_!} &
	\mathsf{Alg}_{F}
	\ar[shift left=0.25em]{l}{\psi^{\**}}
	\end{tikzcd}
	\end{equation}
	both functors in the bottom adjunction
	extend the functors in the top adjunction.
	As such, the monad
	$F_P = \psi^{\**} \psi_!$ on $\Alg_P$
	extends the monad $\psi^{\**} \psi_!$ on $\Kl_P$ which, 
	following \ref{MONALGP REM}, can be thought of as the monad $\bar{F}$.
	
	As noted in Remark \ref{ADJSRMON REM}, the monad $F_T$ is in general hard to describe. 
	Nonetheless, should $P,F$ preserve reflexive coequalizers in $\mathcal{C}$,
	one has that reflexive coequalizers in algebras
	are underlying coequalizers in $\mathcal{C}$
	\cite[Thm. 5.6.5]{Ri17}
	and thus preserved by both functors in the bottom adjunction, and hence also by $F_T$.
	It thus follows that $F_TF_TX$
	can be described as the left vertical coequalizer in the diagram below 
	(where the two squares coincide, 
	with the right square included to more easily describe the vertical maps) 
	\begin{equation}\label{FRMONDES EQ}
	\begin{tikzcd}[column sep =40pt]
	P \bar{F}\bar{F} P X
	\ar[shift left=.3em]{d}{}
	\ar[shift right=.3em]{d}[swap]{}
	\ar{r}{P\mu P}
	&
	P \bar{F} P X
	\ar[shift left=.3em]{d}{}
	\ar[shift right=.3em]{d}[swap]{}
	&%
	F\bar{F} P X
	\ar[shift left=.3em]{d}{F\bar{F} m}
	\ar[shift right=.3em]{d}[swap]{\mu\bar{F} \circ F\tau}
	\ar{r}{\mu P}
	&
	F P X
	\ar[shift left=.3em]{d}{F m}
	\ar[shift right=.3em]{d}[swap]{\mu}
	\\
	P \bar{F}\bar{F} X
	\ar{r}{P\mu}
	&
	P \bar{F} X
	&%
	F\bar{F} X
	\ar{r}{\mu}
	&
	F X
	\end{tikzcd}
	\end{equation}
	while the multiplication $F_TF_TX \to F_TX$
	is the induced is the induced map on the vertical coequalizers.
\end{remark}



%\begin{remark}
%When applied to the example of operads, 
%the right coequalizer in \eqref{FRMONDES EQ}
%(together with realization of categories)
%shows that 
%\[F_r X = \mathsf{Lan}_{(T \in \Omega^{0,r})^{op}}
%\left(\bigotimes_{v \in V(T)} X(T_v)\right)
%\]
%where $\Omega^{0,r}$ is the category of trees and degeneracies.
%Similarly, the left coequalizer shows that
%\[
%F_rF_r X = \mathsf{Lan}_{((T_0 \hookrightarrow T_1) \in \Omega^{1,r})^{op}}
%\left(\bigotimes_{v \in V(T_{1})} X(T_{1,v})\right)
%\]
%$\Omega^{0,r}$ is the category whose objects are inner planar face maps $T_0 \hookrightarrow T_1$
%and whose morphisms are degeneracies between them.
%\end{remark}


\begin{remark}
	Proposition \ref{PARTADJP PROP} 
	admits a complementary result by interchanging the roles of $P$ and $\bar{F}$ and the roles of Kleisli categories and algebra categories.
	
	Explicitly, one has a functor
	$\psi_!\colon \Alg_{\bar{F}} \to \Alg_F$
	which: sends an $\bar{F}$-algebra $X$
	with multiplication $\bar{F} X \xrightarrow{m} X$
	to $PX$ with $F$-algebra multiplication given by
	$FPX \xrightarrow{\mu} FX=P\bar{F}X \xrightarrow{Pm} PX$;
	sends an arrow
	$X \xrightarrow{f} Y$
	to $PX \xrightarrow{Pf} PY$.
	
	Then in the square below one has that the bottom horizontal maps are adjoint and that
	$\iota \psi_! = \psi_!\iota$.
	\begin{equation}\label{PARTADJBARF EQ}
	\begin{tikzcd}
	\mathsf{Kl}_{\bar{F}} 
	\ar[shift left=0.25em]{r}{\psi_!} 
	\ar[hookrightarrow]{d}[swap]{\iota}
	&
	\mathsf{Kl}_{F} 
	\ar[hookrightarrow]{d}{\iota}
	\\
	\mathsf{Alg}_{\bar{F}} 
	\ar[shift left=0.25em]{r}{\psi_!} 
	&
	\mathsf{Alg}_{F}  
	\ar[shift left=0.25em]{l}{\psi^{\**}}
	\end{tikzcd}
	\end{equation}
	Moreover, in analogy to Remark \ref{MONALGP REM}
	one can check that the adjunction monad
	$\psi^{\**} \psi_!$ on $\Alg_{\bar{F}}$ has the same underlying functor, unit and multiplication as the monad $P$.
\end{remark}









\bibliography{biblio-new}{}
\bibliographystyle{amsalpha2}



\end{document}


%%% Local Variables:
%%% mode: latex
%%% TeX-master: t
%%% End:
