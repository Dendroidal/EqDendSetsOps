\documentclass[a4paper,10pt
,draft
]{article}%



% ---- Commands on draft --------

\usepackage[dvipsnames]{xcolor}% adds colors
\usepackage{ifdraft}
\ifdraft{
	\color[RGB]{63,63,63}
	\pagecolor[RGB]{220,220,204}
	\usepackage[notref]{showkeys}
	\usepackage{todonotes}
}
%{
%	\usepackage[disable]{todonotes}
%}



\pdfcompresslevel=0
\pdfobjcompresslevel=0

\usepackage{xr-hyper}
\usepackage[pagebackref, colorlinks, citecolor=PineGreen, linkcolor=PineGreen]{hyperref}
\hypersetup{
	final,
	pdftitle={Edits to ``On the homotopy theory of equivariant colored operads''},
	pdfauthor={Bonventre, P. and Pereira, L. A.},
	linktoc=page
}

\externaldocument[TAS-]{TameAndSquare} % cite using names from other half
\externaldocument[OC-]{OneColor} % cite using names from other half
\externaldocument{AllColors-v2}



\usepackage{amsmath, amsthm}% {amsfonts, amssymb}


% ------ New Characters --------------------------------------

\usepackage[latin1]{inputenc}%
\usepackage{MnSymbol}


\usepackage[normalem]{ulem}% underlining
%\usepackage{dsfont}% double strike-through
\usepackage{bbm}% more bb


\DeclareMathAlphabet\mathbb{U}{msb}{m}{n}
\usepackage{upgreek}
\usepackage{mathrsfs}



\usepackage[normalem]{ulem}

%----- Enumerate ---------------------------------------------
\usepackage[inline,shortlabels]{enumitem}% % can use \begin{enumerate*} for inparaenum
\setenumerate{label=(\roman*)}



% ---------- Page Typesetting ----------
\usepackage[final]{microtype}
\usepackage{relsize}
\usepackage{geometry}


%-------- Tikz ---------------------------

\usepackage{tikz}%
\usetikzlibrary{matrix,arrows,decorations.pathmorphing,
cd,patterns,calc}
\tikzset{%
  treenode/.style = {shape=rectangle, rounded corners,%
                     draw, align=center,%
                     top color=white, bottom color=blue!20},%
  root/.style     = {treenode, font=\Large, bottom color=red!30},%
  env/.style      = {treenode, font=\ttfamily\normalsize},%
  dummy/.style    = {circle,draw,inner sep=0pt,minimum size=2mm}%
}%

\usetikzlibrary[decorations.pathreplacing]


% ----- Labels Changed? --------

\makeatletter

\def\@testdef #1#2#3{%
  \def\reserved@a{#3}\expandafter \ifx \csname #1@#2\endcsname
  \reserved@a  \else
  \typeout{^^Jlabel #2 changed:^^J%
    \meaning\reserved@a^^J%
    \expandafter\meaning\csname #1@#2\endcsname^^J}%
  \@tempswatrue \fi}

\makeatother


%%%%%%%%%%%%%%%%%%%%%%%%% INTERNAL REFERENCES %%%%%%%%%%%%%%%%%%%%%%%%%%%%%%%%%%%

\numberwithin{equation}{section} 
\numberwithin{figure}{section}

\usepackage{mathtools}
\mathtoolsset{showonlyrefs,showmanualtags} % Only number equations which are referenced with eqref


% ------- New Theorems/ Definition/ Names-----------------------

 % \theoremstyle{plain} % bold name, italic text
\newtheorem{theorem}[equation]{Theorem}%
\newtheorem*{theorem*}{Theorem}%
\newtheorem{lemma}[equation]{Lemma}%
\newtheorem{proposition}[equation]{Proposition}%
\newtheorem{corollary}[equation]{Corollary}%
\newtheorem{conjecture}[equation]{Conjecture}%
\newtheorem*{conjecture*}{Conjecture}%
\newtheorem{claim}[equation]{Claim}%

%%%%%% Fancy Numbering for Theorems
\newtheorem{innercustomgeneric}{\customgenericname}
\providecommand{\customgenericname}{}
\newcommand{\newcustomtheorem}[2]{%
  \newenvironment{#1}[1]
  {%
   \renewcommand\customgenericname{#2}%
   \renewcommand\theinnercustomgeneric{##1}%
   \innercustomgeneric
  }
  {\endinnercustomgeneric}
}

\newcustomtheorem{customthm}{Theorem}
\newcustomtheorem{customcor}{Corollary}
%%%%%%%%%%%%%

\theoremstyle{definition} % bold name, plain text
\newtheorem{definition}[equation]{Definition}%
\newtheorem*{definition*}{Definition}%
\newtheorem{example}[equation]{Example}%
\newtheorem{remark}[equation]{Remark}%
\newtheorem{notation}[equation]{Notation}%
\newtheorem{convention}[equation]{Convention}%
\newtheorem{assumption}[equation]{Assumption}%
\newtheorem{exercise}{Exercise}%


% %%%%%%%%%%%%%%%%%%%%%%%%%%%%%%%%%%%%%%%%%%%%%%%%%%%%%%%%%%%%%%%%%%%%%%%%%%%%%%%%
% ------------------------------ COMMANDS ------------------------------

% ---------- macros

\newcommand{\set}[1]{\left\{#1\right\}}%
\newcommand{\sets}[2]{\left\{ #1 \;|\; #2\right\}}%
\newcommand{\longto}{\longrightarrow}%
\newcommand{\into}{\hookrightarrow}%
\newcommand{\onto}{\twoheadrightarrow}%

\usepackage{harpoon}
\newcommand{\vect}[1]{\text{\overrightharp{\ensuremath{#1}}}}


% ---------- operators

\newcommand{\Sym}{\ensuremath{\mathsf{Sym}}}%
\newcommand{\Fin}{\mathsf{F}}%
\newcommand{\Set}{\ensuremath{\mathsf{Set}}}
\newcommand{\Top}{\ensuremath{\mathsf{Top}}}
\newcommand{\sSet}{\ensuremath{\mathsf{sSet}}}%
\newcommand{\Cat}{\mathsf{Cat}}
\newcommand{\sCat}{\mathsf{sCat}}
\newcommand{\Op}{\mathsf{Op}}%
\newcommand{\sOp}{\ensuremath{\mathsf{sOp}}}%
\newcommand{\fgt}{\ensuremath{\mathsf{fgt}}}%
\newcommand{\dSet}{\mathsf{dSet}}
\newcommand{\Fun}{\mathsf{Fun}}
\newcommand{\Fib}{\mathsf{Fib}}
\newcommand{\Alg}{\mathsf{Alg}}
\newcommand{\Kl}{\mathsf{Kl}}



\DeclareMathOperator{\hocmp}{hocmp}%
\DeclareMathOperator{\cmp}{cmp}%
\DeclareMathOperator{\hofiber}{hofiber}%
\DeclareMathOperator{\fiber}{fiber}%
\DeclareMathOperator{\hocofiber}{hocof}%
\DeclareMathOperator{\hocof}{hocof}%
\DeclareMathOperator{\holim}{holim}%
\DeclareMathOperator{\hocolim}{hocolim}%
\DeclareMathOperator{\colim}{colim}%
\DeclareMathOperator{\Lan}{Lan}%
\DeclareMathOperator{\Ran}{Ran}%
\DeclareMathOperator{\Map}{Map}%
\DeclareMathOperator{\Id}{Id}%
\DeclareMathOperator{\mlf}{mlf}%
\DeclareMathOperator{\Hom}{Hom}%
\DeclareMathOperator{\Ho}{Ho}
\DeclareMathOperator{\Aut}{Aut}%
\DeclareMathOperator{\Stab}{Stab}
\DeclareMathOperator{\Iso}{Iso}
\DeclareMathOperator{\Ob}{Ob}

% ---------- shortcuts

\newcommand{\F}{\ensuremath{\mathcal F}}
\newcommand{\V}{\ensuremath{\mathcal V}}
\newcommand{\Q}{\ensuremath{\mathcal Q}}
\renewcommand{\O}{\ensuremath{\mathcal O}}
\renewcommand{\P}{\ensuremath{\mathcal P}}
\newcommand{\C}{\ensuremath{\mathcal C}}
\newcommand{\A}{\ensuremath{\mathcal A}}
\newcommand{\G}{\ensuremath{\mathcal G}}

\newcommand{\del}{\partial}%

\newcommand{\ki}{\chi}
\newcommand{\ksi}{\xi}
\newcommand{\Ksi}{\Xi}

\newcommand{\lltimes}{\underline{\ltimes}}

% detecting $\V$-categories:

\newcommand{\I}{\mathbb I}
\newcommand{\J}{\mathbb J}
\newcommand{\1}{\ensuremath{\mathbbm 1}}%{\ensuremath{\mathbb{id}}} %\eta

% lazy shortcuts

\newcommand{\SC}{\Sigma_{\mathfrak C}}
\newcommand{\OC}{\Omega_{\mathfrak C}}
\newcommand{\UV}{\underline{\mathcal V}}
\newcommand{\UC}{\underline{\mathfrak C}}










% %%%%%%%%%%%%%%%%%%%%%%%%%%%%%%%%%%%%%%%%%%%%%%%%%%%%%%%%%%%%%%%%%%%%%%%%%%%%%%%%%%%%%%%%%%%%%%%%%%%%
% ------------------------------ MAIN BODY ------------------------------

% ---- Title --------

\title{Edits to ``On the homotopy theory of equivariant colored operads''}

\author{Peter Bonventre, Lu\'is A. Pereira}%

\date{\today}


\begin{document} 
  
\maketitle
 


All numbering in this document refer to the numbering of AllColors-v2.



\section{General comments}

{\color{blue}
We feel that:
\begin{itemize}
\item Our arguments, even the ones following/mimicing/reproving results of BM or Cav, are more robust and cleanly presented that the current published literature (or lack there of, as mentioned by the referee).
        Such additional clarity in proof and statement is needed in order to adapt and apply these results to the more general equivariant setting.
\item Even if the statement of the propositions is analogous, much work is needed in order to state the exact definitions and constructions even in order to make sense of the statements
        (e.g. constructing, identifying, and describing the generating (trivial) cofibrations or  representable functors).
\item In many cases, much of the above follows from an intuitive understanding of the appropriate combinatorics, namely $G$-trees.
\end{itemize}
}

The prequel ``The homotopy theory of equivariant operads with fixed colors'' has been submitted, but we have not heard back.
We are confident in our work in that paper.

      

\section{Non-changes}

\begin{enumerate}
\item[(30)] The paper ends with the proof of a lemma from 3 pages before, directly after the last result of the paper. We do not feel that a concluding paragraph is necessary, as the start of the section directs the reader to the important results.
\end{enumerate}

\section{Changes}

\begin{enumerate}
\item[(9)] Added the reference equation (2.13), corrected typos, and slightly rephrased Definition \ref{GCC_DEF} to promote clarity
\item[(10)] We have decided to avoid a notational index, but have gone over the paper to insert notational reminders as needed.
\item[(11)] We have added a sentence to Remark \ref{GOTC_REM2} making a general statement regarding some of these comparisons, as well as refer back to Remark \ref{GTRIV REM}, Remark 4.8 of [BPd], and Remark 6.18 of [BPb].
        
        There is a lot that can be said on this topic.
        However, at the end of the day, the conditions for equivariant model structures and those for projective model structures usually cannot be compared;
        the different end goals, combined with the desire to be as flexible as possible, leads to similar-looking but quite distinct conditions.
\item[(17)] The ``only if'' direction is immediate: the right lifting property is a subset of the required lifting properties of being $\F$-path lifting.
\item[(18)] Added a clause clarifying the argument.
\item[(29)] Added the dual statement to Lemma \ref{TRANSFLIFT LEMMA}.
\item[(31)] References updated. All bibtex entries for published articles and books taken directly from MathSciNet.
\end{enumerate}



\section{Minor changes and typos}
 

\begin{enumerate}
\item[(1)] Hypentations inserted.
\item[(2)] ``in'' removed.
\item[(3)] Replaced some ``which'' with ``that'' throughout
\item[(4)] ``and'' added
\item[(5)] ``constrution'' replaced with ``construction''
\item[(6)] Added reference to [Yau]
\item[(7)] Quotation closed.
\item[(8)] Not a correction.
\item[(12)] ``sections'' capitalized.
\item[(13)] Clarified ``$H$-fixed colors''
\end{enumerate}

\section{Other minor changes}

\begin{enumerate}
\item Slightly rephrased Remark \ref{COLCHSQ REM}
\end{enumerate}




\bibliography{biblio3}{}

\bibliographystyle{alpha}


\end{document} 




%%% Local Variables:
%%% mode: latex
%%% TeX-master: t
%%% End:
