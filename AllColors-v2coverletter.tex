\documentclass[a4paper,10pt
%,draft
]{article}%



% ---- Commands on draft --------

\usepackage[dvipsnames]{xcolor}% adds colors
\usepackage{ifdraft}
\ifdraft{
	\color[RGB]{63,63,63}
	\pagecolor[RGB]{220,220,204}
	\usepackage[notref]{showkeys}
	\usepackage{todonotes}
}
%{
%	\usepackage[disable]{todonotes}
%}



\pdfcompresslevel=0
\pdfobjcompresslevel=0

\usepackage{xr-hyper}
\usepackage[pagebackref, colorlinks, citecolor=PineGreen, linkcolor=PineGreen]{hyperref}
\hypersetup{
	final,
	pdftitle={Edits to ``On the homotopy theory of equivariant colored operads''},
	pdfauthor={Bonventre, P. and Pereira, L. A.},
	linktoc=page
}

\externaldocument[TAS-]{TameAndSquare} % cite using names from other half
\externaldocument[OC-]{OneColor} % cite using names from other half
\externaldocument{AllColors-v2}



\usepackage{amsmath, amsthm}% {amsfonts, amssymb}


% ------ New Characters --------------------------------------

\usepackage[latin1]{inputenc}%
\usepackage{MnSymbol}


\usepackage[normalem]{ulem}% underlining
%\usepackage{dsfont}% double strike-through
\usepackage{bbm}% more bb


\DeclareMathAlphabet\mathbb{U}{msb}{m}{n}
\usepackage{upgreek}
\usepackage{mathrsfs}



\usepackage[normalem]{ulem}

%----- Enumerate ---------------------------------------------
\usepackage[inline,shortlabels]{enumitem}% % can use \begin{enumerate*} for inparaenum
\setenumerate{label=(\roman*)}



% ---------- Page Typesetting ----------
\usepackage[final]{microtype}
\usepackage{relsize}
\usepackage{geometry}


%-------- Tikz ---------------------------

\usepackage{tikz}%
\usetikzlibrary{matrix,arrows,decorations.pathmorphing,
cd,patterns,calc}
\tikzset{%
  treenode/.style = {shape=rectangle, rounded corners,%
                     draw, align=center,%
                     top color=white, bottom color=blue!20},%
  root/.style     = {treenode, font=\Large, bottom color=red!30},%
  env/.style      = {treenode, font=\ttfamily\normalsize},%
  dummy/.style    = {circle,draw,inner sep=0pt,minimum size=2mm}%
}%

\usetikzlibrary[decorations.pathreplacing]


% ----- Labels Changed? --------

\makeatletter

\def\@testdef #1#2#3{%
  \def\reserved@a{#3}\expandafter \ifx \csname #1@#2\endcsname
  \reserved@a  \else
  \typeout{^^Jlabel #2 changed:^^J%
    \meaning\reserved@a^^J%
    \expandafter\meaning\csname #1@#2\endcsname^^J}%
  \@tempswatrue \fi}

\makeatother


%%%%%%%%%%%%%%%%%%%%%%%%% INTERNAL REFERENCES %%%%%%%%%%%%%%%%%%%%%%%%%%%%%%%%%%%

\numberwithin{equation}{section} 
\numberwithin{figure}{section}

\usepackage{mathtools}
\mathtoolsset{showonlyrefs,showmanualtags} % Only number equations which are referenced with eqref


% ------- New Theorems/ Definition/ Names-----------------------

 % \theoremstyle{plain} % bold name, italic text
\newtheorem{theorem}[equation]{Theorem}%
\newtheorem*{theorem*}{Theorem}%
\newtheorem{lemma}[equation]{Lemma}%
\newtheorem{proposition}[equation]{Proposition}%
\newtheorem{corollary}[equation]{Corollary}%
\newtheorem{conjecture}[equation]{Conjecture}%
\newtheorem*{conjecture*}{Conjecture}%
\newtheorem{claim}[equation]{Claim}%

%%%%%% Fancy Numbering for Theorems
\newtheorem{innercustomgeneric}{\customgenericname}
\providecommand{\customgenericname}{}
\newcommand{\newcustomtheorem}[2]{%
  \newenvironment{#1}[1]
  {%
   \renewcommand\customgenericname{#2}%
   \renewcommand\theinnercustomgeneric{##1}%
   \innercustomgeneric
  }
  {\endinnercustomgeneric}
}

\newcustomtheorem{customthm}{Theorem}
\newcustomtheorem{customcor}{Corollary}
%%%%%%%%%%%%%

\theoremstyle{definition} % bold name, plain text
\newtheorem{definition}[equation]{Definition}%
\newtheorem*{definition*}{Definition}%
\newtheorem{example}[equation]{Example}%
\newtheorem{remark}[equation]{Remark}%
\newtheorem{notation}[equation]{Notation}%
\newtheorem{convention}[equation]{Convention}%
\newtheorem{assumption}[equation]{Assumption}%
\newtheorem{exercise}{Exercise}%


% %%%%%%%%%%%%%%%%%%%%%%%%%%%%%%%%%%%%%%%%%%%%%%%%%%%%%%%%%%%%%%%%%%%%%%%%%%%%%%%%
% ------------------------------ COMMANDS ------------------------------

% ---------- macros

\newcommand{\set}[1]{\left\{#1\right\}}%
\newcommand{\sets}[2]{\left\{ #1 \;|\; #2\right\}}%
\newcommand{\longto}{\longrightarrow}%
\newcommand{\into}{\hookrightarrow}%
\newcommand{\onto}{\twoheadrightarrow}%

\usepackage{harpoon}
\newcommand{\vect}[1]{\text{\overrightharp{\ensuremath{#1}}}}


% ---------- operators

\newcommand{\Sym}{\ensuremath{\mathsf{Sym}}}%
\newcommand{\Fin}{\mathsf{F}}%
\newcommand{\Set}{\ensuremath{\mathsf{Set}}}
\newcommand{\Top}{\ensuremath{\mathsf{Top}}}
\newcommand{\sSet}{\ensuremath{\mathsf{sSet}}}%
\newcommand{\Cat}{\mathsf{Cat}}
\newcommand{\sCat}{\mathsf{sCat}}
\newcommand{\Op}{\mathsf{Op}}%
\newcommand{\sOp}{\ensuremath{\mathsf{sOp}}}%
\newcommand{\fgt}{\ensuremath{\mathsf{fgt}}}%
\newcommand{\dSet}{\mathsf{dSet}}
\newcommand{\Fun}{\mathsf{Fun}}
\newcommand{\Fib}{\mathsf{Fib}}
\newcommand{\Alg}{\mathsf{Alg}}
\newcommand{\Kl}{\mathsf{Kl}}



\DeclareMathOperator{\hocmp}{hocmp}%
\DeclareMathOperator{\cmp}{cmp}%
\DeclareMathOperator{\hofiber}{hofiber}%
\DeclareMathOperator{\fiber}{fiber}%
\DeclareMathOperator{\hocofiber}{hocof}%
\DeclareMathOperator{\hocof}{hocof}%
\DeclareMathOperator{\holim}{holim}%
\DeclareMathOperator{\hocolim}{hocolim}%
\DeclareMathOperator{\colim}{colim}%
\DeclareMathOperator{\Lan}{Lan}%
\DeclareMathOperator{\Ran}{Ran}%
\DeclareMathOperator{\Map}{Map}%
\DeclareMathOperator{\Id}{Id}%
\DeclareMathOperator{\mlf}{mlf}%
\DeclareMathOperator{\Hom}{Hom}%
\DeclareMathOperator{\Ho}{Ho}
\DeclareMathOperator{\Aut}{Aut}%
\DeclareMathOperator{\Stab}{Stab}
\DeclareMathOperator{\Iso}{Iso}
\DeclareMathOperator{\Ob}{Ob}

% ---------- shortcuts

\newcommand{\F}{\ensuremath{\mathcal F}}
\newcommand{\V}{\ensuremath{\mathcal V}}
\newcommand{\Q}{\ensuremath{\mathcal Q}}
\renewcommand{\O}{\ensuremath{\mathcal O}}
\renewcommand{\P}{\ensuremath{\mathcal P}}
\newcommand{\C}{\ensuremath{\mathcal C}}
\newcommand{\A}{\ensuremath{\mathcal A}}
\newcommand{\G}{\ensuremath{\mathcal G}}

\newcommand{\del}{\partial}%

\newcommand{\ki}{\chi}
\newcommand{\ksi}{\xi}
\newcommand{\Ksi}{\Xi}

\newcommand{\lltimes}{\underline{\ltimes}}

% detecting $\V$-categories:

\newcommand{\I}{\mathbb I}
\newcommand{\J}{\mathbb J}
\newcommand{\1}{\ensuremath{\mathbbm 1}}%{\ensuremath{\mathbb{id}}} %\eta

% lazy shortcuts

\newcommand{\SC}{\Sigma_{\mathfrak C}}
\newcommand{\OC}{\Omega_{\mathfrak C}}
\newcommand{\UV}{\underline{\mathcal V}}
\newcommand{\UC}{\underline{\mathfrak C}}










% %%%%%%%%%%%%%%%%%%%%%%%%%%%%%%%%%%%%%%%%%%%%%%%%%%%%%%%%%%%%%%%%%%%%%%%%%%%%%%%%%%%%%%%%%%%%%%%%%%%%
% ------------------------------ MAIN BODY ------------------------------

% ---- Title --------

\title{Edits to ``On the homotopy theory of equivariant colored operads''}

\author{Peter Bonventre, Lu\'is A. Pereira}%

\date{\today}


\begin{document} 
  
\maketitle
 



\section{General comments}

The referee report included 31 line-by-line comments, 
all of which consisted of typos/minor changes 
to wording/minor requests for extra detail in proofs, and which have been addressed in the following sections (the numbering ot items refer to the numbering in the referee report).

In addition, the referee notes that our results on this paper depend heavily on [BPd], 
which we have submitted but not yet heard back from.
However, we feel fairly confident of our results in that paper,
as the techniques therein largely adapt techniques from our work [BPb] on ``genuine equivariant operads'' (which has been accepted for publication on Advances in Mathematics),
and our perspective is that the work in [BPb] presents strictly harder technical challenges than those in [BPd].


Lastly, the referee notes that ``statements for the $G$-equivariant version of a result from [BM13] often reduce to the original statement from [BM13], though a new argument is needed at one point'',
with that exceptional ``new argument'' referring to the 
argument about essential surjectivity, 
which requires a substantial revision and improvement over the non-equivariant argument (though we'd note that our arguments classifying fibrations are also novel, and have no analogues in any of [BM13],[CM13b]).
This assessment on the referee's part is not incorrect, though we feel it is somewhat ironic.
Much like [CM13b] was the culmination of a larger project featuring [MW09, CM11, CM13a], 
this paper is the $G$-equivariant culmination of a larger project featuring [Per18, BPc, BPa, BPd].
When one compares the projects as a whole, 
there are multiple points where $G$-equivariance introduces 
substantial difficulties, most often since key technical lemmas in the Cisinski-Moerdijk project (typically the kind of lemmas whose proof is delayed to the back of the paper to be read only by the most interested reader) do not play nicely with equivariance, and need to be revised from the ground up 
(much like what happened with essential surjectivity in this paper).
As such, we read the referee's assessment that ``some of our results seem easy consequences of results in [BM13]''
as saying that we were successful in confining the key 
technical difficulties to the prequel [BPd].


%We feel that
%our arguments, even the ones following results of [BM] or [Cav], are more robust and cleanly presented that the current published literature (or lack there of, as mentioned by the referee).
%Such additional clarity in proof and statement is needed in order to adapt and apply these results to the more general equivariant setting.
%Additionally, while the statement of several equivariant results appear similar, much work is needed in order to state the proper definitions and constructions necessary to make the statement itself
%(e.g. constructing, identifying, and describing the generating (trivial) cofibrations or representable functors).
%In many cases, much of the above follows from an intuitive understanding of the appropriate combinatorics, namely $G$-trees,
%as established in [BP21].
%Overall, we feel these results are a valuable contribution to the literature.

%The prequel ``The homotopy theory of equivariant operads with fixed colors'' has been submitted, but we have not heard back.
%We are confident in our work in that paper.

      

\section{Non-changes}

\begin{enumerate}
\item[(14)] The map is only defined in a single fiber, the fiber over $\mathfrak C = \mathfrak C_\O = \mathfrak C_\P$.
\item[(20)] $\mathbb J$ and $\mathcal C_f$ are themselves the (co)fibrant replacements, of $\tilde{\mathbbm 1}$ and $\mathcal C$ respectively. As such, we believe that the referee's suggestion would change the intended meaning of the sentence.
\item[(30)] The paper ends with the proof of a lemma from 3 pages before, directly after the last result of the paper. We do not feel that a concluding paragraph is necessary, as the start of the section directs the reader to the important results.
\end{enumerate}

\section{Changes}

\begin{enumerate}
\item[(9)] We rephrased and (hopefully) clarified Definition \ref{GCC_DEF}. In particular, we added a reference to equation (2.13), which was previously unnumbered.
\item[(10)] We have decided to avoid a notation index, but have gone over the paper to insert additional notation reminders as needed.
\item[(11)] We've extended Remark \ref{GOTC_REM2} to say a little more about how our monoidal global axiom relates to 
$h$-monoidality.

From a bird's eye point of view, the comparison is simple: our conditions are a ``genuine analogue'' of conditions like $h$-monoidality. 
However, from a technical point of view the story is rather subtle and well, technical, so we've decided to keep the discussion short. 
Briefly, if one abstracts $h$-monoidality to work with 
algebras over any operad (rather than just monoids, i.e. algebras over $Com$), 
one is naturally lead to conditions such as the one
in [WY18, Thm. 6.1.1].
And, throughout our project, the role of [WY18, Thm. 6.1.1]
is played by the cofibrant pushout powers condition
(this is discussed in Remark 4.8 of [BPd], where that condition first appears).
In practice, the cofibrant pushout powers condition demands more of the underlying category than something like [WY18, Thm. 6.1.1],
but has a number of desirable formal properties that allow
for (somewhat lengthy and involved) proofs that we would not 
know how to perform if using something like 
[WY18, Thm. 6.1.1] or
$h$-monoidality.

Or, to summarize, our conditions are significantly more similar to 
$h$-monoidality in purpose than they are in actual usage.


\item[(16)] Added a sentence in Section \ref{EXAMPLES SEC} about the listed conditions.
\item[(17)] The ``only if'' direction is immediate: the right lifting property is a subset of the required lifting properties of being $\F$-path lifting.
\item[(18)] Added a clause clarifying the argument.
\item[(29)] Added the dual statement to Lemma \ref{TRANSFLIFT LEMMA}.
\item[(31)] References updated. All bibtex entries for published articles and books taken directly from MathSciNet.

\end{enumerate}



\section{Minor changes and typos}
 

\begin{enumerate}
\item[(1)] Hyphenation of ``well-known'' and ``non-trivial'' was added.
\item[(2)] ``in'' in ``in below'' was removed.
\item[(3)] In the statement of Theorem A, as well as elsewhere in the paper, replaced ``which'' with ``that'', as requested.
\item[(4)] ``and'' added at the end of the first condition in Theorem A.
\item[(5)] Fixed the typo ``constrution''. 
\item[(6)] Added a reference to [Yau16].
\item[(7)] Quotation closed.
\item[(8)] Not a correction.
\item[(12)] ``sections'' capitalized.
\item[(13)] Clarified ``$H$-fixed colors''
\item[(15)] The argument in the proof of Proposition 3.11 was reworded for clarity.
\item[(19)] Gave second conjecture its own number.
\item[(21)] Fixed the quotation marks in the proof of Proposition \ref{RIGHTPROPER PROP}.
\item[(22)] Fized the $\mathbb{I}$ to $\mathbb{J}$
in the proof of Proposition 3.55.
\item[(23)] Fixed the quotation marks at the beginning of section 3.5
\item[(24)] Fixed the type ``sets1'' in the proof of proposition 3.56.
\item[(25)] Changed ``analogue'' to ``analogous'' at the start of Section 3.6.
\item[(26)] Added ``$\pi_0$'' to the composite and removed ``the class of'' from ``the class of the natural isomorphism'' in the second paragraph after (3.81).
\item[(27)] Fixed ``were'' to ``where'' in Lemma 3.84.
\item[(28)] Changed ``which'' to ``that'' in the statement of Lemma 3.85.
\end{enumerate}

\section{Other minor changes}

\begin{enumerate}
\item Slightly rephrased Remark \ref{COLCHSQ REM}
\end{enumerate}




% \bibliography{biblio3}{}

% \bibliographystyle{alpha}


\end{document} 




%%% Local Variables:
%%% mode: latex
%%% TeX-master: t
%%% End:
