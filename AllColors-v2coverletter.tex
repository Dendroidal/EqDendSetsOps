% LaTeX file for resume 
% This file uses the resume document class (res.cls)

\documentclass{article} 
% the margin option causes section titles to appear to the left of body text 
\textwidth=5.2in % increase textwidth to get smaller right margin
%\usepackage{helvetica} % uses helvetica postscript font (download helvetica.sty)
%\usepackage{newcent}   % uses new century schoolbook postscript font 
\usepackage{url}
\usepackage{hyperref}
\hypersetup{
  % colorlinks,
  final,
  pdftitle={Equivariant Dendroidal Segal Spaces},
  pdfauthor={Bonventre, P. and Pereira, L. A.},
  % pdfsubject={Your subject here},
  % pdfkeywords={keyword1, keyword2},
  linktoc=page
}

\usepackage{xr}
\externaldocument{EqSegSp&G-infty-ops}

\input{commands.tex}%

%-------- TIKZ -----------------------------------------
\usepackage{tikz}%
\usetikzlibrary{matrix,arrows,decorations.pathmorphing,
cd,patterns,calc}
\tikzset{%
  treenode/.style = {shape=rectangle, rounded corners,%
                     draw, align=center,%
                     top color=white, bottom color=blue!20},%
  root/.style     = {treenode, font=\Large, bottom color=red!30},%
  env/.style      = {treenode, font=\ttfamily\normalsize},%
  dummy/.style    = {circle,draw,inner sep=0pt,minimum size=2mm}%
}%

\usetikzlibrary[decorations.pathreplacing]
% \usetikzlibrary{external}\tikzexternalize
% \makeatletters
% \renewcommand{\todo}[2][]{\tikzexternaldisable\@todo[#1]{#2}\tikzexternalenable}

% \makeatother

\begin{document} 
 
\title{Edits to ``On the homotopy theory of equivariant colored operads''
\\[12pt]} % the \\[12pt] adds a blank line after name
 
\author{Bonventre, P. and Pereira, L. A.}
 
\maketitle
 




\section{General comments}

{\color{blue}
We feel that:
\begin{itemize}
\item Our arguments, even the ones following/mimicing/reproving results of BM or Cav, are more robust and cleanly presented that the current published literature (or lack there of, as mentioned by the referee).
        Such additional clarity in proof and statement is needed in order to adapt and apply these results to the more general equivariant setting.
\item Even if the statement of the propositions is analogous, much work is needed in order to state the exact definitions and constructions even in order to make sense of the statements
        (e.g. constructing, identifying, and describing the generating (trivial) cofibrations or  representable functors).
\item In many cases, much of the above follows from an intuitive understanding of the appropriate combinatorics, namely $G$-trees.
\end{itemize}
}

The prequel ``The homotopy theory of equivariant operads with fixed colors'' has been submitted, but we have not heard back.
We are confident in our work in that paper.

      

\section{Changes / Alternatives}

\begin{enumerate}
\item[(9)] Added the reference equation (2.13), corrected typos, and slightly rephrased Definition 2.38 to promote clarity
\item[(10)] We have decided to avoid a notational index, but have gone over the paper to insert notational reminders as needed.
\end{enumerate}


\section{Minor changes and typos}
 

\begin{enumerate}
\item[(1)] Hypentations inserted.
\item[(2)] ``in'' removed.
\item[(3)] Replaced some ``which'' with ``that'' throughout
        
\end{enumerate}






\bibliography{biblio}{}

\bibliographystyle{alpha}


\end{document} 




%%% Local Variables:
%%% mode: latex
%%% TeX-master: t
%%% End:
